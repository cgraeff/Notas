\DeclareMathOperator{\sen}{sen}
\DeclareMathOperator{\Tr}{Tr}
\newcommand{\mean}[1]{\left\langle{#1}\right\rangle}
\newcommand{\degree}[1]{\np[\tcdegree]{#1}}
\renewcommand{\mod}[1]{\ensuremath{|#1|}}
\newcommand{\versi}{\hat{\imath}}
\newcommand{\versj}{\hat{\jmath}}
\newcommand{\versk}{\hat{k}}
\newcommand{\fat}{f_{at}}
\newcommand{\vecfat}{\vec{f}_{at}}
\newcommand{\comment}[1]{\marginnote{\footnotesize\textbf{\texttt{#1}}}}
\newcommand{\etal}{\emph{et al.}}

%\def\mathnote#1{%
%  \stepcounter{equation}\tag{\theequation{\rlap{\hspace\marginparsep\smash{\parbox[c]{\marginparwidth}{%
%  \footnotesize\emph{\textbf{#1}}}}}}}
%}

\newcommand{\mathnote}[2][0pt]{\marginnote{\footnotesize\emph{\textbf{#2}}}[#1]}

\newcommand*\keystroke[1]{%
  \tikz[baseline=(key.base)]
    \node[%
      draw,
      fill=white,
      drop shadow={shadow xshift=0.25ex,shadow yshift=-0.25ex,fill=black,opacity=0.75},
      rectangle,
      rounded corners=1pt,
      inner sep=1pt,
      line width=0.5pt,
      font=\footnotesize\sffamily
    ](key) {#1\strut}
  ;
}
