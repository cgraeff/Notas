%%%%%%%%%%%%%%%%%%%%%%%%%%%%%%%%%%%%%%%%%%%%%%%%%%%%%%%%%%%%%%%%%%
%%%%%%%%%%%%%%%%%%%%%%%%%%%%%%%%%%%%%%%%%%%%%%%%%%%%%%%%%%%%%%%%%%
%%%%%%%%%%%%%%%%%%%%%%%%%%%%%%%%%%%%%%%%%%%%%%%%%%%%%%%%%%%%%%%%%%
\documentclass[prc, reprint, amsmath, linenumbers,10pt]{revtex4-1}
%%%%%%%%%%%%%%%%%%%%%%%%%%%%%%%%%%%%%%%%%%%%%%%%%%%%%%%%%%%%%%%%%%
%%%%%%%%%%%%%%%%%%%%%%%%%%%%%%%%%%%%%%%%%%%%%%%%%%%%%%%%%%%%%%%%%%
%%%%%%%%%%%%%%%%%%%%%%%%%%%%%%%%%%%%%%%%%%%%%%%%%%%%%%%%%%%%%%%%%%

%
% Packages
%
\usepackage[utf8]{inputenc}
\usepackage[T1]{fontenc}
\usepackage{lipsum}
\usepackage{graphicx}
\usepackage[np]{numprint}

%
% Macros
%
\newcommand{\tr}{\rm{Tr}}
\newcommand{\mean}[1]{\left\langle{#1}\right\rangle}
\newcommand{\comment}[1]{{\bf\textit{#1}}}

%%%%%%%%%%%%%%%%
%%%%%%%%%%%%%%%%
\begin{document}
%%%%%%%%%%%%%%%%
%%%%%%%%%%%%%%%%

%
%%
%%%
%%%% Front Matter
%%%
%%
%

\title{The title}

\author{Clebson Abati Graeff}
\affiliation{Universidade Tecnológica Federal do Paraná, campus Pato Branco \\ Via do Conhecimento, Km 1 CEP 85503-390 Pato Branco -- PR, Brazil}
\email{cgraeff@utfpr.edu.br}

\author{Débora Pérez Menezes}
\affiliation{Departamento de Física, Universidade Federal de Santa Catarina, Florianópolis, SC, CP 476, CEP 88.040-900, Brazil}
\email{debora.p.m@ufsc.br}

%%%

\begin{abstract}
An article usually includes an abstract, a concise summary of the work
covered at length in the main body of the article. 
\begin{description}
\item[Usage]
Secondary publications and information retrieval purposes.
\item[PACS numbers]
May be entered using the \verb+\pacs{#1}+ command.
\item[Structure]
You may use the \texttt{description} environment to structure your abstract;
use the optional argument of the \verb+\item+ command to give the category of each item. 
\end{description}
\end{abstract}

%%%

\pacs{Valid PACS appear here}

%%%

\maketitle

%
%%
%%%
%%%% Main Matter
%%%
%%
%

%%%%%%%%%%%%%%%%%%%%%%
\section{Introduction}
%%%%%%%%%%%%%%%%%%%%%%

\comment{Fazer como o Marcelo e deixar enviar em branco?}

%%%%%%%%%%%%%%%%%%%
\section{Formalism}
%%%%%%%%%%%%%%%%%%%

\comment{Lagrangiana genérica SU(2)?}

%%%%%%%%%%%%%%%%%%%%%%%%%
\subsection{Quark Matter}
%%%%%%%%%%%%%%%%%%%%%%%%%

The quarks phase is described by a NJL model SU(2) lagrangian, including a vector-isoscalar term, given by\cite{Buballa2005}
\begin{equation}\label{Eq:LagNJL-SU2-Bub}
\begin{split}
	\mathcal{L} =&~ \bar{\psi}(i\gamma^\mu\partial_\mu - m_0)\psi \\
	&+ G_S[(\bar{\psi}\psi)^2 + (\bar{\psi}i\gamma_5\vec{\tau}\psi)^2] - G_V(\bar{\psi}\gamma^\mu \psi)^2.
\end{split}
\end{equation}
%
Here $\psi$ represents the quark field, $m_0$ the quark bare mass, and $G_S$ and $G_V$ are coupling constants that are chosen by fitting the pion mass $m_\pi = \np[MeV]{135.0}$ and decay constant $f_\pi = \np[MeV]{92.4}$. As the theory is non-renormalizable, a momentum cutoff $\Lambda$ is employed, which acts as a new parameter.

From the lagrangian $\mathcal{L}$, the hamiltonian density $\mathcal{H}$ can be obtained, which leads to the thermodynamic potential per volume $V$ at temperature $T$ by means of
\begin{equation}
	\omega(T, \mu) = -\frac{T}{V} \ln \tr \exp\left(-\frac{1}{T}\int d^3x(\mathcal{H} - \mu\psi^\dagger\psi\right),
\end{equation}
%
where $\tr$ stands for a trace over all states of the system, resulting in \comment{(sinal!)}
\begin{equation}\label{Eq:Pot_Termo_Temp_Finita}
\begin{split}
	\omega(T, \mu; m, \mu_R) =&~ \omega_M(T, \mu_R) \\
	&+ \frac{(m - m_0)^2}{4G_s} - \frac{(\mu - \mu_R)^2}{4G_v} +  \textrm{const.},
\end{split}
\end{equation}
and
\begin{equation}
\begin{split}\label{Eq:Por_Termo_Temp_Finita_Fermi_Gas_Contrib}
	\omega_M(T, \mu_R) = -2 n_f n_c \int \frac{d^3p}{(2\pi)^3} \{&E_p \\
	&+ T\ln(1+e^{-(E_p-\mu_R)/T} \\
	&+ T\ln(1 + e^{-(E_p+\mu_R)/T})\}.
\end{split}
\end{equation}
%
where $n_f$ and $n_c$ stand for the number of flavors and the number of colors, respectively, and $E_p = \sqrt{p^2 + m^2}$.

Here the renormalized chemical potential $\mu_R$ and the constituent mass $m$ are obtained by requiring that $\partial \omega / \partial \mu_R = 0$ and $\partial \omega / \partial m = 0$, respectively, resulting in
\begin{align}
	\mu_R &= \mu - 2 G_V \rho \\
	m &= m_0 - 2 G_S \rho_s
\end{align}
%
and
\begin{align}
	\rho &= 2 n_f n_c \int \frac{d^3p}{(2\pi)^3} (n_p(T, \mu_R) - \bar{n}_p(T, \mu_R))\\
	\rho_s &= - 2 n_f n_c \int\frac{d^3p}{(2\pi)^3} \frac{m}{E_p}(1 - n_p -\bar{n}_p).
\end{align}
%
The Fermi ocupation numbers for quarks $n_p(T, \mu)$ and antiquarks $\bar{n}_p(T, \mu)$ are given by
\begin{align}
	n_p &= \frac{1}{1 + e^{(E_p - \mu)/T}} & \bar{n}_p &= \frac{1}{1 + e^{(E_p + \mu)/T}}
\end{align}

The constant in the potential has no influence in the results \comment{(tem: desloca $p$ e $\varepsilon$ para cima, mas isso não tem significado físico)}, consequently, it may be chosen so that at $T = \mu = 0$ the thermodynamic potential is zero at the value $m = m_{\rm{vac}}$ which minimizes $\omega$. This process may be represented by a  
\begin{equation}
	\tilde\omega(T, \mu; m, \mu_R) = \omega(T, \mu; m, \mu_R) - \omega(0, 0; m_{\rm{vac}}, 0),
\end{equation}
%
so that the quantities we are interested in --~the pressure $p$ and the energy density $\varepsilon$~-- are obtained through
\begin{align}
		p(T, \mu) &= -\tilde\omega(T, \mu; m, \mu_R) \label{Exp_pressao_T}\\
		\varepsilon(T, \mu) &= -p(T, \mu) + T s(T, \mu) + \mu n(T,\mu). \label{Exp_energia_T}
\end{align}
% Having found a pair of solutions $m$ and $\mu_R$, other thermodynamic quantities can be obtained in the standard way. Since the system is uniform, pressure and energy density are given by
	
Taking the limit $T \to 0$, the expressions for $\rho$, $\rho_s$, and $\tilde\omega$ assume the form
\begin{align}
	\rho &=  2 n_f n_c \int \frac{d^3p}{(2\pi)^3} \theta(p_F - p)\\
	\rho_s &= - 2 n_f n_c \int\frac{d^3p}{(2\pi)^3} \frac{m}{E_p} \theta(p_F - p)\\
	\omega_M &= -2 n_f n_c \int\frac{d^3p}{(2\pi)^3} [E_p + (\mu_R - E_p)\theta(p_F - p)].
\end{align}
%
The chemical potential obeys the relation
\begin{equation}
	\mu_R = \sqrt{p_F^2 + m^2}.
\end{equation}

The solution of the above equations consist in determining the solution to the equation for $m$ self-consistently for each value of $\rho$ or $\mu$ (in this case $p_F = \sqrt{\mu_R^2 - m^2}\theta(\mu_R^2 - m^2)$). \comment{(+info nesse parágrafo; resultados?)}

\begin{table*}
\caption{Parameters sets for the lagrangian density~\eqref{Eq:LagNJL-SU2-Bub} \cite{Buballa1996, Buballa2005}. \label{Tab:Parametros_NJL}}
\begin{ruledtabular}
\begin{tabular}{lcccccccc}
Model &  $\Lambda$ & $G_S$ ($\rm{fm}^2$) & $G_V$ ($\rm{fm}^2$) & $m_0$ (MeV) & $m$ (MeV) \\
\hline
Buballa-1 & 650 & 0.19721 & -- & 0 & 313 \\
Buballa-2 & 600 & 0.26498 & -- & 0 & 400 \\
Buballa-3 & 570 & 0.34034 & -- & 0 & 500 \\
%BuballaR-1 & 664.3 & 0.18176 & $\propto G_S$ & 5.0 & 300 \\
BuballaR-2 & 587.9 & 0.27449 & $\propto G_S$ & 5.6 & 400 \\
%BuballaR-3 & 569.3 & 0.33759 & $\propto G_S$ & 5.5 & 500 \\
%BuballaR-4 & 568.6 & 0.38178 & $\propto G_S$ & 5.1 & 600
\end{tabular}
\end{ruledtabular}
\end{table*}

%%%%%%%%%%%%%%%%%%%%%%%%%%
\subsection{Hadron Matter}
%%%%%%%%%%%%%%%%%%%%%%%%%%

Even though the original NJL model is unable to describe the saturation properties of the nuclear matter, this can be fixed by the inclusion of a \comment{scalar-isovector?} channel~\cite{Koch1987}. An extended NJL model (eNJL)~\cite{Pais2016} which includes such channel is given by the lagrangian density
\begin{equation}\label{Eq:Lagrangiana_eNLJ_Pais}
\begin{split}
	\mathcal{L} =&~ \bar{\psi}(i\gamma^\mu\partial_\mu - m_0)\psi + G_s[(\bar{\psi}\psi)^2 + (\bar{\psi}i\gamma_5\vec{\tau}\psi)^2] \\
	& - G_v(\bar{\psi}\gamma^\mu\psi)^2 - G_{sv}[(\bar{\psi}\psi)^2 + (\bar{\psi}i\gamma_5\vec{\tau}\psi)^2](\bar{\psi}\gamma^\mu\psi)^2 \\
	& - G_\rho[(\bar{\psi}\gamma^\mu\vec{\tau}\psi)^2 + (\bar{\psi}\gamma_5\gamma^\mu\vec{\tau}\psi)^2] \\
	& - G_{v\rho}(\bar{\psi}\gamma^\mu\psi)^2[(\bar{\psi}\gamma^\mu\vec{\tau}\psi)^2 + (\bar{\psi}\gamma_5\gamma^\mu\vec{\tau}\psi)^2] \\
	& - G_{s\rho} [(\bar{\psi}\psi)^2 + (\bar{\psi}i\gamma_5\vec{\tau}\psi)^2][(\bar{\psi}\gamma^\mu\vec{\tau}\psi)^2 + (\bar{\psi}\gamma_5\gamma^\mu\vec{\tau}\psi)^2].
\end{split}
\end{equation}
%
where $\psi$ represents the nucleon field and the constants $G_i$ represents the coupling constants for the different channels.

The thermodynamic potential obtained from~\eqref{Eq:Lagrangiana_eNLJ_Pais} is given by \comment{(hocus-pocus)}
\begin{equation}\label{Eq:potencial_termodinamico}
\begin{split}
	\omega(\mu) =&~ \varepsilon_{\rm{kin}} + m\rho_s - G_s\rho_s^2 + G_v\rho^2 + G_{sv}\rho_s^2\rho^2 + G_\rho\rho_3^2 \\
	&+ G_{v\rho}\rho^2\rho_3^2 + G_{s\rho}\rho_s^2\rho_3^2 - \mu_p\rho_p - \mu_n\rho_n,
\end{split}
\end{equation}
%
where $\rho_B$ is the total barionic density, which is the sum of the proton $\rho_B^p$ and neutron $\rho_B^n$ barionic densities. At zero temperature, those densities are given by
\begin{equation}
	\rho_i = \int_0^{p_F^i}\frac{dp}{\pi^2}p^2; \qquad i = p,n
\end{equation}
%
where $p_F^i$ stands for the Fermi momentum of each particle. The kinectic energy contribution is given by
\begin{align}
	\varepsilon_{\rm{kin}} &= 2 N_c\sum_i \int \frac{d^3p}{(2\pi)^3}\frac{p^2}{E_i}(n_{i-}-n_{i+})\theta(\Lambda^2 - p^2) \\
	&= 2 n_c \sum_i \int \frac{d^3p}{(2\pi)^3}\frac{p^2}{E_p^i}(1 - \theta(p_F^i - p))\theta(\Lambda^2 - p^2),
\end{align}
%
with the second expression being valid for $T = 0$.

The effective mass $m$ and the chemical potentials appearing in the thermodynamic potencial $\omega$ are determined by requiring that $\partial\omega/\partial m = 0$ and $\partial\omega/\partial p_F^i = $, resulting in
\begin{align}\label{Eq:Gap}
	m &= m_0 - 2G_s\rho_s + 2G_{sv}\rho_s\rho^2 + 2 G_{s\rho}\rho_s\rho_3^2 \\
	\mu_i &= E_{p_F}^i + 2G_v\rho + 2G_{sv}\rho\rho_s^2 \pm 2G_\rho\rho_3+2G_{v\rho}\rho_3^2\rho \nonumber \\
	&\phantom{=} \pm 2G_{v\rho}\rho^2\rho_3 \pm 2 G_{s\rho}\rho_3\rho_s^2,
\end{align}
%
with $i = p,n$, and e $E_{p_F}^i = \sqrt{M^2 + (p_F^i)^2}$. The scalar density $\rho_s$ is given by sum of the proton and neutron scalar densities
\begin{align}
	\rho_s^i &= - 2 n_c \int \frac{d^3p}{(2\pi)^3}\frac{m_0^i m_i}{E_p^i}(n_{i-} - n_{i+})\theta(\Lambda^2 - p^2) \\
	&= - 2 n_c \int \frac{d^3p}{(2\pi)^3}\frac{m_0^i m_i}{E_p^i}\theta(p_F - p)\theta(\Lambda^2 - p^2)
\end{align}
%
where the second expression being valid for $T = 0$.

The equations of state can be obtained from
\begin{align}
	P &= -\omega(\mu) + \omega_{\rm{vac}} \\
	\varepsilon &= -P + \mu_p \rho_B^p + \mu_n \rho_B^n,
\end{align}
%
where $\omega_{\rm{vac}}$ ... \comment{mudei $\varepsilon_0 \to \omega_{\rm{vac}}$, pois acredito que é isso (discutir os cálculos que me levaram a pensar isso). Fica a questão de como mostrar que $m_{\rm{vac}} = m_n$. Confirmando isso, acho melhor deixar mais próximo da seção anterior.} 


\begin{table*}
\caption{Conjuntos de parâmetros para a lagrangiana~\eqref{Eq:Lagrangiana_eNLJ_Pais}\cite{Pais2016}. \label{Tab:Parametros_eNJL}}
\begin{ruledtabular}
\begin{tabular}{lcccccccc}
Model & $G_s$ ($\rm{fm}^2$) & $G_v$ ($\rm{fm}^2$) & $G_{sv}$ ($\rm{fm}^8$) & $G_\rho$ ($\rm{fm}^2$) & $G_{v\rho}$ ($\rm{fm}^8$) & $G_{s\rho}$ ($\rm{fm}^8$) & $\Lambda$ (MeV) & $m$ (MeV) \\
\hline
eNJL1 & 4.855 & 4.65 & -6.583 & 0.5876 & 0 & 0 & 388.189 & 0 \\
eNJL1$\omega\rho$1 & 4.855 & 4.65 & -6.583 & 0.5976 & -1 & 0 & 388.189 & 0 \\
eNJL1$\omega\rho$2 & 4.855 & 4.65 & -6.583 & 0.6476 & -6 & 0 & 388.189 & 0 \\
eNJL2 & 3.8 & 3.8 & -4.228 & 0.6313 & 0 & 0 & 422.384 & 0 \\
eNJL2$\omega\rho$1 & 3.8 & 3.8 & -4.228 & 0.6413 & -1 & 0 & 422.384 & 0 \\
eNJL3 & 1.93 & 3.0 & -1.8 & 0.65 & 0 & 0 & 534.815 & 0 \\
eNJL3$\sigma\rho$1 & 1.93 & 3.0 & -1.8 & 0.0269 & 0 & 0.5 & 534.815 & 0 \\
eNJL1m & 1.3833 & 1.781 & -2.943 & 0.7 & 0 & 0 & 478.248 & 450 \\
eNJL1m$\sigma\rho$1 & 1.3833 & 1.781 & -2.943 & 0.0739 & 0 & 1 & 478.248 & 450 \\
eNJL2m & 1.078 & 1.955 & -2.74 & 0.75 & 0 & 0 & 502.466 & 450 \\
eNJL2m$\sigma\rho$1 & 1.078 & 1.955 & -2.74 & -0.1114 & 0 & 1 & 502.466 & 450 \\
\end{tabular}
\end{ruledtabular}
\end{table*}

%%%%%%%%%%%%%%%%%%
\section{Binodals}
%%%%%%%%%%%%%%%%%%

The phase coexistence may be determined using the Gibbs' conditions \comment{(ref)}
\begin{align}
\mu_B^Q &= \mu_B^H \\
T^Q &= T^H \\
P^Q &= P^H
\end{align}
%
where the indexes $H$ and $Q$ refer to the hadrons and quarks phases. The chemical potentials are given by \cite{Cavagnoli2011}
\begin{align}
	\mu_B^H &= \frac{\mu_p + \mu_n}{2} \\
	\mu_B^Q &= \frac{3}{2} (\mu_u + \mu_d) = 3 \mu_q.
\end{align}
%
The phase coexistence may then be obtained by simply plotting $P \times \mu_B^i$, $i = Q, H$, and looking for the intersection of both lines.

\comment{(comentar que estamos usando exclusivamente $y_p = 0.5$?)}

%%%%%%%%%%%%%%%%%
\section{Results}
%%%%%%%%%%%%%%%%%

\comment{(Resultados para $T = 0$. Que critério adotar na escolha das combinações de parametrização para as quais elaboraremos gráficos?)}

%\begin{figure}[!htpb]
%\includegraphics[width=\linewidth]{graph/Buballa_1-eNJL1-quark-hadron_phase_transition.pdf}
%\caption{Teste.}
%\end{figure}

%
%%
%%%
%%%% Back Matter
%%%
%%
%

\begin{acknowledgments}
We wish to acknowledge the support of the author community in using
REV\TeX{}, offering suggestions and encouragement, testing new versions,
\dots.
\end{acknowledgments}

%%%

%\appendix
%\section{Appendix title}

%%%

\bibliography{references}

%%%%%%%%%%%%%%
%%%%%%%%%%%%%%
%%%%%%%%%%%%%%
\end{document}
%%%%%%%%%%%%%%
%%%%%%%%%%%%%%
%%%%%%%%%%%%%%
