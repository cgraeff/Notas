\documentclass[plainsections,alongposter]{sciposterlocal}

\usepackage{multicol}
\usepackage{palatino} % Better fonts
\usepackage{parskip}
\usepackage{amsmath}
\usepackage{booktabs}
\usepackage[utf8]{inputenc}
\usepackage{graphicx}
\usepackage[np]{numprint}

%
% Macros
%
\newcommand{\tr}{\rm{Tr}}
\newcommand{\mean}[1]{\left\langle{#1}\right\rangle}
\newcommand{\comment}[1]{{\bf\textit{#1}}}

%%%%%%%%%%%% Header info
\title{The QCD phase-diagram obtained from NJL and extended-NJL models for quark and hadron phases}

\author{Graeff, C. A.$^\dagger$; Menezes, D. P.$^*$}
\email{cgraeff@fsc.ufsc.br, debora.p.m@ufsc.br}

\institute{$^\dagger$~Universidade Tecnológica Federal do Paraná -- Pato Branco, PR - Brazil \\
$^*$~Universidade Federal de Santa Catarina -- Florian\'opolis, SC - Brazil \\
}

\leftlogo{Brasao_UFSC_PDF.pdf}
\rightlogo{logo_UTFPR_cor.pdf}
%%%%%%%%%%%%

%%% Color for sections titles
\definecolor{SectionColor}{rgb}{0.22,0.33,0.6}

%%% Set roman fonts instead of sans serif
\renewcommand{\familydefault}{\rmdefault}

%%% No rules between columns, please
\setlength{\columnseprule}{0pt}

%%% BibTeX bibliography style
\bibliographystyle{plain}

%%%%%%%%%%%%%%%%%%%%
%%% The Document %%%
%%%%%%%%%%%%%%%%%%%%
\begin{document}

\maketitle % The Header

\begin{multicols}{2} % Start content area

%%%%%%%%%%%%%%%%%%%
\section*{Abstract}
{ \it
We analyse the hadron/quark-gluon-plasma phase transition described by the Nambu-Jona-Lasinio (NJL) model [quark phase] and the extended Nambu-Jona-Lasinio model (eNJL) [hadron phase]. While the original formulation of NJL model is not capable of describing hadronic properties due to its lack of confinement, it can be extended with a scalar-vector interaction so it exhibits this property, the so-called  eNJL model. As part of this analysis, we obtain the equations of state within the SU(2) versions of both models for
for the hadron and the quark phases and determine the binodal surface. 
}

%%%%%%%%%%%%%%%%%%%%%%%
\section*{Motivation}

Lorem ipsum.


 
%%%%%%%%%%%%%%%%%%%%%%%%
\section*{Quarks phase}

The quarks phase is described by a NJL model SU(2) lagrangian, including a vector-isoscalar term, given by\cite{Buballa2005}
\begin{equation}\label{Eq:LagNJL-SU2-Bub}
\begin{split}
	\mathcal{L} =&~ \bar{\psi}(i\gamma^\mu\partial_\mu - m_0)\psi \\
	&+ G_S[(\bar{\psi}\psi)^2 + (\bar{\psi}i\gamma_5\vec{\tau}\psi)^2] - G_V(\bar{\psi}\gamma^\mu \psi)^2.
\end{split}
\end{equation}
%
Here $\psi$ represents the quark field, $m_0$ the quark bare mass, and $G_S$ and $G_V$ are coupling constants that are chosen by fitting the pion mass $m_\pi = \np[MeV]{135.0}$ and decay constant $f_\pi = \np[MeV]{92.4}$. As the theory is non-renormalizable, a momentum cutoff $\Lambda$ is employed, which acts as a new parameter.

%%%%%%%%%%%%%%%%%%%%%%%
\section*{Hadrons phase}
Even though the original NJL model is unable to describe the saturation properties of the nuclear matter, this can be fixed by the inclusion of a \comment{scalar-isovector?} channel~\cite{Koch1987}. An extended NJL model (eNJL)~\cite{Pais2016} which includes such channel is given by the lagrangian density
\begin{equation}\label{Eq:Lagrangiana_eNLJ_Pais}
\begin{split}
	\mathcal{L} =&~ \bar{\psi}(i\gamma^\mu\partial_\mu - m_0)\psi + G_s[(\bar{\psi}\psi)^2 + (\bar{\psi}i\gamma_5\vec{\tau}\psi)^2] \\
	& - G_v(\bar{\psi}\gamma^\mu\psi)^2 - G_{sv}[(\bar{\psi}\psi)^2 + (\bar{\psi}i\gamma_5\vec{\tau}\psi)^2](\bar{\psi}\gamma^\mu\psi)^2 \\
	& - G_\rho[(\bar{\psi}\gamma^\mu\vec{\tau}\psi)^2 + (\bar{\psi}\gamma_5\gamma^\mu\vec{\tau}\psi)^2] \\
	& - G_{v\rho}(\bar{\psi}\gamma^\mu\psi)^2[(\bar{\psi}\gamma^\mu\vec{\tau}\psi)^2 + (\bar{\psi}\gamma_5\gamma^\mu\vec{\tau}\psi)^2] \\
	& - G_{s\rho} [(\bar{\psi}\psi)^2 + (\bar{\psi}i\gamma_5\vec{\tau}\psi)^2][(\bar{\psi}\gamma^\mu\vec{\tau}\psi)^2 + (\bar{\psi}\gamma_5\gamma^\mu\vec{\tau}\psi)^2].
\end{split}
\end{equation}
%
where $\psi$ represents the nucleon field and the constants $G_i$ represents the coupling constants for the different channels.

%%%%%%%%%%%%
\columnbreak

%%%%%%%%%%%%%%%%%%%
\section*{Binodals}

The phase coexistence may be determined using the Gibbs' conditions \comment{(ref)}
\begin{align}
\mu_B^Q &= \mu_B^H \\
T^Q &= T^H \\
P^Q &= P^H
\end{align}
%
where the indexes $H$ and $Q$ refer to the hadrons and quarks phases. The chemical potentials are given by \cite{Cavagnoli2011}
\begin{align}
	\mu_B^H &= \frac{\mu_p + \mu_n}{2} \\
	\mu_B^Q &= \frac{3}{2} (\mu_u + \mu_d) = 3 \mu_q.
\end{align}
%
The phase coexistence may then be obtained by simply plotting $P \times \mu_B^i$, $i = Q, H$, and looking for the intersection of both lines.

%%%%%%%%%%%%%%%%%%%%
\section*{Results}
% Link para o github nesta seção



%%%%%%%%%%%%%%%%%%%%%%%%%%%%%%%%%%%%%%
\section*{Conclusions}


%%%%%%%%%%%%%%%%%%%%%%%%
%%% The bibliography %%% 
%%%%%%%%%%%%%%%%%%%%%%%%
{ \tiny
\bibliography{references}
}

\end{multicols}
\end{document}
