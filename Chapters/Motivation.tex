%%%%%%%%%%%%%%%%%%%
\chapter{Motivação}
%%%%%%%%%%%%%%%%%%%

\begin{fullwidth}\it
Cada capítulo tem um texto, uma sinopse, dizendo o que ai ser feito e por quê.
\end{fullwidth}

%%%%%%%%%%%%%%%%%%%%%%%%%%%%%%
\section{Estrelas de nêutrons}
%%%%%%%%%%%%%%%%%%%%%%%%%%%%%%

Neutron stars are believed to be the remnants of supernova
explosions [1–3]. They are born hot and rich in leptons. During
the very beginning of the evolution, an energy of the order
of \np[ergs]{10e53} is radiated away from the neutron star by
neutrinos, in a process known as deleptonization. Theoretical
studies involving different possible equations of state
that result in different matter composition have to be performed.
This is because the temporal evolution of the star
in the so-called Kelvin–Helmholtz epoch, during which the
remnant compact object changes from a hot and lepton-rich
to a cold and deleptonized star, depends on two key ingredients:
the equation of state (EoS) and its associated

1. J.M. Lattimer, M. Prakash, Science 304, 536 (2004)
2. M. Prakash, I. Bombaci, M. Prakash, P.J. Ellis, J.M. Lattimer,
R. Knorren, Phys. Rep. 280, 1 (1997)
3. N.K. Glendenning, Compact Stars (Springer, New York, 2000)

Minimizing the grand-canonical thermodynamical potential
$\Omega$ with respect to $\mu$ leads to three gap equations

%%%%%%%%%%%%%%%%%%%%%%%%
\section{Próximas ações}
%%%%%%%%%%%%%%%%%%%%%%%%

\begin{itemize}
	\item Verificar em algum livro (Glendenning, Shapiro) ou artigos de revisão motivações para o estudo da matéria de hádrons, quarks; Fazer meio que um geralzão da área;
	\item Verificar em artigos de revisão sobre colisões, com a ideia de fazer um geralzão de motivações para o estudo da matéria de hádrons, quarks;
\end{itemize}
