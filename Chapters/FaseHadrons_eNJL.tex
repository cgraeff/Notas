\chapter{Fase de Hadrons: eNJL}

\section{Simetria quiral}

Ver discussão em Ref.\cite{Vogl}
% Vogl, U., and W. Weise. - Progress in Particle and Nuclear Physics 27 (1991): 195-272 - The Nambu and Jona-Lasinio model: its implications for hadrons and nuclei
\section{Termodinâmica}

Temos que $dS$ ou $dU$ -- deve ser $dU$, pois o potencial químico é a ``quantidade de energia ganha ao se inserir mais uma partícula no sistema'' -- tem um termo $\mu dN$. Vamos trabalhar com densidade bariônica, mas essa densidade é só o número de partículas $N$ dividido pelo volume (as outras variáveis também serão trabalhadas divididas pelo volume). Logo, temos $\mu d\rho$. Por isso, temos que $\mu = d\epsilon/d\rho$, onde $\epsilon$ é a densidade de energia $\epsilon = E/V$. 

Glendenning\cite{Glendenning}:
Degenerate ideal Fermi gas: ideal pois não tem interações entre as partículas, degenerado pois todos os estados até uma certa energia -- a energia de Fermi -- estão ocupados. Nesse caso, a soma sobre todos os estados ocupados (que são autoestados de momento, pois não há interação) deve se dar sobre o momento. Isso pode ser escrito como a integral
\begin{equation}
	\int_0^{k_f} \frac{d^3k}{(2\pi^3)}.
\end{equation}
%
(pelo que lembro, é um cálculo realizado em um octante, contando quantos estados existem entre $p$ e $p+dp$ levando-se em conta que são ondas estacionárias em uma caixa de lado $L$. Nesse caso $k$ (que está associado ao momento) é um inteiro vezes o comprimento de onda dividido por dois. Tentar achar isso Ref [63] do Glendenning.

O gás pode ser considerado degenerado se $T \ll E_F = \sqrt{k_F^2 + m^2}$.

A densidade é obtida simplesmente somando os estados ocupados. A energia é calculada somando a energia de cada estado ocupado. A pressão eu não sei:
\begin{align}
	\rho &= \\
	\epsilon &= \\
	p &=
\end{align}

In thermodynamics, chemical potential, also known as partial molar free energy, is a form of potential energy that can be absorbed or released during a chemical reaction. It may also change during a phase transition. The chemical potential of a species in a mixture can be defined as the slope of the free energy of the system with respect to a change in the number of moles of just that species. Thus, it is the partial derivative of the free energy with respect to the amount of the species, all other species' concentrations in the mixture remaining constant, and at constant temperature. When pressure is constant, chemical potential is the partial molar Gibbs free energy. At chemical equilibrium or in phase equilibrium the total sum of chemical potentials is zero, as the free energy is at a minimum. \url{https://en.wikipedia.org/wiki/Chemical_potential}



%%%%%%%%%%%%%%%%%%%%%%%%%%%%%%%%%%%%%%%%%%%%
\section{Artigo: Pais, Menezes, Providência}
%%%%%%%%%%%%%%%%%%%%%%%%%%%%%%%%%%%%%%%%%%%%

Sobre o modelo\cite{Pais}:
\begin{quote}
The NJL model can be extended [...] to yield reasonable saturation properties of nuclear matter, the field $\psi$ being the nucleon field. An effective density dependent coupling constant is obtained if the following extended NJL (eNJL) Lagrangian density, which actually pushes chiral symmetry restoration to higher densities, is considered,
\begin{equation}\label{Eq:Lagrangiana_eNLJ_Pais}
\begin{split}
	\mathcal{L} &= \bar{\psi}(i\gamma^\mu\partial_\mu)\psi + G_s[(\bar{\psi}\psi)^2 + (\bar{\psi}i\gamma_5\vec{\tau}\psi)^2] - G_v(\bar{\psi}\gamma^\mu\psi)^2 \\
	&\phantom{=}- G_{sv}[(\bar{\psi}\psi)^2 + (\bar{\psi}i\gamma_5\vec{\tau}\psi)^2](\bar{\psi}\gamma^\mu\psi)^2 - G_\rho[(\bar{\psi}\gamma^\mu\vec{\tau}\psi)^2 + (\bar{\psi}\gamma_5\gamma^\mu\vec{\tau}\psi)^2] \\
	&\phantom{=}- G_{v\rho}(\bar{\psi}\gamma^\mu\psi)^2[(\bar{\psi}\gamma^\mu\vec{\tau}\psi)^2 + (\bar{\psi}\gamma_5\gamma^\mu\vec{\tau}\psi)^2].
\end{split}
\end{equation}
\end{quote}

Outras informações relevantes (ainda do artigo)\footnote{Qual é a função do termo em $G_{s\rho}$? Qual é o papel do termo em $G_v$? (Ver o que tem no Walecka).}:
\begin{itemize}
	\item Para a matéria nuclear, a degenerescência é $2 N_f$;
	\item O \emph{cutoff} $\Lambda$ é tal que a massa do nucleon no vácuo seja de 939 MeV, determinada variacionalmente;
	\item O termo proporcional $G_s$ simula uma repulsão de curto alcance entre os nucleons (chiral invariant);
	\item ``The term in $G_{sv}$ accounts for the density dependence of the scalar coupling. For the nuclear matter, the NJL model leads to binding, but the binding energy per particle does no have a minimum except at a rather high density where the nucleon mass is small or vanishing. The introduction of the $G_{sv}$ coupling term is required to correct this.''
	\item O termo proporcional a $G_\rho$ (isovetor-vetor) é incluido para descrever a matéria nuclear assimétrica (em isospin); 
\end{itemize}

A partir da lagrangiana \eqref{Eq:Lagrangiana_eNLJ_Pais}, é possível determinar\footnote{Como? Através de $\omega(T,\mu) = -\frac{T}{V} \ln \mathcal{Z}$?} o potencial termodinâmico\footnote{Potencial Grand-canônico, ou potencial de Landau.} por unidade de volume, dado por
\begin{equation}\label{Eq:potencial_termodinamico}
	\omega(\mu) = \varepsilon_{\rm{kin}} - G_s\rho_s^2 + G_v\rho^2 + G_{sv}\rho_s^2\rho^2 + G_\rho\rho_3^2 + G_{v\rho}\rho^2\rho_3^2 - \mu_p\rho_p - \mu_n\rho_n,
\end{equation}
%
onde
\begin{itemize}
	\item $\rho$ é a densidade bariônica, dada pela soma das densidades de nêutron e próton\footnote{São densidades numéricas de partículas, ou seja, representam o número de partículas por unidade de volume.}:
	\begin{equation}
		\rho = \rho_p + \rho_n.
	\end{equation}

	\item As densidades bariônicas de próton e nêutron são dadas por\footnote{De onde vem essa expressão para $\rho_i$. Explicar, explicar o momento de Fermi também.}
	\begin{equation}
		\rho_i = \int_0^{k_F^i}\frac{dp}{\pi^2}p^2; \qquad i = p,n; \quad k_F^i = \textrm{momento de Fermi},
	\end{equation}
	%
	ou, caso $\rho_i$ sejam conhecidos
	\begin{equation}\label{Eq:Mom_Fermi_a_partir_de_rho}
		p_F^i = \sqrt[3]{3\pi^2\rho_i}.
	\end{equation}
	
	\item $\mu_p$ e $\mu_n$ representam os potenciais químicos de próton e nêutron, respectivamente.
\end{itemize}

O termo cinético na expressão acima pode ser calculado através de (primeiro termo da Eq. (1) em \cite{PRC_68_035804_2003}, o resto é energia potencial)\footnote{Degenerescência: O 2 se refere às duas possibilidades de spin; Podemos ter um $N_f$ que representa o número de sabores. Acredito que o sinal não seja do termo cinético, então tem que retirar daqui. No programa em Fortran está definido sem esse sinal. Apesar de que o valor de $\varepsilon_{\rm{kin}}$ fica negativo sem esse sinal; energia cinética é estritamente positiva!.}
\begin{align}
	\varepsilon_{\rm{kin}} &= \langle\bar{\psi}(\vec{\gamma}\cdot\vec{p}\psi\rangle \\
	&= - 2 N_c\sum_i \int \frac{d^3p}{(2\pi)^3}\frac{p^2 + m_i M_i}{E_i}(n_{i-}-n_{i+})\theta(\Lambda^2 - p^2),
\end{align}
%
onde
\begin{itemize}
	\item A soma se dá sobre as espécies de partículas;
	\item $N_c$ representa o número de cores\footnote{No nosso caso, 1?};
	\item $\theta$ é a função degrau, $\Lambda$ é o \emph{cutoff};
	\item $n_{i\pm}$ são as funções de distribuição de Fermi para estados de energia positiva e negativa (respectivamente), dados por
	\begin{equation}
		n_{i\pm} = \frac{1}{1 + \exp(\pm[\beta(E_i\mp\mu_i)])}
	\end{equation}
	%
	onde $i = p, n$ (no nosso caso, no artigo é $u, d, s$) e $\beta = T^{-1}$
	\item $M_i$ é a massa constituinte do nucleon em questão (quark, no artigo).
	\item $E_i = \sqrt{p^2 + M_i^2}$
	\item $m_i$ no artigo são as massas (nuas?) dos quarks, e no nosso caso?\footnote{Ver isso.}
\end{itemize}

Se tomarmos $T \to 0$, temos que $n_{i-} \to 1$ e $n_{i+} \to 0$; Além disso, se o integrando só depende do módulo de $\vec{p}$, então (\cite{Glendenning}, p. 92)
\begin{equation}
	\int\frac{d^3p}{(2\pi)^3} \to \frac{1}{2\pi^2}\int p^2dp.
\end{equation}
%
Logo, temos
\begin{align}
	\varepsilon &= -2 N_c \frac{1}{2\pi^2}\sum_i \int p^2 dp \frac{p^2 + m_i M_i}{\sqrt{p^2 + M_i^2}} \theta(\Lambda^2 - p^2) \\
	&= -\frac{N_c}{\pi^2}\sum_i\left[\int \frac{p^4dp}{\sqrt{p^2 + M_i^2}}\theta(\Lambda^2 - p^2) + \int m_i M_i \frac{p^2 dp}{\sqrt{p^2 + M_i^2}}\theta(\Lambda^2 - p^2)\right]
\end{align}
%
Podemos utilizar as relações (\cite{Glendenning} p. 94\footnote{Na Ref. o primeiro termo da segunda expressão aparece sem o $k$ multiplicando, o que dimensionalmente está incorreto.})
\begin{align}
	\int \frac{k^4}{\sqrt{k^2 + m^2}} dk &= \frac{1}{4}\left[k^3\epsilon - \frac{3}{2} m^2k\epsilon + \frac{3}{2}m^4\ln\frac{\epsilon + k}{m} \right]\\
	\int \frac{k^2}{\sqrt{k^2 + m^2}} dk &= \frac{1}{2}\left[k\epsilon - \frac{1}{2}m^2\ln\frac{\epsilon + k}{m}\right] \label{Eq:Integ_momento_quad}
\end{align}
%
onde $\epsilon = \sqrt{k^2+m^2}$. Tomando o caso $m_i \to 0$\footnote{No prog. \texttt{eos\_enjl1-dens-assym- clean-rho-vr.f}: $\varepsilon \propto [F_2(M, k_F^i) - F_2(M, \Lambda)]$ ao invés de $\varepsilon \propto [F_2(M_i, \Lambda) - F_2(M_i, 0)]$; Isso se deve à retirada da contribuição do vácuo. Além disso, aparentemente os $M_i$ podem ser diferentes. No prog. são iguais, imagino que seja por considerarmos $m_n = m_p = m_N$.}, obtemos
\begin{equation}\label{Eq:Energia_kin}
	\varepsilon_{\rm{kin}} = -\frac{N_c}{\pi^2}\sum_i \Big[\underbrace{\frac{1}{8}\Big((2p^3 - 3M_i^2p)\sqrt{p^2 + M_i^2} + 3M_i^4\ln\frac{p + \sqrt{p^2 + M_i^2}}{M_i}\Big)}_{F_2(m,p)}\Big]_0^\Lambda
\end{equation}

A densidade escalar $\rho_s$ é dada por\footnote{De onde vem essa expressão?}
\begin{equation}\label{Eq:Dens_Escalar}
	\rho_s^i = \frac{M}{\pi^2}[F_0(M, p_F^i) - F_0(M, \Lambda)], \quad i = p, n,
\end{equation}
%
onde
\begin{equation}
	F_0(M, x) = \int_0^x \frac{dp}{\pi^2}\frac{p^2}{\sqrt{M^2 + p^2}}, \quad i = p, n.
\end{equation}
%
Utilizando a Equação~\eqref{Eq:Integ_momento_quad}, podemos reescrever a equação acima como\footnote{Esse $\pi^2$ no denominador está com cara de que não deveria estar aí. A parece um $\pi^2$ no denominador na definição de $\rho_s$ em termos de $F_0(M, x)$ e se eu deixar nos dois lugares, não há raízes para a equação do Gap. Por outro lado, mesmo que eu deixe, $\rho_s$ é menor que zero!}
\begin{equation}
	F_0(M, x) = \frac{1}{2\pi^2}\left[x\sqrt{x^2+M^2} - M^2 \ln \frac{x + \sqrt{x^2+M^2}}{M}\right].
\end{equation}

A massa\footnote{constituinte?} $M$ na equação acima é dada por\footnote{Essa equação é conhecida como \emph{Gap equation} (?).}
\begin{equation}\label{Eq:Gap}
	M = -2G_s\rho_s + 2G_{sv}\rho_s\rho^2,
\end{equation}
%
com $\rho_s = \rho_s^p + \rho_s^n$. Temos, portanto, uma interdependência entre as equações. Para que seja possível solucionar tais equações, podemos definir uma função $f(M)$ de tal forma que
\begin{equation}\label{Eq:Gap_zero}
	f(M) = M + 2G_s\rho_s - 2G_{sv}\rho_s\rho^2.
\end{equation}
%
Para solucionarmos a equação acima, basta utilizarmos uma rotina para encontrar zeros de funções, por exemplo biseção ou Newton-Raphson, encontrando o valor de $M$ para o qual $f(M) = 0$. A densidade escalar $\rho_s$ pode ser calculada através da expressão~\eqref{Eq:Dens_Escalar}\footnote{Na prática é mais fácil salvar o último valor de $\rho_s$ calculado pela rotina que tenta encontrar o zero da função $f(M)$.}.

Os potenciais químicos são dados por\footnote{Como essas expressões são calculadas?}
\begin{equation}\label{Eq:Potenciais_Quimicos}
	\mu_i = E_{p_F}^i + 2G_v\rho + 2G_{sv}\rho\rho_s^2 \pm 2G_\rho\rho_3+2G_{v\rho}\rho_3^2\rho \pm 2G_{v\rho}\rho^2\rho_3,
\end{equation}
%
onde $i = p,n$, os sinais superiores se referem ao caso de prótons, e $E_{p_F}^i = \sqrt{M^2 + (p_F^i)^2}$.

As equações de estado para pressão $P$ e densidade de energia $\varepsilon$ são dadas por\footnote{Como são calculadas?}
\begin{align}
	P &= -\omega(\mu) + \epsilon_0 \label{Eq:Pressao}\\
	\varepsilon &= -P + \mu_p\rho_p + \mu_n\rho_n. \label{Eq:Densidade_energia}
\end{align}

%%%%%%%%%%%%%%%%%%%%%%%%%%%%%
\section{Análise dimensional}
%%%%%%%%%%%%%%%%%%%%%%%%%%%%%

Devido ao fato de que $\hbar = c = 1$, adimensionais, e que não carregamos quaisquer unidades, é comum que algumas equações tenham dimensões discrepantes entre os membros esquerdo e direito, ou mesmo entre termos de um mesmo membro. Podemos acertar as dimensões multiplicando por potências de $\hbar c$. Nas unidades usuais em Física Nuclear, temos que $\hbar c = 197.326\rm{MeV}\cdot\rm{fm}$. Logo, todas as grandezas têm dimensões que envolvem MeV ou fm, ou uma combinação de ambos\sidenote{Note que de qualquer forma as unidades são atípicas, pois --~por exemplo~-- a unidade de massa é o eV, que na realidade tem dimensão de energia. Isso pode ser explicado através de $E^2 = p^2 c^2 + m^2 c^4$, onde assumimos que $c = 1$, adimensional.}:

\begin{itemize}
	\item Como $E^2 = p^2c^2 + m^2c^4$, temos $[E] = [m] = [p]$;
	\item Como $\rho = \bar{\psi}\gamma^0\psi$ é o número de partículas por unidade de volume, temos que $[\rho] = \rm{fm}^{-3}$
	\item O item acima implica que $[\psi] = \rm{fm}^{-3/2}$. Consequentemente, $[\rho_s] = [\bar{\psi}\psi] = \rm{fm}^{-3}$. 
	\item Como o potencial químico esta relacionado à variação de energia ao se adicionar ou retirar partículas do sistema, sua unidade é a de energia (MeV).
	\item O potencial termodinâmico $\omega$ é o potencial grande-canônico $\Omega$\footnote{Potencial de Landau} por unidade de volume, portanto tem dimensão de energia por unidade de volume ($\rm{MeV}\cdot\rm{fm}^{-3}$), assim como a densidade de energia $\varepsilon$ e a pressão $P$.
\end{itemize}

Dessa forma, as equações discutidas acima precisam ter suas unidades checadas e --~quando necessário~-- corrigidas, de forma a ficarem consistentes. Temos então:
\begin{fullwidth}
\begin{itemize}

\item O potencial termodinâmico (por unidade de volume) $\omega$ deve ter dimensão energia por volume (no nosso caso, $\rm{MeV}/\rm{fm}^3$):
\begin{equation}
	\underbrace{\omega}_{\frac{\rm{MeV}}{\rm{fm}^3}} = \underbrace{\varepsilon_{\rm{kin}}}_{\frac{\rm{MeV}}{\rm{fm}^3}} - \underbrace{G_s}_{\rm{fm}^2}\underbrace{\rho_s^2}_{\rm{fm}^{-6}} + \underbrace{G_v}_{\rm{fm}^2}\underbrace{\rho^2}_{\rm{fm}^{-6}} + \underbrace{G_{sv}}_{\rm{fm}^8}\underbrace{\rho_s^2\rho^2}_{\rm{fm}^{-12}} + \underbrace{G_\rho}_{\rm{fm}^2}\underbrace{\rho_3^2}_{\rm{fm}^{-6}} + \underbrace{G_{v\rho}}_{\rm{fm}^2}\underbrace{\rho^2\rho_3^2}_{\rm{fm}^{-12}} + \underbrace{\mu_p}_{\rm{MeV}}\underbrace{\rho_p}_{\rm{fm}^{-3}} + \underbrace{\mu_n}_{\rm{MeV}}\underbrace{\rho_n}_{\rm{fm}^{-3}}.
\end{equation}
%
Os termos envolvendo as constantes $G_i$ necessitam ser multiplicados por $\hbar c$, cuja dimensão é $\rm{MeV}\cdot\rm{fm}$, resultando na dimensão $\rm{MeV}/\rm{fm}^3$ para tais termos.

\item A pressão deve ter unidade de energia por unidade de volume ($\rm{MeV}/\rm{fm}^3$):
\begin{equation}
	\underbrace{P}_{\frac{\rm{MeV}}{\rm{fm}^3}} = - \underbrace{\omega}_{\frac{\rm{MeV}}{\rm{fm}^3}} + \underbrace{\varepsilon_0}_{\frac{\rm{MeV}}{\rm{fm}^3}},
\end{equation}
%
assim como a densidade de energia
\begin{equation}
	\underbrace{\varepsilon}_{\frac{\rm{MeV}}{\rm{fm}^3}} = - \underbrace{P}_{\frac{\rm{MeV}}{\rm{fm}^3}} + \underbrace{\mu_p}_{\rm{MeV}}\underbrace{\rho_p}_{\rm{fm}^{-3}} + \underbrace{\mu_n}_{\rm{MeV}}\underbrace{\rho_n}_{\rm{fm}^{-3}},
\end{equation}
%
e podemos ver que ambas estão com todas as dimensões corretas.

\item A equação para o cálculo da massa
\begin{equation}
	\underbrace{M}_{\rm{MeV}} = -2 \underbrace{G_s}_{\rm{fm}^2} \underbrace{\rho_s}_{\rm{fm}^{-3}} + 2 \underbrace{G_{sv}}_{\rm{fm}^8}\underbrace{\rho_s\rho^2}_{\rm{fm}^{-9}}
\end{equation}
%
necessita ser multiplicada por $\hbar c$ no lado direito.

\item A equação para a densidade bariônica
\begin{equation}
	\underbrace{\rho_i}_{\rm{fm}^{-3}} = \underbrace{\int_0^{p_F^i} \frac{p^2 dp}{\pi^2}}_{\rm{MeV}^3} = \underbrace{\frac{1}{3\pi^2}p^3\Big|_0^{p_F^i}}_{\rm{MeV}^3},
\end{equation}
%
assim como a equação para a densidade escalar
\begin{equation}
	\underbrace{\rho_s^i}_{\rm{fm}^{-3}} = \underbrace{\frac{M}{\pi^2}[F_0(M,p_F^i) - F_0(M, \Lambda)]}_{\rm{MeV}^3},
\end{equation}
%
onde usamos (vide Eq.~\eqref{Eq:Integ_momento_quad})
\begin{equation}
	F_0(m,p) = \underbrace{\int_0^x\frac{dp}{\pi^2} \frac{p^2}{\sqrt{p^2 + m^2}}}_{\rm{MeV}^2},
\end{equation}
%
devem ser multiplicadas por $(\hbar c)^{-3}$.

\item Para o potencial químico temos
\begin{equation}
	\underbrace{\mu_i}_{\rm{MeV}} = \underbrace{E_{p_F}^i}_{\rm{MeV}} +~2 \underbrace{G_v}_{\rm{fm}^2}\underbrace{\rho}_{\rm{fm^{-3}}} +~2 \underbrace{G_{sv}}_{\rm{fm}^8}\underbrace{\rho\rho_s^2}_{\rm{fm^{-9}}} \pm~2 \underbrace{G_\rho}_{\rm{fm}^2}\underbrace{\rho_3}_{\rm{fm^{-3}}} +~2 \underbrace{G_{v\rho}}_{\rm{fm}^8}\underbrace{\rho_3^2\rho}_{\rm{fm^{-9}}} \pm~2 \underbrace{G_{v\rho}}_{\rm{fm}^8}\underbrace{\rho^2\rho_3}_{\rm{fm^{-9}}},
\end{equation}
%
onde
\begin{equation}
	\underbrace{E_{p_F}^i}_{\rm{MeV}} = \sqrt{M^2 + p_F^2}; \quad [M] = [p_F] = \rm{MeV}.
\end{equation}
%
Portanto, verificamos que os termos proporcionais a $G_i$ devem ser multiplicados por $\hbar c$.

\item O termo cinético da energia é dado pela Equação~\eqref{Eq:Energia_kin}. O termo definido como $F_2(m, p)$ tem dimensão de $\rm{MeV}^4$, já que todos as parcelas são produto de $m$, $p$, e $\epsilon = \sqrt{p^2+m_i^2}$. Além disso, o argumento do logarítimo é adimensional. No entanto, como $\varepsilon_{\rm{kin}}$ é uma densidade de energia, a dimensão correta é $\rm{MeV} / \rm{fm}^3$. Portanto, é necessário multiplicar a expressão para $\varepsilon_{\rm{kin}}$ por $(\hbar c)^{-3}$.
\end{itemize}
\end{fullwidth}

%%%%%%%%%%%%%%%%%%%%%%%%%%%%%%%%%%%%%%%%%%%%%%%
\section{Solução da equação para $M$, $\rho_s$}
%%%%%%%%%%%%%%%%%%%%%%%%%%%%%%%%%%%%%%%%%%%%%%%

A solução da Equação~\eqref{Eq:Gap} é encontrada zerando a Equação~\eqref{Eq:Gap_zero}. No entanto, essa última depende de $\rho$ e dos momentos de Fermi para próton e nêutron (o que é uma dependência indireta de $\rho$ e da fração de prótons). Assim, a forma da função pode se alterar de acordo com os valores de tais variáveis. A Figura~\ref{Fig:Gap_zero_graph} mostra curvas de $f(M)$ para diferentes valores de $\rho$.

\begin{figure*}
	\begin{tikzpicture}[gnuplot]
%% generated with GNUPLOT 5.0p2 (Lua 5.2; terminal rev. 99, script rev. 100)
%% Fri Mar 18 15:48:26 2016
\path (0.000,0.000) rectangle (14.000,9.000);
\gpcolor{color=gp lt color border}
\gpsetlinetype{gp lt border}
\gpsetdashtype{gp dt solid}
\gpsetlinewidth{1.00}
\draw[gp path] (1.320,1.680)--(1.500,1.680);
\draw[gp path] (13.447,1.680)--(13.267,1.680);
\node[gp node right] at (1.136,1.680) {$0$};
\draw[gp path] (1.320,3.070)--(1.500,3.070);
\draw[gp path] (13.447,3.070)--(13.267,3.070);
\node[gp node right] at (1.136,3.070) {$100$};
\draw[gp path] (1.320,4.460)--(1.500,4.460);
\draw[gp path] (13.447,4.460)--(13.267,4.460);
\node[gp node right] at (1.136,4.460) {$200$};
\draw[gp path] (1.320,5.851)--(1.500,5.851);
\draw[gp path] (13.447,5.851)--(13.267,5.851);
\node[gp node right] at (1.136,5.851) {$300$};
\draw[gp path] (1.320,7.241)--(1.500,7.241);
\draw[gp path] (13.447,7.241)--(13.267,7.241);
\node[gp node right] at (1.136,7.241) {$400$};
\draw[gp path] (1.320,8.631)--(1.500,8.631);
\draw[gp path] (13.447,8.631)--(13.267,8.631);
\node[gp node right] at (1.136,8.631) {$500$};
\draw[gp path] (1.320,0.985)--(1.320,1.165);
\draw[gp path] (1.320,8.631)--(1.320,8.451);
\node[gp node center] at (1.320,0.677) {$0$};
\draw[gp path] (3.745,0.985)--(3.745,1.165);
\draw[gp path] (3.745,8.631)--(3.745,8.451);
\node[gp node center] at (3.745,0.677) {$100$};
\draw[gp path] (6.171,0.985)--(6.171,1.165);
\draw[gp path] (6.171,8.631)--(6.171,8.451);
\node[gp node center] at (6.171,0.677) {$200$};
\draw[gp path] (8.596,0.985)--(8.596,1.165);
\draw[gp path] (8.596,8.631)--(8.596,8.451);
\node[gp node center] at (8.596,0.677) {$300$};
\draw[gp path] (11.022,0.985)--(11.022,1.165);
\draw[gp path] (11.022,8.631)--(11.022,8.451);
\node[gp node center] at (11.022,0.677) {$400$};
\draw[gp path] (13.447,0.985)--(13.447,1.165);
\draw[gp path] (13.447,8.631)--(13.447,8.451);
\node[gp node center] at (13.447,0.677) {$500$};
\draw[gp path] (1.320,8.631)--(1.320,0.985)--(13.447,0.985)--(13.447,8.631)--cycle;
\node[gp node center,rotate=-270] at (0.246,4.808) {$p_F$};
\node[gp node center] at (7.383,0.215) {$\mu_R$};
\node[gp node left] at (2.788,8.297) {$p_F = \sqrt{\mu_R^2 - m^2}\theta(\mu_R^2 - m^2), m = 100$};
\gpcolor{rgb color={0.580,0.000,0.827}}
\gpsetlinewidth{3.00}
\draw[gp path] (1.688,8.297)--(2.604,8.297);
\draw[gp path] (1.320,1.680)--(1.332,1.680)--(1.344,1.680)--(1.356,1.680)--(1.369,1.680)%
  --(1.381,1.680)--(1.393,1.680)--(1.405,1.680)--(1.417,1.680)--(1.429,1.680)--(1.441,1.680)%
  --(1.453,1.680)--(1.466,1.680)--(1.478,1.680)--(1.490,1.680)--(1.502,1.680)--(1.514,1.680)%
  --(1.526,1.680)--(1.538,1.680)--(1.550,1.680)--(1.563,1.680)--(1.575,1.680)--(1.587,1.680)%
  --(1.599,1.680)--(1.611,1.680)--(1.623,1.680)--(1.635,1.680)--(1.647,1.680)--(1.660,1.680)%
  --(1.672,1.680)--(1.684,1.680)--(1.696,1.680)--(1.708,1.680)--(1.720,1.680)--(1.732,1.680)%
  --(1.744,1.680)--(1.757,1.680)--(1.769,1.680)--(1.781,1.680)--(1.793,1.680)--(1.805,1.680)%
  --(1.817,1.680)--(1.829,1.680)--(1.841,1.680)--(1.854,1.680)--(1.866,1.680)--(1.878,1.680)%
  --(1.890,1.680)--(1.902,1.680)--(1.914,1.680)--(1.926,1.680)--(1.938,1.680)--(1.951,1.680)%
  --(1.963,1.680)--(1.975,1.680)--(1.987,1.680)--(1.999,1.680)--(2.011,1.680)--(2.023,1.680)%
  --(2.035,1.680)--(2.048,1.680)--(2.060,1.680)--(2.072,1.680)--(2.084,1.680)--(2.096,1.680)%
  --(2.108,1.680)--(2.120,1.680)--(2.133,1.680)--(2.145,1.680)--(2.157,1.680)--(2.169,1.680)%
  --(2.181,1.680)--(2.193,1.680)--(2.205,1.680)--(2.217,1.680)--(2.230,1.680)--(2.242,1.680)%
  --(2.254,1.680)--(2.266,1.680)--(2.278,1.680)--(2.290,1.680)--(2.302,1.680)--(2.314,1.680)%
  --(2.327,1.680)--(2.339,1.680)--(2.351,1.680)--(2.363,1.680)--(2.375,1.680)--(2.387,1.680)%
  --(2.399,1.680)--(2.411,1.680)--(2.424,1.680)--(2.436,1.680)--(2.448,1.680)--(2.460,1.680)%
  --(2.472,1.680)--(2.484,1.680)--(2.496,1.680)--(2.508,1.680)--(2.521,1.680)--(2.533,1.680)%
  --(2.545,1.680)--(2.557,1.680)--(2.569,1.680)--(2.581,1.680)--(2.593,1.680)--(2.605,1.680)%
  --(2.618,1.680)--(2.630,1.680)--(2.642,1.680)--(2.654,1.680)--(2.666,1.680)--(2.678,1.680)%
  --(2.690,1.680)--(2.702,1.680)--(2.715,1.680)--(2.727,1.680)--(2.739,1.680)--(2.751,1.680)%
  --(2.763,1.680)--(2.775,1.680)--(2.787,1.680)--(2.799,1.680)--(2.812,1.680)--(2.824,1.680)%
  --(2.836,1.680)--(2.848,1.680)--(2.860,1.680)--(2.872,1.680)--(2.884,1.680)--(2.897,1.680)%
  --(2.909,1.680)--(2.921,1.680)--(2.933,1.680)--(2.945,1.680)--(2.957,1.680)--(2.969,1.680)%
  --(2.981,1.680)--(2.994,1.680)--(3.006,1.680)--(3.018,1.680)--(3.030,1.680)--(3.042,1.680)%
  --(3.054,1.680)--(3.066,1.680)--(3.078,1.680)--(3.091,1.680)--(3.103,1.680)--(3.115,1.680)%
  --(3.127,1.680)--(3.139,1.680)--(3.151,1.680)--(3.163,1.680)--(3.175,1.680)--(3.188,1.680)%
  --(3.200,1.680)--(3.212,1.680)--(3.224,1.680)--(3.236,1.680)--(3.248,1.680)--(3.260,1.680)%
  --(3.272,1.680)--(3.285,1.680)--(3.297,1.680)--(3.309,1.680)--(3.321,1.680)--(3.333,1.680)%
  --(3.345,1.680)--(3.357,1.680)--(3.369,1.680)--(3.382,1.680)--(3.394,1.680)--(3.406,1.680)%
  --(3.418,1.680)--(3.430,1.680)--(3.442,1.680)--(3.454,1.680)--(3.466,1.680)--(3.479,1.680)%
  --(3.491,1.680)--(3.503,1.680)--(3.515,1.680)--(3.527,1.680)--(3.539,1.680)--(3.551,1.680)%
  --(3.563,1.680)--(3.576,1.680)--(3.588,1.680)--(3.600,1.680)--(3.612,1.680)--(3.624,1.680)%
  --(3.636,1.680)--(3.648,1.680)--(3.661,1.680)--(3.673,1.680)--(3.685,1.680)--(3.697,1.680)%
  --(3.709,1.680)--(3.721,1.680)--(3.733,1.680)--(3.745,1.680)--(3.758,1.819)--(3.770,1.877)%
  --(3.782,1.922)--(3.794,1.960)--(3.806,1.993)--(3.818,2.023)--(3.830,2.051)--(3.842,2.077)%
  --(3.855,2.102)--(3.867,2.125)--(3.879,2.147)--(3.891,2.169)--(3.903,2.189)--(3.915,2.209)%
  --(3.927,2.229)--(3.939,2.247)--(3.952,2.265)--(3.964,2.283)--(3.976,2.300)--(3.988,2.317)%
  --(4.000,2.334)--(4.012,2.350)--(4.024,2.366)--(4.036,2.381)--(4.049,2.397)--(4.061,2.412)%
  --(4.073,2.426)--(4.085,2.441)--(4.097,2.455)--(4.109,2.470)--(4.121,2.484)--(4.133,2.497)%
  --(4.146,2.511)--(4.158,2.524)--(4.170,2.538)--(4.182,2.551)--(4.194,2.564)--(4.206,2.577)%
  --(4.218,2.590)--(4.230,2.602)--(4.243,2.615)--(4.255,2.627)--(4.267,2.639)--(4.279,2.652)%
  --(4.291,2.664)--(4.303,2.676)--(4.315,2.688)--(4.327,2.699)--(4.340,2.711)--(4.352,2.723)%
  --(4.364,2.734)--(4.376,2.746)--(4.388,2.757)--(4.400,2.768)--(4.412,2.780)--(4.425,2.791)%
  --(4.437,2.802)--(4.449,2.813)--(4.461,2.824)--(4.473,2.835)--(4.485,2.846)--(4.497,2.857)%
  --(4.509,2.867)--(4.522,2.878)--(4.534,2.889)--(4.546,2.899)--(4.558,2.910)--(4.570,2.920)%
  --(4.582,2.931)--(4.594,2.941)--(4.606,2.951)--(4.619,2.961)--(4.631,2.972)--(4.643,2.982)%
  --(4.655,2.992)--(4.667,3.002)--(4.679,3.012)--(4.691,3.022)--(4.703,3.032)--(4.716,3.042)%
  --(4.728,3.052)--(4.740,3.062)--(4.752,3.072)--(4.764,3.082)--(4.776,3.091)--(4.788,3.101)%
  --(4.800,3.111)--(4.813,3.121)--(4.825,3.130)--(4.837,3.140)--(4.849,3.149)--(4.861,3.159)%
  --(4.873,3.168)--(4.885,3.178)--(4.897,3.187)--(4.910,3.197)--(4.922,3.206)--(4.934,3.216)%
  --(4.946,3.225)--(4.958,3.234)--(4.970,3.244)--(4.982,3.253)--(4.994,3.262)--(5.007,3.271)%
  --(5.019,3.281)--(5.031,3.290)--(5.043,3.299)--(5.055,3.308)--(5.067,3.317)--(5.079,3.326)%
  --(5.091,3.336)--(5.104,3.345)--(5.116,3.354)--(5.128,3.363)--(5.140,3.372)--(5.152,3.381)%
  --(5.164,3.390)--(5.176,3.399)--(5.189,3.408)--(5.201,3.416)--(5.213,3.425)--(5.225,3.434)%
  --(5.237,3.443)--(5.249,3.452)--(5.261,3.461)--(5.273,3.470)--(5.286,3.478)--(5.298,3.487)%
  --(5.310,3.496)--(5.322,3.505)--(5.334,3.513)--(5.346,3.522)--(5.358,3.531)--(5.370,3.539)%
  --(5.383,3.548)--(5.395,3.557)--(5.407,3.565)--(5.419,3.574)--(5.431,3.583)--(5.443,3.591)%
  --(5.455,3.600)--(5.467,3.608)--(5.480,3.617)--(5.492,3.626)--(5.504,3.634)--(5.516,3.643)%
  --(5.528,3.651)--(5.540,3.660)--(5.552,3.668)--(5.564,3.677)--(5.577,3.685)--(5.589,3.694)%
  --(5.601,3.702)--(5.613,3.710)--(5.625,3.719)--(5.637,3.727)--(5.649,3.736)--(5.661,3.744)%
  --(5.674,3.752)--(5.686,3.761)--(5.698,3.769)--(5.710,3.777)--(5.722,3.786)--(5.734,3.794)%
  --(5.746,3.802)--(5.758,3.811)--(5.771,3.819)--(5.783,3.827)--(5.795,3.836)--(5.807,3.844)%
  --(5.819,3.852)--(5.831,3.860)--(5.843,3.869)--(5.855,3.877)--(5.868,3.885)--(5.880,3.893)%
  --(5.892,3.901)--(5.904,3.910)--(5.916,3.918)--(5.928,3.926)--(5.940,3.934)--(5.953,3.942)%
  --(5.965,3.950)--(5.977,3.959)--(5.989,3.967)--(6.001,3.975)--(6.013,3.983)--(6.025,3.991)%
  --(6.037,3.999)--(6.050,4.007)--(6.062,4.015)--(6.074,4.024)--(6.086,4.032)--(6.098,4.040)%
  --(6.110,4.048)--(6.122,4.056)--(6.134,4.064)--(6.147,4.072)--(6.159,4.080)--(6.171,4.088)%
  --(6.183,4.096)--(6.195,4.104)--(6.207,4.112)--(6.219,4.120)--(6.231,4.128)--(6.244,4.136)%
  --(6.256,4.144)--(6.268,4.152)--(6.280,4.160)--(6.292,4.168)--(6.304,4.176)--(6.316,4.184)%
  --(6.328,4.192)--(6.341,4.200)--(6.353,4.208)--(6.365,4.216)--(6.377,4.223)--(6.389,4.231)%
  --(6.401,4.239)--(6.413,4.247)--(6.425,4.255)--(6.438,4.263)--(6.450,4.271)--(6.462,4.279)%
  --(6.474,4.287)--(6.486,4.295)--(6.498,4.302)--(6.510,4.310)--(6.522,4.318)--(6.535,4.326)%
  --(6.547,4.334)--(6.559,4.342)--(6.571,4.350)--(6.583,4.357)--(6.595,4.365)--(6.607,4.373)%
  --(6.619,4.381)--(6.632,4.389)--(6.644,4.396)--(6.656,4.404)--(6.668,4.412)--(6.680,4.420)%
  --(6.692,4.428)--(6.704,4.435)--(6.717,4.443)--(6.729,4.451)--(6.741,4.459)--(6.753,4.467)%
  --(6.765,4.474)--(6.777,4.482)--(6.789,4.490)--(6.801,4.498)--(6.814,4.505)--(6.826,4.513)%
  --(6.838,4.521)--(6.850,4.529)--(6.862,4.536)--(6.874,4.544)--(6.886,4.552)--(6.898,4.559)%
  --(6.911,4.567)--(6.923,4.575)--(6.935,4.583)--(6.947,4.590)--(6.959,4.598)--(6.971,4.606)%
  --(6.983,4.613)--(6.995,4.621)--(7.008,4.629)--(7.020,4.636)--(7.032,4.644)--(7.044,4.652)%
  --(7.056,4.660)--(7.068,4.667)--(7.080,4.675)--(7.092,4.682)--(7.105,4.690)--(7.117,4.698)%
  --(7.129,4.705)--(7.141,4.713)--(7.153,4.721)--(7.165,4.728)--(7.177,4.736)--(7.189,4.744)%
  --(7.202,4.751)--(7.214,4.759)--(7.226,4.767)--(7.238,4.774)--(7.250,4.782)--(7.262,4.789)%
  --(7.274,4.797)--(7.286,4.805)--(7.299,4.812)--(7.311,4.820)--(7.323,4.827)--(7.335,4.835)%
  --(7.347,4.843)--(7.359,4.850)--(7.371,4.858)--(7.384,4.865)--(7.396,4.873)--(7.408,4.881)%
  --(7.420,4.888)--(7.432,4.896)--(7.444,4.903)--(7.456,4.911)--(7.468,4.918)--(7.481,4.926)%
  --(7.493,4.934)--(7.505,4.941)--(7.517,4.949)--(7.529,4.956)--(7.541,4.964)--(7.553,4.971)%
  --(7.565,4.979)--(7.578,4.986)--(7.590,4.994)--(7.602,5.001)--(7.614,5.009)--(7.626,5.017)%
  --(7.638,5.024)--(7.650,5.032)--(7.662,5.039)--(7.675,5.047)--(7.687,5.054)--(7.699,5.062)%
  --(7.711,5.069)--(7.723,5.077)--(7.735,5.084)--(7.747,5.092)--(7.759,5.099)--(7.772,5.107)%
  --(7.784,5.114)--(7.796,5.122)--(7.808,5.129)--(7.820,5.137)--(7.832,5.144)--(7.844,5.152)%
  --(7.856,5.159)--(7.869,5.167)--(7.881,5.174)--(7.893,5.182)--(7.905,5.189)--(7.917,5.197)%
  --(7.929,5.204)--(7.941,5.212)--(7.953,5.219)--(7.966,5.226)--(7.978,5.234)--(7.990,5.241)%
  --(8.002,5.249)--(8.014,5.256)--(8.026,5.264)--(8.038,5.271)--(8.050,5.279)--(8.063,5.286)%
  --(8.075,5.294)--(8.087,5.301)--(8.099,5.308)--(8.111,5.316)--(8.123,5.323)--(8.135,5.331)%
  --(8.148,5.338)--(8.160,5.346)--(8.172,5.353)--(8.184,5.361)--(8.196,5.368)--(8.208,5.375)%
  --(8.220,5.383)--(8.232,5.390)--(8.245,5.398)--(8.257,5.405)--(8.269,5.412)--(8.281,5.420)%
  --(8.293,5.427)--(8.305,5.435)--(8.317,5.442)--(8.329,5.450)--(8.342,5.457)--(8.354,5.464)%
  --(8.366,5.472)--(8.378,5.479)--(8.390,5.487)--(8.402,5.494)--(8.414,5.501)--(8.426,5.509)%
  --(8.439,5.516)--(8.451,5.524)--(8.463,5.531)--(8.475,5.538)--(8.487,5.546)--(8.499,5.553)%
  --(8.511,5.560)--(8.523,5.568)--(8.536,5.575)--(8.548,5.583)--(8.560,5.590)--(8.572,5.597)%
  --(8.584,5.605)--(8.596,5.612)--(8.608,5.619)--(8.620,5.627)--(8.633,5.634)--(8.645,5.642)%
  --(8.657,5.649)--(8.669,5.656)--(8.681,5.664)--(8.693,5.671)--(8.705,5.678)--(8.717,5.686)%
  --(8.730,5.693)--(8.742,5.700)--(8.754,5.708)--(8.766,5.715)--(8.778,5.723)--(8.790,5.730)%
  --(8.802,5.737)--(8.814,5.745)--(8.827,5.752)--(8.839,5.759)--(8.851,5.767)--(8.863,5.774)%
  --(8.875,5.781)--(8.887,5.789)--(8.899,5.796)--(8.912,5.803)--(8.924,5.811)--(8.936,5.818)%
  --(8.948,5.825)--(8.960,5.833)--(8.972,5.840)--(8.984,5.847)--(8.996,5.855)--(9.009,5.862)%
  --(9.021,5.869)--(9.033,5.877)--(9.045,5.884)--(9.057,5.891)--(9.069,5.899)--(9.081,5.906)%
  --(9.093,5.913)--(9.106,5.921)--(9.118,5.928)--(9.130,5.935)--(9.142,5.942)--(9.154,5.950)%
  --(9.166,5.957)--(9.178,5.964)--(9.190,5.972)--(9.203,5.979)--(9.215,5.986)--(9.227,5.994)%
  --(9.239,6.001)--(9.251,6.008)--(9.263,6.016)--(9.275,6.023)--(9.287,6.030)--(9.300,6.037)%
  --(9.312,6.045)--(9.324,6.052)--(9.336,6.059)--(9.348,6.067)--(9.360,6.074)--(9.372,6.081)%
  --(9.384,6.088)--(9.397,6.096)--(9.409,6.103)--(9.421,6.110)--(9.433,6.118)--(9.445,6.125)%
  --(9.457,6.132)--(9.469,6.139)--(9.481,6.147)--(9.494,6.154)--(9.506,6.161)--(9.518,6.169)%
  --(9.530,6.176)--(9.542,6.183)--(9.554,6.190)--(9.566,6.198)--(9.578,6.205)--(9.591,6.212)%
  --(9.603,6.219)--(9.615,6.227)--(9.627,6.234)--(9.639,6.241)--(9.651,6.249)--(9.663,6.256)%
  --(9.676,6.263)--(9.688,6.270)--(9.700,6.278)--(9.712,6.285)--(9.724,6.292)--(9.736,6.299)%
  --(9.748,6.307)--(9.760,6.314)--(9.773,6.321)--(9.785,6.328)--(9.797,6.336)--(9.809,6.343)%
  --(9.821,6.350)--(9.833,6.357)--(9.845,6.365)--(9.857,6.372)--(9.870,6.379)--(9.882,6.386)%
  --(9.894,6.394)--(9.906,6.401)--(9.918,6.408)--(9.930,6.415)--(9.942,6.423)--(9.954,6.430)%
  --(9.967,6.437)--(9.979,6.444)--(9.991,6.452)--(10.003,6.459)--(10.015,6.466)--(10.027,6.473)%
  --(10.039,6.481)--(10.051,6.488)--(10.064,6.495)--(10.076,6.502)--(10.088,6.509)--(10.100,6.517)%
  --(10.112,6.524)--(10.124,6.531)--(10.136,6.538)--(10.148,6.546)--(10.161,6.553)--(10.173,6.560)%
  --(10.185,6.567)--(10.197,6.575)--(10.209,6.582)--(10.221,6.589)--(10.233,6.596)--(10.245,6.603)%
  --(10.258,6.611)--(10.270,6.618)--(10.282,6.625)--(10.294,6.632)--(10.306,6.640)--(10.318,6.647)%
  --(10.330,6.654)--(10.342,6.661)--(10.355,6.668)--(10.367,6.676)--(10.379,6.683)--(10.391,6.690)%
  --(10.403,6.697)--(10.415,6.704)--(10.427,6.712)--(10.440,6.719)--(10.452,6.726)--(10.464,6.733)%
  --(10.476,6.741)--(10.488,6.748)--(10.500,6.755)--(10.512,6.762)--(10.524,6.769)--(10.537,6.777)%
  --(10.549,6.784)--(10.561,6.791)--(10.573,6.798)--(10.585,6.805)--(10.597,6.813)--(10.609,6.820)%
  --(10.621,6.827)--(10.634,6.834)--(10.646,6.841)--(10.658,6.849)--(10.670,6.856)--(10.682,6.863)%
  --(10.694,6.870)--(10.706,6.877)--(10.718,6.885)--(10.731,6.892)--(10.743,6.899)--(10.755,6.906)%
  --(10.767,6.913)--(10.779,6.921)--(10.791,6.928)--(10.803,6.935)--(10.815,6.942)--(10.828,6.949)%
  --(10.840,6.956)--(10.852,6.964)--(10.864,6.971)--(10.876,6.978)--(10.888,6.985)--(10.900,6.992)%
  --(10.912,7.000)--(10.925,7.007)--(10.937,7.014)--(10.949,7.021)--(10.961,7.028)--(10.973,7.036)%
  --(10.985,7.043)--(10.997,7.050)--(11.009,7.057)--(11.022,7.064)--(11.034,7.071)--(11.046,7.079)%
  --(11.058,7.086)--(11.070,7.093)--(11.082,7.100)--(11.094,7.107)--(11.106,7.114)--(11.119,7.122)%
  --(11.131,7.129)--(11.143,7.136)--(11.155,7.143)--(11.167,7.150)--(11.179,7.158)--(11.191,7.165)%
  --(11.204,7.172)--(11.216,7.179)--(11.228,7.186)--(11.240,7.193)--(11.252,7.201)--(11.264,7.208)%
  --(11.276,7.215)--(11.288,7.222)--(11.301,7.229)--(11.313,7.236)--(11.325,7.244)--(11.337,7.251)%
  --(11.349,7.258)--(11.361,7.265)--(11.373,7.272)--(11.385,7.279)--(11.398,7.287)--(11.410,7.294)%
  --(11.422,7.301)--(11.434,7.308)--(11.446,7.315)--(11.458,7.322)--(11.470,7.329)--(11.482,7.337)%
  --(11.495,7.344)--(11.507,7.351)--(11.519,7.358)--(11.531,7.365)--(11.543,7.372)--(11.555,7.380)%
  --(11.567,7.387)--(11.579,7.394)--(11.592,7.401)--(11.604,7.408)--(11.616,7.415)--(11.628,7.422)%
  --(11.640,7.430)--(11.652,7.437)--(11.664,7.444)--(11.676,7.451)--(11.689,7.458)--(11.701,7.465)%
  --(11.713,7.473)--(11.725,7.480)--(11.737,7.487)--(11.749,7.494)--(11.761,7.501)--(11.773,7.508)%
  --(11.786,7.515)--(11.798,7.523)--(11.810,7.530)--(11.822,7.537)--(11.834,7.544)--(11.846,7.551)%
  --(11.858,7.558)--(11.870,7.565)--(11.883,7.573)--(11.895,7.580)--(11.907,7.587)--(11.919,7.594)%
  --(11.931,7.601)--(11.943,7.608)--(11.955,7.615)--(11.968,7.623)--(11.980,7.630)--(11.992,7.637)%
  --(12.004,7.644)--(12.016,7.651)--(12.028,7.658)--(12.040,7.665)--(12.052,7.673)--(12.065,7.680)%
  --(12.077,7.687)--(12.089,7.694)--(12.101,7.701)--(12.113,7.708)--(12.125,7.715)--(12.137,7.722)%
  --(12.149,7.730)--(12.162,7.737)--(12.174,7.744)--(12.186,7.751)--(12.198,7.758)--(12.210,7.765)%
  --(12.222,7.772)--(12.234,7.779)--(12.246,7.787)--(12.259,7.794)--(12.271,7.801)--(12.283,7.808)%
  --(12.295,7.815)--(12.307,7.822)--(12.319,7.829)--(12.331,7.837)--(12.343,7.844)--(12.356,7.851)%
  --(12.368,7.858)--(12.380,7.865)--(12.392,7.872)--(12.404,7.879)--(12.416,7.886)--(12.428,7.894)%
  --(12.440,7.901)--(12.453,7.908)--(12.465,7.915)--(12.477,7.922)--(12.489,7.929)--(12.501,7.936)%
  --(12.513,7.943)--(12.525,7.950)--(12.537,7.958)--(12.550,7.965)--(12.562,7.972)--(12.574,7.979)%
  --(12.586,7.986)--(12.598,7.993)--(12.610,8.000)--(12.622,8.007)--(12.634,8.015)--(12.647,8.022)%
  --(12.659,8.029)--(12.671,8.036)--(12.683,8.043)--(12.695,8.050)--(12.707,8.057)--(12.719,8.064)%
  --(12.732,8.071)--(12.744,8.079)--(12.756,8.086)--(12.768,8.093)--(12.780,8.100)--(12.792,8.107)%
  --(12.804,8.114)--(12.816,8.121)--(12.829,8.128)--(12.841,8.135)--(12.853,8.143)--(12.865,8.150)%
  --(12.877,8.157)--(12.889,8.164)--(12.901,8.171)--(12.913,8.178)--(12.926,8.185)--(12.938,8.192)%
  --(12.950,8.199)--(12.962,8.207)--(12.974,8.214)--(12.986,8.221)--(12.998,8.228)--(13.010,8.235)%
  --(13.023,8.242)--(13.035,8.249)--(13.047,8.256)--(13.059,8.263)--(13.071,8.270)--(13.083,8.278)%
  --(13.095,8.285)--(13.107,8.292)--(13.120,8.299)--(13.132,8.306)--(13.144,8.313)--(13.156,8.320)%
  --(13.168,8.327)--(13.180,8.334)--(13.192,8.342)--(13.204,8.349)--(13.217,8.356)--(13.229,8.363)%
  --(13.241,8.370)--(13.253,8.377)--(13.265,8.384)--(13.277,8.391)--(13.289,8.398)--(13.301,8.405)%
  --(13.314,8.413)--(13.326,8.420)--(13.338,8.427)--(13.350,8.434)--(13.362,8.441)--(13.374,8.448)%
  --(13.386,8.455)--(13.398,8.462)--(13.411,8.469)--(13.423,8.476)--(13.435,8.483)--(13.447,8.491);
\gpcolor{rgb color={0.000,0.620,0.451}}
\gpsetlinewidth{1.00}
\draw[gp path] (1.320,1.680)--(1.332,1.680)--(1.344,1.680)--(1.356,1.680)--(1.369,1.680)%
  --(1.381,1.680)--(1.393,1.680)--(1.405,1.680)--(1.417,1.680)--(1.429,1.680)--(1.441,1.680)%
  --(1.453,1.680)--(1.466,1.680)--(1.478,1.680)--(1.490,1.680)--(1.502,1.680)--(1.514,1.680)%
  --(1.526,1.680)--(1.538,1.680)--(1.550,1.680)--(1.563,1.680)--(1.575,1.680)--(1.587,1.680)%
  --(1.599,1.680)--(1.611,1.680)--(1.623,1.680)--(1.635,1.680)--(1.647,1.680)--(1.660,1.680)%
  --(1.672,1.680)--(1.684,1.680)--(1.696,1.680)--(1.708,1.680)--(1.720,1.680)--(1.732,1.680)%
  --(1.744,1.680)--(1.757,1.680)--(1.769,1.680)--(1.781,1.680)--(1.793,1.680)--(1.805,1.680)%
  --(1.817,1.680)--(1.829,1.680)--(1.841,1.680)--(1.854,1.680)--(1.866,1.680)--(1.878,1.680)%
  --(1.890,1.680)--(1.902,1.680)--(1.914,1.680)--(1.926,1.680)--(1.938,1.680)--(1.951,1.680)%
  --(1.963,1.680)--(1.975,1.680)--(1.987,1.680)--(1.999,1.680)--(2.011,1.680)--(2.023,1.680)%
  --(2.035,1.680)--(2.048,1.680)--(2.060,1.680)--(2.072,1.680)--(2.084,1.680)--(2.096,1.680)%
  --(2.108,1.680)--(2.120,1.680)--(2.133,1.680)--(2.145,1.680)--(2.157,1.680)--(2.169,1.680)%
  --(2.181,1.680)--(2.193,1.680)--(2.205,1.680)--(2.217,1.680)--(2.230,1.680)--(2.242,1.680)%
  --(2.254,1.680)--(2.266,1.680)--(2.278,1.680)--(2.290,1.680)--(2.302,1.680)--(2.314,1.680)%
  --(2.327,1.680)--(2.339,1.680)--(2.351,1.680)--(2.363,1.680)--(2.375,1.680)--(2.387,1.680)%
  --(2.399,1.680)--(2.411,1.680)--(2.424,1.680)--(2.436,1.680)--(2.448,1.680)--(2.460,1.680)%
  --(2.472,1.680)--(2.484,1.680)--(2.496,1.680)--(2.508,1.680)--(2.521,1.680)--(2.533,1.680)%
  --(2.545,1.680)--(2.557,1.680)--(2.569,1.680)--(2.581,1.680)--(2.593,1.680)--(2.605,1.680)%
  --(2.618,1.680)--(2.630,1.680)--(2.642,1.680)--(2.654,1.680)--(2.666,1.680)--(2.678,1.680)%
  --(2.690,1.680)--(2.702,1.680)--(2.715,1.680)--(2.727,1.680)--(2.739,1.680)--(2.751,1.680)%
  --(2.763,1.680)--(2.775,1.680)--(2.787,1.680)--(2.799,1.680)--(2.812,1.680)--(2.824,1.680)%
  --(2.836,1.680)--(2.848,1.680)--(2.860,1.680)--(2.872,1.680)--(2.884,1.680)--(2.897,1.680)%
  --(2.909,1.680)--(2.921,1.680)--(2.933,1.680)--(2.945,1.680)--(2.957,1.680)--(2.969,1.680)%
  --(2.981,1.680)--(2.994,1.680)--(3.006,1.680)--(3.018,1.680)--(3.030,1.680)--(3.042,1.680)%
  --(3.054,1.680)--(3.066,1.680)--(3.078,1.680)--(3.091,1.680)--(3.103,1.680)--(3.115,1.680)%
  --(3.127,1.680)--(3.139,1.680)--(3.151,1.680)--(3.163,1.680)--(3.175,1.680)--(3.188,1.680)%
  --(3.200,1.680)--(3.212,1.680)--(3.224,1.680)--(3.236,1.680)--(3.248,1.680)--(3.260,1.680)%
  --(3.272,1.680)--(3.285,1.680)--(3.297,1.680)--(3.309,1.680)--(3.321,1.680)--(3.333,1.680)%
  --(3.345,1.680)--(3.357,1.680)--(3.369,1.680)--(3.382,1.680)--(3.394,1.680)--(3.406,1.680)%
  --(3.418,1.680)--(3.430,1.680)--(3.442,1.680)--(3.454,1.680)--(3.466,1.680)--(3.479,1.680)%
  --(3.491,1.680)--(3.503,1.680)--(3.515,1.680)--(3.527,1.680)--(3.539,1.680)--(3.551,1.680)%
  --(3.563,1.680)--(3.576,1.680)--(3.588,1.680)--(3.600,1.680)--(3.612,1.680)--(3.624,1.680)%
  --(3.636,1.680)--(3.648,1.680)--(3.661,1.680)--(3.673,1.680)--(3.685,1.680)--(3.697,1.680)%
  --(3.709,1.680)--(3.721,1.680)--(3.733,1.680)--(3.745,1.680)--(3.758,1.680)--(3.770,1.680)%
  --(3.782,1.680)--(3.794,1.680)--(3.806,1.680)--(3.818,1.680)--(3.830,1.680)--(3.842,1.680)%
  --(3.855,1.680)--(3.867,1.680)--(3.879,1.680)--(3.891,1.680)--(3.903,1.680)--(3.915,1.680)%
  --(3.927,1.680)--(3.939,1.680)--(3.952,1.680)--(3.964,1.680)--(3.976,1.680)--(3.988,1.680)%
  --(4.000,1.680)--(4.012,1.680)--(4.024,1.680)--(4.036,1.680)--(4.049,1.680)--(4.061,1.680)%
  --(4.073,1.680)--(4.085,1.680)--(4.097,1.680)--(4.109,1.680)--(4.121,1.680)--(4.133,1.680)%
  --(4.146,1.680)--(4.158,1.680)--(4.170,1.680)--(4.182,1.680)--(4.194,1.680)--(4.206,1.680)%
  --(4.218,1.680)--(4.230,1.680)--(4.243,1.680)--(4.255,1.680)--(4.267,1.680)--(4.279,1.680)%
  --(4.291,1.680)--(4.303,1.680)--(4.315,1.680)--(4.327,1.680)--(4.340,1.680)--(4.352,1.680)%
  --(4.364,1.680)--(4.376,1.680)--(4.388,1.680)--(4.400,1.680)--(4.412,1.680)--(4.425,1.680)%
  --(4.437,1.680)--(4.449,1.680)--(4.461,1.680)--(4.473,1.680)--(4.485,1.680)--(4.497,1.680)%
  --(4.509,1.680)--(4.522,1.680)--(4.534,1.680)--(4.546,1.680)--(4.558,1.680)--(4.570,1.680)%
  --(4.582,1.680)--(4.594,1.680)--(4.606,1.680)--(4.619,1.680)--(4.631,1.680)--(4.643,1.680)%
  --(4.655,1.680)--(4.667,1.680)--(4.679,1.680)--(4.691,1.680)--(4.703,1.680)--(4.716,1.680)%
  --(4.728,1.680)--(4.740,1.680)--(4.752,1.680)--(4.764,1.680)--(4.776,1.680)--(4.788,1.680)%
  --(4.800,1.680)--(4.813,1.680)--(4.825,1.680)--(4.837,1.680)--(4.849,1.680)--(4.861,1.680)%
  --(4.873,1.680)--(4.885,1.680)--(4.897,1.680)--(4.910,1.680)--(4.922,1.680)--(4.934,1.680)%
  --(4.946,1.680)--(4.958,1.680)--(4.970,1.680)--(4.982,1.680)--(4.994,1.680)--(5.007,1.680)%
  --(5.019,1.680)--(5.031,1.680)--(5.043,1.680)--(5.055,1.680)--(5.067,1.680)--(5.079,1.680)%
  --(5.091,1.680)--(5.104,1.680)--(5.116,1.680)--(5.128,1.680)--(5.140,1.680)--(5.152,1.680)%
  --(5.164,1.680)--(5.176,1.680)--(5.189,1.680)--(5.201,1.680)--(5.213,1.680)--(5.225,1.680)%
  --(5.237,1.680)--(5.249,1.680)--(5.261,1.680)--(5.273,1.680)--(5.286,1.680)--(5.298,1.680)%
  --(5.310,1.680)--(5.322,1.680)--(5.334,1.680)--(5.346,1.680)--(5.358,1.680)--(5.370,1.680)%
  --(5.383,1.680)--(5.395,1.680)--(5.407,1.680)--(5.419,1.680)--(5.431,1.680)--(5.443,1.680)%
  --(5.455,1.680)--(5.467,1.680)--(5.480,1.680)--(5.492,1.680)--(5.504,1.680)--(5.516,1.680)%
  --(5.528,1.680)--(5.540,1.680)--(5.552,1.680)--(5.564,1.680)--(5.577,1.680)--(5.589,1.680)%
  --(5.601,1.680)--(5.613,1.680)--(5.625,1.680)--(5.637,1.680)--(5.649,1.680)--(5.661,1.680)%
  --(5.674,1.680)--(5.686,1.680)--(5.698,1.680)--(5.710,1.680)--(5.722,1.680)--(5.734,1.680)%
  --(5.746,1.680)--(5.758,1.680)--(5.771,1.680)--(5.783,1.680)--(5.795,1.680)--(5.807,1.680)%
  --(5.819,1.680)--(5.831,1.680)--(5.843,1.680)--(5.855,1.680)--(5.868,1.680)--(5.880,1.680)%
  --(5.892,1.680)--(5.904,1.680)--(5.916,1.680)--(5.928,1.680)--(5.940,1.680)--(5.953,1.680)%
  --(5.965,1.680)--(5.977,1.680)--(5.989,1.680)--(6.001,1.680)--(6.013,1.680)--(6.025,1.680)%
  --(6.037,1.680)--(6.050,1.680)--(6.062,1.680)--(6.074,1.680)--(6.086,1.680)--(6.098,1.680)%
  --(6.110,1.680)--(6.122,1.680)--(6.134,1.680)--(6.147,1.680)--(6.159,1.680)--(6.171,1.680)%
  --(6.183,1.680)--(6.195,1.680)--(6.207,1.680)--(6.219,1.680)--(6.231,1.680)--(6.244,1.680)%
  --(6.256,1.680)--(6.268,1.680)--(6.280,1.680)--(6.292,1.680)--(6.304,1.680)--(6.316,1.680)%
  --(6.328,1.680)--(6.341,1.680)--(6.353,1.680)--(6.365,1.680)--(6.377,1.680)--(6.389,1.680)%
  --(6.401,1.680)--(6.413,1.680)--(6.425,1.680)--(6.438,1.680)--(6.450,1.680)--(6.462,1.680)%
  --(6.474,1.680)--(6.486,1.680)--(6.498,1.680)--(6.510,1.680)--(6.522,1.680)--(6.535,1.680)%
  --(6.547,1.680)--(6.559,1.680)--(6.571,1.680)--(6.583,1.680)--(6.595,1.680)--(6.607,1.680)%
  --(6.619,1.680)--(6.632,1.680)--(6.644,1.680)--(6.656,1.680)--(6.668,1.680)--(6.680,1.680)%
  --(6.692,1.680)--(6.704,1.680)--(6.717,1.680)--(6.729,1.680)--(6.741,1.680)--(6.753,1.680)%
  --(6.765,1.680)--(6.777,1.680)--(6.789,1.680)--(6.801,1.680)--(6.814,1.680)--(6.826,1.680)%
  --(6.838,1.680)--(6.850,1.680)--(6.862,1.680)--(6.874,1.680)--(6.886,1.680)--(6.898,1.680)%
  --(6.911,1.680)--(6.923,1.680)--(6.935,1.680)--(6.947,1.680)--(6.959,1.680)--(6.971,1.680)%
  --(6.983,1.680)--(6.995,1.680)--(7.008,1.680)--(7.020,1.680)--(7.032,1.680)--(7.044,1.680)%
  --(7.056,1.680)--(7.068,1.680)--(7.080,1.680)--(7.092,1.680)--(7.105,1.680)--(7.117,1.680)%
  --(7.129,1.680)--(7.141,1.680)--(7.153,1.680)--(7.165,1.680)--(7.177,1.680)--(7.189,1.680)%
  --(7.202,1.680)--(7.214,1.680)--(7.226,1.680)--(7.238,1.680)--(7.250,1.680)--(7.262,1.680)%
  --(7.274,1.680)--(7.286,1.680)--(7.299,1.680)--(7.311,1.680)--(7.323,1.680)--(7.335,1.680)%
  --(7.347,1.680)--(7.359,1.680)--(7.371,1.680)--(7.384,1.680)--(7.396,1.680)--(7.408,1.680)%
  --(7.420,1.680)--(7.432,1.680)--(7.444,1.680)--(7.456,1.680)--(7.468,1.680)--(7.481,1.680)%
  --(7.493,1.680)--(7.505,1.680)--(7.517,1.680)--(7.529,1.680)--(7.541,1.680)--(7.553,1.680)%
  --(7.565,1.680)--(7.578,1.680)--(7.590,1.680)--(7.602,1.680)--(7.614,1.680)--(7.626,1.680)%
  --(7.638,1.680)--(7.650,1.680)--(7.662,1.680)--(7.675,1.680)--(7.687,1.680)--(7.699,1.680)%
  --(7.711,1.680)--(7.723,1.680)--(7.735,1.680)--(7.747,1.680)--(7.759,1.680)--(7.772,1.680)%
  --(7.784,1.680)--(7.796,1.680)--(7.808,1.680)--(7.820,1.680)--(7.832,1.680)--(7.844,1.680)%
  --(7.856,1.680)--(7.869,1.680)--(7.881,1.680)--(7.893,1.680)--(7.905,1.680)--(7.917,1.680)%
  --(7.929,1.680)--(7.941,1.680)--(7.953,1.680)--(7.966,1.680)--(7.978,1.680)--(7.990,1.680)%
  --(8.002,1.680)--(8.014,1.680)--(8.026,1.680)--(8.038,1.680)--(8.050,1.680)--(8.063,1.680)%
  --(8.075,1.680)--(8.087,1.680)--(8.099,1.680)--(8.111,1.680)--(8.123,1.680)--(8.135,1.680)%
  --(8.148,1.680)--(8.160,1.680)--(8.172,1.680)--(8.184,1.680)--(8.196,1.680)--(8.208,1.680)%
  --(8.220,1.680)--(8.232,1.680)--(8.245,1.680)--(8.257,1.680)--(8.269,1.680)--(8.281,1.680)%
  --(8.293,1.680)--(8.305,1.680)--(8.317,1.680)--(8.329,1.680)--(8.342,1.680)--(8.354,1.680)%
  --(8.366,1.680)--(8.378,1.680)--(8.390,1.680)--(8.402,1.680)--(8.414,1.680)--(8.426,1.680)%
  --(8.439,1.680)--(8.451,1.680)--(8.463,1.680)--(8.475,1.680)--(8.487,1.680)--(8.499,1.680)%
  --(8.511,1.680)--(8.523,1.680)--(8.536,1.680)--(8.548,1.680)--(8.560,1.680)--(8.572,1.680)%
  --(8.584,1.680)--(8.596,1.680)--(8.608,1.680)--(8.620,1.680)--(8.633,1.680)--(8.645,1.680)%
  --(8.657,1.680)--(8.669,1.680)--(8.681,1.680)--(8.693,1.680)--(8.705,1.680)--(8.717,1.680)%
  --(8.730,1.680)--(8.742,1.680)--(8.754,1.680)--(8.766,1.680)--(8.778,1.680)--(8.790,1.680)%
  --(8.802,1.680)--(8.814,1.680)--(8.827,1.680)--(8.839,1.680)--(8.851,1.680)--(8.863,1.680)%
  --(8.875,1.680)--(8.887,1.680)--(8.899,1.680)--(8.912,1.680)--(8.924,1.680)--(8.936,1.680)%
  --(8.948,1.680)--(8.960,1.680)--(8.972,1.680)--(8.984,1.680)--(8.996,1.680)--(9.009,1.680)%
  --(9.021,1.680)--(9.033,1.680)--(9.045,1.680)--(9.057,1.680)--(9.069,1.680)--(9.081,1.680)%
  --(9.093,1.680)--(9.106,1.680)--(9.118,1.680)--(9.130,1.680)--(9.142,1.680)--(9.154,1.680)%
  --(9.166,1.680)--(9.178,1.680)--(9.190,1.680)--(9.203,1.680)--(9.215,1.680)--(9.227,1.680)%
  --(9.239,1.680)--(9.251,1.680)--(9.263,1.680)--(9.275,1.680)--(9.287,1.680)--(9.300,1.680)%
  --(9.312,1.680)--(9.324,1.680)--(9.336,1.680)--(9.348,1.680)--(9.360,1.680)--(9.372,1.680)%
  --(9.384,1.680)--(9.397,1.680)--(9.409,1.680)--(9.421,1.680)--(9.433,1.680)--(9.445,1.680)%
  --(9.457,1.680)--(9.469,1.680)--(9.481,1.680)--(9.494,1.680)--(9.506,1.680)--(9.518,1.680)%
  --(9.530,1.680)--(9.542,1.680)--(9.554,1.680)--(9.566,1.680)--(9.578,1.680)--(9.591,1.680)%
  --(9.603,1.680)--(9.615,1.680)--(9.627,1.680)--(9.639,1.680)--(9.651,1.680)--(9.663,1.680)%
  --(9.676,1.680)--(9.688,1.680)--(9.700,1.680)--(9.712,1.680)--(9.724,1.680)--(9.736,1.680)%
  --(9.748,1.680)--(9.760,1.680)--(9.773,1.680)--(9.785,1.680)--(9.797,1.680)--(9.809,1.680)%
  --(9.821,1.680)--(9.833,1.680)--(9.845,1.680)--(9.857,1.680)--(9.870,1.680)--(9.882,1.680)%
  --(9.894,1.680)--(9.906,1.680)--(9.918,1.680)--(9.930,1.680)--(9.942,1.680)--(9.954,1.680)%
  --(9.967,1.680)--(9.979,1.680)--(9.991,1.680)--(10.003,1.680)--(10.015,1.680)--(10.027,1.680)%
  --(10.039,1.680)--(10.051,1.680)--(10.064,1.680)--(10.076,1.680)--(10.088,1.680)--(10.100,1.680)%
  --(10.112,1.680)--(10.124,1.680)--(10.136,1.680)--(10.148,1.680)--(10.161,1.680)--(10.173,1.680)%
  --(10.185,1.680)--(10.197,1.680)--(10.209,1.680)--(10.221,1.680)--(10.233,1.680)--(10.245,1.680)%
  --(10.258,1.680)--(10.270,1.680)--(10.282,1.680)--(10.294,1.680)--(10.306,1.680)--(10.318,1.680)%
  --(10.330,1.680)--(10.342,1.680)--(10.355,1.680)--(10.367,1.680)--(10.379,1.680)--(10.391,1.680)%
  --(10.403,1.680)--(10.415,1.680)--(10.427,1.680)--(10.440,1.680)--(10.452,1.680)--(10.464,1.680)%
  --(10.476,1.680)--(10.488,1.680)--(10.500,1.680)--(10.512,1.680)--(10.524,1.680)--(10.537,1.680)%
  --(10.549,1.680)--(10.561,1.680)--(10.573,1.680)--(10.585,1.680)--(10.597,1.680)--(10.609,1.680)%
  --(10.621,1.680)--(10.634,1.680)--(10.646,1.680)--(10.658,1.680)--(10.670,1.680)--(10.682,1.680)%
  --(10.694,1.680)--(10.706,1.680)--(10.718,1.680)--(10.731,1.680)--(10.743,1.680)--(10.755,1.680)%
  --(10.767,1.680)--(10.779,1.680)--(10.791,1.680)--(10.803,1.680)--(10.815,1.680)--(10.828,1.680)%
  --(10.840,1.680)--(10.852,1.680)--(10.864,1.680)--(10.876,1.680)--(10.888,1.680)--(10.900,1.680)%
  --(10.912,1.680)--(10.925,1.680)--(10.937,1.680)--(10.949,1.680)--(10.961,1.680)--(10.973,1.680)%
  --(10.985,1.680)--(10.997,1.680)--(11.009,1.680)--(11.022,1.680)--(11.034,1.680)--(11.046,1.680)%
  --(11.058,1.680)--(11.070,1.680)--(11.082,1.680)--(11.094,1.680)--(11.106,1.680)--(11.119,1.680)%
  --(11.131,1.680)--(11.143,1.680)--(11.155,1.680)--(11.167,1.680)--(11.179,1.680)--(11.191,1.680)%
  --(11.204,1.680)--(11.216,1.680)--(11.228,1.680)--(11.240,1.680)--(11.252,1.680)--(11.264,1.680)%
  --(11.276,1.680)--(11.288,1.680)--(11.301,1.680)--(11.313,1.680)--(11.325,1.680)--(11.337,1.680)%
  --(11.349,1.680)--(11.361,1.680)--(11.373,1.680)--(11.385,1.680)--(11.398,1.680)--(11.410,1.680)%
  --(11.422,1.680)--(11.434,1.680)--(11.446,1.680)--(11.458,1.680)--(11.470,1.680)--(11.482,1.680)%
  --(11.495,1.680)--(11.507,1.680)--(11.519,1.680)--(11.531,1.680)--(11.543,1.680)--(11.555,1.680)%
  --(11.567,1.680)--(11.579,1.680)--(11.592,1.680)--(11.604,1.680)--(11.616,1.680)--(11.628,1.680)%
  --(11.640,1.680)--(11.652,1.680)--(11.664,1.680)--(11.676,1.680)--(11.689,1.680)--(11.701,1.680)%
  --(11.713,1.680)--(11.725,1.680)--(11.737,1.680)--(11.749,1.680)--(11.761,1.680)--(11.773,1.680)%
  --(11.786,1.680)--(11.798,1.680)--(11.810,1.680)--(11.822,1.680)--(11.834,1.680)--(11.846,1.680)%
  --(11.858,1.680)--(11.870,1.680)--(11.883,1.680)--(11.895,1.680)--(11.907,1.680)--(11.919,1.680)%
  --(11.931,1.680)--(11.943,1.680)--(11.955,1.680)--(11.968,1.680)--(11.980,1.680)--(11.992,1.680)%
  --(12.004,1.680)--(12.016,1.680)--(12.028,1.680)--(12.040,1.680)--(12.052,1.680)--(12.065,1.680)%
  --(12.077,1.680)--(12.089,1.680)--(12.101,1.680)--(12.113,1.680)--(12.125,1.680)--(12.137,1.680)%
  --(12.149,1.680)--(12.162,1.680)--(12.174,1.680)--(12.186,1.680)--(12.198,1.680)--(12.210,1.680)%
  --(12.222,1.680)--(12.234,1.680)--(12.246,1.680)--(12.259,1.680)--(12.271,1.680)--(12.283,1.680)%
  --(12.295,1.680)--(12.307,1.680)--(12.319,1.680)--(12.331,1.680)--(12.343,1.680)--(12.356,1.680)%
  --(12.368,1.680)--(12.380,1.680)--(12.392,1.680)--(12.404,1.680)--(12.416,1.680)--(12.428,1.680)%
  --(12.440,1.680)--(12.453,1.680)--(12.465,1.680)--(12.477,1.680)--(12.489,1.680)--(12.501,1.680)%
  --(12.513,1.680)--(12.525,1.680)--(12.537,1.680)--(12.550,1.680)--(12.562,1.680)--(12.574,1.680)%
  --(12.586,1.680)--(12.598,1.680)--(12.610,1.680)--(12.622,1.680)--(12.634,1.680)--(12.647,1.680)%
  --(12.659,1.680)--(12.671,1.680)--(12.683,1.680)--(12.695,1.680)--(12.707,1.680)--(12.719,1.680)%
  --(12.732,1.680)--(12.744,1.680)--(12.756,1.680)--(12.768,1.680)--(12.780,1.680)--(12.792,1.680)%
  --(12.804,1.680)--(12.816,1.680)--(12.829,1.680)--(12.841,1.680)--(12.853,1.680)--(12.865,1.680)%
  --(12.877,1.680)--(12.889,1.680)--(12.901,1.680)--(12.913,1.680)--(12.926,1.680)--(12.938,1.680)%
  --(12.950,1.680)--(12.962,1.680)--(12.974,1.680)--(12.986,1.680)--(12.998,1.680)--(13.010,1.680)%
  --(13.023,1.680)--(13.035,1.680)--(13.047,1.680)--(13.059,1.680)--(13.071,1.680)--(13.083,1.680)%
  --(13.095,1.680)--(13.107,1.680)--(13.120,1.680)--(13.132,1.680)--(13.144,1.680)--(13.156,1.680)%
  --(13.168,1.680)--(13.180,1.680)--(13.192,1.680)--(13.204,1.680)--(13.217,1.680)--(13.229,1.680)%
  --(13.241,1.680)--(13.253,1.680)--(13.265,1.680)--(13.277,1.680)--(13.289,1.680)--(13.301,1.680)%
  --(13.314,1.680)--(13.326,1.680)--(13.338,1.680)--(13.350,1.680)--(13.362,1.680)--(13.374,1.680)%
  --(13.386,1.680)--(13.398,1.680)--(13.411,1.680)--(13.423,1.680)--(13.435,1.680)--(13.447,1.680);
\gpcolor{color=gp lt color border}
\draw[gp path] (1.320,8.631)--(1.320,0.985)--(13.447,0.985)--(13.447,8.631)--cycle;
%% coordinates of the plot area
\gpdefrectangularnode{gp plot 1}{\pgfpoint{1.320cm}{0.985cm}}{\pgfpoint{13.447cm}{8.631cm}}
\end{tikzpicture}
%% gnuplot variables

	\caption{Exemplos de formas diferentes para $f(M)$ para valores diferentes de $\rho$. Os momentos de Fermi foram determinados a partir de \eqref{Eq:Mom_Fermi_a_partir_de_rho} com fração de prótons 1/2.}
	\label{Fig:Gap_zero_graph}
\end{figure*}

Portanto, ao se utilizar um método qualquer para se determinar o zero da função, são necessários parâmetros que tornem possível encontrar o zero da função. Como a função só tem um zero\footnote{Desprezando a solução trivial $M = 0$.}, temos uma situação mais simples, pois não há risco de que tenhamos que encontrar mais que um mínimo (se esse fosse o caso, teríamos que encontrar aquele que minimiza alguma energia\cite{Buballa}. Verificar.). Adotando o método da biseção, que é o mais simples, só precisamos de dois pontos que devem estar à esquerda e à direita do zero. Adotamos os valores \np{1.0e-3} e 1500 (em MeV), sendo o primeiro bastante próximo de zero e o segundo aparentemente se posiciona à direita do zero para vários valores de densidade. Não podemos adotar zero para o valor à esquerda pois $f(M=0) = 0$. A Fig.~\ref{Fig:mass_graph} mostra os valores de $M$ obtidos de acordo com o acima exposto, enquanto a Fig.~\ref{Fig:scalar_density_graph} mostra a curva das densidades escalares correspondentes.

\begin{figure*}
	\begin{tikzpicture}[gnuplot]
%% generated with GNUPLOT 5.0p2 (Lua 5.2; terminal rev. 99, script rev. 100)
%% Mon Mar  7 16:11:42 2016
\path (0.000,0.000) rectangle (14.000,9.000);
\gpcolor{color=gp lt color border}
\gpsetlinetype{gp lt border}
\gpsetdashtype{gp dt solid}
\gpsetlinewidth{1.00}
\draw[gp path] (1.504,0.985)--(1.684,0.985);
\draw[gp path] (13.447,0.985)--(13.267,0.985);
\node[gp node right] at (1.320,0.985) {$0$};
\draw[gp path] (1.504,1.750)--(1.684,1.750);
\draw[gp path] (13.447,1.750)--(13.267,1.750);
\node[gp node right] at (1.320,1.750) {$100$};
\draw[gp path] (1.504,2.514)--(1.684,2.514);
\draw[gp path] (13.447,2.514)--(13.267,2.514);
\node[gp node right] at (1.320,2.514) {$200$};
\draw[gp path] (1.504,3.279)--(1.684,3.279);
\draw[gp path] (13.447,3.279)--(13.267,3.279);
\node[gp node right] at (1.320,3.279) {$300$};
\draw[gp path] (1.504,4.043)--(1.684,4.043);
\draw[gp path] (13.447,4.043)--(13.267,4.043);
\node[gp node right] at (1.320,4.043) {$400$};
\draw[gp path] (1.504,4.808)--(1.684,4.808);
\draw[gp path] (13.447,4.808)--(13.267,4.808);
\node[gp node right] at (1.320,4.808) {$500$};
\draw[gp path] (1.504,5.573)--(1.684,5.573);
\draw[gp path] (13.447,5.573)--(13.267,5.573);
\node[gp node right] at (1.320,5.573) {$600$};
\draw[gp path] (1.504,6.337)--(1.684,6.337);
\draw[gp path] (13.447,6.337)--(13.267,6.337);
\node[gp node right] at (1.320,6.337) {$700$};
\draw[gp path] (1.504,7.102)--(1.684,7.102);
\draw[gp path] (13.447,7.102)--(13.267,7.102);
\node[gp node right] at (1.320,7.102) {$800$};
\draw[gp path] (1.504,7.866)--(1.684,7.866);
\draw[gp path] (13.447,7.866)--(13.267,7.866);
\node[gp node right] at (1.320,7.866) {$900$};
\draw[gp path] (1.504,8.631)--(1.684,8.631);
\draw[gp path] (13.447,8.631)--(13.267,8.631);
\node[gp node right] at (1.320,8.631) {$1000$};
\draw[gp path] (1.504,0.985)--(1.504,1.165);
\draw[gp path] (1.504,8.631)--(1.504,8.451);
\node[gp node center] at (1.504,0.677) {$0$};
\draw[gp path] (3.210,0.985)--(3.210,1.165);
\draw[gp path] (3.210,8.631)--(3.210,8.451);
\node[gp node center] at (3.210,0.677) {$0.5$};
\draw[gp path] (4.916,0.985)--(4.916,1.165);
\draw[gp path] (4.916,8.631)--(4.916,8.451);
\node[gp node center] at (4.916,0.677) {$1$};
\draw[gp path] (6.622,0.985)--(6.622,1.165);
\draw[gp path] (6.622,8.631)--(6.622,8.451);
\node[gp node center] at (6.622,0.677) {$1.5$};
\draw[gp path] (8.329,0.985)--(8.329,1.165);
\draw[gp path] (8.329,8.631)--(8.329,8.451);
\node[gp node center] at (8.329,0.677) {$2$};
\draw[gp path] (10.035,0.985)--(10.035,1.165);
\draw[gp path] (10.035,8.631)--(10.035,8.451);
\node[gp node center] at (10.035,0.677) {$2.5$};
\draw[gp path] (11.741,0.985)--(11.741,1.165);
\draw[gp path] (11.741,8.631)--(11.741,8.451);
\node[gp node center] at (11.741,0.677) {$3$};
\draw[gp path] (13.447,0.985)--(13.447,1.165);
\draw[gp path] (13.447,8.631)--(13.447,8.451);
\node[gp node center] at (13.447,0.677) {$3.5$};
\draw[gp path] (1.504,8.631)--(1.504,0.985)--(13.447,0.985)--(13.447,8.631)--cycle;
\node[gp node center,rotate=-270] at (0.246,4.808) {$M$ (MeV)};
\node[gp node center] at (7.475,0.215) {$\rho$ ($\rm{fm}^{-3}$)};
\gpcolor{rgb color={0.580,0.000,0.827}}
\draw[gp path] (1.548,8.107)--(1.559,8.096)--(1.569,8.082)--(1.579,8.071)--(1.589,8.059)%
  --(1.599,8.045)--(1.610,8.034)--(1.620,8.023)--(1.630,8.012)--(1.640,8.001)--(1.650,7.989)%
  --(1.661,7.978)--(1.671,7.967)--(1.681,7.956)--(1.691,7.945)--(1.702,7.933)--(1.712,7.925)%
  --(1.722,7.914)--(1.732,7.903)--(1.742,7.894)--(1.753,7.883)--(1.763,7.875)--(1.773,7.863)%
  --(1.783,7.855)--(1.793,7.844)--(1.804,7.835)--(1.814,7.827)--(1.824,7.819)--(1.834,7.807)%
  --(1.845,7.799)--(1.855,7.791)--(1.865,7.782)--(1.875,7.774)--(1.885,7.765)--(1.896,7.757)%
  --(1.906,7.749)--(1.916,7.740)--(1.926,7.732)--(1.936,7.723)--(1.947,7.715)--(1.957,7.709)%
  --(1.967,7.701)--(1.977,7.693)--(1.987,7.684)--(1.998,7.679)--(2.008,7.670)--(2.018,7.665)%
  --(2.028,7.656)--(2.039,7.648)--(2.049,7.642)--(2.059,7.637)--(2.069,7.628)--(2.079,7.623)%
  --(2.090,7.614)--(2.100,7.609)--(2.110,7.603)--(2.120,7.595)--(2.130,7.589)--(2.141,7.583)%
  --(2.151,7.575)--(2.161,7.569)--(2.171,7.564)--(2.182,7.558)--(2.192,7.553)--(2.202,7.547)%
  --(2.212,7.539)--(2.222,7.533)--(2.233,7.527)--(2.243,7.522)--(2.253,7.516)--(2.263,7.511)%
  --(2.273,7.505)--(2.284,7.499)--(2.294,7.494)--(2.304,7.488)--(2.314,7.483)--(2.325,7.477)%
  --(2.335,7.471)--(2.345,7.469)--(2.355,7.463)--(2.365,7.457)--(2.376,7.452)--(2.386,7.446)%
  --(2.396,7.441)--(2.406,7.435)--(2.416,7.432)--(2.427,7.427)--(2.437,7.421)--(2.447,7.415)%
  --(2.457,7.413)--(2.468,7.407)--(2.478,7.401)--(2.488,7.396)--(2.498,7.390)--(2.508,7.387)%
  --(2.519,7.382)--(2.529,7.376)--(2.539,7.373)--(2.549,7.368)--(2.559,7.362)--(2.570,7.357)%
  --(2.580,7.354)--(2.590,7.348)--(2.600,7.343)--(2.610,7.337)--(2.621,7.334)--(2.631,7.329)%
  --(2.641,7.323)--(2.651,7.320)--(2.662,7.315)--(2.672,7.309)--(2.682,7.303)--(2.692,7.301)%
  --(2.702,7.295)--(2.713,7.289)--(2.723,7.287)--(2.733,7.281)--(2.743,7.275)--(2.753,7.270)%
  --(2.764,7.267)--(2.774,7.261)--(2.784,7.256)--(2.794,7.250)--(2.805,7.245)--(2.815,7.242)%
  --(2.825,7.236)--(2.835,7.231)--(2.845,7.225)--(2.856,7.219)--(2.866,7.214)--(2.876,7.208)%
  --(2.886,7.205)--(2.896,7.200)--(2.907,7.194)--(2.917,7.189)--(2.927,7.183)--(2.937,7.177)%
  --(2.948,7.172)--(2.958,7.166)--(2.968,7.161)--(2.978,7.155)--(2.988,7.149)--(2.999,7.144)%
  --(3.009,7.138)--(3.019,7.133)--(3.029,7.127)--(3.039,7.119)--(3.050,7.113)--(3.060,7.107)%
  --(3.070,7.102)--(3.080,7.096)--(3.090,7.088)--(3.101,7.082)--(3.111,7.077)--(3.121,7.071)%
  --(3.131,7.063)--(3.142,7.057)--(3.152,7.049)--(3.162,7.043)--(3.172,7.037)--(3.182,7.029)%
  --(3.193,7.023)--(3.203,7.015)--(3.213,7.007)--(3.223,7.001)--(3.233,6.993)--(3.244,6.987)%
  --(3.254,6.979)--(3.264,6.970)--(3.274,6.965)--(3.285,6.956)--(3.295,6.948)--(3.305,6.939)%
  --(3.315,6.931)--(3.325,6.923)--(3.336,6.914)--(3.346,6.906)--(3.356,6.897)--(3.366,6.889)%
  --(3.376,6.881)--(3.387,6.872)--(3.397,6.864)--(3.407,6.855)--(3.417,6.847)--(3.428,6.836)%
  --(3.438,6.827)--(3.448,6.819)--(3.458,6.808)--(3.468,6.799)--(3.479,6.791)--(3.489,6.780)%
  --(3.499,6.771)--(3.509,6.760)--(3.519,6.749)--(3.530,6.741)--(3.540,6.729)--(3.550,6.718)%
  --(3.560,6.710)--(3.570,6.699)--(3.581,6.687)--(3.591,6.676)--(3.601,6.665)--(3.611,6.654)%
  --(3.622,6.643)--(3.632,6.631)--(3.642,6.620)--(3.652,6.606)--(3.662,6.595)--(3.673,6.584)%
  --(3.683,6.573)--(3.693,6.559)--(3.703,6.547)--(3.713,6.533)--(3.724,6.522)--(3.734,6.508)%
  --(3.744,6.497)--(3.754,6.483)--(3.765,6.469)--(3.775,6.455)--(3.785,6.444)--(3.795,6.430)%
  --(3.805,6.416)--(3.816,6.402)--(3.826,6.388)--(3.836,6.374)--(3.846,6.357)--(3.856,6.343)%
  --(3.867,6.329)--(3.877,6.315)--(3.887,6.298)--(3.897,6.284)--(3.908,6.270)--(3.918,6.253)%
  --(3.928,6.236)--(3.938,6.222)--(3.948,6.206)--(3.959,6.189)--(3.969,6.175)--(3.979,6.158)%
  --(3.989,6.141)--(3.999,6.124)--(4.010,6.108)--(4.020,6.091)--(4.030,6.074)--(4.040,6.057)%
  --(4.051,6.038)--(4.061,6.021)--(4.071,6.004)--(4.081,5.984)--(4.091,5.968)--(4.102,5.948)%
  --(4.112,5.931)--(4.122,5.912)--(4.132,5.892)--(4.142,5.875)--(4.153,5.856)--(4.163,5.836)%
  --(4.173,5.817)--(4.183,5.798)--(4.193,5.778)--(4.204,5.758)--(4.214,5.737)--(4.224,5.718)%
  --(4.234,5.698)--(4.245,5.677)--(4.255,5.656)--(4.265,5.635)--(4.275,5.616)--(4.285,5.595)%
  --(4.296,5.572)--(4.306,5.551)--(4.316,5.530)--(4.326,5.508)--(4.336,5.487)--(4.347,5.464)%
  --(4.357,5.443)--(4.367,5.421)--(4.377,5.399)--(4.388,5.376)--(4.398,5.354)--(4.408,5.331)%
  --(4.418,5.308)--(4.428,5.285)--(4.439,5.261)--(4.449,5.239)--(4.459,5.215)--(4.469,5.191)%
  --(4.479,5.169)--(4.490,5.145)--(4.500,5.121)--(4.510,5.098)--(4.520,5.074)--(4.531,5.049)%
  --(4.541,5.025)--(4.551,5.001)--(4.561,4.976)--(4.571,4.952)--(4.582,4.927)--(4.592,4.903)%
  --(4.602,4.878)--(4.612,4.853)--(4.622,4.829)--(4.633,4.804)--(4.643,4.778)--(4.653,4.753)%
  --(4.663,4.728)--(4.673,4.703)--(4.684,4.678)--(4.694,4.652)--(4.704,4.627)--(4.714,4.602)%
  --(4.725,4.577)--(4.735,4.552)--(4.745,4.526)--(4.755,4.501)--(4.765,4.476)--(4.776,4.449)%
  --(4.786,4.424)--(4.796,4.399)--(4.806,4.374)--(4.816,4.349)--(4.827,4.323)--(4.837,4.298)%
  --(4.847,4.273)--(4.857,4.248)--(4.868,4.223)--(4.878,4.197)--(4.888,4.172)--(4.898,4.147)%
  --(4.908,4.122)--(4.919,4.097)--(4.929,4.071)--(4.939,4.046)--(4.949,4.022)--(4.959,3.997)%
  --(4.970,3.972)--(4.980,3.948)--(4.990,3.924)--(5.000,3.899)--(5.011,3.875)--(5.021,3.852)%
  --(5.031,3.826)--(5.041,3.803)--(5.051,3.779)--(5.062,3.756)--(5.072,3.733)--(5.082,3.709)%
  --(5.092,3.685)--(5.102,3.663)--(5.113,3.639)--(5.123,3.616)--(5.133,3.594)--(5.143,3.572)%
  --(5.154,3.549)--(5.164,3.527)--(5.174,3.504)--(5.184,3.482)--(5.194,3.461)--(5.205,3.439)%
  --(5.215,3.418)--(5.225,3.397)--(5.235,3.376)--(5.245,3.355)--(5.256,3.334)--(5.266,3.313)%
  --(5.276,3.293)--(5.286,3.272)--(5.296,3.252)--(5.307,3.231)--(5.317,3.212)--(5.327,3.192)%
  --(5.337,3.173)--(5.348,3.154)--(5.358,3.135)--(5.368,3.117)--(5.378,3.097)--(5.388,3.079)%
  --(5.399,3.061)--(5.409,3.042)--(5.419,3.024)--(5.429,3.006)--(5.439,2.989)--(5.450,2.971)%
  --(5.460,2.954)--(5.470,2.936)--(5.480,2.919)--(5.491,2.903)--(5.501,2.886)--(5.511,2.870)%
  --(5.521,2.853)--(5.531,2.837)--(5.542,2.821)--(5.552,2.805)--(5.562,2.789)--(5.572,2.774)%
  --(5.582,2.758)--(5.593,2.744)--(5.603,2.728)--(5.613,2.714)--(5.623,2.699)--(5.634,2.684)%
  --(5.644,2.670)--(5.654,2.656)--(5.664,2.642)--(5.674,2.628)--(5.685,2.614)--(5.695,2.600)%
  --(5.705,2.586)--(5.715,2.573)--(5.725,2.560)--(5.736,2.547)--(5.746,2.534)--(5.756,2.521)%
  --(5.766,2.509)--(5.776,2.496)--(5.787,2.483)--(5.797,2.471)--(5.807,2.459)--(5.817,2.447)%
  --(5.828,2.435)--(5.838,2.423)--(5.848,2.412)--(5.858,2.400)--(5.868,2.389)--(5.879,2.378)%
  --(5.889,2.366)--(5.899,2.355)--(5.909,2.344)--(5.919,2.334)--(5.930,2.323)--(5.940,2.313)%
  --(5.950,2.302)--(5.960,2.292)--(5.971,2.281)--(5.981,2.271)--(5.991,2.261)--(6.001,2.251)%
  --(6.011,2.241)--(6.022,2.231)--(6.032,2.222)--(6.042,2.212)--(6.052,2.203)--(6.062,2.194)%
  --(6.073,2.184)--(6.083,2.175)--(6.093,2.166)--(6.103,2.157)--(6.114,2.148)--(6.124,2.140)%
  --(6.134,2.131)--(6.144,2.122)--(6.154,2.114)--(6.165,2.105)--(6.175,2.097)--(6.185,2.089)%
  --(6.195,2.080)--(6.205,2.072)--(6.216,2.064)--(6.226,2.056)--(6.236,2.049)--(6.246,2.041)%
  --(6.257,2.033)--(6.267,2.026)--(6.277,2.018)--(6.287,2.010)--(6.297,2.003)--(6.308,1.995)%
  --(6.318,1.988)--(6.328,1.981)--(6.338,1.974)--(6.348,1.967)--(6.359,1.960)--(6.369,1.953)%
  --(6.379,1.946)--(6.389,1.939)--(6.399,1.932)--(6.410,1.926)--(6.420,1.919)--(6.430,1.913)%
  --(6.440,1.907)--(6.451,1.900)--(6.461,1.893)--(6.471,1.887)--(6.481,1.881)--(6.491,1.875)%
  --(6.502,1.869)--(6.512,1.862)--(6.522,1.857)--(6.532,1.851)--(6.542,1.845)--(6.553,1.839)%
  --(6.563,1.833)--(6.573,1.827)--(6.583,1.822)--(6.594,1.816)--(6.604,1.811)--(6.614,1.805)%
  --(6.624,1.799)--(6.634,1.794)--(6.645,1.789)--(6.655,1.783)--(6.665,1.778)--(6.675,1.773)%
  --(6.685,1.768)--(6.696,1.762)--(6.706,1.757)--(6.716,1.753)--(6.726,1.748)--(6.737,1.743)%
  --(6.747,1.738)--(6.757,1.733)--(6.767,1.728)--(6.777,1.723)--(6.788,1.718)--(6.798,1.713)%
  --(6.808,1.709)--(6.818,1.704)--(6.828,1.699)--(6.839,1.695)--(6.849,1.691)--(6.859,1.686)%
  --(6.869,1.682)--(6.879,1.678)--(6.890,1.673)--(6.900,1.669)--(6.910,1.664)--(6.920,1.660)%
  --(6.931,1.656)--(6.941,1.652)--(6.951,1.648)--(6.961,1.643)--(6.971,1.639)--(6.982,1.636)%
  --(6.992,1.631)--(7.002,1.627)--(7.012,1.624)--(7.022,1.620)--(7.033,1.616)--(7.043,1.612)%
  --(7.053,1.608)--(7.063,1.604)--(7.074,1.600)--(7.084,1.597)--(7.094,1.593)--(7.104,1.590)%
  --(7.114,1.586)--(7.125,1.582)--(7.135,1.579)--(7.145,1.575)--(7.155,1.572)--(7.165,1.568)%
  --(7.176,1.565)--(7.186,1.561)--(7.196,1.558)--(7.206,1.555)--(7.217,1.551)--(7.227,1.548)%
  --(7.237,1.544)--(7.247,1.541)--(7.257,1.538)--(7.268,1.535)--(7.278,1.532)--(7.288,1.529)%
  --(7.298,1.526)--(7.308,1.522)--(7.319,1.519)--(7.329,1.516)--(7.339,1.513)--(7.349,1.510)%
  --(7.360,1.507)--(7.370,1.504)--(7.380,1.501)--(7.390,1.498)--(7.400,1.495)--(7.411,1.492)%
  --(7.421,1.490)--(7.431,1.487)--(7.441,1.484)--(7.451,1.481)--(7.462,1.478)--(7.472,1.476)%
  --(7.482,1.473)--(7.492,1.470)--(7.502,1.467)--(7.513,1.465)--(7.523,1.462)--(7.533,1.459)%
  --(7.543,1.457)--(7.554,1.454)--(7.564,1.452)--(7.574,1.449)--(7.584,1.447)--(7.594,1.444)%
  --(7.605,1.442)--(7.615,1.439)--(7.625,1.437)--(7.635,1.434)--(7.645,1.432)--(7.656,1.429)%
  --(7.666,1.427)--(7.676,1.425)--(7.686,1.422)--(7.697,1.420)--(7.707,1.418)--(7.717,1.415)%
  --(7.727,1.413)--(7.737,1.411)--(7.748,1.409)--(7.758,1.406)--(7.768,1.404)--(7.778,1.402)%
  --(7.788,1.400)--(7.799,1.397)--(7.809,1.395)--(7.819,1.393)--(7.829,1.391)--(7.840,1.389)%
  --(7.850,1.387)--(7.860,1.385)--(7.870,1.383)--(7.880,1.381)--(7.891,1.379)--(7.901,1.376)%
  --(7.911,1.374)--(7.921,1.373)--(7.931,1.371)--(7.942,1.368)--(7.952,1.367)--(7.962,1.365)%
  --(7.972,1.363)--(7.982,1.361)--(7.993,1.359)--(8.003,1.357)--(8.013,1.355)--(8.023,1.353)%
  --(8.034,1.351)--(8.044,1.350)--(8.054,1.348)--(8.064,1.346)--(8.074,1.344)--(8.085,1.342)%
  --(8.095,1.340)--(8.105,1.339)--(8.115,1.337)--(8.125,1.335)--(8.136,1.333)--(8.146,1.332)%
  --(8.156,1.330)--(8.166,1.329)--(8.177,1.327)--(8.187,1.325)--(8.197,1.323)--(8.207,1.322)%
  --(8.217,1.320)--(8.228,1.318)--(8.238,1.317)--(8.248,1.315)--(8.258,1.313)--(8.268,1.312)%
  --(8.279,1.310)--(8.289,1.309)--(8.299,1.307)--(8.309,1.306)--(8.320,1.304)--(8.330,1.303)%
  --(8.340,1.301)--(8.350,1.299)--(8.360,1.298)--(8.371,1.297)--(8.381,1.295)--(8.391,1.294)%
  --(8.401,1.292)--(8.411,1.291)--(8.422,1.289)--(8.432,1.288)--(8.442,1.287)--(8.452,1.285)%
  --(8.463,1.284)--(8.473,1.282)--(8.483,1.281)--(8.493,1.280)--(8.503,1.278)--(8.514,1.277)%
  --(8.524,1.275)--(8.534,1.274)--(8.544,1.273)--(8.554,1.271)--(8.565,1.270)--(8.575,1.269)%
  --(8.585,1.267)--(8.595,1.266)--(8.605,1.265)--(8.616,1.263)--(8.626,1.262)--(8.636,1.261)%
  --(8.646,1.260)--(8.657,1.259)--(8.667,1.257)--(8.677,1.256)--(8.687,1.255)--(8.697,1.254)%
  --(8.708,1.252)--(8.718,1.251)--(8.728,1.250)--(8.738,1.249)--(8.748,1.248)--(8.759,1.246)%
  --(8.769,1.245)--(8.779,1.244)--(8.789,1.243)--(8.800,1.242)--(8.810,1.241)--(8.820,1.240)%
  --(8.830,1.238)--(8.840,1.237)--(8.851,1.236)--(8.861,1.235)--(8.871,1.234)--(8.881,1.233)%
  --(8.891,1.232)--(8.902,1.231)--(8.912,1.229)--(8.922,1.228)--(8.932,1.227)--(8.943,1.226)%
  --(8.953,1.225)--(8.963,1.224)--(8.973,1.223)--(8.983,1.222)--(8.994,1.221)--(9.004,1.220)%
  --(9.014,1.219)--(9.024,1.218)--(9.034,1.217)--(9.045,1.216)--(9.055,1.215)--(9.065,1.214)%
  --(9.075,1.213)--(9.085,1.212)--(9.096,1.211)--(9.106,1.210)--(9.116,1.210)--(9.126,1.208)%
  --(9.137,1.207)--(9.147,1.206)--(9.157,1.206)--(9.167,1.205)--(9.177,1.204)--(9.188,1.203)%
  --(9.198,1.202)--(9.208,1.201)--(9.218,1.200)--(9.228,1.199)--(9.239,1.198)--(9.249,1.197)%
  --(9.259,1.197)--(9.269,1.196)--(9.280,1.195)--(9.290,1.194)--(9.300,1.193)--(9.310,1.192)%
  --(9.320,1.191)--(9.331,1.191)--(9.341,1.190)--(9.351,1.189)--(9.361,1.188)--(9.371,1.187)%
  --(9.382,1.186)--(9.392,1.185)--(9.402,1.185)--(9.412,1.184)--(9.423,1.183)--(9.433,1.182)%
  --(9.443,1.182)--(9.453,1.180)--(9.463,1.180)--(9.474,1.179)--(9.484,1.178)--(9.494,1.178)%
  --(9.504,1.177)--(9.514,1.176)--(9.525,1.175)--(9.535,1.175)--(9.545,1.174)--(9.555,1.173)%
  --(9.565,1.172)--(9.576,1.171)--(9.586,1.171)--(9.596,1.170)--(9.606,1.169)--(9.617,1.169)%
  --(9.627,1.168)--(9.637,1.167)--(9.647,1.166)--(9.657,1.166)--(9.668,1.165)--(9.678,1.164)%
  --(9.688,1.163)--(9.698,1.163)--(9.708,1.162)--(9.719,1.162)--(9.729,1.161)--(9.739,1.160)%
  --(9.749,1.159)--(9.760,1.159)--(9.770,1.158)--(9.780,1.157)--(9.790,1.157)--(9.800,1.156)%
  --(9.811,1.155)--(9.821,1.155)--(9.831,1.154)--(9.841,1.154)--(9.851,1.153)--(9.862,1.152)%
  --(9.872,1.151)--(9.882,1.151)--(9.892,1.150)--(9.903,1.150)--(9.913,1.149)--(9.923,1.148)%
  --(9.933,1.148)--(9.943,1.147)--(9.954,1.147)--(9.964,1.146)--(9.974,1.145)--(9.984,1.145)%
  --(9.994,1.144)--(10.005,1.143)--(10.015,1.143)--(10.025,1.142)--(10.035,1.142)--(10.046,1.141)%
  --(10.056,1.141)--(10.066,1.140)--(10.076,1.140)--(10.086,1.139)--(10.097,1.138)--(10.107,1.138)%
  --(10.117,1.137)--(10.127,1.136)--(10.137,1.136)--(10.148,1.135)--(10.158,1.135)--(10.168,1.134)%
  --(10.178,1.134)--(10.188,1.133)--(10.199,1.133)--(10.209,1.132)--(10.219,1.131)--(10.229,1.131)%
  --(10.240,1.130)--(10.250,1.130)--(10.260,1.129)--(10.270,1.129)--(10.280,1.128)--(10.291,1.128)%
  --(10.301,1.127)--(10.311,1.127)--(10.321,1.126)--(10.331,1.126)--(10.342,1.125)--(10.352,1.125)%
  --(10.362,1.124)--(10.372,1.124)--(10.383,1.123)--(10.393,1.123)--(10.403,1.122)--(10.413,1.122)%
  --(10.423,1.121)--(10.434,1.121)--(10.444,1.120)--(10.454,1.120)--(10.464,1.120)--(10.474,1.119)%
  --(10.485,1.119)--(10.495,1.118)--(10.505,1.117)--(10.515,1.117)--(10.526,1.117)--(10.536,1.116)%
  --(10.546,1.116)--(10.556,1.115)--(10.566,1.115)--(10.577,1.114)--(10.587,1.114)--(10.597,1.113)%
  --(10.607,1.113)--(10.617,1.113)--(10.628,1.112)--(10.638,1.112)--(10.648,1.111)--(10.658,1.111)%
  --(10.668,1.110)--(10.679,1.110)--(10.689,1.109)--(10.699,1.109)--(10.709,1.109)--(10.720,1.108)%
  --(10.730,1.108)--(10.740,1.107)--(10.750,1.107)--(10.760,1.107)--(10.771,1.106)--(10.781,1.106)%
  --(10.791,1.105)--(10.801,1.105)--(10.811,1.105)--(10.822,1.104)--(10.832,1.103)--(10.842,1.103)%
  --(10.852,1.103)--(10.863,1.102)--(10.873,1.102)--(10.883,1.102)--(10.893,1.101)--(10.903,1.101)%
  --(10.914,1.100)--(10.924,1.100)--(10.934,1.100)--(10.944,1.099)--(10.954,1.099)--(10.965,1.099)%
  --(10.975,1.098)--(10.985,1.098)--(10.995,1.098)--(11.006,1.097)--(11.016,1.097)--(11.026,1.096)%
  --(11.036,1.096)--(11.046,1.095)--(11.057,1.095)--(11.067,1.095)--(11.077,1.094)--(11.087,1.094)%
  --(11.097,1.094)--(11.108,1.093)--(11.118,1.093)--(11.128,1.093)--(11.138,1.092)--(11.149,1.092)%
  --(11.159,1.092)--(11.169,1.091)--(11.179,1.091)--(11.189,1.091)--(11.200,1.090)--(11.210,1.090)%
  --(11.220,1.089)--(11.230,1.089)--(11.240,1.089)--(11.251,1.088)--(11.261,1.088)--(11.271,1.088)%
  --(11.281,1.088)--(11.291,1.087)--(11.302,1.087)--(11.312,1.087)--(11.322,1.086)--(11.332,1.086)%
  --(11.343,1.086)--(11.353,1.085)--(11.363,1.085)--(11.373,1.085)--(11.383,1.084)--(11.394,1.084)%
  --(11.404,1.084)--(11.414,1.084)--(11.424,1.083)--(11.434,1.083)--(11.445,1.082)--(11.455,1.082)%
  --(11.465,1.082)--(11.475,1.081)--(11.486,1.081)--(11.496,1.081)--(11.506,1.081)--(11.516,1.080)%
  --(11.526,1.080)--(11.537,1.080)--(11.547,1.079)--(11.557,1.079)--(11.567,1.079)--(11.577,1.079)%
  --(11.588,1.078)--(11.598,1.078)--(11.608,1.078)--(11.618,1.077)--(11.629,1.077)--(11.639,1.077)%
  --(11.649,1.077)--(11.659,1.076)--(11.669,1.076)--(11.680,1.076)--(11.690,1.075)--(11.700,1.075)%
  --(11.710,1.075)--(11.720,1.074)--(11.731,1.074)--(11.741,1.074)--(11.751,1.074);
\gpcolor{color=gp lt color border}
\draw[gp path] (1.504,8.631)--(1.504,0.985)--(13.447,0.985)--(13.447,8.631)--cycle;
%% coordinates of the plot area
\gpdefrectangularnode{gp plot 1}{\pgfpoint{1.504cm}{0.985cm}}{\pgfpoint{13.447cm}{8.631cm}}
\end{tikzpicture}
%% gnuplot variables

	\caption{Gráfico mostrando a massa em função da densidade bariônica para fração de prótons 1/2. Note que $M$ diminui até zero em $\rho \approx 0.35$. Nesse ponto ocorre a restauração da simetria quiral.}
	\label{Fig:mass_graph}
\end{figure*}

\begin{figure*}
	\begin{tikzpicture}[gnuplot]
%% generated with GNUPLOT 5.0p2 (Lua 5.2; terminal rev. 99, script rev. 100)
%% Tue Apr  5 14:26:13 2016
\path (0.000,0.000) rectangle (14.000,9.000);
\gpcolor{color=gp lt color border}
\gpsetlinetype{gp lt border}
\gpsetdashtype{gp dt solid}
\gpsetlinewidth{1.00}
\draw[gp path] (1.504,0.985)--(1.684,0.985);
\draw[gp path] (13.447,0.985)--(13.267,0.985);
\node[gp node right] at (1.320,0.985) {$-4$};
\draw[gp path] (1.504,2.077)--(1.684,2.077);
\draw[gp path] (13.447,2.077)--(13.267,2.077);
\node[gp node right] at (1.320,2.077) {$-3.5$};
\draw[gp path] (1.504,3.170)--(1.684,3.170);
\draw[gp path] (13.447,3.170)--(13.267,3.170);
\node[gp node right] at (1.320,3.170) {$-3$};
\draw[gp path] (1.504,4.262)--(1.684,4.262);
\draw[gp path] (13.447,4.262)--(13.267,4.262);
\node[gp node right] at (1.320,4.262) {$-2.5$};
\draw[gp path] (1.504,5.354)--(1.684,5.354);
\draw[gp path] (13.447,5.354)--(13.267,5.354);
\node[gp node right] at (1.320,5.354) {$-2$};
\draw[gp path] (1.504,6.446)--(1.684,6.446);
\draw[gp path] (13.447,6.446)--(13.267,6.446);
\node[gp node right] at (1.320,6.446) {$-1.5$};
\draw[gp path] (1.504,7.539)--(1.684,7.539);
\draw[gp path] (13.447,7.539)--(13.267,7.539);
\node[gp node right] at (1.320,7.539) {$-1$};
\draw[gp path] (1.504,8.631)--(1.684,8.631);
\draw[gp path] (13.447,8.631)--(13.267,8.631);
\node[gp node right] at (1.320,8.631) {$-0.5$};
\draw[gp path] (1.504,0.985)--(1.504,1.165);
\draw[gp path] (1.504,8.631)--(1.504,8.451);
\node[gp node center] at (1.504,0.677) {$0$};
\draw[gp path] (3.495,0.985)--(3.495,1.165);
\draw[gp path] (3.495,8.631)--(3.495,8.451);
\node[gp node center] at (3.495,0.677) {$0.05$};
\draw[gp path] (5.485,0.985)--(5.485,1.165);
\draw[gp path] (5.485,8.631)--(5.485,8.451);
\node[gp node center] at (5.485,0.677) {$0.1$};
\draw[gp path] (7.476,0.985)--(7.476,1.165);
\draw[gp path] (7.476,8.631)--(7.476,8.451);
\node[gp node center] at (7.476,0.677) {$0.15$};
\draw[gp path] (9.466,0.985)--(9.466,1.165);
\draw[gp path] (9.466,8.631)--(9.466,8.451);
\node[gp node center] at (9.466,0.677) {$0.2$};
\draw[gp path] (11.456,0.985)--(11.456,1.165);
\draw[gp path] (11.456,8.631)--(11.456,8.451);
\node[gp node center] at (11.456,0.677) {$0.25$};
\draw[gp path] (13.447,0.985)--(13.447,1.165);
\draw[gp path] (13.447,8.631)--(13.447,8.451);
\node[gp node center] at (13.447,0.677) {$0.3$};
\draw[gp path] (1.504,8.631)--(1.504,0.985)--(13.447,0.985)--(13.447,8.631)--cycle;
\node[gp node center,rotate=-270] at (0.246,4.808) {$\rho_s$ ($\rm{fm}^{-3}$)};
\node[gp node center] at (7.475,0.215) {$\rho$ ($\rm{fm}^{-3}$)};
\gpcolor{rgb color={0.580,0.000,0.827}}
\draw[gp path] (1.912,1.174)--(1.922,1.179)--(1.932,1.184)--(1.942,1.189)--(1.952,1.190)%
  --(1.962,1.195)--(1.973,1.200)--(1.983,1.205)--(1.993,1.210)--(2.003,1.215)--(2.013,1.220)%
  --(2.023,1.225)--(2.033,1.230)--(2.043,1.235)--(2.053,1.240)--(2.063,1.245)--(2.073,1.247)%
  --(2.083,1.252)--(2.093,1.257)--(2.103,1.262)--(2.113,1.267)--(2.123,1.272)--(2.134,1.277)%
  --(2.144,1.282)--(2.154,1.287)--(2.164,1.292)--(2.174,1.297)--(2.184,1.302)--(2.194,1.303)%
  --(2.204,1.308)--(2.214,1.313)--(2.224,1.318)--(2.234,1.323)--(2.244,1.328)--(2.254,1.333)%
  --(2.264,1.338)--(2.274,1.343)--(2.284,1.348)--(2.295,1.353)--(2.305,1.355)--(2.315,1.360)%
  --(2.325,1.365)--(2.335,1.370)--(2.345,1.375)--(2.355,1.380)--(2.365,1.385)--(2.375,1.390)%
  --(2.385,1.395)--(2.395,1.400)--(2.405,1.405)--(2.415,1.410)--(2.425,1.411)--(2.435,1.416)%
  --(2.445,1.421)--(2.456,1.426)--(2.466,1.431)--(2.476,1.436)--(2.486,1.441)--(2.496,1.446)%
  --(2.506,1.451)--(2.516,1.456)--(2.526,1.461)--(2.536,1.466)--(2.546,1.471)--(2.556,1.473)%
  --(2.566,1.478)--(2.576,1.483)--(2.586,1.488)--(2.596,1.493)--(2.606,1.498)--(2.617,1.503)%
  --(2.627,1.508)--(2.637,1.513)--(2.647,1.518)--(2.657,1.523)--(2.667,1.528)--(2.677,1.533)%
  --(2.687,1.535)--(2.697,1.540)--(2.707,1.545)--(2.717,1.550)--(2.727,1.555)--(2.737,1.560)%
  --(2.747,1.565)--(2.757,1.570)--(2.767,1.575)--(2.778,1.580)--(2.788,1.585)--(2.798,1.590)%
  --(2.808,1.595)--(2.818,1.596)--(2.828,1.601)--(2.838,1.606)--(2.848,1.611)--(2.858,1.616)%
  --(2.868,1.621)--(2.878,1.626)--(2.888,1.631)--(2.898,1.636)--(2.908,1.642)--(2.918,1.647)%
  --(2.928,1.652)--(2.938,1.657)--(2.949,1.662)--(2.959,1.663)--(2.969,1.668)--(2.979,1.673)%
  --(2.989,1.678)--(2.999,1.683)--(3.009,1.688)--(3.019,1.693)--(3.029,1.698)--(3.039,1.703)%
  --(3.049,1.708)--(3.059,1.713)--(3.069,1.718)--(3.079,1.723)--(3.089,1.729)--(3.099,1.734)%
  --(3.110,1.735)--(3.120,1.740)--(3.130,1.745)--(3.140,1.750)--(3.150,1.755)--(3.160,1.760)%
  --(3.170,1.765)--(3.180,1.770)--(3.190,1.775)--(3.200,1.780)--(3.210,1.785)--(3.220,1.790)%
  --(3.230,1.796)--(3.240,1.801)--(3.250,1.806)--(3.260,1.807)--(3.271,1.812)--(3.281,1.817)%
  --(3.291,1.822)--(3.301,1.827)--(3.311,1.832)--(3.321,1.837)--(3.331,1.842)--(3.341,1.847)%
  --(3.351,1.852)--(3.361,1.858)--(3.371,1.863)--(3.381,1.868)--(3.391,1.873)--(3.401,1.878)%
  --(3.411,1.883)--(3.421,1.888)--(3.432,1.893)--(3.442,1.894)--(3.452,1.899)--(3.462,1.904)%
  --(3.472,1.910)--(3.482,1.915)--(3.492,1.920)--(3.502,1.925)--(3.512,1.930)--(3.522,1.935)%
  --(3.532,1.940)--(3.542,1.945)--(3.552,1.950)--(3.562,1.955)--(3.572,1.960)--(3.582,1.965)%
  --(3.593,1.970)--(3.603,1.975)--(3.613,1.980)--(3.623,1.985)--(3.633,1.987)--(3.643,1.992)%
  --(3.653,1.997)--(3.663,2.002)--(3.673,2.007)--(3.683,2.012)--(3.693,2.017)--(3.703,2.022)%
  --(3.713,2.027)--(3.723,2.033)--(3.733,2.038)--(3.743,2.043)--(3.754,2.048)--(3.764,2.053)%
  --(3.774,2.058)--(3.784,2.063)--(3.794,2.068)--(3.804,2.073)--(3.814,2.078)--(3.824,2.083)%
  --(3.834,2.088)--(3.844,2.093)--(3.854,2.099)--(3.864,2.100)--(3.874,2.105)--(3.884,2.110)%
  --(3.894,2.115)--(3.904,2.120)--(3.915,2.125)--(3.925,2.130)--(3.935,2.136)--(3.945,2.141)%
  --(3.955,2.146)--(3.965,2.151)--(3.975,2.156)--(3.985,2.161)--(3.995,2.166)--(4.005,2.171)%
  --(4.015,2.176)--(4.025,2.181)--(4.035,2.186)--(4.045,2.192)--(4.055,2.197)--(4.065,2.202)%
  --(4.076,2.207)--(4.086,2.212)--(4.096,2.217)--(4.106,2.222)--(4.116,2.227)--(4.126,2.232)%
  --(4.136,2.237)--(4.146,2.243)--(4.156,2.248)--(4.166,2.252)--(4.176,2.255)--(4.186,2.260)%
  --(4.196,2.265)--(4.206,2.270)--(4.216,2.275)--(4.226,2.281)--(4.237,2.286)--(4.247,2.291)%
  --(4.257,2.296)--(4.267,2.301)--(4.277,2.306)--(4.287,2.311)--(4.297,2.316)--(4.307,2.321)%
  --(4.317,2.327)--(4.327,2.332)--(4.337,2.337)--(4.347,2.342)--(4.357,2.347)--(4.367,2.352)%
  --(4.377,2.357)--(4.387,2.362)--(4.397,2.367)--(4.408,2.371)--(4.418,2.376)--(4.428,2.381)%
  --(4.438,2.386)--(4.448,2.391)--(4.458,2.396)--(4.468,2.401)--(4.478,2.406)--(4.488,2.412)%
  --(4.498,2.417)--(4.508,2.422)--(4.518,2.427)--(4.528,2.432)--(4.538,2.437)--(4.548,2.442)%
  --(4.558,2.447)--(4.569,2.453)--(4.579,2.458)--(4.589,2.463)--(4.599,2.468)--(4.609,2.473)%
  --(4.619,2.478)--(4.629,2.483)--(4.639,2.489)--(4.649,2.494)--(4.659,2.499)--(4.669,2.504)%
  --(4.679,2.509)--(4.689,2.514)--(4.699,2.519)--(4.709,2.524)--(4.719,2.530)--(4.730,2.535)%
  --(4.740,2.540)--(4.750,2.545)--(4.760,2.550)--(4.770,2.555)--(4.780,2.560)--(4.790,2.566)%
  --(4.800,2.571)--(4.810,2.576)--(4.820,2.581)--(4.830,2.586)--(4.840,2.591)--(4.850,2.597)%
  --(4.860,2.602)--(4.870,2.607)--(4.880,2.612)--(4.891,2.617)--(4.901,2.622)--(4.911,2.627)%
  --(4.921,2.633)--(4.931,2.638)--(4.941,2.643)--(4.951,2.648)--(4.961,2.653)--(4.971,2.658)%
  --(4.981,2.663)--(4.991,2.669)--(5.001,2.674)--(5.011,2.679)--(5.021,2.684)--(5.031,2.691)%
  --(5.041,2.696)--(5.052,2.702)--(5.062,2.707)--(5.072,2.712)--(5.082,2.717)--(5.092,2.722)%
  --(5.102,2.727)--(5.112,2.733)--(5.122,2.738)--(5.132,2.743)--(5.142,2.748)--(5.152,2.753)%
  --(5.162,2.758)--(5.172,2.764)--(5.182,2.769)--(5.192,2.774)--(5.202,2.779)--(5.213,2.784)%
  --(5.223,2.789)--(5.233,2.795)--(5.243,2.800)--(5.253,2.807)--(5.263,2.812)--(5.273,2.817)%
  --(5.283,2.822)--(5.293,2.828)--(5.303,2.833)--(5.313,2.838)--(5.323,2.843)--(5.333,2.848)%
  --(5.343,2.853)--(5.353,2.859)--(5.363,2.864)--(5.374,2.869)--(5.384,2.874)--(5.394,2.879)%
  --(5.404,2.887)--(5.414,2.892)--(5.424,2.897)--(5.434,2.902)--(5.444,2.907)--(5.454,2.912)%
  --(5.464,2.918)--(5.474,2.923)--(5.484,2.928)--(5.494,2.933)--(5.504,2.938)--(5.514,2.944)%
  --(5.524,2.951)--(5.535,2.956)--(5.545,2.961)--(5.555,2.966)--(5.565,2.972)--(5.575,2.977)%
  --(5.585,2.982)--(5.595,2.987)--(5.605,2.992)--(5.615,2.998)--(5.625,3.005)--(5.635,3.010)%
  --(5.645,3.015)--(5.655,3.020)--(5.665,3.026)--(5.675,3.031)--(5.685,3.036)--(5.696,3.041)%
  --(5.706,3.046)--(5.716,3.054)--(5.726,3.059)--(5.736,3.064)--(5.746,3.069)--(5.756,3.074)%
  --(5.766,3.080)--(5.776,3.085)--(5.786,3.090)--(5.796,3.097)--(5.806,3.102)--(5.816,3.108)%
  --(5.826,3.113)--(5.836,3.118)--(5.846,3.123)--(5.857,3.129)--(5.867,3.134)--(5.877,3.141)%
  --(5.887,3.146)--(5.897,3.151)--(5.907,3.157)--(5.917,3.162)--(5.927,3.167)--(5.937,3.172)%
  --(5.947,3.180)--(5.957,3.185)--(5.967,3.190)--(5.977,3.195)--(5.987,3.200)--(5.997,3.206)%
  --(6.007,3.211)--(6.017,3.218)--(6.028,3.223)--(6.038,3.229)--(6.048,3.234)--(6.058,3.239)%
  --(6.068,3.244)--(6.078,3.251)--(6.088,3.257)--(6.098,3.262)--(6.108,3.267)--(6.118,3.272)%
  --(6.128,3.278)--(6.138,3.285)--(6.148,3.290)--(6.158,3.295)--(6.168,3.301)--(6.178,3.306)%
  --(6.189,3.313)--(6.199,3.318)--(6.209,3.324)--(6.219,3.329)--(6.229,3.334)--(6.239,3.341)%
  --(6.249,3.347)--(6.259,3.352)--(6.269,3.357)--(6.279,3.362)--(6.289,3.368)--(6.299,3.375)%
  --(6.309,3.380)--(6.319,3.385)--(6.329,3.391)--(6.339,3.398)--(6.350,3.403)--(6.360,3.408)%
  --(6.370,3.414)--(6.380,3.419)--(6.390,3.426)--(6.400,3.431)--(6.410,3.437)--(6.420,3.442)%
  --(6.430,3.447)--(6.440,3.455)--(6.450,3.460)--(6.460,3.465)--(6.470,3.470)--(6.480,3.478)%
  --(6.490,3.483)--(6.500,3.488)--(6.511,3.493)--(6.521,3.501)--(6.531,3.506)--(6.541,3.511)%
  --(6.551,3.516)--(6.561,3.522)--(6.571,3.529)--(6.581,3.534)--(6.591,3.540)--(6.601,3.545)%
  --(6.611,3.552)--(6.621,3.557)--(6.631,3.563)--(6.641,3.570)--(6.651,3.575)--(6.661,3.581)%
  --(6.672,3.586)--(6.682,3.593)--(6.692,3.599)--(6.702,3.604)--(6.712,3.609)--(6.722,3.616)%
  --(6.732,3.622)--(6.742,3.627)--(6.752,3.632)--(6.762,3.640)--(6.772,3.645)--(6.782,3.650)%
  --(6.792,3.657)--(6.802,3.663)--(6.812,3.668)--(6.822,3.673)--(6.833,3.681)--(6.843,3.686)%
  --(6.853,3.691)--(6.863,3.699)--(6.873,3.704)--(6.883,3.709)--(6.893,3.717)--(6.903,3.722)%
  --(6.913,3.727)--(6.923,3.732)--(6.933,3.740)--(6.943,3.745)--(6.953,3.750)--(6.963,3.758)%
  --(6.973,3.763)--(6.983,3.768)--(6.994,3.776)--(7.004,3.781)--(7.014,3.786)--(7.024,3.794)%
  --(7.034,3.799)--(7.044,3.804)--(7.054,3.812)--(7.064,3.817)--(7.074,3.822)--(7.084,3.830)%
  --(7.094,3.835)--(7.104,3.840)--(7.114,3.848)--(7.124,3.853)--(7.134,3.858)--(7.144,3.866)%
  --(7.155,3.871)--(7.165,3.876)--(7.175,3.884)--(7.185,3.889)--(7.195,3.896)--(7.205,3.902)%
  --(7.215,3.907)--(7.225,3.915)--(7.235,3.920)--(7.245,3.925)--(7.255,3.933)--(7.265,3.938)%
  --(7.275,3.945)--(7.285,3.951)--(7.295,3.956)--(7.305,3.963)--(7.316,3.969)--(7.326,3.974)%
  --(7.336,3.981)--(7.346,3.987)--(7.356,3.994)--(7.366,3.999)--(7.376,4.005)--(7.386,4.012)%
  --(7.396,4.018)--(7.406,4.025)--(7.416,4.030)--(7.426,4.038)--(7.436,4.043)--(7.446,4.048)%
  --(7.456,4.056)--(7.466,4.061)--(7.476,4.069)--(7.487,4.074)--(7.497,4.079)--(7.507,4.087)%
  --(7.517,4.092)--(7.527,4.100)--(7.537,4.105)--(7.547,4.112)--(7.557,4.118)--(7.567,4.125)%
  --(7.577,4.130)--(7.587,4.136)--(7.597,4.143)--(7.607,4.149)--(7.617,4.156)--(7.627,4.161)%
  --(7.637,4.169)--(7.648,4.174)--(7.658,4.182)--(7.668,4.187)--(7.678,4.195)--(7.688,4.200)%
  --(7.698,4.207)--(7.708,4.213)--(7.718,4.220)--(7.728,4.226)--(7.738,4.231)--(7.748,4.238)%
  --(7.758,4.244)--(7.768,4.251)--(7.778,4.257)--(7.788,4.264)--(7.798,4.269)--(7.809,4.277)%
  --(7.819,4.282)--(7.829,4.290)--(7.839,4.295)--(7.849,4.303)--(7.859,4.310)--(7.869,4.316)%
  --(7.879,4.323)--(7.889,4.328)--(7.899,4.336)--(7.909,4.341)--(7.919,4.349)--(7.929,4.354)%
  --(7.939,4.362)--(7.949,4.367)--(7.959,4.375)--(7.970,4.380)--(7.980,4.387)--(7.990,4.393)%
  --(8.000,4.400)--(8.010,4.408)--(8.020,4.413)--(8.030,4.421)--(8.040,4.426)--(8.050,4.434)%
  --(8.060,4.439)--(8.070,4.447)--(8.080,4.454)--(8.090,4.459)--(8.100,4.467)--(8.110,4.472)%
  --(8.120,4.480)--(8.131,4.487)--(8.141,4.493)--(8.151,4.500)--(8.161,4.506)--(8.171,4.513)%
  --(8.181,4.521)--(8.191,4.526)--(8.201,4.534)--(8.211,4.539)--(8.221,4.547)--(8.231,4.554)%
  --(8.241,4.560)--(8.251,4.567)--(8.261,4.573)--(8.271,4.580)--(8.281,4.588)--(8.292,4.593)%
  --(8.302,4.601)--(8.312,4.608)--(8.322,4.614)--(8.332,4.621)--(8.342,4.629)--(8.352,4.634)%
  --(8.362,4.642)--(8.372,4.649)--(8.382,4.655)--(8.392,4.662)--(8.402,4.670)--(8.412,4.675)%
  --(8.422,4.683)--(8.432,4.691)--(8.442,4.696)--(8.453,4.704)--(8.463,4.711)--(8.473,4.717)%
  --(8.483,4.724)--(8.493,4.732)--(8.503,4.739)--(8.513,4.745)--(8.523,4.752)--(8.533,4.760)%
  --(8.543,4.765)--(8.553,4.773)--(8.563,4.781)--(8.573,4.788)--(8.583,4.794)--(8.593,4.801)%
  --(8.603,4.809)--(8.614,4.816)--(8.624,4.822)--(8.634,4.829)--(8.644,4.837)--(8.654,4.845)%
  --(8.664,4.850)--(8.674,4.858)--(8.684,4.865)--(8.694,4.873)--(8.704,4.878)--(8.714,4.886)%
  --(8.724,4.894)--(8.734,4.901)--(8.744,4.909)--(8.754,4.914)--(8.764,4.922)--(8.775,4.930)%
  --(8.785,4.937)--(8.795,4.945)--(8.805,4.950)--(8.815,4.958)--(8.825,4.966)--(8.835,4.973)%
  --(8.845,4.981)--(8.855,4.989)--(8.865,4.994)--(8.875,5.002)--(8.885,5.009)--(8.895,5.017)%
  --(8.905,5.025)--(8.915,5.032)--(8.925,5.040)--(8.935,5.045)--(8.946,5.053)--(8.956,5.061)%
  --(8.966,5.068)--(8.976,5.076)--(8.986,5.084)--(8.996,5.091)--(9.006,5.099)--(9.016,5.107)%
  --(9.026,5.112)--(9.036,5.120)--(9.046,5.128)--(9.056,5.135)--(9.066,5.143)--(9.076,5.151)%
  --(9.086,5.158)--(9.096,5.166)--(9.107,5.174)--(9.117,5.181)--(9.127,5.189)--(9.137,5.197)%
  --(9.147,5.204)--(9.157,5.212)--(9.167,5.220)--(9.177,5.228)--(9.187,5.235)--(9.197,5.243)%
  --(9.207,5.248)--(9.217,5.256)--(9.227,5.264)--(9.237,5.272)--(9.247,5.279)--(9.257,5.287)%
  --(9.268,5.297)--(9.278,5.305)--(9.288,5.312)--(9.298,5.320)--(9.308,5.328)--(9.318,5.336)%
  --(9.328,5.343)--(9.338,5.351)--(9.348,5.359)--(9.358,5.366)--(9.368,5.374)--(9.378,5.382)%
  --(9.388,5.390)--(9.398,5.397)--(9.408,5.405)--(9.418,5.413)--(9.429,5.421)--(9.439,5.431)%
  --(9.449,5.438)--(9.459,5.446)--(9.469,5.454)--(9.479,5.462)--(9.489,5.469)--(9.499,5.477)%
  --(9.509,5.485)--(9.519,5.495)--(9.529,5.503)--(9.539,5.510)--(9.549,5.518)--(9.559,5.526)%
  --(9.569,5.534)--(9.579,5.544)--(9.590,5.551)--(9.600,5.559)--(9.610,5.567)--(9.620,5.575)%
  --(9.630,5.585)--(9.640,5.593)--(9.650,5.600)--(9.660,5.608)--(9.670,5.616)--(9.680,5.626)%
  --(9.690,5.634)--(9.700,5.641)--(9.710,5.649)--(9.720,5.659)--(9.730,5.667)--(9.740,5.675)%
  --(9.751,5.683)--(9.761,5.693)--(9.771,5.701)--(9.781,5.708)--(9.791,5.718)--(9.801,5.726)%
  --(9.811,5.734)--(9.821,5.744)--(9.831,5.752)--(9.841,5.760)--(9.851,5.770)--(9.861,5.778)%
  --(9.871,5.785)--(9.881,5.796)--(9.891,5.803)--(9.901,5.811)--(9.912,5.821)--(9.922,5.829)%
  --(9.932,5.839)--(9.942,5.847)--(9.952,5.857)--(9.962,5.865)--(9.972,5.873)--(9.982,5.883)%
  --(9.992,5.891)--(10.002,5.901)--(10.012,5.909)--(10.022,5.919)--(10.032,5.927)--(10.042,5.937)%
  --(10.052,5.945)--(10.062,5.955)--(10.073,5.963)--(10.083,5.973)--(10.093,5.981)--(10.103,5.991)%
  --(10.113,5.999)--(10.123,6.009)--(10.133,6.017)--(10.143,6.027)--(10.153,6.035)--(10.163,6.045)%
  --(10.173,6.055)--(10.183,6.063)--(10.193,6.073)--(10.203,6.081)--(10.213,6.091)--(10.223,6.101)%
  --(10.234,6.109)--(10.244,6.119)--(10.254,6.130)--(10.264,6.137)--(10.274,6.148)--(10.284,6.158)%
  --(10.294,6.166)--(10.304,6.176)--(10.314,6.186)--(10.324,6.196)--(10.334,6.204)--(10.344,6.214)%
  --(10.354,6.225)--(10.364,6.235)--(10.374,6.243)--(10.384,6.253)--(10.395,6.263)--(10.405,6.274)%
  --(10.415,6.284)--(10.425,6.292)--(10.435,6.302)--(10.445,6.312)--(10.455,6.322)--(10.465,6.333)%
  --(10.475,6.343)--(10.485,6.353)--(10.495,6.361)--(10.505,6.371)--(10.515,6.382)--(10.525,6.392)%
  --(10.535,6.402)--(10.545,6.412)--(10.555,6.423)--(10.566,6.433)--(10.576,6.443)--(10.586,6.453)%
  --(10.596,6.464)--(10.606,6.474)--(10.616,6.484)--(10.626,6.495)--(10.636,6.505)--(10.646,6.515)%
  --(10.656,6.525)--(10.666,6.538)--(10.676,6.548)--(10.686,6.559)--(10.696,6.569)--(10.706,6.579)%
  --(10.716,6.590)--(10.727,6.602)--(10.737,6.613)--(10.747,6.623)--(10.757,6.633)--(10.767,6.644)%
  --(10.777,6.656)--(10.787,6.667)--(10.797,6.677)--(10.807,6.690)--(10.817,6.700)--(10.827,6.710)%
  --(10.837,6.723)--(10.847,6.734)--(10.857,6.744)--(10.867,6.757)--(10.877,6.767)--(10.888,6.780)%
  --(10.898,6.790)--(10.908,6.800)--(10.918,6.813)--(10.928,6.823)--(10.938,6.836)--(10.948,6.847)%
  --(10.958,6.859)--(10.968,6.872)--(10.978,6.883)--(10.988,6.895)--(10.998,6.906)--(11.008,6.919)%
  --(11.018,6.931)--(11.028,6.942)--(11.038,6.955)--(11.049,6.967)--(11.059,6.978)--(11.069,6.991)%
  --(11.079,7.003)--(11.089,7.016)--(11.099,7.029)--(11.109,7.039)--(11.119,7.052)--(11.129,7.065)%
  --(11.139,7.078)--(11.149,7.091)--(11.159,7.104)--(11.169,7.116)--(11.179,7.129)--(11.189,7.142)%
  --(11.199,7.155)--(11.210,7.168)--(11.220,7.181)--(11.230,7.194)--(11.240,7.209)--(11.250,7.222)%
  --(11.260,7.235)--(11.270,7.247)--(11.280,7.260)--(11.290,7.276)--(11.300,7.289)--(11.310,7.301)%
  --(11.320,7.317)--(11.330,7.330)--(11.340,7.345)--(11.350,7.358)--(11.360,7.373)--(11.371,7.386)%
  --(11.381,7.401)--(11.391,7.415)--(11.401,7.431)--(11.411,7.445)--(11.421,7.459)--(11.431,7.474)%
  --(11.441,7.488)--(11.451,7.504)--(11.461,7.519)--(11.471,7.535)--(11.481,7.549)--(11.491,7.564)%
  --(11.501,7.580)--(11.511,7.596)--(11.521,7.612)--(11.532,7.627)--(11.542,7.644)--(11.552,7.659)%
  --(11.562,7.676)--(11.572,7.691)--(11.582,7.708)--(11.592,7.725)--(11.602,7.741)--(11.612,7.758)%
  --(11.622,7.775)--(11.632,7.793)--(11.642,7.810)--(11.652,7.828)--(11.662,7.844)--(11.672,7.862)%
  --(11.682,7.880)--(11.693,7.898)--(11.703,7.917)--(11.713,7.935)--(11.723,7.953)--(11.733,7.973)%
  --(11.743,7.992)--(11.753,8.011)--(11.763,8.031)--(11.773,8.051)--(11.783,8.070)--(11.793,8.091)%
  --(11.803,8.111)--(11.813,8.132)--(11.823,8.153)--(11.833,8.174)--(11.843,8.195)--(11.854,8.217)%
  --(11.864,8.240)--(11.874,8.262)--(11.884,8.285)--(11.894,8.308)--(11.904,8.331)--(11.914,8.356)%
  --(11.924,8.380)--(11.934,8.404)--(11.944,8.430)--(11.954,8.456)--(11.964,8.483);
\gpcolor{color=gp lt color border}
\draw[gp path] (1.504,8.631)--(1.504,0.985)--(13.447,0.985)--(13.447,8.631)--cycle;
%% coordinates of the plot area
\gpdefrectangularnode{gp plot 1}{\pgfpoint{1.504cm}{0.985cm}}{\pgfpoint{13.447cm}{8.631cm}}
\end{tikzpicture}
%% gnuplot variables

	\caption{Gráfico da densidade escalar em função da densidade bariônica para fração de prótons 1/2. O resultado mostrado nesse gráfico é obtido juntamente com os resultados mostrados na Fig.~\ref{Fig:mass_graph} através da solução da Equação~\ref{Eq:Gap_zero}.}
	\label{Fig:scalar_density_graph}
\end{figure*}

%%%%%%%%%%%%%%%%%%%%%%%%%%%%%%%%%%%%%%%%%%%%%%%%%%%%%%%%%%%%%%%%%%%%%%%%%%%%%
\section{Potenciais químicos, termodinâmico, pressão, e densidade de energia}
%%%%%%%%%%%%%%%%%%%%%%%%%%%%%%%%%%%%%%%%%%%%%%%%%%%%%%%%%%%%%%%%%%%%%%%%%%%%%

As figuras abaixo mostram os resultados obtidos através das massas e densidades escalares obtidas na seção anterior, utilizando as Equações~\eqref{Eq:Potenciais_Quimicos}, \eqref{Eq:potencial_termodinamico}, \eqref{Eq:Pressao} e~\eqref{Eq:Densidade_energia}.
\begin{figure*}
	\begin{tikzpicture}[gnuplot]
%% generated with GNUPLOT 5.0p2 (Lua 5.2; terminal rev. 99, script rev. 100)
%% Fri Mar 18 15:48:26 2016
\path (0.000,0.000) rectangle (14.000,9.000);
\gpcolor{color=gp lt color border}
\gpsetlinetype{gp lt border}
\gpsetdashtype{gp dt solid}
\gpsetlinewidth{1.00}
\draw[gp path] (1.320,1.680)--(1.500,1.680);
\draw[gp path] (13.447,1.680)--(13.267,1.680);
\node[gp node right] at (1.136,1.680) {$0$};
\draw[gp path] (1.320,3.070)--(1.500,3.070);
\draw[gp path] (13.447,3.070)--(13.267,3.070);
\node[gp node right] at (1.136,3.070) {$100$};
\draw[gp path] (1.320,4.460)--(1.500,4.460);
\draw[gp path] (13.447,4.460)--(13.267,4.460);
\node[gp node right] at (1.136,4.460) {$200$};
\draw[gp path] (1.320,5.851)--(1.500,5.851);
\draw[gp path] (13.447,5.851)--(13.267,5.851);
\node[gp node right] at (1.136,5.851) {$300$};
\draw[gp path] (1.320,7.241)--(1.500,7.241);
\draw[gp path] (13.447,7.241)--(13.267,7.241);
\node[gp node right] at (1.136,7.241) {$400$};
\draw[gp path] (1.320,8.631)--(1.500,8.631);
\draw[gp path] (13.447,8.631)--(13.267,8.631);
\node[gp node right] at (1.136,8.631) {$500$};
\draw[gp path] (1.320,0.985)--(1.320,1.165);
\draw[gp path] (1.320,8.631)--(1.320,8.451);
\node[gp node center] at (1.320,0.677) {$0$};
\draw[gp path] (3.745,0.985)--(3.745,1.165);
\draw[gp path] (3.745,8.631)--(3.745,8.451);
\node[gp node center] at (3.745,0.677) {$100$};
\draw[gp path] (6.171,0.985)--(6.171,1.165);
\draw[gp path] (6.171,8.631)--(6.171,8.451);
\node[gp node center] at (6.171,0.677) {$200$};
\draw[gp path] (8.596,0.985)--(8.596,1.165);
\draw[gp path] (8.596,8.631)--(8.596,8.451);
\node[gp node center] at (8.596,0.677) {$300$};
\draw[gp path] (11.022,0.985)--(11.022,1.165);
\draw[gp path] (11.022,8.631)--(11.022,8.451);
\node[gp node center] at (11.022,0.677) {$400$};
\draw[gp path] (13.447,0.985)--(13.447,1.165);
\draw[gp path] (13.447,8.631)--(13.447,8.451);
\node[gp node center] at (13.447,0.677) {$500$};
\draw[gp path] (1.320,8.631)--(1.320,0.985)--(13.447,0.985)--(13.447,8.631)--cycle;
\node[gp node center,rotate=-270] at (0.246,4.808) {$p_F$};
\node[gp node center] at (7.383,0.215) {$\mu_R$};
\node[gp node left] at (2.788,8.297) {$p_F = \sqrt{\mu_R^2 - m^2}\theta(\mu_R^2 - m^2), m = 100$};
\gpcolor{rgb color={0.580,0.000,0.827}}
\gpsetlinewidth{3.00}
\draw[gp path] (1.688,8.297)--(2.604,8.297);
\draw[gp path] (1.320,1.680)--(1.332,1.680)--(1.344,1.680)--(1.356,1.680)--(1.369,1.680)%
  --(1.381,1.680)--(1.393,1.680)--(1.405,1.680)--(1.417,1.680)--(1.429,1.680)--(1.441,1.680)%
  --(1.453,1.680)--(1.466,1.680)--(1.478,1.680)--(1.490,1.680)--(1.502,1.680)--(1.514,1.680)%
  --(1.526,1.680)--(1.538,1.680)--(1.550,1.680)--(1.563,1.680)--(1.575,1.680)--(1.587,1.680)%
  --(1.599,1.680)--(1.611,1.680)--(1.623,1.680)--(1.635,1.680)--(1.647,1.680)--(1.660,1.680)%
  --(1.672,1.680)--(1.684,1.680)--(1.696,1.680)--(1.708,1.680)--(1.720,1.680)--(1.732,1.680)%
  --(1.744,1.680)--(1.757,1.680)--(1.769,1.680)--(1.781,1.680)--(1.793,1.680)--(1.805,1.680)%
  --(1.817,1.680)--(1.829,1.680)--(1.841,1.680)--(1.854,1.680)--(1.866,1.680)--(1.878,1.680)%
  --(1.890,1.680)--(1.902,1.680)--(1.914,1.680)--(1.926,1.680)--(1.938,1.680)--(1.951,1.680)%
  --(1.963,1.680)--(1.975,1.680)--(1.987,1.680)--(1.999,1.680)--(2.011,1.680)--(2.023,1.680)%
  --(2.035,1.680)--(2.048,1.680)--(2.060,1.680)--(2.072,1.680)--(2.084,1.680)--(2.096,1.680)%
  --(2.108,1.680)--(2.120,1.680)--(2.133,1.680)--(2.145,1.680)--(2.157,1.680)--(2.169,1.680)%
  --(2.181,1.680)--(2.193,1.680)--(2.205,1.680)--(2.217,1.680)--(2.230,1.680)--(2.242,1.680)%
  --(2.254,1.680)--(2.266,1.680)--(2.278,1.680)--(2.290,1.680)--(2.302,1.680)--(2.314,1.680)%
  --(2.327,1.680)--(2.339,1.680)--(2.351,1.680)--(2.363,1.680)--(2.375,1.680)--(2.387,1.680)%
  --(2.399,1.680)--(2.411,1.680)--(2.424,1.680)--(2.436,1.680)--(2.448,1.680)--(2.460,1.680)%
  --(2.472,1.680)--(2.484,1.680)--(2.496,1.680)--(2.508,1.680)--(2.521,1.680)--(2.533,1.680)%
  --(2.545,1.680)--(2.557,1.680)--(2.569,1.680)--(2.581,1.680)--(2.593,1.680)--(2.605,1.680)%
  --(2.618,1.680)--(2.630,1.680)--(2.642,1.680)--(2.654,1.680)--(2.666,1.680)--(2.678,1.680)%
  --(2.690,1.680)--(2.702,1.680)--(2.715,1.680)--(2.727,1.680)--(2.739,1.680)--(2.751,1.680)%
  --(2.763,1.680)--(2.775,1.680)--(2.787,1.680)--(2.799,1.680)--(2.812,1.680)--(2.824,1.680)%
  --(2.836,1.680)--(2.848,1.680)--(2.860,1.680)--(2.872,1.680)--(2.884,1.680)--(2.897,1.680)%
  --(2.909,1.680)--(2.921,1.680)--(2.933,1.680)--(2.945,1.680)--(2.957,1.680)--(2.969,1.680)%
  --(2.981,1.680)--(2.994,1.680)--(3.006,1.680)--(3.018,1.680)--(3.030,1.680)--(3.042,1.680)%
  --(3.054,1.680)--(3.066,1.680)--(3.078,1.680)--(3.091,1.680)--(3.103,1.680)--(3.115,1.680)%
  --(3.127,1.680)--(3.139,1.680)--(3.151,1.680)--(3.163,1.680)--(3.175,1.680)--(3.188,1.680)%
  --(3.200,1.680)--(3.212,1.680)--(3.224,1.680)--(3.236,1.680)--(3.248,1.680)--(3.260,1.680)%
  --(3.272,1.680)--(3.285,1.680)--(3.297,1.680)--(3.309,1.680)--(3.321,1.680)--(3.333,1.680)%
  --(3.345,1.680)--(3.357,1.680)--(3.369,1.680)--(3.382,1.680)--(3.394,1.680)--(3.406,1.680)%
  --(3.418,1.680)--(3.430,1.680)--(3.442,1.680)--(3.454,1.680)--(3.466,1.680)--(3.479,1.680)%
  --(3.491,1.680)--(3.503,1.680)--(3.515,1.680)--(3.527,1.680)--(3.539,1.680)--(3.551,1.680)%
  --(3.563,1.680)--(3.576,1.680)--(3.588,1.680)--(3.600,1.680)--(3.612,1.680)--(3.624,1.680)%
  --(3.636,1.680)--(3.648,1.680)--(3.661,1.680)--(3.673,1.680)--(3.685,1.680)--(3.697,1.680)%
  --(3.709,1.680)--(3.721,1.680)--(3.733,1.680)--(3.745,1.680)--(3.758,1.819)--(3.770,1.877)%
  --(3.782,1.922)--(3.794,1.960)--(3.806,1.993)--(3.818,2.023)--(3.830,2.051)--(3.842,2.077)%
  --(3.855,2.102)--(3.867,2.125)--(3.879,2.147)--(3.891,2.169)--(3.903,2.189)--(3.915,2.209)%
  --(3.927,2.229)--(3.939,2.247)--(3.952,2.265)--(3.964,2.283)--(3.976,2.300)--(3.988,2.317)%
  --(4.000,2.334)--(4.012,2.350)--(4.024,2.366)--(4.036,2.381)--(4.049,2.397)--(4.061,2.412)%
  --(4.073,2.426)--(4.085,2.441)--(4.097,2.455)--(4.109,2.470)--(4.121,2.484)--(4.133,2.497)%
  --(4.146,2.511)--(4.158,2.524)--(4.170,2.538)--(4.182,2.551)--(4.194,2.564)--(4.206,2.577)%
  --(4.218,2.590)--(4.230,2.602)--(4.243,2.615)--(4.255,2.627)--(4.267,2.639)--(4.279,2.652)%
  --(4.291,2.664)--(4.303,2.676)--(4.315,2.688)--(4.327,2.699)--(4.340,2.711)--(4.352,2.723)%
  --(4.364,2.734)--(4.376,2.746)--(4.388,2.757)--(4.400,2.768)--(4.412,2.780)--(4.425,2.791)%
  --(4.437,2.802)--(4.449,2.813)--(4.461,2.824)--(4.473,2.835)--(4.485,2.846)--(4.497,2.857)%
  --(4.509,2.867)--(4.522,2.878)--(4.534,2.889)--(4.546,2.899)--(4.558,2.910)--(4.570,2.920)%
  --(4.582,2.931)--(4.594,2.941)--(4.606,2.951)--(4.619,2.961)--(4.631,2.972)--(4.643,2.982)%
  --(4.655,2.992)--(4.667,3.002)--(4.679,3.012)--(4.691,3.022)--(4.703,3.032)--(4.716,3.042)%
  --(4.728,3.052)--(4.740,3.062)--(4.752,3.072)--(4.764,3.082)--(4.776,3.091)--(4.788,3.101)%
  --(4.800,3.111)--(4.813,3.121)--(4.825,3.130)--(4.837,3.140)--(4.849,3.149)--(4.861,3.159)%
  --(4.873,3.168)--(4.885,3.178)--(4.897,3.187)--(4.910,3.197)--(4.922,3.206)--(4.934,3.216)%
  --(4.946,3.225)--(4.958,3.234)--(4.970,3.244)--(4.982,3.253)--(4.994,3.262)--(5.007,3.271)%
  --(5.019,3.281)--(5.031,3.290)--(5.043,3.299)--(5.055,3.308)--(5.067,3.317)--(5.079,3.326)%
  --(5.091,3.336)--(5.104,3.345)--(5.116,3.354)--(5.128,3.363)--(5.140,3.372)--(5.152,3.381)%
  --(5.164,3.390)--(5.176,3.399)--(5.189,3.408)--(5.201,3.416)--(5.213,3.425)--(5.225,3.434)%
  --(5.237,3.443)--(5.249,3.452)--(5.261,3.461)--(5.273,3.470)--(5.286,3.478)--(5.298,3.487)%
  --(5.310,3.496)--(5.322,3.505)--(5.334,3.513)--(5.346,3.522)--(5.358,3.531)--(5.370,3.539)%
  --(5.383,3.548)--(5.395,3.557)--(5.407,3.565)--(5.419,3.574)--(5.431,3.583)--(5.443,3.591)%
  --(5.455,3.600)--(5.467,3.608)--(5.480,3.617)--(5.492,3.626)--(5.504,3.634)--(5.516,3.643)%
  --(5.528,3.651)--(5.540,3.660)--(5.552,3.668)--(5.564,3.677)--(5.577,3.685)--(5.589,3.694)%
  --(5.601,3.702)--(5.613,3.710)--(5.625,3.719)--(5.637,3.727)--(5.649,3.736)--(5.661,3.744)%
  --(5.674,3.752)--(5.686,3.761)--(5.698,3.769)--(5.710,3.777)--(5.722,3.786)--(5.734,3.794)%
  --(5.746,3.802)--(5.758,3.811)--(5.771,3.819)--(5.783,3.827)--(5.795,3.836)--(5.807,3.844)%
  --(5.819,3.852)--(5.831,3.860)--(5.843,3.869)--(5.855,3.877)--(5.868,3.885)--(5.880,3.893)%
  --(5.892,3.901)--(5.904,3.910)--(5.916,3.918)--(5.928,3.926)--(5.940,3.934)--(5.953,3.942)%
  --(5.965,3.950)--(5.977,3.959)--(5.989,3.967)--(6.001,3.975)--(6.013,3.983)--(6.025,3.991)%
  --(6.037,3.999)--(6.050,4.007)--(6.062,4.015)--(6.074,4.024)--(6.086,4.032)--(6.098,4.040)%
  --(6.110,4.048)--(6.122,4.056)--(6.134,4.064)--(6.147,4.072)--(6.159,4.080)--(6.171,4.088)%
  --(6.183,4.096)--(6.195,4.104)--(6.207,4.112)--(6.219,4.120)--(6.231,4.128)--(6.244,4.136)%
  --(6.256,4.144)--(6.268,4.152)--(6.280,4.160)--(6.292,4.168)--(6.304,4.176)--(6.316,4.184)%
  --(6.328,4.192)--(6.341,4.200)--(6.353,4.208)--(6.365,4.216)--(6.377,4.223)--(6.389,4.231)%
  --(6.401,4.239)--(6.413,4.247)--(6.425,4.255)--(6.438,4.263)--(6.450,4.271)--(6.462,4.279)%
  --(6.474,4.287)--(6.486,4.295)--(6.498,4.302)--(6.510,4.310)--(6.522,4.318)--(6.535,4.326)%
  --(6.547,4.334)--(6.559,4.342)--(6.571,4.350)--(6.583,4.357)--(6.595,4.365)--(6.607,4.373)%
  --(6.619,4.381)--(6.632,4.389)--(6.644,4.396)--(6.656,4.404)--(6.668,4.412)--(6.680,4.420)%
  --(6.692,4.428)--(6.704,4.435)--(6.717,4.443)--(6.729,4.451)--(6.741,4.459)--(6.753,4.467)%
  --(6.765,4.474)--(6.777,4.482)--(6.789,4.490)--(6.801,4.498)--(6.814,4.505)--(6.826,4.513)%
  --(6.838,4.521)--(6.850,4.529)--(6.862,4.536)--(6.874,4.544)--(6.886,4.552)--(6.898,4.559)%
  --(6.911,4.567)--(6.923,4.575)--(6.935,4.583)--(6.947,4.590)--(6.959,4.598)--(6.971,4.606)%
  --(6.983,4.613)--(6.995,4.621)--(7.008,4.629)--(7.020,4.636)--(7.032,4.644)--(7.044,4.652)%
  --(7.056,4.660)--(7.068,4.667)--(7.080,4.675)--(7.092,4.682)--(7.105,4.690)--(7.117,4.698)%
  --(7.129,4.705)--(7.141,4.713)--(7.153,4.721)--(7.165,4.728)--(7.177,4.736)--(7.189,4.744)%
  --(7.202,4.751)--(7.214,4.759)--(7.226,4.767)--(7.238,4.774)--(7.250,4.782)--(7.262,4.789)%
  --(7.274,4.797)--(7.286,4.805)--(7.299,4.812)--(7.311,4.820)--(7.323,4.827)--(7.335,4.835)%
  --(7.347,4.843)--(7.359,4.850)--(7.371,4.858)--(7.384,4.865)--(7.396,4.873)--(7.408,4.881)%
  --(7.420,4.888)--(7.432,4.896)--(7.444,4.903)--(7.456,4.911)--(7.468,4.918)--(7.481,4.926)%
  --(7.493,4.934)--(7.505,4.941)--(7.517,4.949)--(7.529,4.956)--(7.541,4.964)--(7.553,4.971)%
  --(7.565,4.979)--(7.578,4.986)--(7.590,4.994)--(7.602,5.001)--(7.614,5.009)--(7.626,5.017)%
  --(7.638,5.024)--(7.650,5.032)--(7.662,5.039)--(7.675,5.047)--(7.687,5.054)--(7.699,5.062)%
  --(7.711,5.069)--(7.723,5.077)--(7.735,5.084)--(7.747,5.092)--(7.759,5.099)--(7.772,5.107)%
  --(7.784,5.114)--(7.796,5.122)--(7.808,5.129)--(7.820,5.137)--(7.832,5.144)--(7.844,5.152)%
  --(7.856,5.159)--(7.869,5.167)--(7.881,5.174)--(7.893,5.182)--(7.905,5.189)--(7.917,5.197)%
  --(7.929,5.204)--(7.941,5.212)--(7.953,5.219)--(7.966,5.226)--(7.978,5.234)--(7.990,5.241)%
  --(8.002,5.249)--(8.014,5.256)--(8.026,5.264)--(8.038,5.271)--(8.050,5.279)--(8.063,5.286)%
  --(8.075,5.294)--(8.087,5.301)--(8.099,5.308)--(8.111,5.316)--(8.123,5.323)--(8.135,5.331)%
  --(8.148,5.338)--(8.160,5.346)--(8.172,5.353)--(8.184,5.361)--(8.196,5.368)--(8.208,5.375)%
  --(8.220,5.383)--(8.232,5.390)--(8.245,5.398)--(8.257,5.405)--(8.269,5.412)--(8.281,5.420)%
  --(8.293,5.427)--(8.305,5.435)--(8.317,5.442)--(8.329,5.450)--(8.342,5.457)--(8.354,5.464)%
  --(8.366,5.472)--(8.378,5.479)--(8.390,5.487)--(8.402,5.494)--(8.414,5.501)--(8.426,5.509)%
  --(8.439,5.516)--(8.451,5.524)--(8.463,5.531)--(8.475,5.538)--(8.487,5.546)--(8.499,5.553)%
  --(8.511,5.560)--(8.523,5.568)--(8.536,5.575)--(8.548,5.583)--(8.560,5.590)--(8.572,5.597)%
  --(8.584,5.605)--(8.596,5.612)--(8.608,5.619)--(8.620,5.627)--(8.633,5.634)--(8.645,5.642)%
  --(8.657,5.649)--(8.669,5.656)--(8.681,5.664)--(8.693,5.671)--(8.705,5.678)--(8.717,5.686)%
  --(8.730,5.693)--(8.742,5.700)--(8.754,5.708)--(8.766,5.715)--(8.778,5.723)--(8.790,5.730)%
  --(8.802,5.737)--(8.814,5.745)--(8.827,5.752)--(8.839,5.759)--(8.851,5.767)--(8.863,5.774)%
  --(8.875,5.781)--(8.887,5.789)--(8.899,5.796)--(8.912,5.803)--(8.924,5.811)--(8.936,5.818)%
  --(8.948,5.825)--(8.960,5.833)--(8.972,5.840)--(8.984,5.847)--(8.996,5.855)--(9.009,5.862)%
  --(9.021,5.869)--(9.033,5.877)--(9.045,5.884)--(9.057,5.891)--(9.069,5.899)--(9.081,5.906)%
  --(9.093,5.913)--(9.106,5.921)--(9.118,5.928)--(9.130,5.935)--(9.142,5.942)--(9.154,5.950)%
  --(9.166,5.957)--(9.178,5.964)--(9.190,5.972)--(9.203,5.979)--(9.215,5.986)--(9.227,5.994)%
  --(9.239,6.001)--(9.251,6.008)--(9.263,6.016)--(9.275,6.023)--(9.287,6.030)--(9.300,6.037)%
  --(9.312,6.045)--(9.324,6.052)--(9.336,6.059)--(9.348,6.067)--(9.360,6.074)--(9.372,6.081)%
  --(9.384,6.088)--(9.397,6.096)--(9.409,6.103)--(9.421,6.110)--(9.433,6.118)--(9.445,6.125)%
  --(9.457,6.132)--(9.469,6.139)--(9.481,6.147)--(9.494,6.154)--(9.506,6.161)--(9.518,6.169)%
  --(9.530,6.176)--(9.542,6.183)--(9.554,6.190)--(9.566,6.198)--(9.578,6.205)--(9.591,6.212)%
  --(9.603,6.219)--(9.615,6.227)--(9.627,6.234)--(9.639,6.241)--(9.651,6.249)--(9.663,6.256)%
  --(9.676,6.263)--(9.688,6.270)--(9.700,6.278)--(9.712,6.285)--(9.724,6.292)--(9.736,6.299)%
  --(9.748,6.307)--(9.760,6.314)--(9.773,6.321)--(9.785,6.328)--(9.797,6.336)--(9.809,6.343)%
  --(9.821,6.350)--(9.833,6.357)--(9.845,6.365)--(9.857,6.372)--(9.870,6.379)--(9.882,6.386)%
  --(9.894,6.394)--(9.906,6.401)--(9.918,6.408)--(9.930,6.415)--(9.942,6.423)--(9.954,6.430)%
  --(9.967,6.437)--(9.979,6.444)--(9.991,6.452)--(10.003,6.459)--(10.015,6.466)--(10.027,6.473)%
  --(10.039,6.481)--(10.051,6.488)--(10.064,6.495)--(10.076,6.502)--(10.088,6.509)--(10.100,6.517)%
  --(10.112,6.524)--(10.124,6.531)--(10.136,6.538)--(10.148,6.546)--(10.161,6.553)--(10.173,6.560)%
  --(10.185,6.567)--(10.197,6.575)--(10.209,6.582)--(10.221,6.589)--(10.233,6.596)--(10.245,6.603)%
  --(10.258,6.611)--(10.270,6.618)--(10.282,6.625)--(10.294,6.632)--(10.306,6.640)--(10.318,6.647)%
  --(10.330,6.654)--(10.342,6.661)--(10.355,6.668)--(10.367,6.676)--(10.379,6.683)--(10.391,6.690)%
  --(10.403,6.697)--(10.415,6.704)--(10.427,6.712)--(10.440,6.719)--(10.452,6.726)--(10.464,6.733)%
  --(10.476,6.741)--(10.488,6.748)--(10.500,6.755)--(10.512,6.762)--(10.524,6.769)--(10.537,6.777)%
  --(10.549,6.784)--(10.561,6.791)--(10.573,6.798)--(10.585,6.805)--(10.597,6.813)--(10.609,6.820)%
  --(10.621,6.827)--(10.634,6.834)--(10.646,6.841)--(10.658,6.849)--(10.670,6.856)--(10.682,6.863)%
  --(10.694,6.870)--(10.706,6.877)--(10.718,6.885)--(10.731,6.892)--(10.743,6.899)--(10.755,6.906)%
  --(10.767,6.913)--(10.779,6.921)--(10.791,6.928)--(10.803,6.935)--(10.815,6.942)--(10.828,6.949)%
  --(10.840,6.956)--(10.852,6.964)--(10.864,6.971)--(10.876,6.978)--(10.888,6.985)--(10.900,6.992)%
  --(10.912,7.000)--(10.925,7.007)--(10.937,7.014)--(10.949,7.021)--(10.961,7.028)--(10.973,7.036)%
  --(10.985,7.043)--(10.997,7.050)--(11.009,7.057)--(11.022,7.064)--(11.034,7.071)--(11.046,7.079)%
  --(11.058,7.086)--(11.070,7.093)--(11.082,7.100)--(11.094,7.107)--(11.106,7.114)--(11.119,7.122)%
  --(11.131,7.129)--(11.143,7.136)--(11.155,7.143)--(11.167,7.150)--(11.179,7.158)--(11.191,7.165)%
  --(11.204,7.172)--(11.216,7.179)--(11.228,7.186)--(11.240,7.193)--(11.252,7.201)--(11.264,7.208)%
  --(11.276,7.215)--(11.288,7.222)--(11.301,7.229)--(11.313,7.236)--(11.325,7.244)--(11.337,7.251)%
  --(11.349,7.258)--(11.361,7.265)--(11.373,7.272)--(11.385,7.279)--(11.398,7.287)--(11.410,7.294)%
  --(11.422,7.301)--(11.434,7.308)--(11.446,7.315)--(11.458,7.322)--(11.470,7.329)--(11.482,7.337)%
  --(11.495,7.344)--(11.507,7.351)--(11.519,7.358)--(11.531,7.365)--(11.543,7.372)--(11.555,7.380)%
  --(11.567,7.387)--(11.579,7.394)--(11.592,7.401)--(11.604,7.408)--(11.616,7.415)--(11.628,7.422)%
  --(11.640,7.430)--(11.652,7.437)--(11.664,7.444)--(11.676,7.451)--(11.689,7.458)--(11.701,7.465)%
  --(11.713,7.473)--(11.725,7.480)--(11.737,7.487)--(11.749,7.494)--(11.761,7.501)--(11.773,7.508)%
  --(11.786,7.515)--(11.798,7.523)--(11.810,7.530)--(11.822,7.537)--(11.834,7.544)--(11.846,7.551)%
  --(11.858,7.558)--(11.870,7.565)--(11.883,7.573)--(11.895,7.580)--(11.907,7.587)--(11.919,7.594)%
  --(11.931,7.601)--(11.943,7.608)--(11.955,7.615)--(11.968,7.623)--(11.980,7.630)--(11.992,7.637)%
  --(12.004,7.644)--(12.016,7.651)--(12.028,7.658)--(12.040,7.665)--(12.052,7.673)--(12.065,7.680)%
  --(12.077,7.687)--(12.089,7.694)--(12.101,7.701)--(12.113,7.708)--(12.125,7.715)--(12.137,7.722)%
  --(12.149,7.730)--(12.162,7.737)--(12.174,7.744)--(12.186,7.751)--(12.198,7.758)--(12.210,7.765)%
  --(12.222,7.772)--(12.234,7.779)--(12.246,7.787)--(12.259,7.794)--(12.271,7.801)--(12.283,7.808)%
  --(12.295,7.815)--(12.307,7.822)--(12.319,7.829)--(12.331,7.837)--(12.343,7.844)--(12.356,7.851)%
  --(12.368,7.858)--(12.380,7.865)--(12.392,7.872)--(12.404,7.879)--(12.416,7.886)--(12.428,7.894)%
  --(12.440,7.901)--(12.453,7.908)--(12.465,7.915)--(12.477,7.922)--(12.489,7.929)--(12.501,7.936)%
  --(12.513,7.943)--(12.525,7.950)--(12.537,7.958)--(12.550,7.965)--(12.562,7.972)--(12.574,7.979)%
  --(12.586,7.986)--(12.598,7.993)--(12.610,8.000)--(12.622,8.007)--(12.634,8.015)--(12.647,8.022)%
  --(12.659,8.029)--(12.671,8.036)--(12.683,8.043)--(12.695,8.050)--(12.707,8.057)--(12.719,8.064)%
  --(12.732,8.071)--(12.744,8.079)--(12.756,8.086)--(12.768,8.093)--(12.780,8.100)--(12.792,8.107)%
  --(12.804,8.114)--(12.816,8.121)--(12.829,8.128)--(12.841,8.135)--(12.853,8.143)--(12.865,8.150)%
  --(12.877,8.157)--(12.889,8.164)--(12.901,8.171)--(12.913,8.178)--(12.926,8.185)--(12.938,8.192)%
  --(12.950,8.199)--(12.962,8.207)--(12.974,8.214)--(12.986,8.221)--(12.998,8.228)--(13.010,8.235)%
  --(13.023,8.242)--(13.035,8.249)--(13.047,8.256)--(13.059,8.263)--(13.071,8.270)--(13.083,8.278)%
  --(13.095,8.285)--(13.107,8.292)--(13.120,8.299)--(13.132,8.306)--(13.144,8.313)--(13.156,8.320)%
  --(13.168,8.327)--(13.180,8.334)--(13.192,8.342)--(13.204,8.349)--(13.217,8.356)--(13.229,8.363)%
  --(13.241,8.370)--(13.253,8.377)--(13.265,8.384)--(13.277,8.391)--(13.289,8.398)--(13.301,8.405)%
  --(13.314,8.413)--(13.326,8.420)--(13.338,8.427)--(13.350,8.434)--(13.362,8.441)--(13.374,8.448)%
  --(13.386,8.455)--(13.398,8.462)--(13.411,8.469)--(13.423,8.476)--(13.435,8.483)--(13.447,8.491);
\gpcolor{rgb color={0.000,0.620,0.451}}
\gpsetlinewidth{1.00}
\draw[gp path] (1.320,1.680)--(1.332,1.680)--(1.344,1.680)--(1.356,1.680)--(1.369,1.680)%
  --(1.381,1.680)--(1.393,1.680)--(1.405,1.680)--(1.417,1.680)--(1.429,1.680)--(1.441,1.680)%
  --(1.453,1.680)--(1.466,1.680)--(1.478,1.680)--(1.490,1.680)--(1.502,1.680)--(1.514,1.680)%
  --(1.526,1.680)--(1.538,1.680)--(1.550,1.680)--(1.563,1.680)--(1.575,1.680)--(1.587,1.680)%
  --(1.599,1.680)--(1.611,1.680)--(1.623,1.680)--(1.635,1.680)--(1.647,1.680)--(1.660,1.680)%
  --(1.672,1.680)--(1.684,1.680)--(1.696,1.680)--(1.708,1.680)--(1.720,1.680)--(1.732,1.680)%
  --(1.744,1.680)--(1.757,1.680)--(1.769,1.680)--(1.781,1.680)--(1.793,1.680)--(1.805,1.680)%
  --(1.817,1.680)--(1.829,1.680)--(1.841,1.680)--(1.854,1.680)--(1.866,1.680)--(1.878,1.680)%
  --(1.890,1.680)--(1.902,1.680)--(1.914,1.680)--(1.926,1.680)--(1.938,1.680)--(1.951,1.680)%
  --(1.963,1.680)--(1.975,1.680)--(1.987,1.680)--(1.999,1.680)--(2.011,1.680)--(2.023,1.680)%
  --(2.035,1.680)--(2.048,1.680)--(2.060,1.680)--(2.072,1.680)--(2.084,1.680)--(2.096,1.680)%
  --(2.108,1.680)--(2.120,1.680)--(2.133,1.680)--(2.145,1.680)--(2.157,1.680)--(2.169,1.680)%
  --(2.181,1.680)--(2.193,1.680)--(2.205,1.680)--(2.217,1.680)--(2.230,1.680)--(2.242,1.680)%
  --(2.254,1.680)--(2.266,1.680)--(2.278,1.680)--(2.290,1.680)--(2.302,1.680)--(2.314,1.680)%
  --(2.327,1.680)--(2.339,1.680)--(2.351,1.680)--(2.363,1.680)--(2.375,1.680)--(2.387,1.680)%
  --(2.399,1.680)--(2.411,1.680)--(2.424,1.680)--(2.436,1.680)--(2.448,1.680)--(2.460,1.680)%
  --(2.472,1.680)--(2.484,1.680)--(2.496,1.680)--(2.508,1.680)--(2.521,1.680)--(2.533,1.680)%
  --(2.545,1.680)--(2.557,1.680)--(2.569,1.680)--(2.581,1.680)--(2.593,1.680)--(2.605,1.680)%
  --(2.618,1.680)--(2.630,1.680)--(2.642,1.680)--(2.654,1.680)--(2.666,1.680)--(2.678,1.680)%
  --(2.690,1.680)--(2.702,1.680)--(2.715,1.680)--(2.727,1.680)--(2.739,1.680)--(2.751,1.680)%
  --(2.763,1.680)--(2.775,1.680)--(2.787,1.680)--(2.799,1.680)--(2.812,1.680)--(2.824,1.680)%
  --(2.836,1.680)--(2.848,1.680)--(2.860,1.680)--(2.872,1.680)--(2.884,1.680)--(2.897,1.680)%
  --(2.909,1.680)--(2.921,1.680)--(2.933,1.680)--(2.945,1.680)--(2.957,1.680)--(2.969,1.680)%
  --(2.981,1.680)--(2.994,1.680)--(3.006,1.680)--(3.018,1.680)--(3.030,1.680)--(3.042,1.680)%
  --(3.054,1.680)--(3.066,1.680)--(3.078,1.680)--(3.091,1.680)--(3.103,1.680)--(3.115,1.680)%
  --(3.127,1.680)--(3.139,1.680)--(3.151,1.680)--(3.163,1.680)--(3.175,1.680)--(3.188,1.680)%
  --(3.200,1.680)--(3.212,1.680)--(3.224,1.680)--(3.236,1.680)--(3.248,1.680)--(3.260,1.680)%
  --(3.272,1.680)--(3.285,1.680)--(3.297,1.680)--(3.309,1.680)--(3.321,1.680)--(3.333,1.680)%
  --(3.345,1.680)--(3.357,1.680)--(3.369,1.680)--(3.382,1.680)--(3.394,1.680)--(3.406,1.680)%
  --(3.418,1.680)--(3.430,1.680)--(3.442,1.680)--(3.454,1.680)--(3.466,1.680)--(3.479,1.680)%
  --(3.491,1.680)--(3.503,1.680)--(3.515,1.680)--(3.527,1.680)--(3.539,1.680)--(3.551,1.680)%
  --(3.563,1.680)--(3.576,1.680)--(3.588,1.680)--(3.600,1.680)--(3.612,1.680)--(3.624,1.680)%
  --(3.636,1.680)--(3.648,1.680)--(3.661,1.680)--(3.673,1.680)--(3.685,1.680)--(3.697,1.680)%
  --(3.709,1.680)--(3.721,1.680)--(3.733,1.680)--(3.745,1.680)--(3.758,1.680)--(3.770,1.680)%
  --(3.782,1.680)--(3.794,1.680)--(3.806,1.680)--(3.818,1.680)--(3.830,1.680)--(3.842,1.680)%
  --(3.855,1.680)--(3.867,1.680)--(3.879,1.680)--(3.891,1.680)--(3.903,1.680)--(3.915,1.680)%
  --(3.927,1.680)--(3.939,1.680)--(3.952,1.680)--(3.964,1.680)--(3.976,1.680)--(3.988,1.680)%
  --(4.000,1.680)--(4.012,1.680)--(4.024,1.680)--(4.036,1.680)--(4.049,1.680)--(4.061,1.680)%
  --(4.073,1.680)--(4.085,1.680)--(4.097,1.680)--(4.109,1.680)--(4.121,1.680)--(4.133,1.680)%
  --(4.146,1.680)--(4.158,1.680)--(4.170,1.680)--(4.182,1.680)--(4.194,1.680)--(4.206,1.680)%
  --(4.218,1.680)--(4.230,1.680)--(4.243,1.680)--(4.255,1.680)--(4.267,1.680)--(4.279,1.680)%
  --(4.291,1.680)--(4.303,1.680)--(4.315,1.680)--(4.327,1.680)--(4.340,1.680)--(4.352,1.680)%
  --(4.364,1.680)--(4.376,1.680)--(4.388,1.680)--(4.400,1.680)--(4.412,1.680)--(4.425,1.680)%
  --(4.437,1.680)--(4.449,1.680)--(4.461,1.680)--(4.473,1.680)--(4.485,1.680)--(4.497,1.680)%
  --(4.509,1.680)--(4.522,1.680)--(4.534,1.680)--(4.546,1.680)--(4.558,1.680)--(4.570,1.680)%
  --(4.582,1.680)--(4.594,1.680)--(4.606,1.680)--(4.619,1.680)--(4.631,1.680)--(4.643,1.680)%
  --(4.655,1.680)--(4.667,1.680)--(4.679,1.680)--(4.691,1.680)--(4.703,1.680)--(4.716,1.680)%
  --(4.728,1.680)--(4.740,1.680)--(4.752,1.680)--(4.764,1.680)--(4.776,1.680)--(4.788,1.680)%
  --(4.800,1.680)--(4.813,1.680)--(4.825,1.680)--(4.837,1.680)--(4.849,1.680)--(4.861,1.680)%
  --(4.873,1.680)--(4.885,1.680)--(4.897,1.680)--(4.910,1.680)--(4.922,1.680)--(4.934,1.680)%
  --(4.946,1.680)--(4.958,1.680)--(4.970,1.680)--(4.982,1.680)--(4.994,1.680)--(5.007,1.680)%
  --(5.019,1.680)--(5.031,1.680)--(5.043,1.680)--(5.055,1.680)--(5.067,1.680)--(5.079,1.680)%
  --(5.091,1.680)--(5.104,1.680)--(5.116,1.680)--(5.128,1.680)--(5.140,1.680)--(5.152,1.680)%
  --(5.164,1.680)--(5.176,1.680)--(5.189,1.680)--(5.201,1.680)--(5.213,1.680)--(5.225,1.680)%
  --(5.237,1.680)--(5.249,1.680)--(5.261,1.680)--(5.273,1.680)--(5.286,1.680)--(5.298,1.680)%
  --(5.310,1.680)--(5.322,1.680)--(5.334,1.680)--(5.346,1.680)--(5.358,1.680)--(5.370,1.680)%
  --(5.383,1.680)--(5.395,1.680)--(5.407,1.680)--(5.419,1.680)--(5.431,1.680)--(5.443,1.680)%
  --(5.455,1.680)--(5.467,1.680)--(5.480,1.680)--(5.492,1.680)--(5.504,1.680)--(5.516,1.680)%
  --(5.528,1.680)--(5.540,1.680)--(5.552,1.680)--(5.564,1.680)--(5.577,1.680)--(5.589,1.680)%
  --(5.601,1.680)--(5.613,1.680)--(5.625,1.680)--(5.637,1.680)--(5.649,1.680)--(5.661,1.680)%
  --(5.674,1.680)--(5.686,1.680)--(5.698,1.680)--(5.710,1.680)--(5.722,1.680)--(5.734,1.680)%
  --(5.746,1.680)--(5.758,1.680)--(5.771,1.680)--(5.783,1.680)--(5.795,1.680)--(5.807,1.680)%
  --(5.819,1.680)--(5.831,1.680)--(5.843,1.680)--(5.855,1.680)--(5.868,1.680)--(5.880,1.680)%
  --(5.892,1.680)--(5.904,1.680)--(5.916,1.680)--(5.928,1.680)--(5.940,1.680)--(5.953,1.680)%
  --(5.965,1.680)--(5.977,1.680)--(5.989,1.680)--(6.001,1.680)--(6.013,1.680)--(6.025,1.680)%
  --(6.037,1.680)--(6.050,1.680)--(6.062,1.680)--(6.074,1.680)--(6.086,1.680)--(6.098,1.680)%
  --(6.110,1.680)--(6.122,1.680)--(6.134,1.680)--(6.147,1.680)--(6.159,1.680)--(6.171,1.680)%
  --(6.183,1.680)--(6.195,1.680)--(6.207,1.680)--(6.219,1.680)--(6.231,1.680)--(6.244,1.680)%
  --(6.256,1.680)--(6.268,1.680)--(6.280,1.680)--(6.292,1.680)--(6.304,1.680)--(6.316,1.680)%
  --(6.328,1.680)--(6.341,1.680)--(6.353,1.680)--(6.365,1.680)--(6.377,1.680)--(6.389,1.680)%
  --(6.401,1.680)--(6.413,1.680)--(6.425,1.680)--(6.438,1.680)--(6.450,1.680)--(6.462,1.680)%
  --(6.474,1.680)--(6.486,1.680)--(6.498,1.680)--(6.510,1.680)--(6.522,1.680)--(6.535,1.680)%
  --(6.547,1.680)--(6.559,1.680)--(6.571,1.680)--(6.583,1.680)--(6.595,1.680)--(6.607,1.680)%
  --(6.619,1.680)--(6.632,1.680)--(6.644,1.680)--(6.656,1.680)--(6.668,1.680)--(6.680,1.680)%
  --(6.692,1.680)--(6.704,1.680)--(6.717,1.680)--(6.729,1.680)--(6.741,1.680)--(6.753,1.680)%
  --(6.765,1.680)--(6.777,1.680)--(6.789,1.680)--(6.801,1.680)--(6.814,1.680)--(6.826,1.680)%
  --(6.838,1.680)--(6.850,1.680)--(6.862,1.680)--(6.874,1.680)--(6.886,1.680)--(6.898,1.680)%
  --(6.911,1.680)--(6.923,1.680)--(6.935,1.680)--(6.947,1.680)--(6.959,1.680)--(6.971,1.680)%
  --(6.983,1.680)--(6.995,1.680)--(7.008,1.680)--(7.020,1.680)--(7.032,1.680)--(7.044,1.680)%
  --(7.056,1.680)--(7.068,1.680)--(7.080,1.680)--(7.092,1.680)--(7.105,1.680)--(7.117,1.680)%
  --(7.129,1.680)--(7.141,1.680)--(7.153,1.680)--(7.165,1.680)--(7.177,1.680)--(7.189,1.680)%
  --(7.202,1.680)--(7.214,1.680)--(7.226,1.680)--(7.238,1.680)--(7.250,1.680)--(7.262,1.680)%
  --(7.274,1.680)--(7.286,1.680)--(7.299,1.680)--(7.311,1.680)--(7.323,1.680)--(7.335,1.680)%
  --(7.347,1.680)--(7.359,1.680)--(7.371,1.680)--(7.384,1.680)--(7.396,1.680)--(7.408,1.680)%
  --(7.420,1.680)--(7.432,1.680)--(7.444,1.680)--(7.456,1.680)--(7.468,1.680)--(7.481,1.680)%
  --(7.493,1.680)--(7.505,1.680)--(7.517,1.680)--(7.529,1.680)--(7.541,1.680)--(7.553,1.680)%
  --(7.565,1.680)--(7.578,1.680)--(7.590,1.680)--(7.602,1.680)--(7.614,1.680)--(7.626,1.680)%
  --(7.638,1.680)--(7.650,1.680)--(7.662,1.680)--(7.675,1.680)--(7.687,1.680)--(7.699,1.680)%
  --(7.711,1.680)--(7.723,1.680)--(7.735,1.680)--(7.747,1.680)--(7.759,1.680)--(7.772,1.680)%
  --(7.784,1.680)--(7.796,1.680)--(7.808,1.680)--(7.820,1.680)--(7.832,1.680)--(7.844,1.680)%
  --(7.856,1.680)--(7.869,1.680)--(7.881,1.680)--(7.893,1.680)--(7.905,1.680)--(7.917,1.680)%
  --(7.929,1.680)--(7.941,1.680)--(7.953,1.680)--(7.966,1.680)--(7.978,1.680)--(7.990,1.680)%
  --(8.002,1.680)--(8.014,1.680)--(8.026,1.680)--(8.038,1.680)--(8.050,1.680)--(8.063,1.680)%
  --(8.075,1.680)--(8.087,1.680)--(8.099,1.680)--(8.111,1.680)--(8.123,1.680)--(8.135,1.680)%
  --(8.148,1.680)--(8.160,1.680)--(8.172,1.680)--(8.184,1.680)--(8.196,1.680)--(8.208,1.680)%
  --(8.220,1.680)--(8.232,1.680)--(8.245,1.680)--(8.257,1.680)--(8.269,1.680)--(8.281,1.680)%
  --(8.293,1.680)--(8.305,1.680)--(8.317,1.680)--(8.329,1.680)--(8.342,1.680)--(8.354,1.680)%
  --(8.366,1.680)--(8.378,1.680)--(8.390,1.680)--(8.402,1.680)--(8.414,1.680)--(8.426,1.680)%
  --(8.439,1.680)--(8.451,1.680)--(8.463,1.680)--(8.475,1.680)--(8.487,1.680)--(8.499,1.680)%
  --(8.511,1.680)--(8.523,1.680)--(8.536,1.680)--(8.548,1.680)--(8.560,1.680)--(8.572,1.680)%
  --(8.584,1.680)--(8.596,1.680)--(8.608,1.680)--(8.620,1.680)--(8.633,1.680)--(8.645,1.680)%
  --(8.657,1.680)--(8.669,1.680)--(8.681,1.680)--(8.693,1.680)--(8.705,1.680)--(8.717,1.680)%
  --(8.730,1.680)--(8.742,1.680)--(8.754,1.680)--(8.766,1.680)--(8.778,1.680)--(8.790,1.680)%
  --(8.802,1.680)--(8.814,1.680)--(8.827,1.680)--(8.839,1.680)--(8.851,1.680)--(8.863,1.680)%
  --(8.875,1.680)--(8.887,1.680)--(8.899,1.680)--(8.912,1.680)--(8.924,1.680)--(8.936,1.680)%
  --(8.948,1.680)--(8.960,1.680)--(8.972,1.680)--(8.984,1.680)--(8.996,1.680)--(9.009,1.680)%
  --(9.021,1.680)--(9.033,1.680)--(9.045,1.680)--(9.057,1.680)--(9.069,1.680)--(9.081,1.680)%
  --(9.093,1.680)--(9.106,1.680)--(9.118,1.680)--(9.130,1.680)--(9.142,1.680)--(9.154,1.680)%
  --(9.166,1.680)--(9.178,1.680)--(9.190,1.680)--(9.203,1.680)--(9.215,1.680)--(9.227,1.680)%
  --(9.239,1.680)--(9.251,1.680)--(9.263,1.680)--(9.275,1.680)--(9.287,1.680)--(9.300,1.680)%
  --(9.312,1.680)--(9.324,1.680)--(9.336,1.680)--(9.348,1.680)--(9.360,1.680)--(9.372,1.680)%
  --(9.384,1.680)--(9.397,1.680)--(9.409,1.680)--(9.421,1.680)--(9.433,1.680)--(9.445,1.680)%
  --(9.457,1.680)--(9.469,1.680)--(9.481,1.680)--(9.494,1.680)--(9.506,1.680)--(9.518,1.680)%
  --(9.530,1.680)--(9.542,1.680)--(9.554,1.680)--(9.566,1.680)--(9.578,1.680)--(9.591,1.680)%
  --(9.603,1.680)--(9.615,1.680)--(9.627,1.680)--(9.639,1.680)--(9.651,1.680)--(9.663,1.680)%
  --(9.676,1.680)--(9.688,1.680)--(9.700,1.680)--(9.712,1.680)--(9.724,1.680)--(9.736,1.680)%
  --(9.748,1.680)--(9.760,1.680)--(9.773,1.680)--(9.785,1.680)--(9.797,1.680)--(9.809,1.680)%
  --(9.821,1.680)--(9.833,1.680)--(9.845,1.680)--(9.857,1.680)--(9.870,1.680)--(9.882,1.680)%
  --(9.894,1.680)--(9.906,1.680)--(9.918,1.680)--(9.930,1.680)--(9.942,1.680)--(9.954,1.680)%
  --(9.967,1.680)--(9.979,1.680)--(9.991,1.680)--(10.003,1.680)--(10.015,1.680)--(10.027,1.680)%
  --(10.039,1.680)--(10.051,1.680)--(10.064,1.680)--(10.076,1.680)--(10.088,1.680)--(10.100,1.680)%
  --(10.112,1.680)--(10.124,1.680)--(10.136,1.680)--(10.148,1.680)--(10.161,1.680)--(10.173,1.680)%
  --(10.185,1.680)--(10.197,1.680)--(10.209,1.680)--(10.221,1.680)--(10.233,1.680)--(10.245,1.680)%
  --(10.258,1.680)--(10.270,1.680)--(10.282,1.680)--(10.294,1.680)--(10.306,1.680)--(10.318,1.680)%
  --(10.330,1.680)--(10.342,1.680)--(10.355,1.680)--(10.367,1.680)--(10.379,1.680)--(10.391,1.680)%
  --(10.403,1.680)--(10.415,1.680)--(10.427,1.680)--(10.440,1.680)--(10.452,1.680)--(10.464,1.680)%
  --(10.476,1.680)--(10.488,1.680)--(10.500,1.680)--(10.512,1.680)--(10.524,1.680)--(10.537,1.680)%
  --(10.549,1.680)--(10.561,1.680)--(10.573,1.680)--(10.585,1.680)--(10.597,1.680)--(10.609,1.680)%
  --(10.621,1.680)--(10.634,1.680)--(10.646,1.680)--(10.658,1.680)--(10.670,1.680)--(10.682,1.680)%
  --(10.694,1.680)--(10.706,1.680)--(10.718,1.680)--(10.731,1.680)--(10.743,1.680)--(10.755,1.680)%
  --(10.767,1.680)--(10.779,1.680)--(10.791,1.680)--(10.803,1.680)--(10.815,1.680)--(10.828,1.680)%
  --(10.840,1.680)--(10.852,1.680)--(10.864,1.680)--(10.876,1.680)--(10.888,1.680)--(10.900,1.680)%
  --(10.912,1.680)--(10.925,1.680)--(10.937,1.680)--(10.949,1.680)--(10.961,1.680)--(10.973,1.680)%
  --(10.985,1.680)--(10.997,1.680)--(11.009,1.680)--(11.022,1.680)--(11.034,1.680)--(11.046,1.680)%
  --(11.058,1.680)--(11.070,1.680)--(11.082,1.680)--(11.094,1.680)--(11.106,1.680)--(11.119,1.680)%
  --(11.131,1.680)--(11.143,1.680)--(11.155,1.680)--(11.167,1.680)--(11.179,1.680)--(11.191,1.680)%
  --(11.204,1.680)--(11.216,1.680)--(11.228,1.680)--(11.240,1.680)--(11.252,1.680)--(11.264,1.680)%
  --(11.276,1.680)--(11.288,1.680)--(11.301,1.680)--(11.313,1.680)--(11.325,1.680)--(11.337,1.680)%
  --(11.349,1.680)--(11.361,1.680)--(11.373,1.680)--(11.385,1.680)--(11.398,1.680)--(11.410,1.680)%
  --(11.422,1.680)--(11.434,1.680)--(11.446,1.680)--(11.458,1.680)--(11.470,1.680)--(11.482,1.680)%
  --(11.495,1.680)--(11.507,1.680)--(11.519,1.680)--(11.531,1.680)--(11.543,1.680)--(11.555,1.680)%
  --(11.567,1.680)--(11.579,1.680)--(11.592,1.680)--(11.604,1.680)--(11.616,1.680)--(11.628,1.680)%
  --(11.640,1.680)--(11.652,1.680)--(11.664,1.680)--(11.676,1.680)--(11.689,1.680)--(11.701,1.680)%
  --(11.713,1.680)--(11.725,1.680)--(11.737,1.680)--(11.749,1.680)--(11.761,1.680)--(11.773,1.680)%
  --(11.786,1.680)--(11.798,1.680)--(11.810,1.680)--(11.822,1.680)--(11.834,1.680)--(11.846,1.680)%
  --(11.858,1.680)--(11.870,1.680)--(11.883,1.680)--(11.895,1.680)--(11.907,1.680)--(11.919,1.680)%
  --(11.931,1.680)--(11.943,1.680)--(11.955,1.680)--(11.968,1.680)--(11.980,1.680)--(11.992,1.680)%
  --(12.004,1.680)--(12.016,1.680)--(12.028,1.680)--(12.040,1.680)--(12.052,1.680)--(12.065,1.680)%
  --(12.077,1.680)--(12.089,1.680)--(12.101,1.680)--(12.113,1.680)--(12.125,1.680)--(12.137,1.680)%
  --(12.149,1.680)--(12.162,1.680)--(12.174,1.680)--(12.186,1.680)--(12.198,1.680)--(12.210,1.680)%
  --(12.222,1.680)--(12.234,1.680)--(12.246,1.680)--(12.259,1.680)--(12.271,1.680)--(12.283,1.680)%
  --(12.295,1.680)--(12.307,1.680)--(12.319,1.680)--(12.331,1.680)--(12.343,1.680)--(12.356,1.680)%
  --(12.368,1.680)--(12.380,1.680)--(12.392,1.680)--(12.404,1.680)--(12.416,1.680)--(12.428,1.680)%
  --(12.440,1.680)--(12.453,1.680)--(12.465,1.680)--(12.477,1.680)--(12.489,1.680)--(12.501,1.680)%
  --(12.513,1.680)--(12.525,1.680)--(12.537,1.680)--(12.550,1.680)--(12.562,1.680)--(12.574,1.680)%
  --(12.586,1.680)--(12.598,1.680)--(12.610,1.680)--(12.622,1.680)--(12.634,1.680)--(12.647,1.680)%
  --(12.659,1.680)--(12.671,1.680)--(12.683,1.680)--(12.695,1.680)--(12.707,1.680)--(12.719,1.680)%
  --(12.732,1.680)--(12.744,1.680)--(12.756,1.680)--(12.768,1.680)--(12.780,1.680)--(12.792,1.680)%
  --(12.804,1.680)--(12.816,1.680)--(12.829,1.680)--(12.841,1.680)--(12.853,1.680)--(12.865,1.680)%
  --(12.877,1.680)--(12.889,1.680)--(12.901,1.680)--(12.913,1.680)--(12.926,1.680)--(12.938,1.680)%
  --(12.950,1.680)--(12.962,1.680)--(12.974,1.680)--(12.986,1.680)--(12.998,1.680)--(13.010,1.680)%
  --(13.023,1.680)--(13.035,1.680)--(13.047,1.680)--(13.059,1.680)--(13.071,1.680)--(13.083,1.680)%
  --(13.095,1.680)--(13.107,1.680)--(13.120,1.680)--(13.132,1.680)--(13.144,1.680)--(13.156,1.680)%
  --(13.168,1.680)--(13.180,1.680)--(13.192,1.680)--(13.204,1.680)--(13.217,1.680)--(13.229,1.680)%
  --(13.241,1.680)--(13.253,1.680)--(13.265,1.680)--(13.277,1.680)--(13.289,1.680)--(13.301,1.680)%
  --(13.314,1.680)--(13.326,1.680)--(13.338,1.680)--(13.350,1.680)--(13.362,1.680)--(13.374,1.680)%
  --(13.386,1.680)--(13.398,1.680)--(13.411,1.680)--(13.423,1.680)--(13.435,1.680)--(13.447,1.680);
\gpcolor{color=gp lt color border}
\draw[gp path] (1.320,8.631)--(1.320,0.985)--(13.447,0.985)--(13.447,8.631)--cycle;
%% coordinates of the plot area
\gpdefrectangularnode{gp plot 1}{\pgfpoint{1.320cm}{0.985cm}}{\pgfpoint{13.447cm}{8.631cm}}
\end{tikzpicture}
%% gnuplot variables

	\caption{Gráfico dos potenciais químicos de próton e nêutron calculados a partir da Equação~\eqref{Eq:Potenciais_Quimicos}. Como a fração de próton é 1/2, ambos os potenciais tem o mesmo valor.}
	\label{Fig:chemical_potential_graph}
\end{figure*}

\begin{figure*}
	\begin{tikzpicture}[gnuplot]
%% generated with GNUPLOT 5.0p2 (Lua 5.2; terminal rev. 99, script rev. 100)
%% Fri Mar 18 15:48:26 2016
\path (0.000,0.000) rectangle (14.000,9.000);
\gpcolor{color=gp lt color border}
\gpsetlinetype{gp lt border}
\gpsetdashtype{gp dt solid}
\gpsetlinewidth{1.00}
\draw[gp path] (1.320,1.680)--(1.500,1.680);
\draw[gp path] (13.447,1.680)--(13.267,1.680);
\node[gp node right] at (1.136,1.680) {$0$};
\draw[gp path] (1.320,3.070)--(1.500,3.070);
\draw[gp path] (13.447,3.070)--(13.267,3.070);
\node[gp node right] at (1.136,3.070) {$100$};
\draw[gp path] (1.320,4.460)--(1.500,4.460);
\draw[gp path] (13.447,4.460)--(13.267,4.460);
\node[gp node right] at (1.136,4.460) {$200$};
\draw[gp path] (1.320,5.851)--(1.500,5.851);
\draw[gp path] (13.447,5.851)--(13.267,5.851);
\node[gp node right] at (1.136,5.851) {$300$};
\draw[gp path] (1.320,7.241)--(1.500,7.241);
\draw[gp path] (13.447,7.241)--(13.267,7.241);
\node[gp node right] at (1.136,7.241) {$400$};
\draw[gp path] (1.320,8.631)--(1.500,8.631);
\draw[gp path] (13.447,8.631)--(13.267,8.631);
\node[gp node right] at (1.136,8.631) {$500$};
\draw[gp path] (1.320,0.985)--(1.320,1.165);
\draw[gp path] (1.320,8.631)--(1.320,8.451);
\node[gp node center] at (1.320,0.677) {$0$};
\draw[gp path] (3.745,0.985)--(3.745,1.165);
\draw[gp path] (3.745,8.631)--(3.745,8.451);
\node[gp node center] at (3.745,0.677) {$100$};
\draw[gp path] (6.171,0.985)--(6.171,1.165);
\draw[gp path] (6.171,8.631)--(6.171,8.451);
\node[gp node center] at (6.171,0.677) {$200$};
\draw[gp path] (8.596,0.985)--(8.596,1.165);
\draw[gp path] (8.596,8.631)--(8.596,8.451);
\node[gp node center] at (8.596,0.677) {$300$};
\draw[gp path] (11.022,0.985)--(11.022,1.165);
\draw[gp path] (11.022,8.631)--(11.022,8.451);
\node[gp node center] at (11.022,0.677) {$400$};
\draw[gp path] (13.447,0.985)--(13.447,1.165);
\draw[gp path] (13.447,8.631)--(13.447,8.451);
\node[gp node center] at (13.447,0.677) {$500$};
\draw[gp path] (1.320,8.631)--(1.320,0.985)--(13.447,0.985)--(13.447,8.631)--cycle;
\node[gp node center,rotate=-270] at (0.246,4.808) {$p_F$};
\node[gp node center] at (7.383,0.215) {$\mu_R$};
\node[gp node left] at (2.788,8.297) {$p_F = \sqrt{\mu_R^2 - m^2}\theta(\mu_R^2 - m^2), m = 100$};
\gpcolor{rgb color={0.580,0.000,0.827}}
\gpsetlinewidth{3.00}
\draw[gp path] (1.688,8.297)--(2.604,8.297);
\draw[gp path] (1.320,1.680)--(1.332,1.680)--(1.344,1.680)--(1.356,1.680)--(1.369,1.680)%
  --(1.381,1.680)--(1.393,1.680)--(1.405,1.680)--(1.417,1.680)--(1.429,1.680)--(1.441,1.680)%
  --(1.453,1.680)--(1.466,1.680)--(1.478,1.680)--(1.490,1.680)--(1.502,1.680)--(1.514,1.680)%
  --(1.526,1.680)--(1.538,1.680)--(1.550,1.680)--(1.563,1.680)--(1.575,1.680)--(1.587,1.680)%
  --(1.599,1.680)--(1.611,1.680)--(1.623,1.680)--(1.635,1.680)--(1.647,1.680)--(1.660,1.680)%
  --(1.672,1.680)--(1.684,1.680)--(1.696,1.680)--(1.708,1.680)--(1.720,1.680)--(1.732,1.680)%
  --(1.744,1.680)--(1.757,1.680)--(1.769,1.680)--(1.781,1.680)--(1.793,1.680)--(1.805,1.680)%
  --(1.817,1.680)--(1.829,1.680)--(1.841,1.680)--(1.854,1.680)--(1.866,1.680)--(1.878,1.680)%
  --(1.890,1.680)--(1.902,1.680)--(1.914,1.680)--(1.926,1.680)--(1.938,1.680)--(1.951,1.680)%
  --(1.963,1.680)--(1.975,1.680)--(1.987,1.680)--(1.999,1.680)--(2.011,1.680)--(2.023,1.680)%
  --(2.035,1.680)--(2.048,1.680)--(2.060,1.680)--(2.072,1.680)--(2.084,1.680)--(2.096,1.680)%
  --(2.108,1.680)--(2.120,1.680)--(2.133,1.680)--(2.145,1.680)--(2.157,1.680)--(2.169,1.680)%
  --(2.181,1.680)--(2.193,1.680)--(2.205,1.680)--(2.217,1.680)--(2.230,1.680)--(2.242,1.680)%
  --(2.254,1.680)--(2.266,1.680)--(2.278,1.680)--(2.290,1.680)--(2.302,1.680)--(2.314,1.680)%
  --(2.327,1.680)--(2.339,1.680)--(2.351,1.680)--(2.363,1.680)--(2.375,1.680)--(2.387,1.680)%
  --(2.399,1.680)--(2.411,1.680)--(2.424,1.680)--(2.436,1.680)--(2.448,1.680)--(2.460,1.680)%
  --(2.472,1.680)--(2.484,1.680)--(2.496,1.680)--(2.508,1.680)--(2.521,1.680)--(2.533,1.680)%
  --(2.545,1.680)--(2.557,1.680)--(2.569,1.680)--(2.581,1.680)--(2.593,1.680)--(2.605,1.680)%
  --(2.618,1.680)--(2.630,1.680)--(2.642,1.680)--(2.654,1.680)--(2.666,1.680)--(2.678,1.680)%
  --(2.690,1.680)--(2.702,1.680)--(2.715,1.680)--(2.727,1.680)--(2.739,1.680)--(2.751,1.680)%
  --(2.763,1.680)--(2.775,1.680)--(2.787,1.680)--(2.799,1.680)--(2.812,1.680)--(2.824,1.680)%
  --(2.836,1.680)--(2.848,1.680)--(2.860,1.680)--(2.872,1.680)--(2.884,1.680)--(2.897,1.680)%
  --(2.909,1.680)--(2.921,1.680)--(2.933,1.680)--(2.945,1.680)--(2.957,1.680)--(2.969,1.680)%
  --(2.981,1.680)--(2.994,1.680)--(3.006,1.680)--(3.018,1.680)--(3.030,1.680)--(3.042,1.680)%
  --(3.054,1.680)--(3.066,1.680)--(3.078,1.680)--(3.091,1.680)--(3.103,1.680)--(3.115,1.680)%
  --(3.127,1.680)--(3.139,1.680)--(3.151,1.680)--(3.163,1.680)--(3.175,1.680)--(3.188,1.680)%
  --(3.200,1.680)--(3.212,1.680)--(3.224,1.680)--(3.236,1.680)--(3.248,1.680)--(3.260,1.680)%
  --(3.272,1.680)--(3.285,1.680)--(3.297,1.680)--(3.309,1.680)--(3.321,1.680)--(3.333,1.680)%
  --(3.345,1.680)--(3.357,1.680)--(3.369,1.680)--(3.382,1.680)--(3.394,1.680)--(3.406,1.680)%
  --(3.418,1.680)--(3.430,1.680)--(3.442,1.680)--(3.454,1.680)--(3.466,1.680)--(3.479,1.680)%
  --(3.491,1.680)--(3.503,1.680)--(3.515,1.680)--(3.527,1.680)--(3.539,1.680)--(3.551,1.680)%
  --(3.563,1.680)--(3.576,1.680)--(3.588,1.680)--(3.600,1.680)--(3.612,1.680)--(3.624,1.680)%
  --(3.636,1.680)--(3.648,1.680)--(3.661,1.680)--(3.673,1.680)--(3.685,1.680)--(3.697,1.680)%
  --(3.709,1.680)--(3.721,1.680)--(3.733,1.680)--(3.745,1.680)--(3.758,1.819)--(3.770,1.877)%
  --(3.782,1.922)--(3.794,1.960)--(3.806,1.993)--(3.818,2.023)--(3.830,2.051)--(3.842,2.077)%
  --(3.855,2.102)--(3.867,2.125)--(3.879,2.147)--(3.891,2.169)--(3.903,2.189)--(3.915,2.209)%
  --(3.927,2.229)--(3.939,2.247)--(3.952,2.265)--(3.964,2.283)--(3.976,2.300)--(3.988,2.317)%
  --(4.000,2.334)--(4.012,2.350)--(4.024,2.366)--(4.036,2.381)--(4.049,2.397)--(4.061,2.412)%
  --(4.073,2.426)--(4.085,2.441)--(4.097,2.455)--(4.109,2.470)--(4.121,2.484)--(4.133,2.497)%
  --(4.146,2.511)--(4.158,2.524)--(4.170,2.538)--(4.182,2.551)--(4.194,2.564)--(4.206,2.577)%
  --(4.218,2.590)--(4.230,2.602)--(4.243,2.615)--(4.255,2.627)--(4.267,2.639)--(4.279,2.652)%
  --(4.291,2.664)--(4.303,2.676)--(4.315,2.688)--(4.327,2.699)--(4.340,2.711)--(4.352,2.723)%
  --(4.364,2.734)--(4.376,2.746)--(4.388,2.757)--(4.400,2.768)--(4.412,2.780)--(4.425,2.791)%
  --(4.437,2.802)--(4.449,2.813)--(4.461,2.824)--(4.473,2.835)--(4.485,2.846)--(4.497,2.857)%
  --(4.509,2.867)--(4.522,2.878)--(4.534,2.889)--(4.546,2.899)--(4.558,2.910)--(4.570,2.920)%
  --(4.582,2.931)--(4.594,2.941)--(4.606,2.951)--(4.619,2.961)--(4.631,2.972)--(4.643,2.982)%
  --(4.655,2.992)--(4.667,3.002)--(4.679,3.012)--(4.691,3.022)--(4.703,3.032)--(4.716,3.042)%
  --(4.728,3.052)--(4.740,3.062)--(4.752,3.072)--(4.764,3.082)--(4.776,3.091)--(4.788,3.101)%
  --(4.800,3.111)--(4.813,3.121)--(4.825,3.130)--(4.837,3.140)--(4.849,3.149)--(4.861,3.159)%
  --(4.873,3.168)--(4.885,3.178)--(4.897,3.187)--(4.910,3.197)--(4.922,3.206)--(4.934,3.216)%
  --(4.946,3.225)--(4.958,3.234)--(4.970,3.244)--(4.982,3.253)--(4.994,3.262)--(5.007,3.271)%
  --(5.019,3.281)--(5.031,3.290)--(5.043,3.299)--(5.055,3.308)--(5.067,3.317)--(5.079,3.326)%
  --(5.091,3.336)--(5.104,3.345)--(5.116,3.354)--(5.128,3.363)--(5.140,3.372)--(5.152,3.381)%
  --(5.164,3.390)--(5.176,3.399)--(5.189,3.408)--(5.201,3.416)--(5.213,3.425)--(5.225,3.434)%
  --(5.237,3.443)--(5.249,3.452)--(5.261,3.461)--(5.273,3.470)--(5.286,3.478)--(5.298,3.487)%
  --(5.310,3.496)--(5.322,3.505)--(5.334,3.513)--(5.346,3.522)--(5.358,3.531)--(5.370,3.539)%
  --(5.383,3.548)--(5.395,3.557)--(5.407,3.565)--(5.419,3.574)--(5.431,3.583)--(5.443,3.591)%
  --(5.455,3.600)--(5.467,3.608)--(5.480,3.617)--(5.492,3.626)--(5.504,3.634)--(5.516,3.643)%
  --(5.528,3.651)--(5.540,3.660)--(5.552,3.668)--(5.564,3.677)--(5.577,3.685)--(5.589,3.694)%
  --(5.601,3.702)--(5.613,3.710)--(5.625,3.719)--(5.637,3.727)--(5.649,3.736)--(5.661,3.744)%
  --(5.674,3.752)--(5.686,3.761)--(5.698,3.769)--(5.710,3.777)--(5.722,3.786)--(5.734,3.794)%
  --(5.746,3.802)--(5.758,3.811)--(5.771,3.819)--(5.783,3.827)--(5.795,3.836)--(5.807,3.844)%
  --(5.819,3.852)--(5.831,3.860)--(5.843,3.869)--(5.855,3.877)--(5.868,3.885)--(5.880,3.893)%
  --(5.892,3.901)--(5.904,3.910)--(5.916,3.918)--(5.928,3.926)--(5.940,3.934)--(5.953,3.942)%
  --(5.965,3.950)--(5.977,3.959)--(5.989,3.967)--(6.001,3.975)--(6.013,3.983)--(6.025,3.991)%
  --(6.037,3.999)--(6.050,4.007)--(6.062,4.015)--(6.074,4.024)--(6.086,4.032)--(6.098,4.040)%
  --(6.110,4.048)--(6.122,4.056)--(6.134,4.064)--(6.147,4.072)--(6.159,4.080)--(6.171,4.088)%
  --(6.183,4.096)--(6.195,4.104)--(6.207,4.112)--(6.219,4.120)--(6.231,4.128)--(6.244,4.136)%
  --(6.256,4.144)--(6.268,4.152)--(6.280,4.160)--(6.292,4.168)--(6.304,4.176)--(6.316,4.184)%
  --(6.328,4.192)--(6.341,4.200)--(6.353,4.208)--(6.365,4.216)--(6.377,4.223)--(6.389,4.231)%
  --(6.401,4.239)--(6.413,4.247)--(6.425,4.255)--(6.438,4.263)--(6.450,4.271)--(6.462,4.279)%
  --(6.474,4.287)--(6.486,4.295)--(6.498,4.302)--(6.510,4.310)--(6.522,4.318)--(6.535,4.326)%
  --(6.547,4.334)--(6.559,4.342)--(6.571,4.350)--(6.583,4.357)--(6.595,4.365)--(6.607,4.373)%
  --(6.619,4.381)--(6.632,4.389)--(6.644,4.396)--(6.656,4.404)--(6.668,4.412)--(6.680,4.420)%
  --(6.692,4.428)--(6.704,4.435)--(6.717,4.443)--(6.729,4.451)--(6.741,4.459)--(6.753,4.467)%
  --(6.765,4.474)--(6.777,4.482)--(6.789,4.490)--(6.801,4.498)--(6.814,4.505)--(6.826,4.513)%
  --(6.838,4.521)--(6.850,4.529)--(6.862,4.536)--(6.874,4.544)--(6.886,4.552)--(6.898,4.559)%
  --(6.911,4.567)--(6.923,4.575)--(6.935,4.583)--(6.947,4.590)--(6.959,4.598)--(6.971,4.606)%
  --(6.983,4.613)--(6.995,4.621)--(7.008,4.629)--(7.020,4.636)--(7.032,4.644)--(7.044,4.652)%
  --(7.056,4.660)--(7.068,4.667)--(7.080,4.675)--(7.092,4.682)--(7.105,4.690)--(7.117,4.698)%
  --(7.129,4.705)--(7.141,4.713)--(7.153,4.721)--(7.165,4.728)--(7.177,4.736)--(7.189,4.744)%
  --(7.202,4.751)--(7.214,4.759)--(7.226,4.767)--(7.238,4.774)--(7.250,4.782)--(7.262,4.789)%
  --(7.274,4.797)--(7.286,4.805)--(7.299,4.812)--(7.311,4.820)--(7.323,4.827)--(7.335,4.835)%
  --(7.347,4.843)--(7.359,4.850)--(7.371,4.858)--(7.384,4.865)--(7.396,4.873)--(7.408,4.881)%
  --(7.420,4.888)--(7.432,4.896)--(7.444,4.903)--(7.456,4.911)--(7.468,4.918)--(7.481,4.926)%
  --(7.493,4.934)--(7.505,4.941)--(7.517,4.949)--(7.529,4.956)--(7.541,4.964)--(7.553,4.971)%
  --(7.565,4.979)--(7.578,4.986)--(7.590,4.994)--(7.602,5.001)--(7.614,5.009)--(7.626,5.017)%
  --(7.638,5.024)--(7.650,5.032)--(7.662,5.039)--(7.675,5.047)--(7.687,5.054)--(7.699,5.062)%
  --(7.711,5.069)--(7.723,5.077)--(7.735,5.084)--(7.747,5.092)--(7.759,5.099)--(7.772,5.107)%
  --(7.784,5.114)--(7.796,5.122)--(7.808,5.129)--(7.820,5.137)--(7.832,5.144)--(7.844,5.152)%
  --(7.856,5.159)--(7.869,5.167)--(7.881,5.174)--(7.893,5.182)--(7.905,5.189)--(7.917,5.197)%
  --(7.929,5.204)--(7.941,5.212)--(7.953,5.219)--(7.966,5.226)--(7.978,5.234)--(7.990,5.241)%
  --(8.002,5.249)--(8.014,5.256)--(8.026,5.264)--(8.038,5.271)--(8.050,5.279)--(8.063,5.286)%
  --(8.075,5.294)--(8.087,5.301)--(8.099,5.308)--(8.111,5.316)--(8.123,5.323)--(8.135,5.331)%
  --(8.148,5.338)--(8.160,5.346)--(8.172,5.353)--(8.184,5.361)--(8.196,5.368)--(8.208,5.375)%
  --(8.220,5.383)--(8.232,5.390)--(8.245,5.398)--(8.257,5.405)--(8.269,5.412)--(8.281,5.420)%
  --(8.293,5.427)--(8.305,5.435)--(8.317,5.442)--(8.329,5.450)--(8.342,5.457)--(8.354,5.464)%
  --(8.366,5.472)--(8.378,5.479)--(8.390,5.487)--(8.402,5.494)--(8.414,5.501)--(8.426,5.509)%
  --(8.439,5.516)--(8.451,5.524)--(8.463,5.531)--(8.475,5.538)--(8.487,5.546)--(8.499,5.553)%
  --(8.511,5.560)--(8.523,5.568)--(8.536,5.575)--(8.548,5.583)--(8.560,5.590)--(8.572,5.597)%
  --(8.584,5.605)--(8.596,5.612)--(8.608,5.619)--(8.620,5.627)--(8.633,5.634)--(8.645,5.642)%
  --(8.657,5.649)--(8.669,5.656)--(8.681,5.664)--(8.693,5.671)--(8.705,5.678)--(8.717,5.686)%
  --(8.730,5.693)--(8.742,5.700)--(8.754,5.708)--(8.766,5.715)--(8.778,5.723)--(8.790,5.730)%
  --(8.802,5.737)--(8.814,5.745)--(8.827,5.752)--(8.839,5.759)--(8.851,5.767)--(8.863,5.774)%
  --(8.875,5.781)--(8.887,5.789)--(8.899,5.796)--(8.912,5.803)--(8.924,5.811)--(8.936,5.818)%
  --(8.948,5.825)--(8.960,5.833)--(8.972,5.840)--(8.984,5.847)--(8.996,5.855)--(9.009,5.862)%
  --(9.021,5.869)--(9.033,5.877)--(9.045,5.884)--(9.057,5.891)--(9.069,5.899)--(9.081,5.906)%
  --(9.093,5.913)--(9.106,5.921)--(9.118,5.928)--(9.130,5.935)--(9.142,5.942)--(9.154,5.950)%
  --(9.166,5.957)--(9.178,5.964)--(9.190,5.972)--(9.203,5.979)--(9.215,5.986)--(9.227,5.994)%
  --(9.239,6.001)--(9.251,6.008)--(9.263,6.016)--(9.275,6.023)--(9.287,6.030)--(9.300,6.037)%
  --(9.312,6.045)--(9.324,6.052)--(9.336,6.059)--(9.348,6.067)--(9.360,6.074)--(9.372,6.081)%
  --(9.384,6.088)--(9.397,6.096)--(9.409,6.103)--(9.421,6.110)--(9.433,6.118)--(9.445,6.125)%
  --(9.457,6.132)--(9.469,6.139)--(9.481,6.147)--(9.494,6.154)--(9.506,6.161)--(9.518,6.169)%
  --(9.530,6.176)--(9.542,6.183)--(9.554,6.190)--(9.566,6.198)--(9.578,6.205)--(9.591,6.212)%
  --(9.603,6.219)--(9.615,6.227)--(9.627,6.234)--(9.639,6.241)--(9.651,6.249)--(9.663,6.256)%
  --(9.676,6.263)--(9.688,6.270)--(9.700,6.278)--(9.712,6.285)--(9.724,6.292)--(9.736,6.299)%
  --(9.748,6.307)--(9.760,6.314)--(9.773,6.321)--(9.785,6.328)--(9.797,6.336)--(9.809,6.343)%
  --(9.821,6.350)--(9.833,6.357)--(9.845,6.365)--(9.857,6.372)--(9.870,6.379)--(9.882,6.386)%
  --(9.894,6.394)--(9.906,6.401)--(9.918,6.408)--(9.930,6.415)--(9.942,6.423)--(9.954,6.430)%
  --(9.967,6.437)--(9.979,6.444)--(9.991,6.452)--(10.003,6.459)--(10.015,6.466)--(10.027,6.473)%
  --(10.039,6.481)--(10.051,6.488)--(10.064,6.495)--(10.076,6.502)--(10.088,6.509)--(10.100,6.517)%
  --(10.112,6.524)--(10.124,6.531)--(10.136,6.538)--(10.148,6.546)--(10.161,6.553)--(10.173,6.560)%
  --(10.185,6.567)--(10.197,6.575)--(10.209,6.582)--(10.221,6.589)--(10.233,6.596)--(10.245,6.603)%
  --(10.258,6.611)--(10.270,6.618)--(10.282,6.625)--(10.294,6.632)--(10.306,6.640)--(10.318,6.647)%
  --(10.330,6.654)--(10.342,6.661)--(10.355,6.668)--(10.367,6.676)--(10.379,6.683)--(10.391,6.690)%
  --(10.403,6.697)--(10.415,6.704)--(10.427,6.712)--(10.440,6.719)--(10.452,6.726)--(10.464,6.733)%
  --(10.476,6.741)--(10.488,6.748)--(10.500,6.755)--(10.512,6.762)--(10.524,6.769)--(10.537,6.777)%
  --(10.549,6.784)--(10.561,6.791)--(10.573,6.798)--(10.585,6.805)--(10.597,6.813)--(10.609,6.820)%
  --(10.621,6.827)--(10.634,6.834)--(10.646,6.841)--(10.658,6.849)--(10.670,6.856)--(10.682,6.863)%
  --(10.694,6.870)--(10.706,6.877)--(10.718,6.885)--(10.731,6.892)--(10.743,6.899)--(10.755,6.906)%
  --(10.767,6.913)--(10.779,6.921)--(10.791,6.928)--(10.803,6.935)--(10.815,6.942)--(10.828,6.949)%
  --(10.840,6.956)--(10.852,6.964)--(10.864,6.971)--(10.876,6.978)--(10.888,6.985)--(10.900,6.992)%
  --(10.912,7.000)--(10.925,7.007)--(10.937,7.014)--(10.949,7.021)--(10.961,7.028)--(10.973,7.036)%
  --(10.985,7.043)--(10.997,7.050)--(11.009,7.057)--(11.022,7.064)--(11.034,7.071)--(11.046,7.079)%
  --(11.058,7.086)--(11.070,7.093)--(11.082,7.100)--(11.094,7.107)--(11.106,7.114)--(11.119,7.122)%
  --(11.131,7.129)--(11.143,7.136)--(11.155,7.143)--(11.167,7.150)--(11.179,7.158)--(11.191,7.165)%
  --(11.204,7.172)--(11.216,7.179)--(11.228,7.186)--(11.240,7.193)--(11.252,7.201)--(11.264,7.208)%
  --(11.276,7.215)--(11.288,7.222)--(11.301,7.229)--(11.313,7.236)--(11.325,7.244)--(11.337,7.251)%
  --(11.349,7.258)--(11.361,7.265)--(11.373,7.272)--(11.385,7.279)--(11.398,7.287)--(11.410,7.294)%
  --(11.422,7.301)--(11.434,7.308)--(11.446,7.315)--(11.458,7.322)--(11.470,7.329)--(11.482,7.337)%
  --(11.495,7.344)--(11.507,7.351)--(11.519,7.358)--(11.531,7.365)--(11.543,7.372)--(11.555,7.380)%
  --(11.567,7.387)--(11.579,7.394)--(11.592,7.401)--(11.604,7.408)--(11.616,7.415)--(11.628,7.422)%
  --(11.640,7.430)--(11.652,7.437)--(11.664,7.444)--(11.676,7.451)--(11.689,7.458)--(11.701,7.465)%
  --(11.713,7.473)--(11.725,7.480)--(11.737,7.487)--(11.749,7.494)--(11.761,7.501)--(11.773,7.508)%
  --(11.786,7.515)--(11.798,7.523)--(11.810,7.530)--(11.822,7.537)--(11.834,7.544)--(11.846,7.551)%
  --(11.858,7.558)--(11.870,7.565)--(11.883,7.573)--(11.895,7.580)--(11.907,7.587)--(11.919,7.594)%
  --(11.931,7.601)--(11.943,7.608)--(11.955,7.615)--(11.968,7.623)--(11.980,7.630)--(11.992,7.637)%
  --(12.004,7.644)--(12.016,7.651)--(12.028,7.658)--(12.040,7.665)--(12.052,7.673)--(12.065,7.680)%
  --(12.077,7.687)--(12.089,7.694)--(12.101,7.701)--(12.113,7.708)--(12.125,7.715)--(12.137,7.722)%
  --(12.149,7.730)--(12.162,7.737)--(12.174,7.744)--(12.186,7.751)--(12.198,7.758)--(12.210,7.765)%
  --(12.222,7.772)--(12.234,7.779)--(12.246,7.787)--(12.259,7.794)--(12.271,7.801)--(12.283,7.808)%
  --(12.295,7.815)--(12.307,7.822)--(12.319,7.829)--(12.331,7.837)--(12.343,7.844)--(12.356,7.851)%
  --(12.368,7.858)--(12.380,7.865)--(12.392,7.872)--(12.404,7.879)--(12.416,7.886)--(12.428,7.894)%
  --(12.440,7.901)--(12.453,7.908)--(12.465,7.915)--(12.477,7.922)--(12.489,7.929)--(12.501,7.936)%
  --(12.513,7.943)--(12.525,7.950)--(12.537,7.958)--(12.550,7.965)--(12.562,7.972)--(12.574,7.979)%
  --(12.586,7.986)--(12.598,7.993)--(12.610,8.000)--(12.622,8.007)--(12.634,8.015)--(12.647,8.022)%
  --(12.659,8.029)--(12.671,8.036)--(12.683,8.043)--(12.695,8.050)--(12.707,8.057)--(12.719,8.064)%
  --(12.732,8.071)--(12.744,8.079)--(12.756,8.086)--(12.768,8.093)--(12.780,8.100)--(12.792,8.107)%
  --(12.804,8.114)--(12.816,8.121)--(12.829,8.128)--(12.841,8.135)--(12.853,8.143)--(12.865,8.150)%
  --(12.877,8.157)--(12.889,8.164)--(12.901,8.171)--(12.913,8.178)--(12.926,8.185)--(12.938,8.192)%
  --(12.950,8.199)--(12.962,8.207)--(12.974,8.214)--(12.986,8.221)--(12.998,8.228)--(13.010,8.235)%
  --(13.023,8.242)--(13.035,8.249)--(13.047,8.256)--(13.059,8.263)--(13.071,8.270)--(13.083,8.278)%
  --(13.095,8.285)--(13.107,8.292)--(13.120,8.299)--(13.132,8.306)--(13.144,8.313)--(13.156,8.320)%
  --(13.168,8.327)--(13.180,8.334)--(13.192,8.342)--(13.204,8.349)--(13.217,8.356)--(13.229,8.363)%
  --(13.241,8.370)--(13.253,8.377)--(13.265,8.384)--(13.277,8.391)--(13.289,8.398)--(13.301,8.405)%
  --(13.314,8.413)--(13.326,8.420)--(13.338,8.427)--(13.350,8.434)--(13.362,8.441)--(13.374,8.448)%
  --(13.386,8.455)--(13.398,8.462)--(13.411,8.469)--(13.423,8.476)--(13.435,8.483)--(13.447,8.491);
\gpcolor{rgb color={0.000,0.620,0.451}}
\gpsetlinewidth{1.00}
\draw[gp path] (1.320,1.680)--(1.332,1.680)--(1.344,1.680)--(1.356,1.680)--(1.369,1.680)%
  --(1.381,1.680)--(1.393,1.680)--(1.405,1.680)--(1.417,1.680)--(1.429,1.680)--(1.441,1.680)%
  --(1.453,1.680)--(1.466,1.680)--(1.478,1.680)--(1.490,1.680)--(1.502,1.680)--(1.514,1.680)%
  --(1.526,1.680)--(1.538,1.680)--(1.550,1.680)--(1.563,1.680)--(1.575,1.680)--(1.587,1.680)%
  --(1.599,1.680)--(1.611,1.680)--(1.623,1.680)--(1.635,1.680)--(1.647,1.680)--(1.660,1.680)%
  --(1.672,1.680)--(1.684,1.680)--(1.696,1.680)--(1.708,1.680)--(1.720,1.680)--(1.732,1.680)%
  --(1.744,1.680)--(1.757,1.680)--(1.769,1.680)--(1.781,1.680)--(1.793,1.680)--(1.805,1.680)%
  --(1.817,1.680)--(1.829,1.680)--(1.841,1.680)--(1.854,1.680)--(1.866,1.680)--(1.878,1.680)%
  --(1.890,1.680)--(1.902,1.680)--(1.914,1.680)--(1.926,1.680)--(1.938,1.680)--(1.951,1.680)%
  --(1.963,1.680)--(1.975,1.680)--(1.987,1.680)--(1.999,1.680)--(2.011,1.680)--(2.023,1.680)%
  --(2.035,1.680)--(2.048,1.680)--(2.060,1.680)--(2.072,1.680)--(2.084,1.680)--(2.096,1.680)%
  --(2.108,1.680)--(2.120,1.680)--(2.133,1.680)--(2.145,1.680)--(2.157,1.680)--(2.169,1.680)%
  --(2.181,1.680)--(2.193,1.680)--(2.205,1.680)--(2.217,1.680)--(2.230,1.680)--(2.242,1.680)%
  --(2.254,1.680)--(2.266,1.680)--(2.278,1.680)--(2.290,1.680)--(2.302,1.680)--(2.314,1.680)%
  --(2.327,1.680)--(2.339,1.680)--(2.351,1.680)--(2.363,1.680)--(2.375,1.680)--(2.387,1.680)%
  --(2.399,1.680)--(2.411,1.680)--(2.424,1.680)--(2.436,1.680)--(2.448,1.680)--(2.460,1.680)%
  --(2.472,1.680)--(2.484,1.680)--(2.496,1.680)--(2.508,1.680)--(2.521,1.680)--(2.533,1.680)%
  --(2.545,1.680)--(2.557,1.680)--(2.569,1.680)--(2.581,1.680)--(2.593,1.680)--(2.605,1.680)%
  --(2.618,1.680)--(2.630,1.680)--(2.642,1.680)--(2.654,1.680)--(2.666,1.680)--(2.678,1.680)%
  --(2.690,1.680)--(2.702,1.680)--(2.715,1.680)--(2.727,1.680)--(2.739,1.680)--(2.751,1.680)%
  --(2.763,1.680)--(2.775,1.680)--(2.787,1.680)--(2.799,1.680)--(2.812,1.680)--(2.824,1.680)%
  --(2.836,1.680)--(2.848,1.680)--(2.860,1.680)--(2.872,1.680)--(2.884,1.680)--(2.897,1.680)%
  --(2.909,1.680)--(2.921,1.680)--(2.933,1.680)--(2.945,1.680)--(2.957,1.680)--(2.969,1.680)%
  --(2.981,1.680)--(2.994,1.680)--(3.006,1.680)--(3.018,1.680)--(3.030,1.680)--(3.042,1.680)%
  --(3.054,1.680)--(3.066,1.680)--(3.078,1.680)--(3.091,1.680)--(3.103,1.680)--(3.115,1.680)%
  --(3.127,1.680)--(3.139,1.680)--(3.151,1.680)--(3.163,1.680)--(3.175,1.680)--(3.188,1.680)%
  --(3.200,1.680)--(3.212,1.680)--(3.224,1.680)--(3.236,1.680)--(3.248,1.680)--(3.260,1.680)%
  --(3.272,1.680)--(3.285,1.680)--(3.297,1.680)--(3.309,1.680)--(3.321,1.680)--(3.333,1.680)%
  --(3.345,1.680)--(3.357,1.680)--(3.369,1.680)--(3.382,1.680)--(3.394,1.680)--(3.406,1.680)%
  --(3.418,1.680)--(3.430,1.680)--(3.442,1.680)--(3.454,1.680)--(3.466,1.680)--(3.479,1.680)%
  --(3.491,1.680)--(3.503,1.680)--(3.515,1.680)--(3.527,1.680)--(3.539,1.680)--(3.551,1.680)%
  --(3.563,1.680)--(3.576,1.680)--(3.588,1.680)--(3.600,1.680)--(3.612,1.680)--(3.624,1.680)%
  --(3.636,1.680)--(3.648,1.680)--(3.661,1.680)--(3.673,1.680)--(3.685,1.680)--(3.697,1.680)%
  --(3.709,1.680)--(3.721,1.680)--(3.733,1.680)--(3.745,1.680)--(3.758,1.680)--(3.770,1.680)%
  --(3.782,1.680)--(3.794,1.680)--(3.806,1.680)--(3.818,1.680)--(3.830,1.680)--(3.842,1.680)%
  --(3.855,1.680)--(3.867,1.680)--(3.879,1.680)--(3.891,1.680)--(3.903,1.680)--(3.915,1.680)%
  --(3.927,1.680)--(3.939,1.680)--(3.952,1.680)--(3.964,1.680)--(3.976,1.680)--(3.988,1.680)%
  --(4.000,1.680)--(4.012,1.680)--(4.024,1.680)--(4.036,1.680)--(4.049,1.680)--(4.061,1.680)%
  --(4.073,1.680)--(4.085,1.680)--(4.097,1.680)--(4.109,1.680)--(4.121,1.680)--(4.133,1.680)%
  --(4.146,1.680)--(4.158,1.680)--(4.170,1.680)--(4.182,1.680)--(4.194,1.680)--(4.206,1.680)%
  --(4.218,1.680)--(4.230,1.680)--(4.243,1.680)--(4.255,1.680)--(4.267,1.680)--(4.279,1.680)%
  --(4.291,1.680)--(4.303,1.680)--(4.315,1.680)--(4.327,1.680)--(4.340,1.680)--(4.352,1.680)%
  --(4.364,1.680)--(4.376,1.680)--(4.388,1.680)--(4.400,1.680)--(4.412,1.680)--(4.425,1.680)%
  --(4.437,1.680)--(4.449,1.680)--(4.461,1.680)--(4.473,1.680)--(4.485,1.680)--(4.497,1.680)%
  --(4.509,1.680)--(4.522,1.680)--(4.534,1.680)--(4.546,1.680)--(4.558,1.680)--(4.570,1.680)%
  --(4.582,1.680)--(4.594,1.680)--(4.606,1.680)--(4.619,1.680)--(4.631,1.680)--(4.643,1.680)%
  --(4.655,1.680)--(4.667,1.680)--(4.679,1.680)--(4.691,1.680)--(4.703,1.680)--(4.716,1.680)%
  --(4.728,1.680)--(4.740,1.680)--(4.752,1.680)--(4.764,1.680)--(4.776,1.680)--(4.788,1.680)%
  --(4.800,1.680)--(4.813,1.680)--(4.825,1.680)--(4.837,1.680)--(4.849,1.680)--(4.861,1.680)%
  --(4.873,1.680)--(4.885,1.680)--(4.897,1.680)--(4.910,1.680)--(4.922,1.680)--(4.934,1.680)%
  --(4.946,1.680)--(4.958,1.680)--(4.970,1.680)--(4.982,1.680)--(4.994,1.680)--(5.007,1.680)%
  --(5.019,1.680)--(5.031,1.680)--(5.043,1.680)--(5.055,1.680)--(5.067,1.680)--(5.079,1.680)%
  --(5.091,1.680)--(5.104,1.680)--(5.116,1.680)--(5.128,1.680)--(5.140,1.680)--(5.152,1.680)%
  --(5.164,1.680)--(5.176,1.680)--(5.189,1.680)--(5.201,1.680)--(5.213,1.680)--(5.225,1.680)%
  --(5.237,1.680)--(5.249,1.680)--(5.261,1.680)--(5.273,1.680)--(5.286,1.680)--(5.298,1.680)%
  --(5.310,1.680)--(5.322,1.680)--(5.334,1.680)--(5.346,1.680)--(5.358,1.680)--(5.370,1.680)%
  --(5.383,1.680)--(5.395,1.680)--(5.407,1.680)--(5.419,1.680)--(5.431,1.680)--(5.443,1.680)%
  --(5.455,1.680)--(5.467,1.680)--(5.480,1.680)--(5.492,1.680)--(5.504,1.680)--(5.516,1.680)%
  --(5.528,1.680)--(5.540,1.680)--(5.552,1.680)--(5.564,1.680)--(5.577,1.680)--(5.589,1.680)%
  --(5.601,1.680)--(5.613,1.680)--(5.625,1.680)--(5.637,1.680)--(5.649,1.680)--(5.661,1.680)%
  --(5.674,1.680)--(5.686,1.680)--(5.698,1.680)--(5.710,1.680)--(5.722,1.680)--(5.734,1.680)%
  --(5.746,1.680)--(5.758,1.680)--(5.771,1.680)--(5.783,1.680)--(5.795,1.680)--(5.807,1.680)%
  --(5.819,1.680)--(5.831,1.680)--(5.843,1.680)--(5.855,1.680)--(5.868,1.680)--(5.880,1.680)%
  --(5.892,1.680)--(5.904,1.680)--(5.916,1.680)--(5.928,1.680)--(5.940,1.680)--(5.953,1.680)%
  --(5.965,1.680)--(5.977,1.680)--(5.989,1.680)--(6.001,1.680)--(6.013,1.680)--(6.025,1.680)%
  --(6.037,1.680)--(6.050,1.680)--(6.062,1.680)--(6.074,1.680)--(6.086,1.680)--(6.098,1.680)%
  --(6.110,1.680)--(6.122,1.680)--(6.134,1.680)--(6.147,1.680)--(6.159,1.680)--(6.171,1.680)%
  --(6.183,1.680)--(6.195,1.680)--(6.207,1.680)--(6.219,1.680)--(6.231,1.680)--(6.244,1.680)%
  --(6.256,1.680)--(6.268,1.680)--(6.280,1.680)--(6.292,1.680)--(6.304,1.680)--(6.316,1.680)%
  --(6.328,1.680)--(6.341,1.680)--(6.353,1.680)--(6.365,1.680)--(6.377,1.680)--(6.389,1.680)%
  --(6.401,1.680)--(6.413,1.680)--(6.425,1.680)--(6.438,1.680)--(6.450,1.680)--(6.462,1.680)%
  --(6.474,1.680)--(6.486,1.680)--(6.498,1.680)--(6.510,1.680)--(6.522,1.680)--(6.535,1.680)%
  --(6.547,1.680)--(6.559,1.680)--(6.571,1.680)--(6.583,1.680)--(6.595,1.680)--(6.607,1.680)%
  --(6.619,1.680)--(6.632,1.680)--(6.644,1.680)--(6.656,1.680)--(6.668,1.680)--(6.680,1.680)%
  --(6.692,1.680)--(6.704,1.680)--(6.717,1.680)--(6.729,1.680)--(6.741,1.680)--(6.753,1.680)%
  --(6.765,1.680)--(6.777,1.680)--(6.789,1.680)--(6.801,1.680)--(6.814,1.680)--(6.826,1.680)%
  --(6.838,1.680)--(6.850,1.680)--(6.862,1.680)--(6.874,1.680)--(6.886,1.680)--(6.898,1.680)%
  --(6.911,1.680)--(6.923,1.680)--(6.935,1.680)--(6.947,1.680)--(6.959,1.680)--(6.971,1.680)%
  --(6.983,1.680)--(6.995,1.680)--(7.008,1.680)--(7.020,1.680)--(7.032,1.680)--(7.044,1.680)%
  --(7.056,1.680)--(7.068,1.680)--(7.080,1.680)--(7.092,1.680)--(7.105,1.680)--(7.117,1.680)%
  --(7.129,1.680)--(7.141,1.680)--(7.153,1.680)--(7.165,1.680)--(7.177,1.680)--(7.189,1.680)%
  --(7.202,1.680)--(7.214,1.680)--(7.226,1.680)--(7.238,1.680)--(7.250,1.680)--(7.262,1.680)%
  --(7.274,1.680)--(7.286,1.680)--(7.299,1.680)--(7.311,1.680)--(7.323,1.680)--(7.335,1.680)%
  --(7.347,1.680)--(7.359,1.680)--(7.371,1.680)--(7.384,1.680)--(7.396,1.680)--(7.408,1.680)%
  --(7.420,1.680)--(7.432,1.680)--(7.444,1.680)--(7.456,1.680)--(7.468,1.680)--(7.481,1.680)%
  --(7.493,1.680)--(7.505,1.680)--(7.517,1.680)--(7.529,1.680)--(7.541,1.680)--(7.553,1.680)%
  --(7.565,1.680)--(7.578,1.680)--(7.590,1.680)--(7.602,1.680)--(7.614,1.680)--(7.626,1.680)%
  --(7.638,1.680)--(7.650,1.680)--(7.662,1.680)--(7.675,1.680)--(7.687,1.680)--(7.699,1.680)%
  --(7.711,1.680)--(7.723,1.680)--(7.735,1.680)--(7.747,1.680)--(7.759,1.680)--(7.772,1.680)%
  --(7.784,1.680)--(7.796,1.680)--(7.808,1.680)--(7.820,1.680)--(7.832,1.680)--(7.844,1.680)%
  --(7.856,1.680)--(7.869,1.680)--(7.881,1.680)--(7.893,1.680)--(7.905,1.680)--(7.917,1.680)%
  --(7.929,1.680)--(7.941,1.680)--(7.953,1.680)--(7.966,1.680)--(7.978,1.680)--(7.990,1.680)%
  --(8.002,1.680)--(8.014,1.680)--(8.026,1.680)--(8.038,1.680)--(8.050,1.680)--(8.063,1.680)%
  --(8.075,1.680)--(8.087,1.680)--(8.099,1.680)--(8.111,1.680)--(8.123,1.680)--(8.135,1.680)%
  --(8.148,1.680)--(8.160,1.680)--(8.172,1.680)--(8.184,1.680)--(8.196,1.680)--(8.208,1.680)%
  --(8.220,1.680)--(8.232,1.680)--(8.245,1.680)--(8.257,1.680)--(8.269,1.680)--(8.281,1.680)%
  --(8.293,1.680)--(8.305,1.680)--(8.317,1.680)--(8.329,1.680)--(8.342,1.680)--(8.354,1.680)%
  --(8.366,1.680)--(8.378,1.680)--(8.390,1.680)--(8.402,1.680)--(8.414,1.680)--(8.426,1.680)%
  --(8.439,1.680)--(8.451,1.680)--(8.463,1.680)--(8.475,1.680)--(8.487,1.680)--(8.499,1.680)%
  --(8.511,1.680)--(8.523,1.680)--(8.536,1.680)--(8.548,1.680)--(8.560,1.680)--(8.572,1.680)%
  --(8.584,1.680)--(8.596,1.680)--(8.608,1.680)--(8.620,1.680)--(8.633,1.680)--(8.645,1.680)%
  --(8.657,1.680)--(8.669,1.680)--(8.681,1.680)--(8.693,1.680)--(8.705,1.680)--(8.717,1.680)%
  --(8.730,1.680)--(8.742,1.680)--(8.754,1.680)--(8.766,1.680)--(8.778,1.680)--(8.790,1.680)%
  --(8.802,1.680)--(8.814,1.680)--(8.827,1.680)--(8.839,1.680)--(8.851,1.680)--(8.863,1.680)%
  --(8.875,1.680)--(8.887,1.680)--(8.899,1.680)--(8.912,1.680)--(8.924,1.680)--(8.936,1.680)%
  --(8.948,1.680)--(8.960,1.680)--(8.972,1.680)--(8.984,1.680)--(8.996,1.680)--(9.009,1.680)%
  --(9.021,1.680)--(9.033,1.680)--(9.045,1.680)--(9.057,1.680)--(9.069,1.680)--(9.081,1.680)%
  --(9.093,1.680)--(9.106,1.680)--(9.118,1.680)--(9.130,1.680)--(9.142,1.680)--(9.154,1.680)%
  --(9.166,1.680)--(9.178,1.680)--(9.190,1.680)--(9.203,1.680)--(9.215,1.680)--(9.227,1.680)%
  --(9.239,1.680)--(9.251,1.680)--(9.263,1.680)--(9.275,1.680)--(9.287,1.680)--(9.300,1.680)%
  --(9.312,1.680)--(9.324,1.680)--(9.336,1.680)--(9.348,1.680)--(9.360,1.680)--(9.372,1.680)%
  --(9.384,1.680)--(9.397,1.680)--(9.409,1.680)--(9.421,1.680)--(9.433,1.680)--(9.445,1.680)%
  --(9.457,1.680)--(9.469,1.680)--(9.481,1.680)--(9.494,1.680)--(9.506,1.680)--(9.518,1.680)%
  --(9.530,1.680)--(9.542,1.680)--(9.554,1.680)--(9.566,1.680)--(9.578,1.680)--(9.591,1.680)%
  --(9.603,1.680)--(9.615,1.680)--(9.627,1.680)--(9.639,1.680)--(9.651,1.680)--(9.663,1.680)%
  --(9.676,1.680)--(9.688,1.680)--(9.700,1.680)--(9.712,1.680)--(9.724,1.680)--(9.736,1.680)%
  --(9.748,1.680)--(9.760,1.680)--(9.773,1.680)--(9.785,1.680)--(9.797,1.680)--(9.809,1.680)%
  --(9.821,1.680)--(9.833,1.680)--(9.845,1.680)--(9.857,1.680)--(9.870,1.680)--(9.882,1.680)%
  --(9.894,1.680)--(9.906,1.680)--(9.918,1.680)--(9.930,1.680)--(9.942,1.680)--(9.954,1.680)%
  --(9.967,1.680)--(9.979,1.680)--(9.991,1.680)--(10.003,1.680)--(10.015,1.680)--(10.027,1.680)%
  --(10.039,1.680)--(10.051,1.680)--(10.064,1.680)--(10.076,1.680)--(10.088,1.680)--(10.100,1.680)%
  --(10.112,1.680)--(10.124,1.680)--(10.136,1.680)--(10.148,1.680)--(10.161,1.680)--(10.173,1.680)%
  --(10.185,1.680)--(10.197,1.680)--(10.209,1.680)--(10.221,1.680)--(10.233,1.680)--(10.245,1.680)%
  --(10.258,1.680)--(10.270,1.680)--(10.282,1.680)--(10.294,1.680)--(10.306,1.680)--(10.318,1.680)%
  --(10.330,1.680)--(10.342,1.680)--(10.355,1.680)--(10.367,1.680)--(10.379,1.680)--(10.391,1.680)%
  --(10.403,1.680)--(10.415,1.680)--(10.427,1.680)--(10.440,1.680)--(10.452,1.680)--(10.464,1.680)%
  --(10.476,1.680)--(10.488,1.680)--(10.500,1.680)--(10.512,1.680)--(10.524,1.680)--(10.537,1.680)%
  --(10.549,1.680)--(10.561,1.680)--(10.573,1.680)--(10.585,1.680)--(10.597,1.680)--(10.609,1.680)%
  --(10.621,1.680)--(10.634,1.680)--(10.646,1.680)--(10.658,1.680)--(10.670,1.680)--(10.682,1.680)%
  --(10.694,1.680)--(10.706,1.680)--(10.718,1.680)--(10.731,1.680)--(10.743,1.680)--(10.755,1.680)%
  --(10.767,1.680)--(10.779,1.680)--(10.791,1.680)--(10.803,1.680)--(10.815,1.680)--(10.828,1.680)%
  --(10.840,1.680)--(10.852,1.680)--(10.864,1.680)--(10.876,1.680)--(10.888,1.680)--(10.900,1.680)%
  --(10.912,1.680)--(10.925,1.680)--(10.937,1.680)--(10.949,1.680)--(10.961,1.680)--(10.973,1.680)%
  --(10.985,1.680)--(10.997,1.680)--(11.009,1.680)--(11.022,1.680)--(11.034,1.680)--(11.046,1.680)%
  --(11.058,1.680)--(11.070,1.680)--(11.082,1.680)--(11.094,1.680)--(11.106,1.680)--(11.119,1.680)%
  --(11.131,1.680)--(11.143,1.680)--(11.155,1.680)--(11.167,1.680)--(11.179,1.680)--(11.191,1.680)%
  --(11.204,1.680)--(11.216,1.680)--(11.228,1.680)--(11.240,1.680)--(11.252,1.680)--(11.264,1.680)%
  --(11.276,1.680)--(11.288,1.680)--(11.301,1.680)--(11.313,1.680)--(11.325,1.680)--(11.337,1.680)%
  --(11.349,1.680)--(11.361,1.680)--(11.373,1.680)--(11.385,1.680)--(11.398,1.680)--(11.410,1.680)%
  --(11.422,1.680)--(11.434,1.680)--(11.446,1.680)--(11.458,1.680)--(11.470,1.680)--(11.482,1.680)%
  --(11.495,1.680)--(11.507,1.680)--(11.519,1.680)--(11.531,1.680)--(11.543,1.680)--(11.555,1.680)%
  --(11.567,1.680)--(11.579,1.680)--(11.592,1.680)--(11.604,1.680)--(11.616,1.680)--(11.628,1.680)%
  --(11.640,1.680)--(11.652,1.680)--(11.664,1.680)--(11.676,1.680)--(11.689,1.680)--(11.701,1.680)%
  --(11.713,1.680)--(11.725,1.680)--(11.737,1.680)--(11.749,1.680)--(11.761,1.680)--(11.773,1.680)%
  --(11.786,1.680)--(11.798,1.680)--(11.810,1.680)--(11.822,1.680)--(11.834,1.680)--(11.846,1.680)%
  --(11.858,1.680)--(11.870,1.680)--(11.883,1.680)--(11.895,1.680)--(11.907,1.680)--(11.919,1.680)%
  --(11.931,1.680)--(11.943,1.680)--(11.955,1.680)--(11.968,1.680)--(11.980,1.680)--(11.992,1.680)%
  --(12.004,1.680)--(12.016,1.680)--(12.028,1.680)--(12.040,1.680)--(12.052,1.680)--(12.065,1.680)%
  --(12.077,1.680)--(12.089,1.680)--(12.101,1.680)--(12.113,1.680)--(12.125,1.680)--(12.137,1.680)%
  --(12.149,1.680)--(12.162,1.680)--(12.174,1.680)--(12.186,1.680)--(12.198,1.680)--(12.210,1.680)%
  --(12.222,1.680)--(12.234,1.680)--(12.246,1.680)--(12.259,1.680)--(12.271,1.680)--(12.283,1.680)%
  --(12.295,1.680)--(12.307,1.680)--(12.319,1.680)--(12.331,1.680)--(12.343,1.680)--(12.356,1.680)%
  --(12.368,1.680)--(12.380,1.680)--(12.392,1.680)--(12.404,1.680)--(12.416,1.680)--(12.428,1.680)%
  --(12.440,1.680)--(12.453,1.680)--(12.465,1.680)--(12.477,1.680)--(12.489,1.680)--(12.501,1.680)%
  --(12.513,1.680)--(12.525,1.680)--(12.537,1.680)--(12.550,1.680)--(12.562,1.680)--(12.574,1.680)%
  --(12.586,1.680)--(12.598,1.680)--(12.610,1.680)--(12.622,1.680)--(12.634,1.680)--(12.647,1.680)%
  --(12.659,1.680)--(12.671,1.680)--(12.683,1.680)--(12.695,1.680)--(12.707,1.680)--(12.719,1.680)%
  --(12.732,1.680)--(12.744,1.680)--(12.756,1.680)--(12.768,1.680)--(12.780,1.680)--(12.792,1.680)%
  --(12.804,1.680)--(12.816,1.680)--(12.829,1.680)--(12.841,1.680)--(12.853,1.680)--(12.865,1.680)%
  --(12.877,1.680)--(12.889,1.680)--(12.901,1.680)--(12.913,1.680)--(12.926,1.680)--(12.938,1.680)%
  --(12.950,1.680)--(12.962,1.680)--(12.974,1.680)--(12.986,1.680)--(12.998,1.680)--(13.010,1.680)%
  --(13.023,1.680)--(13.035,1.680)--(13.047,1.680)--(13.059,1.680)--(13.071,1.680)--(13.083,1.680)%
  --(13.095,1.680)--(13.107,1.680)--(13.120,1.680)--(13.132,1.680)--(13.144,1.680)--(13.156,1.680)%
  --(13.168,1.680)--(13.180,1.680)--(13.192,1.680)--(13.204,1.680)--(13.217,1.680)--(13.229,1.680)%
  --(13.241,1.680)--(13.253,1.680)--(13.265,1.680)--(13.277,1.680)--(13.289,1.680)--(13.301,1.680)%
  --(13.314,1.680)--(13.326,1.680)--(13.338,1.680)--(13.350,1.680)--(13.362,1.680)--(13.374,1.680)%
  --(13.386,1.680)--(13.398,1.680)--(13.411,1.680)--(13.423,1.680)--(13.435,1.680)--(13.447,1.680);
\gpcolor{color=gp lt color border}
\draw[gp path] (1.320,8.631)--(1.320,0.985)--(13.447,0.985)--(13.447,8.631)--cycle;
%% coordinates of the plot area
\gpdefrectangularnode{gp plot 1}{\pgfpoint{1.320cm}{0.985cm}}{\pgfpoint{13.447cm}{8.631cm}}
\end{tikzpicture}
%% gnuplot variables

	\caption{Gráfico do potencial grand canônico por unidade de volume obtido através da Equação~\eqref{Eq:potencial_termodinamico}.}
	\label{Fig:thermodynamic_potential_graph}
\end{figure*}

\begin{figure*}
	\begin{tikzpicture}[gnuplot]
%% generated with GNUPLOT 5.0p2 (Lua 5.2; terminal rev. 99, script rev. 100)
%% Fri Mar 18 15:48:26 2016
\path (0.000,0.000) rectangle (14.000,9.000);
\gpcolor{color=gp lt color border}
\gpsetlinetype{gp lt border}
\gpsetdashtype{gp dt solid}
\gpsetlinewidth{1.00}
\draw[gp path] (1.320,1.680)--(1.500,1.680);
\draw[gp path] (13.447,1.680)--(13.267,1.680);
\node[gp node right] at (1.136,1.680) {$0$};
\draw[gp path] (1.320,3.070)--(1.500,3.070);
\draw[gp path] (13.447,3.070)--(13.267,3.070);
\node[gp node right] at (1.136,3.070) {$100$};
\draw[gp path] (1.320,4.460)--(1.500,4.460);
\draw[gp path] (13.447,4.460)--(13.267,4.460);
\node[gp node right] at (1.136,4.460) {$200$};
\draw[gp path] (1.320,5.851)--(1.500,5.851);
\draw[gp path] (13.447,5.851)--(13.267,5.851);
\node[gp node right] at (1.136,5.851) {$300$};
\draw[gp path] (1.320,7.241)--(1.500,7.241);
\draw[gp path] (13.447,7.241)--(13.267,7.241);
\node[gp node right] at (1.136,7.241) {$400$};
\draw[gp path] (1.320,8.631)--(1.500,8.631);
\draw[gp path] (13.447,8.631)--(13.267,8.631);
\node[gp node right] at (1.136,8.631) {$500$};
\draw[gp path] (1.320,0.985)--(1.320,1.165);
\draw[gp path] (1.320,8.631)--(1.320,8.451);
\node[gp node center] at (1.320,0.677) {$0$};
\draw[gp path] (3.745,0.985)--(3.745,1.165);
\draw[gp path] (3.745,8.631)--(3.745,8.451);
\node[gp node center] at (3.745,0.677) {$100$};
\draw[gp path] (6.171,0.985)--(6.171,1.165);
\draw[gp path] (6.171,8.631)--(6.171,8.451);
\node[gp node center] at (6.171,0.677) {$200$};
\draw[gp path] (8.596,0.985)--(8.596,1.165);
\draw[gp path] (8.596,8.631)--(8.596,8.451);
\node[gp node center] at (8.596,0.677) {$300$};
\draw[gp path] (11.022,0.985)--(11.022,1.165);
\draw[gp path] (11.022,8.631)--(11.022,8.451);
\node[gp node center] at (11.022,0.677) {$400$};
\draw[gp path] (13.447,0.985)--(13.447,1.165);
\draw[gp path] (13.447,8.631)--(13.447,8.451);
\node[gp node center] at (13.447,0.677) {$500$};
\draw[gp path] (1.320,8.631)--(1.320,0.985)--(13.447,0.985)--(13.447,8.631)--cycle;
\node[gp node center,rotate=-270] at (0.246,4.808) {$p_F$};
\node[gp node center] at (7.383,0.215) {$\mu_R$};
\node[gp node left] at (2.788,8.297) {$p_F = \sqrt{\mu_R^2 - m^2}\theta(\mu_R^2 - m^2), m = 100$};
\gpcolor{rgb color={0.580,0.000,0.827}}
\gpsetlinewidth{3.00}
\draw[gp path] (1.688,8.297)--(2.604,8.297);
\draw[gp path] (1.320,1.680)--(1.332,1.680)--(1.344,1.680)--(1.356,1.680)--(1.369,1.680)%
  --(1.381,1.680)--(1.393,1.680)--(1.405,1.680)--(1.417,1.680)--(1.429,1.680)--(1.441,1.680)%
  --(1.453,1.680)--(1.466,1.680)--(1.478,1.680)--(1.490,1.680)--(1.502,1.680)--(1.514,1.680)%
  --(1.526,1.680)--(1.538,1.680)--(1.550,1.680)--(1.563,1.680)--(1.575,1.680)--(1.587,1.680)%
  --(1.599,1.680)--(1.611,1.680)--(1.623,1.680)--(1.635,1.680)--(1.647,1.680)--(1.660,1.680)%
  --(1.672,1.680)--(1.684,1.680)--(1.696,1.680)--(1.708,1.680)--(1.720,1.680)--(1.732,1.680)%
  --(1.744,1.680)--(1.757,1.680)--(1.769,1.680)--(1.781,1.680)--(1.793,1.680)--(1.805,1.680)%
  --(1.817,1.680)--(1.829,1.680)--(1.841,1.680)--(1.854,1.680)--(1.866,1.680)--(1.878,1.680)%
  --(1.890,1.680)--(1.902,1.680)--(1.914,1.680)--(1.926,1.680)--(1.938,1.680)--(1.951,1.680)%
  --(1.963,1.680)--(1.975,1.680)--(1.987,1.680)--(1.999,1.680)--(2.011,1.680)--(2.023,1.680)%
  --(2.035,1.680)--(2.048,1.680)--(2.060,1.680)--(2.072,1.680)--(2.084,1.680)--(2.096,1.680)%
  --(2.108,1.680)--(2.120,1.680)--(2.133,1.680)--(2.145,1.680)--(2.157,1.680)--(2.169,1.680)%
  --(2.181,1.680)--(2.193,1.680)--(2.205,1.680)--(2.217,1.680)--(2.230,1.680)--(2.242,1.680)%
  --(2.254,1.680)--(2.266,1.680)--(2.278,1.680)--(2.290,1.680)--(2.302,1.680)--(2.314,1.680)%
  --(2.327,1.680)--(2.339,1.680)--(2.351,1.680)--(2.363,1.680)--(2.375,1.680)--(2.387,1.680)%
  --(2.399,1.680)--(2.411,1.680)--(2.424,1.680)--(2.436,1.680)--(2.448,1.680)--(2.460,1.680)%
  --(2.472,1.680)--(2.484,1.680)--(2.496,1.680)--(2.508,1.680)--(2.521,1.680)--(2.533,1.680)%
  --(2.545,1.680)--(2.557,1.680)--(2.569,1.680)--(2.581,1.680)--(2.593,1.680)--(2.605,1.680)%
  --(2.618,1.680)--(2.630,1.680)--(2.642,1.680)--(2.654,1.680)--(2.666,1.680)--(2.678,1.680)%
  --(2.690,1.680)--(2.702,1.680)--(2.715,1.680)--(2.727,1.680)--(2.739,1.680)--(2.751,1.680)%
  --(2.763,1.680)--(2.775,1.680)--(2.787,1.680)--(2.799,1.680)--(2.812,1.680)--(2.824,1.680)%
  --(2.836,1.680)--(2.848,1.680)--(2.860,1.680)--(2.872,1.680)--(2.884,1.680)--(2.897,1.680)%
  --(2.909,1.680)--(2.921,1.680)--(2.933,1.680)--(2.945,1.680)--(2.957,1.680)--(2.969,1.680)%
  --(2.981,1.680)--(2.994,1.680)--(3.006,1.680)--(3.018,1.680)--(3.030,1.680)--(3.042,1.680)%
  --(3.054,1.680)--(3.066,1.680)--(3.078,1.680)--(3.091,1.680)--(3.103,1.680)--(3.115,1.680)%
  --(3.127,1.680)--(3.139,1.680)--(3.151,1.680)--(3.163,1.680)--(3.175,1.680)--(3.188,1.680)%
  --(3.200,1.680)--(3.212,1.680)--(3.224,1.680)--(3.236,1.680)--(3.248,1.680)--(3.260,1.680)%
  --(3.272,1.680)--(3.285,1.680)--(3.297,1.680)--(3.309,1.680)--(3.321,1.680)--(3.333,1.680)%
  --(3.345,1.680)--(3.357,1.680)--(3.369,1.680)--(3.382,1.680)--(3.394,1.680)--(3.406,1.680)%
  --(3.418,1.680)--(3.430,1.680)--(3.442,1.680)--(3.454,1.680)--(3.466,1.680)--(3.479,1.680)%
  --(3.491,1.680)--(3.503,1.680)--(3.515,1.680)--(3.527,1.680)--(3.539,1.680)--(3.551,1.680)%
  --(3.563,1.680)--(3.576,1.680)--(3.588,1.680)--(3.600,1.680)--(3.612,1.680)--(3.624,1.680)%
  --(3.636,1.680)--(3.648,1.680)--(3.661,1.680)--(3.673,1.680)--(3.685,1.680)--(3.697,1.680)%
  --(3.709,1.680)--(3.721,1.680)--(3.733,1.680)--(3.745,1.680)--(3.758,1.819)--(3.770,1.877)%
  --(3.782,1.922)--(3.794,1.960)--(3.806,1.993)--(3.818,2.023)--(3.830,2.051)--(3.842,2.077)%
  --(3.855,2.102)--(3.867,2.125)--(3.879,2.147)--(3.891,2.169)--(3.903,2.189)--(3.915,2.209)%
  --(3.927,2.229)--(3.939,2.247)--(3.952,2.265)--(3.964,2.283)--(3.976,2.300)--(3.988,2.317)%
  --(4.000,2.334)--(4.012,2.350)--(4.024,2.366)--(4.036,2.381)--(4.049,2.397)--(4.061,2.412)%
  --(4.073,2.426)--(4.085,2.441)--(4.097,2.455)--(4.109,2.470)--(4.121,2.484)--(4.133,2.497)%
  --(4.146,2.511)--(4.158,2.524)--(4.170,2.538)--(4.182,2.551)--(4.194,2.564)--(4.206,2.577)%
  --(4.218,2.590)--(4.230,2.602)--(4.243,2.615)--(4.255,2.627)--(4.267,2.639)--(4.279,2.652)%
  --(4.291,2.664)--(4.303,2.676)--(4.315,2.688)--(4.327,2.699)--(4.340,2.711)--(4.352,2.723)%
  --(4.364,2.734)--(4.376,2.746)--(4.388,2.757)--(4.400,2.768)--(4.412,2.780)--(4.425,2.791)%
  --(4.437,2.802)--(4.449,2.813)--(4.461,2.824)--(4.473,2.835)--(4.485,2.846)--(4.497,2.857)%
  --(4.509,2.867)--(4.522,2.878)--(4.534,2.889)--(4.546,2.899)--(4.558,2.910)--(4.570,2.920)%
  --(4.582,2.931)--(4.594,2.941)--(4.606,2.951)--(4.619,2.961)--(4.631,2.972)--(4.643,2.982)%
  --(4.655,2.992)--(4.667,3.002)--(4.679,3.012)--(4.691,3.022)--(4.703,3.032)--(4.716,3.042)%
  --(4.728,3.052)--(4.740,3.062)--(4.752,3.072)--(4.764,3.082)--(4.776,3.091)--(4.788,3.101)%
  --(4.800,3.111)--(4.813,3.121)--(4.825,3.130)--(4.837,3.140)--(4.849,3.149)--(4.861,3.159)%
  --(4.873,3.168)--(4.885,3.178)--(4.897,3.187)--(4.910,3.197)--(4.922,3.206)--(4.934,3.216)%
  --(4.946,3.225)--(4.958,3.234)--(4.970,3.244)--(4.982,3.253)--(4.994,3.262)--(5.007,3.271)%
  --(5.019,3.281)--(5.031,3.290)--(5.043,3.299)--(5.055,3.308)--(5.067,3.317)--(5.079,3.326)%
  --(5.091,3.336)--(5.104,3.345)--(5.116,3.354)--(5.128,3.363)--(5.140,3.372)--(5.152,3.381)%
  --(5.164,3.390)--(5.176,3.399)--(5.189,3.408)--(5.201,3.416)--(5.213,3.425)--(5.225,3.434)%
  --(5.237,3.443)--(5.249,3.452)--(5.261,3.461)--(5.273,3.470)--(5.286,3.478)--(5.298,3.487)%
  --(5.310,3.496)--(5.322,3.505)--(5.334,3.513)--(5.346,3.522)--(5.358,3.531)--(5.370,3.539)%
  --(5.383,3.548)--(5.395,3.557)--(5.407,3.565)--(5.419,3.574)--(5.431,3.583)--(5.443,3.591)%
  --(5.455,3.600)--(5.467,3.608)--(5.480,3.617)--(5.492,3.626)--(5.504,3.634)--(5.516,3.643)%
  --(5.528,3.651)--(5.540,3.660)--(5.552,3.668)--(5.564,3.677)--(5.577,3.685)--(5.589,3.694)%
  --(5.601,3.702)--(5.613,3.710)--(5.625,3.719)--(5.637,3.727)--(5.649,3.736)--(5.661,3.744)%
  --(5.674,3.752)--(5.686,3.761)--(5.698,3.769)--(5.710,3.777)--(5.722,3.786)--(5.734,3.794)%
  --(5.746,3.802)--(5.758,3.811)--(5.771,3.819)--(5.783,3.827)--(5.795,3.836)--(5.807,3.844)%
  --(5.819,3.852)--(5.831,3.860)--(5.843,3.869)--(5.855,3.877)--(5.868,3.885)--(5.880,3.893)%
  --(5.892,3.901)--(5.904,3.910)--(5.916,3.918)--(5.928,3.926)--(5.940,3.934)--(5.953,3.942)%
  --(5.965,3.950)--(5.977,3.959)--(5.989,3.967)--(6.001,3.975)--(6.013,3.983)--(6.025,3.991)%
  --(6.037,3.999)--(6.050,4.007)--(6.062,4.015)--(6.074,4.024)--(6.086,4.032)--(6.098,4.040)%
  --(6.110,4.048)--(6.122,4.056)--(6.134,4.064)--(6.147,4.072)--(6.159,4.080)--(6.171,4.088)%
  --(6.183,4.096)--(6.195,4.104)--(6.207,4.112)--(6.219,4.120)--(6.231,4.128)--(6.244,4.136)%
  --(6.256,4.144)--(6.268,4.152)--(6.280,4.160)--(6.292,4.168)--(6.304,4.176)--(6.316,4.184)%
  --(6.328,4.192)--(6.341,4.200)--(6.353,4.208)--(6.365,4.216)--(6.377,4.223)--(6.389,4.231)%
  --(6.401,4.239)--(6.413,4.247)--(6.425,4.255)--(6.438,4.263)--(6.450,4.271)--(6.462,4.279)%
  --(6.474,4.287)--(6.486,4.295)--(6.498,4.302)--(6.510,4.310)--(6.522,4.318)--(6.535,4.326)%
  --(6.547,4.334)--(6.559,4.342)--(6.571,4.350)--(6.583,4.357)--(6.595,4.365)--(6.607,4.373)%
  --(6.619,4.381)--(6.632,4.389)--(6.644,4.396)--(6.656,4.404)--(6.668,4.412)--(6.680,4.420)%
  --(6.692,4.428)--(6.704,4.435)--(6.717,4.443)--(6.729,4.451)--(6.741,4.459)--(6.753,4.467)%
  --(6.765,4.474)--(6.777,4.482)--(6.789,4.490)--(6.801,4.498)--(6.814,4.505)--(6.826,4.513)%
  --(6.838,4.521)--(6.850,4.529)--(6.862,4.536)--(6.874,4.544)--(6.886,4.552)--(6.898,4.559)%
  --(6.911,4.567)--(6.923,4.575)--(6.935,4.583)--(6.947,4.590)--(6.959,4.598)--(6.971,4.606)%
  --(6.983,4.613)--(6.995,4.621)--(7.008,4.629)--(7.020,4.636)--(7.032,4.644)--(7.044,4.652)%
  --(7.056,4.660)--(7.068,4.667)--(7.080,4.675)--(7.092,4.682)--(7.105,4.690)--(7.117,4.698)%
  --(7.129,4.705)--(7.141,4.713)--(7.153,4.721)--(7.165,4.728)--(7.177,4.736)--(7.189,4.744)%
  --(7.202,4.751)--(7.214,4.759)--(7.226,4.767)--(7.238,4.774)--(7.250,4.782)--(7.262,4.789)%
  --(7.274,4.797)--(7.286,4.805)--(7.299,4.812)--(7.311,4.820)--(7.323,4.827)--(7.335,4.835)%
  --(7.347,4.843)--(7.359,4.850)--(7.371,4.858)--(7.384,4.865)--(7.396,4.873)--(7.408,4.881)%
  --(7.420,4.888)--(7.432,4.896)--(7.444,4.903)--(7.456,4.911)--(7.468,4.918)--(7.481,4.926)%
  --(7.493,4.934)--(7.505,4.941)--(7.517,4.949)--(7.529,4.956)--(7.541,4.964)--(7.553,4.971)%
  --(7.565,4.979)--(7.578,4.986)--(7.590,4.994)--(7.602,5.001)--(7.614,5.009)--(7.626,5.017)%
  --(7.638,5.024)--(7.650,5.032)--(7.662,5.039)--(7.675,5.047)--(7.687,5.054)--(7.699,5.062)%
  --(7.711,5.069)--(7.723,5.077)--(7.735,5.084)--(7.747,5.092)--(7.759,5.099)--(7.772,5.107)%
  --(7.784,5.114)--(7.796,5.122)--(7.808,5.129)--(7.820,5.137)--(7.832,5.144)--(7.844,5.152)%
  --(7.856,5.159)--(7.869,5.167)--(7.881,5.174)--(7.893,5.182)--(7.905,5.189)--(7.917,5.197)%
  --(7.929,5.204)--(7.941,5.212)--(7.953,5.219)--(7.966,5.226)--(7.978,5.234)--(7.990,5.241)%
  --(8.002,5.249)--(8.014,5.256)--(8.026,5.264)--(8.038,5.271)--(8.050,5.279)--(8.063,5.286)%
  --(8.075,5.294)--(8.087,5.301)--(8.099,5.308)--(8.111,5.316)--(8.123,5.323)--(8.135,5.331)%
  --(8.148,5.338)--(8.160,5.346)--(8.172,5.353)--(8.184,5.361)--(8.196,5.368)--(8.208,5.375)%
  --(8.220,5.383)--(8.232,5.390)--(8.245,5.398)--(8.257,5.405)--(8.269,5.412)--(8.281,5.420)%
  --(8.293,5.427)--(8.305,5.435)--(8.317,5.442)--(8.329,5.450)--(8.342,5.457)--(8.354,5.464)%
  --(8.366,5.472)--(8.378,5.479)--(8.390,5.487)--(8.402,5.494)--(8.414,5.501)--(8.426,5.509)%
  --(8.439,5.516)--(8.451,5.524)--(8.463,5.531)--(8.475,5.538)--(8.487,5.546)--(8.499,5.553)%
  --(8.511,5.560)--(8.523,5.568)--(8.536,5.575)--(8.548,5.583)--(8.560,5.590)--(8.572,5.597)%
  --(8.584,5.605)--(8.596,5.612)--(8.608,5.619)--(8.620,5.627)--(8.633,5.634)--(8.645,5.642)%
  --(8.657,5.649)--(8.669,5.656)--(8.681,5.664)--(8.693,5.671)--(8.705,5.678)--(8.717,5.686)%
  --(8.730,5.693)--(8.742,5.700)--(8.754,5.708)--(8.766,5.715)--(8.778,5.723)--(8.790,5.730)%
  --(8.802,5.737)--(8.814,5.745)--(8.827,5.752)--(8.839,5.759)--(8.851,5.767)--(8.863,5.774)%
  --(8.875,5.781)--(8.887,5.789)--(8.899,5.796)--(8.912,5.803)--(8.924,5.811)--(8.936,5.818)%
  --(8.948,5.825)--(8.960,5.833)--(8.972,5.840)--(8.984,5.847)--(8.996,5.855)--(9.009,5.862)%
  --(9.021,5.869)--(9.033,5.877)--(9.045,5.884)--(9.057,5.891)--(9.069,5.899)--(9.081,5.906)%
  --(9.093,5.913)--(9.106,5.921)--(9.118,5.928)--(9.130,5.935)--(9.142,5.942)--(9.154,5.950)%
  --(9.166,5.957)--(9.178,5.964)--(9.190,5.972)--(9.203,5.979)--(9.215,5.986)--(9.227,5.994)%
  --(9.239,6.001)--(9.251,6.008)--(9.263,6.016)--(9.275,6.023)--(9.287,6.030)--(9.300,6.037)%
  --(9.312,6.045)--(9.324,6.052)--(9.336,6.059)--(9.348,6.067)--(9.360,6.074)--(9.372,6.081)%
  --(9.384,6.088)--(9.397,6.096)--(9.409,6.103)--(9.421,6.110)--(9.433,6.118)--(9.445,6.125)%
  --(9.457,6.132)--(9.469,6.139)--(9.481,6.147)--(9.494,6.154)--(9.506,6.161)--(9.518,6.169)%
  --(9.530,6.176)--(9.542,6.183)--(9.554,6.190)--(9.566,6.198)--(9.578,6.205)--(9.591,6.212)%
  --(9.603,6.219)--(9.615,6.227)--(9.627,6.234)--(9.639,6.241)--(9.651,6.249)--(9.663,6.256)%
  --(9.676,6.263)--(9.688,6.270)--(9.700,6.278)--(9.712,6.285)--(9.724,6.292)--(9.736,6.299)%
  --(9.748,6.307)--(9.760,6.314)--(9.773,6.321)--(9.785,6.328)--(9.797,6.336)--(9.809,6.343)%
  --(9.821,6.350)--(9.833,6.357)--(9.845,6.365)--(9.857,6.372)--(9.870,6.379)--(9.882,6.386)%
  --(9.894,6.394)--(9.906,6.401)--(9.918,6.408)--(9.930,6.415)--(9.942,6.423)--(9.954,6.430)%
  --(9.967,6.437)--(9.979,6.444)--(9.991,6.452)--(10.003,6.459)--(10.015,6.466)--(10.027,6.473)%
  --(10.039,6.481)--(10.051,6.488)--(10.064,6.495)--(10.076,6.502)--(10.088,6.509)--(10.100,6.517)%
  --(10.112,6.524)--(10.124,6.531)--(10.136,6.538)--(10.148,6.546)--(10.161,6.553)--(10.173,6.560)%
  --(10.185,6.567)--(10.197,6.575)--(10.209,6.582)--(10.221,6.589)--(10.233,6.596)--(10.245,6.603)%
  --(10.258,6.611)--(10.270,6.618)--(10.282,6.625)--(10.294,6.632)--(10.306,6.640)--(10.318,6.647)%
  --(10.330,6.654)--(10.342,6.661)--(10.355,6.668)--(10.367,6.676)--(10.379,6.683)--(10.391,6.690)%
  --(10.403,6.697)--(10.415,6.704)--(10.427,6.712)--(10.440,6.719)--(10.452,6.726)--(10.464,6.733)%
  --(10.476,6.741)--(10.488,6.748)--(10.500,6.755)--(10.512,6.762)--(10.524,6.769)--(10.537,6.777)%
  --(10.549,6.784)--(10.561,6.791)--(10.573,6.798)--(10.585,6.805)--(10.597,6.813)--(10.609,6.820)%
  --(10.621,6.827)--(10.634,6.834)--(10.646,6.841)--(10.658,6.849)--(10.670,6.856)--(10.682,6.863)%
  --(10.694,6.870)--(10.706,6.877)--(10.718,6.885)--(10.731,6.892)--(10.743,6.899)--(10.755,6.906)%
  --(10.767,6.913)--(10.779,6.921)--(10.791,6.928)--(10.803,6.935)--(10.815,6.942)--(10.828,6.949)%
  --(10.840,6.956)--(10.852,6.964)--(10.864,6.971)--(10.876,6.978)--(10.888,6.985)--(10.900,6.992)%
  --(10.912,7.000)--(10.925,7.007)--(10.937,7.014)--(10.949,7.021)--(10.961,7.028)--(10.973,7.036)%
  --(10.985,7.043)--(10.997,7.050)--(11.009,7.057)--(11.022,7.064)--(11.034,7.071)--(11.046,7.079)%
  --(11.058,7.086)--(11.070,7.093)--(11.082,7.100)--(11.094,7.107)--(11.106,7.114)--(11.119,7.122)%
  --(11.131,7.129)--(11.143,7.136)--(11.155,7.143)--(11.167,7.150)--(11.179,7.158)--(11.191,7.165)%
  --(11.204,7.172)--(11.216,7.179)--(11.228,7.186)--(11.240,7.193)--(11.252,7.201)--(11.264,7.208)%
  --(11.276,7.215)--(11.288,7.222)--(11.301,7.229)--(11.313,7.236)--(11.325,7.244)--(11.337,7.251)%
  --(11.349,7.258)--(11.361,7.265)--(11.373,7.272)--(11.385,7.279)--(11.398,7.287)--(11.410,7.294)%
  --(11.422,7.301)--(11.434,7.308)--(11.446,7.315)--(11.458,7.322)--(11.470,7.329)--(11.482,7.337)%
  --(11.495,7.344)--(11.507,7.351)--(11.519,7.358)--(11.531,7.365)--(11.543,7.372)--(11.555,7.380)%
  --(11.567,7.387)--(11.579,7.394)--(11.592,7.401)--(11.604,7.408)--(11.616,7.415)--(11.628,7.422)%
  --(11.640,7.430)--(11.652,7.437)--(11.664,7.444)--(11.676,7.451)--(11.689,7.458)--(11.701,7.465)%
  --(11.713,7.473)--(11.725,7.480)--(11.737,7.487)--(11.749,7.494)--(11.761,7.501)--(11.773,7.508)%
  --(11.786,7.515)--(11.798,7.523)--(11.810,7.530)--(11.822,7.537)--(11.834,7.544)--(11.846,7.551)%
  --(11.858,7.558)--(11.870,7.565)--(11.883,7.573)--(11.895,7.580)--(11.907,7.587)--(11.919,7.594)%
  --(11.931,7.601)--(11.943,7.608)--(11.955,7.615)--(11.968,7.623)--(11.980,7.630)--(11.992,7.637)%
  --(12.004,7.644)--(12.016,7.651)--(12.028,7.658)--(12.040,7.665)--(12.052,7.673)--(12.065,7.680)%
  --(12.077,7.687)--(12.089,7.694)--(12.101,7.701)--(12.113,7.708)--(12.125,7.715)--(12.137,7.722)%
  --(12.149,7.730)--(12.162,7.737)--(12.174,7.744)--(12.186,7.751)--(12.198,7.758)--(12.210,7.765)%
  --(12.222,7.772)--(12.234,7.779)--(12.246,7.787)--(12.259,7.794)--(12.271,7.801)--(12.283,7.808)%
  --(12.295,7.815)--(12.307,7.822)--(12.319,7.829)--(12.331,7.837)--(12.343,7.844)--(12.356,7.851)%
  --(12.368,7.858)--(12.380,7.865)--(12.392,7.872)--(12.404,7.879)--(12.416,7.886)--(12.428,7.894)%
  --(12.440,7.901)--(12.453,7.908)--(12.465,7.915)--(12.477,7.922)--(12.489,7.929)--(12.501,7.936)%
  --(12.513,7.943)--(12.525,7.950)--(12.537,7.958)--(12.550,7.965)--(12.562,7.972)--(12.574,7.979)%
  --(12.586,7.986)--(12.598,7.993)--(12.610,8.000)--(12.622,8.007)--(12.634,8.015)--(12.647,8.022)%
  --(12.659,8.029)--(12.671,8.036)--(12.683,8.043)--(12.695,8.050)--(12.707,8.057)--(12.719,8.064)%
  --(12.732,8.071)--(12.744,8.079)--(12.756,8.086)--(12.768,8.093)--(12.780,8.100)--(12.792,8.107)%
  --(12.804,8.114)--(12.816,8.121)--(12.829,8.128)--(12.841,8.135)--(12.853,8.143)--(12.865,8.150)%
  --(12.877,8.157)--(12.889,8.164)--(12.901,8.171)--(12.913,8.178)--(12.926,8.185)--(12.938,8.192)%
  --(12.950,8.199)--(12.962,8.207)--(12.974,8.214)--(12.986,8.221)--(12.998,8.228)--(13.010,8.235)%
  --(13.023,8.242)--(13.035,8.249)--(13.047,8.256)--(13.059,8.263)--(13.071,8.270)--(13.083,8.278)%
  --(13.095,8.285)--(13.107,8.292)--(13.120,8.299)--(13.132,8.306)--(13.144,8.313)--(13.156,8.320)%
  --(13.168,8.327)--(13.180,8.334)--(13.192,8.342)--(13.204,8.349)--(13.217,8.356)--(13.229,8.363)%
  --(13.241,8.370)--(13.253,8.377)--(13.265,8.384)--(13.277,8.391)--(13.289,8.398)--(13.301,8.405)%
  --(13.314,8.413)--(13.326,8.420)--(13.338,8.427)--(13.350,8.434)--(13.362,8.441)--(13.374,8.448)%
  --(13.386,8.455)--(13.398,8.462)--(13.411,8.469)--(13.423,8.476)--(13.435,8.483)--(13.447,8.491);
\gpcolor{rgb color={0.000,0.620,0.451}}
\gpsetlinewidth{1.00}
\draw[gp path] (1.320,1.680)--(1.332,1.680)--(1.344,1.680)--(1.356,1.680)--(1.369,1.680)%
  --(1.381,1.680)--(1.393,1.680)--(1.405,1.680)--(1.417,1.680)--(1.429,1.680)--(1.441,1.680)%
  --(1.453,1.680)--(1.466,1.680)--(1.478,1.680)--(1.490,1.680)--(1.502,1.680)--(1.514,1.680)%
  --(1.526,1.680)--(1.538,1.680)--(1.550,1.680)--(1.563,1.680)--(1.575,1.680)--(1.587,1.680)%
  --(1.599,1.680)--(1.611,1.680)--(1.623,1.680)--(1.635,1.680)--(1.647,1.680)--(1.660,1.680)%
  --(1.672,1.680)--(1.684,1.680)--(1.696,1.680)--(1.708,1.680)--(1.720,1.680)--(1.732,1.680)%
  --(1.744,1.680)--(1.757,1.680)--(1.769,1.680)--(1.781,1.680)--(1.793,1.680)--(1.805,1.680)%
  --(1.817,1.680)--(1.829,1.680)--(1.841,1.680)--(1.854,1.680)--(1.866,1.680)--(1.878,1.680)%
  --(1.890,1.680)--(1.902,1.680)--(1.914,1.680)--(1.926,1.680)--(1.938,1.680)--(1.951,1.680)%
  --(1.963,1.680)--(1.975,1.680)--(1.987,1.680)--(1.999,1.680)--(2.011,1.680)--(2.023,1.680)%
  --(2.035,1.680)--(2.048,1.680)--(2.060,1.680)--(2.072,1.680)--(2.084,1.680)--(2.096,1.680)%
  --(2.108,1.680)--(2.120,1.680)--(2.133,1.680)--(2.145,1.680)--(2.157,1.680)--(2.169,1.680)%
  --(2.181,1.680)--(2.193,1.680)--(2.205,1.680)--(2.217,1.680)--(2.230,1.680)--(2.242,1.680)%
  --(2.254,1.680)--(2.266,1.680)--(2.278,1.680)--(2.290,1.680)--(2.302,1.680)--(2.314,1.680)%
  --(2.327,1.680)--(2.339,1.680)--(2.351,1.680)--(2.363,1.680)--(2.375,1.680)--(2.387,1.680)%
  --(2.399,1.680)--(2.411,1.680)--(2.424,1.680)--(2.436,1.680)--(2.448,1.680)--(2.460,1.680)%
  --(2.472,1.680)--(2.484,1.680)--(2.496,1.680)--(2.508,1.680)--(2.521,1.680)--(2.533,1.680)%
  --(2.545,1.680)--(2.557,1.680)--(2.569,1.680)--(2.581,1.680)--(2.593,1.680)--(2.605,1.680)%
  --(2.618,1.680)--(2.630,1.680)--(2.642,1.680)--(2.654,1.680)--(2.666,1.680)--(2.678,1.680)%
  --(2.690,1.680)--(2.702,1.680)--(2.715,1.680)--(2.727,1.680)--(2.739,1.680)--(2.751,1.680)%
  --(2.763,1.680)--(2.775,1.680)--(2.787,1.680)--(2.799,1.680)--(2.812,1.680)--(2.824,1.680)%
  --(2.836,1.680)--(2.848,1.680)--(2.860,1.680)--(2.872,1.680)--(2.884,1.680)--(2.897,1.680)%
  --(2.909,1.680)--(2.921,1.680)--(2.933,1.680)--(2.945,1.680)--(2.957,1.680)--(2.969,1.680)%
  --(2.981,1.680)--(2.994,1.680)--(3.006,1.680)--(3.018,1.680)--(3.030,1.680)--(3.042,1.680)%
  --(3.054,1.680)--(3.066,1.680)--(3.078,1.680)--(3.091,1.680)--(3.103,1.680)--(3.115,1.680)%
  --(3.127,1.680)--(3.139,1.680)--(3.151,1.680)--(3.163,1.680)--(3.175,1.680)--(3.188,1.680)%
  --(3.200,1.680)--(3.212,1.680)--(3.224,1.680)--(3.236,1.680)--(3.248,1.680)--(3.260,1.680)%
  --(3.272,1.680)--(3.285,1.680)--(3.297,1.680)--(3.309,1.680)--(3.321,1.680)--(3.333,1.680)%
  --(3.345,1.680)--(3.357,1.680)--(3.369,1.680)--(3.382,1.680)--(3.394,1.680)--(3.406,1.680)%
  --(3.418,1.680)--(3.430,1.680)--(3.442,1.680)--(3.454,1.680)--(3.466,1.680)--(3.479,1.680)%
  --(3.491,1.680)--(3.503,1.680)--(3.515,1.680)--(3.527,1.680)--(3.539,1.680)--(3.551,1.680)%
  --(3.563,1.680)--(3.576,1.680)--(3.588,1.680)--(3.600,1.680)--(3.612,1.680)--(3.624,1.680)%
  --(3.636,1.680)--(3.648,1.680)--(3.661,1.680)--(3.673,1.680)--(3.685,1.680)--(3.697,1.680)%
  --(3.709,1.680)--(3.721,1.680)--(3.733,1.680)--(3.745,1.680)--(3.758,1.680)--(3.770,1.680)%
  --(3.782,1.680)--(3.794,1.680)--(3.806,1.680)--(3.818,1.680)--(3.830,1.680)--(3.842,1.680)%
  --(3.855,1.680)--(3.867,1.680)--(3.879,1.680)--(3.891,1.680)--(3.903,1.680)--(3.915,1.680)%
  --(3.927,1.680)--(3.939,1.680)--(3.952,1.680)--(3.964,1.680)--(3.976,1.680)--(3.988,1.680)%
  --(4.000,1.680)--(4.012,1.680)--(4.024,1.680)--(4.036,1.680)--(4.049,1.680)--(4.061,1.680)%
  --(4.073,1.680)--(4.085,1.680)--(4.097,1.680)--(4.109,1.680)--(4.121,1.680)--(4.133,1.680)%
  --(4.146,1.680)--(4.158,1.680)--(4.170,1.680)--(4.182,1.680)--(4.194,1.680)--(4.206,1.680)%
  --(4.218,1.680)--(4.230,1.680)--(4.243,1.680)--(4.255,1.680)--(4.267,1.680)--(4.279,1.680)%
  --(4.291,1.680)--(4.303,1.680)--(4.315,1.680)--(4.327,1.680)--(4.340,1.680)--(4.352,1.680)%
  --(4.364,1.680)--(4.376,1.680)--(4.388,1.680)--(4.400,1.680)--(4.412,1.680)--(4.425,1.680)%
  --(4.437,1.680)--(4.449,1.680)--(4.461,1.680)--(4.473,1.680)--(4.485,1.680)--(4.497,1.680)%
  --(4.509,1.680)--(4.522,1.680)--(4.534,1.680)--(4.546,1.680)--(4.558,1.680)--(4.570,1.680)%
  --(4.582,1.680)--(4.594,1.680)--(4.606,1.680)--(4.619,1.680)--(4.631,1.680)--(4.643,1.680)%
  --(4.655,1.680)--(4.667,1.680)--(4.679,1.680)--(4.691,1.680)--(4.703,1.680)--(4.716,1.680)%
  --(4.728,1.680)--(4.740,1.680)--(4.752,1.680)--(4.764,1.680)--(4.776,1.680)--(4.788,1.680)%
  --(4.800,1.680)--(4.813,1.680)--(4.825,1.680)--(4.837,1.680)--(4.849,1.680)--(4.861,1.680)%
  --(4.873,1.680)--(4.885,1.680)--(4.897,1.680)--(4.910,1.680)--(4.922,1.680)--(4.934,1.680)%
  --(4.946,1.680)--(4.958,1.680)--(4.970,1.680)--(4.982,1.680)--(4.994,1.680)--(5.007,1.680)%
  --(5.019,1.680)--(5.031,1.680)--(5.043,1.680)--(5.055,1.680)--(5.067,1.680)--(5.079,1.680)%
  --(5.091,1.680)--(5.104,1.680)--(5.116,1.680)--(5.128,1.680)--(5.140,1.680)--(5.152,1.680)%
  --(5.164,1.680)--(5.176,1.680)--(5.189,1.680)--(5.201,1.680)--(5.213,1.680)--(5.225,1.680)%
  --(5.237,1.680)--(5.249,1.680)--(5.261,1.680)--(5.273,1.680)--(5.286,1.680)--(5.298,1.680)%
  --(5.310,1.680)--(5.322,1.680)--(5.334,1.680)--(5.346,1.680)--(5.358,1.680)--(5.370,1.680)%
  --(5.383,1.680)--(5.395,1.680)--(5.407,1.680)--(5.419,1.680)--(5.431,1.680)--(5.443,1.680)%
  --(5.455,1.680)--(5.467,1.680)--(5.480,1.680)--(5.492,1.680)--(5.504,1.680)--(5.516,1.680)%
  --(5.528,1.680)--(5.540,1.680)--(5.552,1.680)--(5.564,1.680)--(5.577,1.680)--(5.589,1.680)%
  --(5.601,1.680)--(5.613,1.680)--(5.625,1.680)--(5.637,1.680)--(5.649,1.680)--(5.661,1.680)%
  --(5.674,1.680)--(5.686,1.680)--(5.698,1.680)--(5.710,1.680)--(5.722,1.680)--(5.734,1.680)%
  --(5.746,1.680)--(5.758,1.680)--(5.771,1.680)--(5.783,1.680)--(5.795,1.680)--(5.807,1.680)%
  --(5.819,1.680)--(5.831,1.680)--(5.843,1.680)--(5.855,1.680)--(5.868,1.680)--(5.880,1.680)%
  --(5.892,1.680)--(5.904,1.680)--(5.916,1.680)--(5.928,1.680)--(5.940,1.680)--(5.953,1.680)%
  --(5.965,1.680)--(5.977,1.680)--(5.989,1.680)--(6.001,1.680)--(6.013,1.680)--(6.025,1.680)%
  --(6.037,1.680)--(6.050,1.680)--(6.062,1.680)--(6.074,1.680)--(6.086,1.680)--(6.098,1.680)%
  --(6.110,1.680)--(6.122,1.680)--(6.134,1.680)--(6.147,1.680)--(6.159,1.680)--(6.171,1.680)%
  --(6.183,1.680)--(6.195,1.680)--(6.207,1.680)--(6.219,1.680)--(6.231,1.680)--(6.244,1.680)%
  --(6.256,1.680)--(6.268,1.680)--(6.280,1.680)--(6.292,1.680)--(6.304,1.680)--(6.316,1.680)%
  --(6.328,1.680)--(6.341,1.680)--(6.353,1.680)--(6.365,1.680)--(6.377,1.680)--(6.389,1.680)%
  --(6.401,1.680)--(6.413,1.680)--(6.425,1.680)--(6.438,1.680)--(6.450,1.680)--(6.462,1.680)%
  --(6.474,1.680)--(6.486,1.680)--(6.498,1.680)--(6.510,1.680)--(6.522,1.680)--(6.535,1.680)%
  --(6.547,1.680)--(6.559,1.680)--(6.571,1.680)--(6.583,1.680)--(6.595,1.680)--(6.607,1.680)%
  --(6.619,1.680)--(6.632,1.680)--(6.644,1.680)--(6.656,1.680)--(6.668,1.680)--(6.680,1.680)%
  --(6.692,1.680)--(6.704,1.680)--(6.717,1.680)--(6.729,1.680)--(6.741,1.680)--(6.753,1.680)%
  --(6.765,1.680)--(6.777,1.680)--(6.789,1.680)--(6.801,1.680)--(6.814,1.680)--(6.826,1.680)%
  --(6.838,1.680)--(6.850,1.680)--(6.862,1.680)--(6.874,1.680)--(6.886,1.680)--(6.898,1.680)%
  --(6.911,1.680)--(6.923,1.680)--(6.935,1.680)--(6.947,1.680)--(6.959,1.680)--(6.971,1.680)%
  --(6.983,1.680)--(6.995,1.680)--(7.008,1.680)--(7.020,1.680)--(7.032,1.680)--(7.044,1.680)%
  --(7.056,1.680)--(7.068,1.680)--(7.080,1.680)--(7.092,1.680)--(7.105,1.680)--(7.117,1.680)%
  --(7.129,1.680)--(7.141,1.680)--(7.153,1.680)--(7.165,1.680)--(7.177,1.680)--(7.189,1.680)%
  --(7.202,1.680)--(7.214,1.680)--(7.226,1.680)--(7.238,1.680)--(7.250,1.680)--(7.262,1.680)%
  --(7.274,1.680)--(7.286,1.680)--(7.299,1.680)--(7.311,1.680)--(7.323,1.680)--(7.335,1.680)%
  --(7.347,1.680)--(7.359,1.680)--(7.371,1.680)--(7.384,1.680)--(7.396,1.680)--(7.408,1.680)%
  --(7.420,1.680)--(7.432,1.680)--(7.444,1.680)--(7.456,1.680)--(7.468,1.680)--(7.481,1.680)%
  --(7.493,1.680)--(7.505,1.680)--(7.517,1.680)--(7.529,1.680)--(7.541,1.680)--(7.553,1.680)%
  --(7.565,1.680)--(7.578,1.680)--(7.590,1.680)--(7.602,1.680)--(7.614,1.680)--(7.626,1.680)%
  --(7.638,1.680)--(7.650,1.680)--(7.662,1.680)--(7.675,1.680)--(7.687,1.680)--(7.699,1.680)%
  --(7.711,1.680)--(7.723,1.680)--(7.735,1.680)--(7.747,1.680)--(7.759,1.680)--(7.772,1.680)%
  --(7.784,1.680)--(7.796,1.680)--(7.808,1.680)--(7.820,1.680)--(7.832,1.680)--(7.844,1.680)%
  --(7.856,1.680)--(7.869,1.680)--(7.881,1.680)--(7.893,1.680)--(7.905,1.680)--(7.917,1.680)%
  --(7.929,1.680)--(7.941,1.680)--(7.953,1.680)--(7.966,1.680)--(7.978,1.680)--(7.990,1.680)%
  --(8.002,1.680)--(8.014,1.680)--(8.026,1.680)--(8.038,1.680)--(8.050,1.680)--(8.063,1.680)%
  --(8.075,1.680)--(8.087,1.680)--(8.099,1.680)--(8.111,1.680)--(8.123,1.680)--(8.135,1.680)%
  --(8.148,1.680)--(8.160,1.680)--(8.172,1.680)--(8.184,1.680)--(8.196,1.680)--(8.208,1.680)%
  --(8.220,1.680)--(8.232,1.680)--(8.245,1.680)--(8.257,1.680)--(8.269,1.680)--(8.281,1.680)%
  --(8.293,1.680)--(8.305,1.680)--(8.317,1.680)--(8.329,1.680)--(8.342,1.680)--(8.354,1.680)%
  --(8.366,1.680)--(8.378,1.680)--(8.390,1.680)--(8.402,1.680)--(8.414,1.680)--(8.426,1.680)%
  --(8.439,1.680)--(8.451,1.680)--(8.463,1.680)--(8.475,1.680)--(8.487,1.680)--(8.499,1.680)%
  --(8.511,1.680)--(8.523,1.680)--(8.536,1.680)--(8.548,1.680)--(8.560,1.680)--(8.572,1.680)%
  --(8.584,1.680)--(8.596,1.680)--(8.608,1.680)--(8.620,1.680)--(8.633,1.680)--(8.645,1.680)%
  --(8.657,1.680)--(8.669,1.680)--(8.681,1.680)--(8.693,1.680)--(8.705,1.680)--(8.717,1.680)%
  --(8.730,1.680)--(8.742,1.680)--(8.754,1.680)--(8.766,1.680)--(8.778,1.680)--(8.790,1.680)%
  --(8.802,1.680)--(8.814,1.680)--(8.827,1.680)--(8.839,1.680)--(8.851,1.680)--(8.863,1.680)%
  --(8.875,1.680)--(8.887,1.680)--(8.899,1.680)--(8.912,1.680)--(8.924,1.680)--(8.936,1.680)%
  --(8.948,1.680)--(8.960,1.680)--(8.972,1.680)--(8.984,1.680)--(8.996,1.680)--(9.009,1.680)%
  --(9.021,1.680)--(9.033,1.680)--(9.045,1.680)--(9.057,1.680)--(9.069,1.680)--(9.081,1.680)%
  --(9.093,1.680)--(9.106,1.680)--(9.118,1.680)--(9.130,1.680)--(9.142,1.680)--(9.154,1.680)%
  --(9.166,1.680)--(9.178,1.680)--(9.190,1.680)--(9.203,1.680)--(9.215,1.680)--(9.227,1.680)%
  --(9.239,1.680)--(9.251,1.680)--(9.263,1.680)--(9.275,1.680)--(9.287,1.680)--(9.300,1.680)%
  --(9.312,1.680)--(9.324,1.680)--(9.336,1.680)--(9.348,1.680)--(9.360,1.680)--(9.372,1.680)%
  --(9.384,1.680)--(9.397,1.680)--(9.409,1.680)--(9.421,1.680)--(9.433,1.680)--(9.445,1.680)%
  --(9.457,1.680)--(9.469,1.680)--(9.481,1.680)--(9.494,1.680)--(9.506,1.680)--(9.518,1.680)%
  --(9.530,1.680)--(9.542,1.680)--(9.554,1.680)--(9.566,1.680)--(9.578,1.680)--(9.591,1.680)%
  --(9.603,1.680)--(9.615,1.680)--(9.627,1.680)--(9.639,1.680)--(9.651,1.680)--(9.663,1.680)%
  --(9.676,1.680)--(9.688,1.680)--(9.700,1.680)--(9.712,1.680)--(9.724,1.680)--(9.736,1.680)%
  --(9.748,1.680)--(9.760,1.680)--(9.773,1.680)--(9.785,1.680)--(9.797,1.680)--(9.809,1.680)%
  --(9.821,1.680)--(9.833,1.680)--(9.845,1.680)--(9.857,1.680)--(9.870,1.680)--(9.882,1.680)%
  --(9.894,1.680)--(9.906,1.680)--(9.918,1.680)--(9.930,1.680)--(9.942,1.680)--(9.954,1.680)%
  --(9.967,1.680)--(9.979,1.680)--(9.991,1.680)--(10.003,1.680)--(10.015,1.680)--(10.027,1.680)%
  --(10.039,1.680)--(10.051,1.680)--(10.064,1.680)--(10.076,1.680)--(10.088,1.680)--(10.100,1.680)%
  --(10.112,1.680)--(10.124,1.680)--(10.136,1.680)--(10.148,1.680)--(10.161,1.680)--(10.173,1.680)%
  --(10.185,1.680)--(10.197,1.680)--(10.209,1.680)--(10.221,1.680)--(10.233,1.680)--(10.245,1.680)%
  --(10.258,1.680)--(10.270,1.680)--(10.282,1.680)--(10.294,1.680)--(10.306,1.680)--(10.318,1.680)%
  --(10.330,1.680)--(10.342,1.680)--(10.355,1.680)--(10.367,1.680)--(10.379,1.680)--(10.391,1.680)%
  --(10.403,1.680)--(10.415,1.680)--(10.427,1.680)--(10.440,1.680)--(10.452,1.680)--(10.464,1.680)%
  --(10.476,1.680)--(10.488,1.680)--(10.500,1.680)--(10.512,1.680)--(10.524,1.680)--(10.537,1.680)%
  --(10.549,1.680)--(10.561,1.680)--(10.573,1.680)--(10.585,1.680)--(10.597,1.680)--(10.609,1.680)%
  --(10.621,1.680)--(10.634,1.680)--(10.646,1.680)--(10.658,1.680)--(10.670,1.680)--(10.682,1.680)%
  --(10.694,1.680)--(10.706,1.680)--(10.718,1.680)--(10.731,1.680)--(10.743,1.680)--(10.755,1.680)%
  --(10.767,1.680)--(10.779,1.680)--(10.791,1.680)--(10.803,1.680)--(10.815,1.680)--(10.828,1.680)%
  --(10.840,1.680)--(10.852,1.680)--(10.864,1.680)--(10.876,1.680)--(10.888,1.680)--(10.900,1.680)%
  --(10.912,1.680)--(10.925,1.680)--(10.937,1.680)--(10.949,1.680)--(10.961,1.680)--(10.973,1.680)%
  --(10.985,1.680)--(10.997,1.680)--(11.009,1.680)--(11.022,1.680)--(11.034,1.680)--(11.046,1.680)%
  --(11.058,1.680)--(11.070,1.680)--(11.082,1.680)--(11.094,1.680)--(11.106,1.680)--(11.119,1.680)%
  --(11.131,1.680)--(11.143,1.680)--(11.155,1.680)--(11.167,1.680)--(11.179,1.680)--(11.191,1.680)%
  --(11.204,1.680)--(11.216,1.680)--(11.228,1.680)--(11.240,1.680)--(11.252,1.680)--(11.264,1.680)%
  --(11.276,1.680)--(11.288,1.680)--(11.301,1.680)--(11.313,1.680)--(11.325,1.680)--(11.337,1.680)%
  --(11.349,1.680)--(11.361,1.680)--(11.373,1.680)--(11.385,1.680)--(11.398,1.680)--(11.410,1.680)%
  --(11.422,1.680)--(11.434,1.680)--(11.446,1.680)--(11.458,1.680)--(11.470,1.680)--(11.482,1.680)%
  --(11.495,1.680)--(11.507,1.680)--(11.519,1.680)--(11.531,1.680)--(11.543,1.680)--(11.555,1.680)%
  --(11.567,1.680)--(11.579,1.680)--(11.592,1.680)--(11.604,1.680)--(11.616,1.680)--(11.628,1.680)%
  --(11.640,1.680)--(11.652,1.680)--(11.664,1.680)--(11.676,1.680)--(11.689,1.680)--(11.701,1.680)%
  --(11.713,1.680)--(11.725,1.680)--(11.737,1.680)--(11.749,1.680)--(11.761,1.680)--(11.773,1.680)%
  --(11.786,1.680)--(11.798,1.680)--(11.810,1.680)--(11.822,1.680)--(11.834,1.680)--(11.846,1.680)%
  --(11.858,1.680)--(11.870,1.680)--(11.883,1.680)--(11.895,1.680)--(11.907,1.680)--(11.919,1.680)%
  --(11.931,1.680)--(11.943,1.680)--(11.955,1.680)--(11.968,1.680)--(11.980,1.680)--(11.992,1.680)%
  --(12.004,1.680)--(12.016,1.680)--(12.028,1.680)--(12.040,1.680)--(12.052,1.680)--(12.065,1.680)%
  --(12.077,1.680)--(12.089,1.680)--(12.101,1.680)--(12.113,1.680)--(12.125,1.680)--(12.137,1.680)%
  --(12.149,1.680)--(12.162,1.680)--(12.174,1.680)--(12.186,1.680)--(12.198,1.680)--(12.210,1.680)%
  --(12.222,1.680)--(12.234,1.680)--(12.246,1.680)--(12.259,1.680)--(12.271,1.680)--(12.283,1.680)%
  --(12.295,1.680)--(12.307,1.680)--(12.319,1.680)--(12.331,1.680)--(12.343,1.680)--(12.356,1.680)%
  --(12.368,1.680)--(12.380,1.680)--(12.392,1.680)--(12.404,1.680)--(12.416,1.680)--(12.428,1.680)%
  --(12.440,1.680)--(12.453,1.680)--(12.465,1.680)--(12.477,1.680)--(12.489,1.680)--(12.501,1.680)%
  --(12.513,1.680)--(12.525,1.680)--(12.537,1.680)--(12.550,1.680)--(12.562,1.680)--(12.574,1.680)%
  --(12.586,1.680)--(12.598,1.680)--(12.610,1.680)--(12.622,1.680)--(12.634,1.680)--(12.647,1.680)%
  --(12.659,1.680)--(12.671,1.680)--(12.683,1.680)--(12.695,1.680)--(12.707,1.680)--(12.719,1.680)%
  --(12.732,1.680)--(12.744,1.680)--(12.756,1.680)--(12.768,1.680)--(12.780,1.680)--(12.792,1.680)%
  --(12.804,1.680)--(12.816,1.680)--(12.829,1.680)--(12.841,1.680)--(12.853,1.680)--(12.865,1.680)%
  --(12.877,1.680)--(12.889,1.680)--(12.901,1.680)--(12.913,1.680)--(12.926,1.680)--(12.938,1.680)%
  --(12.950,1.680)--(12.962,1.680)--(12.974,1.680)--(12.986,1.680)--(12.998,1.680)--(13.010,1.680)%
  --(13.023,1.680)--(13.035,1.680)--(13.047,1.680)--(13.059,1.680)--(13.071,1.680)--(13.083,1.680)%
  --(13.095,1.680)--(13.107,1.680)--(13.120,1.680)--(13.132,1.680)--(13.144,1.680)--(13.156,1.680)%
  --(13.168,1.680)--(13.180,1.680)--(13.192,1.680)--(13.204,1.680)--(13.217,1.680)--(13.229,1.680)%
  --(13.241,1.680)--(13.253,1.680)--(13.265,1.680)--(13.277,1.680)--(13.289,1.680)--(13.301,1.680)%
  --(13.314,1.680)--(13.326,1.680)--(13.338,1.680)--(13.350,1.680)--(13.362,1.680)--(13.374,1.680)%
  --(13.386,1.680)--(13.398,1.680)--(13.411,1.680)--(13.423,1.680)--(13.435,1.680)--(13.447,1.680);
\gpcolor{color=gp lt color border}
\draw[gp path] (1.320,8.631)--(1.320,0.985)--(13.447,0.985)--(13.447,8.631)--cycle;
%% coordinates of the plot area
\gpdefrectangularnode{gp plot 1}{\pgfpoint{1.320cm}{0.985cm}}{\pgfpoint{13.447cm}{8.631cm}}
\end{tikzpicture}
%% gnuplot variables

	\caption{Gráfico da pressão obtido através da Equação~\eqref{Eq:Pressao}. Devido ao fato de que estamos usando $\varepsilon_o = 0$ por enquanto, a escala vertical do gráfico está deslocada no sentido positivo.}
	\label{Fig:pressure_graph}
\end{figure*}

