\chapter{Fase de Hadrons: eNJL}

\section{Simetria quiral}

Ver discussão em Ref.\cite{Vogl}
% Vogl, U., and W. Weise. - Progress in Particle and Nuclear Physics 27 (1991): 195-272 - The Nambu and Jona-Lasinio model: its implications for hadrons and nuclei
\section{Termodinâmica}

Temos que $dS$ ou $dU$ -- deve ser $dU$, pois o potencial químico é a ``quantidade de energia ganha ao se inserir mais uma partícula no sistema'' -- tem um termo $\mu dN$. Vamos trabalhar com densidade bariônica, mas essa densidade é só o número de partículas $N$ dividido pelo volume (as outras variáveis também serão trabalhadas divididas pelo volume). Logo, temos $\mu d\rho$. Por isso, temos que $\mu = d\epsilon/d\rho$, onde $\epsilon$ é a densidade de energia $\epsilon = E/V$. 

Glendenning\cite{Glendenning}:
Degenerate ideal Fermi gas: ideal pois não tem interações entre as partículas, degenerado pois todos os estados até uma certa energia -- a energia de Fermi -- estão ocupados. Nesse caso, a soma sobre todos os estados ocupados (que são autoestados de momento, pois não há interação) deve se dar sobre o momento. Isso pode ser escrito como a integral
\begin{equation}
	\int_0^{k_f} \frac{d^3k}{(2\pi^3)}.
\end{equation}
%
(pelo que lembro, é um cálculo realizado em um octante, contando quantos estados existem entre $p$ e $p+dp$ levando-se em conta que são ondas estacionárias em uma caixa de lado $L$. Nesse caso $k$ (que está associado ao momento) é um inteiro vezes o comprimento de onda dividido por dois. Tentar achar isso Ref [63] do Glendenning.

O gás pode ser considerado degenerado se $T \ll E_F = \sqrt{k_F^2 + m^2}$.

A densidade é obtida simplesmente somando os estados ocupados. A energia é calculada somando a energia de cada estado ocupado. A pressão eu não sei:
\begin{align}
	\rho &= \\
	\epsilon &= \\
	p &=
\end{align}

In thermodynamics, chemical potential, also known as partial molar free energy, is a form of potential energy that can be absorbed or released during a chemical reaction. It may also change during a phase transition. The chemical potential of a species in a mixture can be defined as the slope of the free energy of the system with respect to a change in the number of moles of just that species. Thus, it is the partial derivative of the free energy with respect to the amount of the species, all other species' concentrations in the mixture remaining constant, and at constant temperature. When pressure is constant, chemical potential is the partial molar Gibbs free energy. At chemical equilibrium or in phase equilibrium the total sum of chemical potentials is zero, as the free energy is at a minimum. \url{https://en.wikipedia.org/wiki/Chemical_potential}



%%%%%%%%%%%%%%%%%%%%%%%%%%%%%%%%%%%%%%%%%%%%
\section{Artigo: Pais, Menezes, Providência}
%%%%%%%%%%%%%%%%%%%%%%%%%%%%%%%%%%%%%%%%%%%%

Sobre o modelo\cite{Pais}:
\begin{quote}
The NJL model can be extended [...] to yield reasonable saturation properties of nuclear matter, the field $\psi$ being the nucleon field. An effective density dependent coupling constant is obtained if the following extended NJL (eNJL) Lagrangian density, which actually pushes chiral symmetry restoration to higher densities, is considered,
\begin{equation}\label{Eq:Lagrangiana_eNLJ_Pais}
\begin{split}
	\mathcal{L} &= \bar{\psi}(i\gamma^\mu\partial_\mu)\psi + G_s[(\bar{\psi}\psi)^2 + (\bar{\psi}i\gamma_5\vec{\tau}\psi)^2] - G_v(\bar{\psi}\gamma^\mu\psi)^2 \\
	&\phantom{=}- G_{sv}[(\bar{\psi}\psi)^2 + (\bar{\psi}i\gamma_5\vec{\tau}\psi)^2](\bar{\psi}\gamma^\mu\psi)^2 - G_\rho[(\bar{\psi}\gamma^\mu\vec{\tau}\psi)^2 + (\bar{\psi}\gamma_5\gamma^\mu\vec{\tau}\psi)^2] \\
	&\phantom{=}- G_{v\rho}(\bar{\psi}\gamma^\mu\psi)^2[(\bar{\psi}\gamma^\mu\vec{\tau}\psi)^2 + (\bar{\psi}\gamma_5\gamma^\mu\vec{\tau}\psi)^2].
\end{split}
\end{equation}
\end{quote}

Outras informações relevantes (ainda do artigo)\footnote{Qual é a função do termo em $G_{s\rho}$? Qual é o papel do termo em $G_v$? (Ver o que tem no Walecka).}:
\begin{itemize}
	\item Para a matéria nuclear, a degenerescência é $2 N_f$;
	\item O \emph{cutoff} $\Lambda$ é tal que a massa do nucleon no vácuo seja de 939 MeV, determinada variacionalmente;
	\item O termo proporcional $G_s$ simula uma repulsão de curto alcance entre os nucleons (chiral invariant);
	\item ``The term in $G_{sv}$ accounts for the density dependence of the scalar coupling. For the nuclear matter, the NJL model leads to binding, but the binding energy per particle does no have a minimum except at a rather high density where the nucleon mass is small or vanishing. The introduction of the $G_{sv}$ coupling term is required to correct this.''
	\item O termo proporcional a $G_\rho$ (isovetor-vetor) é incluido para descrever a matéria nuclear assimétrica (em isospin); 
\end{itemize}

A partir da lagrangiana \eqref{Eq:Lagrangiana_eNLJ_Pais}, é possível determinar\footnote{Como? Através de $\omega(T,\mu) = -\frac{T}{V} \ln \mathcal{Z}$?} o potencial termodinâmico\footnote{Potencial Grand-canônico, ou potencial de Landau.} por unidade de volume, dado por
\begin{equation}\label{Eq:potencial_termodinamico}
	\omega(\mu) = \varepsilon_{\rm{kin}} - G_s\rho_s^2 + G_v\rho^2 + G_{sv}\rho_s^2\rho^2 + G_\rho\rho_3^2 + G_{v\rho}\rho^2\rho_3^2 - \mu_p\rho_p - \mu_n\rho_n,
\end{equation}
%
onde
\begin{itemize}
	\item $\rho$ é a densidade bariônica, dada pela soma das densidades de nêutron e próton\footnote{São densidades numéricas de partículas, ou seja, representam o número de partículas por unidade de volume.}:
	\begin{equation}
		\rho = \rho_p + \rho_n.
	\end{equation}

	\item As densidades bariônicas de próton e nêutron são dadas por\footnote{De onde vem essa expressão para $\rho_i$. Explicar, explicar o momento de Fermi também.}
	\begin{equation}
		\rho_i = \int_0^{k_F^i}\frac{dp}{\pi^2}p^2; \qquad i = p,n; \quad k_F^i = \textrm{momento de Fermi},
	\end{equation}
	%
	ou, caso $\rho_i$ sejam conhecidos
	\begin{equation}\label{Eq:Mom_Fermi_a_partir_de_rho}
		p_F^i = \sqrt[3]{3\pi^2\rho_i}.
	\end{equation}
	
	\item $\mu_p$ e $\mu_n$ representam os potenciais químicos de próton e nêutron, respectivamente.
\end{itemize}

O termo cinético na expressão acima pode ser calculado através de (primeiro termo da Eq. (1) em \cite{PRC_68_035804_2003}, o resto é energia potencial)\footnote{Degenerescência: O 2 se refere às duas possibilidades de spin; Podemos ter um $N_f$ que representa o número de sabores. Acredito que o sinal não seja do termo cinético, então tem que retirar daqui. No programa em Fortran está definido sem esse sinal. Apesar de que o valor de $\varepsilon_{\rm{kin}}$ fica negativo sem esse sinal; energia cinética é estritamente positiva!.}
\begin{align}
	\varepsilon_{\rm{kin}} &= \langle\bar{\psi}(\vec{\gamma}\cdot\vec{p}\psi\rangle \\
	&= - 2 N_c\sum_i \int \frac{d^3p}{(2\pi)^3}\frac{p^2 + m_i M_i}{E_i}(n_{i-}-n_{i+})\theta(\Lambda^2 - p^2),
\end{align}
%
onde
\begin{itemize}
	\item A soma se dá sobre as espécies de partículas;
	\item $N_c$ representa o número de cores\footnote{No nosso caso, 1?};
	\item $\theta$ é a função degrau, $\Lambda$ é o \emph{cutoff};
	\item $n_{i\pm}$ são as funções de distribuição de Fermi para estados de energia positiva e negativa (respectivamente), dados por
	\begin{equation}
		n_{i\pm} = \frac{1}{1 + \exp(\pm[\beta(E_i\mp\mu_i)])}
	\end{equation}
	%
	onde $i = p, n$ (no nosso caso, no artigo é $u, d, s$) e $\beta = T^{-1}$
	\item $M_i$ é a massa constituinte do nucleon em questão (quark, no artigo).
	\item $E_i = \sqrt{p^2 + M_i^2}$
	\item $m_i$ no artigo são as massas (nuas?) dos quarks, e no nosso caso?\footnote{Ver isso.}
\end{itemize}

Se tomarmos $T \to 0$, temos que $n_{i-} \to 1$ e $n_{i+} \to 0$; Além disso, se o integrando só depende do módulo de $\vec{p}$, então (\cite{Glendenning}, p. 92)
\begin{equation}
	\int\frac{d^3p}{(2\pi)^3} \to \frac{1}{2\pi^2}\int p^2dp.
\end{equation}
%
Logo, temos
\begin{align}
	\varepsilon &= -2 N_c \frac{1}{2\pi^2}\sum_i \int p^2 dp \frac{p^2 + m_i M_i}{\sqrt{p^2 + M_i^2}} \theta(\Lambda^2 - p^2) \\
	&= -\frac{N_c}{\pi^2}\sum_i\left[\int \frac{p^4dp}{\sqrt{p^2 + M_i^2}}\theta(\Lambda^2 - p^2) + \int m_i M_i \frac{p^2 dp}{\sqrt{p^2 + M_i^2}}\theta(\Lambda^2 - p^2)\right]
\end{align}
%
Podemos utilizar as relações (\cite{Glendenning} p. 94\footnote{Na Ref. o primeiro termo da segunda expressão aparece sem o $k$ multiplicando, o que dimensionalmente está incorreto.})
\begin{align}
	\int \frac{k^4}{\sqrt{k^2 + m^2}} dk &= \frac{1}{4}\left[k^3\epsilon - \frac{3}{2} m^2k\epsilon + \frac{3}{2}m^4\ln\frac{\epsilon + k}{m} \right]\\
	\int \frac{k^2}{\sqrt{k^2 + m^2}} dk &= \frac{1}{2}\left[k\epsilon - \frac{1}{2}m^2\ln\frac{\epsilon + k}{m}\right] \label{Eq:Integ_momento_quad}
\end{align}
%
onde $\epsilon = \sqrt{k^2+m^2}$. Tomando o caso $m_i \to 0$\footnote{No prog. \texttt{eos\_enjl1-dens-assym- clean-rho-vr.f}: $\varepsilon \propto [F_2(M, k_F^i) - F_2(M, \Lambda)]$ ao invés de $\varepsilon \propto [F_2(M_i, \Lambda) - F_2(M_i, 0)]$; Isso se deve à retirada da contribuição do vácuo. Além disso, aparentemente os $M_i$ podem ser diferentes. No prog. são iguais, imagino que seja por considerarmos $m_n = m_p = m_N$.}, obtemos
\begin{equation}\label{Eq:Energia_kin}
	\varepsilon_{\rm{kin}} = -\frac{N_c}{\pi^2}\sum_i \Big[\underbrace{\frac{1}{8}\Big((2p^3 - 3M_i^2p)\sqrt{p^2 + M_i^2} + 3M_i^4\ln\frac{p + \sqrt{p^2 + M_i^2}}{M_i}\Big)}_{F_2(m,p)}\Big]_0^\Lambda
\end{equation}

A densidade escalar $\rho_s$ é dada por\footnote{De onde vem essa expressão?}
\begin{equation}\label{Eq:Dens_Escalar}
	\rho_s^i = \frac{M}{\pi^2}[F_0(M, p_F^i) - F_0(M, \Lambda)], \quad i = p, n,
\end{equation}
%
onde
\begin{equation}
	F_0(M, x) = \int_0^x \frac{dp}{\pi^2}\frac{p^2}{\sqrt{M^2 + p^2}}, \quad i = p, n.
\end{equation}
%
Utilizando a Equação~\eqref{Eq:Integ_momento_quad}, podemos reescrever a equação acima como\footnote{Esse $\pi^2$ no denominador está com cara de que não deveria estar aí. A parece um $\pi^2$ no denominador na definição de $\rho_s$ em termos de $F_0(M, x)$ e se eu deixar nos dois lugares, não há raízes para a equação do Gap. Por outro lado, mesmo que eu deixe, $\rho_s$ é menor que zero!}
\begin{equation}
	F_0(M, x) = \frac{1}{2\pi^2}\left[x\sqrt{x^2+M^2} - M^2 \ln \frac{x + \sqrt{x^2+M^2}}{M}\right].
\end{equation}

A massa\footnote{constituinte?} $M$ na equação acima é dada por\footnote{Essa equação é conhecida como \emph{Gap equation} (?).}
\begin{equation}\label{Eq:Gap}
	M = -2G_s\rho_s + 2G_{sv}\rho_s\rho^2,
\end{equation}
%
com $\rho_s = \rho_s^p + \rho_s^n$. Temos, portanto, uma interdependência entre as equações. Para que seja possível solucionar tais equações, podemos definir uma função $f(M)$ de tal forma que
\begin{equation}\label{Eq:Gap_zero}
	f(M) = M + 2G_s\rho_s - 2G_{sv}\rho_s\rho^2.
\end{equation}
%
Para solucionarmos a equação acima, basta utilizarmos uma rotina para encontrar zeros de funções, por exemplo biseção ou Newton-Raphson, encontrando o valor de $M$ para o qual $f(M) = 0$. A densidade escalar $\rho_s$ pode ser calculada através da expressão~\eqref{Eq:Dens_Escalar}\footnote{Na prática é mais fácil salvar o último valor de $\rho_s$ calculado pela rotina que tenta encontrar o zero da função $f(M)$.}.

Os potenciais químicos são dados por\footnote{Como essas expressões são calculadas?}
\begin{equation}\label{Eq:Potenciais_Quimicos}
	\mu_i = E_{p_F}^i + 2G_v\rho + 2G_{sv}\rho\rho_s^2 \pm 2G_\rho\rho_3+2G_{v\rho}\rho_3^2\rho \pm 2G_{v\rho}\rho^2\rho_3,
\end{equation}
%
onde $i = p,n$, os sinais superiores se referem ao caso de prótons, e $E_{p_F}^i = \sqrt{M^2 + (p_F^i)^2}$.

As equações de estado para pressão $P$ e densidade de energia $\varepsilon$ são dadas por\footnote{Como são calculadas?}
\begin{align}
	P &= -\omega(\mu) + \epsilon_0 \label{Eq:Pressao}\\
	\varepsilon &= -P + \mu_p\rho_p + \mu_n\rho_n. \label{Eq:Densidade_energia}
\end{align}

%%%%%%%%%%%%%%%%%%%%%%%%%%%%%
\section{Análise dimensional}
%%%%%%%%%%%%%%%%%%%%%%%%%%%%%

Devido ao fato de que $\hbar = c = 1$, adimensionais, e que não carregamos quaisquer unidades, é comum que algumas equações tenham dimensões discrepantes entre os membros esquerdo e direito, ou mesmo entre termos de um mesmo membro. Podemos acertar as dimensões multiplicando por potências de $\hbar c$. Nas unidades usuais em Física Nuclear, temos que $\hbar c = 197.326\rm{MeV}\cdot\rm{fm}$. Logo, todas as grandezas têm dimensões que envolvem MeV ou fm, ou uma combinação de ambos\sidenote{Note que de qualquer forma as unidades são atípicas, pois --~por exemplo~-- a unidade de massa é o eV, que na realidade tem dimensão de energia. Isso pode ser explicado através de $E^2 = p^2 c^2 + m^2 c^4$, onde assumimos que $c = 1$, adimensional.}:

\begin{itemize}
	\item Como $E^2 = p^2c^2 + m^2c^4$, temos $[E] = [m] = [p]$;
	\item Como $\rho = \bar{\psi}\gamma^0\psi$ é o número de partículas por unidade de volume, temos que $[\rho] = \rm{fm}^{-3}$
	\item O item acima implica que $[\psi] = \rm{fm}^{-3/2}$. Consequentemente, $[\rho_s] = [\bar{\psi}\psi] = \rm{fm}^{-3}$. 
	\item Como o potencial químico esta relacionado à variação de energia ao se adicionar ou retirar partículas do sistema, sua unidade é a de energia (MeV).
	\item O potencial termodinâmico $\omega$ é o potencial grande-canônico $\Omega$\footnote{Potencial de Landau} por unidade de volume, portanto tem dimensão de energia por unidade de volume ($\rm{MeV}\cdot\rm{fm}^{-3}$), assim como a densidade de energia $\varepsilon$ e a pressão $P$.
\end{itemize}

Dessa forma, as equações discutidas acima precisam ter suas unidades checadas e --~quando necessário~-- corrigidas, de forma a ficarem consistentes. Temos então:
\begin{fullwidth}
\begin{itemize}

\item O potencial termodinâmico (por unidade de volume) $\omega$ deve ter dimensão energia por volume (no nosso caso, $\rm{MeV}/\rm{fm}^3$):
\begin{equation}
	\underbrace{\omega}_{\frac{\rm{MeV}}{\rm{fm}^3}} = \underbrace{\varepsilon_{\rm{kin}}}_{\frac{\rm{MeV}}{\rm{fm}^3}} - \underbrace{G_s}_{\rm{fm}^2}\underbrace{\rho_s^2}_{\rm{fm}^{-6}} + \underbrace{G_v}_{\rm{fm}^2}\underbrace{\rho^2}_{\rm{fm}^{-6}} + \underbrace{G_{sv}}_{\rm{fm}^8}\underbrace{\rho_s^2\rho^2}_{\rm{fm}^{-12}} + \underbrace{G_\rho}_{\rm{fm}^2}\underbrace{\rho_3^2}_{\rm{fm}^{-6}} + \underbrace{G_{v\rho}}_{\rm{fm}^2}\underbrace{\rho^2\rho_3^2}_{\rm{fm}^{-12}} + \underbrace{\mu_p}_{\rm{MeV}}\underbrace{\rho_p}_{\rm{fm}^{-3}} + \underbrace{\mu_n}_{\rm{MeV}}\underbrace{\rho_n}_{\rm{fm}^{-3}}.
\end{equation}
%
Os termos envolvendo as constantes $G_i$ necessitam ser multiplicados por $\hbar c$, cuja dimensão é $\rm{MeV}\cdot\rm{fm}$, resultando na dimensão $\rm{MeV}/\rm{fm}^3$ para tais termos.

\item A pressão deve ter unidade de energia por unidade de volume ($\rm{MeV}/\rm{fm}^3$):
\begin{equation}
	\underbrace{P}_{\frac{\rm{MeV}}{\rm{fm}^3}} = - \underbrace{\omega}_{\frac{\rm{MeV}}{\rm{fm}^3}} + \underbrace{\varepsilon_0}_{\frac{\rm{MeV}}{\rm{fm}^3}},
\end{equation}
%
assim como a densidade de energia
\begin{equation}
	\underbrace{\varepsilon}_{\frac{\rm{MeV}}{\rm{fm}^3}} = - \underbrace{P}_{\frac{\rm{MeV}}{\rm{fm}^3}} + \underbrace{\mu_p}_{\rm{MeV}}\underbrace{\rho_p}_{\rm{fm}^{-3}} + \underbrace{\mu_n}_{\rm{MeV}}\underbrace{\rho_n}_{\rm{fm}^{-3}},
\end{equation}
%
e podemos ver que ambas estão com todas as dimensões corretas.

\item A equação para o cálculo da massa
\begin{equation}
	\underbrace{M}_{\rm{MeV}} = -2 \underbrace{G_s}_{\rm{fm}^2} \underbrace{\rho_s}_{\rm{fm}^{-3}} + 2 \underbrace{G_{sv}}_{\rm{fm}^8}\underbrace{\rho_s\rho^2}_{\rm{fm}^{-9}}
\end{equation}
%
necessita ser multiplicada por $\hbar c$ no lado direito.

\item A equação para a densidade bariônica
\begin{equation}
	\underbrace{\rho_i}_{\rm{fm}^{-3}} = \underbrace{\int_0^{p_F^i} \frac{p^2 dp}{\pi^2}}_{\rm{MeV}^3} = \underbrace{\frac{1}{3\pi^2}p^3\Big|_0^{p_F^i}}_{\rm{MeV}^3},
\end{equation}
%
assim como a equação para a densidade escalar
\begin{equation}
	\underbrace{\rho_s^i}_{\rm{fm}^{-3}} = \underbrace{\frac{M}{\pi^2}[F_0(M,p_F^i) - F_0(M, \Lambda)]}_{\rm{MeV}^3},
\end{equation}
%
onde usamos (vide Eq.~\eqref{Eq:Integ_momento_quad})
\begin{equation}
	F_0(m,p) = \underbrace{\int_0^x\frac{dp}{\pi^2} \frac{p^2}{\sqrt{p^2 + m^2}}}_{\rm{MeV}^2},
\end{equation}
%
devem ser multiplicadas por $(\hbar c)^{-3}$.

\item Para o potencial químico temos
\begin{equation}
	\underbrace{\mu_i}_{\rm{MeV}} = \underbrace{E_{p_F}^i}_{\rm{MeV}} +~2 \underbrace{G_v}_{\rm{fm}^2}\underbrace{\rho}_{\rm{fm^{-3}}} +~2 \underbrace{G_{sv}}_{\rm{fm}^8}\underbrace{\rho\rho_s^2}_{\rm{fm^{-9}}} \pm~2 \underbrace{G_\rho}_{\rm{fm}^2}\underbrace{\rho_3}_{\rm{fm^{-3}}} +~2 \underbrace{G_{v\rho}}_{\rm{fm}^8}\underbrace{\rho_3^2\rho}_{\rm{fm^{-9}}} \pm~2 \underbrace{G_{v\rho}}_{\rm{fm}^8}\underbrace{\rho^2\rho_3}_{\rm{fm^{-9}}},
\end{equation}
%
onde
\begin{equation}
	\underbrace{E_{p_F}^i}_{\rm{MeV}} = \sqrt{M^2 + p_F^2}; \quad [M] = [p_F] = \rm{MeV}.
\end{equation}
%
Portanto, verificamos que os termos proporcionais a $G_i$ devem ser multiplicados por $\hbar c$.

\item O termo cinético da energia é dado pela Equação~\eqref{Eq:Energia_kin}. O termo definido como $F_2(m, p)$ tem dimensão de $\rm{MeV}^4$, já que todos as parcelas são produto de $m$, $p$, e $\epsilon = \sqrt{p^2+m_i^2}$. Além disso, o argumento do logarítimo é adimensional. No entanto, como $\varepsilon_{\rm{kin}}$ é uma densidade de energia, a dimensão correta é $\rm{MeV} / \rm{fm}^3$. Portanto, é necessário multiplicar a expressão para $\varepsilon_{\rm{kin}}$ por $(\hbar c)^{-3}$.
\end{itemize}
\end{fullwidth}

%%%%%%%%%%%%%%%%%%%%%%%%%%%%%%%%%%%%%%%%%%%%%%%
\section{Solução da equação para $M$, $\rho_s$}
%%%%%%%%%%%%%%%%%%%%%%%%%%%%%%%%%%%%%%%%%%%%%%%

A solução da Equação~\eqref{Eq:Gap} é encontrada zerando a Equação~\eqref{Eq:Gap_zero}. No entanto, essa última depende de $\rho$ e dos momentos de Fermi para próton e nêutron (o que é uma dependência indireta de $\rho$ e da fração de prótons). Assim, a forma da função pode se alterar de acordo com os valores de tais variáveis. A Figura~\ref{Fig:Gap_zero_graph} mostra curvas de $f(M)$ para diferentes valores de $\rho$.

\begin{figure*}
	\begin{tikzpicture}[gnuplot]
%% generated with GNUPLOT 5.0p2 (Lua 5.2; terminal rev. 99, script rev. 100)
%% Thu Feb 25 17:15:55 2016
\path (0.000,0.000) rectangle (14.000,9.000);
\gpcolor{color=gp lt color border}
\gpsetlinetype{gp lt border}
\gpsetdashtype{gp dt solid}
\gpsetlinewidth{1.00}
\draw[gp path] (1.504,0.985)--(1.684,0.985);
\draw[gp path] (13.447,0.985)--(13.267,0.985);
\node[gp node right] at (1.320,0.985) {$-400$};
\draw[gp path] (1.504,1.750)--(1.684,1.750);
\draw[gp path] (13.447,1.750)--(13.267,1.750);
\node[gp node right] at (1.320,1.750) {$-200$};
\draw[gp path] (1.504,2.514)--(1.684,2.514);
\draw[gp path] (13.447,2.514)--(13.267,2.514);
\node[gp node right] at (1.320,2.514) {$0$};
\draw[gp path] (1.504,3.279)--(1.684,3.279);
\draw[gp path] (13.447,3.279)--(13.267,3.279);
\node[gp node right] at (1.320,3.279) {$200$};
\draw[gp path] (1.504,4.043)--(1.684,4.043);
\draw[gp path] (13.447,4.043)--(13.267,4.043);
\node[gp node right] at (1.320,4.043) {$400$};
\draw[gp path] (1.504,4.808)--(1.684,4.808);
\draw[gp path] (13.447,4.808)--(13.267,4.808);
\node[gp node right] at (1.320,4.808) {$600$};
\draw[gp path] (1.504,5.573)--(1.684,5.573);
\draw[gp path] (13.447,5.573)--(13.267,5.573);
\node[gp node right] at (1.320,5.573) {$800$};
\draw[gp path] (1.504,6.337)--(1.684,6.337);
\draw[gp path] (13.447,6.337)--(13.267,6.337);
\node[gp node right] at (1.320,6.337) {$1000$};
\draw[gp path] (1.504,7.102)--(1.684,7.102);
\draw[gp path] (13.447,7.102)--(13.267,7.102);
\node[gp node right] at (1.320,7.102) {$1200$};
\draw[gp path] (1.504,7.866)--(1.684,7.866);
\draw[gp path] (13.447,7.866)--(13.267,7.866);
\node[gp node right] at (1.320,7.866) {$1400$};
\draw[gp path] (1.504,8.631)--(1.684,8.631);
\draw[gp path] (13.447,8.631)--(13.267,8.631);
\node[gp node right] at (1.320,8.631) {$1600$};
\draw[gp path] (1.504,0.985)--(1.504,1.165);
\draw[gp path] (1.504,8.631)--(1.504,8.451);
\node[gp node center] at (1.504,0.677) {$0$};
\draw[gp path] (2.698,0.985)--(2.698,1.165);
\draw[gp path] (2.698,8.631)--(2.698,8.451);
\node[gp node center] at (2.698,0.677) {$200$};
\draw[gp path] (3.893,0.985)--(3.893,1.165);
\draw[gp path] (3.893,8.631)--(3.893,8.451);
\node[gp node center] at (3.893,0.677) {$400$};
\draw[gp path] (5.087,0.985)--(5.087,1.165);
\draw[gp path] (5.087,8.631)--(5.087,8.451);
\node[gp node center] at (5.087,0.677) {$600$};
\draw[gp path] (6.281,0.985)--(6.281,1.165);
\draw[gp path] (6.281,8.631)--(6.281,8.451);
\node[gp node center] at (6.281,0.677) {$800$};
\draw[gp path] (7.476,0.985)--(7.476,1.165);
\draw[gp path] (7.476,8.631)--(7.476,8.451);
\node[gp node center] at (7.476,0.677) {$1000$};
\draw[gp path] (8.670,0.985)--(8.670,1.165);
\draw[gp path] (8.670,8.631)--(8.670,8.451);
\node[gp node center] at (8.670,0.677) {$1200$};
\draw[gp path] (9.864,0.985)--(9.864,1.165);
\draw[gp path] (9.864,8.631)--(9.864,8.451);
\node[gp node center] at (9.864,0.677) {$1400$};
\draw[gp path] (11.058,0.985)--(11.058,1.165);
\draw[gp path] (11.058,8.631)--(11.058,8.451);
\node[gp node center] at (11.058,0.677) {$1600$};
\draw[gp path] (12.253,0.985)--(12.253,1.165);
\draw[gp path] (12.253,8.631)--(12.253,8.451);
\node[gp node center] at (12.253,0.677) {$1800$};
\draw[gp path] (13.447,0.985)--(13.447,1.165);
\draw[gp path] (13.447,8.631)--(13.447,8.451);
\node[gp node center] at (13.447,0.677) {$2000$};
\draw[gp path] (1.504,8.631)--(1.504,0.985)--(13.447,0.985)--(13.447,8.631)--cycle;
\node[gp node center,rotate=-270] at (0.246,4.808) {$f(M) = M + G_s\rho_s - 2G_{sv}\rho_s\rho^2$ (MeV)};
\node[gp node center] at (7.475,0.215) {$M$ (MeV)};
\node[gp node left] at (2.972,8.297) {$\rho \approx 0.1$};
\gpcolor{rgb color={0.580,0.000,0.827}}
\draw[gp path] (1.872,8.297)--(2.788,8.297);
\draw[gp path] (1.504,2.514)--(1.507,2.511)--(1.510,2.508)--(1.513,2.506)--(1.516,2.503)%
  --(1.519,2.500)--(1.522,2.497)--(1.525,2.494)--(1.528,2.491)--(1.531,2.488)--(1.534,2.485)%
  --(1.537,2.482)--(1.540,2.480)--(1.543,2.477)--(1.546,2.474)--(1.549,2.471)--(1.552,2.468)%
  --(1.555,2.465)--(1.558,2.462)--(1.561,2.459)--(1.564,2.457)--(1.567,2.454)--(1.570,2.451)%
  --(1.573,2.448)--(1.576,2.445)--(1.579,2.442)--(1.582,2.439)--(1.585,2.436)--(1.588,2.434)%
  --(1.591,2.431)--(1.594,2.428)--(1.597,2.425)--(1.600,2.422)--(1.603,2.419)--(1.606,2.416)%
  --(1.609,2.414)--(1.611,2.411)--(1.614,2.408)--(1.617,2.405)--(1.620,2.402)--(1.623,2.399)%
  --(1.626,2.396)--(1.629,2.394)--(1.632,2.391)--(1.635,2.388)--(1.638,2.385)--(1.641,2.382)%
  --(1.644,2.379)--(1.647,2.376)--(1.650,2.374)--(1.653,2.371)--(1.656,2.368)--(1.659,2.365)%
  --(1.662,2.362)--(1.665,2.359)--(1.668,2.357)--(1.671,2.354)--(1.674,2.351)--(1.677,2.348)%
  --(1.680,2.345)--(1.683,2.343)--(1.686,2.340)--(1.689,2.337)--(1.692,2.334)--(1.695,2.331)%
  --(1.698,2.328)--(1.701,2.326)--(1.704,2.323)--(1.707,2.320)--(1.710,2.317)--(1.713,2.314)%
  --(1.716,2.312)--(1.719,2.309)--(1.722,2.306)--(1.725,2.303)--(1.728,2.301)--(1.731,2.298)%
  --(1.734,2.295)--(1.737,2.292)--(1.740,2.289)--(1.743,2.287)--(1.746,2.284)--(1.749,2.281)%
  --(1.752,2.278)--(1.755,2.276)--(1.758,2.273)--(1.761,2.270)--(1.764,2.267)--(1.767,2.265)%
  --(1.770,2.262)--(1.773,2.259)--(1.776,2.256)--(1.779,2.254)--(1.782,2.251)--(1.785,2.248)%
  --(1.788,2.246)--(1.791,2.243)--(1.794,2.240)--(1.797,2.237)--(1.800,2.235)--(1.803,2.232)%
  --(1.806,2.229)--(1.809,2.227)--(1.812,2.224)--(1.815,2.221)--(1.818,2.219)--(1.820,2.216)%
  --(1.823,2.213)--(1.826,2.211)--(1.829,2.208)--(1.832,2.205)--(1.835,2.203)--(1.838,2.200)%
  --(1.841,2.197)--(1.844,2.195)--(1.847,2.192)--(1.850,2.189)--(1.853,2.187)--(1.856,2.184)%
  --(1.859,2.181)--(1.862,2.179)--(1.865,2.176)--(1.868,2.174)--(1.871,2.171)--(1.874,2.168)%
  --(1.877,2.166)--(1.880,2.163)--(1.883,2.161)--(1.886,2.158)--(1.889,2.155)--(1.892,2.153)%
  --(1.895,2.150)--(1.898,2.148)--(1.901,2.145)--(1.904,2.143)--(1.907,2.140)--(1.910,2.137)%
  --(1.913,2.135)--(1.916,2.132)--(1.919,2.130)--(1.922,2.127)--(1.925,2.125)--(1.928,2.122)%
  --(1.931,2.120)--(1.934,2.117)--(1.937,2.115)--(1.940,2.112)--(1.943,2.110)--(1.946,2.107)%
  --(1.949,2.105)--(1.952,2.102)--(1.955,2.100)--(1.958,2.097)--(1.961,2.095)--(1.964,2.092)%
  --(1.967,2.090)--(1.970,2.087)--(1.973,2.085)--(1.976,2.083)--(1.979,2.080)--(1.982,2.078)%
  --(1.985,2.075)--(1.988,2.073)--(1.991,2.070)--(1.994,2.068)--(1.997,2.066)--(2.000,2.063)%
  --(2.003,2.061)--(2.006,2.058)--(2.009,2.056)--(2.012,2.054)--(2.015,2.051)--(2.018,2.049)%
  --(2.021,2.046)--(2.024,2.044)--(2.027,2.042)--(2.029,2.039)--(2.032,2.037)--(2.035,2.035)%
  --(2.038,2.032)--(2.041,2.030)--(2.044,2.028)--(2.047,2.025)--(2.050,2.023)--(2.053,2.021)%
  --(2.056,2.018)--(2.059,2.016)--(2.062,2.014)--(2.065,2.012)--(2.068,2.009)--(2.071,2.007)%
  --(2.074,2.005)--(2.077,2.002)--(2.080,2.000)--(2.083,1.998)--(2.086,1.996)--(2.089,1.994)%
  --(2.092,1.991)--(2.095,1.989)--(2.098,1.987)--(2.101,1.985)--(2.104,1.982)--(2.107,1.980)%
  --(2.110,1.978)--(2.113,1.976)--(2.116,1.974)--(2.119,1.971)--(2.122,1.969)--(2.125,1.967)%
  --(2.128,1.965)--(2.131,1.963)--(2.134,1.961)--(2.137,1.958)--(2.140,1.956)--(2.143,1.954)%
  --(2.146,1.952)--(2.149,1.950)--(2.152,1.948)--(2.155,1.946)--(2.158,1.944)--(2.161,1.942)%
  --(2.164,1.939)--(2.167,1.937)--(2.170,1.935)--(2.173,1.933)--(2.176,1.931)--(2.179,1.929)%
  --(2.182,1.927)--(2.185,1.925)--(2.188,1.923)--(2.191,1.921)--(2.194,1.919)--(2.197,1.917)%
  --(2.200,1.915)--(2.203,1.913)--(2.206,1.911)--(2.209,1.909)--(2.212,1.907)--(2.215,1.905)%
  --(2.218,1.903)--(2.221,1.901)--(2.224,1.899)--(2.227,1.897)--(2.230,1.895)--(2.233,1.893)%
  --(2.236,1.891)--(2.238,1.889)--(2.241,1.887)--(2.244,1.885)--(2.247,1.883)--(2.250,1.882)%
  --(2.253,1.880)--(2.256,1.878)--(2.259,1.876)--(2.262,1.874)--(2.265,1.872)--(2.268,1.870)%
  --(2.271,1.868)--(2.274,1.867)--(2.277,1.865)--(2.280,1.863)--(2.283,1.861)--(2.286,1.859)%
  --(2.289,1.857)--(2.292,1.856)--(2.295,1.854)--(2.298,1.852)--(2.301,1.850)--(2.304,1.848)%
  --(2.307,1.847)--(2.310,1.845)--(2.313,1.843)--(2.316,1.841)--(2.319,1.839)--(2.322,1.838)%
  --(2.325,1.836)--(2.328,1.834)--(2.331,1.832)--(2.334,1.831)--(2.337,1.829)--(2.340,1.827)%
  --(2.343,1.826)--(2.346,1.824)--(2.349,1.822)--(2.352,1.821)--(2.355,1.819)--(2.358,1.817)%
  --(2.361,1.816)--(2.364,1.814)--(2.367,1.812)--(2.370,1.811)--(2.373,1.809)--(2.376,1.807)%
  --(2.379,1.806)--(2.382,1.804)--(2.385,1.802)--(2.388,1.801)--(2.391,1.799)--(2.394,1.798)%
  --(2.397,1.796)--(2.400,1.794)--(2.403,1.793)--(2.406,1.791)--(2.409,1.790)--(2.412,1.788)%
  --(2.415,1.787)--(2.418,1.785)--(2.421,1.784)--(2.424,1.782)--(2.427,1.780)--(2.430,1.779)%
  --(2.433,1.777)--(2.436,1.776)--(2.439,1.774)--(2.442,1.773)--(2.445,1.771)--(2.447,1.770)%
  --(2.450,1.768)--(2.453,1.767)--(2.456,1.766)--(2.459,1.764)--(2.462,1.763)--(2.465,1.761)%
  --(2.468,1.760)--(2.471,1.758)--(2.474,1.757)--(2.477,1.756)--(2.480,1.754)--(2.483,1.753)%
  --(2.486,1.751)--(2.489,1.750)--(2.492,1.749)--(2.495,1.747)--(2.498,1.746)--(2.501,1.744)%
  --(2.504,1.743)--(2.507,1.742)--(2.510,1.740)--(2.513,1.739)--(2.516,1.738)--(2.519,1.736)%
  --(2.522,1.735)--(2.525,1.734)--(2.528,1.732)--(2.531,1.731)--(2.534,1.730)--(2.537,1.729)%
  --(2.540,1.727)--(2.543,1.726)--(2.546,1.725)--(2.549,1.724)--(2.552,1.722)--(2.555,1.721)%
  --(2.558,1.720)--(2.561,1.719)--(2.564,1.717)--(2.567,1.716)--(2.570,1.715)--(2.573,1.714)%
  --(2.576,1.713)--(2.579,1.711)--(2.582,1.710)--(2.585,1.709)--(2.588,1.708)--(2.591,1.707)%
  --(2.594,1.705)--(2.597,1.704)--(2.600,1.703)--(2.603,1.702)--(2.606,1.701)--(2.609,1.700)%
  --(2.612,1.699)--(2.615,1.697)--(2.618,1.696)--(2.621,1.695)--(2.624,1.694)--(2.627,1.693)%
  --(2.630,1.692)--(2.633,1.691)--(2.636,1.690)--(2.639,1.689)--(2.642,1.688)--(2.645,1.687)%
  --(2.648,1.686)--(2.651,1.685)--(2.654,1.684)--(2.656,1.682)--(2.659,1.681)--(2.662,1.680)%
  --(2.665,1.679)--(2.668,1.678)--(2.671,1.677)--(2.674,1.676)--(2.677,1.675)--(2.680,1.674)%
  --(2.683,1.673)--(2.686,1.673)--(2.689,1.672)--(2.692,1.671)--(2.695,1.670)--(2.698,1.669)%
  --(2.701,1.668)--(2.704,1.667)--(2.707,1.666)--(2.710,1.665)--(2.713,1.664)--(2.716,1.663)%
  --(2.719,1.662)--(2.722,1.661)--(2.725,1.660)--(2.728,1.660)--(2.731,1.659)--(2.734,1.658)%
  --(2.737,1.657)--(2.740,1.656)--(2.743,1.655)--(2.746,1.654)--(2.749,1.654)--(2.752,1.653)%
  --(2.755,1.652)--(2.758,1.651)--(2.761,1.650)--(2.764,1.649)--(2.767,1.649)--(2.770,1.648)%
  --(2.773,1.647)--(2.776,1.646)--(2.779,1.646)--(2.782,1.645)--(2.785,1.644)--(2.788,1.643)%
  --(2.791,1.642)--(2.794,1.642)--(2.797,1.641)--(2.800,1.640)--(2.803,1.639)--(2.806,1.639)%
  --(2.809,1.638)--(2.812,1.637)--(2.815,1.637)--(2.818,1.636)--(2.821,1.635)--(2.824,1.635)%
  --(2.827,1.634)--(2.830,1.633)--(2.833,1.632)--(2.836,1.632)--(2.839,1.631)--(2.842,1.630)%
  --(2.845,1.630)--(2.848,1.629)--(2.851,1.629)--(2.854,1.628)--(2.857,1.627)--(2.860,1.627)%
  --(2.863,1.626)--(2.866,1.625)--(2.868,1.625)--(2.871,1.624)--(2.874,1.624)--(2.877,1.623)%
  --(2.880,1.622)--(2.883,1.622)--(2.886,1.621)--(2.889,1.621)--(2.892,1.620)--(2.895,1.620)%
  --(2.898,1.619)--(2.901,1.618)--(2.904,1.618)--(2.907,1.617)--(2.910,1.617)--(2.913,1.616)%
  --(2.916,1.616)--(2.919,1.615)--(2.922,1.615)--(2.925,1.614)--(2.928,1.614)--(2.931,1.613)%
  --(2.934,1.613)--(2.937,1.612)--(2.940,1.612)--(2.943,1.611)--(2.946,1.611)--(2.949,1.610)%
  --(2.952,1.610)--(2.955,1.610)--(2.958,1.609)--(2.961,1.609)--(2.964,1.608)--(2.967,1.608)%
  --(2.970,1.607)--(2.973,1.607)--(2.976,1.607)--(2.979,1.606)--(2.982,1.606)--(2.985,1.605)%
  --(2.988,1.605)--(2.991,1.605)--(2.994,1.604)--(2.997,1.604)--(3.000,1.603)--(3.003,1.603)%
  --(3.006,1.603)--(3.009,1.602)--(3.012,1.602)--(3.015,1.602)--(3.018,1.601)--(3.021,1.601)%
  --(3.024,1.601)--(3.027,1.600)--(3.030,1.600)--(3.033,1.600)--(3.036,1.599)--(3.039,1.599)%
  --(3.042,1.599)--(3.045,1.599)--(3.048,1.598)--(3.051,1.598)--(3.054,1.598)--(3.057,1.597)%
  --(3.060,1.597)--(3.063,1.597)--(3.066,1.597)--(3.069,1.596)--(3.072,1.596)--(3.075,1.596)%
  --(3.077,1.596)--(3.080,1.595)--(3.083,1.595)--(3.086,1.595)--(3.089,1.595)--(3.092,1.595)%
  --(3.095,1.594)--(3.098,1.594)--(3.101,1.594)--(3.104,1.594)--(3.107,1.594)--(3.110,1.593)%
  --(3.113,1.593)--(3.116,1.593)--(3.119,1.593)--(3.122,1.593)--(3.125,1.593)--(3.128,1.592)%
  --(3.131,1.592)--(3.134,1.592)--(3.137,1.592)--(3.140,1.592)--(3.143,1.592)--(3.146,1.592)%
  --(3.149,1.591)--(3.152,1.591)--(3.155,1.591)--(3.158,1.591)--(3.161,1.591)--(3.164,1.591)%
  --(3.167,1.591)--(3.170,1.591)--(3.173,1.591)--(3.176,1.591)--(3.179,1.590)--(3.182,1.590)%
  --(3.185,1.590)--(3.188,1.590)--(3.191,1.590)--(3.194,1.590)--(3.197,1.590)--(3.200,1.590)%
  --(3.203,1.590)--(3.206,1.590)--(3.209,1.590)--(3.212,1.590)--(3.215,1.590)--(3.218,1.590)%
  --(3.221,1.590)--(3.224,1.590)--(3.227,1.590)--(3.230,1.590)--(3.233,1.590)--(3.236,1.590)%
  --(3.239,1.590)--(3.242,1.590)--(3.245,1.590)--(3.248,1.590)--(3.251,1.590)--(3.254,1.590)%
  --(3.257,1.590)--(3.260,1.590)--(3.263,1.590)--(3.266,1.590)--(3.269,1.590)--(3.272,1.590)%
  --(3.275,1.590)--(3.278,1.591)--(3.281,1.591)--(3.284,1.591)--(3.286,1.591)--(3.289,1.591)%
  --(3.292,1.591)--(3.295,1.591)--(3.298,1.591)--(3.301,1.591)--(3.304,1.591)--(3.307,1.591)%
  --(3.310,1.592)--(3.313,1.592)--(3.316,1.592)--(3.319,1.592)--(3.322,1.592)--(3.325,1.592)%
  --(3.328,1.592)--(3.331,1.593)--(3.334,1.593)--(3.337,1.593)--(3.340,1.593)--(3.343,1.593)%
  --(3.346,1.593)--(3.349,1.593)--(3.352,1.594)--(3.355,1.594)--(3.358,1.594)--(3.361,1.594)%
  --(3.364,1.594)--(3.367,1.595)--(3.370,1.595)--(3.373,1.595)--(3.376,1.595)--(3.379,1.595)%
  --(3.382,1.596)--(3.385,1.596)--(3.388,1.596)--(3.391,1.596)--(3.394,1.597)--(3.397,1.597)%
  --(3.400,1.597)--(3.403,1.597)--(3.406,1.597)--(3.409,1.598)--(3.412,1.598)--(3.415,1.598)%
  --(3.418,1.599)--(3.421,1.599)--(3.424,1.599)--(3.427,1.599)--(3.430,1.600)--(3.433,1.600)%
  --(3.436,1.600)--(3.439,1.600)--(3.442,1.601)--(3.445,1.601)--(3.448,1.601)--(3.451,1.602)%
  --(3.454,1.602)--(3.457,1.602)--(3.460,1.603)--(3.463,1.603)--(3.466,1.603)--(3.469,1.604)%
  --(3.472,1.604)--(3.475,1.604)--(3.478,1.605)--(3.481,1.605)--(3.484,1.605)--(3.487,1.606)%
  --(3.490,1.606)--(3.493,1.606)--(3.495,1.607)--(3.498,1.607)--(3.501,1.607)--(3.504,1.608)%
  --(3.507,1.608)--(3.510,1.608)--(3.513,1.609)--(3.516,1.609)--(3.519,1.610)--(3.522,1.610)%
  --(3.525,1.610)--(3.528,1.611)--(3.531,1.611)--(3.534,1.612)--(3.537,1.612)--(3.540,1.612)%
  --(3.543,1.613)--(3.546,1.613)--(3.549,1.614)--(3.552,1.614)--(3.555,1.615)--(3.558,1.615)%
  --(3.561,1.615)--(3.564,1.616)--(3.567,1.616)--(3.570,1.617)--(3.573,1.617)--(3.576,1.618)%
  --(3.579,1.618)--(3.582,1.619)--(3.585,1.619)--(3.588,1.619)--(3.591,1.620)--(3.594,1.620)%
  --(3.597,1.621)--(3.600,1.621)--(3.603,1.622)--(3.606,1.622)--(3.609,1.623)--(3.612,1.623)%
  --(3.615,1.624)--(3.618,1.624)--(3.621,1.625)--(3.624,1.625)--(3.627,1.626)--(3.630,1.626)%
  --(3.633,1.627)--(3.636,1.627)--(3.639,1.628)--(3.642,1.628)--(3.645,1.629)--(3.648,1.630)%
  --(3.651,1.630)--(3.654,1.631)--(3.657,1.631)--(3.660,1.632)--(3.663,1.632)--(3.666,1.633)%
  --(3.669,1.633)--(3.672,1.634)--(3.675,1.634)--(3.678,1.635)--(3.681,1.636)--(3.684,1.636)%
  --(3.687,1.637)--(3.690,1.637)--(3.693,1.638)--(3.696,1.638)--(3.699,1.639)--(3.702,1.640)%
  --(3.704,1.640)--(3.707,1.641)--(3.710,1.641)--(3.713,1.642)--(3.716,1.643)--(3.719,1.643)%
  --(3.722,1.644)--(3.725,1.644)--(3.728,1.645)--(3.731,1.646)--(3.734,1.646)--(3.737,1.647)%
  --(3.740,1.647)--(3.743,1.648)--(3.746,1.649)--(3.749,1.649)--(3.752,1.650)--(3.755,1.651)%
  --(3.758,1.651)--(3.761,1.652)--(3.764,1.653)--(3.767,1.653)--(3.770,1.654)--(3.773,1.655)%
  --(3.776,1.655)--(3.779,1.656)--(3.782,1.657)--(3.785,1.657)--(3.788,1.658)--(3.791,1.659)%
  --(3.794,1.659)--(3.797,1.660)--(3.800,1.661)--(3.803,1.661)--(3.806,1.662)--(3.809,1.663)%
  --(3.812,1.663)--(3.815,1.664)--(3.818,1.665)--(3.821,1.665)--(3.824,1.666)--(3.827,1.667)%
  --(3.830,1.668)--(3.833,1.668)--(3.836,1.669)--(3.839,1.670)--(3.842,1.670)--(3.845,1.671)%
  --(3.848,1.672)--(3.851,1.673)--(3.854,1.673)--(3.857,1.674)--(3.860,1.675)--(3.863,1.676)%
  --(3.866,1.676)--(3.869,1.677)--(3.872,1.678)--(3.875,1.679)--(3.878,1.679)--(3.881,1.680)%
  --(3.884,1.681)--(3.887,1.682)--(3.890,1.682)--(3.893,1.683)--(3.896,1.684)--(3.899,1.685)%
  --(3.902,1.685)--(3.905,1.686)--(3.908,1.687)--(3.911,1.688)--(3.914,1.689)--(3.916,1.689)%
  --(3.919,1.690)--(3.922,1.691)--(3.925,1.692)--(3.928,1.693)--(3.931,1.693)--(3.934,1.694)%
  --(3.937,1.695)--(3.940,1.696)--(3.943,1.697)--(3.946,1.697)--(3.949,1.698)--(3.952,1.699)%
  --(3.955,1.700)--(3.958,1.701)--(3.961,1.701)--(3.964,1.702)--(3.967,1.703)--(3.970,1.704)%
  --(3.973,1.705)--(3.976,1.706)--(3.979,1.706)--(3.982,1.707)--(3.985,1.708)--(3.988,1.709)%
  --(3.991,1.710)--(3.994,1.711)--(3.997,1.712)--(4.000,1.712)--(4.003,1.713)--(4.006,1.714)%
  --(4.009,1.715)--(4.012,1.716)--(4.015,1.717)--(4.018,1.718)--(4.021,1.718)--(4.024,1.719)%
  --(4.027,1.720)--(4.030,1.721)--(4.033,1.722)--(4.036,1.723)--(4.039,1.724)--(4.042,1.725)%
  --(4.045,1.725)--(4.048,1.726)--(4.051,1.727)--(4.054,1.728)--(4.057,1.729)--(4.060,1.730)%
  --(4.063,1.731)--(4.066,1.732)--(4.069,1.733)--(4.072,1.734)--(4.075,1.735)--(4.078,1.735)%
  --(4.081,1.736)--(4.084,1.737)--(4.087,1.738)--(4.090,1.739)--(4.093,1.740)--(4.096,1.741)%
  --(4.099,1.742)--(4.102,1.743)--(4.105,1.744)--(4.108,1.745)--(4.111,1.746)--(4.114,1.747)%
  --(4.117,1.748)--(4.120,1.748)--(4.123,1.749)--(4.125,1.750)--(4.128,1.751)--(4.131,1.752)%
  --(4.134,1.753)--(4.137,1.754)--(4.140,1.755)--(4.143,1.756)--(4.146,1.757)--(4.149,1.758)%
  --(4.152,1.759)--(4.155,1.760)--(4.158,1.761)--(4.161,1.762)--(4.164,1.763)--(4.167,1.764)%
  --(4.170,1.765)--(4.173,1.766)--(4.176,1.767)--(4.179,1.768)--(4.182,1.769)--(4.185,1.770)%
  --(4.188,1.771)--(4.191,1.772)--(4.194,1.773)--(4.197,1.774)--(4.200,1.775)--(4.203,1.776)%
  --(4.206,1.777)--(4.209,1.778)--(4.212,1.779)--(4.215,1.780)--(4.218,1.781)--(4.221,1.782)%
  --(4.224,1.783)--(4.227,1.784)--(4.230,1.785)--(4.233,1.786)--(4.236,1.787)--(4.239,1.788)%
  --(4.242,1.789)--(4.245,1.790)--(4.248,1.791)--(4.251,1.792)--(4.254,1.793)--(4.257,1.794)%
  --(4.260,1.795)--(4.263,1.796)--(4.266,1.797)--(4.269,1.798)--(4.272,1.799)--(4.275,1.800)%
  --(4.278,1.801)--(4.281,1.802)--(4.284,1.804)--(4.287,1.805)--(4.290,1.806)--(4.293,1.807)%
  --(4.296,1.808)--(4.299,1.809)--(4.302,1.810)--(4.305,1.811)--(4.308,1.812)--(4.311,1.813)%
  --(4.314,1.814)--(4.317,1.815)--(4.320,1.816)--(4.323,1.817)--(4.326,1.818)--(4.329,1.820)%
  --(4.332,1.821)--(4.334,1.822)--(4.337,1.823)--(4.340,1.824)--(4.343,1.825)--(4.346,1.826)%
  --(4.349,1.827)--(4.352,1.828)--(4.355,1.829)--(4.358,1.830)--(4.361,1.832)--(4.364,1.833)%
  --(4.367,1.834)--(4.370,1.835)--(4.373,1.836)--(4.376,1.837)--(4.379,1.838)--(4.382,1.839)%
  --(4.385,1.840)--(4.388,1.842)--(4.391,1.843)--(4.394,1.844)--(4.397,1.845)--(4.400,1.846)%
  --(4.403,1.847)--(4.406,1.848)--(4.409,1.849)--(4.412,1.851)--(4.415,1.852)--(4.418,1.853)%
  --(4.421,1.854)--(4.424,1.855)--(4.427,1.856)--(4.430,1.857)--(4.433,1.858)--(4.436,1.860)%
  --(4.439,1.861)--(4.442,1.862)--(4.445,1.863)--(4.448,1.864)--(4.451,1.865)--(4.454,1.867)%
  --(4.457,1.868)--(4.460,1.869)--(4.463,1.870)--(4.466,1.871)--(4.469,1.872)--(4.472,1.873)%
  --(4.475,1.875)--(4.478,1.876)--(4.481,1.877)--(4.484,1.878)--(4.487,1.879)--(4.490,1.880)%
  --(4.493,1.882)--(4.496,1.883)--(4.499,1.884)--(4.502,1.885)--(4.505,1.886)--(4.508,1.888)%
  --(4.511,1.889)--(4.514,1.890)--(4.517,1.891)--(4.520,1.892)--(4.523,1.893)--(4.526,1.895)%
  --(4.529,1.896)--(4.532,1.897)--(4.535,1.898)--(4.538,1.899)--(4.541,1.901)--(4.543,1.902)%
  --(4.546,1.903)--(4.549,1.904)--(4.552,1.905)--(4.555,1.907)--(4.558,1.908)--(4.561,1.909)%
  --(4.564,1.910)--(4.567,1.911)--(4.570,1.913)--(4.573,1.914)--(4.576,1.915)--(4.579,1.916)%
  --(4.582,1.917)--(4.585,1.919)--(4.588,1.920)--(4.591,1.921)--(4.594,1.922)--(4.597,1.924)%
  --(4.600,1.925)--(4.603,1.926)--(4.606,1.927)--(4.609,1.929)--(4.612,1.930)--(4.615,1.931)%
  --(4.618,1.932)--(4.621,1.933)--(4.624,1.935)--(4.627,1.936)--(4.630,1.937)--(4.633,1.938)%
  --(4.636,1.940)--(4.639,1.941)--(4.642,1.942)--(4.645,1.943)--(4.648,1.945)--(4.651,1.946)%
  --(4.654,1.947)--(4.657,1.948)--(4.660,1.950)--(4.663,1.951)--(4.666,1.952)--(4.669,1.953)%
  --(4.672,1.955)--(4.675,1.956)--(4.678,1.957)--(4.681,1.958)--(4.684,1.960)--(4.687,1.961)%
  --(4.690,1.962)--(4.693,1.963)--(4.696,1.965)--(4.699,1.966)--(4.702,1.967)--(4.705,1.969)%
  --(4.708,1.970)--(4.711,1.971)--(4.714,1.972)--(4.717,1.974)--(4.720,1.975)--(4.723,1.976)%
  --(4.726,1.978)--(4.729,1.979)--(4.732,1.980)--(4.735,1.981)--(4.738,1.983)--(4.741,1.984)%
  --(4.744,1.985)--(4.747,1.987)--(4.750,1.988)--(4.752,1.989)--(4.755,1.990)--(4.758,1.992)%
  --(4.761,1.993)--(4.764,1.994)--(4.767,1.996)--(4.770,1.997)--(4.773,1.998)--(4.776,2.000)%
  --(4.779,2.001)--(4.782,2.002)--(4.785,2.003)--(4.788,2.005)--(4.791,2.006)--(4.794,2.007)%
  --(4.797,2.009)--(4.800,2.010)--(4.803,2.011)--(4.806,2.013)--(4.809,2.014)--(4.812,2.015)%
  --(4.815,2.017)--(4.818,2.018)--(4.821,2.019)--(4.824,2.021)--(4.827,2.022)--(4.830,2.023)%
  --(4.833,2.025)--(4.836,2.026)--(4.839,2.027)--(4.842,2.028)--(4.845,2.030)--(4.848,2.031)%
  --(4.851,2.032)--(4.854,2.034)--(4.857,2.035)--(4.860,2.036)--(4.863,2.038)--(4.866,2.039)%
  --(4.869,2.041)--(4.872,2.042)--(4.875,2.043)--(4.878,2.045)--(4.881,2.046)--(4.884,2.047)%
  --(4.887,2.049)--(4.890,2.050)--(4.893,2.051)--(4.896,2.053)--(4.899,2.054)--(4.902,2.055)%
  --(4.905,2.057)--(4.908,2.058)--(4.911,2.059)--(4.914,2.061)--(4.917,2.062)--(4.920,2.063)%
  --(4.923,2.065)--(4.926,2.066)--(4.929,2.068)--(4.932,2.069)--(4.935,2.070)--(4.938,2.072)%
  --(4.941,2.073)--(4.944,2.074)--(4.947,2.076)--(4.950,2.077)--(4.953,2.078)--(4.956,2.080)%
  --(4.959,2.081)--(4.961,2.083)--(4.964,2.084)--(4.967,2.085)--(4.970,2.087)--(4.973,2.088)%
  --(4.976,2.089)--(4.979,2.091)--(4.982,2.092)--(4.985,2.094)--(4.988,2.095)--(4.991,2.096)%
  --(4.994,2.098)--(4.997,2.099)--(5.000,2.101)--(5.003,2.102)--(5.006,2.103)--(5.009,2.105)%
  --(5.012,2.106)--(5.015,2.107)--(5.018,2.109)--(5.021,2.110)--(5.024,2.112)--(5.027,2.113)%
  --(5.030,2.114)--(5.033,2.116)--(5.036,2.117)--(5.039,2.119)--(5.042,2.120)--(5.045,2.121)%
  --(5.048,2.123)--(5.051,2.124)--(5.054,2.126)--(5.057,2.127)--(5.060,2.128)--(5.063,2.130)%
  --(5.066,2.131)--(5.069,2.133)--(5.072,2.134)--(5.075,2.136)--(5.078,2.137)--(5.081,2.138)%
  --(5.084,2.140)--(5.087,2.141)--(5.090,2.143)--(5.093,2.144)--(5.096,2.145)--(5.099,2.147)%
  --(5.102,2.148)--(5.105,2.150)--(5.108,2.151)--(5.111,2.153)--(5.114,2.154)--(5.117,2.155)%
  --(5.120,2.157)--(5.123,2.158)--(5.126,2.160)--(5.129,2.161)--(5.132,2.163)--(5.135,2.164)%
  --(5.138,2.165)--(5.141,2.167)--(5.144,2.168)--(5.147,2.170)--(5.150,2.171)--(5.153,2.173)%
  --(5.156,2.174)--(5.159,2.175)--(5.162,2.177)--(5.165,2.178)--(5.168,2.180)--(5.171,2.181)%
  --(5.173,2.183)--(5.176,2.184)--(5.179,2.186)--(5.182,2.187)--(5.185,2.188)--(5.188,2.190)%
  --(5.191,2.191)--(5.194,2.193)--(5.197,2.194)--(5.200,2.196)--(5.203,2.197)--(5.206,2.199)%
  --(5.209,2.200)--(5.212,2.201)--(5.215,2.203)--(5.218,2.204)--(5.221,2.206)--(5.224,2.207)%
  --(5.227,2.209)--(5.230,2.210)--(5.233,2.212)--(5.236,2.213)--(5.239,2.215)--(5.242,2.216)%
  --(5.245,2.218)--(5.248,2.219)--(5.251,2.220)--(5.254,2.222)--(5.257,2.223)--(5.260,2.225)%
  --(5.263,2.226)--(5.266,2.228)--(5.269,2.229)--(5.272,2.231)--(5.275,2.232)--(5.278,2.234)%
  --(5.281,2.235)--(5.284,2.237)--(5.287,2.238)--(5.290,2.240)--(5.293,2.241)--(5.296,2.243)%
  --(5.299,2.244)--(5.302,2.245)--(5.305,2.247)--(5.308,2.248)--(5.311,2.250)--(5.314,2.251)%
  --(5.317,2.253)--(5.320,2.254)--(5.323,2.256)--(5.326,2.257)--(5.329,2.259)--(5.332,2.260)%
  --(5.335,2.262)--(5.338,2.263)--(5.341,2.265)--(5.344,2.266)--(5.347,2.268)--(5.350,2.269)%
  --(5.353,2.271)--(5.356,2.272)--(5.359,2.274)--(5.362,2.275)--(5.365,2.277)--(5.368,2.278)%
  --(5.371,2.280)--(5.374,2.281)--(5.377,2.283)--(5.380,2.284)--(5.382,2.286)--(5.385,2.287)%
  --(5.388,2.289)--(5.391,2.290)--(5.394,2.292)--(5.397,2.293)--(5.400,2.295)--(5.403,2.296)%
  --(5.406,2.298)--(5.409,2.299)--(5.412,2.301)--(5.415,2.302)--(5.418,2.304)--(5.421,2.305)%
  --(5.424,2.307)--(5.427,2.308)--(5.430,2.310)--(5.433,2.311)--(5.436,2.313)--(5.439,2.314)%
  --(5.442,2.316)--(5.445,2.317)--(5.448,2.319)--(5.451,2.320)--(5.454,2.322)--(5.457,2.323)%
  --(5.460,2.325)--(5.463,2.326)--(5.466,2.328)--(5.469,2.329)--(5.472,2.331)--(5.475,2.333)%
  --(5.478,2.334)--(5.481,2.336)--(5.484,2.337)--(5.487,2.339)--(5.490,2.340)--(5.493,2.342)%
  --(5.496,2.343)--(5.499,2.345)--(5.502,2.346)--(5.505,2.348)--(5.508,2.349)--(5.511,2.351)%
  --(5.514,2.352)--(5.517,2.354)--(5.520,2.355)--(5.523,2.357)--(5.526,2.359)--(5.529,2.360)%
  --(5.532,2.362)--(5.535,2.363)--(5.538,2.365)--(5.541,2.366)--(5.544,2.368)--(5.547,2.369)%
  --(5.550,2.371)--(5.553,2.372)--(5.556,2.374)--(5.559,2.375)--(5.562,2.377)--(5.565,2.379)%
  --(5.568,2.380)--(5.571,2.382)--(5.574,2.383)--(5.577,2.385)--(5.580,2.386)--(5.583,2.388)%
  --(5.586,2.389)--(5.589,2.391)--(5.591,2.392)--(5.594,2.394)--(5.597,2.396)--(5.600,2.397)%
  --(5.603,2.399)--(5.606,2.400)--(5.609,2.402)--(5.612,2.403)--(5.615,2.405)--(5.618,2.406)%
  --(5.621,2.408)--(5.624,2.410)--(5.627,2.411)--(5.630,2.413)--(5.633,2.414)--(5.636,2.416)%
  --(5.639,2.417)--(5.642,2.419)--(5.645,2.420)--(5.648,2.422)--(5.651,2.424)--(5.654,2.425)%
  --(5.657,2.427)--(5.660,2.428)--(5.663,2.430)--(5.666,2.431)--(5.669,2.433)--(5.672,2.434)%
  --(5.675,2.436)--(5.678,2.438)--(5.681,2.439)--(5.684,2.441)--(5.687,2.442)--(5.690,2.444)%
  --(5.693,2.445)--(5.696,2.447)--(5.699,2.449)--(5.702,2.450)--(5.705,2.452)--(5.708,2.453)%
  --(5.711,2.455)--(5.714,2.456)--(5.717,2.458)--(5.720,2.460)--(5.723,2.461)--(5.726,2.463)%
  --(5.729,2.464)--(5.732,2.466)--(5.735,2.467)--(5.738,2.469)--(5.741,2.471)--(5.744,2.472)%
  --(5.747,2.474)--(5.750,2.475)--(5.753,2.477)--(5.756,2.479)--(5.759,2.480)--(5.762,2.482)%
  --(5.765,2.483)--(5.768,2.485)--(5.771,2.486)--(5.774,2.488)--(5.777,2.490)--(5.780,2.491)%
  --(5.783,2.493)--(5.786,2.494)--(5.789,2.496)--(5.792,2.498)--(5.795,2.499)--(5.798,2.501)%
  --(5.800,2.502)--(5.803,2.504)--(5.806,2.506)--(5.809,2.507)--(5.812,2.509)--(5.815,2.510)%
  --(5.818,2.512)--(5.821,2.513)--(5.824,2.515)--(5.827,2.517)--(5.830,2.518)--(5.833,2.520)%
  --(5.836,2.521)--(5.839,2.523)--(5.842,2.525)--(5.845,2.526)--(5.848,2.528)--(5.851,2.529)%
  --(5.854,2.531)--(5.857,2.533)--(5.860,2.534)--(5.863,2.536)--(5.866,2.537)--(5.869,2.539)%
  --(5.872,2.541)--(5.875,2.542)--(5.878,2.544)--(5.881,2.545)--(5.884,2.547)--(5.887,2.549)%
  --(5.890,2.550)--(5.893,2.552)--(5.896,2.554)--(5.899,2.555)--(5.902,2.557)--(5.905,2.558)%
  --(5.908,2.560)--(5.911,2.562)--(5.914,2.563)--(5.917,2.565)--(5.920,2.566)--(5.923,2.568)%
  --(5.926,2.570)--(5.929,2.571)--(5.932,2.573)--(5.935,2.574)--(5.938,2.576)--(5.941,2.578)%
  --(5.944,2.579)--(5.947,2.581)--(5.950,2.583)--(5.953,2.584)--(5.956,2.586)--(5.959,2.587)%
  --(5.962,2.589)--(5.965,2.591)--(5.968,2.592)--(5.971,2.594)--(5.974,2.595)--(5.977,2.597)%
  --(5.980,2.599)--(5.983,2.600)--(5.986,2.602)--(5.989,2.604)--(5.992,2.605)--(5.995,2.607)%
  --(5.998,2.608)--(6.001,2.610)--(6.004,2.612)--(6.007,2.613)--(6.009,2.615)--(6.012,2.617)%
  --(6.015,2.618)--(6.018,2.620)--(6.021,2.621)--(6.024,2.623)--(6.027,2.625)--(6.030,2.626)%
  --(6.033,2.628)--(6.036,2.630)--(6.039,2.631)--(6.042,2.633)--(6.045,2.635)--(6.048,2.636)%
  --(6.051,2.638)--(6.054,2.639)--(6.057,2.641)--(6.060,2.643)--(6.063,2.644)--(6.066,2.646)%
  --(6.069,2.648)--(6.072,2.649)--(6.075,2.651)--(6.078,2.652)--(6.081,2.654)--(6.084,2.656)%
  --(6.087,2.657)--(6.090,2.659)--(6.093,2.661)--(6.096,2.662)--(6.099,2.664)--(6.102,2.666)%
  --(6.105,2.667)--(6.108,2.669)--(6.111,2.671)--(6.114,2.672)--(6.117,2.674)--(6.120,2.675)%
  --(6.123,2.677)--(6.126,2.679)--(6.129,2.680)--(6.132,2.682)--(6.135,2.684)--(6.138,2.685)%
  --(6.141,2.687)--(6.144,2.689)--(6.147,2.690)--(6.150,2.692)--(6.153,2.694)--(6.156,2.695)%
  --(6.159,2.697)--(6.162,2.699)--(6.165,2.700)--(6.168,2.702)--(6.171,2.703)--(6.174,2.705)%
  --(6.177,2.707)--(6.180,2.708)--(6.183,2.710)--(6.186,2.712)--(6.189,2.713)--(6.192,2.715)%
  --(6.195,2.717)--(6.198,2.718)--(6.201,2.720)--(6.204,2.722)--(6.207,2.723)--(6.210,2.725)%
  --(6.213,2.727)--(6.216,2.728)--(6.218,2.730)--(6.221,2.732)--(6.224,2.733)--(6.227,2.735)%
  --(6.230,2.737)--(6.233,2.738)--(6.236,2.740)--(6.239,2.742)--(6.242,2.743)--(6.245,2.745)%
  --(6.248,2.747)--(6.251,2.748)--(6.254,2.750)--(6.257,2.751)--(6.260,2.753)--(6.263,2.755)%
  --(6.266,2.756)--(6.269,2.758)--(6.272,2.760)--(6.275,2.761)--(6.278,2.763)--(6.281,2.765)%
  --(6.284,2.766)--(6.287,2.768)--(6.290,2.770)--(6.293,2.771)--(6.296,2.773)--(6.299,2.775)%
  --(6.302,2.776)--(6.305,2.778)--(6.308,2.780)--(6.311,2.781)--(6.314,2.783)--(6.317,2.785)%
  --(6.320,2.786)--(6.323,2.788)--(6.326,2.790)--(6.329,2.792)--(6.332,2.793)--(6.335,2.795)%
  --(6.338,2.797)--(6.341,2.798)--(6.344,2.800)--(6.347,2.802)--(6.350,2.803)--(6.353,2.805)%
  --(6.356,2.807)--(6.359,2.808)--(6.362,2.810)--(6.365,2.812)--(6.368,2.813)--(6.371,2.815)%
  --(6.374,2.817)--(6.377,2.818)--(6.380,2.820)--(6.383,2.822)--(6.386,2.823)--(6.389,2.825)%
  --(6.392,2.827)--(6.395,2.828)--(6.398,2.830)--(6.401,2.832)--(6.404,2.833)--(6.407,2.835)%
  --(6.410,2.837)--(6.413,2.838)--(6.416,2.840)--(6.419,2.842)--(6.422,2.844)--(6.425,2.845)%
  --(6.428,2.847)--(6.430,2.849)--(6.433,2.850)--(6.436,2.852)--(6.439,2.854)--(6.442,2.855)%
  --(6.445,2.857)--(6.448,2.859)--(6.451,2.860)--(6.454,2.862)--(6.457,2.864)--(6.460,2.865)%
  --(6.463,2.867)--(6.466,2.869)--(6.469,2.870)--(6.472,2.872)--(6.475,2.874)--(6.478,2.876)%
  --(6.481,2.877)--(6.484,2.879)--(6.487,2.881)--(6.490,2.882)--(6.493,2.884)--(6.496,2.886)%
  --(6.499,2.887)--(6.502,2.889)--(6.505,2.891)--(6.508,2.892)--(6.511,2.894)--(6.514,2.896)%
  --(6.517,2.898)--(6.520,2.899)--(6.523,2.901)--(6.526,2.903)--(6.529,2.904)--(6.532,2.906)%
  --(6.535,2.908)--(6.538,2.909)--(6.541,2.911)--(6.544,2.913)--(6.547,2.915)--(6.550,2.916)%
  --(6.553,2.918)--(6.556,2.920)--(6.559,2.921)--(6.562,2.923)--(6.565,2.925)--(6.568,2.926)%
  --(6.571,2.928)--(6.574,2.930)--(6.577,2.932)--(6.580,2.933)--(6.583,2.935)--(6.586,2.937)%
  --(6.589,2.938)--(6.592,2.940)--(6.595,2.942)--(6.598,2.943)--(6.601,2.945)--(6.604,2.947)%
  --(6.607,2.949)--(6.610,2.950)--(6.613,2.952)--(6.616,2.954)--(6.619,2.955)--(6.622,2.957)%
  --(6.625,2.959)--(6.628,2.960)--(6.631,2.962)--(6.634,2.964)--(6.637,2.966)--(6.639,2.967)%
  --(6.642,2.969)--(6.645,2.971)--(6.648,2.972)--(6.651,2.974)--(6.654,2.976)--(6.657,2.978)%
  --(6.660,2.979)--(6.663,2.981)--(6.666,2.983)--(6.669,2.984)--(6.672,2.986)--(6.675,2.988)%
  --(6.678,2.990)--(6.681,2.991)--(6.684,2.993)--(6.687,2.995)--(6.690,2.996)--(6.693,2.998)%
  --(6.696,3.000)--(6.699,3.002)--(6.702,3.003)--(6.705,3.005)--(6.708,3.007)--(6.711,3.008)%
  --(6.714,3.010)--(6.717,3.012)--(6.720,3.014)--(6.723,3.015)--(6.726,3.017)--(6.729,3.019)%
  --(6.732,3.020)--(6.735,3.022)--(6.738,3.024)--(6.741,3.026)--(6.744,3.027)--(6.747,3.029)%
  --(6.750,3.031)--(6.753,3.032)--(6.756,3.034)--(6.759,3.036)--(6.762,3.038)--(6.765,3.039)%
  --(6.768,3.041)--(6.771,3.043)--(6.774,3.044)--(6.777,3.046)--(6.780,3.048)--(6.783,3.050)%
  --(6.786,3.051)--(6.789,3.053)--(6.792,3.055)--(6.795,3.056)--(6.798,3.058)--(6.801,3.060)%
  --(6.804,3.062)--(6.807,3.063)--(6.810,3.065)--(6.813,3.067)--(6.816,3.069)--(6.819,3.070)%
  --(6.822,3.072)--(6.825,3.074)--(6.828,3.075)--(6.831,3.077)--(6.834,3.079)--(6.837,3.081)%
  --(6.840,3.082)--(6.843,3.084)--(6.846,3.086)--(6.848,3.088)--(6.851,3.089)--(6.854,3.091)%
  --(6.857,3.093)--(6.860,3.094)--(6.863,3.096)--(6.866,3.098)--(6.869,3.100)--(6.872,3.101)%
  --(6.875,3.103)--(6.878,3.105)--(6.881,3.107)--(6.884,3.108)--(6.887,3.110)--(6.890,3.112)%
  --(6.893,3.114)--(6.896,3.115)--(6.899,3.117)--(6.902,3.119)--(6.905,3.120)--(6.908,3.122)%
  --(6.911,3.124)--(6.914,3.126)--(6.917,3.127)--(6.920,3.129)--(6.923,3.131)--(6.926,3.133)%
  --(6.929,3.134)--(6.932,3.136)--(6.935,3.138)--(6.938,3.140)--(6.941,3.141)--(6.944,3.143)%
  --(6.947,3.145)--(6.950,3.146)--(6.953,3.148)--(6.956,3.150)--(6.959,3.152)--(6.962,3.153)%
  --(6.965,3.155)--(6.968,3.157)--(6.971,3.159)--(6.974,3.160)--(6.977,3.162)--(6.980,3.164)%
  --(6.983,3.166)--(6.986,3.167)--(6.989,3.169)--(6.992,3.171)--(6.995,3.173)--(6.998,3.174)%
  --(7.001,3.176)--(7.004,3.178)--(7.007,3.179)--(7.010,3.181)--(7.013,3.183)--(7.016,3.185)%
  --(7.019,3.186)--(7.022,3.188)--(7.025,3.190)--(7.028,3.192)--(7.031,3.193)--(7.034,3.195)%
  --(7.037,3.197)--(7.040,3.199)--(7.043,3.200)--(7.046,3.202)--(7.049,3.204)--(7.052,3.206)%
  --(7.055,3.207)--(7.057,3.209)--(7.060,3.211)--(7.063,3.213)--(7.066,3.214)--(7.069,3.216)%
  --(7.072,3.218)--(7.075,3.220)--(7.078,3.221)--(7.081,3.223)--(7.084,3.225)--(7.087,3.227)%
  --(7.090,3.228)--(7.093,3.230)--(7.096,3.232)--(7.099,3.234)--(7.102,3.235)--(7.105,3.237)%
  --(7.108,3.239)--(7.111,3.241)--(7.114,3.242)--(7.117,3.244)--(7.120,3.246)--(7.123,3.248)%
  --(7.126,3.249)--(7.129,3.251)--(7.132,3.253)--(7.135,3.255)--(7.138,3.256)--(7.141,3.258)%
  --(7.144,3.260)--(7.147,3.262)--(7.150,3.263)--(7.153,3.265)--(7.156,3.267)--(7.159,3.269)%
  --(7.162,3.270)--(7.165,3.272)--(7.168,3.274)--(7.171,3.276)--(7.174,3.277)--(7.177,3.279)%
  --(7.180,3.281)--(7.183,3.283)--(7.186,3.284)--(7.189,3.286)--(7.192,3.288)--(7.195,3.290)%
  --(7.198,3.291)--(7.201,3.293)--(7.204,3.295)--(7.207,3.297)--(7.210,3.298)--(7.213,3.300)%
  --(7.216,3.302)--(7.219,3.304)--(7.222,3.305)--(7.225,3.307)--(7.228,3.309)--(7.231,3.311)%
  --(7.234,3.313)--(7.237,3.314)--(7.240,3.316)--(7.243,3.318)--(7.246,3.320)--(7.249,3.321)%
  --(7.252,3.323)--(7.255,3.325)--(7.258,3.327)--(7.261,3.328)--(7.264,3.330)--(7.266,3.332)%
  --(7.269,3.334)--(7.272,3.335)--(7.275,3.337)--(7.278,3.339)--(7.281,3.341)--(7.284,3.342)%
  --(7.287,3.344)--(7.290,3.346)--(7.293,3.348)--(7.296,3.350)--(7.299,3.351)--(7.302,3.353)%
  --(7.305,3.355)--(7.308,3.357)--(7.311,3.358)--(7.314,3.360)--(7.317,3.362)--(7.320,3.364)%
  --(7.323,3.365)--(7.326,3.367)--(7.329,3.369)--(7.332,3.371)--(7.335,3.372)--(7.338,3.374)%
  --(7.341,3.376)--(7.344,3.378)--(7.347,3.380)--(7.350,3.381)--(7.353,3.383)--(7.356,3.385)%
  --(7.359,3.387)--(7.362,3.388)--(7.365,3.390)--(7.368,3.392)--(7.371,3.394)--(7.374,3.395)%
  --(7.377,3.397)--(7.380,3.399)--(7.383,3.401)--(7.386,3.402)--(7.389,3.404)--(7.392,3.406)%
  --(7.395,3.408)--(7.398,3.410)--(7.401,3.411)--(7.404,3.413)--(7.407,3.415)--(7.410,3.417)%
  --(7.413,3.418)--(7.416,3.420)--(7.419,3.422)--(7.422,3.424)--(7.425,3.426)--(7.428,3.427)%
  --(7.431,3.429)--(7.434,3.431)--(7.437,3.433)--(7.440,3.434)--(7.443,3.436)--(7.446,3.438)%
  --(7.449,3.440)--(7.452,3.441)--(7.455,3.443)--(7.458,3.445)--(7.461,3.447)--(7.464,3.449)%
  --(7.467,3.450)--(7.470,3.452)--(7.473,3.454)--(7.476,3.456)--(7.478,3.457)--(7.481,3.459)%
  --(7.484,3.461)--(7.487,3.463)--(7.490,3.465)--(7.493,3.466)--(7.496,3.468)--(7.499,3.470)%
  --(7.502,3.472)--(7.505,3.473)--(7.508,3.475)--(7.511,3.477)--(7.514,3.479)--(7.517,3.481)%
  --(7.520,3.482)--(7.523,3.484)--(7.526,3.486)--(7.529,3.488)--(7.532,3.489)--(7.535,3.491)%
  --(7.538,3.493)--(7.541,3.495)--(7.544,3.497)--(7.547,3.498)--(7.550,3.500)--(7.553,3.502)%
  --(7.556,3.504)--(7.559,3.505)--(7.562,3.507)--(7.565,3.509)--(7.568,3.511)--(7.571,3.513)%
  --(7.574,3.514)--(7.577,3.516)--(7.580,3.518)--(7.583,3.520)--(7.586,3.521)--(7.589,3.523)%
  --(7.592,3.525)--(7.595,3.527)--(7.598,3.529)--(7.601,3.530)--(7.604,3.532)--(7.607,3.534)%
  --(7.610,3.536)--(7.613,3.537)--(7.616,3.539)--(7.619,3.541)--(7.622,3.543)--(7.625,3.545)%
  --(7.628,3.546)--(7.631,3.548)--(7.634,3.550)--(7.637,3.552)--(7.640,3.554)--(7.643,3.555)%
  --(7.646,3.557)--(7.649,3.559)--(7.652,3.561)--(7.655,3.562)--(7.658,3.564)--(7.661,3.566)%
  --(7.664,3.568)--(7.667,3.570)--(7.670,3.571)--(7.673,3.573)--(7.676,3.575)--(7.679,3.577)%
  --(7.682,3.579)--(7.685,3.580)--(7.687,3.582)--(7.690,3.584)--(7.693,3.586)--(7.696,3.587)%
  --(7.699,3.589)--(7.702,3.591)--(7.705,3.593)--(7.708,3.595)--(7.711,3.596)--(7.714,3.598)%
  --(7.717,3.600)--(7.720,3.602)--(7.723,3.604)--(7.726,3.605)--(7.729,3.607)--(7.732,3.609)%
  --(7.735,3.611)--(7.738,3.613)--(7.741,3.614)--(7.744,3.616)--(7.747,3.618)--(7.750,3.620)%
  --(7.753,3.621)--(7.756,3.623)--(7.759,3.625)--(7.762,3.627)--(7.765,3.629)--(7.768,3.630)%
  --(7.771,3.632)--(7.774,3.634)--(7.777,3.636)--(7.780,3.638)--(7.783,3.639)--(7.786,3.641)%
  --(7.789,3.643)--(7.792,3.645)--(7.795,3.647)--(7.798,3.648)--(7.801,3.650)--(7.804,3.652)%
  --(7.807,3.654)--(7.810,3.656)--(7.813,3.657)--(7.816,3.659)--(7.819,3.661)--(7.822,3.663)%
  --(7.825,3.664)--(7.828,3.666)--(7.831,3.668)--(7.834,3.670)--(7.837,3.672)--(7.840,3.673)%
  --(7.843,3.675)--(7.846,3.677)--(7.849,3.679)--(7.852,3.681)--(7.855,3.682)--(7.858,3.684)%
  --(7.861,3.686)--(7.864,3.688)--(7.867,3.690)--(7.870,3.691)--(7.873,3.693)--(7.876,3.695)%
  --(7.879,3.697)--(7.882,3.699)--(7.885,3.700)--(7.888,3.702)--(7.891,3.704)--(7.894,3.706)%
  --(7.896,3.708)--(7.899,3.709)--(7.902,3.711)--(7.905,3.713)--(7.908,3.715)--(7.911,3.717)%
  --(7.914,3.718)--(7.917,3.720)--(7.920,3.722)--(7.923,3.724)--(7.926,3.726)--(7.929,3.727)%
  --(7.932,3.729)--(7.935,3.731)--(7.938,3.733)--(7.941,3.735)--(7.944,3.736)--(7.947,3.738)%
  --(7.950,3.740)--(7.953,3.742)--(7.956,3.744)--(7.959,3.745)--(7.962,3.747)--(7.965,3.749)%
  --(7.968,3.751)--(7.971,3.753)--(7.974,3.754)--(7.977,3.756)--(7.980,3.758)--(7.983,3.760)%
  --(7.986,3.762)--(7.989,3.763)--(7.992,3.765)--(7.995,3.767)--(7.998,3.769)--(8.001,3.771)%
  --(8.004,3.772)--(8.007,3.774)--(8.010,3.776)--(8.013,3.778)--(8.016,3.780)--(8.019,3.781)%
  --(8.022,3.783)--(8.025,3.785)--(8.028,3.787)--(8.031,3.789)--(8.034,3.790)--(8.037,3.792)%
  --(8.040,3.794)--(8.043,3.796)--(8.046,3.798)--(8.049,3.799)--(8.052,3.801)--(8.055,3.803)%
  --(8.058,3.805)--(8.061,3.807)--(8.064,3.809)--(8.067,3.810)--(8.070,3.812)--(8.073,3.814)%
  --(8.076,3.816)--(8.079,3.818)--(8.082,3.819)--(8.085,3.821)--(8.088,3.823)--(8.091,3.825)%
  --(8.094,3.827)--(8.097,3.828)--(8.100,3.830)--(8.103,3.832)--(8.105,3.834)--(8.108,3.836)%
  --(8.111,3.837)--(8.114,3.839)--(8.117,3.841)--(8.120,3.843)--(8.123,3.845)--(8.126,3.846)%
  --(8.129,3.848)--(8.132,3.850)--(8.135,3.852)--(8.138,3.854)--(8.141,3.855)--(8.144,3.857)%
  --(8.147,3.859)--(8.150,3.861)--(8.153,3.863)--(8.156,3.865)--(8.159,3.866)--(8.162,3.868)%
  --(8.165,3.870)--(8.168,3.872)--(8.171,3.874)--(8.174,3.875)--(8.177,3.877)--(8.180,3.879)%
  --(8.183,3.881)--(8.186,3.883)--(8.189,3.884)--(8.192,3.886)--(8.195,3.888)--(8.198,3.890)%
  --(8.201,3.892)--(8.204,3.894)--(8.207,3.895)--(8.210,3.897)--(8.213,3.899)--(8.216,3.901)%
  --(8.219,3.903)--(8.222,3.904)--(8.225,3.906)--(8.228,3.908)--(8.231,3.910)--(8.234,3.912)%
  --(8.237,3.913)--(8.240,3.915)--(8.243,3.917)--(8.246,3.919)--(8.249,3.921)--(8.252,3.923)%
  --(8.255,3.924)--(8.258,3.926)--(8.261,3.928)--(8.264,3.930)--(8.267,3.932)--(8.270,3.933)%
  --(8.273,3.935)--(8.276,3.937)--(8.279,3.939)--(8.282,3.941)--(8.285,3.942)--(8.288,3.944)%
  --(8.291,3.946)--(8.294,3.948)--(8.297,3.950)--(8.300,3.952)--(8.303,3.953)--(8.306,3.955)%
  --(8.309,3.957)--(8.312,3.959)--(8.314,3.961)--(8.317,3.962)--(8.320,3.964)--(8.323,3.966)%
  --(8.326,3.968)--(8.329,3.970)--(8.332,3.972)--(8.335,3.973)--(8.338,3.975)--(8.341,3.977)%
  --(8.344,3.979)--(8.347,3.981)--(8.350,3.982)--(8.353,3.984)--(8.356,3.986)--(8.359,3.988)%
  --(8.362,3.990)--(8.365,3.992)--(8.368,3.993)--(8.371,3.995)--(8.374,3.997)--(8.377,3.999)%
  --(8.380,4.001)--(8.383,4.002)--(8.386,4.004)--(8.389,4.006)--(8.392,4.008)--(8.395,4.010)%
  --(8.398,4.012)--(8.401,4.013)--(8.404,4.015)--(8.407,4.017)--(8.410,4.019)--(8.413,4.021)%
  --(8.416,4.022)--(8.419,4.024)--(8.422,4.026)--(8.425,4.028)--(8.428,4.030)--(8.431,4.032)%
  --(8.434,4.033)--(8.437,4.035)--(8.440,4.037)--(8.443,4.039)--(8.446,4.041)--(8.449,4.042)%
  --(8.452,4.044)--(8.455,4.046)--(8.458,4.048)--(8.461,4.050)--(8.464,4.052)--(8.467,4.053)%
  --(8.470,4.055)--(8.473,4.057)--(8.476,4.059)--(8.479,4.061)--(8.482,4.063)--(8.485,4.064)%
  --(8.488,4.066)--(8.491,4.068)--(8.494,4.070)--(8.497,4.072)--(8.500,4.073)--(8.503,4.075)%
  --(8.506,4.077)--(8.509,4.079)--(8.512,4.081)--(8.515,4.083)--(8.518,4.084)--(8.521,4.086)%
  --(8.523,4.088)--(8.526,4.090)--(8.529,4.092)--(8.532,4.093)--(8.535,4.095)--(8.538,4.097)%
  --(8.541,4.099)--(8.544,4.101)--(8.547,4.103)--(8.550,4.104)--(8.553,4.106)--(8.556,4.108)%
  --(8.559,4.110)--(8.562,4.112)--(8.565,4.114)--(8.568,4.115)--(8.571,4.117)--(8.574,4.119)%
  --(8.577,4.121)--(8.580,4.123)--(8.583,4.125)--(8.586,4.126)--(8.589,4.128)--(8.592,4.130)%
  --(8.595,4.132)--(8.598,4.134)--(8.601,4.135)--(8.604,4.137)--(8.607,4.139)--(8.610,4.141)%
  --(8.613,4.143)--(8.616,4.145)--(8.619,4.146)--(8.622,4.148)--(8.625,4.150)--(8.628,4.152)%
  --(8.631,4.154)--(8.634,4.156)--(8.637,4.157)--(8.640,4.159)--(8.643,4.161)--(8.646,4.163)%
  --(8.649,4.165)--(8.652,4.167)--(8.655,4.168)--(8.658,4.170)--(8.661,4.172)--(8.664,4.174)%
  --(8.667,4.176)--(8.670,4.178)--(8.673,4.179)--(8.676,4.181)--(8.679,4.183)--(8.682,4.185)%
  --(8.685,4.187)--(8.688,4.188)--(8.691,4.190)--(8.694,4.192)--(8.697,4.194)--(8.700,4.196)%
  --(8.703,4.198)--(8.706,4.199)--(8.709,4.201)--(8.712,4.203)--(8.715,4.205)--(8.718,4.207)%
  --(8.721,4.209)--(8.724,4.210)--(8.727,4.212)--(8.730,4.214)--(8.733,4.216)--(8.735,4.218)%
  --(8.738,4.220)--(8.741,4.221)--(8.744,4.223)--(8.747,4.225)--(8.750,4.227)--(8.753,4.229)%
  --(8.756,4.231)--(8.759,4.232)--(8.762,4.234)--(8.765,4.236)--(8.768,4.238)--(8.771,4.240)%
  --(8.774,4.242)--(8.777,4.243)--(8.780,4.245)--(8.783,4.247)--(8.786,4.249)--(8.789,4.251)%
  --(8.792,4.253)--(8.795,4.254)--(8.798,4.256)--(8.801,4.258)--(8.804,4.260)--(8.807,4.262)%
  --(8.810,4.264)--(8.813,4.265)--(8.816,4.267)--(8.819,4.269)--(8.822,4.271)--(8.825,4.273)%
  --(8.828,4.275)--(8.831,4.276)--(8.834,4.278)--(8.837,4.280)--(8.840,4.282)--(8.843,4.284)%
  --(8.846,4.286)--(8.849,4.287)--(8.852,4.289)--(8.855,4.291)--(8.858,4.293)--(8.861,4.295)%
  --(8.864,4.297)--(8.867,4.298)--(8.870,4.300)--(8.873,4.302)--(8.876,4.304)--(8.879,4.306)%
  --(8.882,4.308)--(8.885,4.309)--(8.888,4.311)--(8.891,4.313)--(8.894,4.315)--(8.897,4.317)%
  --(8.900,4.319)--(8.903,4.320)--(8.906,4.322)--(8.909,4.324)--(8.912,4.326)--(8.915,4.328)%
  --(8.918,4.330)--(8.921,4.331)--(8.924,4.333)--(8.927,4.335)--(8.930,4.337)--(8.933,4.339)%
  --(8.936,4.341)--(8.939,4.342)--(8.942,4.344)--(8.944,4.346)--(8.947,4.348)--(8.950,4.350)%
  --(8.953,4.352)--(8.956,4.353)--(8.959,4.355)--(8.962,4.357)--(8.965,4.359)--(8.968,4.361)%
  --(8.971,4.363)--(8.974,4.365)--(8.977,4.366)--(8.980,4.368)--(8.983,4.370)--(8.986,4.372)%
  --(8.989,4.374)--(8.992,4.376)--(8.995,4.377)--(8.998,4.379)--(9.001,4.381)--(9.004,4.383)%
  --(9.007,4.385)--(9.010,4.387)--(9.013,4.388)--(9.016,4.390)--(9.019,4.392)--(9.022,4.394)%
  --(9.025,4.396)--(9.028,4.398)--(9.031,4.399)--(9.034,4.401)--(9.037,4.403)--(9.040,4.405)%
  --(9.043,4.407)--(9.046,4.409)--(9.049,4.410)--(9.052,4.412)--(9.055,4.414)--(9.058,4.416)%
  --(9.061,4.418)--(9.064,4.420)--(9.067,4.422)--(9.070,4.423)--(9.073,4.425)--(9.076,4.427)%
  --(9.079,4.429)--(9.082,4.431)--(9.085,4.433)--(9.088,4.434)--(9.091,4.436)--(9.094,4.438)%
  --(9.097,4.440)--(9.100,4.442)--(9.103,4.444)--(9.106,4.445)--(9.109,4.447)--(9.112,4.449)%
  --(9.115,4.451)--(9.118,4.453)--(9.121,4.455)--(9.124,4.456)--(9.127,4.458)--(9.130,4.460)%
  --(9.133,4.462)--(9.136,4.464)--(9.139,4.466)--(9.142,4.468)--(9.145,4.469)--(9.148,4.471)%
  --(9.151,4.473)--(9.153,4.475)--(9.156,4.477)--(9.159,4.479)--(9.162,4.480)--(9.165,4.482)%
  --(9.168,4.484)--(9.171,4.486)--(9.174,4.488)--(9.177,4.490)--(9.180,4.492)--(9.183,4.493)%
  --(9.186,4.495)--(9.189,4.497)--(9.192,4.499)--(9.195,4.501)--(9.198,4.503)--(9.201,4.504)%
  --(9.204,4.506)--(9.207,4.508)--(9.210,4.510)--(9.213,4.512)--(9.216,4.514)--(9.219,4.515)%
  --(9.222,4.517)--(9.225,4.519)--(9.228,4.521)--(9.231,4.523)--(9.234,4.525)--(9.237,4.527)%
  --(9.240,4.528)--(9.243,4.530)--(9.246,4.532)--(9.249,4.534)--(9.252,4.536)--(9.255,4.538)%
  --(9.258,4.539)--(9.261,4.541)--(9.264,4.543)--(9.267,4.545)--(9.270,4.547)--(9.273,4.549)%
  --(9.276,4.551)--(9.279,4.552)--(9.282,4.554)--(9.285,4.556)--(9.288,4.558)--(9.291,4.560)%
  --(9.294,4.562)--(9.297,4.563)--(9.300,4.565)--(9.303,4.567)--(9.306,4.569)--(9.309,4.571)%
  --(9.312,4.573)--(9.315,4.575)--(9.318,4.576)--(9.321,4.578)--(9.324,4.580)--(9.327,4.582)%
  --(9.330,4.584)--(9.333,4.586)--(9.336,4.587)--(9.339,4.589)--(9.342,4.591)--(9.345,4.593)%
  --(9.348,4.595)--(9.351,4.597)--(9.354,4.599)--(9.357,4.600)--(9.360,4.602)--(9.362,4.604)%
  --(9.365,4.606)--(9.368,4.608)--(9.371,4.610)--(9.374,4.611)--(9.377,4.613)--(9.380,4.615)%
  --(9.383,4.617)--(9.386,4.619)--(9.389,4.621)--(9.392,4.623)--(9.395,4.624)--(9.398,4.626)%
  --(9.401,4.628)--(9.404,4.630)--(9.407,4.632)--(9.410,4.634)--(9.413,4.635)--(9.416,4.637)%
  --(9.419,4.639)--(9.422,4.641)--(9.425,4.643)--(9.428,4.645)--(9.431,4.647)--(9.434,4.648)%
  --(9.437,4.650)--(9.440,4.652)--(9.443,4.654)--(9.446,4.656)--(9.449,4.658)--(9.452,4.660)%
  --(9.455,4.661)--(9.458,4.663)--(9.461,4.665)--(9.464,4.667)--(9.467,4.669)--(9.470,4.671)%
  --(9.473,4.672)--(9.476,4.674)--(9.479,4.676)--(9.482,4.678)--(9.485,4.680)--(9.488,4.682)%
  --(9.491,4.684)--(9.494,4.685)--(9.497,4.687)--(9.500,4.689)--(9.503,4.691)--(9.506,4.693)%
  --(9.509,4.695)--(9.512,4.697)--(9.515,4.698)--(9.518,4.700)--(9.521,4.702)--(9.524,4.704)%
  --(9.527,4.706)--(9.530,4.708)--(9.533,4.709)--(9.536,4.711)--(9.539,4.713)--(9.542,4.715)%
  --(9.545,4.717)--(9.548,4.719)--(9.551,4.721)--(9.554,4.722)--(9.557,4.724)--(9.560,4.726)%
  --(9.563,4.728)--(9.566,4.730)--(9.569,4.732)--(9.571,4.734)--(9.574,4.735)--(9.577,4.737)%
  --(9.580,4.739)--(9.583,4.741)--(9.586,4.743)--(9.589,4.745)--(9.592,4.747)--(9.595,4.748)%
  --(9.598,4.750)--(9.601,4.752)--(9.604,4.754)--(9.607,4.756)--(9.610,4.758)--(9.613,4.759)%
  --(9.616,4.761)--(9.619,4.763)--(9.622,4.765)--(9.625,4.767)--(9.628,4.769)--(9.631,4.771)%
  --(9.634,4.772)--(9.637,4.774)--(9.640,4.776)--(9.643,4.778)--(9.646,4.780)--(9.649,4.782)%
  --(9.652,4.784)--(9.655,4.785)--(9.658,4.787)--(9.661,4.789)--(9.664,4.791)--(9.667,4.793)%
  --(9.670,4.795)--(9.673,4.797)--(9.676,4.798)--(9.679,4.800)--(9.682,4.802)--(9.685,4.804)%
  --(9.688,4.806)--(9.691,4.808)--(9.694,4.810)--(9.697,4.811)--(9.700,4.813)--(9.703,4.815)%
  --(9.706,4.817)--(9.709,4.819)--(9.712,4.821)--(9.715,4.823)--(9.718,4.824)--(9.721,4.826)%
  --(9.724,4.828)--(9.727,4.830)--(9.730,4.832)--(9.733,4.834)--(9.736,4.836)--(9.739,4.837)%
  --(9.742,4.839)--(9.745,4.841)--(9.748,4.843)--(9.751,4.845)--(9.754,4.847)--(9.757,4.849)%
  --(9.760,4.850)--(9.763,4.852)--(9.766,4.854)--(9.769,4.856)--(9.772,4.858)--(9.775,4.860)%
  --(9.778,4.861)--(9.780,4.863)--(9.783,4.865)--(9.786,4.867)--(9.789,4.869)--(9.792,4.871)%
  --(9.795,4.873)--(9.798,4.874)--(9.801,4.876)--(9.804,4.878)--(9.807,4.880)--(9.810,4.882)%
  --(9.813,4.884)--(9.816,4.886)--(9.819,4.887)--(9.822,4.889)--(9.825,4.891)--(9.828,4.893)%
  --(9.831,4.895)--(9.834,4.897)--(9.837,4.899)--(9.840,4.900)--(9.843,4.902)--(9.846,4.904)%
  --(9.849,4.906)--(9.852,4.908)--(9.855,4.910)--(9.858,4.912)--(9.861,4.913)--(9.864,4.915)%
  --(9.867,4.917)--(9.870,4.919)--(9.873,4.921)--(9.876,4.923)--(9.879,4.925)--(9.882,4.926)%
  --(9.885,4.928)--(9.888,4.930)--(9.891,4.932)--(9.894,4.934)--(9.897,4.936)--(9.900,4.938)%
  --(9.903,4.940)--(9.906,4.941)--(9.909,4.943)--(9.912,4.945)--(9.915,4.947)--(9.918,4.949)%
  --(9.921,4.951)--(9.924,4.953)--(9.927,4.954)--(9.930,4.956)--(9.933,4.958)--(9.936,4.960)%
  --(9.939,4.962)--(9.942,4.964)--(9.945,4.966)--(9.948,4.967)--(9.951,4.969)--(9.954,4.971)%
  --(9.957,4.973)--(9.960,4.975)--(9.963,4.977)--(9.966,4.979)--(9.969,4.980)--(9.972,4.982)%
  --(9.975,4.984)--(9.978,4.986)--(9.981,4.988)--(9.984,4.990)--(9.987,4.992)--(9.990,4.993)%
  --(9.992,4.995)--(9.995,4.997)--(9.998,4.999)--(10.001,5.001)--(10.004,5.003)--(10.007,5.005)%
  --(10.010,5.006)--(10.013,5.008)--(10.016,5.010)--(10.019,5.012)--(10.022,5.014)--(10.025,5.016)%
  --(10.028,5.018)--(10.031,5.019)--(10.034,5.021)--(10.037,5.023)--(10.040,5.025)--(10.043,5.027)%
  --(10.046,5.029)--(10.049,5.031)--(10.052,5.032)--(10.055,5.034)--(10.058,5.036)--(10.061,5.038)%
  --(10.064,5.040)--(10.067,5.042)--(10.070,5.044)--(10.073,5.046)--(10.076,5.047)--(10.079,5.049)%
  --(10.082,5.051)--(10.085,5.053)--(10.088,5.055)--(10.091,5.057)--(10.094,5.059)--(10.097,5.060)%
  --(10.100,5.062)--(10.103,5.064)--(10.106,5.066)--(10.109,5.068)--(10.112,5.070)--(10.115,5.072)%
  --(10.118,5.073)--(10.121,5.075)--(10.124,5.077)--(10.127,5.079)--(10.130,5.081)--(10.133,5.083)%
  --(10.136,5.085)--(10.139,5.086)--(10.142,5.088)--(10.145,5.090)--(10.148,5.092)--(10.151,5.094)%
  --(10.154,5.096)--(10.157,5.098)--(10.160,5.100)--(10.163,5.101)--(10.166,5.103)--(10.169,5.105)%
  --(10.172,5.107)--(10.175,5.109)--(10.178,5.111)--(10.181,5.113)--(10.184,5.114)--(10.187,5.116)%
  --(10.190,5.118)--(10.193,5.120)--(10.196,5.122)--(10.199,5.124)--(10.201,5.126)--(10.204,5.127)%
  --(10.207,5.129)--(10.210,5.131)--(10.213,5.133)--(10.216,5.135)--(10.219,5.137)--(10.222,5.139)%
  --(10.225,5.141)--(10.228,5.142)--(10.231,5.144)--(10.234,5.146)--(10.237,5.148)--(10.240,5.150)%
  --(10.243,5.152)--(10.246,5.154)--(10.249,5.155)--(10.252,5.157)--(10.255,5.159)--(10.258,5.161)%
  --(10.261,5.163)--(10.264,5.165)--(10.267,5.167)--(10.270,5.168)--(10.273,5.170)--(10.276,5.172)%
  --(10.279,5.174)--(10.282,5.176)--(10.285,5.178)--(10.288,5.180)--(10.291,5.182)--(10.294,5.183)%
  --(10.297,5.185)--(10.300,5.187)--(10.303,5.189)--(10.306,5.191)--(10.309,5.193)--(10.312,5.195)%
  --(10.315,5.196)--(10.318,5.198)--(10.321,5.200)--(10.324,5.202)--(10.327,5.204)--(10.330,5.206)%
  --(10.333,5.208)--(10.336,5.210)--(10.339,5.211)--(10.342,5.213)--(10.345,5.215)--(10.348,5.217)%
  --(10.351,5.219)--(10.354,5.221)--(10.357,5.223)--(10.360,5.224)--(10.363,5.226)--(10.366,5.228)%
  --(10.369,5.230)--(10.372,5.232)--(10.375,5.234)--(10.378,5.236)--(10.381,5.237)--(10.384,5.239)%
  --(10.387,5.241)--(10.390,5.243)--(10.393,5.245)--(10.396,5.247)--(10.399,5.249)--(10.402,5.251)%
  --(10.405,5.252)--(10.408,5.254)--(10.410,5.256)--(10.413,5.258)--(10.416,5.260)--(10.419,5.262)%
  --(10.422,5.264)--(10.425,5.265)--(10.428,5.267)--(10.431,5.269)--(10.434,5.271)--(10.437,5.273)%
  --(10.440,5.275)--(10.443,5.277)--(10.446,5.279)--(10.449,5.280)--(10.452,5.282)--(10.455,5.284)%
  --(10.458,5.286)--(10.461,5.288)--(10.464,5.290)--(10.467,5.292)--(10.470,5.294)--(10.473,5.295)%
  --(10.476,5.297)--(10.479,5.299)--(10.482,5.301)--(10.485,5.303)--(10.488,5.305)--(10.491,5.307)%
  --(10.494,5.308)--(10.497,5.310)--(10.500,5.312)--(10.503,5.314)--(10.506,5.316)--(10.509,5.318)%
  --(10.512,5.320)--(10.515,5.322)--(10.518,5.323)--(10.521,5.325)--(10.524,5.327)--(10.527,5.329)%
  --(10.530,5.331)--(10.533,5.333)--(10.536,5.335)--(10.539,5.336)--(10.542,5.338)--(10.545,5.340)%
  --(10.548,5.342)--(10.551,5.344)--(10.554,5.346)--(10.557,5.348)--(10.560,5.350)--(10.563,5.351)%
  --(10.566,5.353)--(10.569,5.355)--(10.572,5.357)--(10.575,5.359)--(10.578,5.361)--(10.581,5.363)%
  --(10.584,5.364)--(10.587,5.366)--(10.590,5.368)--(10.593,5.370)--(10.596,5.372)--(10.599,5.374)%
  --(10.602,5.376)--(10.605,5.378)--(10.608,5.379)--(10.611,5.381)--(10.614,5.383)--(10.617,5.385)%
  --(10.619,5.387)--(10.622,5.389)--(10.625,5.391)--(10.628,5.393)--(10.631,5.394)--(10.634,5.396)%
  --(10.637,5.398)--(10.640,5.400)--(10.643,5.402)--(10.646,5.404)--(10.649,5.406)--(10.652,5.407)%
  --(10.655,5.409)--(10.658,5.411)--(10.661,5.413)--(10.664,5.415)--(10.667,5.417)--(10.670,5.419)%
  --(10.673,5.421)--(10.676,5.422)--(10.679,5.424)--(10.682,5.426)--(10.685,5.428)--(10.688,5.430)%
  --(10.691,5.432)--(10.694,5.434)--(10.697,5.436)--(10.700,5.437)--(10.703,5.439)--(10.706,5.441)%
  --(10.709,5.443)--(10.712,5.445)--(10.715,5.447)--(10.718,5.449)--(10.721,5.451)--(10.724,5.452)%
  --(10.727,5.454)--(10.730,5.456)--(10.733,5.458)--(10.736,5.460)--(10.739,5.462)--(10.742,5.464)%
  --(10.745,5.465)--(10.748,5.467)--(10.751,5.469)--(10.754,5.471)--(10.757,5.473)--(10.760,5.475)%
  --(10.763,5.477)--(10.766,5.479)--(10.769,5.480)--(10.772,5.482)--(10.775,5.484)--(10.778,5.486)%
  --(10.781,5.488)--(10.784,5.490)--(10.787,5.492)--(10.790,5.494)--(10.793,5.495)--(10.796,5.497)%
  --(10.799,5.499)--(10.802,5.501)--(10.805,5.503)--(10.808,5.505)--(10.811,5.507)--(10.814,5.509)%
  --(10.817,5.510)--(10.820,5.512)--(10.823,5.514)--(10.826,5.516)--(10.828,5.518)--(10.831,5.520)%
  --(10.834,5.522)--(10.837,5.524)--(10.840,5.525)--(10.843,5.527)--(10.846,5.529)--(10.849,5.531)%
  --(10.852,5.533)--(10.855,5.535)--(10.858,5.537)--(10.861,5.538)--(10.864,5.540)--(10.867,5.542)%
  --(10.870,5.544)--(10.873,5.546)--(10.876,5.548)--(10.879,5.550)--(10.882,5.552)--(10.885,5.553)%
  --(10.888,5.555)--(10.891,5.557)--(10.894,5.559)--(10.897,5.561)--(10.900,5.563)--(10.903,5.565)%
  --(10.906,5.567)--(10.909,5.568)--(10.912,5.570)--(10.915,5.572)--(10.918,5.574)--(10.921,5.576)%
  --(10.924,5.578)--(10.927,5.580)--(10.930,5.582)--(10.933,5.583)--(10.936,5.585)--(10.939,5.587)%
  --(10.942,5.589)--(10.945,5.591)--(10.948,5.593)--(10.951,5.595)--(10.954,5.597)--(10.957,5.598)%
  --(10.960,5.600)--(10.963,5.602)--(10.966,5.604)--(10.969,5.606)--(10.972,5.608)--(10.975,5.610)%
  --(10.978,5.612)--(10.981,5.613)--(10.984,5.615)--(10.987,5.617)--(10.990,5.619)--(10.993,5.621)%
  --(10.996,5.623)--(10.999,5.625)--(11.002,5.627)--(11.005,5.628)--(11.008,5.630)--(11.011,5.632)%
  --(11.014,5.634)--(11.017,5.636)--(11.020,5.638)--(11.023,5.640)--(11.026,5.642)--(11.029,5.643)%
  --(11.032,5.645)--(11.035,5.647)--(11.037,5.649)--(11.040,5.651)--(11.043,5.653)--(11.046,5.655)%
  --(11.049,5.657)--(11.052,5.658)--(11.055,5.660)--(11.058,5.662)--(11.061,5.664)--(11.064,5.666)%
  --(11.067,5.668)--(11.070,5.670)--(11.073,5.672)--(11.076,5.673)--(11.079,5.675)--(11.082,5.677)%
  --(11.085,5.679)--(11.088,5.681)--(11.091,5.683)--(11.094,5.685)--(11.097,5.687)--(11.100,5.688)%
  --(11.103,5.690)--(11.106,5.692)--(11.109,5.694)--(11.112,5.696)--(11.115,5.698)--(11.118,5.700)%
  --(11.121,5.702)--(11.124,5.703)--(11.127,5.705)--(11.130,5.707)--(11.133,5.709)--(11.136,5.711)%
  --(11.139,5.713)--(11.142,5.715)--(11.145,5.717)--(11.148,5.718)--(11.151,5.720)--(11.154,5.722)%
  --(11.157,5.724)--(11.160,5.726)--(11.163,5.728)--(11.166,5.730)--(11.169,5.732)--(11.172,5.733)%
  --(11.175,5.735)--(11.178,5.737)--(11.181,5.739)--(11.184,5.741)--(11.187,5.743)--(11.190,5.745)%
  --(11.193,5.747)--(11.196,5.748)--(11.199,5.750)--(11.202,5.752)--(11.205,5.754)--(11.208,5.756)%
  --(11.211,5.758)--(11.214,5.760)--(11.217,5.762)--(11.220,5.763)--(11.223,5.765)--(11.226,5.767)%
  --(11.229,5.769)--(11.232,5.771)--(11.235,5.773)--(11.238,5.775)--(11.241,5.777)--(11.244,5.778)%
  --(11.247,5.780)--(11.249,5.782)--(11.252,5.784)--(11.255,5.786)--(11.258,5.788)--(11.261,5.790)%
  --(11.264,5.792)--(11.267,5.793)--(11.270,5.795)--(11.273,5.797)--(11.276,5.799)--(11.279,5.801)%
  --(11.282,5.803)--(11.285,5.805)--(11.288,5.807)--(11.291,5.808)--(11.294,5.810)--(11.297,5.812)%
  --(11.300,5.814)--(11.303,5.816)--(11.306,5.818)--(11.309,5.820)--(11.312,5.822)--(11.315,5.823)%
  --(11.318,5.825)--(11.321,5.827)--(11.324,5.829)--(11.327,5.831)--(11.330,5.833)--(11.333,5.835)%
  --(11.336,5.837)--(11.339,5.839)--(11.342,5.840)--(11.345,5.842)--(11.348,5.844)--(11.351,5.846)%
  --(11.354,5.848)--(11.357,5.850)--(11.360,5.852)--(11.363,5.854)--(11.366,5.855)--(11.369,5.857)%
  --(11.372,5.859)--(11.375,5.861)--(11.378,5.863)--(11.381,5.865)--(11.384,5.867)--(11.387,5.869)%
  --(11.390,5.870)--(11.393,5.872)--(11.396,5.874)--(11.399,5.876)--(11.402,5.878)--(11.405,5.880)%
  --(11.408,5.882)--(11.411,5.884)--(11.414,5.885)--(11.417,5.887)--(11.420,5.889)--(11.423,5.891)%
  --(11.426,5.893)--(11.429,5.895)--(11.432,5.897)--(11.435,5.899)--(11.438,5.901)--(11.441,5.902)%
  --(11.444,5.904)--(11.447,5.906)--(11.450,5.908)--(11.453,5.910)--(11.456,5.912)--(11.458,5.914)%
  --(11.461,5.916)--(11.464,5.917)--(11.467,5.919)--(11.470,5.921)--(11.473,5.923)--(11.476,5.925)%
  --(11.479,5.927)--(11.482,5.929)--(11.485,5.931)--(11.488,5.932)--(11.491,5.934)--(11.494,5.936)%
  --(11.497,5.938)--(11.500,5.940)--(11.503,5.942)--(11.506,5.944)--(11.509,5.946)--(11.512,5.947)%
  --(11.515,5.949)--(11.518,5.951)--(11.521,5.953)--(11.524,5.955)--(11.527,5.957)--(11.530,5.959)%
  --(11.533,5.961)--(11.536,5.963)--(11.539,5.964)--(11.542,5.966)--(11.545,5.968)--(11.548,5.970)%
  --(11.551,5.972)--(11.554,5.974)--(11.557,5.976)--(11.560,5.978)--(11.563,5.979)--(11.566,5.981)%
  --(11.569,5.983)--(11.572,5.985)--(11.575,5.987)--(11.578,5.989)--(11.581,5.991)--(11.584,5.993)%
  --(11.587,5.994)--(11.590,5.996)--(11.593,5.998)--(11.596,6.000)--(11.599,6.002)--(11.602,6.004)%
  --(11.605,6.006)--(11.608,6.008)--(11.611,6.010)--(11.614,6.011)--(11.617,6.013)--(11.620,6.015)%
  --(11.623,6.017)--(11.626,6.019)--(11.629,6.021)--(11.632,6.023)--(11.635,6.025)--(11.638,6.026)%
  --(11.641,6.028)--(11.644,6.030)--(11.647,6.032)--(11.650,6.034)--(11.653,6.036)--(11.656,6.038)%
  --(11.659,6.040)--(11.662,6.041)--(11.665,6.043)--(11.667,6.045)--(11.670,6.047)--(11.673,6.049)%
  --(11.676,6.051)--(11.679,6.053)--(11.682,6.055)--(11.685,6.057)--(11.688,6.058)--(11.691,6.060)%
  --(11.694,6.062)--(11.697,6.064)--(11.700,6.066)--(11.703,6.068)--(11.706,6.070)--(11.709,6.072)%
  --(11.712,6.073)--(11.715,6.075)--(11.718,6.077)--(11.721,6.079)--(11.724,6.081)--(11.727,6.083)%
  --(11.730,6.085)--(11.733,6.087)--(11.736,6.089)--(11.739,6.090)--(11.742,6.092)--(11.745,6.094)%
  --(11.748,6.096)--(11.751,6.098)--(11.754,6.100)--(11.757,6.102)--(11.760,6.104)--(11.763,6.105)%
  --(11.766,6.107)--(11.769,6.109)--(11.772,6.111)--(11.775,6.113)--(11.778,6.115)--(11.781,6.117)%
  --(11.784,6.119)--(11.787,6.120)--(11.790,6.122)--(11.793,6.124)--(11.796,6.126)--(11.799,6.128)%
  --(11.802,6.130)--(11.805,6.132)--(11.808,6.134)--(11.811,6.136)--(11.814,6.137)--(11.817,6.139)%
  --(11.820,6.141)--(11.823,6.143)--(11.826,6.145)--(11.829,6.147)--(11.832,6.149)--(11.835,6.151)%
  --(11.838,6.152)--(11.841,6.154)--(11.844,6.156)--(11.847,6.158)--(11.850,6.160)--(11.853,6.162)%
  --(11.856,6.164)--(11.859,6.166)--(11.862,6.168)--(11.865,6.169)--(11.868,6.171)--(11.871,6.173)%
  --(11.874,6.175)--(11.876,6.177)--(11.879,6.179)--(11.882,6.181)--(11.885,6.183)--(11.888,6.184)%
  --(11.891,6.186)--(11.894,6.188)--(11.897,6.190)--(11.900,6.192)--(11.903,6.194)--(11.906,6.196)%
  --(11.909,6.198)--(11.912,6.200)--(11.915,6.201)--(11.918,6.203)--(11.921,6.205)--(11.924,6.207)%
  --(11.927,6.209)--(11.930,6.211)--(11.933,6.213)--(11.936,6.215)--(11.939,6.217)--(11.942,6.218)%
  --(11.945,6.220)--(11.948,6.222)--(11.951,6.224)--(11.954,6.226)--(11.957,6.228)--(11.960,6.230)%
  --(11.963,6.232)--(11.966,6.233)--(11.969,6.235)--(11.972,6.237)--(11.975,6.239)--(11.978,6.241)%
  --(11.981,6.243)--(11.984,6.245)--(11.987,6.247)--(11.990,6.249)--(11.993,6.250)--(11.996,6.252)%
  --(11.999,6.254)--(12.002,6.256)--(12.005,6.258)--(12.008,6.260)--(12.011,6.262)--(12.014,6.264)%
  --(12.017,6.265)--(12.020,6.267)--(12.023,6.269)--(12.026,6.271)--(12.029,6.273)--(12.032,6.275)%
  --(12.035,6.277)--(12.038,6.279)--(12.041,6.281)--(12.044,6.282)--(12.047,6.284)--(12.050,6.286)%
  --(12.053,6.288)--(12.056,6.290)--(12.059,6.292)--(12.062,6.294)--(12.065,6.296)--(12.068,6.297)%
  --(12.071,6.299)--(12.074,6.301)--(12.077,6.303)--(12.080,6.305)--(12.083,6.307)--(12.085,6.309)%
  --(12.088,6.311)--(12.091,6.313)--(12.094,6.314)--(12.097,6.316)--(12.100,6.318)--(12.103,6.320)%
  --(12.106,6.322)--(12.109,6.324)--(12.112,6.326)--(12.115,6.328)--(12.118,6.330)--(12.121,6.331)%
  --(12.124,6.333)--(12.127,6.335)--(12.130,6.337)--(12.133,6.339)--(12.136,6.341)--(12.139,6.343)%
  --(12.142,6.345)--(12.145,6.346)--(12.148,6.348)--(12.151,6.350)--(12.154,6.352)--(12.157,6.354)%
  --(12.160,6.356)--(12.163,6.358)--(12.166,6.360)--(12.169,6.362)--(12.172,6.363)--(12.175,6.365)%
  --(12.178,6.367)--(12.181,6.369)--(12.184,6.371)--(12.187,6.373)--(12.190,6.375)--(12.193,6.377)%
  --(12.196,6.379)--(12.199,6.380)--(12.202,6.382)--(12.205,6.384)--(12.208,6.386)--(12.211,6.388)%
  --(12.214,6.390)--(12.217,6.392)--(12.220,6.394)--(12.223,6.395)--(12.226,6.397)--(12.229,6.399)%
  --(12.232,6.401)--(12.235,6.403)--(12.238,6.405)--(12.241,6.407)--(12.244,6.409)--(12.247,6.411)%
  --(12.250,6.412)--(12.253,6.414)--(12.256,6.416)--(12.259,6.418)--(12.262,6.420)--(12.265,6.422)%
  --(12.268,6.424)--(12.271,6.426)--(12.274,6.428)--(12.277,6.429)--(12.280,6.431)--(12.283,6.433)%
  --(12.286,6.435)--(12.289,6.437)--(12.292,6.439)--(12.295,6.441)--(12.297,6.443)--(12.300,6.445)%
  --(12.303,6.446)--(12.306,6.448)--(12.309,6.450)--(12.312,6.452)--(12.315,6.454)--(12.318,6.456)%
  --(12.321,6.458)--(12.324,6.460)--(12.327,6.461)--(12.330,6.463)--(12.333,6.465)--(12.336,6.467)%
  --(12.339,6.469)--(12.342,6.471)--(12.345,6.473)--(12.348,6.475)--(12.351,6.477)--(12.354,6.478)%
  --(12.357,6.480)--(12.360,6.482)--(12.363,6.484)--(12.366,6.486)--(12.369,6.488)--(12.372,6.490)%
  --(12.375,6.492)--(12.378,6.494)--(12.381,6.495)--(12.384,6.497)--(12.387,6.499)--(12.390,6.501)%
  --(12.393,6.503)--(12.396,6.505)--(12.399,6.507)--(12.402,6.509)--(12.405,6.511)--(12.408,6.512)%
  --(12.411,6.514)--(12.414,6.516)--(12.417,6.518)--(12.420,6.520)--(12.423,6.522)--(12.426,6.524)%
  --(12.429,6.526)--(12.432,6.528)--(12.435,6.529)--(12.438,6.531)--(12.441,6.533)--(12.444,6.535)%
  --(12.447,6.537)--(12.450,6.539)--(12.453,6.541)--(12.456,6.543)--(12.459,6.544)--(12.462,6.546)%
  --(12.465,6.548)--(12.468,6.550)--(12.471,6.552)--(12.474,6.554)--(12.477,6.556)--(12.480,6.558)%
  --(12.483,6.560)--(12.486,6.561)--(12.489,6.563)--(12.492,6.565)--(12.495,6.567)--(12.498,6.569)%
  --(12.501,6.571)--(12.504,6.573)--(12.506,6.575)--(12.509,6.577)--(12.512,6.578)--(12.515,6.580)%
  --(12.518,6.582)--(12.521,6.584)--(12.524,6.586)--(12.527,6.588)--(12.530,6.590)--(12.533,6.592)%
  --(12.536,6.594)--(12.539,6.595)--(12.542,6.597)--(12.545,6.599)--(12.548,6.601)--(12.551,6.603)%
  --(12.554,6.605)--(12.557,6.607)--(12.560,6.609)--(12.563,6.611)--(12.566,6.612)--(12.569,6.614)%
  --(12.572,6.616)--(12.575,6.618)--(12.578,6.620)--(12.581,6.622)--(12.584,6.624)--(12.587,6.626)%
  --(12.590,6.628)--(12.593,6.629)--(12.596,6.631)--(12.599,6.633)--(12.602,6.635)--(12.605,6.637)%
  --(12.608,6.639)--(12.611,6.641)--(12.614,6.643)--(12.617,6.645)--(12.620,6.646)--(12.623,6.648)%
  --(12.626,6.650)--(12.629,6.652)--(12.632,6.654)--(12.635,6.656)--(12.638,6.658)--(12.641,6.660)%
  --(12.644,6.662)--(12.647,6.663)--(12.650,6.665)--(12.653,6.667)--(12.656,6.669)--(12.659,6.671)%
  --(12.662,6.673)--(12.665,6.675)--(12.668,6.677)--(12.671,6.678)--(12.674,6.680)--(12.677,6.682)%
  --(12.680,6.684)--(12.683,6.686)--(12.686,6.688)--(12.689,6.690)--(12.692,6.692)--(12.695,6.694)%
  --(12.698,6.695)--(12.701,6.697)--(12.704,6.699)--(12.707,6.701)--(12.710,6.703)--(12.713,6.705)%
  --(12.715,6.707)--(12.718,6.709)--(12.721,6.711)--(12.724,6.712)--(12.727,6.714)--(12.730,6.716)%
  --(12.733,6.718)--(12.736,6.720)--(12.739,6.722)--(12.742,6.724)--(12.745,6.726)--(12.748,6.728)%
  --(12.751,6.729)--(12.754,6.731)--(12.757,6.733)--(12.760,6.735)--(12.763,6.737)--(12.766,6.739)%
  --(12.769,6.741)--(12.772,6.743)--(12.775,6.745)--(12.778,6.746)--(12.781,6.748)--(12.784,6.750)%
  --(12.787,6.752)--(12.790,6.754)--(12.793,6.756)--(12.796,6.758)--(12.799,6.760)--(12.802,6.762)%
  --(12.805,6.763)--(12.808,6.765)--(12.811,6.767)--(12.814,6.769)--(12.817,6.771)--(12.820,6.773)%
  --(12.823,6.775)--(12.826,6.777)--(12.829,6.779)--(12.832,6.780)--(12.835,6.782)--(12.838,6.784)%
  --(12.841,6.786)--(12.844,6.788)--(12.847,6.790)--(12.850,6.792)--(12.853,6.794)--(12.856,6.796)%
  --(12.859,6.797)--(12.862,6.799)--(12.865,6.801)--(12.868,6.803)--(12.871,6.805)--(12.874,6.807)%
  --(12.877,6.809)--(12.880,6.811)--(12.883,6.813)--(12.886,6.814)--(12.889,6.816)--(12.892,6.818)%
  --(12.895,6.820)--(12.898,6.822)--(12.901,6.824)--(12.904,6.826)--(12.907,6.828)--(12.910,6.830)%
  --(12.913,6.831)--(12.916,6.833)--(12.919,6.835)--(12.922,6.837)--(12.924,6.839)--(12.927,6.841)%
  --(12.930,6.843)--(12.933,6.845)--(12.936,6.847)--(12.939,6.849)--(12.942,6.850)--(12.945,6.852)%
  --(12.948,6.854)--(12.951,6.856)--(12.954,6.858)--(12.957,6.860)--(12.960,6.862)--(12.963,6.864)%
  --(12.966,6.866)--(12.969,6.867)--(12.972,6.869)--(12.975,6.871)--(12.978,6.873)--(12.981,6.875)%
  --(12.984,6.877)--(12.987,6.879)--(12.990,6.881)--(12.993,6.883)--(12.996,6.884)--(12.999,6.886)%
  --(13.002,6.888)--(13.005,6.890)--(13.008,6.892)--(13.011,6.894)--(13.014,6.896)--(13.017,6.898)%
  --(13.020,6.900)--(13.023,6.901)--(13.026,6.903)--(13.029,6.905)--(13.032,6.907)--(13.035,6.909)%
  --(13.038,6.911)--(13.041,6.913)--(13.044,6.915)--(13.047,6.917)--(13.050,6.918)--(13.053,6.920)%
  --(13.056,6.922)--(13.059,6.924)--(13.062,6.926)--(13.065,6.928)--(13.068,6.930)--(13.071,6.932)%
  --(13.074,6.934)--(13.077,6.935)--(13.080,6.937)--(13.083,6.939)--(13.086,6.941)--(13.089,6.943)%
  --(13.092,6.945)--(13.095,6.947)--(13.098,6.949)--(13.101,6.951)--(13.104,6.952)--(13.107,6.954)%
  --(13.110,6.956)--(13.113,6.958)--(13.116,6.960)--(13.119,6.962)--(13.122,6.964)--(13.125,6.966)%
  --(13.128,6.968)--(13.131,6.969)--(13.133,6.971)--(13.136,6.973)--(13.139,6.975)--(13.142,6.977)%
  --(13.145,6.979)--(13.148,6.981)--(13.151,6.983)--(13.154,6.985)--(13.157,6.986)--(13.160,6.988)%
  --(13.163,6.990)--(13.166,6.992)--(13.169,6.994)--(13.172,6.996)--(13.175,6.998)--(13.178,7.000)%
  --(13.181,7.002)--(13.184,7.004)--(13.187,7.005)--(13.190,7.007)--(13.193,7.009)--(13.196,7.011)%
  --(13.199,7.013)--(13.202,7.015)--(13.205,7.017)--(13.208,7.019)--(13.211,7.021)--(13.214,7.022)%
  --(13.217,7.024)--(13.220,7.026)--(13.223,7.028)--(13.226,7.030)--(13.229,7.032)--(13.232,7.034)%
  --(13.235,7.036)--(13.238,7.038)--(13.241,7.039)--(13.244,7.041)--(13.247,7.043)--(13.250,7.045)%
  --(13.253,7.047)--(13.256,7.049)--(13.259,7.051)--(13.262,7.053)--(13.265,7.055)--(13.268,7.056)%
  --(13.271,7.058)--(13.274,7.060)--(13.277,7.062)--(13.280,7.064)--(13.283,7.066)--(13.286,7.068)%
  --(13.289,7.070)--(13.292,7.072)--(13.295,7.073)--(13.298,7.075)--(13.301,7.077)--(13.304,7.079)%
  --(13.307,7.081)--(13.310,7.083)--(13.313,7.085)--(13.316,7.087)--(13.319,7.089)--(13.322,7.091)%
  --(13.325,7.092)--(13.328,7.094)--(13.331,7.096)--(13.334,7.098)--(13.337,7.100)--(13.340,7.102)%
  --(13.342,7.104)--(13.345,7.106)--(13.348,7.108)--(13.351,7.109)--(13.354,7.111)--(13.357,7.113)%
  --(13.360,7.115)--(13.363,7.117)--(13.366,7.119)--(13.369,7.121)--(13.372,7.123)--(13.375,7.125)%
  --(13.378,7.126)--(13.381,7.128)--(13.384,7.130)--(13.387,7.132)--(13.390,7.134)--(13.393,7.136)%
  --(13.396,7.138)--(13.399,7.140)--(13.402,7.142)--(13.405,7.143)--(13.408,7.145)--(13.411,7.147)%
  --(13.414,7.149)--(13.417,7.151)--(13.420,7.153)--(13.423,7.155)--(13.426,7.157)--(13.429,7.159)%
  --(13.432,7.161)--(13.435,7.162)--(13.438,7.164)--(13.441,7.166)--(13.444,7.168);
\gpcolor{color=gp lt color border}
\node[gp node left] at (2.972,7.989) {$\rho \approx 0.2$};
\gpcolor{rgb color={0.000,0.620,0.451}}
\draw[gp path] (1.872,7.989)--(2.788,7.989);
\draw[gp path] (1.504,2.514)--(1.507,2.513)--(1.510,2.511)--(1.513,2.509)--(1.516,2.508)%
  --(1.519,2.506)--(1.522,2.505)--(1.525,2.503)--(1.528,2.502)--(1.531,2.500)--(1.534,2.498)%
  --(1.537,2.497)--(1.540,2.495)--(1.543,2.494)--(1.546,2.492)--(1.549,2.491)--(1.552,2.489)%
  --(1.555,2.487)--(1.558,2.486)--(1.561,2.484)--(1.564,2.483)--(1.567,2.481)--(1.570,2.479)%
  --(1.573,2.478)--(1.576,2.476)--(1.579,2.475)--(1.582,2.473)--(1.585,2.472)--(1.588,2.470)%
  --(1.591,2.468)--(1.594,2.467)--(1.597,2.465)--(1.600,2.464)--(1.603,2.462)--(1.606,2.461)%
  --(1.609,2.459)--(1.611,2.457)--(1.614,2.456)--(1.617,2.454)--(1.620,2.453)--(1.623,2.451)%
  --(1.626,2.450)--(1.629,2.448)--(1.632,2.447)--(1.635,2.445)--(1.638,2.443)--(1.641,2.442)%
  --(1.644,2.440)--(1.647,2.439)--(1.650,2.437)--(1.653,2.436)--(1.656,2.434)--(1.659,2.433)%
  --(1.662,2.431)--(1.665,2.429)--(1.668,2.428)--(1.671,2.426)--(1.674,2.425)--(1.677,2.423)%
  --(1.680,2.422)--(1.683,2.420)--(1.686,2.419)--(1.689,2.417)--(1.692,2.416)--(1.695,2.414)%
  --(1.698,2.413)--(1.701,2.411)--(1.704,2.409)--(1.707,2.408)--(1.710,2.406)--(1.713,2.405)%
  --(1.716,2.403)--(1.719,2.402)--(1.722,2.400)--(1.725,2.399)--(1.728,2.397)--(1.731,2.396)%
  --(1.734,2.394)--(1.737,2.393)--(1.740,2.391)--(1.743,2.390)--(1.746,2.388)--(1.749,2.387)%
  --(1.752,2.385)--(1.755,2.384)--(1.758,2.382)--(1.761,2.381)--(1.764,2.379)--(1.767,2.378)%
  --(1.770,2.376)--(1.773,2.375)--(1.776,2.373)--(1.779,2.372)--(1.782,2.370)--(1.785,2.369)%
  --(1.788,2.367)--(1.791,2.366)--(1.794,2.364)--(1.797,2.363)--(1.800,2.361)--(1.803,2.360)%
  --(1.806,2.358)--(1.809,2.357)--(1.812,2.356)--(1.815,2.354)--(1.818,2.353)--(1.820,2.351)%
  --(1.823,2.350)--(1.826,2.348)--(1.829,2.347)--(1.832,2.345)--(1.835,2.344)--(1.838,2.342)%
  --(1.841,2.341)--(1.844,2.340)--(1.847,2.338)--(1.850,2.337)--(1.853,2.335)--(1.856,2.334)%
  --(1.859,2.332)--(1.862,2.331)--(1.865,2.330)--(1.868,2.328)--(1.871,2.327)--(1.874,2.325)%
  --(1.877,2.324)--(1.880,2.323)--(1.883,2.321)--(1.886,2.320)--(1.889,2.318)--(1.892,2.317)%
  --(1.895,2.316)--(1.898,2.314)--(1.901,2.313)--(1.904,2.311)--(1.907,2.310)--(1.910,2.309)%
  --(1.913,2.307)--(1.916,2.306)--(1.919,2.305)--(1.922,2.303)--(1.925,2.302)--(1.928,2.300)%
  --(1.931,2.299)--(1.934,2.298)--(1.937,2.296)--(1.940,2.295)--(1.943,2.294)--(1.946,2.292)%
  --(1.949,2.291)--(1.952,2.290)--(1.955,2.288)--(1.958,2.287)--(1.961,2.286)--(1.964,2.284)%
  --(1.967,2.283)--(1.970,2.282)--(1.973,2.280)--(1.976,2.279)--(1.979,2.278)--(1.982,2.276)%
  --(1.985,2.275)--(1.988,2.274)--(1.991,2.273)--(1.994,2.271)--(1.997,2.270)--(2.000,2.269)%
  --(2.003,2.267)--(2.006,2.266)--(2.009,2.265)--(2.012,2.264)--(2.015,2.262)--(2.018,2.261)%
  --(2.021,2.260)--(2.024,2.259)--(2.027,2.257)--(2.029,2.256)--(2.032,2.255)--(2.035,2.254)%
  --(2.038,2.252)--(2.041,2.251)--(2.044,2.250)--(2.047,2.249)--(2.050,2.247)--(2.053,2.246)%
  --(2.056,2.245)--(2.059,2.244)--(2.062,2.242)--(2.065,2.241)--(2.068,2.240)--(2.071,2.239)%
  --(2.074,2.238)--(2.077,2.236)--(2.080,2.235)--(2.083,2.234)--(2.086,2.233)--(2.089,2.232)%
  --(2.092,2.231)--(2.095,2.229)--(2.098,2.228)--(2.101,2.227)--(2.104,2.226)--(2.107,2.225)%
  --(2.110,2.224)--(2.113,2.222)--(2.116,2.221)--(2.119,2.220)--(2.122,2.219)--(2.125,2.218)%
  --(2.128,2.217)--(2.131,2.216)--(2.134,2.214)--(2.137,2.213)--(2.140,2.212)--(2.143,2.211)%
  --(2.146,2.210)--(2.149,2.209)--(2.152,2.208)--(2.155,2.207)--(2.158,2.206)--(2.161,2.204)%
  --(2.164,2.203)--(2.167,2.202)--(2.170,2.201)--(2.173,2.200)--(2.176,2.199)--(2.179,2.198)%
  --(2.182,2.197)--(2.185,2.196)--(2.188,2.195)--(2.191,2.194)--(2.194,2.193)--(2.197,2.192)%
  --(2.200,2.191)--(2.203,2.190)--(2.206,2.189)--(2.209,2.188)--(2.212,2.187)--(2.215,2.186)%
  --(2.218,2.184)--(2.221,2.183)--(2.224,2.182)--(2.227,2.181)--(2.230,2.180)--(2.233,2.179)%
  --(2.236,2.178)--(2.238,2.178)--(2.241,2.177)--(2.244,2.176)--(2.247,2.175)--(2.250,2.174)%
  --(2.253,2.173)--(2.256,2.172)--(2.259,2.171)--(2.262,2.170)--(2.265,2.169)--(2.268,2.168)%
  --(2.271,2.167)--(2.274,2.166)--(2.277,2.165)--(2.280,2.164)--(2.283,2.163)--(2.286,2.162)%
  --(2.289,2.161)--(2.292,2.160)--(2.295,2.160)--(2.298,2.159)--(2.301,2.158)--(2.304,2.157)%
  --(2.307,2.156)--(2.310,2.155)--(2.313,2.154)--(2.316,2.153)--(2.319,2.152)--(2.322,2.152)%
  --(2.325,2.151)--(2.328,2.150)--(2.331,2.149)--(2.334,2.148)--(2.337,2.147)--(2.340,2.146)%
  --(2.343,2.146)--(2.346,2.145)--(2.349,2.144)--(2.352,2.143)--(2.355,2.142)--(2.358,2.142)%
  --(2.361,2.141)--(2.364,2.140)--(2.367,2.139)--(2.370,2.138)--(2.373,2.138)--(2.376,2.137)%
  --(2.379,2.136)--(2.382,2.135)--(2.385,2.134)--(2.388,2.134)--(2.391,2.133)--(2.394,2.132)%
  --(2.397,2.131)--(2.400,2.131)--(2.403,2.130)--(2.406,2.129)--(2.409,2.128)--(2.412,2.128)%
  --(2.415,2.127)--(2.418,2.126)--(2.421,2.125)--(2.424,2.125)--(2.427,2.124)--(2.430,2.123)%
  --(2.433,2.123)--(2.436,2.122)--(2.439,2.121)--(2.442,2.120)--(2.445,2.120)--(2.447,2.119)%
  --(2.450,2.118)--(2.453,2.118)--(2.456,2.117)--(2.459,2.116)--(2.462,2.116)--(2.465,2.115)%
  --(2.468,2.114)--(2.471,2.114)--(2.474,2.113)--(2.477,2.112)--(2.480,2.112)--(2.483,2.111)%
  --(2.486,2.111)--(2.489,2.110)--(2.492,2.109)--(2.495,2.109)--(2.498,2.108)--(2.501,2.108)%
  --(2.504,2.107)--(2.507,2.106)--(2.510,2.106)--(2.513,2.105)--(2.516,2.105)--(2.519,2.104)%
  --(2.522,2.103)--(2.525,2.103)--(2.528,2.102)--(2.531,2.102)--(2.534,2.101)--(2.537,2.101)%
  --(2.540,2.100)--(2.543,2.099)--(2.546,2.099)--(2.549,2.098)--(2.552,2.098)--(2.555,2.097)%
  --(2.558,2.097)--(2.561,2.096)--(2.564,2.096)--(2.567,2.095)--(2.570,2.095)--(2.573,2.094)%
  --(2.576,2.094)--(2.579,2.093)--(2.582,2.093)--(2.585,2.092)--(2.588,2.092)--(2.591,2.091)%
  --(2.594,2.091)--(2.597,2.090)--(2.600,2.090)--(2.603,2.090)--(2.606,2.089)--(2.609,2.089)%
  --(2.612,2.088)--(2.615,2.088)--(2.618,2.087)--(2.621,2.087)--(2.624,2.086)--(2.627,2.086)%
  --(2.630,2.086)--(2.633,2.085)--(2.636,2.085)--(2.639,2.084)--(2.642,2.084)--(2.645,2.084)%
  --(2.648,2.083)--(2.651,2.083)--(2.654,2.082)--(2.656,2.082)--(2.659,2.082)--(2.662,2.081)%
  --(2.665,2.081)--(2.668,2.081)--(2.671,2.080)--(2.674,2.080)--(2.677,2.080)--(2.680,2.079)%
  --(2.683,2.079)--(2.686,2.079)--(2.689,2.078)--(2.692,2.078)--(2.695,2.078)--(2.698,2.077)%
  --(2.701,2.077)--(2.704,2.077)--(2.707,2.076)--(2.710,2.076)--(2.713,2.076)--(2.716,2.075)%
  --(2.719,2.075)--(2.722,2.075)--(2.725,2.075)--(2.728,2.074)--(2.731,2.074)--(2.734,2.074)%
  --(2.737,2.074)--(2.740,2.073)--(2.743,2.073)--(2.746,2.073)--(2.749,2.073)--(2.752,2.072)%
  --(2.755,2.072)--(2.758,2.072)--(2.761,2.072)--(2.764,2.071)--(2.767,2.071)--(2.770,2.071)%
  --(2.773,2.071)--(2.776,2.071)--(2.779,2.070)--(2.782,2.070)--(2.785,2.070)--(2.788,2.070)%
  --(2.791,2.070)--(2.794,2.069)--(2.797,2.069)--(2.800,2.069)--(2.803,2.069)--(2.806,2.069)%
  --(2.809,2.069)--(2.812,2.068)--(2.815,2.068)--(2.818,2.068)--(2.821,2.068)--(2.824,2.068)%
  --(2.827,2.068)--(2.830,2.068)--(2.833,2.067)--(2.836,2.067)--(2.839,2.067)--(2.842,2.067)%
  --(2.845,2.067)--(2.848,2.067)--(2.851,2.067)--(2.854,2.067)--(2.857,2.067)--(2.860,2.066)%
  --(2.863,2.066)--(2.866,2.066)--(2.868,2.066)--(2.871,2.066)--(2.874,2.066)--(2.877,2.066)%
  --(2.880,2.066)--(2.883,2.066)--(2.886,2.066)--(2.889,2.066)--(2.892,2.066)--(2.895,2.066)%
  --(2.898,2.066)--(2.901,2.066)--(2.904,2.066)--(2.907,2.066)--(2.910,2.066)--(2.913,2.066)%
  --(2.916,2.065)--(2.919,2.065)--(2.922,2.065)--(2.925,2.065)--(2.928,2.065)--(2.931,2.065)%
  --(2.934,2.065)--(2.937,2.066)--(2.940,2.066)--(2.943,2.066)--(2.946,2.066)--(2.949,2.066)%
  --(2.952,2.066)--(2.955,2.066)--(2.958,2.066)--(2.961,2.066)--(2.964,2.066)--(2.967,2.066)%
  --(2.970,2.066)--(2.973,2.066)--(2.976,2.066)--(2.979,2.066)--(2.982,2.066)--(2.985,2.066)%
  --(2.988,2.066)--(2.991,2.066)--(2.994,2.067)--(2.997,2.067)--(3.000,2.067)--(3.003,2.067)%
  --(3.006,2.067)--(3.009,2.067)--(3.012,2.067)--(3.015,2.067)--(3.018,2.067)--(3.021,2.067)%
  --(3.024,2.068)--(3.027,2.068)--(3.030,2.068)--(3.033,2.068)--(3.036,2.068)--(3.039,2.068)%
  --(3.042,2.068)--(3.045,2.069)--(3.048,2.069)--(3.051,2.069)--(3.054,2.069)--(3.057,2.069)%
  --(3.060,2.069)--(3.063,2.070)--(3.066,2.070)--(3.069,2.070)--(3.072,2.070)--(3.075,2.070)%
  --(3.077,2.070)--(3.080,2.071)--(3.083,2.071)--(3.086,2.071)--(3.089,2.071)--(3.092,2.072)%
  --(3.095,2.072)--(3.098,2.072)--(3.101,2.072)--(3.104,2.072)--(3.107,2.073)--(3.110,2.073)%
  --(3.113,2.073)--(3.116,2.073)--(3.119,2.074)--(3.122,2.074)--(3.125,2.074)--(3.128,2.074)%
  --(3.131,2.075)--(3.134,2.075)--(3.137,2.075)--(3.140,2.075)--(3.143,2.076)--(3.146,2.076)%
  --(3.149,2.076)--(3.152,2.076)--(3.155,2.077)--(3.158,2.077)--(3.161,2.077)--(3.164,2.078)%
  --(3.167,2.078)--(3.170,2.078)--(3.173,2.079)--(3.176,2.079)--(3.179,2.079)--(3.182,2.079)%
  --(3.185,2.080)--(3.188,2.080)--(3.191,2.080)--(3.194,2.081)--(3.197,2.081)--(3.200,2.081)%
  --(3.203,2.082)--(3.206,2.082)--(3.209,2.082)--(3.212,2.083)--(3.215,2.083)--(3.218,2.084)%
  --(3.221,2.084)--(3.224,2.084)--(3.227,2.085)--(3.230,2.085)--(3.233,2.085)--(3.236,2.086)%
  --(3.239,2.086)--(3.242,2.087)--(3.245,2.087)--(3.248,2.087)--(3.251,2.088)--(3.254,2.088)%
  --(3.257,2.089)--(3.260,2.089)--(3.263,2.089)--(3.266,2.090)--(3.269,2.090)--(3.272,2.091)%
  --(3.275,2.091)--(3.278,2.091)--(3.281,2.092)--(3.284,2.092)--(3.286,2.093)--(3.289,2.093)%
  --(3.292,2.094)--(3.295,2.094)--(3.298,2.094)--(3.301,2.095)--(3.304,2.095)--(3.307,2.096)%
  --(3.310,2.096)--(3.313,2.097)--(3.316,2.097)--(3.319,2.098)--(3.322,2.098)--(3.325,2.099)%
  --(3.328,2.099)--(3.331,2.100)--(3.334,2.100)--(3.337,2.101)--(3.340,2.101)--(3.343,2.102)%
  --(3.346,2.102)--(3.349,2.103)--(3.352,2.103)--(3.355,2.104)--(3.358,2.104)--(3.361,2.105)%
  --(3.364,2.105)--(3.367,2.106)--(3.370,2.106)--(3.373,2.107)--(3.376,2.107)--(3.379,2.108)%
  --(3.382,2.108)--(3.385,2.109)--(3.388,2.109)--(3.391,2.110)--(3.394,2.110)--(3.397,2.111)%
  --(3.400,2.111)--(3.403,2.112)--(3.406,2.113)--(3.409,2.113)--(3.412,2.114)--(3.415,2.114)%
  --(3.418,2.115)--(3.421,2.115)--(3.424,2.116)--(3.427,2.117)--(3.430,2.117)--(3.433,2.118)%
  --(3.436,2.118)--(3.439,2.119)--(3.442,2.119)--(3.445,2.120)--(3.448,2.121)--(3.451,2.121)%
  --(3.454,2.122)--(3.457,2.122)--(3.460,2.123)--(3.463,2.124)--(3.466,2.124)--(3.469,2.125)%
  --(3.472,2.126)--(3.475,2.126)--(3.478,2.127)--(3.481,2.127)--(3.484,2.128)--(3.487,2.129)%
  --(3.490,2.129)--(3.493,2.130)--(3.495,2.131)--(3.498,2.131)--(3.501,2.132)--(3.504,2.133)%
  --(3.507,2.133)--(3.510,2.134)--(3.513,2.135)--(3.516,2.135)--(3.519,2.136)--(3.522,2.137)%
  --(3.525,2.137)--(3.528,2.138)--(3.531,2.139)--(3.534,2.139)--(3.537,2.140)--(3.540,2.141)%
  --(3.543,2.141)--(3.546,2.142)--(3.549,2.143)--(3.552,2.143)--(3.555,2.144)--(3.558,2.145)%
  --(3.561,2.145)--(3.564,2.146)--(3.567,2.147)--(3.570,2.148)--(3.573,2.148)--(3.576,2.149)%
  --(3.579,2.150)--(3.582,2.150)--(3.585,2.151)--(3.588,2.152)--(3.591,2.153)--(3.594,2.153)%
  --(3.597,2.154)--(3.600,2.155)--(3.603,2.156)--(3.606,2.156)--(3.609,2.157)--(3.612,2.158)%
  --(3.615,2.159)--(3.618,2.159)--(3.621,2.160)--(3.624,2.161)--(3.627,2.162)--(3.630,2.162)%
  --(3.633,2.163)--(3.636,2.164)--(3.639,2.165)--(3.642,2.165)--(3.645,2.166)--(3.648,2.167)%
  --(3.651,2.168)--(3.654,2.168)--(3.657,2.169)--(3.660,2.170)--(3.663,2.171)--(3.666,2.172)%
  --(3.669,2.172)--(3.672,2.173)--(3.675,2.174)--(3.678,2.175)--(3.681,2.176)--(3.684,2.176)%
  --(3.687,2.177)--(3.690,2.178)--(3.693,2.179)--(3.696,2.180)--(3.699,2.180)--(3.702,2.181)%
  --(3.704,2.182)--(3.707,2.183)--(3.710,2.184)--(3.713,2.185)--(3.716,2.185)--(3.719,2.186)%
  --(3.722,2.187)--(3.725,2.188)--(3.728,2.189)--(3.731,2.190)--(3.734,2.190)--(3.737,2.191)%
  --(3.740,2.192)--(3.743,2.193)--(3.746,2.194)--(3.749,2.195)--(3.752,2.196)--(3.755,2.196)%
  --(3.758,2.197)--(3.761,2.198)--(3.764,2.199)--(3.767,2.200)--(3.770,2.201)--(3.773,2.202)%
  --(3.776,2.203)--(3.779,2.203)--(3.782,2.204)--(3.785,2.205)--(3.788,2.206)--(3.791,2.207)%
  --(3.794,2.208)--(3.797,2.209)--(3.800,2.210)--(3.803,2.211)--(3.806,2.211)--(3.809,2.212)%
  --(3.812,2.213)--(3.815,2.214)--(3.818,2.215)--(3.821,2.216)--(3.824,2.217)--(3.827,2.218)%
  --(3.830,2.219)--(3.833,2.220)--(3.836,2.221)--(3.839,2.222)--(3.842,2.222)--(3.845,2.223)%
  --(3.848,2.224)--(3.851,2.225)--(3.854,2.226)--(3.857,2.227)--(3.860,2.228)--(3.863,2.229)%
  --(3.866,2.230)--(3.869,2.231)--(3.872,2.232)--(3.875,2.233)--(3.878,2.234)--(3.881,2.235)%
  --(3.884,2.236)--(3.887,2.237)--(3.890,2.238)--(3.893,2.239)--(3.896,2.239)--(3.899,2.240)%
  --(3.902,2.241)--(3.905,2.242)--(3.908,2.243)--(3.911,2.244)--(3.914,2.245)--(3.916,2.246)%
  --(3.919,2.247)--(3.922,2.248)--(3.925,2.249)--(3.928,2.250)--(3.931,2.251)--(3.934,2.252)%
  --(3.937,2.253)--(3.940,2.254)--(3.943,2.255)--(3.946,2.256)--(3.949,2.257)--(3.952,2.258)%
  --(3.955,2.259)--(3.958,2.260)--(3.961,2.261)--(3.964,2.262)--(3.967,2.263)--(3.970,2.264)%
  --(3.973,2.265)--(3.976,2.266)--(3.979,2.267)--(3.982,2.268)--(3.985,2.269)--(3.988,2.270)%
  --(3.991,2.271)--(3.994,2.272)--(3.997,2.273)--(4.000,2.274)--(4.003,2.275)--(4.006,2.277)%
  --(4.009,2.278)--(4.012,2.279)--(4.015,2.280)--(4.018,2.281)--(4.021,2.282)--(4.024,2.283)%
  --(4.027,2.284)--(4.030,2.285)--(4.033,2.286)--(4.036,2.287)--(4.039,2.288)--(4.042,2.289)%
  --(4.045,2.290)--(4.048,2.291)--(4.051,2.292)--(4.054,2.293)--(4.057,2.294)--(4.060,2.295)%
  --(4.063,2.297)--(4.066,2.298)--(4.069,2.299)--(4.072,2.300)--(4.075,2.301)--(4.078,2.302)%
  --(4.081,2.303)--(4.084,2.304)--(4.087,2.305)--(4.090,2.306)--(4.093,2.307)--(4.096,2.308)%
  --(4.099,2.310)--(4.102,2.311)--(4.105,2.312)--(4.108,2.313)--(4.111,2.314)--(4.114,2.315)%
  --(4.117,2.316)--(4.120,2.317)--(4.123,2.318)--(4.125,2.319)--(4.128,2.321)--(4.131,2.322)%
  --(4.134,2.323)--(4.137,2.324)--(4.140,2.325)--(4.143,2.326)--(4.146,2.327)--(4.149,2.328)%
  --(4.152,2.330)--(4.155,2.331)--(4.158,2.332)--(4.161,2.333)--(4.164,2.334)--(4.167,2.335)%
  --(4.170,2.336)--(4.173,2.337)--(4.176,2.339)--(4.179,2.340)--(4.182,2.341)--(4.185,2.342)%
  --(4.188,2.343)--(4.191,2.344)--(4.194,2.345)--(4.197,2.347)--(4.200,2.348)--(4.203,2.349)%
  --(4.206,2.350)--(4.209,2.351)--(4.212,2.352)--(4.215,2.353)--(4.218,2.355)--(4.221,2.356)%
  --(4.224,2.357)--(4.227,2.358)--(4.230,2.359)--(4.233,2.360)--(4.236,2.362)--(4.239,2.363)%
  --(4.242,2.364)--(4.245,2.365)--(4.248,2.366)--(4.251,2.367)--(4.254,2.369)--(4.257,2.370)%
  --(4.260,2.371)--(4.263,2.372)--(4.266,2.373)--(4.269,2.375)--(4.272,2.376)--(4.275,2.377)%
  --(4.278,2.378)--(4.281,2.379)--(4.284,2.380)--(4.287,2.382)--(4.290,2.383)--(4.293,2.384)%
  --(4.296,2.385)--(4.299,2.386)--(4.302,2.388)--(4.305,2.389)--(4.308,2.390)--(4.311,2.391)%
  --(4.314,2.393)--(4.317,2.394)--(4.320,2.395)--(4.323,2.396)--(4.326,2.397)--(4.329,2.399)%
  --(4.332,2.400)--(4.334,2.401)--(4.337,2.402)--(4.340,2.403)--(4.343,2.405)--(4.346,2.406)%
  --(4.349,2.407)--(4.352,2.408)--(4.355,2.410)--(4.358,2.411)--(4.361,2.412)--(4.364,2.413)%
  --(4.367,2.414)--(4.370,2.416)--(4.373,2.417)--(4.376,2.418)--(4.379,2.419)--(4.382,2.421)%
  --(4.385,2.422)--(4.388,2.423)--(4.391,2.424)--(4.394,2.426)--(4.397,2.427)--(4.400,2.428)%
  --(4.403,2.429)--(4.406,2.431)--(4.409,2.432)--(4.412,2.433)--(4.415,2.434)--(4.418,2.436)%
  --(4.421,2.437)--(4.424,2.438)--(4.427,2.439)--(4.430,2.441)--(4.433,2.442)--(4.436,2.443)%
  --(4.439,2.444)--(4.442,2.446)--(4.445,2.447)--(4.448,2.448)--(4.451,2.450)--(4.454,2.451)%
  --(4.457,2.452)--(4.460,2.453)--(4.463,2.455)--(4.466,2.456)--(4.469,2.457)--(4.472,2.458)%
  --(4.475,2.460)--(4.478,2.461)--(4.481,2.462)--(4.484,2.464)--(4.487,2.465)--(4.490,2.466)%
  --(4.493,2.467)--(4.496,2.469)--(4.499,2.470)--(4.502,2.471)--(4.505,2.473)--(4.508,2.474)%
  --(4.511,2.475)--(4.514,2.477)--(4.517,2.478)--(4.520,2.479)--(4.523,2.480)--(4.526,2.482)%
  --(4.529,2.483)--(4.532,2.484)--(4.535,2.486)--(4.538,2.487)--(4.541,2.488)--(4.543,2.490)%
  --(4.546,2.491)--(4.549,2.492)--(4.552,2.494)--(4.555,2.495)--(4.558,2.496)--(4.561,2.497)%
  --(4.564,2.499)--(4.567,2.500)--(4.570,2.501)--(4.573,2.503)--(4.576,2.504)--(4.579,2.505)%
  --(4.582,2.507)--(4.585,2.508)--(4.588,2.509)--(4.591,2.511)--(4.594,2.512)--(4.597,2.513)%
  --(4.600,2.515)--(4.603,2.516)--(4.606,2.517)--(4.609,2.519)--(4.612,2.520)--(4.615,2.521)%
  --(4.618,2.523)--(4.621,2.524)--(4.624,2.525)--(4.627,2.527)--(4.630,2.528)--(4.633,2.529)%
  --(4.636,2.531)--(4.639,2.532)--(4.642,2.533)--(4.645,2.535)--(4.648,2.536)--(4.651,2.538)%
  --(4.654,2.539)--(4.657,2.540)--(4.660,2.542)--(4.663,2.543)--(4.666,2.544)--(4.669,2.546)%
  --(4.672,2.547)--(4.675,2.548)--(4.678,2.550)--(4.681,2.551)--(4.684,2.552)--(4.687,2.554)%
  --(4.690,2.555)--(4.693,2.557)--(4.696,2.558)--(4.699,2.559)--(4.702,2.561)--(4.705,2.562)%
  --(4.708,2.563)--(4.711,2.565)--(4.714,2.566)--(4.717,2.568)--(4.720,2.569)--(4.723,2.570)%
  --(4.726,2.572)--(4.729,2.573)--(4.732,2.574)--(4.735,2.576)--(4.738,2.577)--(4.741,2.579)%
  --(4.744,2.580)--(4.747,2.581)--(4.750,2.583)--(4.752,2.584)--(4.755,2.586)--(4.758,2.587)%
  --(4.761,2.588)--(4.764,2.590)--(4.767,2.591)--(4.770,2.592)--(4.773,2.594)--(4.776,2.595)%
  --(4.779,2.597)--(4.782,2.598)--(4.785,2.599)--(4.788,2.601)--(4.791,2.602)--(4.794,2.604)%
  --(4.797,2.605)--(4.800,2.606)--(4.803,2.608)--(4.806,2.609)--(4.809,2.611)--(4.812,2.612)%
  --(4.815,2.614)--(4.818,2.615)--(4.821,2.616)--(4.824,2.618)--(4.827,2.619)--(4.830,2.621)%
  --(4.833,2.622)--(4.836,2.623)--(4.839,2.625)--(4.842,2.626)--(4.845,2.628)--(4.848,2.629)%
  --(4.851,2.630)--(4.854,2.632)--(4.857,2.633)--(4.860,2.635)--(4.863,2.636)--(4.866,2.638)%
  --(4.869,2.639)--(4.872,2.640)--(4.875,2.642)--(4.878,2.643)--(4.881,2.645)--(4.884,2.646)%
  --(4.887,2.648)--(4.890,2.649)--(4.893,2.650)--(4.896,2.652)--(4.899,2.653)--(4.902,2.655)%
  --(4.905,2.656)--(4.908,2.658)--(4.911,2.659)--(4.914,2.661)--(4.917,2.662)--(4.920,2.663)%
  --(4.923,2.665)--(4.926,2.666)--(4.929,2.668)--(4.932,2.669)--(4.935,2.671)--(4.938,2.672)%
  --(4.941,2.674)--(4.944,2.675)--(4.947,2.676)--(4.950,2.678)--(4.953,2.679)--(4.956,2.681)%
  --(4.959,2.682)--(4.961,2.684)--(4.964,2.685)--(4.967,2.687)--(4.970,2.688)--(4.973,2.690)%
  --(4.976,2.691)--(4.979,2.692)--(4.982,2.694)--(4.985,2.695)--(4.988,2.697)--(4.991,2.698)%
  --(4.994,2.700)--(4.997,2.701)--(5.000,2.703)--(5.003,2.704)--(5.006,2.706)--(5.009,2.707)%
  --(5.012,2.709)--(5.015,2.710)--(5.018,2.712)--(5.021,2.713)--(5.024,2.714)--(5.027,2.716)%
  --(5.030,2.717)--(5.033,2.719)--(5.036,2.720)--(5.039,2.722)--(5.042,2.723)--(5.045,2.725)%
  --(5.048,2.726)--(5.051,2.728)--(5.054,2.729)--(5.057,2.731)--(5.060,2.732)--(5.063,2.734)%
  --(5.066,2.735)--(5.069,2.737)--(5.072,2.738)--(5.075,2.740)--(5.078,2.741)--(5.081,2.743)%
  --(5.084,2.744)--(5.087,2.746)--(5.090,2.747)--(5.093,2.749)--(5.096,2.750)--(5.099,2.752)%
  --(5.102,2.753)--(5.105,2.755)--(5.108,2.756)--(5.111,2.758)--(5.114,2.759)--(5.117,2.761)%
  --(5.120,2.762)--(5.123,2.764)--(5.126,2.765)--(5.129,2.767)--(5.132,2.768)--(5.135,2.770)%
  --(5.138,2.771)--(5.141,2.773)--(5.144,2.774)--(5.147,2.776)--(5.150,2.777)--(5.153,2.779)%
  --(5.156,2.780)--(5.159,2.782)--(5.162,2.783)--(5.165,2.785)--(5.168,2.786)--(5.171,2.788)%
  --(5.173,2.789)--(5.176,2.791)--(5.179,2.792)--(5.182,2.794)--(5.185,2.795)--(5.188,2.797)%
  --(5.191,2.798)--(5.194,2.800)--(5.197,2.801)--(5.200,2.803)--(5.203,2.804)--(5.206,2.806)%
  --(5.209,2.807)--(5.212,2.809)--(5.215,2.810)--(5.218,2.812)--(5.221,2.813)--(5.224,2.815)%
  --(5.227,2.816)--(5.230,2.818)--(5.233,2.819)--(5.236,2.821)--(5.239,2.823)--(5.242,2.824)%
  --(5.245,2.826)--(5.248,2.827)--(5.251,2.829)--(5.254,2.830)--(5.257,2.832)--(5.260,2.833)%
  --(5.263,2.835)--(5.266,2.836)--(5.269,2.838)--(5.272,2.839)--(5.275,2.841)--(5.278,2.842)%
  --(5.281,2.844)--(5.284,2.846)--(5.287,2.847)--(5.290,2.849)--(5.293,2.850)--(5.296,2.852)%
  --(5.299,2.853)--(5.302,2.855)--(5.305,2.856)--(5.308,2.858)--(5.311,2.859)--(5.314,2.861)%
  --(5.317,2.863)--(5.320,2.864)--(5.323,2.866)--(5.326,2.867)--(5.329,2.869)--(5.332,2.870)%
  --(5.335,2.872)--(5.338,2.873)--(5.341,2.875)--(5.344,2.876)--(5.347,2.878)--(5.350,2.880)%
  --(5.353,2.881)--(5.356,2.883)--(5.359,2.884)--(5.362,2.886)--(5.365,2.887)--(5.368,2.889)%
  --(5.371,2.890)--(5.374,2.892)--(5.377,2.894)--(5.380,2.895)--(5.382,2.897)--(5.385,2.898)%
  --(5.388,2.900)--(5.391,2.901)--(5.394,2.903)--(5.397,2.904)--(5.400,2.906)--(5.403,2.908)%
  --(5.406,2.909)--(5.409,2.911)--(5.412,2.912)--(5.415,2.914)--(5.418,2.915)--(5.421,2.917)%
  --(5.424,2.919)--(5.427,2.920)--(5.430,2.922)--(5.433,2.923)--(5.436,2.925)--(5.439,2.926)%
  --(5.442,2.928)--(5.445,2.930)--(5.448,2.931)--(5.451,2.933)--(5.454,2.934)--(5.457,2.936)%
  --(5.460,2.937)--(5.463,2.939)--(5.466,2.941)--(5.469,2.942)--(5.472,2.944)--(5.475,2.945)%
  --(5.478,2.947)--(5.481,2.948)--(5.484,2.950)--(5.487,2.952)--(5.490,2.953)--(5.493,2.955)%
  --(5.496,2.956)--(5.499,2.958)--(5.502,2.960)--(5.505,2.961)--(5.508,2.963)--(5.511,2.964)%
  --(5.514,2.966)--(5.517,2.967)--(5.520,2.969)--(5.523,2.971)--(5.526,2.972)--(5.529,2.974)%
  --(5.532,2.975)--(5.535,2.977)--(5.538,2.979)--(5.541,2.980)--(5.544,2.982)--(5.547,2.983)%
  --(5.550,2.985)--(5.553,2.987)--(5.556,2.988)--(5.559,2.990)--(5.562,2.991)--(5.565,2.993)%
  --(5.568,2.995)--(5.571,2.996)--(5.574,2.998)--(5.577,2.999)--(5.580,3.001)--(5.583,3.003)%
  --(5.586,3.004)--(5.589,3.006)--(5.591,3.007)--(5.594,3.009)--(5.597,3.011)--(5.600,3.012)%
  --(5.603,3.014)--(5.606,3.015)--(5.609,3.017)--(5.612,3.019)--(5.615,3.020)--(5.618,3.022)%
  --(5.621,3.023)--(5.624,3.025)--(5.627,3.027)--(5.630,3.028)--(5.633,3.030)--(5.636,3.031)%
  --(5.639,3.033)--(5.642,3.035)--(5.645,3.036)--(5.648,3.038)--(5.651,3.039)--(5.654,3.041)%
  --(5.657,3.043)--(5.660,3.044)--(5.663,3.046)--(5.666,3.048)--(5.669,3.049)--(5.672,3.051)%
  --(5.675,3.052)--(5.678,3.054)--(5.681,3.056)--(5.684,3.057)--(5.687,3.059)--(5.690,3.060)%
  --(5.693,3.062)--(5.696,3.064)--(5.699,3.065)--(5.702,3.067)--(5.705,3.069)--(5.708,3.070)%
  --(5.711,3.072)--(5.714,3.073)--(5.717,3.075)--(5.720,3.077)--(5.723,3.078)--(5.726,3.080)%
  --(5.729,3.082)--(5.732,3.083)--(5.735,3.085)--(5.738,3.086)--(5.741,3.088)--(5.744,3.090)%
  --(5.747,3.091)--(5.750,3.093)--(5.753,3.095)--(5.756,3.096)--(5.759,3.098)--(5.762,3.099)%
  --(5.765,3.101)--(5.768,3.103)--(5.771,3.104)--(5.774,3.106)--(5.777,3.108)--(5.780,3.109)%
  --(5.783,3.111)--(5.786,3.112)--(5.789,3.114)--(5.792,3.116)--(5.795,3.117)--(5.798,3.119)%
  --(5.800,3.121)--(5.803,3.122)--(5.806,3.124)--(5.809,3.126)--(5.812,3.127)--(5.815,3.129)%
  --(5.818,3.131)--(5.821,3.132)--(5.824,3.134)--(5.827,3.135)--(5.830,3.137)--(5.833,3.139)%
  --(5.836,3.140)--(5.839,3.142)--(5.842,3.144)--(5.845,3.145)--(5.848,3.147)--(5.851,3.149)%
  --(5.854,3.150)--(5.857,3.152)--(5.860,3.153)--(5.863,3.155)--(5.866,3.157)--(5.869,3.158)%
  --(5.872,3.160)--(5.875,3.162)--(5.878,3.163)--(5.881,3.165)--(5.884,3.167)--(5.887,3.168)%
  --(5.890,3.170)--(5.893,3.172)--(5.896,3.173)--(5.899,3.175)--(5.902,3.177)--(5.905,3.178)%
  --(5.908,3.180)--(5.911,3.182)--(5.914,3.183)--(5.917,3.185)--(5.920,3.186)--(5.923,3.188)%
  --(5.926,3.190)--(5.929,3.191)--(5.932,3.193)--(5.935,3.195)--(5.938,3.196)--(5.941,3.198)%
  --(5.944,3.200)--(5.947,3.201)--(5.950,3.203)--(5.953,3.205)--(5.956,3.206)--(5.959,3.208)%
  --(5.962,3.210)--(5.965,3.211)--(5.968,3.213)--(5.971,3.215)--(5.974,3.216)--(5.977,3.218)%
  --(5.980,3.220)--(5.983,3.221)--(5.986,3.223)--(5.989,3.225)--(5.992,3.226)--(5.995,3.228)%
  --(5.998,3.230)--(6.001,3.231)--(6.004,3.233)--(6.007,3.235)--(6.009,3.236)--(6.012,3.238)%
  --(6.015,3.240)--(6.018,3.241)--(6.021,3.243)--(6.024,3.245)--(6.027,3.246)--(6.030,3.248)%
  --(6.033,3.250)--(6.036,3.251)--(6.039,3.253)--(6.042,3.255)--(6.045,3.256)--(6.048,3.258)%
  --(6.051,3.260)--(6.054,3.261)--(6.057,3.263)--(6.060,3.265)--(6.063,3.266)--(6.066,3.268)%
  --(6.069,3.270)--(6.072,3.271)--(6.075,3.273)--(6.078,3.275)--(6.081,3.276)--(6.084,3.278)%
  --(6.087,3.280)--(6.090,3.281)--(6.093,3.283)--(6.096,3.285)--(6.099,3.286)--(6.102,3.288)%
  --(6.105,3.290)--(6.108,3.292)--(6.111,3.293)--(6.114,3.295)--(6.117,3.297)--(6.120,3.298)%
  --(6.123,3.300)--(6.126,3.302)--(6.129,3.303)--(6.132,3.305)--(6.135,3.307)--(6.138,3.308)%
  --(6.141,3.310)--(6.144,3.312)--(6.147,3.313)--(6.150,3.315)--(6.153,3.317)--(6.156,3.318)%
  --(6.159,3.320)--(6.162,3.322)--(6.165,3.323)--(6.168,3.325)--(6.171,3.327)--(6.174,3.329)%
  --(6.177,3.330)--(6.180,3.332)--(6.183,3.334)--(6.186,3.335)--(6.189,3.337)--(6.192,3.339)%
  --(6.195,3.340)--(6.198,3.342)--(6.201,3.344)--(6.204,3.345)--(6.207,3.347)--(6.210,3.349)%
  --(6.213,3.351)--(6.216,3.352)--(6.218,3.354)--(6.221,3.356)--(6.224,3.357)--(6.227,3.359)%
  --(6.230,3.361)--(6.233,3.362)--(6.236,3.364)--(6.239,3.366)--(6.242,3.367)--(6.245,3.369)%
  --(6.248,3.371)--(6.251,3.373)--(6.254,3.374)--(6.257,3.376)--(6.260,3.378)--(6.263,3.379)%
  --(6.266,3.381)--(6.269,3.383)--(6.272,3.384)--(6.275,3.386)--(6.278,3.388)--(6.281,3.390)%
  --(6.284,3.391)--(6.287,3.393)--(6.290,3.395)--(6.293,3.396)--(6.296,3.398)--(6.299,3.400)%
  --(6.302,3.401)--(6.305,3.403)--(6.308,3.405)--(6.311,3.407)--(6.314,3.408)--(6.317,3.410)%
  --(6.320,3.412)--(6.323,3.413)--(6.326,3.415)--(6.329,3.417)--(6.332,3.418)--(6.335,3.420)%
  --(6.338,3.422)--(6.341,3.424)--(6.344,3.425)--(6.347,3.427)--(6.350,3.429)--(6.353,3.430)%
  --(6.356,3.432)--(6.359,3.434)--(6.362,3.436)--(6.365,3.437)--(6.368,3.439)--(6.371,3.441)%
  --(6.374,3.442)--(6.377,3.444)--(6.380,3.446)--(6.383,3.448)--(6.386,3.449)--(6.389,3.451)%
  --(6.392,3.453)--(6.395,3.454)--(6.398,3.456)--(6.401,3.458)--(6.404,3.459)--(6.407,3.461)%
  --(6.410,3.463)--(6.413,3.465)--(6.416,3.466)--(6.419,3.468)--(6.422,3.470)--(6.425,3.471)%
  --(6.428,3.473)--(6.430,3.475)--(6.433,3.477)--(6.436,3.478)--(6.439,3.480)--(6.442,3.482)%
  --(6.445,3.484)--(6.448,3.485)--(6.451,3.487)--(6.454,3.489)--(6.457,3.490)--(6.460,3.492)%
  --(6.463,3.494)--(6.466,3.496)--(6.469,3.497)--(6.472,3.499)--(6.475,3.501)--(6.478,3.502)%
  --(6.481,3.504)--(6.484,3.506)--(6.487,3.508)--(6.490,3.509)--(6.493,3.511)--(6.496,3.513)%
  --(6.499,3.514)--(6.502,3.516)--(6.505,3.518)--(6.508,3.520)--(6.511,3.521)--(6.514,3.523)%
  --(6.517,3.525)--(6.520,3.527)--(6.523,3.528)--(6.526,3.530)--(6.529,3.532)--(6.532,3.533)%
  --(6.535,3.535)--(6.538,3.537)--(6.541,3.539)--(6.544,3.540)--(6.547,3.542)--(6.550,3.544)%
  --(6.553,3.546)--(6.556,3.547)--(6.559,3.549)--(6.562,3.551)--(6.565,3.552)--(6.568,3.554)%
  --(6.571,3.556)--(6.574,3.558)--(6.577,3.559)--(6.580,3.561)--(6.583,3.563)--(6.586,3.565)%
  --(6.589,3.566)--(6.592,3.568)--(6.595,3.570)--(6.598,3.571)--(6.601,3.573)--(6.604,3.575)%
  --(6.607,3.577)--(6.610,3.578)--(6.613,3.580)--(6.616,3.582)--(6.619,3.584)--(6.622,3.585)%
  --(6.625,3.587)--(6.628,3.589)--(6.631,3.591)--(6.634,3.592)--(6.637,3.594)--(6.639,3.596)%
  --(6.642,3.597)--(6.645,3.599)--(6.648,3.601)--(6.651,3.603)--(6.654,3.604)--(6.657,3.606)%
  --(6.660,3.608)--(6.663,3.610)--(6.666,3.611)--(6.669,3.613)--(6.672,3.615)--(6.675,3.617)%
  --(6.678,3.618)--(6.681,3.620)--(6.684,3.622)--(6.687,3.624)--(6.690,3.625)--(6.693,3.627)%
  --(6.696,3.629)--(6.699,3.630)--(6.702,3.632)--(6.705,3.634)--(6.708,3.636)--(6.711,3.637)%
  --(6.714,3.639)--(6.717,3.641)--(6.720,3.643)--(6.723,3.644)--(6.726,3.646)--(6.729,3.648)%
  --(6.732,3.650)--(6.735,3.651)--(6.738,3.653)--(6.741,3.655)--(6.744,3.657)--(6.747,3.658)%
  --(6.750,3.660)--(6.753,3.662)--(6.756,3.664)--(6.759,3.665)--(6.762,3.667)--(6.765,3.669)%
  --(6.768,3.671)--(6.771,3.672)--(6.774,3.674)--(6.777,3.676)--(6.780,3.678)--(6.783,3.679)%
  --(6.786,3.681)--(6.789,3.683)--(6.792,3.685)--(6.795,3.686)--(6.798,3.688)--(6.801,3.690)%
  --(6.804,3.692)--(6.807,3.693)--(6.810,3.695)--(6.813,3.697)--(6.816,3.699)--(6.819,3.700)%
  --(6.822,3.702)--(6.825,3.704)--(6.828,3.706)--(6.831,3.707)--(6.834,3.709)--(6.837,3.711)%
  --(6.840,3.713)--(6.843,3.714)--(6.846,3.716)--(6.848,3.718)--(6.851,3.720)--(6.854,3.721)%
  --(6.857,3.723)--(6.860,3.725)--(6.863,3.727)--(6.866,3.728)--(6.869,3.730)--(6.872,3.732)%
  --(6.875,3.734)--(6.878,3.735)--(6.881,3.737)--(6.884,3.739)--(6.887,3.741)--(6.890,3.742)%
  --(6.893,3.744)--(6.896,3.746)--(6.899,3.748)--(6.902,3.749)--(6.905,3.751)--(6.908,3.753)%
  --(6.911,3.755)--(6.914,3.756)--(6.917,3.758)--(6.920,3.760)--(6.923,3.762)--(6.926,3.763)%
  --(6.929,3.765)--(6.932,3.767)--(6.935,3.769)--(6.938,3.771)--(6.941,3.772)--(6.944,3.774)%
  --(6.947,3.776)--(6.950,3.778)--(6.953,3.779)--(6.956,3.781)--(6.959,3.783)--(6.962,3.785)%
  --(6.965,3.786)--(6.968,3.788)--(6.971,3.790)--(6.974,3.792)--(6.977,3.793)--(6.980,3.795)%
  --(6.983,3.797)--(6.986,3.799)--(6.989,3.800)--(6.992,3.802)--(6.995,3.804)--(6.998,3.806)%
  --(7.001,3.808)--(7.004,3.809)--(7.007,3.811)--(7.010,3.813)--(7.013,3.815)--(7.016,3.816)%
  --(7.019,3.818)--(7.022,3.820)--(7.025,3.822)--(7.028,3.823)--(7.031,3.825)--(7.034,3.827)%
  --(7.037,3.829)--(7.040,3.830)--(7.043,3.832)--(7.046,3.834)--(7.049,3.836)--(7.052,3.838)%
  --(7.055,3.839)--(7.057,3.841)--(7.060,3.843)--(7.063,3.845)--(7.066,3.846)--(7.069,3.848)%
  --(7.072,3.850)--(7.075,3.852)--(7.078,3.853)--(7.081,3.855)--(7.084,3.857)--(7.087,3.859)%
  --(7.090,3.861)--(7.093,3.862)--(7.096,3.864)--(7.099,3.866)--(7.102,3.868)--(7.105,3.869)%
  --(7.108,3.871)--(7.111,3.873)--(7.114,3.875)--(7.117,3.876)--(7.120,3.878)--(7.123,3.880)%
  --(7.126,3.882)--(7.129,3.884)--(7.132,3.885)--(7.135,3.887)--(7.138,3.889)--(7.141,3.891)%
  --(7.144,3.892)--(7.147,3.894)--(7.150,3.896)--(7.153,3.898)--(7.156,3.900)--(7.159,3.901)%
  --(7.162,3.903)--(7.165,3.905)--(7.168,3.907)--(7.171,3.908)--(7.174,3.910)--(7.177,3.912)%
  --(7.180,3.914)--(7.183,3.916)--(7.186,3.917)--(7.189,3.919)--(7.192,3.921)--(7.195,3.923)%
  --(7.198,3.924)--(7.201,3.926)--(7.204,3.928)--(7.207,3.930)--(7.210,3.932)--(7.213,3.933)%
  --(7.216,3.935)--(7.219,3.937)--(7.222,3.939)--(7.225,3.940)--(7.228,3.942)--(7.231,3.944)%
  --(7.234,3.946)--(7.237,3.948)--(7.240,3.949)--(7.243,3.951)--(7.246,3.953)--(7.249,3.955)%
  --(7.252,3.956)--(7.255,3.958)--(7.258,3.960)--(7.261,3.962)--(7.264,3.964)--(7.266,3.965)%
  --(7.269,3.967)--(7.272,3.969)--(7.275,3.971)--(7.278,3.972)--(7.281,3.974)--(7.284,3.976)%
  --(7.287,3.978)--(7.290,3.980)--(7.293,3.981)--(7.296,3.983)--(7.299,3.985)--(7.302,3.987)%
  --(7.305,3.989)--(7.308,3.990)--(7.311,3.992)--(7.314,3.994)--(7.317,3.996)--(7.320,3.997)%
  --(7.323,3.999)--(7.326,4.001)--(7.329,4.003)--(7.332,4.005)--(7.335,4.006)--(7.338,4.008)%
  --(7.341,4.010)--(7.344,4.012)--(7.347,4.013)--(7.350,4.015)--(7.353,4.017)--(7.356,4.019)%
  --(7.359,4.021)--(7.362,4.022)--(7.365,4.024)--(7.368,4.026)--(7.371,4.028)--(7.374,4.030)%
  --(7.377,4.031)--(7.380,4.033)--(7.383,4.035)--(7.386,4.037)--(7.389,4.039)--(7.392,4.040)%
  --(7.395,4.042)--(7.398,4.044)--(7.401,4.046)--(7.404,4.047)--(7.407,4.049)--(7.410,4.051)%
  --(7.413,4.053)--(7.416,4.055)--(7.419,4.056)--(7.422,4.058)--(7.425,4.060)--(7.428,4.062)%
  --(7.431,4.064)--(7.434,4.065)--(7.437,4.067)--(7.440,4.069)--(7.443,4.071)--(7.446,4.073)%
  --(7.449,4.074)--(7.452,4.076)--(7.455,4.078)--(7.458,4.080)--(7.461,4.081)--(7.464,4.083)%
  --(7.467,4.085)--(7.470,4.087)--(7.473,4.089)--(7.476,4.090)--(7.478,4.092)--(7.481,4.094)%
  --(7.484,4.096)--(7.487,4.098)--(7.490,4.099)--(7.493,4.101)--(7.496,4.103)--(7.499,4.105)%
  --(7.502,4.107)--(7.505,4.108)--(7.508,4.110)--(7.511,4.112)--(7.514,4.114)--(7.517,4.116)%
  --(7.520,4.117)--(7.523,4.119)--(7.526,4.121)--(7.529,4.123)--(7.532,4.125)--(7.535,4.126)%
  --(7.538,4.128)--(7.541,4.130)--(7.544,4.132)--(7.547,4.134)--(7.550,4.135)--(7.553,4.137)%
  --(7.556,4.139)--(7.559,4.141)--(7.562,4.143)--(7.565,4.144)--(7.568,4.146)--(7.571,4.148)%
  --(7.574,4.150)--(7.577,4.152)--(7.580,4.153)--(7.583,4.155)--(7.586,4.157)--(7.589,4.159)%
  --(7.592,4.160)--(7.595,4.162)--(7.598,4.164)--(7.601,4.166)--(7.604,4.168)--(7.607,4.169)%
  --(7.610,4.171)--(7.613,4.173)--(7.616,4.175)--(7.619,4.177)--(7.622,4.178)--(7.625,4.180)%
  --(7.628,4.182)--(7.631,4.184)--(7.634,4.186)--(7.637,4.188)--(7.640,4.189)--(7.643,4.191)%
  --(7.646,4.193)--(7.649,4.195)--(7.652,4.197)--(7.655,4.198)--(7.658,4.200)--(7.661,4.202)%
  --(7.664,4.204)--(7.667,4.206)--(7.670,4.207)--(7.673,4.209)--(7.676,4.211)--(7.679,4.213)%
  --(7.682,4.215)--(7.685,4.216)--(7.687,4.218)--(7.690,4.220)--(7.693,4.222)--(7.696,4.224)%
  --(7.699,4.225)--(7.702,4.227)--(7.705,4.229)--(7.708,4.231)--(7.711,4.233)--(7.714,4.234)%
  --(7.717,4.236)--(7.720,4.238)--(7.723,4.240)--(7.726,4.242)--(7.729,4.243)--(7.732,4.245)%
  --(7.735,4.247)--(7.738,4.249)--(7.741,4.251)--(7.744,4.252)--(7.747,4.254)--(7.750,4.256)%
  --(7.753,4.258)--(7.756,4.260)--(7.759,4.261)--(7.762,4.263)--(7.765,4.265)--(7.768,4.267)%
  --(7.771,4.269)--(7.774,4.271)--(7.777,4.272)--(7.780,4.274)--(7.783,4.276)--(7.786,4.278)%
  --(7.789,4.280)--(7.792,4.281)--(7.795,4.283)--(7.798,4.285)--(7.801,4.287)--(7.804,4.289)%
  --(7.807,4.290)--(7.810,4.292)--(7.813,4.294)--(7.816,4.296)--(7.819,4.298)--(7.822,4.299)%
  --(7.825,4.301)--(7.828,4.303)--(7.831,4.305)--(7.834,4.307)--(7.837,4.309)--(7.840,4.310)%
  --(7.843,4.312)--(7.846,4.314)--(7.849,4.316)--(7.852,4.318)--(7.855,4.319)--(7.858,4.321)%
  --(7.861,4.323)--(7.864,4.325)--(7.867,4.327)--(7.870,4.328)--(7.873,4.330)--(7.876,4.332)%
  --(7.879,4.334)--(7.882,4.336)--(7.885,4.337)--(7.888,4.339)--(7.891,4.341)--(7.894,4.343)%
  --(7.896,4.345)--(7.899,4.347)--(7.902,4.348)--(7.905,4.350)--(7.908,4.352)--(7.911,4.354)%
  --(7.914,4.356)--(7.917,4.357)--(7.920,4.359)--(7.923,4.361)--(7.926,4.363)--(7.929,4.365)%
  --(7.932,4.367)--(7.935,4.368)--(7.938,4.370)--(7.941,4.372)--(7.944,4.374)--(7.947,4.376)%
  --(7.950,4.377)--(7.953,4.379)--(7.956,4.381)--(7.959,4.383)--(7.962,4.385)--(7.965,4.386)%
  --(7.968,4.388)--(7.971,4.390)--(7.974,4.392)--(7.977,4.394)--(7.980,4.396)--(7.983,4.397)%
  --(7.986,4.399)--(7.989,4.401)--(7.992,4.403)--(7.995,4.405)--(7.998,4.406)--(8.001,4.408)%
  --(8.004,4.410)--(8.007,4.412)--(8.010,4.414)--(8.013,4.416)--(8.016,4.417)--(8.019,4.419)%
  --(8.022,4.421)--(8.025,4.423)--(8.028,4.425)--(8.031,4.426)--(8.034,4.428)--(8.037,4.430)%
  --(8.040,4.432)--(8.043,4.434)--(8.046,4.436)--(8.049,4.437)--(8.052,4.439)--(8.055,4.441)%
  --(8.058,4.443)--(8.061,4.445)--(8.064,4.446)--(8.067,4.448)--(8.070,4.450)--(8.073,4.452)%
  --(8.076,4.454)--(8.079,4.456)--(8.082,4.457)--(8.085,4.459)--(8.088,4.461)--(8.091,4.463)%
  --(8.094,4.465)--(8.097,4.466)--(8.100,4.468)--(8.103,4.470)--(8.105,4.472)--(8.108,4.474)%
  --(8.111,4.476)--(8.114,4.477)--(8.117,4.479)--(8.120,4.481)--(8.123,4.483)--(8.126,4.485)%
  --(8.129,4.487)--(8.132,4.488)--(8.135,4.490)--(8.138,4.492)--(8.141,4.494)--(8.144,4.496)%
  --(8.147,4.497)--(8.150,4.499)--(8.153,4.501)--(8.156,4.503)--(8.159,4.505)--(8.162,4.507)%
  --(8.165,4.508)--(8.168,4.510)--(8.171,4.512)--(8.174,4.514)--(8.177,4.516)--(8.180,4.518)%
  --(8.183,4.519)--(8.186,4.521)--(8.189,4.523)--(8.192,4.525)--(8.195,4.527)--(8.198,4.528)%
  --(8.201,4.530)--(8.204,4.532)--(8.207,4.534)--(8.210,4.536)--(8.213,4.538)--(8.216,4.539)%
  --(8.219,4.541)--(8.222,4.543)--(8.225,4.545)--(8.228,4.547)--(8.231,4.549)--(8.234,4.550)%
  --(8.237,4.552)--(8.240,4.554)--(8.243,4.556)--(8.246,4.558)--(8.249,4.559)--(8.252,4.561)%
  --(8.255,4.563)--(8.258,4.565)--(8.261,4.567)--(8.264,4.569)--(8.267,4.570)--(8.270,4.572)%
  --(8.273,4.574)--(8.276,4.576)--(8.279,4.578)--(8.282,4.580)--(8.285,4.581)--(8.288,4.583)%
  --(8.291,4.585)--(8.294,4.587)--(8.297,4.589)--(8.300,4.591)--(8.303,4.592)--(8.306,4.594)%
  --(8.309,4.596)--(8.312,4.598)--(8.314,4.600)--(8.317,4.602)--(8.320,4.603)--(8.323,4.605)%
  --(8.326,4.607)--(8.329,4.609)--(8.332,4.611)--(8.335,4.613)--(8.338,4.614)--(8.341,4.616)%
  --(8.344,4.618)--(8.347,4.620)--(8.350,4.622)--(8.353,4.623)--(8.356,4.625)--(8.359,4.627)%
  --(8.362,4.629)--(8.365,4.631)--(8.368,4.633)--(8.371,4.634)--(8.374,4.636)--(8.377,4.638)%
  --(8.380,4.640)--(8.383,4.642)--(8.386,4.644)--(8.389,4.645)--(8.392,4.647)--(8.395,4.649)%
  --(8.398,4.651)--(8.401,4.653)--(8.404,4.655)--(8.407,4.656)--(8.410,4.658)--(8.413,4.660)%
  --(8.416,4.662)--(8.419,4.664)--(8.422,4.666)--(8.425,4.667)--(8.428,4.669)--(8.431,4.671)%
  --(8.434,4.673)--(8.437,4.675)--(8.440,4.677)--(8.443,4.678)--(8.446,4.680)--(8.449,4.682)%
  --(8.452,4.684)--(8.455,4.686)--(8.458,4.688)--(8.461,4.689)--(8.464,4.691)--(8.467,4.693)%
  --(8.470,4.695)--(8.473,4.697)--(8.476,4.699)--(8.479,4.700)--(8.482,4.702)--(8.485,4.704)%
  --(8.488,4.706)--(8.491,4.708)--(8.494,4.710)--(8.497,4.711)--(8.500,4.713)--(8.503,4.715)%
  --(8.506,4.717)--(8.509,4.719)--(8.512,4.721)--(8.515,4.722)--(8.518,4.724)--(8.521,4.726)%
  --(8.523,4.728)--(8.526,4.730)--(8.529,4.732)--(8.532,4.733)--(8.535,4.735)--(8.538,4.737)%
  --(8.541,4.739)--(8.544,4.741)--(8.547,4.743)--(8.550,4.744)--(8.553,4.746)--(8.556,4.748)%
  --(8.559,4.750)--(8.562,4.752)--(8.565,4.754)--(8.568,4.755)--(8.571,4.757)--(8.574,4.759)%
  --(8.577,4.761)--(8.580,4.763)--(8.583,4.765)--(8.586,4.767)--(8.589,4.768)--(8.592,4.770)%
  --(8.595,4.772)--(8.598,4.774)--(8.601,4.776)--(8.604,4.778)--(8.607,4.779)--(8.610,4.781)%
  --(8.613,4.783)--(8.616,4.785)--(8.619,4.787)--(8.622,4.789)--(8.625,4.790)--(8.628,4.792)%
  --(8.631,4.794)--(8.634,4.796)--(8.637,4.798)--(8.640,4.800)--(8.643,4.801)--(8.646,4.803)%
  --(8.649,4.805)--(8.652,4.807)--(8.655,4.809)--(8.658,4.811)--(8.661,4.812)--(8.664,4.814)%
  --(8.667,4.816)--(8.670,4.818)--(8.673,4.820)--(8.676,4.822)--(8.679,4.823)--(8.682,4.825)%
  --(8.685,4.827)--(8.688,4.829)--(8.691,4.831)--(8.694,4.833)--(8.697,4.835)--(8.700,4.836)%
  --(8.703,4.838)--(8.706,4.840)--(8.709,4.842)--(8.712,4.844)--(8.715,4.846)--(8.718,4.847)%
  --(8.721,4.849)--(8.724,4.851)--(8.727,4.853)--(8.730,4.855)--(8.733,4.857)--(8.735,4.858)%
  --(8.738,4.860)--(8.741,4.862)--(8.744,4.864)--(8.747,4.866)--(8.750,4.868)--(8.753,4.870)%
  --(8.756,4.871)--(8.759,4.873)--(8.762,4.875)--(8.765,4.877)--(8.768,4.879)--(8.771,4.881)%
  --(8.774,4.882)--(8.777,4.884)--(8.780,4.886)--(8.783,4.888)--(8.786,4.890)--(8.789,4.892)%
  --(8.792,4.893)--(8.795,4.895)--(8.798,4.897)--(8.801,4.899)--(8.804,4.901)--(8.807,4.903)%
  --(8.810,4.905)--(8.813,4.906)--(8.816,4.908)--(8.819,4.910)--(8.822,4.912)--(8.825,4.914)%
  --(8.828,4.916)--(8.831,4.917)--(8.834,4.919)--(8.837,4.921)--(8.840,4.923)--(8.843,4.925)%
  --(8.846,4.927)--(8.849,4.928)--(8.852,4.930)--(8.855,4.932)--(8.858,4.934)--(8.861,4.936)%
  --(8.864,4.938)--(8.867,4.940)--(8.870,4.941)--(8.873,4.943)--(8.876,4.945)--(8.879,4.947)%
  --(8.882,4.949)--(8.885,4.951)--(8.888,4.952)--(8.891,4.954)--(8.894,4.956)--(8.897,4.958)%
  --(8.900,4.960)--(8.903,4.962)--(8.906,4.964)--(8.909,4.965)--(8.912,4.967)--(8.915,4.969)%
  --(8.918,4.971)--(8.921,4.973)--(8.924,4.975)--(8.927,4.976)--(8.930,4.978)--(8.933,4.980)%
  --(8.936,4.982)--(8.939,4.984)--(8.942,4.986)--(8.944,4.988)--(8.947,4.989)--(8.950,4.991)%
  --(8.953,4.993)--(8.956,4.995)--(8.959,4.997)--(8.962,4.999)--(8.965,5.000)--(8.968,5.002)%
  --(8.971,5.004)--(8.974,5.006)--(8.977,5.008)--(8.980,5.010)--(8.983,5.012)--(8.986,5.013)%
  --(8.989,5.015)--(8.992,5.017)--(8.995,5.019)--(8.998,5.021)--(9.001,5.023)--(9.004,5.024)%
  --(9.007,5.026)--(9.010,5.028)--(9.013,5.030)--(9.016,5.032)--(9.019,5.034)--(9.022,5.036)%
  --(9.025,5.037)--(9.028,5.039)--(9.031,5.041)--(9.034,5.043)--(9.037,5.045)--(9.040,5.047)%
  --(9.043,5.049)--(9.046,5.050)--(9.049,5.052)--(9.052,5.054)--(9.055,5.056)--(9.058,5.058)%
  --(9.061,5.060)--(9.064,5.061)--(9.067,5.063)--(9.070,5.065)--(9.073,5.067)--(9.076,5.069)%
  --(9.079,5.071)--(9.082,5.073)--(9.085,5.074)--(9.088,5.076)--(9.091,5.078)--(9.094,5.080)%
  --(9.097,5.082)--(9.100,5.084)--(9.103,5.086)--(9.106,5.087)--(9.109,5.089)--(9.112,5.091)%
  --(9.115,5.093)--(9.118,5.095)--(9.121,5.097)--(9.124,5.098)--(9.127,5.100)--(9.130,5.102)%
  --(9.133,5.104)--(9.136,5.106)--(9.139,5.108)--(9.142,5.110)--(9.145,5.111)--(9.148,5.113)%
  --(9.151,5.115)--(9.153,5.117)--(9.156,5.119)--(9.159,5.121)--(9.162,5.123)--(9.165,5.124)%
  --(9.168,5.126)--(9.171,5.128)--(9.174,5.130)--(9.177,5.132)--(9.180,5.134)--(9.183,5.136)%
  --(9.186,5.137)--(9.189,5.139)--(9.192,5.141)--(9.195,5.143)--(9.198,5.145)--(9.201,5.147)%
  --(9.204,5.148)--(9.207,5.150)--(9.210,5.152)--(9.213,5.154)--(9.216,5.156)--(9.219,5.158)%
  --(9.222,5.160)--(9.225,5.161)--(9.228,5.163)--(9.231,5.165)--(9.234,5.167)--(9.237,5.169)%
  --(9.240,5.171)--(9.243,5.173)--(9.246,5.174)--(9.249,5.176)--(9.252,5.178)--(9.255,5.180)%
  --(9.258,5.182)--(9.261,5.184)--(9.264,5.186)--(9.267,5.187)--(9.270,5.189)--(9.273,5.191)%
  --(9.276,5.193)--(9.279,5.195)--(9.282,5.197)--(9.285,5.199)--(9.288,5.200)--(9.291,5.202)%
  --(9.294,5.204)--(9.297,5.206)--(9.300,5.208)--(9.303,5.210)--(9.306,5.211)--(9.309,5.213)%
  --(9.312,5.215)--(9.315,5.217)--(9.318,5.219)--(9.321,5.221)--(9.324,5.223)--(9.327,5.224)%
  --(9.330,5.226)--(9.333,5.228)--(9.336,5.230)--(9.339,5.232)--(9.342,5.234)--(9.345,5.236)%
  --(9.348,5.237)--(9.351,5.239)--(9.354,5.241)--(9.357,5.243)--(9.360,5.245)--(9.362,5.247)%
  --(9.365,5.249)--(9.368,5.250)--(9.371,5.252)--(9.374,5.254)--(9.377,5.256)--(9.380,5.258)%
  --(9.383,5.260)--(9.386,5.262)--(9.389,5.263)--(9.392,5.265)--(9.395,5.267)--(9.398,5.269)%
  --(9.401,5.271)--(9.404,5.273)--(9.407,5.275)--(9.410,5.276)--(9.413,5.278)--(9.416,5.280)%
  --(9.419,5.282)--(9.422,5.284)--(9.425,5.286)--(9.428,5.288)--(9.431,5.289)--(9.434,5.291)%
  --(9.437,5.293)--(9.440,5.295)--(9.443,5.297)--(9.446,5.299)--(9.449,5.301)--(9.452,5.302)%
  --(9.455,5.304)--(9.458,5.306)--(9.461,5.308)--(9.464,5.310)--(9.467,5.312)--(9.470,5.314)%
  --(9.473,5.315)--(9.476,5.317)--(9.479,5.319)--(9.482,5.321)--(9.485,5.323)--(9.488,5.325)%
  --(9.491,5.327)--(9.494,5.328)--(9.497,5.330)--(9.500,5.332)--(9.503,5.334)--(9.506,5.336)%
  --(9.509,5.338)--(9.512,5.340)--(9.515,5.341)--(9.518,5.343)--(9.521,5.345)--(9.524,5.347)%
  --(9.527,5.349)--(9.530,5.351)--(9.533,5.353)--(9.536,5.354)--(9.539,5.356)--(9.542,5.358)%
  --(9.545,5.360)--(9.548,5.362)--(9.551,5.364)--(9.554,5.366)--(9.557,5.367)--(9.560,5.369)%
  --(9.563,5.371)--(9.566,5.373)--(9.569,5.375)--(9.571,5.377)--(9.574,5.379)--(9.577,5.381)%
  --(9.580,5.382)--(9.583,5.384)--(9.586,5.386)--(9.589,5.388)--(9.592,5.390)--(9.595,5.392)%
  --(9.598,5.394)--(9.601,5.395)--(9.604,5.397)--(9.607,5.399)--(9.610,5.401)--(9.613,5.403)%
  --(9.616,5.405)--(9.619,5.407)--(9.622,5.408)--(9.625,5.410)--(9.628,5.412)--(9.631,5.414)%
  --(9.634,5.416)--(9.637,5.418)--(9.640,5.420)--(9.643,5.421)--(9.646,5.423)--(9.649,5.425)%
  --(9.652,5.427)--(9.655,5.429)--(9.658,5.431)--(9.661,5.433)--(9.664,5.434)--(9.667,5.436)%
  --(9.670,5.438)--(9.673,5.440)--(9.676,5.442)--(9.679,5.444)--(9.682,5.446)--(9.685,5.448)%
  --(9.688,5.449)--(9.691,5.451)--(9.694,5.453)--(9.697,5.455)--(9.700,5.457)--(9.703,5.459)%
  --(9.706,5.461)--(9.709,5.462)--(9.712,5.464)--(9.715,5.466)--(9.718,5.468)--(9.721,5.470)%
  --(9.724,5.472)--(9.727,5.474)--(9.730,5.475)--(9.733,5.477)--(9.736,5.479)--(9.739,5.481)%
  --(9.742,5.483)--(9.745,5.485)--(9.748,5.487)--(9.751,5.488)--(9.754,5.490)--(9.757,5.492)%
  --(9.760,5.494)--(9.763,5.496)--(9.766,5.498)--(9.769,5.500)--(9.772,5.502)--(9.775,5.503)%
  --(9.778,5.505)--(9.780,5.507)--(9.783,5.509)--(9.786,5.511)--(9.789,5.513)--(9.792,5.515)%
  --(9.795,5.516)--(9.798,5.518)--(9.801,5.520)--(9.804,5.522)--(9.807,5.524)--(9.810,5.526)%
  --(9.813,5.528)--(9.816,5.529)--(9.819,5.531)--(9.822,5.533)--(9.825,5.535)--(9.828,5.537)%
  --(9.831,5.539)--(9.834,5.541)--(9.837,5.543)--(9.840,5.544)--(9.843,5.546)--(9.846,5.548)%
  --(9.849,5.550)--(9.852,5.552)--(9.855,5.554)--(9.858,5.556)--(9.861,5.557)--(9.864,5.559)%
  --(9.867,5.561)--(9.870,5.563)--(9.873,5.565)--(9.876,5.567)--(9.879,5.569)--(9.882,5.570)%
  --(9.885,5.572)--(9.888,5.574)--(9.891,5.576)--(9.894,5.578)--(9.897,5.580)--(9.900,5.582)%
  --(9.903,5.584)--(9.906,5.585)--(9.909,5.587)--(9.912,5.589)--(9.915,5.591)--(9.918,5.593)%
  --(9.921,5.595)--(9.924,5.597)--(9.927,5.598)--(9.930,5.600)--(9.933,5.602)--(9.936,5.604)%
  --(9.939,5.606)--(9.942,5.608)--(9.945,5.610)--(9.948,5.612)--(9.951,5.613)--(9.954,5.615)%
  --(9.957,5.617)--(9.960,5.619)--(9.963,5.621)--(9.966,5.623)--(9.969,5.625)--(9.972,5.626)%
  --(9.975,5.628)--(9.978,5.630)--(9.981,5.632)--(9.984,5.634)--(9.987,5.636)--(9.990,5.638)%
  --(9.992,5.640)--(9.995,5.641)--(9.998,5.643)--(10.001,5.645)--(10.004,5.647)--(10.007,5.649)%
  --(10.010,5.651)--(10.013,5.653)--(10.016,5.654)--(10.019,5.656)--(10.022,5.658)--(10.025,5.660)%
  --(10.028,5.662)--(10.031,5.664)--(10.034,5.666)--(10.037,5.668)--(10.040,5.669)--(10.043,5.671)%
  --(10.046,5.673)--(10.049,5.675)--(10.052,5.677)--(10.055,5.679)--(10.058,5.681)--(10.061,5.682)%
  --(10.064,5.684)--(10.067,5.686)--(10.070,5.688)--(10.073,5.690)--(10.076,5.692)--(10.079,5.694)%
  --(10.082,5.696)--(10.085,5.697)--(10.088,5.699)--(10.091,5.701)--(10.094,5.703)--(10.097,5.705)%
  --(10.100,5.707)--(10.103,5.709)--(10.106,5.710)--(10.109,5.712)--(10.112,5.714)--(10.115,5.716)%
  --(10.118,5.718)--(10.121,5.720)--(10.124,5.722)--(10.127,5.724)--(10.130,5.725)--(10.133,5.727)%
  --(10.136,5.729)--(10.139,5.731)--(10.142,5.733)--(10.145,5.735)--(10.148,5.737)--(10.151,5.739)%
  --(10.154,5.740)--(10.157,5.742)--(10.160,5.744)--(10.163,5.746)--(10.166,5.748)--(10.169,5.750)%
  --(10.172,5.752)--(10.175,5.753)--(10.178,5.755)--(10.181,5.757)--(10.184,5.759)--(10.187,5.761)%
  --(10.190,5.763)--(10.193,5.765)--(10.196,5.767)--(10.199,5.768)--(10.201,5.770)--(10.204,5.772)%
  --(10.207,5.774)--(10.210,5.776)--(10.213,5.778)--(10.216,5.780)--(10.219,5.782)--(10.222,5.783)%
  --(10.225,5.785)--(10.228,5.787)--(10.231,5.789)--(10.234,5.791)--(10.237,5.793)--(10.240,5.795)%
  --(10.243,5.796)--(10.246,5.798)--(10.249,5.800)--(10.252,5.802)--(10.255,5.804)--(10.258,5.806)%
  --(10.261,5.808)--(10.264,5.810)--(10.267,5.811)--(10.270,5.813)--(10.273,5.815)--(10.276,5.817)%
  --(10.279,5.819)--(10.282,5.821)--(10.285,5.823)--(10.288,5.825)--(10.291,5.826)--(10.294,5.828)%
  --(10.297,5.830)--(10.300,5.832)--(10.303,5.834)--(10.306,5.836)--(10.309,5.838)--(10.312,5.840)%
  --(10.315,5.841)--(10.318,5.843)--(10.321,5.845)--(10.324,5.847)--(10.327,5.849)--(10.330,5.851)%
  --(10.333,5.853)--(10.336,5.854)--(10.339,5.856)--(10.342,5.858)--(10.345,5.860)--(10.348,5.862)%
  --(10.351,5.864)--(10.354,5.866)--(10.357,5.868)--(10.360,5.869)--(10.363,5.871)--(10.366,5.873)%
  --(10.369,5.875)--(10.372,5.877)--(10.375,5.879)--(10.378,5.881)--(10.381,5.883)--(10.384,5.884)%
  --(10.387,5.886)--(10.390,5.888)--(10.393,5.890)--(10.396,5.892)--(10.399,5.894)--(10.402,5.896)%
  --(10.405,5.898)--(10.408,5.899)--(10.410,5.901)--(10.413,5.903)--(10.416,5.905)--(10.419,5.907)%
  --(10.422,5.909)--(10.425,5.911)--(10.428,5.913)--(10.431,5.914)--(10.434,5.916)--(10.437,5.918)%
  --(10.440,5.920)--(10.443,5.922)--(10.446,5.924)--(10.449,5.926)--(10.452,5.927)--(10.455,5.929)%
  --(10.458,5.931)--(10.461,5.933)--(10.464,5.935)--(10.467,5.937)--(10.470,5.939)--(10.473,5.941)%
  --(10.476,5.942)--(10.479,5.944)--(10.482,5.946)--(10.485,5.948)--(10.488,5.950)--(10.491,5.952)%
  --(10.494,5.954)--(10.497,5.956)--(10.500,5.957)--(10.503,5.959)--(10.506,5.961)--(10.509,5.963)%
  --(10.512,5.965)--(10.515,5.967)--(10.518,5.969)--(10.521,5.971)--(10.524,5.972)--(10.527,5.974)%
  --(10.530,5.976)--(10.533,5.978)--(10.536,5.980)--(10.539,5.982)--(10.542,5.984)--(10.545,5.986)%
  --(10.548,5.987)--(10.551,5.989)--(10.554,5.991)--(10.557,5.993)--(10.560,5.995)--(10.563,5.997)%
  --(10.566,5.999)--(10.569,6.001)--(10.572,6.002)--(10.575,6.004)--(10.578,6.006)--(10.581,6.008)%
  --(10.584,6.010)--(10.587,6.012)--(10.590,6.014)--(10.593,6.016)--(10.596,6.017)--(10.599,6.019)%
  --(10.602,6.021)--(10.605,6.023)--(10.608,6.025)--(10.611,6.027)--(10.614,6.029)--(10.617,6.031)%
  --(10.619,6.032)--(10.622,6.034)--(10.625,6.036)--(10.628,6.038)--(10.631,6.040)--(10.634,6.042)%
  --(10.637,6.044)--(10.640,6.046)--(10.643,6.047)--(10.646,6.049)--(10.649,6.051)--(10.652,6.053)%
  --(10.655,6.055)--(10.658,6.057)--(10.661,6.059)--(10.664,6.061)--(10.667,6.062)--(10.670,6.064)%
  --(10.673,6.066)--(10.676,6.068)--(10.679,6.070)--(10.682,6.072)--(10.685,6.074)--(10.688,6.076)%
  --(10.691,6.077)--(10.694,6.079)--(10.697,6.081)--(10.700,6.083)--(10.703,6.085)--(10.706,6.087)%
  --(10.709,6.089)--(10.712,6.091)--(10.715,6.092)--(10.718,6.094)--(10.721,6.096)--(10.724,6.098)%
  --(10.727,6.100)--(10.730,6.102)--(10.733,6.104)--(10.736,6.106)--(10.739,6.107)--(10.742,6.109)%
  --(10.745,6.111)--(10.748,6.113)--(10.751,6.115)--(10.754,6.117)--(10.757,6.119)--(10.760,6.121)%
  --(10.763,6.122)--(10.766,6.124)--(10.769,6.126)--(10.772,6.128)--(10.775,6.130)--(10.778,6.132)%
  --(10.781,6.134)--(10.784,6.136)--(10.787,6.137)--(10.790,6.139)--(10.793,6.141)--(10.796,6.143)%
  --(10.799,6.145)--(10.802,6.147)--(10.805,6.149)--(10.808,6.151)--(10.811,6.152)--(10.814,6.154)%
  --(10.817,6.156)--(10.820,6.158)--(10.823,6.160)--(10.826,6.162)--(10.828,6.164)--(10.831,6.166)%
  --(10.834,6.168)--(10.837,6.169)--(10.840,6.171)--(10.843,6.173)--(10.846,6.175)--(10.849,6.177)%
  --(10.852,6.179)--(10.855,6.181)--(10.858,6.183)--(10.861,6.184)--(10.864,6.186)--(10.867,6.188)%
  --(10.870,6.190)--(10.873,6.192)--(10.876,6.194)--(10.879,6.196)--(10.882,6.198)--(10.885,6.199)%
  --(10.888,6.201)--(10.891,6.203)--(10.894,6.205)--(10.897,6.207)--(10.900,6.209)--(10.903,6.211)%
  --(10.906,6.213)--(10.909,6.214)--(10.912,6.216)--(10.915,6.218)--(10.918,6.220)--(10.921,6.222)%
  --(10.924,6.224)--(10.927,6.226)--(10.930,6.228)--(10.933,6.229)--(10.936,6.231)--(10.939,6.233)%
  --(10.942,6.235)--(10.945,6.237)--(10.948,6.239)--(10.951,6.241)--(10.954,6.243)--(10.957,6.245)%
  --(10.960,6.246)--(10.963,6.248)--(10.966,6.250)--(10.969,6.252)--(10.972,6.254)--(10.975,6.256)%
  --(10.978,6.258)--(10.981,6.260)--(10.984,6.261)--(10.987,6.263)--(10.990,6.265)--(10.993,6.267)%
  --(10.996,6.269)--(10.999,6.271)--(11.002,6.273)--(11.005,6.275)--(11.008,6.276)--(11.011,6.278)%
  --(11.014,6.280)--(11.017,6.282)--(11.020,6.284)--(11.023,6.286)--(11.026,6.288)--(11.029,6.290)%
  --(11.032,6.291)--(11.035,6.293)--(11.037,6.295)--(11.040,6.297)--(11.043,6.299)--(11.046,6.301)%
  --(11.049,6.303)--(11.052,6.305)--(11.055,6.307)--(11.058,6.308)--(11.061,6.310)--(11.064,6.312)%
  --(11.067,6.314)--(11.070,6.316)--(11.073,6.318)--(11.076,6.320)--(11.079,6.322)--(11.082,6.323)%
  --(11.085,6.325)--(11.088,6.327)--(11.091,6.329)--(11.094,6.331)--(11.097,6.333)--(11.100,6.335)%
  --(11.103,6.337)--(11.106,6.338)--(11.109,6.340)--(11.112,6.342)--(11.115,6.344)--(11.118,6.346)%
  --(11.121,6.348)--(11.124,6.350)--(11.127,6.352)--(11.130,6.354)--(11.133,6.355)--(11.136,6.357)%
  --(11.139,6.359)--(11.142,6.361)--(11.145,6.363)--(11.148,6.365)--(11.151,6.367)--(11.154,6.369)%
  --(11.157,6.370)--(11.160,6.372)--(11.163,6.374)--(11.166,6.376)--(11.169,6.378)--(11.172,6.380)%
  --(11.175,6.382)--(11.178,6.384)--(11.181,6.385)--(11.184,6.387)--(11.187,6.389)--(11.190,6.391)%
  --(11.193,6.393)--(11.196,6.395)--(11.199,6.397)--(11.202,6.399)--(11.205,6.401)--(11.208,6.402)%
  --(11.211,6.404)--(11.214,6.406)--(11.217,6.408)--(11.220,6.410)--(11.223,6.412)--(11.226,6.414)%
  --(11.229,6.416)--(11.232,6.417)--(11.235,6.419)--(11.238,6.421)--(11.241,6.423)--(11.244,6.425)%
  --(11.247,6.427)--(11.249,6.429)--(11.252,6.431)--(11.255,6.433)--(11.258,6.434)--(11.261,6.436)%
  --(11.264,6.438)--(11.267,6.440)--(11.270,6.442)--(11.273,6.444)--(11.276,6.446)--(11.279,6.448)%
  --(11.282,6.449)--(11.285,6.451)--(11.288,6.453)--(11.291,6.455)--(11.294,6.457)--(11.297,6.459)%
  --(11.300,6.461)--(11.303,6.463)--(11.306,6.464)--(11.309,6.466)--(11.312,6.468)--(11.315,6.470)%
  --(11.318,6.472)--(11.321,6.474)--(11.324,6.476)--(11.327,6.478)--(11.330,6.480)--(11.333,6.481)%
  --(11.336,6.483)--(11.339,6.485)--(11.342,6.487)--(11.345,6.489)--(11.348,6.491)--(11.351,6.493)%
  --(11.354,6.495)--(11.357,6.496)--(11.360,6.498)--(11.363,6.500)--(11.366,6.502)--(11.369,6.504)%
  --(11.372,6.506)--(11.375,6.508)--(11.378,6.510)--(11.381,6.512)--(11.384,6.513)--(11.387,6.515)%
  --(11.390,6.517)--(11.393,6.519)--(11.396,6.521)--(11.399,6.523)--(11.402,6.525)--(11.405,6.527)%
  --(11.408,6.528)--(11.411,6.530)--(11.414,6.532)--(11.417,6.534)--(11.420,6.536)--(11.423,6.538)%
  --(11.426,6.540)--(11.429,6.542)--(11.432,6.544)--(11.435,6.545)--(11.438,6.547)--(11.441,6.549)%
  --(11.444,6.551)--(11.447,6.553)--(11.450,6.555)--(11.453,6.557)--(11.456,6.559)--(11.458,6.560)%
  --(11.461,6.562)--(11.464,6.564)--(11.467,6.566)--(11.470,6.568)--(11.473,6.570)--(11.476,6.572)%
  --(11.479,6.574)--(11.482,6.576)--(11.485,6.577)--(11.488,6.579)--(11.491,6.581)--(11.494,6.583)%
  --(11.497,6.585)--(11.500,6.587)--(11.503,6.589)--(11.506,6.591)--(11.509,6.593)--(11.512,6.594)%
  --(11.515,6.596)--(11.518,6.598)--(11.521,6.600)--(11.524,6.602)--(11.527,6.604)--(11.530,6.606)%
  --(11.533,6.608)--(11.536,6.609)--(11.539,6.611)--(11.542,6.613)--(11.545,6.615)--(11.548,6.617)%
  --(11.551,6.619)--(11.554,6.621)--(11.557,6.623)--(11.560,6.625)--(11.563,6.626)--(11.566,6.628)%
  --(11.569,6.630)--(11.572,6.632)--(11.575,6.634)--(11.578,6.636)--(11.581,6.638)--(11.584,6.640)%
  --(11.587,6.641)--(11.590,6.643)--(11.593,6.645)--(11.596,6.647)--(11.599,6.649)--(11.602,6.651)%
  --(11.605,6.653)--(11.608,6.655)--(11.611,6.657)--(11.614,6.658)--(11.617,6.660)--(11.620,6.662)%
  --(11.623,6.664)--(11.626,6.666)--(11.629,6.668)--(11.632,6.670)--(11.635,6.672)--(11.638,6.674)%
  --(11.641,6.675)--(11.644,6.677)--(11.647,6.679)--(11.650,6.681)--(11.653,6.683)--(11.656,6.685)%
  --(11.659,6.687)--(11.662,6.689)--(11.665,6.690)--(11.667,6.692)--(11.670,6.694)--(11.673,6.696)%
  --(11.676,6.698)--(11.679,6.700)--(11.682,6.702)--(11.685,6.704)--(11.688,6.706)--(11.691,6.707)%
  --(11.694,6.709)--(11.697,6.711)--(11.700,6.713)--(11.703,6.715)--(11.706,6.717)--(11.709,6.719)%
  --(11.712,6.721)--(11.715,6.723)--(11.718,6.724)--(11.721,6.726)--(11.724,6.728)--(11.727,6.730)%
  --(11.730,6.732)--(11.733,6.734)--(11.736,6.736)--(11.739,6.738)--(11.742,6.739)--(11.745,6.741)%
  --(11.748,6.743)--(11.751,6.745)--(11.754,6.747)--(11.757,6.749)--(11.760,6.751)--(11.763,6.753)%
  --(11.766,6.755)--(11.769,6.756)--(11.772,6.758)--(11.775,6.760)--(11.778,6.762)--(11.781,6.764)%
  --(11.784,6.766)--(11.787,6.768)--(11.790,6.770)--(11.793,6.772)--(11.796,6.773)--(11.799,6.775)%
  --(11.802,6.777)--(11.805,6.779)--(11.808,6.781)--(11.811,6.783)--(11.814,6.785)--(11.817,6.787)%
  --(11.820,6.789)--(11.823,6.790)--(11.826,6.792)--(11.829,6.794)--(11.832,6.796)--(11.835,6.798)%
  --(11.838,6.800)--(11.841,6.802)--(11.844,6.804)--(11.847,6.805)--(11.850,6.807)--(11.853,6.809)%
  --(11.856,6.811)--(11.859,6.813)--(11.862,6.815)--(11.865,6.817)--(11.868,6.819)--(11.871,6.821)%
  --(11.874,6.822)--(11.876,6.824)--(11.879,6.826)--(11.882,6.828)--(11.885,6.830)--(11.888,6.832)%
  --(11.891,6.834)--(11.894,6.836)--(11.897,6.838)--(11.900,6.839)--(11.903,6.841)--(11.906,6.843)%
  --(11.909,6.845)--(11.912,6.847)--(11.915,6.849)--(11.918,6.851)--(11.921,6.853)--(11.924,6.855)%
  --(11.927,6.856)--(11.930,6.858)--(11.933,6.860)--(11.936,6.862)--(11.939,6.864)--(11.942,6.866)%
  --(11.945,6.868)--(11.948,6.870)--(11.951,6.872)--(11.954,6.873)--(11.957,6.875)--(11.960,6.877)%
  --(11.963,6.879)--(11.966,6.881)--(11.969,6.883)--(11.972,6.885)--(11.975,6.887)--(11.978,6.888)%
  --(11.981,6.890)--(11.984,6.892)--(11.987,6.894)--(11.990,6.896)--(11.993,6.898)--(11.996,6.900)%
  --(11.999,6.902)--(12.002,6.904)--(12.005,6.905)--(12.008,6.907)--(12.011,6.909)--(12.014,6.911)%
  --(12.017,6.913)--(12.020,6.915)--(12.023,6.917)--(12.026,6.919)--(12.029,6.921)--(12.032,6.922)%
  --(12.035,6.924)--(12.038,6.926)--(12.041,6.928)--(12.044,6.930)--(12.047,6.932)--(12.050,6.934)%
  --(12.053,6.936)--(12.056,6.938)--(12.059,6.939)--(12.062,6.941)--(12.065,6.943)--(12.068,6.945)%
  --(12.071,6.947)--(12.074,6.949)--(12.077,6.951)--(12.080,6.953)--(12.083,6.955)--(12.085,6.956)%
  --(12.088,6.958)--(12.091,6.960)--(12.094,6.962)--(12.097,6.964)--(12.100,6.966)--(12.103,6.968)%
  --(12.106,6.970)--(12.109,6.972)--(12.112,6.973)--(12.115,6.975)--(12.118,6.977)--(12.121,6.979)%
  --(12.124,6.981)--(12.127,6.983)--(12.130,6.985)--(12.133,6.987)--(12.136,6.989)--(12.139,6.990)%
  --(12.142,6.992)--(12.145,6.994)--(12.148,6.996)--(12.151,6.998)--(12.154,7.000)--(12.157,7.002)%
  --(12.160,7.004)--(12.163,7.006)--(12.166,7.007)--(12.169,7.009)--(12.172,7.011)--(12.175,7.013)%
  --(12.178,7.015)--(12.181,7.017)--(12.184,7.019)--(12.187,7.021)--(12.190,7.023)--(12.193,7.024)%
  --(12.196,7.026)--(12.199,7.028)--(12.202,7.030)--(12.205,7.032)--(12.208,7.034)--(12.211,7.036)%
  --(12.214,7.038)--(12.217,7.040)--(12.220,7.041)--(12.223,7.043)--(12.226,7.045)--(12.229,7.047)%
  --(12.232,7.049)--(12.235,7.051)--(12.238,7.053)--(12.241,7.055)--(12.244,7.057)--(12.247,7.058)%
  --(12.250,7.060)--(12.253,7.062)--(12.256,7.064)--(12.259,7.066)--(12.262,7.068)--(12.265,7.070)%
  --(12.268,7.072)--(12.271,7.074)--(12.274,7.075)--(12.277,7.077)--(12.280,7.079)--(12.283,7.081)%
  --(12.286,7.083)--(12.289,7.085)--(12.292,7.087)--(12.295,7.089)--(12.297,7.091)--(12.300,7.092)%
  --(12.303,7.094)--(12.306,7.096)--(12.309,7.098)--(12.312,7.100)--(12.315,7.102)--(12.318,7.104)%
  --(12.321,7.106)--(12.324,7.108)--(12.327,7.109)--(12.330,7.111)--(12.333,7.113)--(12.336,7.115)%
  --(12.339,7.117)--(12.342,7.119)--(12.345,7.121)--(12.348,7.123)--(12.351,7.125)--(12.354,7.126)%
  --(12.357,7.128)--(12.360,7.130)--(12.363,7.132)--(12.366,7.134)--(12.369,7.136)--(12.372,7.138)%
  --(12.375,7.140)--(12.378,7.142)--(12.381,7.143)--(12.384,7.145)--(12.387,7.147)--(12.390,7.149)%
  --(12.393,7.151)--(12.396,7.153)--(12.399,7.155)--(12.402,7.157)--(12.405,7.159)--(12.408,7.160)%
  --(12.411,7.162)--(12.414,7.164)--(12.417,7.166)--(12.420,7.168)--(12.423,7.170)--(12.426,7.172)%
  --(12.429,7.174)--(12.432,7.176)--(12.435,7.177)--(12.438,7.179)--(12.441,7.181)--(12.444,7.183)%
  --(12.447,7.185)--(12.450,7.187)--(12.453,7.189)--(12.456,7.191)--(12.459,7.193)--(12.462,7.194)%
  --(12.465,7.196)--(12.468,7.198)--(12.471,7.200)--(12.474,7.202)--(12.477,7.204)--(12.480,7.206)%
  --(12.483,7.208)--(12.486,7.210)--(12.489,7.211)--(12.492,7.213)--(12.495,7.215)--(12.498,7.217)%
  --(12.501,7.219)--(12.504,7.221)--(12.506,7.223)--(12.509,7.225)--(12.512,7.227)--(12.515,7.228)%
  --(12.518,7.230)--(12.521,7.232)--(12.524,7.234)--(12.527,7.236)--(12.530,7.238)--(12.533,7.240)%
  --(12.536,7.242)--(12.539,7.244)--(12.542,7.245)--(12.545,7.247)--(12.548,7.249)--(12.551,7.251)%
  --(12.554,7.253)--(12.557,7.255)--(12.560,7.257)--(12.563,7.259)--(12.566,7.261)--(12.569,7.262)%
  --(12.572,7.264)--(12.575,7.266)--(12.578,7.268)--(12.581,7.270)--(12.584,7.272)--(12.587,7.274)%
  --(12.590,7.276)--(12.593,7.278)--(12.596,7.279)--(12.599,7.281)--(12.602,7.283)--(12.605,7.285)%
  --(12.608,7.287)--(12.611,7.289)--(12.614,7.291)--(12.617,7.293)--(12.620,7.295)--(12.623,7.297)%
  --(12.626,7.298)--(12.629,7.300)--(12.632,7.302)--(12.635,7.304)--(12.638,7.306)--(12.641,7.308)%
  --(12.644,7.310)--(12.647,7.312)--(12.650,7.314)--(12.653,7.315)--(12.656,7.317)--(12.659,7.319)%
  --(12.662,7.321)--(12.665,7.323)--(12.668,7.325)--(12.671,7.327)--(12.674,7.329)--(12.677,7.331)%
  --(12.680,7.332)--(12.683,7.334)--(12.686,7.336)--(12.689,7.338)--(12.692,7.340)--(12.695,7.342)%
  --(12.698,7.344)--(12.701,7.346)--(12.704,7.348)--(12.707,7.349)--(12.710,7.351)--(12.713,7.353)%
  --(12.715,7.355)--(12.718,7.357)--(12.721,7.359)--(12.724,7.361)--(12.727,7.363)--(12.730,7.365)%
  --(12.733,7.366)--(12.736,7.368)--(12.739,7.370)--(12.742,7.372)--(12.745,7.374)--(12.748,7.376)%
  --(12.751,7.378)--(12.754,7.380)--(12.757,7.382)--(12.760,7.384)--(12.763,7.385)--(12.766,7.387)%
  --(12.769,7.389)--(12.772,7.391)--(12.775,7.393)--(12.778,7.395)--(12.781,7.397)--(12.784,7.399)%
  --(12.787,7.401)--(12.790,7.402)--(12.793,7.404)--(12.796,7.406)--(12.799,7.408)--(12.802,7.410)%
  --(12.805,7.412)--(12.808,7.414)--(12.811,7.416)--(12.814,7.418)--(12.817,7.419)--(12.820,7.421)%
  --(12.823,7.423)--(12.826,7.425)--(12.829,7.427)--(12.832,7.429)--(12.835,7.431)--(12.838,7.433)%
  --(12.841,7.435)--(12.844,7.436)--(12.847,7.438)--(12.850,7.440)--(12.853,7.442)--(12.856,7.444)%
  --(12.859,7.446)--(12.862,7.448)--(12.865,7.450)--(12.868,7.452)--(12.871,7.454)--(12.874,7.455)%
  --(12.877,7.457)--(12.880,7.459)--(12.883,7.461)--(12.886,7.463)--(12.889,7.465)--(12.892,7.467)%
  --(12.895,7.469)--(12.898,7.471)--(12.901,7.472)--(12.904,7.474)--(12.907,7.476)--(12.910,7.478)%
  --(12.913,7.480)--(12.916,7.482)--(12.919,7.484)--(12.922,7.486)--(12.924,7.488)--(12.927,7.489)%
  --(12.930,7.491)--(12.933,7.493)--(12.936,7.495)--(12.939,7.497)--(12.942,7.499)--(12.945,7.501)%
  --(12.948,7.503)--(12.951,7.505)--(12.954,7.506)--(12.957,7.508)--(12.960,7.510)--(12.963,7.512)%
  --(12.966,7.514)--(12.969,7.516)--(12.972,7.518)--(12.975,7.520)--(12.978,7.522)--(12.981,7.524)%
  --(12.984,7.525)--(12.987,7.527)--(12.990,7.529)--(12.993,7.531)--(12.996,7.533)--(12.999,7.535)%
  --(13.002,7.537)--(13.005,7.539)--(13.008,7.541)--(13.011,7.542)--(13.014,7.544)--(13.017,7.546)%
  --(13.020,7.548)--(13.023,7.550)--(13.026,7.552)--(13.029,7.554)--(13.032,7.556)--(13.035,7.558)%
  --(13.038,7.559)--(13.041,7.561)--(13.044,7.563)--(13.047,7.565)--(13.050,7.567)--(13.053,7.569)%
  --(13.056,7.571)--(13.059,7.573)--(13.062,7.575)--(13.065,7.577)--(13.068,7.578)--(13.071,7.580)%
  --(13.074,7.582)--(13.077,7.584)--(13.080,7.586)--(13.083,7.588)--(13.086,7.590)--(13.089,7.592)%
  --(13.092,7.594)--(13.095,7.595)--(13.098,7.597)--(13.101,7.599)--(13.104,7.601)--(13.107,7.603)%
  --(13.110,7.605)--(13.113,7.607)--(13.116,7.609)--(13.119,7.611)--(13.122,7.612)--(13.125,7.614)%
  --(13.128,7.616)--(13.131,7.618)--(13.133,7.620)--(13.136,7.622)--(13.139,7.624)--(13.142,7.626)%
  --(13.145,7.628)--(13.148,7.630)--(13.151,7.631)--(13.154,7.633)--(13.157,7.635)--(13.160,7.637)%
  --(13.163,7.639)--(13.166,7.641)--(13.169,7.643)--(13.172,7.645)--(13.175,7.647)--(13.178,7.648)%
  --(13.181,7.650)--(13.184,7.652)--(13.187,7.654)--(13.190,7.656)--(13.193,7.658)--(13.196,7.660)%
  --(13.199,7.662)--(13.202,7.664)--(13.205,7.666)--(13.208,7.667)--(13.211,7.669)--(13.214,7.671)%
  --(13.217,7.673)--(13.220,7.675)--(13.223,7.677)--(13.226,7.679)--(13.229,7.681)--(13.232,7.683)%
  --(13.235,7.684)--(13.238,7.686)--(13.241,7.688)--(13.244,7.690)--(13.247,7.692)--(13.250,7.694)%
  --(13.253,7.696)--(13.256,7.698)--(13.259,7.700)--(13.262,7.701)--(13.265,7.703)--(13.268,7.705)%
  --(13.271,7.707)--(13.274,7.709)--(13.277,7.711)--(13.280,7.713)--(13.283,7.715)--(13.286,7.717)%
  --(13.289,7.719)--(13.292,7.720)--(13.295,7.722)--(13.298,7.724)--(13.301,7.726)--(13.304,7.728)%
  --(13.307,7.730)--(13.310,7.732)--(13.313,7.734)--(13.316,7.736)--(13.319,7.737)--(13.322,7.739)%
  --(13.325,7.741)--(13.328,7.743)--(13.331,7.745)--(13.334,7.747)--(13.337,7.749)--(13.340,7.751)%
  --(13.342,7.753)--(13.345,7.755)--(13.348,7.756)--(13.351,7.758)--(13.354,7.760)--(13.357,7.762)%
  --(13.360,7.764)--(13.363,7.766)--(13.366,7.768)--(13.369,7.770)--(13.372,7.772)--(13.375,7.773)%
  --(13.378,7.775)--(13.381,7.777)--(13.384,7.779)--(13.387,7.781)--(13.390,7.783)--(13.393,7.785)%
  --(13.396,7.787)--(13.399,7.789)--(13.402,7.791)--(13.405,7.792)--(13.408,7.794)--(13.411,7.796)%
  --(13.414,7.798)--(13.417,7.800)--(13.420,7.802)--(13.423,7.804)--(13.426,7.806)--(13.429,7.808)%
  --(13.432,7.809)--(13.435,7.811)--(13.438,7.813)--(13.441,7.815)--(13.444,7.817);
\gpcolor{color=gp lt color border}
\node[gp node left] at (2.972,7.681) {$\rho \approx 0.3$};
\gpcolor{rgb color={0.337,0.706,0.914}}
\draw[gp path] (1.872,7.681)--(2.788,7.681);
\draw[gp path] (1.504,2.514)--(1.507,2.514)--(1.510,2.513)--(1.513,2.513)--(1.516,2.512)%
  --(1.519,2.512)--(1.522,2.511)--(1.525,2.511)--(1.528,2.511)--(1.531,2.510)--(1.534,2.510)%
  --(1.537,2.509)--(1.540,2.509)--(1.543,2.508)--(1.546,2.508)--(1.549,2.507)--(1.552,2.507)%
  --(1.555,2.506)--(1.558,2.506)--(1.561,2.506)--(1.564,2.505)--(1.567,2.505)--(1.570,2.504)%
  --(1.573,2.504)--(1.576,2.503)--(1.579,2.503)--(1.582,2.502)--(1.585,2.502)--(1.588,2.501)%
  --(1.591,2.501)--(1.594,2.501)--(1.597,2.500)--(1.600,2.500)--(1.603,2.499)--(1.606,2.499)%
  --(1.609,2.498)--(1.611,2.498)--(1.614,2.497)--(1.617,2.497)--(1.620,2.496)--(1.623,2.496)%
  --(1.626,2.496)--(1.629,2.495)--(1.632,2.495)--(1.635,2.494)--(1.638,2.494)--(1.641,2.493)%
  --(1.644,2.493)--(1.647,2.492)--(1.650,2.492)--(1.653,2.492)--(1.656,2.491)--(1.659,2.491)%
  --(1.662,2.490)--(1.665,2.490)--(1.668,2.489)--(1.671,2.489)--(1.674,2.489)--(1.677,2.488)%
  --(1.680,2.488)--(1.683,2.487)--(1.686,2.487)--(1.689,2.486)--(1.692,2.486)--(1.695,2.486)%
  --(1.698,2.485)--(1.701,2.485)--(1.704,2.484)--(1.707,2.484)--(1.710,2.483)--(1.713,2.483)%
  --(1.716,2.483)--(1.719,2.482)--(1.722,2.482)--(1.725,2.481)--(1.728,2.481)--(1.731,2.480)%
  --(1.734,2.480)--(1.737,2.480)--(1.740,2.479)--(1.743,2.479)--(1.746,2.478)--(1.749,2.478)%
  --(1.752,2.478)--(1.755,2.477)--(1.758,2.477)--(1.761,2.476)--(1.764,2.476)--(1.767,2.475)%
  --(1.770,2.475)--(1.773,2.475)--(1.776,2.474)--(1.779,2.474)--(1.782,2.473)--(1.785,2.473)%
  --(1.788,2.473)--(1.791,2.472)--(1.794,2.472)--(1.797,2.472)--(1.800,2.471)--(1.803,2.471)%
  --(1.806,2.470)--(1.809,2.470)--(1.812,2.470)--(1.815,2.469)--(1.818,2.469)--(1.820,2.468)%
  --(1.823,2.468)--(1.826,2.468)--(1.829,2.467)--(1.832,2.467)--(1.835,2.467)--(1.838,2.466)%
  --(1.841,2.466)--(1.844,2.465)--(1.847,2.465)--(1.850,2.465)--(1.853,2.464)--(1.856,2.464)%
  --(1.859,2.464)--(1.862,2.463)--(1.865,2.463)--(1.868,2.462)--(1.871,2.462)--(1.874,2.462)%
  --(1.877,2.461)--(1.880,2.461)--(1.883,2.461)--(1.886,2.460)--(1.889,2.460)--(1.892,2.460)%
  --(1.895,2.459)--(1.898,2.459)--(1.901,2.459)--(1.904,2.458)--(1.907,2.458)--(1.910,2.458)%
  --(1.913,2.457)--(1.916,2.457)--(1.919,2.457)--(1.922,2.456)--(1.925,2.456)--(1.928,2.456)%
  --(1.931,2.455)--(1.934,2.455)--(1.937,2.455)--(1.940,2.454)--(1.943,2.454)--(1.946,2.454)%
  --(1.949,2.453)--(1.952,2.453)--(1.955,2.453)--(1.958,2.453)--(1.961,2.452)--(1.964,2.452)%
  --(1.967,2.452)--(1.970,2.451)--(1.973,2.451)--(1.976,2.451)--(1.979,2.450)--(1.982,2.450)%
  --(1.985,2.450)--(1.988,2.450)--(1.991,2.449)--(1.994,2.449)--(1.997,2.449)--(2.000,2.448)%
  --(2.003,2.448)--(2.006,2.448)--(2.009,2.448)--(2.012,2.447)--(2.015,2.447)--(2.018,2.447)%
  --(2.021,2.447)--(2.024,2.446)--(2.027,2.446)--(2.029,2.446)--(2.032,2.446)--(2.035,2.445)%
  --(2.038,2.445)--(2.041,2.445)--(2.044,2.444)--(2.047,2.444)--(2.050,2.444)--(2.053,2.444)%
  --(2.056,2.444)--(2.059,2.443)--(2.062,2.443)--(2.065,2.443)--(2.068,2.443)--(2.071,2.442)%
  --(2.074,2.442)--(2.077,2.442)--(2.080,2.442)--(2.083,2.441)--(2.086,2.441)--(2.089,2.441)%
  --(2.092,2.441)--(2.095,2.441)--(2.098,2.440)--(2.101,2.440)--(2.104,2.440)--(2.107,2.440)%
  --(2.110,2.440)--(2.113,2.439)--(2.116,2.439)--(2.119,2.439)--(2.122,2.439)--(2.125,2.439)%
  --(2.128,2.438)--(2.131,2.438)--(2.134,2.438)--(2.137,2.438)--(2.140,2.438)--(2.143,2.438)%
  --(2.146,2.437)--(2.149,2.437)--(2.152,2.437)--(2.155,2.437)--(2.158,2.437)--(2.161,2.437)%
  --(2.164,2.436)--(2.167,2.436)--(2.170,2.436)--(2.173,2.436)--(2.176,2.436)--(2.179,2.436)%
  --(2.182,2.436)--(2.185,2.435)--(2.188,2.435)--(2.191,2.435)--(2.194,2.435)--(2.197,2.435)%
  --(2.200,2.435)--(2.203,2.435)--(2.206,2.435)--(2.209,2.434)--(2.212,2.434)--(2.215,2.434)%
  --(2.218,2.434)--(2.221,2.434)--(2.224,2.434)--(2.227,2.434)--(2.230,2.434)--(2.233,2.434)%
  --(2.236,2.433)--(2.238,2.433)--(2.241,2.433)--(2.244,2.433)--(2.247,2.433)--(2.250,2.433)%
  --(2.253,2.433)--(2.256,2.433)--(2.259,2.433)--(2.262,2.433)--(2.265,2.433)--(2.268,2.433)%
  --(2.271,2.432)--(2.274,2.432)--(2.277,2.432)--(2.280,2.432)--(2.283,2.432)--(2.286,2.432)%
  --(2.289,2.432)--(2.292,2.432)--(2.295,2.432)--(2.298,2.432)--(2.301,2.432)--(2.304,2.432)%
  --(2.307,2.432)--(2.310,2.432)--(2.313,2.432)--(2.316,2.432)--(2.319,2.432)--(2.322,2.432)%
  --(2.325,2.432)--(2.328,2.432)--(2.331,2.432)--(2.334,2.432)--(2.337,2.432)--(2.340,2.432)%
  --(2.343,2.432)--(2.346,2.432)--(2.349,2.432)--(2.352,2.432)--(2.355,2.432)--(2.358,2.432)%
  --(2.361,2.432)--(2.364,2.432)--(2.367,2.432)--(2.370,2.432)--(2.373,2.432)--(2.376,2.432)%
  --(2.379,2.432)--(2.382,2.432)--(2.385,2.432)--(2.388,2.432)--(2.391,2.432)--(2.394,2.432)%
  --(2.397,2.432)--(2.400,2.432)--(2.403,2.432)--(2.406,2.432)--(2.409,2.432)--(2.412,2.432)%
  --(2.415,2.433)--(2.418,2.433)--(2.421,2.433)--(2.424,2.433)--(2.427,2.433)--(2.430,2.433)%
  --(2.433,2.433)--(2.436,2.433)--(2.439,2.433)--(2.442,2.433)--(2.445,2.433)--(2.447,2.433)%
  --(2.450,2.434)--(2.453,2.434)--(2.456,2.434)--(2.459,2.434)--(2.462,2.434)--(2.465,2.434)%
  --(2.468,2.434)--(2.471,2.434)--(2.474,2.434)--(2.477,2.435)--(2.480,2.435)--(2.483,2.435)%
  --(2.486,2.435)--(2.489,2.435)--(2.492,2.435)--(2.495,2.435)--(2.498,2.436)--(2.501,2.436)%
  --(2.504,2.436)--(2.507,2.436)--(2.510,2.436)--(2.513,2.436)--(2.516,2.436)--(2.519,2.437)%
  --(2.522,2.437)--(2.525,2.437)--(2.528,2.437)--(2.531,2.437)--(2.534,2.438)--(2.537,2.438)%
  --(2.540,2.438)--(2.543,2.438)--(2.546,2.438)--(2.549,2.438)--(2.552,2.439)--(2.555,2.439)%
  --(2.558,2.439)--(2.561,2.439)--(2.564,2.439)--(2.567,2.440)--(2.570,2.440)--(2.573,2.440)%
  --(2.576,2.440)--(2.579,2.441)--(2.582,2.441)--(2.585,2.441)--(2.588,2.441)--(2.591,2.441)%
  --(2.594,2.442)--(2.597,2.442)--(2.600,2.442)--(2.603,2.442)--(2.606,2.443)--(2.609,2.443)%
  --(2.612,2.443)--(2.615,2.443)--(2.618,2.444)--(2.621,2.444)--(2.624,2.444)--(2.627,2.444)%
  --(2.630,2.445)--(2.633,2.445)--(2.636,2.445)--(2.639,2.446)--(2.642,2.446)--(2.645,2.446)%
  --(2.648,2.446)--(2.651,2.447)--(2.654,2.447)--(2.656,2.447)--(2.659,2.448)--(2.662,2.448)%
  --(2.665,2.448)--(2.668,2.448)--(2.671,2.449)--(2.674,2.449)--(2.677,2.449)--(2.680,2.450)%
  --(2.683,2.450)--(2.686,2.450)--(2.689,2.451)--(2.692,2.451)--(2.695,2.451)--(2.698,2.452)%
  --(2.701,2.452)--(2.704,2.452)--(2.707,2.453)--(2.710,2.453)--(2.713,2.453)--(2.716,2.454)%
  --(2.719,2.454)--(2.722,2.454)--(2.725,2.455)--(2.728,2.455)--(2.731,2.456)--(2.734,2.456)%
  --(2.737,2.456)--(2.740,2.457)--(2.743,2.457)--(2.746,2.457)--(2.749,2.458)--(2.752,2.458)%
  --(2.755,2.459)--(2.758,2.459)--(2.761,2.459)--(2.764,2.460)--(2.767,2.460)--(2.770,2.461)%
  --(2.773,2.461)--(2.776,2.461)--(2.779,2.462)--(2.782,2.462)--(2.785,2.463)--(2.788,2.463)%
  --(2.791,2.463)--(2.794,2.464)--(2.797,2.464)--(2.800,2.465)--(2.803,2.465)--(2.806,2.466)%
  --(2.809,2.466)--(2.812,2.466)--(2.815,2.467)--(2.818,2.467)--(2.821,2.468)--(2.824,2.468)%
  --(2.827,2.469)--(2.830,2.469)--(2.833,2.470)--(2.836,2.470)--(2.839,2.471)--(2.842,2.471)%
  --(2.845,2.471)--(2.848,2.472)--(2.851,2.472)--(2.854,2.473)--(2.857,2.473)--(2.860,2.474)%
  --(2.863,2.474)--(2.866,2.475)--(2.868,2.475)--(2.871,2.476)--(2.874,2.476)--(2.877,2.477)%
  --(2.880,2.477)--(2.883,2.478)--(2.886,2.478)--(2.889,2.479)--(2.892,2.479)--(2.895,2.480)%
  --(2.898,2.480)--(2.901,2.481)--(2.904,2.481)--(2.907,2.482)--(2.910,2.482)--(2.913,2.483)%
  --(2.916,2.484)--(2.919,2.484)--(2.922,2.485)--(2.925,2.485)--(2.928,2.486)--(2.931,2.486)%
  --(2.934,2.487)--(2.937,2.487)--(2.940,2.488)--(2.943,2.488)--(2.946,2.489)--(2.949,2.490)%
  --(2.952,2.490)--(2.955,2.491)--(2.958,2.491)--(2.961,2.492)--(2.964,2.492)--(2.967,2.493)%
  --(2.970,2.494)--(2.973,2.494)--(2.976,2.495)--(2.979,2.495)--(2.982,2.496)--(2.985,2.497)%
  --(2.988,2.497)--(2.991,2.498)--(2.994,2.498)--(2.997,2.499)--(3.000,2.500)--(3.003,2.500)%
  --(3.006,2.501)--(3.009,2.501)--(3.012,2.502)--(3.015,2.503)--(3.018,2.503)--(3.021,2.504)%
  --(3.024,2.504)--(3.027,2.505)--(3.030,2.506)--(3.033,2.506)--(3.036,2.507)--(3.039,2.508)%
  --(3.042,2.508)--(3.045,2.509)--(3.048,2.510)--(3.051,2.510)--(3.054,2.511)--(3.057,2.512)%
  --(3.060,2.512)--(3.063,2.513)--(3.066,2.514)--(3.069,2.514)--(3.072,2.515)--(3.075,2.516)%
  --(3.077,2.516)--(3.080,2.517)--(3.083,2.518)--(3.086,2.518)--(3.089,2.519)--(3.092,2.520)%
  --(3.095,2.520)--(3.098,2.521)--(3.101,2.522)--(3.104,2.522)--(3.107,2.523)--(3.110,2.524)%
  --(3.113,2.524)--(3.116,2.525)--(3.119,2.526)--(3.122,2.527)--(3.125,2.527)--(3.128,2.528)%
  --(3.131,2.529)--(3.134,2.529)--(3.137,2.530)--(3.140,2.531)--(3.143,2.532)--(3.146,2.532)%
  --(3.149,2.533)--(3.152,2.534)--(3.155,2.535)--(3.158,2.535)--(3.161,2.536)--(3.164,2.537)%
  --(3.167,2.538)--(3.170,2.538)--(3.173,2.539)--(3.176,2.540)--(3.179,2.541)--(3.182,2.541)%
  --(3.185,2.542)--(3.188,2.543)--(3.191,2.544)--(3.194,2.544)--(3.197,2.545)--(3.200,2.546)%
  --(3.203,2.547)--(3.206,2.547)--(3.209,2.548)--(3.212,2.549)--(3.215,2.550)--(3.218,2.551)%
  --(3.221,2.551)--(3.224,2.552)--(3.227,2.553)--(3.230,2.554)--(3.233,2.554)--(3.236,2.555)%
  --(3.239,2.556)--(3.242,2.557)--(3.245,2.558)--(3.248,2.558)--(3.251,2.559)--(3.254,2.560)%
  --(3.257,2.561)--(3.260,2.562)--(3.263,2.563)--(3.266,2.563)--(3.269,2.564)--(3.272,2.565)%
  --(3.275,2.566)--(3.278,2.567)--(3.281,2.568)--(3.284,2.568)--(3.286,2.569)--(3.289,2.570)%
  --(3.292,2.571)--(3.295,2.572)--(3.298,2.573)--(3.301,2.573)--(3.304,2.574)--(3.307,2.575)%
  --(3.310,2.576)--(3.313,2.577)--(3.316,2.578)--(3.319,2.579)--(3.322,2.579)--(3.325,2.580)%
  --(3.328,2.581)--(3.331,2.582)--(3.334,2.583)--(3.337,2.584)--(3.340,2.585)--(3.343,2.585)%
  --(3.346,2.586)--(3.349,2.587)--(3.352,2.588)--(3.355,2.589)--(3.358,2.590)--(3.361,2.591)%
  --(3.364,2.592)--(3.367,2.593)--(3.370,2.593)--(3.373,2.594)--(3.376,2.595)--(3.379,2.596)%
  --(3.382,2.597)--(3.385,2.598)--(3.388,2.599)--(3.391,2.600)--(3.394,2.601)--(3.397,2.602)%
  --(3.400,2.603)--(3.403,2.603)--(3.406,2.604)--(3.409,2.605)--(3.412,2.606)--(3.415,2.607)%
  --(3.418,2.608)--(3.421,2.609)--(3.424,2.610)--(3.427,2.611)--(3.430,2.612)--(3.433,2.613)%
  --(3.436,2.614)--(3.439,2.615)--(3.442,2.616)--(3.445,2.617)--(3.448,2.617)--(3.451,2.618)%
  --(3.454,2.619)--(3.457,2.620)--(3.460,2.621)--(3.463,2.622)--(3.466,2.623)--(3.469,2.624)%
  --(3.472,2.625)--(3.475,2.626)--(3.478,2.627)--(3.481,2.628)--(3.484,2.629)--(3.487,2.630)%
  --(3.490,2.631)--(3.493,2.632)--(3.495,2.633)--(3.498,2.634)--(3.501,2.635)--(3.504,2.636)%
  --(3.507,2.637)--(3.510,2.638)--(3.513,2.639)--(3.516,2.640)--(3.519,2.641)--(3.522,2.642)%
  --(3.525,2.643)--(3.528,2.644)--(3.531,2.645)--(3.534,2.646)--(3.537,2.647)--(3.540,2.648)%
  --(3.543,2.649)--(3.546,2.650)--(3.549,2.651)--(3.552,2.652)--(3.555,2.653)--(3.558,2.654)%
  --(3.561,2.655)--(3.564,2.656)--(3.567,2.657)--(3.570,2.658)--(3.573,2.659)--(3.576,2.660)%
  --(3.579,2.661)--(3.582,2.662)--(3.585,2.663)--(3.588,2.664)--(3.591,2.665)--(3.594,2.666)%
  --(3.597,2.667)--(3.600,2.668)--(3.603,2.670)--(3.606,2.671)--(3.609,2.672)--(3.612,2.673)%
  --(3.615,2.674)--(3.618,2.675)--(3.621,2.676)--(3.624,2.677)--(3.627,2.678)--(3.630,2.679)%
  --(3.633,2.680)--(3.636,2.681)--(3.639,2.682)--(3.642,2.683)--(3.645,2.684)--(3.648,2.685)%
  --(3.651,2.687)--(3.654,2.688)--(3.657,2.689)--(3.660,2.690)--(3.663,2.691)--(3.666,2.692)%
  --(3.669,2.693)--(3.672,2.694)--(3.675,2.695)--(3.678,2.696)--(3.681,2.697)--(3.684,2.698)%
  --(3.687,2.700)--(3.690,2.701)--(3.693,2.702)--(3.696,2.703)--(3.699,2.704)--(3.702,2.705)%
  --(3.704,2.706)--(3.707,2.707)--(3.710,2.708)--(3.713,2.710)--(3.716,2.711)--(3.719,2.712)%
  --(3.722,2.713)--(3.725,2.714)--(3.728,2.715)--(3.731,2.716)--(3.734,2.717)--(3.737,2.719)%
  --(3.740,2.720)--(3.743,2.721)--(3.746,2.722)--(3.749,2.723)--(3.752,2.724)--(3.755,2.725)%
  --(3.758,2.726)--(3.761,2.728)--(3.764,2.729)--(3.767,2.730)--(3.770,2.731)--(3.773,2.732)%
  --(3.776,2.733)--(3.779,2.734)--(3.782,2.736)--(3.785,2.737)--(3.788,2.738)--(3.791,2.739)%
  --(3.794,2.740)--(3.797,2.741)--(3.800,2.743)--(3.803,2.744)--(3.806,2.745)--(3.809,2.746)%
  --(3.812,2.747)--(3.815,2.748)--(3.818,2.750)--(3.821,2.751)--(3.824,2.752)--(3.827,2.753)%
  --(3.830,2.754)--(3.833,2.755)--(3.836,2.757)--(3.839,2.758)--(3.842,2.759)--(3.845,2.760)%
  --(3.848,2.761)--(3.851,2.762)--(3.854,2.764)--(3.857,2.765)--(3.860,2.766)--(3.863,2.767)%
  --(3.866,2.768)--(3.869,2.770)--(3.872,2.771)--(3.875,2.772)--(3.878,2.773)--(3.881,2.774)%
  --(3.884,2.776)--(3.887,2.777)--(3.890,2.778)--(3.893,2.779)--(3.896,2.780)--(3.899,2.782)%
  --(3.902,2.783)--(3.905,2.784)--(3.908,2.785)--(3.911,2.786)--(3.914,2.788)--(3.916,2.789)%
  --(3.919,2.790)--(3.922,2.791)--(3.925,2.793)--(3.928,2.794)--(3.931,2.795)--(3.934,2.796)%
  --(3.937,2.797)--(3.940,2.799)--(3.943,2.800)--(3.946,2.801)--(3.949,2.802)--(3.952,2.804)%
  --(3.955,2.805)--(3.958,2.806)--(3.961,2.807)--(3.964,2.809)--(3.967,2.810)--(3.970,2.811)%
  --(3.973,2.812)--(3.976,2.814)--(3.979,2.815)--(3.982,2.816)--(3.985,2.817)--(3.988,2.819)%
  --(3.991,2.820)--(3.994,2.821)--(3.997,2.822)--(4.000,2.824)--(4.003,2.825)--(4.006,2.826)%
  --(4.009,2.827)--(4.012,2.829)--(4.015,2.830)--(4.018,2.831)--(4.021,2.832)--(4.024,2.834)%
  --(4.027,2.835)--(4.030,2.836)--(4.033,2.837)--(4.036,2.839)--(4.039,2.840)--(4.042,2.841)%
  --(4.045,2.843)--(4.048,2.844)--(4.051,2.845)--(4.054,2.846)--(4.057,2.848)--(4.060,2.849)%
  --(4.063,2.850)--(4.066,2.852)--(4.069,2.853)--(4.072,2.854)--(4.075,2.855)--(4.078,2.857)%
  --(4.081,2.858)--(4.084,2.859)--(4.087,2.861)--(4.090,2.862)--(4.093,2.863)--(4.096,2.864)%
  --(4.099,2.866)--(4.102,2.867)--(4.105,2.868)--(4.108,2.870)--(4.111,2.871)--(4.114,2.872)%
  --(4.117,2.874)--(4.120,2.875)--(4.123,2.876)--(4.125,2.878)--(4.128,2.879)--(4.131,2.880)%
  --(4.134,2.881)--(4.137,2.883)--(4.140,2.884)--(4.143,2.885)--(4.146,2.887)--(4.149,2.888)%
  --(4.152,2.889)--(4.155,2.891)--(4.158,2.892)--(4.161,2.893)--(4.164,2.895)--(4.167,2.896)%
  --(4.170,2.897)--(4.173,2.899)--(4.176,2.900)--(4.179,2.901)--(4.182,2.903)--(4.185,2.904)%
  --(4.188,2.905)--(4.191,2.907)--(4.194,2.908)--(4.197,2.909)--(4.200,2.911)--(4.203,2.912)%
  --(4.206,2.913)--(4.209,2.915)--(4.212,2.916)--(4.215,2.917)--(4.218,2.919)--(4.221,2.920)%
  --(4.224,2.921)--(4.227,2.923)--(4.230,2.924)--(4.233,2.926)--(4.236,2.927)--(4.239,2.928)%
  --(4.242,2.930)--(4.245,2.931)--(4.248,2.932)--(4.251,2.934)--(4.254,2.935)--(4.257,2.936)%
  --(4.260,2.938)--(4.263,2.939)--(4.266,2.940)--(4.269,2.942)--(4.272,2.943)--(4.275,2.945)%
  --(4.278,2.946)--(4.281,2.947)--(4.284,2.949)--(4.287,2.950)--(4.290,2.951)--(4.293,2.953)%
  --(4.296,2.954)--(4.299,2.956)--(4.302,2.957)--(4.305,2.958)--(4.308,2.960)--(4.311,2.961)%
  --(4.314,2.962)--(4.317,2.964)--(4.320,2.965)--(4.323,2.967)--(4.326,2.968)--(4.329,2.969)%
  --(4.332,2.971)--(4.334,2.972)--(4.337,2.974)--(4.340,2.975)--(4.343,2.976)--(4.346,2.978)%
  --(4.349,2.979)--(4.352,2.981)--(4.355,2.982)--(4.358,2.983)--(4.361,2.985)--(4.364,2.986)%
  --(4.367,2.988)--(4.370,2.989)--(4.373,2.990)--(4.376,2.992)--(4.379,2.993)--(4.382,2.995)%
  --(4.385,2.996)--(4.388,2.997)--(4.391,2.999)--(4.394,3.000)--(4.397,3.002)--(4.400,3.003)%
  --(4.403,3.004)--(4.406,3.006)--(4.409,3.007)--(4.412,3.009)--(4.415,3.010)--(4.418,3.012)%
  --(4.421,3.013)--(4.424,3.014)--(4.427,3.016)--(4.430,3.017)--(4.433,3.019)--(4.436,3.020)%
  --(4.439,3.021)--(4.442,3.023)--(4.445,3.024)--(4.448,3.026)--(4.451,3.027)--(4.454,3.029)%
  --(4.457,3.030)--(4.460,3.031)--(4.463,3.033)--(4.466,3.034)--(4.469,3.036)--(4.472,3.037)%
  --(4.475,3.039)--(4.478,3.040)--(4.481,3.042)--(4.484,3.043)--(4.487,3.044)--(4.490,3.046)%
  --(4.493,3.047)--(4.496,3.049)--(4.499,3.050)--(4.502,3.052)--(4.505,3.053)--(4.508,3.055)%
  --(4.511,3.056)--(4.514,3.057)--(4.517,3.059)--(4.520,3.060)--(4.523,3.062)--(4.526,3.063)%
  --(4.529,3.065)--(4.532,3.066)--(4.535,3.068)--(4.538,3.069)--(4.541,3.071)--(4.543,3.072)%
  --(4.546,3.073)--(4.549,3.075)--(4.552,3.076)--(4.555,3.078)--(4.558,3.079)--(4.561,3.081)%
  --(4.564,3.082)--(4.567,3.084)--(4.570,3.085)--(4.573,3.087)--(4.576,3.088)--(4.579,3.090)%
  --(4.582,3.091)--(4.585,3.092)--(4.588,3.094)--(4.591,3.095)--(4.594,3.097)--(4.597,3.098)%
  --(4.600,3.100)--(4.603,3.101)--(4.606,3.103)--(4.609,3.104)--(4.612,3.106)--(4.615,3.107)%
  --(4.618,3.109)--(4.621,3.110)--(4.624,3.112)--(4.627,3.113)--(4.630,3.115)--(4.633,3.116)%
  --(4.636,3.118)--(4.639,3.119)--(4.642,3.121)--(4.645,3.122)--(4.648,3.124)--(4.651,3.125)%
  --(4.654,3.127)--(4.657,3.128)--(4.660,3.129)--(4.663,3.131)--(4.666,3.132)--(4.669,3.134)%
  --(4.672,3.135)--(4.675,3.137)--(4.678,3.138)--(4.681,3.140)--(4.684,3.141)--(4.687,3.143)%
  --(4.690,3.144)--(4.693,3.146)--(4.696,3.147)--(4.699,3.149)--(4.702,3.150)--(4.705,3.152)%
  --(4.708,3.153)--(4.711,3.155)--(4.714,3.156)--(4.717,3.158)--(4.720,3.159)--(4.723,3.161)%
  --(4.726,3.163)--(4.729,3.164)--(4.732,3.166)--(4.735,3.167)--(4.738,3.169)--(4.741,3.170)%
  --(4.744,3.172)--(4.747,3.173)--(4.750,3.175)--(4.752,3.176)--(4.755,3.178)--(4.758,3.179)%
  --(4.761,3.181)--(4.764,3.182)--(4.767,3.184)--(4.770,3.185)--(4.773,3.187)--(4.776,3.188)%
  --(4.779,3.190)--(4.782,3.191)--(4.785,3.193)--(4.788,3.194)--(4.791,3.196)--(4.794,3.197)%
  --(4.797,3.199)--(4.800,3.200)--(4.803,3.202)--(4.806,3.204)--(4.809,3.205)--(4.812,3.207)%
  --(4.815,3.208)--(4.818,3.210)--(4.821,3.211)--(4.824,3.213)--(4.827,3.214)--(4.830,3.216)%
  --(4.833,3.217)--(4.836,3.219)--(4.839,3.220)--(4.842,3.222)--(4.845,3.223)--(4.848,3.225)%
  --(4.851,3.227)--(4.854,3.228)--(4.857,3.230)--(4.860,3.231)--(4.863,3.233)--(4.866,3.234)%
  --(4.869,3.236)--(4.872,3.237)--(4.875,3.239)--(4.878,3.240)--(4.881,3.242)--(4.884,3.244)%
  --(4.887,3.245)--(4.890,3.247)--(4.893,3.248)--(4.896,3.250)--(4.899,3.251)--(4.902,3.253)%
  --(4.905,3.254)--(4.908,3.256)--(4.911,3.257)--(4.914,3.259)--(4.917,3.261)--(4.920,3.262)%
  --(4.923,3.264)--(4.926,3.265)--(4.929,3.267)--(4.932,3.268)--(4.935,3.270)--(4.938,3.271)%
  --(4.941,3.273)--(4.944,3.275)--(4.947,3.276)--(4.950,3.278)--(4.953,3.279)--(4.956,3.281)%
  --(4.959,3.282)--(4.961,3.284)--(4.964,3.286)--(4.967,3.287)--(4.970,3.289)--(4.973,3.290)%
  --(4.976,3.292)--(4.979,3.293)--(4.982,3.295)--(4.985,3.297)--(4.988,3.298)--(4.991,3.300)%
  --(4.994,3.301)--(4.997,3.303)--(5.000,3.304)--(5.003,3.306)--(5.006,3.308)--(5.009,3.309)%
  --(5.012,3.311)--(5.015,3.312)--(5.018,3.314)--(5.021,3.315)--(5.024,3.317)--(5.027,3.319)%
  --(5.030,3.320)--(5.033,3.322)--(5.036,3.323)--(5.039,3.325)--(5.042,3.326)--(5.045,3.328)%
  --(5.048,3.330)--(5.051,3.331)--(5.054,3.333)--(5.057,3.334)--(5.060,3.336)--(5.063,3.338)%
  --(5.066,3.339)--(5.069,3.341)--(5.072,3.342)--(5.075,3.344)--(5.078,3.345)--(5.081,3.347)%
  --(5.084,3.349)--(5.087,3.350)--(5.090,3.352)--(5.093,3.353)--(5.096,3.355)--(5.099,3.357)%
  --(5.102,3.358)--(5.105,3.360)--(5.108,3.361)--(5.111,3.363)--(5.114,3.365)--(5.117,3.366)%
  --(5.120,3.368)--(5.123,3.369)--(5.126,3.371)--(5.129,3.373)--(5.132,3.374)--(5.135,3.376)%
  --(5.138,3.377)--(5.141,3.379)--(5.144,3.381)--(5.147,3.382)--(5.150,3.384)--(5.153,3.385)%
  --(5.156,3.387)--(5.159,3.389)--(5.162,3.390)--(5.165,3.392)--(5.168,3.393)--(5.171,3.395)%
  --(5.173,3.397)--(5.176,3.398)--(5.179,3.400)--(5.182,3.401)--(5.185,3.403)--(5.188,3.405)%
  --(5.191,3.406)--(5.194,3.408)--(5.197,3.410)--(5.200,3.411)--(5.203,3.413)--(5.206,3.414)%
  --(5.209,3.416)--(5.212,3.418)--(5.215,3.419)--(5.218,3.421)--(5.221,3.422)--(5.224,3.424)%
  --(5.227,3.426)--(5.230,3.427)--(5.233,3.429)--(5.236,3.431)--(5.239,3.432)--(5.242,3.434)%
  --(5.245,3.435)--(5.248,3.437)--(5.251,3.439)--(5.254,3.440)--(5.257,3.442)--(5.260,3.443)%
  --(5.263,3.445)--(5.266,3.447)--(5.269,3.448)--(5.272,3.450)--(5.275,3.452)--(5.278,3.453)%
  --(5.281,3.455)--(5.284,3.456)--(5.287,3.458)--(5.290,3.460)--(5.293,3.461)--(5.296,3.463)%
  --(5.299,3.465)--(5.302,3.466)--(5.305,3.468)--(5.308,3.470)--(5.311,3.471)--(5.314,3.473)%
  --(5.317,3.474)--(5.320,3.476)--(5.323,3.478)--(5.326,3.479)--(5.329,3.481)--(5.332,3.483)%
  --(5.335,3.484)--(5.338,3.486)--(5.341,3.487)--(5.344,3.489)--(5.347,3.491)--(5.350,3.492)%
  --(5.353,3.494)--(5.356,3.496)--(5.359,3.497)--(5.362,3.499)--(5.365,3.501)--(5.368,3.502)%
  --(5.371,3.504)--(5.374,3.506)--(5.377,3.507)--(5.380,3.509)--(5.382,3.510)--(5.385,3.512)%
  --(5.388,3.514)--(5.391,3.515)--(5.394,3.517)--(5.397,3.519)--(5.400,3.520)--(5.403,3.522)%
  --(5.406,3.524)--(5.409,3.525)--(5.412,3.527)--(5.415,3.529)--(5.418,3.530)--(5.421,3.532)%
  --(5.424,3.533)--(5.427,3.535)--(5.430,3.537)--(5.433,3.538)--(5.436,3.540)--(5.439,3.542)%
  --(5.442,3.543)--(5.445,3.545)--(5.448,3.547)--(5.451,3.548)--(5.454,3.550)--(5.457,3.552)%
  --(5.460,3.553)--(5.463,3.555)--(5.466,3.557)--(5.469,3.558)--(5.472,3.560)--(5.475,3.562)%
  --(5.478,3.563)--(5.481,3.565)--(5.484,3.567)--(5.487,3.568)--(5.490,3.570)--(5.493,3.572)%
  --(5.496,3.573)--(5.499,3.575)--(5.502,3.577)--(5.505,3.578)--(5.508,3.580)--(5.511,3.582)%
  --(5.514,3.583)--(5.517,3.585)--(5.520,3.587)--(5.523,3.588)--(5.526,3.590)--(5.529,3.592)%
  --(5.532,3.593)--(5.535,3.595)--(5.538,3.597)--(5.541,3.598)--(5.544,3.600)--(5.547,3.602)%
  --(5.550,3.603)--(5.553,3.605)--(5.556,3.607)--(5.559,3.608)--(5.562,3.610)--(5.565,3.612)%
  --(5.568,3.613)--(5.571,3.615)--(5.574,3.617)--(5.577,3.618)--(5.580,3.620)--(5.583,3.622)%
  --(5.586,3.623)--(5.589,3.625)--(5.591,3.627)--(5.594,3.628)--(5.597,3.630)--(5.600,3.632)%
  --(5.603,3.633)--(5.606,3.635)--(5.609,3.637)--(5.612,3.638)--(5.615,3.640)--(5.618,3.642)%
  --(5.621,3.643)--(5.624,3.645)--(5.627,3.647)--(5.630,3.648)--(5.633,3.650)--(5.636,3.652)%
  --(5.639,3.653)--(5.642,3.655)--(5.645,3.657)--(5.648,3.658)--(5.651,3.660)--(5.654,3.662)%
  --(5.657,3.663)--(5.660,3.665)--(5.663,3.667)--(5.666,3.669)--(5.669,3.670)--(5.672,3.672)%
  --(5.675,3.674)--(5.678,3.675)--(5.681,3.677)--(5.684,3.679)--(5.687,3.680)--(5.690,3.682)%
  --(5.693,3.684)--(5.696,3.685)--(5.699,3.687)--(5.702,3.689)--(5.705,3.690)--(5.708,3.692)%
  --(5.711,3.694)--(5.714,3.696)--(5.717,3.697)--(5.720,3.699)--(5.723,3.701)--(5.726,3.702)%
  --(5.729,3.704)--(5.732,3.706)--(5.735,3.707)--(5.738,3.709)--(5.741,3.711)--(5.744,3.712)%
  --(5.747,3.714)--(5.750,3.716)--(5.753,3.718)--(5.756,3.719)--(5.759,3.721)--(5.762,3.723)%
  --(5.765,3.724)--(5.768,3.726)--(5.771,3.728)--(5.774,3.729)--(5.777,3.731)--(5.780,3.733)%
  --(5.783,3.735)--(5.786,3.736)--(5.789,3.738)--(5.792,3.740)--(5.795,3.741)--(5.798,3.743)%
  --(5.800,3.745)--(5.803,3.746)--(5.806,3.748)--(5.809,3.750)--(5.812,3.752)--(5.815,3.753)%
  --(5.818,3.755)--(5.821,3.757)--(5.824,3.758)--(5.827,3.760)--(5.830,3.762)--(5.833,3.763)%
  --(5.836,3.765)--(5.839,3.767)--(5.842,3.769)--(5.845,3.770)--(5.848,3.772)--(5.851,3.774)%
  --(5.854,3.775)--(5.857,3.777)--(5.860,3.779)--(5.863,3.780)--(5.866,3.782)--(5.869,3.784)%
  --(5.872,3.786)--(5.875,3.787)--(5.878,3.789)--(5.881,3.791)--(5.884,3.792)--(5.887,3.794)%
  --(5.890,3.796)--(5.893,3.798)--(5.896,3.799)--(5.899,3.801)--(5.902,3.803)--(5.905,3.804)%
  --(5.908,3.806)--(5.911,3.808)--(5.914,3.810)--(5.917,3.811)--(5.920,3.813)--(5.923,3.815)%
  --(5.926,3.816)--(5.929,3.818)--(5.932,3.820)--(5.935,3.822)--(5.938,3.823)--(5.941,3.825)%
  --(5.944,3.827)--(5.947,3.828)--(5.950,3.830)--(5.953,3.832)--(5.956,3.834)--(5.959,3.835)%
  --(5.962,3.837)--(5.965,3.839)--(5.968,3.840)--(5.971,3.842)--(5.974,3.844)--(5.977,3.846)%
  --(5.980,3.847)--(5.983,3.849)--(5.986,3.851)--(5.989,3.852)--(5.992,3.854)--(5.995,3.856)%
  --(5.998,3.858)--(6.001,3.859)--(6.004,3.861)--(6.007,3.863)--(6.009,3.865)--(6.012,3.866)%
  --(6.015,3.868)--(6.018,3.870)--(6.021,3.871)--(6.024,3.873)--(6.027,3.875)--(6.030,3.877)%
  --(6.033,3.878)--(6.036,3.880)--(6.039,3.882)--(6.042,3.884)--(6.045,3.885)--(6.048,3.887)%
  --(6.051,3.889)--(6.054,3.890)--(6.057,3.892)--(6.060,3.894)--(6.063,3.896)--(6.066,3.897)%
  --(6.069,3.899)--(6.072,3.901)--(6.075,3.903)--(6.078,3.904)--(6.081,3.906)--(6.084,3.908)%
  --(6.087,3.909)--(6.090,3.911)--(6.093,3.913)--(6.096,3.915)--(6.099,3.916)--(6.102,3.918)%
  --(6.105,3.920)--(6.108,3.922)--(6.111,3.923)--(6.114,3.925)--(6.117,3.927)--(6.120,3.929)%
  --(6.123,3.930)--(6.126,3.932)--(6.129,3.934)--(6.132,3.935)--(6.135,3.937)--(6.138,3.939)%
  --(6.141,3.941)--(6.144,3.942)--(6.147,3.944)--(6.150,3.946)--(6.153,3.948)--(6.156,3.949)%
  --(6.159,3.951)--(6.162,3.953)--(6.165,3.955)--(6.168,3.956)--(6.171,3.958)--(6.174,3.960)%
  --(6.177,3.962)--(6.180,3.963)--(6.183,3.965)--(6.186,3.967)--(6.189,3.968)--(6.192,3.970)%
  --(6.195,3.972)--(6.198,3.974)--(6.201,3.975)--(6.204,3.977)--(6.207,3.979)--(6.210,3.981)%
  --(6.213,3.982)--(6.216,3.984)--(6.218,3.986)--(6.221,3.988)--(6.224,3.989)--(6.227,3.991)%
  --(6.230,3.993)--(6.233,3.995)--(6.236,3.996)--(6.239,3.998)--(6.242,4.000)--(6.245,4.002)%
  --(6.248,4.003)--(6.251,4.005)--(6.254,4.007)--(6.257,4.009)--(6.260,4.010)--(6.263,4.012)%
  --(6.266,4.014)--(6.269,4.016)--(6.272,4.017)--(6.275,4.019)--(6.278,4.021)--(6.281,4.023)%
  --(6.284,4.024)--(6.287,4.026)--(6.290,4.028)--(6.293,4.030)--(6.296,4.031)--(6.299,4.033)%
  --(6.302,4.035)--(6.305,4.037)--(6.308,4.038)--(6.311,4.040)--(6.314,4.042)--(6.317,4.044)%
  --(6.320,4.045)--(6.323,4.047)--(6.326,4.049)--(6.329,4.051)--(6.332,4.052)--(6.335,4.054)%
  --(6.338,4.056)--(6.341,4.058)--(6.344,4.059)--(6.347,4.061)--(6.350,4.063)--(6.353,4.065)%
  --(6.356,4.066)--(6.359,4.068)--(6.362,4.070)--(6.365,4.072)--(6.368,4.073)--(6.371,4.075)%
  --(6.374,4.077)--(6.377,4.079)--(6.380,4.080)--(6.383,4.082)--(6.386,4.084)--(6.389,4.086)%
  --(6.392,4.087)--(6.395,4.089)--(6.398,4.091)--(6.401,4.093)--(6.404,4.094)--(6.407,4.096)%
  --(6.410,4.098)--(6.413,4.100)--(6.416,4.102)--(6.419,4.103)--(6.422,4.105)--(6.425,4.107)%
  --(6.428,4.109)--(6.430,4.110)--(6.433,4.112)--(6.436,4.114)--(6.439,4.116)--(6.442,4.117)%
  --(6.445,4.119)--(6.448,4.121)--(6.451,4.123)--(6.454,4.124)--(6.457,4.126)--(6.460,4.128)%
  --(6.463,4.130)--(6.466,4.131)--(6.469,4.133)--(6.472,4.135)--(6.475,4.137)--(6.478,4.139)%
  --(6.481,4.140)--(6.484,4.142)--(6.487,4.144)--(6.490,4.146)--(6.493,4.147)--(6.496,4.149)%
  --(6.499,4.151)--(6.502,4.153)--(6.505,4.154)--(6.508,4.156)--(6.511,4.158)--(6.514,4.160)%
  --(6.517,4.161)--(6.520,4.163)--(6.523,4.165)--(6.526,4.167)--(6.529,4.169)--(6.532,4.170)%
  --(6.535,4.172)--(6.538,4.174)--(6.541,4.176)--(6.544,4.177)--(6.547,4.179)--(6.550,4.181)%
  --(6.553,4.183)--(6.556,4.184)--(6.559,4.186)--(6.562,4.188)--(6.565,4.190)--(6.568,4.192)%
  --(6.571,4.193)--(6.574,4.195)--(6.577,4.197)--(6.580,4.199)--(6.583,4.200)--(6.586,4.202)%
  --(6.589,4.204)--(6.592,4.206)--(6.595,4.207)--(6.598,4.209)--(6.601,4.211)--(6.604,4.213)%
  --(6.607,4.215)--(6.610,4.216)--(6.613,4.218)--(6.616,4.220)--(6.619,4.222)--(6.622,4.223)%
  --(6.625,4.225)--(6.628,4.227)--(6.631,4.229)--(6.634,4.231)--(6.637,4.232)--(6.639,4.234)%
  --(6.642,4.236)--(6.645,4.238)--(6.648,4.239)--(6.651,4.241)--(6.654,4.243)--(6.657,4.245)%
  --(6.660,4.247)--(6.663,4.248)--(6.666,4.250)--(6.669,4.252)--(6.672,4.254)--(6.675,4.255)%
  --(6.678,4.257)--(6.681,4.259)--(6.684,4.261)--(6.687,4.263)--(6.690,4.264)--(6.693,4.266)%
  --(6.696,4.268)--(6.699,4.270)--(6.702,4.271)--(6.705,4.273)--(6.708,4.275)--(6.711,4.277)%
  --(6.714,4.279)--(6.717,4.280)--(6.720,4.282)--(6.723,4.284)--(6.726,4.286)--(6.729,4.287)%
  --(6.732,4.289)--(6.735,4.291)--(6.738,4.293)--(6.741,4.295)--(6.744,4.296)--(6.747,4.298)%
  --(6.750,4.300)--(6.753,4.302)--(6.756,4.304)--(6.759,4.305)--(6.762,4.307)--(6.765,4.309)%
  --(6.768,4.311)--(6.771,4.312)--(6.774,4.314)--(6.777,4.316)--(6.780,4.318)--(6.783,4.320)%
  --(6.786,4.321)--(6.789,4.323)--(6.792,4.325)--(6.795,4.327)--(6.798,4.328)--(6.801,4.330)%
  --(6.804,4.332)--(6.807,4.334)--(6.810,4.336)--(6.813,4.337)--(6.816,4.339)--(6.819,4.341)%
  --(6.822,4.343)--(6.825,4.345)--(6.828,4.346)--(6.831,4.348)--(6.834,4.350)--(6.837,4.352)%
  --(6.840,4.354)--(6.843,4.355)--(6.846,4.357)--(6.848,4.359)--(6.851,4.361)--(6.854,4.362)%
  --(6.857,4.364)--(6.860,4.366)--(6.863,4.368)--(6.866,4.370)--(6.869,4.371)--(6.872,4.373)%
  --(6.875,4.375)--(6.878,4.377)--(6.881,4.379)--(6.884,4.380)--(6.887,4.382)--(6.890,4.384)%
  --(6.893,4.386)--(6.896,4.388)--(6.899,4.389)--(6.902,4.391)--(6.905,4.393)--(6.908,4.395)%
  --(6.911,4.396)--(6.914,4.398)--(6.917,4.400)--(6.920,4.402)--(6.923,4.404)--(6.926,4.405)%
  --(6.929,4.407)--(6.932,4.409)--(6.935,4.411)--(6.938,4.413)--(6.941,4.414)--(6.944,4.416)%
  --(6.947,4.418)--(6.950,4.420)--(6.953,4.422)--(6.956,4.423)--(6.959,4.425)--(6.962,4.427)%
  --(6.965,4.429)--(6.968,4.431)--(6.971,4.432)--(6.974,4.434)--(6.977,4.436)--(6.980,4.438)%
  --(6.983,4.440)--(6.986,4.441)--(6.989,4.443)--(6.992,4.445)--(6.995,4.447)--(6.998,4.449)%
  --(7.001,4.450)--(7.004,4.452)--(7.007,4.454)--(7.010,4.456)--(7.013,4.458)--(7.016,4.459)%
  --(7.019,4.461)--(7.022,4.463)--(7.025,4.465)--(7.028,4.467)--(7.031,4.468)--(7.034,4.470)%
  --(7.037,4.472)--(7.040,4.474)--(7.043,4.476)--(7.046,4.477)--(7.049,4.479)--(7.052,4.481)%
  --(7.055,4.483)--(7.057,4.485)--(7.060,4.486)--(7.063,4.488)--(7.066,4.490)--(7.069,4.492)%
  --(7.072,4.494)--(7.075,4.495)--(7.078,4.497)--(7.081,4.499)--(7.084,4.501)--(7.087,4.503)%
  --(7.090,4.504)--(7.093,4.506)--(7.096,4.508)--(7.099,4.510)--(7.102,4.512)--(7.105,4.513)%
  --(7.108,4.515)--(7.111,4.517)--(7.114,4.519)--(7.117,4.521)--(7.120,4.522)--(7.123,4.524)%
  --(7.126,4.526)--(7.129,4.528)--(7.132,4.530)--(7.135,4.531)--(7.138,4.533)--(7.141,4.535)%
  --(7.144,4.537)--(7.147,4.539)--(7.150,4.540)--(7.153,4.542)--(7.156,4.544)--(7.159,4.546)%
  --(7.162,4.548)--(7.165,4.549)--(7.168,4.551)--(7.171,4.553)--(7.174,4.555)--(7.177,4.557)%
  --(7.180,4.558)--(7.183,4.560)--(7.186,4.562)--(7.189,4.564)--(7.192,4.566)--(7.195,4.568)%
  --(7.198,4.569)--(7.201,4.571)--(7.204,4.573)--(7.207,4.575)--(7.210,4.577)--(7.213,4.578)%
  --(7.216,4.580)--(7.219,4.582)--(7.222,4.584)--(7.225,4.586)--(7.228,4.587)--(7.231,4.589)%
  --(7.234,4.591)--(7.237,4.593)--(7.240,4.595)--(7.243,4.596)--(7.246,4.598)--(7.249,4.600)%
  --(7.252,4.602)--(7.255,4.604)--(7.258,4.606)--(7.261,4.607)--(7.264,4.609)--(7.266,4.611)%
  --(7.269,4.613)--(7.272,4.615)--(7.275,4.616)--(7.278,4.618)--(7.281,4.620)--(7.284,4.622)%
  --(7.287,4.624)--(7.290,4.625)--(7.293,4.627)--(7.296,4.629)--(7.299,4.631)--(7.302,4.633)%
  --(7.305,4.634)--(7.308,4.636)--(7.311,4.638)--(7.314,4.640)--(7.317,4.642)--(7.320,4.644)%
  --(7.323,4.645)--(7.326,4.647)--(7.329,4.649)--(7.332,4.651)--(7.335,4.653)--(7.338,4.654)%
  --(7.341,4.656)--(7.344,4.658)--(7.347,4.660)--(7.350,4.662)--(7.353,4.664)--(7.356,4.665)%
  --(7.359,4.667)--(7.362,4.669)--(7.365,4.671)--(7.368,4.673)--(7.371,4.674)--(7.374,4.676)%
  --(7.377,4.678)--(7.380,4.680)--(7.383,4.682)--(7.386,4.683)--(7.389,4.685)--(7.392,4.687)%
  --(7.395,4.689)--(7.398,4.691)--(7.401,4.693)--(7.404,4.694)--(7.407,4.696)--(7.410,4.698)%
  --(7.413,4.700)--(7.416,4.702)--(7.419,4.703)--(7.422,4.705)--(7.425,4.707)--(7.428,4.709)%
  --(7.431,4.711)--(7.434,4.713)--(7.437,4.714)--(7.440,4.716)--(7.443,4.718)--(7.446,4.720)%
  --(7.449,4.722)--(7.452,4.723)--(7.455,4.725)--(7.458,4.727)--(7.461,4.729)--(7.464,4.731)%
  --(7.467,4.733)--(7.470,4.734)--(7.473,4.736)--(7.476,4.738)--(7.478,4.740)--(7.481,4.742)%
  --(7.484,4.743)--(7.487,4.745)--(7.490,4.747)--(7.493,4.749)--(7.496,4.751)--(7.499,4.753)%
  --(7.502,4.754)--(7.505,4.756)--(7.508,4.758)--(7.511,4.760)--(7.514,4.762)--(7.517,4.764)%
  --(7.520,4.765)--(7.523,4.767)--(7.526,4.769)--(7.529,4.771)--(7.532,4.773)--(7.535,4.774)%
  --(7.538,4.776)--(7.541,4.778)--(7.544,4.780)--(7.547,4.782)--(7.550,4.784)--(7.553,4.785)%
  --(7.556,4.787)--(7.559,4.789)--(7.562,4.791)--(7.565,4.793)--(7.568,4.795)--(7.571,4.796)%
  --(7.574,4.798)--(7.577,4.800)--(7.580,4.802)--(7.583,4.804)--(7.586,4.805)--(7.589,4.807)%
  --(7.592,4.809)--(7.595,4.811)--(7.598,4.813)--(7.601,4.815)--(7.604,4.816)--(7.607,4.818)%
  --(7.610,4.820)--(7.613,4.822)--(7.616,4.824)--(7.619,4.826)--(7.622,4.827)--(7.625,4.829)%
  --(7.628,4.831)--(7.631,4.833)--(7.634,4.835)--(7.637,4.836)--(7.640,4.838)--(7.643,4.840)%
  --(7.646,4.842)--(7.649,4.844)--(7.652,4.846)--(7.655,4.847)--(7.658,4.849)--(7.661,4.851)%
  --(7.664,4.853)--(7.667,4.855)--(7.670,4.857)--(7.673,4.858)--(7.676,4.860)--(7.679,4.862)%
  --(7.682,4.864)--(7.685,4.866)--(7.687,4.868)--(7.690,4.869)--(7.693,4.871)--(7.696,4.873)%
  --(7.699,4.875)--(7.702,4.877)--(7.705,4.879)--(7.708,4.880)--(7.711,4.882)--(7.714,4.884)%
  --(7.717,4.886)--(7.720,4.888)--(7.723,4.889)--(7.726,4.891)--(7.729,4.893)--(7.732,4.895)%
  --(7.735,4.897)--(7.738,4.899)--(7.741,4.900)--(7.744,4.902)--(7.747,4.904)--(7.750,4.906)%
  --(7.753,4.908)--(7.756,4.910)--(7.759,4.911)--(7.762,4.913)--(7.765,4.915)--(7.768,4.917)%
  --(7.771,4.919)--(7.774,4.921)--(7.777,4.922)--(7.780,4.924)--(7.783,4.926)--(7.786,4.928)%
  --(7.789,4.930)--(7.792,4.932)--(7.795,4.933)--(7.798,4.935)--(7.801,4.937)--(7.804,4.939)%
  --(7.807,4.941)--(7.810,4.943)--(7.813,4.944)--(7.816,4.946)--(7.819,4.948)--(7.822,4.950)%
  --(7.825,4.952)--(7.828,4.954)--(7.831,4.955)--(7.834,4.957)--(7.837,4.959)--(7.840,4.961)%
  --(7.843,4.963)--(7.846,4.965)--(7.849,4.966)--(7.852,4.968)--(7.855,4.970)--(7.858,4.972)%
  --(7.861,4.974)--(7.864,4.976)--(7.867,4.977)--(7.870,4.979)--(7.873,4.981)--(7.876,4.983)%
  --(7.879,4.985)--(7.882,4.987)--(7.885,4.988)--(7.888,4.990)--(7.891,4.992)--(7.894,4.994)%
  --(7.896,4.996)--(7.899,4.998)--(7.902,4.999)--(7.905,5.001)--(7.908,5.003)--(7.911,5.005)%
  --(7.914,5.007)--(7.917,5.009)--(7.920,5.010)--(7.923,5.012)--(7.926,5.014)--(7.929,5.016)%
  --(7.932,5.018)--(7.935,5.020)--(7.938,5.021)--(7.941,5.023)--(7.944,5.025)--(7.947,5.027)%
  --(7.950,5.029)--(7.953,5.031)--(7.956,5.032)--(7.959,5.034)--(7.962,5.036)--(7.965,5.038)%
  --(7.968,5.040)--(7.971,5.042)--(7.974,5.043)--(7.977,5.045)--(7.980,5.047)--(7.983,5.049)%
  --(7.986,5.051)--(7.989,5.053)--(7.992,5.055)--(7.995,5.056)--(7.998,5.058)--(8.001,5.060)%
  --(8.004,5.062)--(8.007,5.064)--(8.010,5.066)--(8.013,5.067)--(8.016,5.069)--(8.019,5.071)%
  --(8.022,5.073)--(8.025,5.075)--(8.028,5.077)--(8.031,5.078)--(8.034,5.080)--(8.037,5.082)%
  --(8.040,5.084)--(8.043,5.086)--(8.046,5.088)--(8.049,5.089)--(8.052,5.091)--(8.055,5.093)%
  --(8.058,5.095)--(8.061,5.097)--(8.064,5.099)--(8.067,5.101)--(8.070,5.102)--(8.073,5.104)%
  --(8.076,5.106)--(8.079,5.108)--(8.082,5.110)--(8.085,5.112)--(8.088,5.113)--(8.091,5.115)%
  --(8.094,5.117)--(8.097,5.119)--(8.100,5.121)--(8.103,5.123)--(8.105,5.124)--(8.108,5.126)%
  --(8.111,5.128)--(8.114,5.130)--(8.117,5.132)--(8.120,5.134)--(8.123,5.135)--(8.126,5.137)%
  --(8.129,5.139)--(8.132,5.141)--(8.135,5.143)--(8.138,5.145)--(8.141,5.147)--(8.144,5.148)%
  --(8.147,5.150)--(8.150,5.152)--(8.153,5.154)--(8.156,5.156)--(8.159,5.158)--(8.162,5.159)%
  --(8.165,5.161)--(8.168,5.163)--(8.171,5.165)--(8.174,5.167)--(8.177,5.169)--(8.180,5.171)%
  --(8.183,5.172)--(8.186,5.174)--(8.189,5.176)--(8.192,5.178)--(8.195,5.180)--(8.198,5.182)%
  --(8.201,5.183)--(8.204,5.185)--(8.207,5.187)--(8.210,5.189)--(8.213,5.191)--(8.216,5.193)%
  --(8.219,5.194)--(8.222,5.196)--(8.225,5.198)--(8.228,5.200)--(8.231,5.202)--(8.234,5.204)%
  --(8.237,5.206)--(8.240,5.207)--(8.243,5.209)--(8.246,5.211)--(8.249,5.213)--(8.252,5.215)%
  --(8.255,5.217)--(8.258,5.218)--(8.261,5.220)--(8.264,5.222)--(8.267,5.224)--(8.270,5.226)%
  --(8.273,5.228)--(8.276,5.230)--(8.279,5.231)--(8.282,5.233)--(8.285,5.235)--(8.288,5.237)%
  --(8.291,5.239)--(8.294,5.241)--(8.297,5.242)--(8.300,5.244)--(8.303,5.246)--(8.306,5.248)%
  --(8.309,5.250)--(8.312,5.252)--(8.314,5.254)--(8.317,5.255)--(8.320,5.257)--(8.323,5.259)%
  --(8.326,5.261)--(8.329,5.263)--(8.332,5.265)--(8.335,5.266)--(8.338,5.268)--(8.341,5.270)%
  --(8.344,5.272)--(8.347,5.274)--(8.350,5.276)--(8.353,5.278)--(8.356,5.279)--(8.359,5.281)%
  --(8.362,5.283)--(8.365,5.285)--(8.368,5.287)--(8.371,5.289)--(8.374,5.291)--(8.377,5.292)%
  --(8.380,5.294)--(8.383,5.296)--(8.386,5.298)--(8.389,5.300)--(8.392,5.302)--(8.395,5.303)%
  --(8.398,5.305)--(8.401,5.307)--(8.404,5.309)--(8.407,5.311)--(8.410,5.313)--(8.413,5.315)%
  --(8.416,5.316)--(8.419,5.318)--(8.422,5.320)--(8.425,5.322)--(8.428,5.324)--(8.431,5.326)%
  --(8.434,5.327)--(8.437,5.329)--(8.440,5.331)--(8.443,5.333)--(8.446,5.335)--(8.449,5.337)%
  --(8.452,5.339)--(8.455,5.340)--(8.458,5.342)--(8.461,5.344)--(8.464,5.346)--(8.467,5.348)%
  --(8.470,5.350)--(8.473,5.352)--(8.476,5.353)--(8.479,5.355)--(8.482,5.357)--(8.485,5.359)%
  --(8.488,5.361)--(8.491,5.363)--(8.494,5.365)--(8.497,5.366)--(8.500,5.368)--(8.503,5.370)%
  --(8.506,5.372)--(8.509,5.374)--(8.512,5.376)--(8.515,5.377)--(8.518,5.379)--(8.521,5.381)%
  --(8.523,5.383)--(8.526,5.385)--(8.529,5.387)--(8.532,5.389)--(8.535,5.390)--(8.538,5.392)%
  --(8.541,5.394)--(8.544,5.396)--(8.547,5.398)--(8.550,5.400)--(8.553,5.402)--(8.556,5.403)%
  --(8.559,5.405)--(8.562,5.407)--(8.565,5.409)--(8.568,5.411)--(8.571,5.413)--(8.574,5.415)%
  --(8.577,5.416)--(8.580,5.418)--(8.583,5.420)--(8.586,5.422)--(8.589,5.424)--(8.592,5.426)%
  --(8.595,5.428)--(8.598,5.429)--(8.601,5.431)--(8.604,5.433)--(8.607,5.435)--(8.610,5.437)%
  --(8.613,5.439)--(8.616,5.440)--(8.619,5.442)--(8.622,5.444)--(8.625,5.446)--(8.628,5.448)%
  --(8.631,5.450)--(8.634,5.452)--(8.637,5.453)--(8.640,5.455)--(8.643,5.457)--(8.646,5.459)%
  --(8.649,5.461)--(8.652,5.463)--(8.655,5.465)--(8.658,5.466)--(8.661,5.468)--(8.664,5.470)%
  --(8.667,5.472)--(8.670,5.474)--(8.673,5.476)--(8.676,5.478)--(8.679,5.479)--(8.682,5.481)%
  --(8.685,5.483)--(8.688,5.485)--(8.691,5.487)--(8.694,5.489)--(8.697,5.491)--(8.700,5.492)%
  --(8.703,5.494)--(8.706,5.496)--(8.709,5.498)--(8.712,5.500)--(8.715,5.502)--(8.718,5.504)%
  --(8.721,5.505)--(8.724,5.507)--(8.727,5.509)--(8.730,5.511)--(8.733,5.513)--(8.735,5.515)%
  --(8.738,5.517)--(8.741,5.518)--(8.744,5.520)--(8.747,5.522)--(8.750,5.524)--(8.753,5.526)%
  --(8.756,5.528)--(8.759,5.530)--(8.762,5.531)--(8.765,5.533)--(8.768,5.535)--(8.771,5.537)%
  --(8.774,5.539)--(8.777,5.541)--(8.780,5.543)--(8.783,5.544)--(8.786,5.546)--(8.789,5.548)%
  --(8.792,5.550)--(8.795,5.552)--(8.798,5.554)--(8.801,5.556)--(8.804,5.557)--(8.807,5.559)%
  --(8.810,5.561)--(8.813,5.563)--(8.816,5.565)--(8.819,5.567)--(8.822,5.569)--(8.825,5.570)%
  --(8.828,5.572)--(8.831,5.574)--(8.834,5.576)--(8.837,5.578)--(8.840,5.580)--(8.843,5.582)%
  --(8.846,5.583)--(8.849,5.585)--(8.852,5.587)--(8.855,5.589)--(8.858,5.591)--(8.861,5.593)%
  --(8.864,5.595)--(8.867,5.596)--(8.870,5.598)--(8.873,5.600)--(8.876,5.602)--(8.879,5.604)%
  --(8.882,5.606)--(8.885,5.608)--(8.888,5.609)--(8.891,5.611)--(8.894,5.613)--(8.897,5.615)%
  --(8.900,5.617)--(8.903,5.619)--(8.906,5.621)--(8.909,5.623)--(8.912,5.624)--(8.915,5.626)%
  --(8.918,5.628)--(8.921,5.630)--(8.924,5.632)--(8.927,5.634)--(8.930,5.636)--(8.933,5.637)%
  --(8.936,5.639)--(8.939,5.641)--(8.942,5.643)--(8.944,5.645)--(8.947,5.647)--(8.950,5.649)%
  --(8.953,5.650)--(8.956,5.652)--(8.959,5.654)--(8.962,5.656)--(8.965,5.658)--(8.968,5.660)%
  --(8.971,5.662)--(8.974,5.663)--(8.977,5.665)--(8.980,5.667)--(8.983,5.669)--(8.986,5.671)%
  --(8.989,5.673)--(8.992,5.675)--(8.995,5.676)--(8.998,5.678)--(9.001,5.680)--(9.004,5.682)%
  --(9.007,5.684)--(9.010,5.686)--(9.013,5.688)--(9.016,5.690)--(9.019,5.691)--(9.022,5.693)%
  --(9.025,5.695)--(9.028,5.697)--(9.031,5.699)--(9.034,5.701)--(9.037,5.703)--(9.040,5.704)%
  --(9.043,5.706)--(9.046,5.708)--(9.049,5.710)--(9.052,5.712)--(9.055,5.714)--(9.058,5.716)%
  --(9.061,5.717)--(9.064,5.719)--(9.067,5.721)--(9.070,5.723)--(9.073,5.725)--(9.076,5.727)%
  --(9.079,5.729)--(9.082,5.731)--(9.085,5.732)--(9.088,5.734)--(9.091,5.736)--(9.094,5.738)%
  --(9.097,5.740)--(9.100,5.742)--(9.103,5.744)--(9.106,5.745)--(9.109,5.747)--(9.112,5.749)%
  --(9.115,5.751)--(9.118,5.753)--(9.121,5.755)--(9.124,5.757)--(9.127,5.758)--(9.130,5.760)%
  --(9.133,5.762)--(9.136,5.764)--(9.139,5.766)--(9.142,5.768)--(9.145,5.770)--(9.148,5.772)%
  --(9.151,5.773)--(9.153,5.775)--(9.156,5.777)--(9.159,5.779)--(9.162,5.781)--(9.165,5.783)%
  --(9.168,5.785)--(9.171,5.786)--(9.174,5.788)--(9.177,5.790)--(9.180,5.792)--(9.183,5.794)%
  --(9.186,5.796)--(9.189,5.798)--(9.192,5.800)--(9.195,5.801)--(9.198,5.803)--(9.201,5.805)%
  --(9.204,5.807)--(9.207,5.809)--(9.210,5.811)--(9.213,5.813)--(9.216,5.814)--(9.219,5.816)%
  --(9.222,5.818)--(9.225,5.820)--(9.228,5.822)--(9.231,5.824)--(9.234,5.826)--(9.237,5.827)%
  --(9.240,5.829)--(9.243,5.831)--(9.246,5.833)--(9.249,5.835)--(9.252,5.837)--(9.255,5.839)%
  --(9.258,5.841)--(9.261,5.842)--(9.264,5.844)--(9.267,5.846)--(9.270,5.848)--(9.273,5.850)%
  --(9.276,5.852)--(9.279,5.854)--(9.282,5.855)--(9.285,5.857)--(9.288,5.859)--(9.291,5.861)%
  --(9.294,5.863)--(9.297,5.865)--(9.300,5.867)--(9.303,5.869)--(9.306,5.870)--(9.309,5.872)%
  --(9.312,5.874)--(9.315,5.876)--(9.318,5.878)--(9.321,5.880)--(9.324,5.882)--(9.327,5.883)%
  --(9.330,5.885)--(9.333,5.887)--(9.336,5.889)--(9.339,5.891)--(9.342,5.893)--(9.345,5.895)%
  --(9.348,5.897)--(9.351,5.898)--(9.354,5.900)--(9.357,5.902)--(9.360,5.904)--(9.362,5.906)%
  --(9.365,5.908)--(9.368,5.910)--(9.371,5.912)--(9.374,5.913)--(9.377,5.915)--(9.380,5.917)%
  --(9.383,5.919)--(9.386,5.921)--(9.389,5.923)--(9.392,5.925)--(9.395,5.926)--(9.398,5.928)%
  --(9.401,5.930)--(9.404,5.932)--(9.407,5.934)--(9.410,5.936)--(9.413,5.938)--(9.416,5.940)%
  --(9.419,5.941)--(9.422,5.943)--(9.425,5.945)--(9.428,5.947)--(9.431,5.949)--(9.434,5.951)%
  --(9.437,5.953)--(9.440,5.954)--(9.443,5.956)--(9.446,5.958)--(9.449,5.960)--(9.452,5.962)%
  --(9.455,5.964)--(9.458,5.966)--(9.461,5.968)--(9.464,5.969)--(9.467,5.971)--(9.470,5.973)%
  --(9.473,5.975)--(9.476,5.977)--(9.479,5.979)--(9.482,5.981)--(9.485,5.983)--(9.488,5.984)%
  --(9.491,5.986)--(9.494,5.988)--(9.497,5.990)--(9.500,5.992)--(9.503,5.994)--(9.506,5.996)%
  --(9.509,5.998)--(9.512,5.999)--(9.515,6.001)--(9.518,6.003)--(9.521,6.005)--(9.524,6.007)%
  --(9.527,6.009)--(9.530,6.011)--(9.533,6.012)--(9.536,6.014)--(9.539,6.016)--(9.542,6.018)%
  --(9.545,6.020)--(9.548,6.022)--(9.551,6.024)--(9.554,6.026)--(9.557,6.027)--(9.560,6.029)%
  --(9.563,6.031)--(9.566,6.033)--(9.569,6.035)--(9.571,6.037)--(9.574,6.039)--(9.577,6.041)%
  --(9.580,6.042)--(9.583,6.044)--(9.586,6.046)--(9.589,6.048)--(9.592,6.050)--(9.595,6.052)%
  --(9.598,6.054)--(9.601,6.056)--(9.604,6.057)--(9.607,6.059)--(9.610,6.061)--(9.613,6.063)%
  --(9.616,6.065)--(9.619,6.067)--(9.622,6.069)--(9.625,6.070)--(9.628,6.072)--(9.631,6.074)%
  --(9.634,6.076)--(9.637,6.078)--(9.640,6.080)--(9.643,6.082)--(9.646,6.084)--(9.649,6.085)%
  --(9.652,6.087)--(9.655,6.089)--(9.658,6.091)--(9.661,6.093)--(9.664,6.095)--(9.667,6.097)%
  --(9.670,6.099)--(9.673,6.100)--(9.676,6.102)--(9.679,6.104)--(9.682,6.106)--(9.685,6.108)%
  --(9.688,6.110)--(9.691,6.112)--(9.694,6.114)--(9.697,6.115)--(9.700,6.117)--(9.703,6.119)%
  --(9.706,6.121)--(9.709,6.123)--(9.712,6.125)--(9.715,6.127)--(9.718,6.129)--(9.721,6.130)%
  --(9.724,6.132)--(9.727,6.134)--(9.730,6.136)--(9.733,6.138)--(9.736,6.140)--(9.739,6.142)%
  --(9.742,6.144)--(9.745,6.145)--(9.748,6.147)--(9.751,6.149)--(9.754,6.151)--(9.757,6.153)%
  --(9.760,6.155)--(9.763,6.157)--(9.766,6.159)--(9.769,6.160)--(9.772,6.162)--(9.775,6.164)%
  --(9.778,6.166)--(9.780,6.168)--(9.783,6.170)--(9.786,6.172)--(9.789,6.174)--(9.792,6.175)%
  --(9.795,6.177)--(9.798,6.179)--(9.801,6.181)--(9.804,6.183)--(9.807,6.185)--(9.810,6.187)%
  --(9.813,6.188)--(9.816,6.190)--(9.819,6.192)--(9.822,6.194)--(9.825,6.196)--(9.828,6.198)%
  --(9.831,6.200)--(9.834,6.202)--(9.837,6.203)--(9.840,6.205)--(9.843,6.207)--(9.846,6.209)%
  --(9.849,6.211)--(9.852,6.213)--(9.855,6.215)--(9.858,6.217)--(9.861,6.218)--(9.864,6.220)%
  --(9.867,6.222)--(9.870,6.224)--(9.873,6.226)--(9.876,6.228)--(9.879,6.230)--(9.882,6.232)%
  --(9.885,6.233)--(9.888,6.235)--(9.891,6.237)--(9.894,6.239)--(9.897,6.241)--(9.900,6.243)%
  --(9.903,6.245)--(9.906,6.247)--(9.909,6.248)--(9.912,6.250)--(9.915,6.252)--(9.918,6.254)%
  --(9.921,6.256)--(9.924,6.258)--(9.927,6.260)--(9.930,6.262)--(9.933,6.264)--(9.936,6.265)%
  --(9.939,6.267)--(9.942,6.269)--(9.945,6.271)--(9.948,6.273)--(9.951,6.275)--(9.954,6.277)%
  --(9.957,6.279)--(9.960,6.280)--(9.963,6.282)--(9.966,6.284)--(9.969,6.286)--(9.972,6.288)%
  --(9.975,6.290)--(9.978,6.292)--(9.981,6.294)--(9.984,6.295)--(9.987,6.297)--(9.990,6.299)%
  --(9.992,6.301)--(9.995,6.303)--(9.998,6.305)--(10.001,6.307)--(10.004,6.309)--(10.007,6.310)%
  --(10.010,6.312)--(10.013,6.314)--(10.016,6.316)--(10.019,6.318)--(10.022,6.320)--(10.025,6.322)%
  --(10.028,6.324)--(10.031,6.325)--(10.034,6.327)--(10.037,6.329)--(10.040,6.331)--(10.043,6.333)%
  --(10.046,6.335)--(10.049,6.337)--(10.052,6.339)--(10.055,6.340)--(10.058,6.342)--(10.061,6.344)%
  --(10.064,6.346)--(10.067,6.348)--(10.070,6.350)--(10.073,6.352)--(10.076,6.354)--(10.079,6.355)%
  --(10.082,6.357)--(10.085,6.359)--(10.088,6.361)--(10.091,6.363)--(10.094,6.365)--(10.097,6.367)%
  --(10.100,6.369)--(10.103,6.370)--(10.106,6.372)--(10.109,6.374)--(10.112,6.376)--(10.115,6.378)%
  --(10.118,6.380)--(10.121,6.382)--(10.124,6.384)--(10.127,6.386)--(10.130,6.387)--(10.133,6.389)%
  --(10.136,6.391)--(10.139,6.393)--(10.142,6.395)--(10.145,6.397)--(10.148,6.399)--(10.151,6.401)%
  --(10.154,6.402)--(10.157,6.404)--(10.160,6.406)--(10.163,6.408)--(10.166,6.410)--(10.169,6.412)%
  --(10.172,6.414)--(10.175,6.416)--(10.178,6.417)--(10.181,6.419)--(10.184,6.421)--(10.187,6.423)%
  --(10.190,6.425)--(10.193,6.427)--(10.196,6.429)--(10.199,6.431)--(10.201,6.432)--(10.204,6.434)%
  --(10.207,6.436)--(10.210,6.438)--(10.213,6.440)--(10.216,6.442)--(10.219,6.444)--(10.222,6.446)%
  --(10.225,6.448)--(10.228,6.449)--(10.231,6.451)--(10.234,6.453)--(10.237,6.455)--(10.240,6.457)%
  --(10.243,6.459)--(10.246,6.461)--(10.249,6.463)--(10.252,6.464)--(10.255,6.466)--(10.258,6.468)%
  --(10.261,6.470)--(10.264,6.472)--(10.267,6.474)--(10.270,6.476)--(10.273,6.478)--(10.276,6.479)%
  --(10.279,6.481)--(10.282,6.483)--(10.285,6.485)--(10.288,6.487)--(10.291,6.489)--(10.294,6.491)%
  --(10.297,6.493)--(10.300,6.494)--(10.303,6.496)--(10.306,6.498)--(10.309,6.500)--(10.312,6.502)%
  --(10.315,6.504)--(10.318,6.506)--(10.321,6.508)--(10.324,6.510)--(10.327,6.511)--(10.330,6.513)%
  --(10.333,6.515)--(10.336,6.517)--(10.339,6.519)--(10.342,6.521)--(10.345,6.523)--(10.348,6.525)%
  --(10.351,6.526)--(10.354,6.528)--(10.357,6.530)--(10.360,6.532)--(10.363,6.534)--(10.366,6.536)%
  --(10.369,6.538)--(10.372,6.540)--(10.375,6.542)--(10.378,6.543)--(10.381,6.545)--(10.384,6.547)%
  --(10.387,6.549)--(10.390,6.551)--(10.393,6.553)--(10.396,6.555)--(10.399,6.557)--(10.402,6.558)%
  --(10.405,6.560)--(10.408,6.562)--(10.410,6.564)--(10.413,6.566)--(10.416,6.568)--(10.419,6.570)%
  --(10.422,6.572)--(10.425,6.573)--(10.428,6.575)--(10.431,6.577)--(10.434,6.579)--(10.437,6.581)%
  --(10.440,6.583)--(10.443,6.585)--(10.446,6.587)--(10.449,6.589)--(10.452,6.590)--(10.455,6.592)%
  --(10.458,6.594)--(10.461,6.596)--(10.464,6.598)--(10.467,6.600)--(10.470,6.602)--(10.473,6.604)%
  --(10.476,6.605)--(10.479,6.607)--(10.482,6.609)--(10.485,6.611)--(10.488,6.613)--(10.491,6.615)%
  --(10.494,6.617)--(10.497,6.619)--(10.500,6.621)--(10.503,6.622)--(10.506,6.624)--(10.509,6.626)%
  --(10.512,6.628)--(10.515,6.630)--(10.518,6.632)--(10.521,6.634)--(10.524,6.636)--(10.527,6.637)%
  --(10.530,6.639)--(10.533,6.641)--(10.536,6.643)--(10.539,6.645)--(10.542,6.647)--(10.545,6.649)%
  --(10.548,6.651)--(10.551,6.653)--(10.554,6.654)--(10.557,6.656)--(10.560,6.658)--(10.563,6.660)%
  --(10.566,6.662)--(10.569,6.664)--(10.572,6.666)--(10.575,6.668)--(10.578,6.669)--(10.581,6.671)%
  --(10.584,6.673)--(10.587,6.675)--(10.590,6.677)--(10.593,6.679)--(10.596,6.681)--(10.599,6.683)%
  --(10.602,6.685)--(10.605,6.686)--(10.608,6.688)--(10.611,6.690)--(10.614,6.692)--(10.617,6.694)%
  --(10.619,6.696)--(10.622,6.698)--(10.625,6.700)--(10.628,6.701)--(10.631,6.703)--(10.634,6.705)%
  --(10.637,6.707)--(10.640,6.709)--(10.643,6.711)--(10.646,6.713)--(10.649,6.715)--(10.652,6.717)%
  --(10.655,6.718)--(10.658,6.720)--(10.661,6.722)--(10.664,6.724)--(10.667,6.726)--(10.670,6.728)%
  --(10.673,6.730)--(10.676,6.732)--(10.679,6.733)--(10.682,6.735)--(10.685,6.737)--(10.688,6.739)%
  --(10.691,6.741)--(10.694,6.743)--(10.697,6.745)--(10.700,6.747)--(10.703,6.749)--(10.706,6.750)%
  --(10.709,6.752)--(10.712,6.754)--(10.715,6.756)--(10.718,6.758)--(10.721,6.760)--(10.724,6.762)%
  --(10.727,6.764)--(10.730,6.766)--(10.733,6.767)--(10.736,6.769)--(10.739,6.771)--(10.742,6.773)%
  --(10.745,6.775)--(10.748,6.777)--(10.751,6.779)--(10.754,6.781)--(10.757,6.782)--(10.760,6.784)%
  --(10.763,6.786)--(10.766,6.788)--(10.769,6.790)--(10.772,6.792)--(10.775,6.794)--(10.778,6.796)%
  --(10.781,6.798)--(10.784,6.799)--(10.787,6.801)--(10.790,6.803)--(10.793,6.805)--(10.796,6.807)%
  --(10.799,6.809)--(10.802,6.811)--(10.805,6.813)--(10.808,6.814)--(10.811,6.816)--(10.814,6.818)%
  --(10.817,6.820)--(10.820,6.822)--(10.823,6.824)--(10.826,6.826)--(10.828,6.828)--(10.831,6.830)%
  --(10.834,6.831)--(10.837,6.833)--(10.840,6.835)--(10.843,6.837)--(10.846,6.839)--(10.849,6.841)%
  --(10.852,6.843)--(10.855,6.845)--(10.858,6.847)--(10.861,6.848)--(10.864,6.850)--(10.867,6.852)%
  --(10.870,6.854)--(10.873,6.856)--(10.876,6.858)--(10.879,6.860)--(10.882,6.862)--(10.885,6.863)%
  --(10.888,6.865)--(10.891,6.867)--(10.894,6.869)--(10.897,6.871)--(10.900,6.873)--(10.903,6.875)%
  --(10.906,6.877)--(10.909,6.879)--(10.912,6.880)--(10.915,6.882)--(10.918,6.884)--(10.921,6.886)%
  --(10.924,6.888)--(10.927,6.890)--(10.930,6.892)--(10.933,6.894)--(10.936,6.896)--(10.939,6.897)%
  --(10.942,6.899)--(10.945,6.901)--(10.948,6.903)--(10.951,6.905)--(10.954,6.907)--(10.957,6.909)%
  --(10.960,6.911)--(10.963,6.913)--(10.966,6.914)--(10.969,6.916)--(10.972,6.918)--(10.975,6.920)%
  --(10.978,6.922)--(10.981,6.924)--(10.984,6.926)--(10.987,6.928)--(10.990,6.929)--(10.993,6.931)%
  --(10.996,6.933)--(10.999,6.935)--(11.002,6.937)--(11.005,6.939)--(11.008,6.941)--(11.011,6.943)%
  --(11.014,6.945)--(11.017,6.946)--(11.020,6.948)--(11.023,6.950)--(11.026,6.952)--(11.029,6.954)%
  --(11.032,6.956)--(11.035,6.958)--(11.037,6.960)--(11.040,6.962)--(11.043,6.963)--(11.046,6.965)%
  --(11.049,6.967)--(11.052,6.969)--(11.055,6.971)--(11.058,6.973)--(11.061,6.975)--(11.064,6.977)%
  --(11.067,6.979)--(11.070,6.980)--(11.073,6.982)--(11.076,6.984)--(11.079,6.986)--(11.082,6.988)%
  --(11.085,6.990)--(11.088,6.992)--(11.091,6.994)--(11.094,6.996)--(11.097,6.997)--(11.100,6.999)%
  --(11.103,7.001)--(11.106,7.003)--(11.109,7.005)--(11.112,7.007)--(11.115,7.009)--(11.118,7.011)%
  --(11.121,7.013)--(11.124,7.014)--(11.127,7.016)--(11.130,7.018)--(11.133,7.020)--(11.136,7.022)%
  --(11.139,7.024)--(11.142,7.026)--(11.145,7.028)--(11.148,7.029)--(11.151,7.031)--(11.154,7.033)%
  --(11.157,7.035)--(11.160,7.037)--(11.163,7.039)--(11.166,7.041)--(11.169,7.043)--(11.172,7.045)%
  --(11.175,7.046)--(11.178,7.048)--(11.181,7.050)--(11.184,7.052)--(11.187,7.054)--(11.190,7.056)%
  --(11.193,7.058)--(11.196,7.060)--(11.199,7.062)--(11.202,7.063)--(11.205,7.065)--(11.208,7.067)%
  --(11.211,7.069)--(11.214,7.071)--(11.217,7.073)--(11.220,7.075)--(11.223,7.077)--(11.226,7.079)%
  --(11.229,7.080)--(11.232,7.082)--(11.235,7.084)--(11.238,7.086)--(11.241,7.088)--(11.244,7.090)%
  --(11.247,7.092)--(11.249,7.094)--(11.252,7.096)--(11.255,7.097)--(11.258,7.099)--(11.261,7.101)%
  --(11.264,7.103)--(11.267,7.105)--(11.270,7.107)--(11.273,7.109)--(11.276,7.111)--(11.279,7.113)%
  --(11.282,7.114)--(11.285,7.116)--(11.288,7.118)--(11.291,7.120)--(11.294,7.122)--(11.297,7.124)%
  --(11.300,7.126)--(11.303,7.128)--(11.306,7.130)--(11.309,7.131)--(11.312,7.133)--(11.315,7.135)%
  --(11.318,7.137)--(11.321,7.139)--(11.324,7.141)--(11.327,7.143)--(11.330,7.145)--(11.333,7.147)%
  --(11.336,7.148)--(11.339,7.150)--(11.342,7.152)--(11.345,7.154)--(11.348,7.156)--(11.351,7.158)%
  --(11.354,7.160)--(11.357,7.162)--(11.360,7.164)--(11.363,7.165)--(11.366,7.167)--(11.369,7.169)%
  --(11.372,7.171)--(11.375,7.173)--(11.378,7.175)--(11.381,7.177)--(11.384,7.179)--(11.387,7.181)%
  --(11.390,7.182)--(11.393,7.184)--(11.396,7.186)--(11.399,7.188)--(11.402,7.190)--(11.405,7.192)%
  --(11.408,7.194)--(11.411,7.196)--(11.414,7.198)--(11.417,7.199)--(11.420,7.201)--(11.423,7.203)%
  --(11.426,7.205)--(11.429,7.207)--(11.432,7.209)--(11.435,7.211)--(11.438,7.213)--(11.441,7.215)%
  --(11.444,7.216)--(11.447,7.218)--(11.450,7.220)--(11.453,7.222)--(11.456,7.224)--(11.458,7.226)%
  --(11.461,7.228)--(11.464,7.230)--(11.467,7.232)--(11.470,7.233)--(11.473,7.235)--(11.476,7.237)%
  --(11.479,7.239)--(11.482,7.241)--(11.485,7.243)--(11.488,7.245)--(11.491,7.247)--(11.494,7.249)%
  --(11.497,7.250)--(11.500,7.252)--(11.503,7.254)--(11.506,7.256)--(11.509,7.258)--(11.512,7.260)%
  --(11.515,7.262)--(11.518,7.264)--(11.521,7.266)--(11.524,7.267)--(11.527,7.269)--(11.530,7.271)%
  --(11.533,7.273)--(11.536,7.275)--(11.539,7.277)--(11.542,7.279)--(11.545,7.281)--(11.548,7.283)%
  --(11.551,7.284)--(11.554,7.286)--(11.557,7.288)--(11.560,7.290)--(11.563,7.292)--(11.566,7.294)%
  --(11.569,7.296)--(11.572,7.298)--(11.575,7.300)--(11.578,7.301)--(11.581,7.303)--(11.584,7.305)%
  --(11.587,7.307)--(11.590,7.309)--(11.593,7.311)--(11.596,7.313)--(11.599,7.315)--(11.602,7.317)%
  --(11.605,7.318)--(11.608,7.320)--(11.611,7.322)--(11.614,7.324)--(11.617,7.326)--(11.620,7.328)%
  --(11.623,7.330)--(11.626,7.332)--(11.629,7.334)--(11.632,7.335)--(11.635,7.337)--(11.638,7.339)%
  --(11.641,7.341)--(11.644,7.343)--(11.647,7.345)--(11.650,7.347)--(11.653,7.349)--(11.656,7.351)%
  --(11.659,7.352)--(11.662,7.354)--(11.665,7.356)--(11.667,7.358)--(11.670,7.360)--(11.673,7.362)%
  --(11.676,7.364)--(11.679,7.366)--(11.682,7.368)--(11.685,7.370)--(11.688,7.371)--(11.691,7.373)%
  --(11.694,7.375)--(11.697,7.377)--(11.700,7.379)--(11.703,7.381)--(11.706,7.383)--(11.709,7.385)%
  --(11.712,7.387)--(11.715,7.388)--(11.718,7.390)--(11.721,7.392)--(11.724,7.394)--(11.727,7.396)%
  --(11.730,7.398)--(11.733,7.400)--(11.736,7.402)--(11.739,7.404)--(11.742,7.405)--(11.745,7.407)%
  --(11.748,7.409)--(11.751,7.411)--(11.754,7.413)--(11.757,7.415)--(11.760,7.417)--(11.763,7.419)%
  --(11.766,7.421)--(11.769,7.422)--(11.772,7.424)--(11.775,7.426)--(11.778,7.428)--(11.781,7.430)%
  --(11.784,7.432)--(11.787,7.434)--(11.790,7.436)--(11.793,7.438)--(11.796,7.439)--(11.799,7.441)%
  --(11.802,7.443)--(11.805,7.445)--(11.808,7.447)--(11.811,7.449)--(11.814,7.451)--(11.817,7.453)%
  --(11.820,7.455)--(11.823,7.456)--(11.826,7.458)--(11.829,7.460)--(11.832,7.462)--(11.835,7.464)%
  --(11.838,7.466)--(11.841,7.468)--(11.844,7.470)--(11.847,7.472)--(11.850,7.474)--(11.853,7.475)%
  --(11.856,7.477)--(11.859,7.479)--(11.862,7.481)--(11.865,7.483)--(11.868,7.485)--(11.871,7.487)%
  --(11.874,7.489)--(11.876,7.491)--(11.879,7.492)--(11.882,7.494)--(11.885,7.496)--(11.888,7.498)%
  --(11.891,7.500)--(11.894,7.502)--(11.897,7.504)--(11.900,7.506)--(11.903,7.508)--(11.906,7.509)%
  --(11.909,7.511)--(11.912,7.513)--(11.915,7.515)--(11.918,7.517)--(11.921,7.519)--(11.924,7.521)%
  --(11.927,7.523)--(11.930,7.525)--(11.933,7.526)--(11.936,7.528)--(11.939,7.530)--(11.942,7.532)%
  --(11.945,7.534)--(11.948,7.536)--(11.951,7.538)--(11.954,7.540)--(11.957,7.542)--(11.960,7.544)%
  --(11.963,7.545)--(11.966,7.547)--(11.969,7.549)--(11.972,7.551)--(11.975,7.553)--(11.978,7.555)%
  --(11.981,7.557)--(11.984,7.559)--(11.987,7.561)--(11.990,7.562)--(11.993,7.564)--(11.996,7.566)%
  --(11.999,7.568)--(12.002,7.570)--(12.005,7.572)--(12.008,7.574)--(12.011,7.576)--(12.014,7.578)%
  --(12.017,7.579)--(12.020,7.581)--(12.023,7.583)--(12.026,7.585)--(12.029,7.587)--(12.032,7.589)%
  --(12.035,7.591)--(12.038,7.593)--(12.041,7.595)--(12.044,7.597)--(12.047,7.598)--(12.050,7.600)%
  --(12.053,7.602)--(12.056,7.604)--(12.059,7.606)--(12.062,7.608)--(12.065,7.610)--(12.068,7.612)%
  --(12.071,7.614)--(12.074,7.615)--(12.077,7.617)--(12.080,7.619)--(12.083,7.621)--(12.085,7.623)%
  --(12.088,7.625)--(12.091,7.627)--(12.094,7.629)--(12.097,7.631)--(12.100,7.632)--(12.103,7.634)%
  --(12.106,7.636)--(12.109,7.638)--(12.112,7.640)--(12.115,7.642)--(12.118,7.644)--(12.121,7.646)%
  --(12.124,7.648)--(12.127,7.650)--(12.130,7.651)--(12.133,7.653)--(12.136,7.655)--(12.139,7.657)%
  --(12.142,7.659)--(12.145,7.661)--(12.148,7.663)--(12.151,7.665)--(12.154,7.667)--(12.157,7.668)%
  --(12.160,7.670)--(12.163,7.672)--(12.166,7.674)--(12.169,7.676)--(12.172,7.678)--(12.175,7.680)%
  --(12.178,7.682)--(12.181,7.684)--(12.184,7.685)--(12.187,7.687)--(12.190,7.689)--(12.193,7.691)%
  --(12.196,7.693)--(12.199,7.695)--(12.202,7.697)--(12.205,7.699)--(12.208,7.701)--(12.211,7.703)%
  --(12.214,7.704)--(12.217,7.706)--(12.220,7.708)--(12.223,7.710)--(12.226,7.712)--(12.229,7.714)%
  --(12.232,7.716)--(12.235,7.718)--(12.238,7.720)--(12.241,7.721)--(12.244,7.723)--(12.247,7.725)%
  --(12.250,7.727)--(12.253,7.729)--(12.256,7.731)--(12.259,7.733)--(12.262,7.735)--(12.265,7.737)%
  --(12.268,7.739)--(12.271,7.740)--(12.274,7.742)--(12.277,7.744)--(12.280,7.746)--(12.283,7.748)%
  --(12.286,7.750)--(12.289,7.752)--(12.292,7.754)--(12.295,7.756)--(12.297,7.757)--(12.300,7.759)%
  --(12.303,7.761)--(12.306,7.763)--(12.309,7.765)--(12.312,7.767)--(12.315,7.769)--(12.318,7.771)%
  --(12.321,7.773)--(12.324,7.774)--(12.327,7.776)--(12.330,7.778)--(12.333,7.780)--(12.336,7.782)%
  --(12.339,7.784)--(12.342,7.786)--(12.345,7.788)--(12.348,7.790)--(12.351,7.792)--(12.354,7.793)%
  --(12.357,7.795)--(12.360,7.797)--(12.363,7.799)--(12.366,7.801)--(12.369,7.803)--(12.372,7.805)%
  --(12.375,7.807)--(12.378,7.809)--(12.381,7.810)--(12.384,7.812)--(12.387,7.814)--(12.390,7.816)%
  --(12.393,7.818)--(12.396,7.820)--(12.399,7.822)--(12.402,7.824)--(12.405,7.826)--(12.408,7.828)%
  --(12.411,7.829)--(12.414,7.831)--(12.417,7.833)--(12.420,7.835)--(12.423,7.837)--(12.426,7.839)%
  --(12.429,7.841)--(12.432,7.843)--(12.435,7.845)--(12.438,7.846)--(12.441,7.848)--(12.444,7.850)%
  --(12.447,7.852)--(12.450,7.854)--(12.453,7.856)--(12.456,7.858)--(12.459,7.860)--(12.462,7.862)%
  --(12.465,7.864)--(12.468,7.865)--(12.471,7.867)--(12.474,7.869)--(12.477,7.871)--(12.480,7.873)%
  --(12.483,7.875)--(12.486,7.877)--(12.489,7.879)--(12.492,7.881)--(12.495,7.882)--(12.498,7.884)%
  --(12.501,7.886)--(12.504,7.888)--(12.506,7.890)--(12.509,7.892)--(12.512,7.894)--(12.515,7.896)%
  --(12.518,7.898)--(12.521,7.900)--(12.524,7.901)--(12.527,7.903)--(12.530,7.905)--(12.533,7.907)%
  --(12.536,7.909)--(12.539,7.911)--(12.542,7.913)--(12.545,7.915)--(12.548,7.917)--(12.551,7.918)%
  --(12.554,7.920)--(12.557,7.922)--(12.560,7.924)--(12.563,7.926)--(12.566,7.928)--(12.569,7.930)%
  --(12.572,7.932)--(12.575,7.934)--(12.578,7.936)--(12.581,7.937)--(12.584,7.939)--(12.587,7.941)%
  --(12.590,7.943)--(12.593,7.945)--(12.596,7.947)--(12.599,7.949)--(12.602,7.951)--(12.605,7.953)%
  --(12.608,7.954)--(12.611,7.956)--(12.614,7.958)--(12.617,7.960)--(12.620,7.962)--(12.623,7.964)%
  --(12.626,7.966)--(12.629,7.968)--(12.632,7.970)--(12.635,7.972)--(12.638,7.973)--(12.641,7.975)%
  --(12.644,7.977)--(12.647,7.979)--(12.650,7.981)--(12.653,7.983)--(12.656,7.985)--(12.659,7.987)%
  --(12.662,7.989)--(12.665,7.990)--(12.668,7.992)--(12.671,7.994)--(12.674,7.996)--(12.677,7.998)%
  --(12.680,8.000)--(12.683,8.002)--(12.686,8.004)--(12.689,8.006)--(12.692,8.008)--(12.695,8.009)%
  --(12.698,8.011)--(12.701,8.013)--(12.704,8.015)--(12.707,8.017)--(12.710,8.019)--(12.713,8.021)%
  --(12.715,8.023)--(12.718,8.025)--(12.721,8.027)--(12.724,8.028)--(12.727,8.030)--(12.730,8.032)%
  --(12.733,8.034)--(12.736,8.036)--(12.739,8.038)--(12.742,8.040)--(12.745,8.042)--(12.748,8.044)%
  --(12.751,8.045)--(12.754,8.047)--(12.757,8.049)--(12.760,8.051)--(12.763,8.053)--(12.766,8.055)%
  --(12.769,8.057)--(12.772,8.059)--(12.775,8.061)--(12.778,8.063)--(12.781,8.064)--(12.784,8.066)%
  --(12.787,8.068)--(12.790,8.070)--(12.793,8.072)--(12.796,8.074)--(12.799,8.076)--(12.802,8.078)%
  --(12.805,8.080)--(12.808,8.081)--(12.811,8.083)--(12.814,8.085)--(12.817,8.087)--(12.820,8.089)%
  --(12.823,8.091)--(12.826,8.093)--(12.829,8.095)--(12.832,8.097)--(12.835,8.099)--(12.838,8.100)%
  --(12.841,8.102)--(12.844,8.104)--(12.847,8.106)--(12.850,8.108)--(12.853,8.110)--(12.856,8.112)%
  --(12.859,8.114)--(12.862,8.116)--(12.865,8.118)--(12.868,8.119)--(12.871,8.121)--(12.874,8.123)%
  --(12.877,8.125)--(12.880,8.127)--(12.883,8.129)--(12.886,8.131)--(12.889,8.133)--(12.892,8.135)%
  --(12.895,8.136)--(12.898,8.138)--(12.901,8.140)--(12.904,8.142)--(12.907,8.144)--(12.910,8.146)%
  --(12.913,8.148)--(12.916,8.150)--(12.919,8.152)--(12.922,8.154)--(12.924,8.155)--(12.927,8.157)%
  --(12.930,8.159)--(12.933,8.161)--(12.936,8.163)--(12.939,8.165)--(12.942,8.167)--(12.945,8.169)%
  --(12.948,8.171)--(12.951,8.173)--(12.954,8.174)--(12.957,8.176)--(12.960,8.178)--(12.963,8.180)%
  --(12.966,8.182)--(12.969,8.184)--(12.972,8.186)--(12.975,8.188)--(12.978,8.190)--(12.981,8.191)%
  --(12.984,8.193)--(12.987,8.195)--(12.990,8.197)--(12.993,8.199)--(12.996,8.201)--(12.999,8.203)%
  --(13.002,8.205)--(13.005,8.207)--(13.008,8.209)--(13.011,8.210)--(13.014,8.212)--(13.017,8.214)%
  --(13.020,8.216)--(13.023,8.218)--(13.026,8.220)--(13.029,8.222)--(13.032,8.224)--(13.035,8.226)%
  --(13.038,8.228)--(13.041,8.229)--(13.044,8.231)--(13.047,8.233)--(13.050,8.235)--(13.053,8.237)%
  --(13.056,8.239)--(13.059,8.241)--(13.062,8.243)--(13.065,8.245)--(13.068,8.247)--(13.071,8.248)%
  --(13.074,8.250)--(13.077,8.252)--(13.080,8.254)--(13.083,8.256)--(13.086,8.258)--(13.089,8.260)%
  --(13.092,8.262)--(13.095,8.264)--(13.098,8.265)--(13.101,8.267)--(13.104,8.269)--(13.107,8.271)%
  --(13.110,8.273)--(13.113,8.275)--(13.116,8.277)--(13.119,8.279)--(13.122,8.281)--(13.125,8.283)%
  --(13.128,8.284)--(13.131,8.286)--(13.133,8.288)--(13.136,8.290)--(13.139,8.292)--(13.142,8.294)%
  --(13.145,8.296)--(13.148,8.298)--(13.151,8.300)--(13.154,8.302)--(13.157,8.303)--(13.160,8.305)%
  --(13.163,8.307)--(13.166,8.309)--(13.169,8.311)--(13.172,8.313)--(13.175,8.315)--(13.178,8.317)%
  --(13.181,8.319)--(13.184,8.320)--(13.187,8.322)--(13.190,8.324)--(13.193,8.326)--(13.196,8.328)%
  --(13.199,8.330)--(13.202,8.332)--(13.205,8.334)--(13.208,8.336)--(13.211,8.338)--(13.214,8.339)%
  --(13.217,8.341)--(13.220,8.343)--(13.223,8.345)--(13.226,8.347)--(13.229,8.349)--(13.232,8.351)%
  --(13.235,8.353)--(13.238,8.355)--(13.241,8.357)--(13.244,8.358)--(13.247,8.360)--(13.250,8.362)%
  --(13.253,8.364)--(13.256,8.366)--(13.259,8.368)--(13.262,8.370)--(13.265,8.372)--(13.268,8.374)%
  --(13.271,8.376)--(13.274,8.377)--(13.277,8.379)--(13.280,8.381)--(13.283,8.383)--(13.286,8.385)%
  --(13.289,8.387)--(13.292,8.389)--(13.295,8.391)--(13.298,8.393)--(13.301,8.395)--(13.304,8.396)%
  --(13.307,8.398)--(13.310,8.400)--(13.313,8.402)--(13.316,8.404)--(13.319,8.406)--(13.322,8.408)%
  --(13.325,8.410)--(13.328,8.412)--(13.331,8.413)--(13.334,8.415)--(13.337,8.417)--(13.340,8.419)%
  --(13.342,8.421)--(13.345,8.423)--(13.348,8.425)--(13.351,8.427)--(13.354,8.429)--(13.357,8.431)%
  --(13.360,8.432)--(13.363,8.434)--(13.366,8.436)--(13.369,8.438)--(13.372,8.440)--(13.375,8.442)%
  --(13.378,8.444)--(13.381,8.446)--(13.384,8.448)--(13.387,8.450)--(13.390,8.451)--(13.393,8.453)%
  --(13.396,8.455)--(13.399,8.457)--(13.402,8.459)--(13.405,8.461)--(13.408,8.463)--(13.411,8.465)%
  --(13.414,8.467)--(13.417,8.469)--(13.420,8.470)--(13.423,8.472)--(13.426,8.474)--(13.429,8.476)%
  --(13.432,8.478)--(13.435,8.480)--(13.438,8.482)--(13.441,8.484)--(13.444,8.486);
\gpcolor{rgb color={0.902,0.624,0.000}}
\draw[gp path] (1.504,2.514)--(1.625,2.514)--(1.745,2.514)--(1.866,2.514)--(1.986,2.514)%
  --(2.107,2.514)--(2.228,2.514)--(2.348,2.514)--(2.469,2.514)--(2.589,2.514)--(2.710,2.514)%
  --(2.831,2.514)--(2.951,2.514)--(3.072,2.514)--(3.192,2.514)--(3.313,2.514)--(3.434,2.514)%
  --(3.554,2.514)--(3.675,2.514)--(3.796,2.514)--(3.916,2.514)--(4.037,2.514)--(4.157,2.514)%
  --(4.278,2.514)--(4.399,2.514)--(4.519,2.514)--(4.640,2.514)--(4.760,2.514)--(4.881,2.514)%
  --(5.002,2.514)--(5.122,2.514)--(5.243,2.514)--(5.363,2.514)--(5.484,2.514)--(5.605,2.514)%
  --(5.725,2.514)--(5.846,2.514)--(5.966,2.514)--(6.087,2.514)--(6.208,2.514)--(6.328,2.514)%
  --(6.449,2.514)--(6.569,2.514)--(6.690,2.514)--(6.811,2.514)--(6.931,2.514)--(7.052,2.514)%
  --(7.172,2.514)--(7.293,2.514)--(7.414,2.514)--(7.534,2.514)--(7.655,2.514)--(7.776,2.514)%
  --(7.896,2.514)--(8.017,2.514)--(8.137,2.514)--(8.258,2.514)--(8.379,2.514)--(8.499,2.514)%
  --(8.620,2.514)--(8.740,2.514)--(8.861,2.514)--(8.982,2.514)--(9.102,2.514)--(9.223,2.514)%
  --(9.343,2.514)--(9.464,2.514)--(9.585,2.514)--(9.705,2.514)--(9.826,2.514)--(9.946,2.514)%
  --(10.067,2.514)--(10.188,2.514)--(10.308,2.514)--(10.429,2.514)--(10.549,2.514)--(10.670,2.514)%
  --(10.791,2.514)--(10.911,2.514)--(11.032,2.514)--(11.152,2.514)--(11.273,2.514)--(11.394,2.514)%
  --(11.514,2.514)--(11.635,2.514)--(11.756,2.514)--(11.876,2.514)--(11.997,2.514)--(12.117,2.514)%
  --(12.238,2.514)--(12.359,2.514)--(12.479,2.514)--(12.600,2.514)--(12.720,2.514)--(12.841,2.514)%
  --(12.962,2.514)--(13.082,2.514)--(13.203,2.514)--(13.323,2.514)--(13.444,2.514);
\gpcolor{color=gp lt color border}
\draw[gp path] (1.504,8.631)--(1.504,0.985)--(13.447,0.985)--(13.447,8.631)--cycle;
%% coordinates of the plot area
\gpdefrectangularnode{gp plot 1}{\pgfpoint{1.504cm}{0.985cm}}{\pgfpoint{13.447cm}{8.631cm}}
\end{tikzpicture}
%% gnuplot variables

	\caption{Exemplos de formas diferentes para $f(M)$ para valores diferentes de $\rho$. Os momentos de Fermi foram determinados a partir de \eqref{Eq:Mom_Fermi_a_partir_de_rho} com fração de prótons 1/2.}
	\label{Fig:Gap_zero_graph}
\end{figure*}

Portanto, ao se utilizar um método qualquer para se determinar o zero da função, são necessários parâmetros que tornem possível encontrar o zero da função. Como a função só tem um zero\footnote{Desprezando a solução trivial $M = 0$.}, temos uma situação mais simples, pois não há risco de que tenhamos que encontrar mais que um mínimo (se esse fosse o caso, teríamos que encontrar aquele que minimiza alguma energia\cite{Buballa}. Verificar.). Adotando o método da biseção, que é o mais simples, só precisamos de dois pontos que devem estar à esquerda e à direita do zero. Adotamos os valores \np{1.0e-3} e 1500 (em MeV), sendo o primeiro bastante próximo de zero e o segundo aparentemente se posiciona à direita do zero para vários valores de densidade. Não podemos adotar zero para o valor à esquerda pois $f(M=0) = 0$. A Fig.~\ref{Fig:mass_graph} mostra os valores de $M$ obtidos de acordo com o acima exposto, enquanto a Fig.~\ref{Fig:scalar_density_graph} mostra a curva das densidades escalares correspondentes.

\begin{figure*}
	\begin{tikzpicture}[gnuplot]
%% generated with GNUPLOT 5.0p2 (Lua 5.2; terminal rev. 99, script rev. 100)
%% Thu Feb 25 17:17:56 2016
\path (0.000,0.000) rectangle (14.000,9.000);
\gpcolor{color=gp lt color border}
\gpsetlinetype{gp lt border}
\gpsetdashtype{gp dt solid}
\gpsetlinewidth{1.00}
\draw[gp path] (1.504,0.985)--(1.684,0.985);
\draw[gp path] (13.447,0.985)--(13.267,0.985);
\node[gp node right] at (1.320,0.985) {$0$};
\draw[gp path] (1.504,1.750)--(1.684,1.750);
\draw[gp path] (13.447,1.750)--(13.267,1.750);
\node[gp node right] at (1.320,1.750) {$100$};
\draw[gp path] (1.504,2.514)--(1.684,2.514);
\draw[gp path] (13.447,2.514)--(13.267,2.514);
\node[gp node right] at (1.320,2.514) {$200$};
\draw[gp path] (1.504,3.279)--(1.684,3.279);
\draw[gp path] (13.447,3.279)--(13.267,3.279);
\node[gp node right] at (1.320,3.279) {$300$};
\draw[gp path] (1.504,4.043)--(1.684,4.043);
\draw[gp path] (13.447,4.043)--(13.267,4.043);
\node[gp node right] at (1.320,4.043) {$400$};
\draw[gp path] (1.504,4.808)--(1.684,4.808);
\draw[gp path] (13.447,4.808)--(13.267,4.808);
\node[gp node right] at (1.320,4.808) {$500$};
\draw[gp path] (1.504,5.573)--(1.684,5.573);
\draw[gp path] (13.447,5.573)--(13.267,5.573);
\node[gp node right] at (1.320,5.573) {$600$};
\draw[gp path] (1.504,6.337)--(1.684,6.337);
\draw[gp path] (13.447,6.337)--(13.267,6.337);
\node[gp node right] at (1.320,6.337) {$700$};
\draw[gp path] (1.504,7.102)--(1.684,7.102);
\draw[gp path] (13.447,7.102)--(13.267,7.102);
\node[gp node right] at (1.320,7.102) {$800$};
\draw[gp path] (1.504,7.866)--(1.684,7.866);
\draw[gp path] (13.447,7.866)--(13.267,7.866);
\node[gp node right] at (1.320,7.866) {$900$};
\draw[gp path] (1.504,8.631)--(1.684,8.631);
\draw[gp path] (13.447,8.631)--(13.267,8.631);
\node[gp node right] at (1.320,8.631) {$1000$};
\draw[gp path] (1.504,0.985)--(1.504,1.165);
\draw[gp path] (1.504,8.631)--(1.504,8.451);
\node[gp node center] at (1.504,0.677) {$0$};
\draw[gp path] (2.997,0.985)--(2.997,1.165);
\draw[gp path] (2.997,8.631)--(2.997,8.451);
\node[gp node center] at (2.997,0.677) {$0.05$};
\draw[gp path] (4.490,0.985)--(4.490,1.165);
\draw[gp path] (4.490,8.631)--(4.490,8.451);
\node[gp node center] at (4.490,0.677) {$0.1$};
\draw[gp path] (5.983,0.985)--(5.983,1.165);
\draw[gp path] (5.983,8.631)--(5.983,8.451);
\node[gp node center] at (5.983,0.677) {$0.15$};
\draw[gp path] (7.476,0.985)--(7.476,1.165);
\draw[gp path] (7.476,8.631)--(7.476,8.451);
\node[gp node center] at (7.476,0.677) {$0.2$};
\draw[gp path] (8.968,0.985)--(8.968,1.165);
\draw[gp path] (8.968,8.631)--(8.968,8.451);
\node[gp node center] at (8.968,0.677) {$0.25$};
\draw[gp path] (10.461,0.985)--(10.461,1.165);
\draw[gp path] (10.461,8.631)--(10.461,8.451);
\node[gp node center] at (10.461,0.677) {$0.3$};
\draw[gp path] (11.954,0.985)--(11.954,1.165);
\draw[gp path] (11.954,8.631)--(11.954,8.451);
\node[gp node center] at (11.954,0.677) {$0.35$};
\draw[gp path] (13.447,0.985)--(13.447,1.165);
\draw[gp path] (13.447,8.631)--(13.447,8.451);
\node[gp node center] at (13.447,0.677) {$0.4$};
\draw[gp path] (1.504,8.631)--(1.504,0.985)--(13.447,0.985)--(13.447,8.631)--cycle;
\node[gp node center,rotate=-270] at (0.246,4.808) {$M$ (MeV)};
\node[gp node center] at (7.475,0.215) {$\rho$ ($\rm{fm}^{-3}$)};
\gpcolor{rgb color={0.580,0.000,0.827}}
\draw[gp path] (1.813,8.001)--(1.823,7.995)--(1.833,7.989)--(1.843,7.984)--(1.853,7.978)%
  --(1.864,7.973)--(1.874,7.967)--(1.884,7.961)--(1.894,7.956)--(1.904,7.950)--(1.914,7.945)%
  --(1.925,7.939)--(1.935,7.933)--(1.945,7.928)--(1.955,7.925)--(1.965,7.919)--(1.975,7.914)%
  --(1.985,7.908)--(1.996,7.903)--(2.006,7.897)--(2.016,7.891)--(2.026,7.886)--(2.036,7.880)%
  --(2.046,7.875)--(2.057,7.869)--(2.067,7.863)--(2.077,7.861)--(2.087,7.855)--(2.097,7.849)%
  --(2.107,7.844)--(2.118,7.838)--(2.128,7.833)--(2.138,7.827)--(2.148,7.821)--(2.158,7.816)%
  --(2.168,7.810)--(2.179,7.805)--(2.189,7.802)--(2.199,7.796)--(2.209,7.791)--(2.219,7.785)%
  --(2.229,7.779)--(2.240,7.774)--(2.250,7.768)--(2.260,7.763)--(2.270,7.757)--(2.280,7.751)%
  --(2.290,7.746)--(2.300,7.743)--(2.311,7.737)--(2.321,7.732)--(2.331,7.726)--(2.341,7.721)%
  --(2.351,7.715)--(2.361,7.709)--(2.372,7.704)--(2.382,7.698)--(2.392,7.693)--(2.402,7.690)%
  --(2.412,7.684)--(2.422,7.679)--(2.433,7.673)--(2.443,7.667)--(2.453,7.662)--(2.463,7.656)%
  --(2.473,7.651)--(2.483,7.645)--(2.494,7.642)--(2.504,7.637)--(2.514,7.631)--(2.524,7.625)%
  --(2.534,7.620)--(2.544,7.614)--(2.555,7.609)--(2.565,7.603)--(2.575,7.597)--(2.585,7.595)%
  --(2.595,7.589)--(2.605,7.583)--(2.616,7.578)--(2.626,7.572)--(2.636,7.567)--(2.646,7.561)%
  --(2.656,7.555)--(2.666,7.553)--(2.676,7.547)--(2.687,7.541)--(2.697,7.536)--(2.707,7.530)%
  --(2.717,7.525)--(2.727,7.519)--(2.737,7.513)--(2.748,7.508)--(2.758,7.505)--(2.768,7.499)%
  --(2.778,7.494)--(2.788,7.488)--(2.798,7.483)--(2.809,7.477)--(2.819,7.471)--(2.829,7.466)%
  --(2.839,7.463)--(2.849,7.457)--(2.859,7.452)--(2.870,7.446)--(2.880,7.441)--(2.890,7.435)%
  --(2.900,7.429)--(2.910,7.424)--(2.920,7.421)--(2.931,7.415)--(2.941,7.410)--(2.951,7.404)%
  --(2.961,7.399)--(2.971,7.393)--(2.981,7.387)--(2.991,7.385)--(3.002,7.379)--(3.012,7.373)%
  --(3.022,7.368)--(3.032,7.362)--(3.042,7.357)--(3.052,7.351)--(3.063,7.345)--(3.073,7.343)%
  --(3.083,7.337)--(3.093,7.331)--(3.103,7.326)--(3.113,7.320)--(3.124,7.315)--(3.134,7.309)%
  --(3.144,7.306)--(3.154,7.301)--(3.164,7.295)--(3.174,7.289)--(3.185,7.284)--(3.195,7.278)%
  --(3.205,7.273)--(3.215,7.270)--(3.225,7.264)--(3.235,7.259)--(3.246,7.253)--(3.256,7.247)%
  --(3.266,7.242)--(3.276,7.236)--(3.286,7.233)--(3.296,7.228)--(3.307,7.222)--(3.317,7.217)%
  --(3.327,7.211)--(3.337,7.205)--(3.347,7.200)--(3.357,7.197)--(3.367,7.191)--(3.378,7.186)%
  --(3.388,7.180)--(3.398,7.175)--(3.408,7.169)--(3.418,7.163)--(3.428,7.161)--(3.439,7.155)%
  --(3.449,7.149)--(3.459,7.144)--(3.469,7.138)--(3.479,7.133)--(3.489,7.127)--(3.500,7.124)%
  --(3.510,7.119)--(3.520,7.113)--(3.530,7.107)--(3.540,7.102)--(3.550,7.096)--(3.561,7.091)%
  --(3.571,7.088)--(3.581,7.082)--(3.591,7.077)--(3.601,7.071)--(3.611,7.065)--(3.622,7.060)%
  --(3.632,7.057)--(3.642,7.051)--(3.652,7.046)--(3.662,7.040)--(3.672,7.035)--(3.682,7.029)%
  --(3.693,7.023)--(3.703,7.021)--(3.713,7.015)--(3.723,7.009)--(3.733,7.004)--(3.743,6.998)%
  --(3.754,6.993)--(3.764,6.990)--(3.774,6.984)--(3.784,6.979)--(3.794,6.973)--(3.804,6.967)%
  --(3.815,6.962)--(3.825,6.956)--(3.835,6.953)--(3.845,6.948)--(3.855,6.942)--(3.865,6.937)%
  --(3.876,6.931)--(3.886,6.925)--(3.896,6.923)--(3.906,6.917)--(3.916,6.911)--(3.926,6.906)%
  --(3.937,6.900)--(3.947,6.895)--(3.957,6.892)--(3.967,6.886)--(3.977,6.881)--(3.987,6.875)%
  --(3.998,6.869)--(4.008,6.864)--(4.018,6.858)--(4.028,6.855)--(4.038,6.850)--(4.048,6.844)%
  --(4.058,6.839)--(4.069,6.833)--(4.079,6.827)--(4.089,6.825)--(4.099,6.819)--(4.109,6.813)%
  --(4.119,6.808)--(4.130,6.802)--(4.140,6.797)--(4.150,6.794)--(4.160,6.788)--(4.170,6.783)%
  --(4.180,6.777)--(4.191,6.771)--(4.201,6.766)--(4.211,6.763)--(4.221,6.757)--(4.231,6.752)%
  --(4.241,6.746)--(4.252,6.741)--(4.262,6.735)--(4.272,6.732)--(4.282,6.727)--(4.292,6.721)%
  --(4.302,6.715)--(4.313,6.710)--(4.323,6.704)--(4.333,6.701)--(4.343,6.696)--(4.353,6.690)%
  --(4.363,6.685)--(4.373,6.679)--(4.384,6.673)--(4.394,6.671)--(4.404,6.665)--(4.414,6.659)%
  --(4.424,6.654)--(4.434,6.648)--(4.445,6.643)--(4.455,6.637)--(4.465,6.634)--(4.475,6.629)%
  --(4.485,6.623)--(4.495,6.617)--(4.506,6.612)--(4.516,6.606)--(4.526,6.603)--(4.536,6.598)%
  --(4.546,6.592)--(4.556,6.587)--(4.567,6.581)--(4.577,6.575)--(4.587,6.573)--(4.597,6.567)%
  --(4.607,6.561)--(4.617,6.556)--(4.628,6.550)--(4.638,6.545)--(4.648,6.542)--(4.658,6.536)%
  --(4.668,6.531)--(4.678,6.525)--(4.689,6.519)--(4.699,6.514)--(4.709,6.511)--(4.719,6.505)%
  --(4.729,6.500)--(4.739,6.494)--(4.749,6.488)--(4.760,6.483)--(4.770,6.480)--(4.780,6.474)%
  --(4.790,6.469)--(4.800,6.463)--(4.810,6.458)--(4.821,6.452)--(4.831,6.449)--(4.841,6.444)%
  --(4.851,6.438)--(4.861,6.432)--(4.871,6.427)--(4.882,6.421)--(4.892,6.418)--(4.902,6.413)%
  --(4.912,6.407)--(4.922,6.402)--(4.932,6.396)--(4.943,6.390)--(4.953,6.388)--(4.963,6.382)%
  --(4.973,6.376)--(4.983,6.371)--(4.993,6.365)--(5.004,6.360)--(5.014,6.357)--(5.024,6.351)%
  --(5.034,6.346)--(5.044,6.340)--(5.054,6.334)--(5.064,6.329)--(5.075,6.323)--(5.085,6.320)%
  --(5.095,6.315)--(5.105,6.309)--(5.115,6.304)--(5.125,6.298)--(5.136,6.292)--(5.146,6.290)%
  --(5.156,6.284)--(5.166,6.278)--(5.176,6.273)--(5.186,6.267)--(5.197,6.262)--(5.207,6.259)%
  --(5.217,6.253)--(5.227,6.248)--(5.237,6.242)--(5.247,6.236)--(5.258,6.231)--(5.268,6.228)%
  --(5.278,6.222)--(5.288,6.217)--(5.298,6.211)--(5.308,6.206)--(5.319,6.200)--(5.329,6.194)%
  --(5.339,6.192)--(5.349,6.186)--(5.359,6.180)--(5.369,6.175)--(5.379,6.169)--(5.390,6.164)%
  --(5.400,6.161)--(5.410,6.155)--(5.420,6.150)--(5.430,6.144)--(5.440,6.138)--(5.451,6.133)%
  --(5.461,6.127)--(5.471,6.124)--(5.481,6.119)--(5.491,6.113)--(5.501,6.108)--(5.512,6.102)%
  --(5.522,6.096)--(5.532,6.094)--(5.542,6.088)--(5.552,6.082)--(5.562,6.077)--(5.573,6.071)%
  --(5.583,6.066)--(5.593,6.060)--(5.603,6.057)--(5.613,6.052)--(5.623,6.046)--(5.634,6.040)%
  --(5.644,6.035)--(5.654,6.029)--(5.664,6.026)--(5.674,6.021)--(5.684,6.015)--(5.695,6.010)%
  --(5.705,6.004)--(5.715,5.998)--(5.725,5.993)--(5.735,5.990)--(5.745,5.984)--(5.755,5.979)%
  --(5.766,5.973)--(5.776,5.968)--(5.786,5.962)--(5.796,5.956)--(5.806,5.954)--(5.816,5.948)%
  --(5.827,5.942)--(5.837,5.937)--(5.847,5.931)--(5.857,5.926)--(5.867,5.920)--(5.877,5.917)%
  --(5.888,5.912)--(5.898,5.906)--(5.908,5.900)--(5.918,5.895)--(5.928,5.889)--(5.938,5.884)%
  --(5.949,5.881)--(5.959,5.875)--(5.969,5.870)--(5.979,5.864)--(5.989,5.858)--(5.999,5.853)%
  --(6.010,5.847)--(6.020,5.844)--(6.030,5.839)--(6.040,5.833)--(6.050,5.828)--(6.060,5.822)%
  --(6.070,5.817)--(6.081,5.812)--(6.091,5.806)--(6.101,5.802)--(6.111,5.796)--(6.121,5.791)%
  --(6.131,5.785)--(6.142,5.781)--(6.152,5.775)--(6.162,5.770)--(6.172,5.764)--(6.182,5.760)%
  --(6.192,5.754)--(6.203,5.749)--(6.213,5.744)--(6.223,5.739)--(6.233,5.733)--(6.243,5.728)%
  --(6.253,5.723)--(6.264,5.718)--(6.274,5.712)--(6.284,5.707)--(6.294,5.701)--(6.304,5.697)%
  --(6.314,5.691)--(6.325,5.686)--(6.335,5.680)--(6.345,5.676)--(6.355,5.670)--(6.365,5.665)%
  --(6.375,5.659)--(6.386,5.655)--(6.396,5.649)--(6.406,5.644)--(6.416,5.638)--(6.426,5.634)%
  --(6.436,5.628)--(6.446,5.623)--(6.457,5.617)--(6.467,5.613)--(6.477,5.607)--(6.487,5.602)%
  --(6.497,5.596)--(6.507,5.590)--(6.518,5.586)--(6.528,5.581)--(6.538,5.575)--(6.548,5.569)%
  --(6.558,5.564)--(6.568,5.560)--(6.579,5.554)--(6.589,5.548)--(6.599,5.543)--(6.609,5.539)%
  --(6.619,5.533)--(6.629,5.527)--(6.640,5.522)--(6.650,5.516)--(6.660,5.512)--(6.670,5.506)%
  --(6.680,5.501)--(6.690,5.495)--(6.701,5.490)--(6.711,5.485)--(6.721,5.480)--(6.731,5.474)%
  --(6.741,5.469)--(6.751,5.463)--(6.761,5.457)--(6.772,5.453)--(6.782,5.448)--(6.792,5.442)%
  --(6.802,5.436)--(6.812,5.431)--(6.822,5.427)--(6.833,5.421)--(6.843,5.415)--(6.853,5.410)%
  --(6.863,5.404)--(6.873,5.399)--(6.883,5.394)--(6.894,5.389)--(6.904,5.383)--(6.914,5.378)%
  --(6.924,5.372)--(6.934,5.366)--(6.944,5.362)--(6.955,5.357)--(6.965,5.351)--(6.975,5.345)%
  --(6.985,5.340)--(6.995,5.334)--(7.005,5.330)--(7.016,5.324)--(7.026,5.319)--(7.036,5.313)%
  --(7.046,5.308)--(7.056,5.302)--(7.066,5.296)--(7.077,5.292)--(7.087,5.287)--(7.097,5.281)%
  --(7.107,5.275)--(7.117,5.270)--(7.127,5.264)--(7.137,5.259)--(7.148,5.253)--(7.158,5.249)%
  --(7.168,5.243)--(7.178,5.238)--(7.188,5.232)--(7.198,5.226)--(7.209,5.221)--(7.219,5.215)%
  --(7.229,5.210)--(7.239,5.204)--(7.249,5.200)--(7.259,5.194)--(7.270,5.189)--(7.280,5.183)%
  --(7.290,5.177)--(7.300,5.172)--(7.310,5.166)--(7.320,5.161)--(7.331,5.155)--(7.341,5.149)%
  --(7.351,5.145)--(7.361,5.140)--(7.371,5.134)--(7.381,5.128)--(7.392,5.123)--(7.402,5.117)%
  --(7.412,5.112)--(7.422,5.106)--(7.432,5.100)--(7.442,5.095)--(7.452,5.089)--(7.463,5.084)%
  --(7.473,5.078)--(7.483,5.072)--(7.493,5.068)--(7.503,5.063)--(7.513,5.057)--(7.524,5.051)%
  --(7.534,5.046)--(7.544,5.040)--(7.554,5.035)--(7.564,5.029)--(7.574,5.023)--(7.585,5.018)%
  --(7.595,5.012)--(7.605,5.007)--(7.615,5.001)--(7.625,4.995)--(7.635,4.990)--(7.646,4.984)%
  --(7.656,4.979)--(7.666,4.973)--(7.676,4.967)--(7.686,4.962)--(7.696,4.956)--(7.707,4.951)%
  --(7.717,4.945)--(7.727,4.939)--(7.737,4.934)--(7.747,4.928)--(7.757,4.923)--(7.767,4.917)%
  --(7.778,4.911)--(7.788,4.906)--(7.798,4.900)--(7.808,4.895)--(7.818,4.889)--(7.828,4.883)%
  --(7.839,4.878)--(7.849,4.872)--(7.859,4.867)--(7.869,4.861)--(7.879,4.855)--(7.889,4.850)%
  --(7.900,4.844)--(7.910,4.839)--(7.920,4.833)--(7.930,4.827)--(7.940,4.822)--(7.950,4.816)%
  --(7.961,4.811)--(7.971,4.804)--(7.981,4.798)--(7.991,4.792)--(8.001,4.787)--(8.011,4.781)%
  --(8.022,4.776)--(8.032,4.770)--(8.042,4.764)--(8.052,4.759)--(8.062,4.753)--(8.072,4.748)%
  --(8.083,4.742)--(8.093,4.736)--(8.103,4.729)--(8.113,4.724)--(8.123,4.718)--(8.133,4.713)%
  --(8.143,4.707)--(8.154,4.701)--(8.164,4.696)--(8.174,4.690)--(8.184,4.683)--(8.194,4.678)%
  --(8.204,4.672)--(8.215,4.666)--(8.225,4.661)--(8.235,4.655)--(8.245,4.650)--(8.255,4.644)%
  --(8.265,4.637)--(8.276,4.631)--(8.286,4.626)--(8.296,4.620)--(8.306,4.615)--(8.316,4.609)%
  --(8.326,4.602)--(8.337,4.596)--(8.347,4.591)--(8.357,4.585)--(8.367,4.580)--(8.377,4.573)%
  --(8.387,4.567)--(8.398,4.561)--(8.408,4.556)--(8.418,4.550)--(8.428,4.543)--(8.438,4.538)%
  --(8.448,4.532)--(8.458,4.526)--(8.469,4.521)--(8.479,4.514)--(8.489,4.508)--(8.499,4.503)%
  --(8.509,4.497)--(8.519,4.490)--(8.530,4.484)--(8.540,4.479)--(8.550,4.473)--(8.560,4.466)%
  --(8.570,4.461)--(8.580,4.455)--(8.591,4.449)--(8.601,4.442)--(8.611,4.437)--(8.621,4.431)%
  --(8.631,4.426)--(8.641,4.419)--(8.652,4.413)--(8.662,4.407)--(8.672,4.400)--(8.682,4.395)%
  --(8.692,4.389)--(8.702,4.382)--(8.713,4.377)--(8.723,4.371)--(8.733,4.364)--(8.743,4.358)%
  --(8.753,4.353)--(8.763,4.346)--(8.774,4.340)--(8.784,4.335)--(8.794,4.328)--(8.804,4.322)%
  --(8.814,4.316)--(8.824,4.309)--(8.834,4.304)--(8.845,4.298)--(8.855,4.291)--(8.865,4.286)%
  --(8.875,4.279)--(8.885,4.273)--(8.895,4.267)--(8.906,4.260)--(8.916,4.255)--(8.926,4.248)%
  --(8.936,4.242)--(8.946,4.237)--(8.956,4.230)--(8.967,4.224)--(8.977,4.217)--(8.987,4.211)%
  --(8.997,4.204)--(9.007,4.199)--(9.017,4.193)--(9.028,4.186)--(9.038,4.181)--(9.048,4.174)%
  --(9.058,4.168)--(9.068,4.161)--(9.078,4.155)--(9.089,4.148)--(9.099,4.143)--(9.109,4.136)%
  --(9.119,4.130)--(9.129,4.123)--(9.139,4.118)--(9.149,4.111)--(9.160,4.105)--(9.170,4.098)%
  --(9.180,4.091)--(9.190,4.085)--(9.200,4.078)--(9.210,4.073)--(9.221,4.066)--(9.231,4.060)%
  --(9.241,4.053)--(9.251,4.048)--(9.261,4.041)--(9.271,4.034)--(9.282,4.028)--(9.292,4.021)%
  --(9.302,4.015)--(9.312,4.008)--(9.322,4.001)--(9.332,3.996)--(9.343,3.989)--(9.353,3.982)%
  --(9.363,3.976)--(9.373,3.969)--(9.383,3.962)--(9.393,3.957)--(9.404,3.950)--(9.414,3.943)%
  --(9.424,3.937)--(9.434,3.930)--(9.444,3.923)--(9.454,3.917)--(9.465,3.910)--(9.475,3.903)%
  --(9.485,3.896)--(9.495,3.891)--(9.505,3.884)--(9.515,3.877)--(9.525,3.870)--(9.536,3.864)%
  --(9.546,3.857)--(9.556,3.850)--(9.566,3.843)--(9.576,3.838)--(9.586,3.831)--(9.597,3.824)%
  --(9.607,3.817)--(9.617,3.810)--(9.627,3.804)--(9.637,3.797)--(9.647,3.790)--(9.658,3.783)%
  --(9.668,3.776)--(9.678,3.769)--(9.688,3.762)--(9.698,3.756)--(9.708,3.749)--(9.719,3.742)%
  --(9.729,3.735)--(9.739,3.728)--(9.749,3.721)--(9.759,3.714)--(9.769,3.707)--(9.780,3.700)%
  --(9.790,3.693)--(9.800,3.686)--(9.810,3.679)--(9.820,3.672)--(9.830,3.665)--(9.840,3.658)%
  --(9.851,3.651)--(9.861,3.644)--(9.871,3.637)--(9.881,3.630)--(9.891,3.623)--(9.901,3.616)%
  --(9.912,3.609)--(9.922,3.602)--(9.932,3.595)--(9.942,3.588)--(9.952,3.581)--(9.962,3.574)%
  --(9.973,3.566)--(9.983,3.559)--(9.993,3.552)--(10.003,3.545)--(10.013,3.538)--(10.023,3.531)%
  --(10.034,3.524)--(10.044,3.516)--(10.054,3.509)--(10.064,3.502)--(10.074,3.495)--(10.084,3.486)%
  --(10.095,3.479)--(10.105,3.472)--(10.115,3.465)--(10.125,3.457)--(10.135,3.450)--(10.145,3.443)%
  --(10.156,3.436)--(10.166,3.427)--(10.176,3.420)--(10.186,3.413)--(10.196,3.405)--(10.206,3.398)%
  --(10.216,3.390)--(10.227,3.383)--(10.237,3.376)--(10.247,3.367)--(10.257,3.360)--(10.267,3.352)%
  --(10.277,3.345)--(10.288,3.336)--(10.298,3.329)--(10.308,3.321)--(10.318,3.314)--(10.328,3.306)%
  --(10.338,3.299)--(10.349,3.290)--(10.359,3.283)--(10.369,3.275)--(10.379,3.268)--(10.389,3.259)%
  --(10.399,3.251)--(10.410,3.244)--(10.420,3.236)--(10.430,3.229)--(10.440,3.220)--(10.450,3.212)%
  --(10.460,3.203)--(10.471,3.196)--(10.481,3.188)--(10.491,3.180)--(10.501,3.171)--(10.511,3.164)%
  --(10.521,3.156)--(10.531,3.147)--(10.542,3.139)--(10.552,3.131)--(10.562,3.124)--(10.572,3.115)%
  --(10.582,3.107)--(10.592,3.098)--(10.603,3.090)--(10.613,3.082)--(10.623,3.073)--(10.633,3.065)%
  --(10.643,3.056)--(10.653,3.048)--(10.664,3.040)--(10.674,3.031)--(10.684,3.023)--(10.694,3.014)%
  --(10.704,3.006)--(10.714,2.996)--(10.725,2.988)--(10.735,2.980)--(10.745,2.971)--(10.755,2.962)%
  --(10.765,2.953)--(10.775,2.945)--(10.786,2.936)--(10.796,2.926)--(10.806,2.918)--(10.816,2.909)%
  --(10.826,2.900)--(10.836,2.891)--(10.846,2.882)--(10.857,2.873)--(10.867,2.864)--(10.877,2.855)%
  --(10.887,2.846)--(10.897,2.836)--(10.907,2.827)--(10.918,2.818)--(10.928,2.809)--(10.938,2.799)%
  --(10.948,2.790)--(10.958,2.781)--(10.968,2.771)--(10.979,2.762)--(10.989,2.752)--(10.999,2.742)%
  --(11.009,2.733)--(11.019,2.723)--(11.029,2.714)--(11.040,2.704)--(11.050,2.694)--(11.060,2.684)%
  --(11.070,2.674)--(11.080,2.665)--(11.090,2.654)--(11.101,2.644)--(11.111,2.635)--(11.121,2.624)%
  --(11.131,2.614)--(11.141,2.604)--(11.151,2.593)--(11.162,2.583)--(11.172,2.572)--(11.182,2.562)%
  --(11.192,2.551)--(11.202,2.541)--(11.212,2.530)--(11.222,2.520)--(11.233,2.509)--(11.243,2.498)%
  --(11.253,2.487)--(11.263,2.476)--(11.273,2.465)--(11.283,2.454)--(11.294,2.443)--(11.304,2.432)%
  --(11.314,2.420)--(11.324,2.408)--(11.334,2.397)--(11.344,2.385)--(11.355,2.374)--(11.365,2.362)%
  --(11.375,2.350)--(11.385,2.338)--(11.395,2.327)--(11.405,2.315)--(11.416,2.302)--(11.426,2.289)%
  --(11.436,2.278)--(11.446,2.265)--(11.456,2.252)--(11.466,2.240)--(11.477,2.226)--(11.487,2.214)%
  --(11.497,2.201)--(11.507,2.187)--(11.517,2.174)--(11.527,2.160)--(11.537,2.147)--(11.548,2.133)%
  --(11.558,2.119)--(11.568,2.105)--(11.578,2.091)--(11.588,2.076)--(11.598,2.061)--(11.609,2.047)%
  --(11.619,2.032)--(11.629,2.016)--(11.639,2.001)--(11.649,1.986)--(11.659,1.970)--(11.670,1.953)%
  --(11.680,1.937)--(11.690,1.921)--(11.700,1.904)--(11.710,1.886)--(11.720,1.869)--(11.731,1.851)%
  --(11.741,1.833)--(11.751,1.814)--(11.761,1.795)--(11.771,1.776)--(11.781,1.756)--(11.792,1.736)%
  --(11.802,1.715)--(11.812,1.693)--(11.822,1.671)--(11.832,1.648)--(11.842,1.624)--(11.853,1.600)%
  --(11.863,1.574)--(11.873,1.548)--(11.883,1.520)--(11.893,1.490)--(11.903,1.459)--(11.913,1.425)%
  --(11.924,1.389)--(11.934,1.350)--(11.944,1.305)--(11.954,1.253)--(11.964,1.189);
\gpcolor{color=gp lt color border}
\draw[gp path] (1.504,8.631)--(1.504,0.985)--(13.447,0.985)--(13.447,8.631)--cycle;
%% coordinates of the plot area
\gpdefrectangularnode{gp plot 1}{\pgfpoint{1.504cm}{0.985cm}}{\pgfpoint{13.447cm}{8.631cm}}
\end{tikzpicture}
%% gnuplot variables

	\caption{Gráfico mostrando a massa em função da densidade bariônica para fração de prótons 1/2. Note que $M$ diminui até zero em $\rho \approx 0.35$. Nesse ponto ocorre a restauração da simetria quiral.}
	\label{Fig:mass_graph}
\end{figure*}

\begin{figure*}
	\begin{tikzpicture}[gnuplot]
%% generated with GNUPLOT 5.0p2 (Lua 5.2; terminal rev. 99, script rev. 100)
%% Thu Feb 25 17:17:56 2016
\path (0.000,0.000) rectangle (14.000,9.000);
\gpcolor{color=gp lt color border}
\gpsetlinetype{gp lt border}
\gpsetdashtype{gp dt solid}
\gpsetlinewidth{1.00}
\draw[gp path] (1.688,0.985)--(1.868,0.985);
\draw[gp path] (13.447,0.985)--(13.267,0.985);
\node[gp node right] at (1.504,0.985) {$-0.5$};
\draw[gp path] (1.688,1.750)--(1.868,1.750);
\draw[gp path] (13.447,1.750)--(13.267,1.750);
\node[gp node right] at (1.504,1.750) {$-0.45$};
\draw[gp path] (1.688,2.514)--(1.868,2.514);
\draw[gp path] (13.447,2.514)--(13.267,2.514);
\node[gp node right] at (1.504,2.514) {$-0.4$};
\draw[gp path] (1.688,3.279)--(1.868,3.279);
\draw[gp path] (13.447,3.279)--(13.267,3.279);
\node[gp node right] at (1.504,3.279) {$-0.35$};
\draw[gp path] (1.688,4.043)--(1.868,4.043);
\draw[gp path] (13.447,4.043)--(13.267,4.043);
\node[gp node right] at (1.504,4.043) {$-0.3$};
\draw[gp path] (1.688,4.808)--(1.868,4.808);
\draw[gp path] (13.447,4.808)--(13.267,4.808);
\node[gp node right] at (1.504,4.808) {$-0.25$};
\draw[gp path] (1.688,5.573)--(1.868,5.573);
\draw[gp path] (13.447,5.573)--(13.267,5.573);
\node[gp node right] at (1.504,5.573) {$-0.2$};
\draw[gp path] (1.688,6.337)--(1.868,6.337);
\draw[gp path] (13.447,6.337)--(13.267,6.337);
\node[gp node right] at (1.504,6.337) {$-0.15$};
\draw[gp path] (1.688,7.102)--(1.868,7.102);
\draw[gp path] (13.447,7.102)--(13.267,7.102);
\node[gp node right] at (1.504,7.102) {$-0.1$};
\draw[gp path] (1.688,7.866)--(1.868,7.866);
\draw[gp path] (13.447,7.866)--(13.267,7.866);
\node[gp node right] at (1.504,7.866) {$-0.05$};
\draw[gp path] (1.688,8.631)--(1.868,8.631);
\draw[gp path] (13.447,8.631)--(13.267,8.631);
\node[gp node right] at (1.504,8.631) {$0$};
\draw[gp path] (1.688,0.985)--(1.688,1.165);
\draw[gp path] (1.688,8.631)--(1.688,8.451);
\node[gp node center] at (1.688,0.677) {$0$};
\draw[gp path] (3.158,0.985)--(3.158,1.165);
\draw[gp path] (3.158,8.631)--(3.158,8.451);
\node[gp node center] at (3.158,0.677) {$0.05$};
\draw[gp path] (4.628,0.985)--(4.628,1.165);
\draw[gp path] (4.628,8.631)--(4.628,8.451);
\node[gp node center] at (4.628,0.677) {$0.1$};
\draw[gp path] (6.098,0.985)--(6.098,1.165);
\draw[gp path] (6.098,8.631)--(6.098,8.451);
\node[gp node center] at (6.098,0.677) {$0.15$};
\draw[gp path] (7.568,0.985)--(7.568,1.165);
\draw[gp path] (7.568,8.631)--(7.568,8.451);
\node[gp node center] at (7.568,0.677) {$0.2$};
\draw[gp path] (9.037,0.985)--(9.037,1.165);
\draw[gp path] (9.037,8.631)--(9.037,8.451);
\node[gp node center] at (9.037,0.677) {$0.25$};
\draw[gp path] (10.507,0.985)--(10.507,1.165);
\draw[gp path] (10.507,8.631)--(10.507,8.451);
\node[gp node center] at (10.507,0.677) {$0.3$};
\draw[gp path] (11.977,0.985)--(11.977,1.165);
\draw[gp path] (11.977,8.631)--(11.977,8.451);
\node[gp node center] at (11.977,0.677) {$0.35$};
\draw[gp path] (13.447,0.985)--(13.447,1.165);
\draw[gp path] (13.447,8.631)--(13.447,8.451);
\node[gp node center] at (13.447,0.677) {$0.4$};
\draw[gp path] (1.688,8.631)--(1.688,0.985)--(13.447,0.985)--(13.447,8.631)--cycle;
\node[gp node center,rotate=-270] at (0.246,4.808) {$\rho_s$ ($\rm{fm}^{-3}$)};
\node[gp node center] at (7.567,0.215) {$\rho$ ($\rm{fm}^{-3}$)};
\gpcolor{rgb color={0.580,0.000,0.827}}
\draw[gp path] (1.992,1.310)--(2.002,1.316)--(2.012,1.322)--(2.022,1.328)--(2.032,1.333)%
  --(2.042,1.339)--(2.052,1.345)--(2.062,1.350)--(2.072,1.356)--(2.082,1.362)--(2.092,1.368)%
  --(2.102,1.373)--(2.112,1.379)--(2.122,1.385)--(2.132,1.390)--(2.142,1.396)--(2.152,1.402)%
  --(2.162,1.408)--(2.172,1.413)--(2.182,1.419)--(2.192,1.425)--(2.202,1.431)--(2.212,1.436)%
  --(2.222,1.442)--(2.232,1.448)--(2.242,1.454)--(2.252,1.459)--(2.262,1.465)--(2.272,1.470)%
  --(2.282,1.476)--(2.292,1.482)--(2.302,1.488)--(2.312,1.493)--(2.322,1.499)--(2.332,1.505)%
  --(2.342,1.511)--(2.352,1.516)--(2.362,1.522)--(2.372,1.528)--(2.382,1.533)--(2.392,1.539)%
  --(2.402,1.545)--(2.412,1.551)--(2.422,1.556)--(2.432,1.562)--(2.442,1.568)--(2.452,1.574)%
  --(2.462,1.579)--(2.472,1.585)--(2.482,1.591)--(2.492,1.596)--(2.502,1.602)--(2.512,1.608)%
  --(2.522,1.614)--(2.532,1.619)--(2.542,1.625)--(2.552,1.631)--(2.562,1.637)--(2.572,1.642)%
  --(2.582,1.648)--(2.592,1.654)--(2.602,1.659)--(2.612,1.665)--(2.622,1.671)--(2.632,1.677)%
  --(2.642,1.682)--(2.652,1.688)--(2.662,1.694)--(2.672,1.699)--(2.682,1.705)--(2.692,1.711)%
  --(2.702,1.717)--(2.712,1.722)--(2.722,1.728)--(2.732,1.734)--(2.742,1.740)--(2.752,1.745)%
  --(2.762,1.751)--(2.772,1.757)--(2.782,1.762)--(2.792,1.768)--(2.802,1.774)--(2.812,1.780)%
  --(2.822,1.786)--(2.832,1.791)--(2.842,1.797)--(2.852,1.803)--(2.862,1.808)--(2.872,1.814)%
  --(2.882,1.820)--(2.892,1.826)--(2.902,1.831)--(2.912,1.837)--(2.922,1.843)--(2.932,1.848)%
  --(2.942,1.854)--(2.952,1.860)--(2.962,1.866)--(2.972,1.872)--(2.982,1.877)--(2.992,1.883)%
  --(3.003,1.889)--(3.013,1.894)--(3.023,1.900)--(3.033,1.906)--(3.043,1.912)--(3.053,1.917)%
  --(3.063,1.923)--(3.073,1.929)--(3.083,1.935)--(3.093,1.940)--(3.103,1.946)--(3.113,1.952)%
  --(3.123,1.958)--(3.133,1.963)--(3.143,1.969)--(3.153,1.975)--(3.163,1.980)--(3.173,1.986)%
  --(3.183,1.992)--(3.193,1.998)--(3.203,2.004)--(3.213,2.009)--(3.223,2.015)--(3.233,2.021)%
  --(3.243,2.026)--(3.253,2.032)--(3.263,2.038)--(3.273,2.044)--(3.283,2.050)--(3.293,2.055)%
  --(3.303,2.061)--(3.313,2.067)--(3.323,2.073)--(3.333,2.078)--(3.343,2.084)--(3.353,2.090)%
  --(3.363,2.096)--(3.373,2.101)--(3.383,2.107)--(3.393,2.113)--(3.403,2.119)--(3.413,2.124)%
  --(3.423,2.130)--(3.433,2.136)--(3.443,2.141)--(3.453,2.147)--(3.463,2.153)--(3.473,2.159)%
  --(3.483,2.165)--(3.493,2.171)--(3.503,2.176)--(3.513,2.182)--(3.523,2.188)--(3.533,2.193)%
  --(3.543,2.199)--(3.553,2.205)--(3.563,2.211)--(3.573,2.217)--(3.583,2.222)--(3.593,2.228)%
  --(3.603,2.234)--(3.613,2.240)--(3.623,2.245)--(3.633,2.251)--(3.643,2.257)--(3.653,2.263)%
  --(3.663,2.268)--(3.673,2.274)--(3.683,2.280)--(3.693,2.286)--(3.703,2.292)--(3.713,2.297)%
  --(3.723,2.303)--(3.733,2.309)--(3.743,2.315)--(3.753,2.320)--(3.763,2.326)--(3.773,2.332)%
  --(3.783,2.338)--(3.793,2.343)--(3.803,2.349)--(3.813,2.355)--(3.823,2.361)--(3.833,2.367)%
  --(3.843,2.373)--(3.853,2.378)--(3.863,2.384)--(3.873,2.390)--(3.883,2.395)--(3.893,2.401)%
  --(3.903,2.407)--(3.913,2.413)--(3.923,2.418)--(3.933,2.424)--(3.943,2.430)--(3.953,2.436)%
  --(3.963,2.442)--(3.973,2.448)--(3.983,2.453)--(3.993,2.459)--(4.003,2.465)--(4.013,2.471)%
  --(4.023,2.477)--(4.033,2.482)--(4.043,2.488)--(4.053,2.494)--(4.063,2.500)--(4.073,2.505)%
  --(4.083,2.511)--(4.093,2.517)--(4.103,2.523)--(4.113,2.528)--(4.123,2.534)--(4.133,2.540)%
  --(4.143,2.546)--(4.153,2.552)--(4.163,2.558)--(4.173,2.563)--(4.183,2.569)--(4.193,2.575)%
  --(4.203,2.581)--(4.213,2.587)--(4.223,2.592)--(4.233,2.598)--(4.243,2.604)--(4.253,2.610)%
  --(4.263,2.615)--(4.273,2.621)--(4.283,2.627)--(4.293,2.633)--(4.303,2.639)--(4.313,2.644)%
  --(4.323,2.650)--(4.333,2.656)--(4.343,2.662)--(4.353,2.667)--(4.363,2.673)--(4.373,2.679)%
  --(4.383,2.685)--(4.393,2.691)--(4.403,2.697)--(4.413,2.702)--(4.423,2.708)--(4.433,2.714)%
  --(4.443,2.720)--(4.453,2.726)--(4.463,2.732)--(4.473,2.737)--(4.483,2.743)--(4.493,2.749)%
  --(4.503,2.755)--(4.513,2.761)--(4.523,2.767)--(4.533,2.772)--(4.543,2.778)--(4.553,2.784)%
  --(4.563,2.790)--(4.573,2.796)--(4.583,2.801)--(4.593,2.807)--(4.603,2.813)--(4.613,2.819)%
  --(4.623,2.825)--(4.633,2.830)--(4.643,2.836)--(4.653,2.842)--(4.663,2.848)--(4.673,2.854)%
  --(4.683,2.860)--(4.693,2.865)--(4.703,2.871)--(4.713,2.877)--(4.723,2.883)--(4.733,2.889)%
  --(4.743,2.894)--(4.753,2.900)--(4.763,2.906)--(4.773,2.912)--(4.783,2.918)--(4.793,2.924)%
  --(4.803,2.929)--(4.813,2.935)--(4.823,2.941)--(4.833,2.947)--(4.843,2.953)--(4.853,2.959)%
  --(4.863,2.964)--(4.873,2.970)--(4.883,2.976)--(4.893,2.982)--(4.903,2.988)--(4.913,2.994)%
  --(4.923,2.999)--(4.933,3.005)--(4.944,3.011)--(4.954,3.017)--(4.964,3.023)--(4.974,3.029)%
  --(4.984,3.035)--(4.994,3.040)--(5.004,3.046)--(5.014,3.052)--(5.024,3.058)--(5.034,3.064)%
  --(5.044,3.070)--(5.054,3.076)--(5.064,3.081)--(5.074,3.087)--(5.084,3.093)--(5.094,3.099)%
  --(5.104,3.105)--(5.114,3.111)--(5.124,3.117)--(5.134,3.123)--(5.144,3.128)--(5.154,3.134)%
  --(5.164,3.140)--(5.174,3.146)--(5.184,3.152)--(5.194,3.158)--(5.204,3.164)--(5.214,3.169)%
  --(5.224,3.175)--(5.234,3.181)--(5.244,3.187)--(5.254,3.193)--(5.264,3.199)--(5.274,3.204)%
  --(5.284,3.210)--(5.294,3.216)--(5.304,3.222)--(5.314,3.228)--(5.324,3.234)--(5.334,3.239)%
  --(5.344,3.245)--(5.354,3.251)--(5.364,3.257)--(5.374,3.263)--(5.384,3.269)--(5.394,3.275)%
  --(5.404,3.281)--(5.414,3.287)--(5.424,3.293)--(5.434,3.299)--(5.444,3.305)--(5.454,3.310)%
  --(5.464,3.316)--(5.474,3.322)--(5.484,3.328)--(5.494,3.334)--(5.504,3.340)--(5.514,3.346)%
  --(5.524,3.351)--(5.534,3.357)--(5.544,3.363)--(5.554,3.369)--(5.564,3.375)--(5.574,3.381)%
  --(5.584,3.387)--(5.594,3.393)--(5.604,3.398)--(5.614,3.404)--(5.624,3.410)--(5.634,3.416)%
  --(5.644,3.422)--(5.654,3.428)--(5.664,3.434)--(5.674,3.440)--(5.684,3.446)--(5.694,3.452)%
  --(5.704,3.458)--(5.714,3.464)--(5.724,3.469)--(5.734,3.475)--(5.744,3.481)--(5.754,3.487)%
  --(5.764,3.493)--(5.774,3.499)--(5.784,3.505)--(5.794,3.511)--(5.804,3.517)--(5.814,3.523)%
  --(5.824,3.529)--(5.834,3.535)--(5.844,3.541)--(5.854,3.546)--(5.864,3.552)--(5.874,3.558)%
  --(5.884,3.564)--(5.894,3.570)--(5.904,3.576)--(5.914,3.582)--(5.924,3.588)--(5.934,3.594)%
  --(5.944,3.600)--(5.954,3.606)--(5.964,3.612)--(5.974,3.618)--(5.984,3.624)--(5.994,3.629)%
  --(6.004,3.635)--(6.014,3.641)--(6.024,3.647)--(6.034,3.653)--(6.044,3.659)--(6.054,3.665)%
  --(6.064,3.671)--(6.074,3.677)--(6.084,3.683)--(6.094,3.689)--(6.104,3.695)--(6.114,3.701)%
  --(6.124,3.707)--(6.134,3.712)--(6.144,3.718)--(6.154,3.724)--(6.164,3.730)--(6.174,3.736)%
  --(6.184,3.742)--(6.194,3.748)--(6.204,3.754)--(6.214,3.760)--(6.224,3.766)--(6.234,3.772)%
  --(6.244,3.778)--(6.254,3.784)--(6.264,3.790)--(6.274,3.796)--(6.284,3.802)--(6.294,3.808)%
  --(6.304,3.814)--(6.314,3.820)--(6.324,3.826)--(6.334,3.832)--(6.344,3.838)--(6.354,3.844)%
  --(6.364,3.850)--(6.374,3.856)--(6.384,3.862)--(6.394,3.868)--(6.404,3.874)--(6.414,3.880)%
  --(6.424,3.886)--(6.434,3.892)--(6.444,3.898)--(6.454,3.903)--(6.464,3.910)--(6.474,3.916)%
  --(6.484,3.922)--(6.494,3.927)--(6.504,3.934)--(6.514,3.940)--(6.524,3.946)--(6.534,3.951)%
  --(6.544,3.957)--(6.554,3.964)--(6.564,3.970)--(6.574,3.975)--(6.584,3.981)--(6.594,3.988)%
  --(6.604,3.994)--(6.614,4.000)--(6.624,4.005)--(6.634,4.012)--(6.644,4.018)--(6.654,4.024)%
  --(6.664,4.030)--(6.674,4.036)--(6.684,4.042)--(6.694,4.048)--(6.704,4.054)--(6.714,4.060)%
  --(6.724,4.066)--(6.734,4.072)--(6.744,4.078)--(6.754,4.084)--(6.764,4.090)--(6.774,4.096)%
  --(6.784,4.102)--(6.794,4.108)--(6.804,4.114)--(6.814,4.120)--(6.824,4.126)--(6.834,4.132)%
  --(6.844,4.138)--(6.854,4.144)--(6.864,4.150)--(6.874,4.156)--(6.885,4.162)--(6.895,4.168)%
  --(6.905,4.175)--(6.915,4.181)--(6.925,4.186)--(6.935,4.193)--(6.945,4.199)--(6.955,4.205)%
  --(6.965,4.211)--(6.975,4.217)--(6.985,4.223)--(6.995,4.229)--(7.005,4.235)--(7.015,4.241)%
  --(7.025,4.247)--(7.035,4.253)--(7.045,4.259)--(7.055,4.265)--(7.065,4.272)--(7.075,4.278)%
  --(7.085,4.284)--(7.095,4.290)--(7.105,4.296)--(7.115,4.302)--(7.125,4.308)--(7.135,4.314)%
  --(7.145,4.320)--(7.155,4.326)--(7.165,4.333)--(7.175,4.338)--(7.185,4.344)--(7.195,4.351)%
  --(7.205,4.357)--(7.215,4.363)--(7.225,4.369)--(7.235,4.375)--(7.245,4.381)--(7.255,4.387)%
  --(7.265,4.393)--(7.275,4.399)--(7.285,4.406)--(7.295,4.412)--(7.305,4.418)--(7.315,4.424)%
  --(7.325,4.430)--(7.335,4.436)--(7.345,4.442)--(7.355,4.448)--(7.365,4.455)--(7.375,4.461)%
  --(7.385,4.467)--(7.395,4.473)--(7.405,4.479)--(7.415,4.485)--(7.425,4.492)--(7.435,4.498)%
  --(7.445,4.504)--(7.455,4.510)--(7.465,4.516)--(7.475,4.522)--(7.485,4.528)--(7.495,4.535)%
  --(7.505,4.541)--(7.515,4.547)--(7.525,4.553)--(7.535,4.559)--(7.545,4.565)--(7.555,4.572)%
  --(7.565,4.578)--(7.575,4.584)--(7.585,4.590)--(7.595,4.596)--(7.605,4.602)--(7.615,4.608)%
  --(7.625,4.615)--(7.635,4.621)--(7.645,4.627)--(7.655,4.633)--(7.665,4.639)--(7.675,4.646)%
  --(7.685,4.652)--(7.695,4.658)--(7.705,4.664)--(7.715,4.671)--(7.725,4.677)--(7.735,4.683)%
  --(7.745,4.689)--(7.755,4.695)--(7.765,4.702)--(7.775,4.708)--(7.785,4.714)--(7.795,4.720)%
  --(7.805,4.727)--(7.815,4.733)--(7.825,4.739)--(7.835,4.745)--(7.845,4.751)--(7.855,4.758)%
  --(7.865,4.764)--(7.875,4.770)--(7.885,4.776)--(7.895,4.783)--(7.905,4.789)--(7.915,4.795)%
  --(7.925,4.801)--(7.935,4.808)--(7.945,4.814)--(7.955,4.820)--(7.965,4.826)--(7.975,4.833)%
  --(7.985,4.839)--(7.995,4.845)--(8.005,4.851)--(8.015,4.858)--(8.025,4.864)--(8.035,4.870)%
  --(8.045,4.876)--(8.055,4.883)--(8.065,4.889)--(8.075,4.896)--(8.085,4.902)--(8.095,4.908)%
  --(8.105,4.914)--(8.115,4.921)--(8.125,4.927)--(8.135,4.933)--(8.145,4.939)--(8.155,4.946)%
  --(8.165,4.952)--(8.175,4.958)--(8.185,4.965)--(8.195,4.971)--(8.205,4.978)--(8.215,4.984)%
  --(8.225,4.990)--(8.235,4.996)--(8.245,5.003)--(8.255,5.009)--(8.265,5.016)--(8.275,5.022)%
  --(8.285,5.028)--(8.295,5.035)--(8.305,5.041)--(8.315,5.047)--(8.325,5.053)--(8.335,5.060)%
  --(8.345,5.066)--(8.355,5.073)--(8.365,5.079)--(8.375,5.085)--(8.385,5.092)--(8.395,5.098)%
  --(8.405,5.105)--(8.415,5.111)--(8.425,5.117)--(8.435,5.124)--(8.445,5.130)--(8.455,5.137)%
  --(8.465,5.143)--(8.475,5.149)--(8.485,5.156)--(8.495,5.162)--(8.505,5.169)--(8.515,5.175)%
  --(8.525,5.182)--(8.535,5.188)--(8.545,5.194)--(8.555,5.201)--(8.565,5.207)--(8.575,5.214)%
  --(8.585,5.220)--(8.595,5.227)--(8.605,5.233)--(8.615,5.240)--(8.625,5.246)--(8.635,5.253)%
  --(8.645,5.259)--(8.655,5.265)--(8.665,5.272)--(8.675,5.279)--(8.685,5.285)--(8.695,5.291)%
  --(8.705,5.298)--(8.715,5.304)--(8.725,5.311)--(8.735,5.317)--(8.745,5.324)--(8.755,5.330)%
  --(8.765,5.337)--(8.775,5.344)--(8.785,5.350)--(8.795,5.356)--(8.805,5.363)--(8.815,5.370)%
  --(8.826,5.376)--(8.836,5.383)--(8.846,5.389)--(8.856,5.396)--(8.866,5.403)--(8.876,5.409)%
  --(8.886,5.415)--(8.896,5.422)--(8.906,5.429)--(8.916,5.435)--(8.926,5.442)--(8.936,5.448)%
  --(8.946,5.455)--(8.956,5.462)--(8.966,5.468)--(8.976,5.475)--(8.986,5.481)--(8.996,5.488)%
  --(9.006,5.495)--(9.016,5.501)--(9.026,5.508)--(9.036,5.514)--(9.046,5.521)--(9.056,5.528)%
  --(9.066,5.535)--(9.076,5.541)--(9.086,5.547)--(9.096,5.554)--(9.106,5.561)--(9.116,5.568)%
  --(9.126,5.574)--(9.136,5.581)--(9.146,5.588)--(9.156,5.595)--(9.166,5.601)--(9.176,5.608)%
  --(9.186,5.614)--(9.196,5.621)--(9.206,5.628)--(9.216,5.635)--(9.226,5.641)--(9.236,5.648)%
  --(9.246,5.655)--(9.256,5.662)--(9.266,5.669)--(9.276,5.675)--(9.286,5.682)--(9.296,5.689)%
  --(9.306,5.696)--(9.316,5.702)--(9.326,5.709)--(9.336,5.716)--(9.346,5.722)--(9.356,5.730)%
  --(9.366,5.736)--(9.376,5.743)--(9.386,5.750)--(9.396,5.757)--(9.406,5.764)--(9.416,5.771)%
  --(9.426,5.777)--(9.436,5.784)--(9.446,5.791)--(9.456,5.798)--(9.466,5.805)--(9.476,5.812)%
  --(9.486,5.818)--(9.496,5.825)--(9.506,5.832)--(9.516,5.839)--(9.526,5.846)--(9.536,5.853)%
  --(9.546,5.860)--(9.556,5.867)--(9.566,5.874)--(9.576,5.881)--(9.586,5.888)--(9.596,5.894)%
  --(9.606,5.902)--(9.616,5.909)--(9.626,5.916)--(9.636,5.922)--(9.646,5.929)--(9.656,5.936)%
  --(9.666,5.944)--(9.676,5.951)--(9.686,5.957)--(9.696,5.964)--(9.706,5.972)--(9.716,5.979)%
  --(9.726,5.986)--(9.736,5.993)--(9.746,6.000)--(9.756,6.007)--(9.766,6.014)--(9.776,6.021)%
  --(9.786,6.028)--(9.796,6.035)--(9.806,6.042)--(9.816,6.050)--(9.826,6.057)--(9.836,6.064)%
  --(9.846,6.071)--(9.856,6.078)--(9.866,6.085)--(9.876,6.093)--(9.886,6.100)--(9.896,6.107)%
  --(9.906,6.114)--(9.916,6.121)--(9.926,6.129)--(9.936,6.136)--(9.946,6.143)--(9.956,6.150)%
  --(9.966,6.158)--(9.976,6.165)--(9.986,6.172)--(9.996,6.179)--(10.006,6.186)--(10.016,6.194)%
  --(10.026,6.202)--(10.036,6.209)--(10.046,6.216)--(10.056,6.223)--(10.066,6.231)--(10.076,6.238)%
  --(10.086,6.245)--(10.096,6.253)--(10.106,6.260)--(10.116,6.268)--(10.126,6.275)--(10.136,6.283)%
  --(10.146,6.290)--(10.156,6.297)--(10.166,6.305)--(10.176,6.313)--(10.186,6.320)--(10.196,6.327)%
  --(10.206,6.335)--(10.216,6.343)--(10.226,6.350)--(10.236,6.357)--(10.246,6.365)--(10.256,6.373)%
  --(10.266,6.381)--(10.276,6.388)--(10.286,6.395)--(10.296,6.403)--(10.306,6.411)--(10.316,6.419)%
  --(10.326,6.426)--(10.336,6.434)--(10.346,6.441)--(10.356,6.450)--(10.366,6.457)--(10.376,6.465)%
  --(10.386,6.472)--(10.396,6.480)--(10.406,6.488)--(10.416,6.496)--(10.426,6.503)--(10.436,6.511)%
  --(10.446,6.520)--(10.456,6.527)--(10.466,6.535)--(10.476,6.543)--(10.486,6.551)--(10.496,6.559)%
  --(10.506,6.567)--(10.516,6.574)--(10.526,6.583)--(10.536,6.591)--(10.546,6.599)--(10.556,6.606)%
  --(10.566,6.615)--(10.576,6.623)--(10.586,6.631)--(10.596,6.639)--(10.606,6.647)--(10.616,6.655)%
  --(10.626,6.663)--(10.636,6.671)--(10.646,6.680)--(10.656,6.688)--(10.666,6.696)--(10.676,6.704)%
  --(10.686,6.713)--(10.696,6.721)--(10.706,6.729)--(10.716,6.737)--(10.726,6.746)--(10.736,6.754)%
  --(10.746,6.763)--(10.756,6.771)--(10.767,6.780)--(10.777,6.788)--(10.787,6.797)--(10.797,6.805)%
  --(10.807,6.814)--(10.817,6.822)--(10.827,6.831)--(10.837,6.840)--(10.847,6.848)--(10.857,6.857)%
  --(10.867,6.866)--(10.877,6.874)--(10.887,6.883)--(10.897,6.891)--(10.907,6.900)--(10.917,6.909)%
  --(10.927,6.918)--(10.937,6.927)--(10.947,6.936)--(10.957,6.945)--(10.967,6.954)--(10.977,6.963)%
  --(10.987,6.972)--(10.997,6.981)--(11.007,6.990)--(11.017,6.999)--(11.027,7.008)--(11.037,7.018)%
  --(11.047,7.027)--(11.057,7.036)--(11.067,7.045)--(11.077,7.055)--(11.087,7.064)--(11.097,7.074)%
  --(11.107,7.083)--(11.117,7.092)--(11.127,7.102)--(11.137,7.112)--(11.147,7.121)--(11.157,7.131)%
  --(11.167,7.141)--(11.177,7.150)--(11.187,7.160)--(11.197,7.170)--(11.207,7.180)--(11.217,7.190)%
  --(11.227,7.200)--(11.237,7.210)--(11.247,7.220)--(11.257,7.230)--(11.267,7.241)--(11.277,7.251)%
  --(11.287,7.261)--(11.297,7.271)--(11.307,7.282)--(11.317,7.292)--(11.327,7.303)--(11.337,7.313)%
  --(11.347,7.324)--(11.357,7.335)--(11.367,7.346)--(11.377,7.357)--(11.387,7.367)--(11.397,7.379)%
  --(11.407,7.390)--(11.417,7.401)--(11.427,7.412)--(11.437,7.423)--(11.447,7.435)--(11.457,7.447)%
  --(11.467,7.458)--(11.477,7.470)--(11.487,7.481)--(11.497,7.493)--(11.507,7.505)--(11.517,7.517)%
  --(11.527,7.529)--(11.537,7.542)--(11.547,7.554)--(11.557,7.567)--(11.567,7.579)--(11.577,7.592)%
  --(11.587,7.605)--(11.597,7.618)--(11.607,7.631)--(11.617,7.645)--(11.627,7.658)--(11.637,7.672)%
  --(11.647,7.685)--(11.657,7.700)--(11.667,7.714)--(11.677,7.728)--(11.687,7.743)--(11.697,7.757)%
  --(11.707,7.772)--(11.717,7.788)--(11.727,7.803)--(11.737,7.819)--(11.747,7.835)--(11.757,7.851)%
  --(11.767,7.867)--(11.777,7.885)--(11.787,7.902)--(11.797,7.920)--(11.807,7.938)--(11.817,7.956)%
  --(11.827,7.975)--(11.837,7.995)--(11.847,8.015)--(11.857,8.035)--(11.867,8.057)--(11.877,8.079)%
  --(11.887,8.102)--(11.897,8.126)--(11.907,8.151)--(11.917,8.178)--(11.927,8.206)--(11.937,8.236)%
  --(11.947,8.269)--(11.957,8.305)--(11.967,8.344)--(11.977,8.391)--(11.987,8.448);
\gpcolor{color=gp lt color border}
\draw[gp path] (1.688,8.631)--(1.688,0.985)--(13.447,0.985)--(13.447,8.631)--cycle;
%% coordinates of the plot area
\gpdefrectangularnode{gp plot 1}{\pgfpoint{1.688cm}{0.985cm}}{\pgfpoint{13.447cm}{8.631cm}}
\end{tikzpicture}
%% gnuplot variables

	\caption{Gráfico da densidade escalar em função da densidade bariônica para fração de prótons 1/2. O resultado mostrado nesse gráfico é obtido juntamente com os resultados mostrados na Fig.~\ref{Fig:mass_graph} através da solução da Equação~\ref{Eq:Gap_zero}.}
	\label{Fig:scalar_density_graph}
\end{figure*}

%%%%%%%%%%%%%%%%%%%%%%%%%%%%%%%%%%%%%%%%%%%%%%%%%%%%%%%%%%%%%%%%%%%%%%%%%%%%%
\section{Potenciais químicos, termodinâmico, pressão, e densidade de energia}
%%%%%%%%%%%%%%%%%%%%%%%%%%%%%%%%%%%%%%%%%%%%%%%%%%%%%%%%%%%%%%%%%%%%%%%%%%%%%

As figuras abaixo mostram os resultados obtidos através das massas e densidades escalares obtidas na seção anterior, utilizando as Equações~\eqref{Eq:Potenciais_Quimicos}, \eqref{Eq:potencial_termodinamico}, \eqref{Eq:Pressao} e~\eqref{Eq:Densidade_energia}.
\begin{figure*}
	\begin{tikzpicture}[gnuplot]
%% generated with GNUPLOT 5.0p2 (Lua 5.2; terminal rev. 99, script rev. 100)
%% Fri Feb 26 16:42:32 2016
\path (0.000,0.000) rectangle (14.000,9.000);
\gpcolor{color=gp lt color border}
\gpsetlinetype{gp lt border}
\gpsetdashtype{gp dt solid}
\gpsetlinewidth{1.00}
\draw[gp path] (1.320,0.985)--(1.500,0.985);
\draw[gp path] (13.447,0.985)--(13.267,0.985);
\node[gp node right] at (1.136,0.985) {$910$};
\draw[gp path] (1.320,1.941)--(1.500,1.941);
\draw[gp path] (13.447,1.941)--(13.267,1.941);
\node[gp node right] at (1.136,1.941) {$920$};
\draw[gp path] (1.320,2.897)--(1.500,2.897);
\draw[gp path] (13.447,2.897)--(13.267,2.897);
\node[gp node right] at (1.136,2.897) {$930$};
\draw[gp path] (1.320,3.852)--(1.500,3.852);
\draw[gp path] (13.447,3.852)--(13.267,3.852);
\node[gp node right] at (1.136,3.852) {$940$};
\draw[gp path] (1.320,4.808)--(1.500,4.808);
\draw[gp path] (13.447,4.808)--(13.267,4.808);
\node[gp node right] at (1.136,4.808) {$950$};
\draw[gp path] (1.320,5.764)--(1.500,5.764);
\draw[gp path] (13.447,5.764)--(13.267,5.764);
\node[gp node right] at (1.136,5.764) {$960$};
\draw[gp path] (1.320,6.720)--(1.500,6.720);
\draw[gp path] (13.447,6.720)--(13.267,6.720);
\node[gp node right] at (1.136,6.720) {$970$};
\draw[gp path] (1.320,7.675)--(1.500,7.675);
\draw[gp path] (13.447,7.675)--(13.267,7.675);
\node[gp node right] at (1.136,7.675) {$980$};
\draw[gp path] (1.320,8.631)--(1.500,8.631);
\draw[gp path] (13.447,8.631)--(13.267,8.631);
\node[gp node right] at (1.136,8.631) {$990$};
\draw[gp path] (1.320,0.985)--(1.320,1.165);
\draw[gp path] (1.320,8.631)--(1.320,8.451);
\node[gp node center] at (1.320,0.677) {$0$};
\draw[gp path] (2.836,0.985)--(2.836,1.165);
\draw[gp path] (2.836,8.631)--(2.836,8.451);
\node[gp node center] at (2.836,0.677) {$0.05$};
\draw[gp path] (4.352,0.985)--(4.352,1.165);
\draw[gp path] (4.352,8.631)--(4.352,8.451);
\node[gp node center] at (4.352,0.677) {$0.1$};
\draw[gp path] (5.868,0.985)--(5.868,1.165);
\draw[gp path] (5.868,8.631)--(5.868,8.451);
\node[gp node center] at (5.868,0.677) {$0.15$};
\draw[gp path] (7.384,0.985)--(7.384,1.165);
\draw[gp path] (7.384,8.631)--(7.384,8.451);
\node[gp node center] at (7.384,0.677) {$0.2$};
\draw[gp path] (8.899,0.985)--(8.899,1.165);
\draw[gp path] (8.899,8.631)--(8.899,8.451);
\node[gp node center] at (8.899,0.677) {$0.25$};
\draw[gp path] (10.415,0.985)--(10.415,1.165);
\draw[gp path] (10.415,8.631)--(10.415,8.451);
\node[gp node center] at (10.415,0.677) {$0.3$};
\draw[gp path] (11.931,0.985)--(11.931,1.165);
\draw[gp path] (11.931,8.631)--(11.931,8.451);
\node[gp node center] at (11.931,0.677) {$0.35$};
\draw[gp path] (13.447,0.985)--(13.447,1.165);
\draw[gp path] (13.447,8.631)--(13.447,8.451);
\node[gp node center] at (13.447,0.677) {$0.4$};
\draw[gp path] (1.320,8.631)--(1.320,0.985)--(13.447,0.985)--(13.447,8.631)--cycle;
\node[gp node center,rotate=-270] at (0.246,4.808) {$\mu$ (MeV)};
\node[gp node center] at (7.383,0.215) {$\rho$ ($\rm{fm}^{-3}$)};
\node[gp node left] at (2.788,8.297) {$\mu_p$};
\gpcolor{rgb color={0.580,0.000,0.827}}
\draw[gp path] (1.688,8.297)--(2.604,8.297);
\draw[gp path] (1.644,3.494)--(1.654,3.478)--(1.664,3.462)--(1.675,3.446)--(1.685,3.431)%
  --(1.695,3.415)--(1.706,3.399)--(1.716,3.383)--(1.726,3.367)--(1.737,3.351)--(1.747,3.335)%
  --(1.757,3.319)--(1.768,3.303)--(1.778,3.321)--(1.788,3.305)--(1.799,3.289)--(1.809,3.273)%
  --(1.819,3.257)--(1.830,3.241)--(1.840,3.225)--(1.850,3.209)--(1.860,3.193)--(1.871,3.177)%
  --(1.881,3.160)--(1.891,3.144)--(1.902,3.163)--(1.912,3.147)--(1.922,3.131)--(1.933,3.115)%
  --(1.943,3.099)--(1.953,3.083)--(1.964,3.067)--(1.974,3.051)--(1.984,3.035)--(1.995,3.019)%
  --(2.005,3.003)--(2.015,3.022)--(2.026,3.006)--(2.036,2.990)--(2.046,2.975)--(2.057,2.959)%
  --(2.067,2.943)--(2.077,2.928)--(2.087,2.912)--(2.098,2.896)--(2.108,2.881)--(2.118,2.865)%
  --(2.129,2.884)--(2.139,2.869)--(2.149,2.853)--(2.160,2.838)--(2.170,2.822)--(2.180,2.807)%
  --(2.191,2.792)--(2.201,2.776)--(2.211,2.761)--(2.222,2.746)--(2.232,2.765)--(2.242,2.750)%
  --(2.253,2.735)--(2.263,2.720)--(2.273,2.704)--(2.284,2.689)--(2.294,2.675)--(2.304,2.660)%
  --(2.314,2.645)--(2.325,2.664)--(2.335,2.649)--(2.345,2.635)--(2.356,2.620)--(2.366,2.605)%
  --(2.376,2.590)--(2.387,2.576)--(2.397,2.561)--(2.407,2.547)--(2.418,2.567)--(2.428,2.552)%
  --(2.438,2.538)--(2.449,2.523)--(2.459,2.509)--(2.469,2.495)--(2.480,2.481)--(2.490,2.466)%
  --(2.500,2.486)--(2.511,2.472)--(2.521,2.458)--(2.531,2.444)--(2.542,2.430)--(2.552,2.416)%
  --(2.562,2.403)--(2.572,2.389)--(2.583,2.375)--(2.593,2.395)--(2.603,2.382)--(2.614,2.368)%
  --(2.624,2.354)--(2.634,2.341)--(2.645,2.327)--(2.655,2.314)--(2.665,2.301)--(2.676,2.321)%
  --(2.686,2.308)--(2.696,2.295)--(2.707,2.281)--(2.717,2.268)--(2.727,2.255)--(2.738,2.242)%
  --(2.748,2.229)--(2.758,2.250)--(2.769,2.237)--(2.779,2.224)--(2.789,2.211)--(2.799,2.199)%
  --(2.810,2.186)--(2.820,2.173)--(2.830,2.195)--(2.841,2.182)--(2.851,2.169)--(2.861,2.157)%
  --(2.872,2.145)--(2.882,2.132)--(2.892,2.120)--(2.903,2.107)--(2.913,2.129)--(2.923,2.117)%
  --(2.934,2.105)--(2.944,2.093)--(2.954,2.081)--(2.965,2.069)--(2.975,2.057)--(2.985,2.079)%
  --(2.996,2.067)--(3.006,2.055)--(3.016,2.043)--(3.026,2.032)--(3.037,2.020)--(3.047,2.008)%
  --(3.057,2.031)--(3.068,2.019)--(3.078,2.008)--(3.088,1.996)--(3.099,1.985)--(3.109,1.974)%
  --(3.119,1.962)--(3.130,1.985)--(3.140,1.974)--(3.150,1.963)--(3.161,1.952)--(3.171,1.941)%
  --(3.181,1.930)--(3.192,1.919)--(3.202,1.942)--(3.212,1.931)--(3.223,1.920)--(3.233,1.910)%
  --(3.243,1.899)--(3.253,1.888)--(3.264,1.878)--(3.274,1.901)--(3.284,1.891)--(3.295,1.880)%
  --(3.305,1.870)--(3.315,1.860)--(3.326,1.849)--(3.336,1.839)--(3.346,1.863)--(3.357,1.853)%
  --(3.367,1.843)--(3.377,1.833)--(3.388,1.823)--(3.398,1.813)--(3.408,1.803)--(3.419,1.827)%
  --(3.429,1.817)--(3.439,1.807)--(3.450,1.798)--(3.460,1.788)--(3.470,1.779)--(3.480,1.803)%
  --(3.491,1.793)--(3.501,1.784)--(3.511,1.775)--(3.522,1.766)--(3.532,1.756)--(3.542,1.747)%
  --(3.553,1.772)--(3.563,1.763)--(3.573,1.754)--(3.584,1.745)--(3.594,1.736)--(3.604,1.727)%
  --(3.615,1.752)--(3.625,1.743)--(3.635,1.734)--(3.646,1.726)--(3.656,1.717)--(3.666,1.709)%
  --(3.677,1.700)--(3.687,1.725)--(3.697,1.717)--(3.707,1.708)--(3.718,1.700)--(3.728,1.692)%
  --(3.738,1.684)--(3.749,1.709)--(3.759,1.701)--(3.769,1.693)--(3.780,1.685)--(3.790,1.677)%
  --(3.800,1.669)--(3.811,1.695)--(3.821,1.687)--(3.831,1.679)--(3.842,1.672)--(3.852,1.664)%
  --(3.862,1.657)--(3.873,1.649)--(3.883,1.675)--(3.893,1.668)--(3.904,1.660)--(3.914,1.653)%
  --(3.924,1.646)--(3.934,1.639)--(3.945,1.665)--(3.955,1.658)--(3.965,1.651)--(3.976,1.644)%
  --(3.986,1.637)--(3.996,1.630)--(4.007,1.656)--(4.017,1.649)--(4.027,1.643)--(4.038,1.636)%
  --(4.048,1.629)--(4.058,1.623)--(4.069,1.650)--(4.079,1.643)--(4.089,1.637)--(4.100,1.630)%
  --(4.110,1.624)--(4.120,1.618)--(4.131,1.645)--(4.141,1.639)--(4.151,1.633)--(4.161,1.627)%
  --(4.172,1.621)--(4.182,1.615)--(4.192,1.642)--(4.203,1.636)--(4.213,1.630)--(4.223,1.624)%
  --(4.234,1.619)--(4.244,1.613)--(4.254,1.641)--(4.265,1.635)--(4.275,1.630)--(4.285,1.624)%
  --(4.296,1.619)--(4.306,1.614)--(4.316,1.608)--(4.327,1.636)--(4.337,1.631)--(4.347,1.626)%
  --(4.358,1.621)--(4.368,1.616)--(4.378,1.611)--(4.388,1.639)--(4.399,1.634)--(4.409,1.629)%
  --(4.419,1.624)--(4.430,1.620)--(4.440,1.615)--(4.450,1.643)--(4.461,1.639)--(4.471,1.634)%
  --(4.481,1.630)--(4.492,1.626)--(4.502,1.621)--(4.512,1.650)--(4.523,1.646)--(4.533,1.641)%
  --(4.543,1.637)--(4.554,1.633)--(4.564,1.629)--(4.574,1.658)--(4.585,1.654)--(4.595,1.650)%
  --(4.605,1.646)--(4.615,1.643)--(4.626,1.639)--(4.636,1.668)--(4.646,1.664)--(4.657,1.661)%
  --(4.667,1.657)--(4.677,1.654)--(4.688,1.650)--(4.698,1.680)--(4.708,1.676)--(4.719,1.673)%
  --(4.729,1.670)--(4.739,1.667)--(4.750,1.664)--(4.760,1.693)--(4.770,1.690)--(4.781,1.687)%
  --(4.791,1.685)--(4.801,1.682)--(4.812,1.679)--(4.822,1.709)--(4.832,1.706)--(4.842,1.703)%
  --(4.853,1.701)--(4.863,1.698)--(4.873,1.696)--(4.884,1.726)--(4.894,1.723)--(4.904,1.721)%
  --(4.915,1.719)--(4.925,1.717)--(4.935,1.715)--(4.946,1.712)--(4.956,1.743)--(4.966,1.741)%
  --(4.977,1.739)--(4.987,1.737)--(4.997,1.735)--(5.008,1.733)--(5.018,1.764)--(5.028,1.762)%
  --(5.039,1.760)--(5.049,1.759)--(5.059,1.757)--(5.069,1.756)--(5.080,1.787)--(5.090,1.785)%
  --(5.100,1.784)--(5.111,1.782)--(5.121,1.781)--(5.131,1.780)--(5.142,1.811)--(5.152,1.810)%
  --(5.162,1.809)--(5.173,1.808)--(5.183,1.807)--(5.193,1.806)--(5.204,1.805)--(5.214,1.837)%
  --(5.224,1.836)--(5.235,1.835)--(5.245,1.835)--(5.255,1.834)--(5.266,1.833)--(5.276,1.865)%
  --(5.286,1.865)--(5.296,1.864)--(5.307,1.864)--(5.317,1.864)--(5.327,1.863)--(5.338,1.863)%
  --(5.348,1.895)--(5.358,1.895)--(5.369,1.895)--(5.379,1.895)--(5.389,1.895)--(5.400,1.895)%
  --(5.410,1.927)--(5.420,1.927)--(5.431,1.928)--(5.441,1.928)--(5.451,1.928)--(5.462,1.929)%
  --(5.472,1.929)--(5.482,1.962)--(5.493,1.962)--(5.503,1.963)--(5.513,1.964)--(5.523,1.964)%
  --(5.534,1.965)--(5.544,1.998)--(5.554,1.999)--(5.565,2.000)--(5.575,2.001)--(5.585,2.002)%
  --(5.596,2.003)--(5.606,2.004)--(5.616,2.037)--(5.627,2.038)--(5.637,2.039)--(5.647,2.040)%
  --(5.658,2.042)--(5.668,2.043)--(5.678,2.045)--(5.689,2.078)--(5.699,2.080)--(5.709,2.081)%
  --(5.720,2.083)--(5.730,2.085)--(5.740,2.086)--(5.750,2.088)--(5.761,2.122)--(5.771,2.124)%
  --(5.781,2.126)--(5.792,2.128)--(5.802,2.130)--(5.812,2.132)--(5.823,2.134)--(5.833,2.168)%
  --(5.843,2.170)--(5.854,2.172)--(5.864,2.175)--(5.874,2.177)--(5.885,2.180)--(5.895,2.182)%
  --(5.905,2.216)--(5.916,2.219)--(5.926,2.221)--(5.936,2.224)--(5.947,2.227)--(5.957,2.238)%
  --(5.967,2.240)--(5.977,2.243)--(5.988,2.262)--(5.998,2.265)--(6.008,2.268)--(6.019,2.271)%
  --(6.029,2.290)--(6.039,2.293)--(6.050,2.297)--(6.060,2.300)--(6.070,2.319)--(6.081,2.322)%
  --(6.091,2.326)--(6.101,2.345)--(6.112,2.348)--(6.122,2.352)--(6.132,2.356)--(6.143,2.375)%
  --(6.153,2.379)--(6.163,2.383)--(6.174,2.387)--(6.184,2.390)--(6.194,2.410)--(6.204,2.414)%
  --(6.215,2.418)--(6.225,2.422)--(6.235,2.442)--(6.246,2.446)--(6.256,2.450)--(6.266,2.455)%
  --(6.277,2.475)--(6.287,2.479)--(6.297,2.483)--(6.308,2.488)--(6.318,2.508)--(6.328,2.513)%
  --(6.339,2.517)--(6.349,2.522)--(6.359,2.542)--(6.370,2.547)--(6.380,2.552)--(6.390,2.557)%
  --(6.401,2.562)--(6.411,2.582)--(6.421,2.587)--(6.431,2.592)--(6.442,2.598)--(6.452,2.603)%
  --(6.462,2.623)--(6.473,2.629)--(6.483,2.634)--(6.493,2.640)--(6.504,2.660)--(6.514,2.666)%
  --(6.524,2.671)--(6.535,2.677)--(6.545,2.683)--(6.555,2.704)--(6.566,2.709)--(6.576,2.715)%
  --(6.586,2.721)--(6.597,2.727)--(6.607,2.748)--(6.617,2.754)--(6.628,2.760)--(6.638,2.766)%
  --(6.648,2.773)--(6.658,2.779)--(6.669,2.800)--(6.679,2.807)--(6.689,2.813)--(6.700,2.819)%
  --(6.710,2.826)--(6.720,2.848)--(6.731,2.854)--(6.741,2.861)--(6.751,2.867)--(6.762,2.874)%
  --(6.772,2.881)--(6.782,2.903)--(6.793,2.910)--(6.803,2.917)--(6.813,2.924)--(6.824,2.931)%
  --(6.834,2.938)--(6.844,2.960)--(6.855,2.967)--(6.865,2.974)--(6.875,2.981)--(6.885,2.989)%
  --(6.896,2.996)--(6.906,3.019)--(6.916,3.026)--(6.927,3.034)--(6.937,3.041)--(6.947,3.049)%
  --(6.958,3.056)--(6.968,3.064)--(6.978,3.087)--(6.989,3.095)--(6.999,3.103)--(7.009,3.110)%
  --(7.020,3.118)--(7.030,3.126)--(7.040,3.134)--(7.051,3.143)--(7.061,3.166)--(7.071,3.174)%
  --(7.082,3.182)--(7.092,3.190)--(7.102,3.199)--(7.112,3.207)--(7.123,3.216)--(7.133,3.224)%
  --(7.143,3.233)--(7.154,3.256)--(7.164,3.265)--(7.174,3.274)--(7.185,3.282)--(7.195,3.291)%
  --(7.205,3.300)--(7.216,3.309)--(7.226,3.318)--(7.236,3.327)--(7.247,3.336)--(7.257,3.360)%
  --(7.267,3.369)--(7.278,3.378)--(7.288,3.387)--(7.298,3.397)--(7.309,3.406)--(7.319,3.415)%
  --(7.329,3.425)--(7.339,3.434)--(7.350,3.444)--(7.360,3.454)--(7.370,3.463)--(7.381,3.473)%
  --(7.391,3.483)--(7.401,3.507)--(7.412,3.517)--(7.422,3.527)--(7.432,3.537)--(7.443,3.547)%
  --(7.453,3.557)--(7.463,3.567)--(7.474,3.577)--(7.484,3.588)--(7.494,3.598)--(7.505,3.608)%
  --(7.515,3.619)--(7.525,3.629)--(7.536,3.639)--(7.546,3.650)--(7.556,3.660)--(7.566,3.671)%
  --(7.577,3.682)--(7.587,3.693)--(7.597,3.703)--(7.608,3.714)--(7.618,3.725)--(7.628,3.736)%
  --(7.639,3.747)--(7.649,3.758)--(7.659,3.769)--(7.670,3.780)--(7.680,3.791)--(7.690,3.803)%
  --(7.701,3.814)--(7.711,3.825)--(7.721,3.837)--(7.732,3.848)--(7.742,3.860)--(7.752,3.871)%
  --(7.763,3.883)--(7.773,3.894)--(7.783,3.906)--(7.793,3.918)--(7.804,3.930)--(7.814,3.942)%
  --(7.824,3.953)--(7.835,3.965)--(7.845,3.977)--(7.855,3.989)--(7.866,4.002)--(7.876,4.014)%
  --(7.886,4.012)--(7.897,4.024)--(7.907,4.036)--(7.917,4.049)--(7.928,4.061)--(7.938,4.074)%
  --(7.948,4.086)--(7.959,4.099)--(7.969,4.111)--(7.979,4.124)--(7.990,4.137)--(8.000,4.150)%
  --(8.010,4.162)--(8.021,4.161)--(8.031,4.174)--(8.041,4.187)--(8.051,4.200)--(8.062,4.213)%
  --(8.072,4.226)--(8.082,4.240)--(8.093,4.253)--(8.103,4.252)--(8.113,4.266)--(8.124,4.279)%
  --(8.134,4.292)--(8.144,4.306)--(8.155,4.320)--(8.165,4.333)--(8.175,4.347)--(8.186,4.347)%
  --(8.196,4.361)--(8.206,4.374)--(8.217,4.388)--(8.227,4.402)--(8.237,4.416)--(8.248,4.416)%
  --(8.258,4.431)--(8.268,4.445)--(8.278,4.459)--(8.289,4.473)--(8.299,4.474)--(8.309,4.488)%
  --(8.320,4.503)--(8.330,4.517)--(8.340,4.532)--(8.351,4.532)--(8.361,4.547)--(8.371,4.562)%
  --(8.382,4.577)--(8.392,4.592)--(8.402,4.593)--(8.413,4.608)--(8.423,4.623)--(8.433,4.638)%
  --(8.444,4.639)--(8.454,4.654)--(8.464,4.669)--(8.475,4.685)--(8.485,4.686)--(8.495,4.702)%
  --(8.505,4.717)--(8.516,4.733)--(8.526,4.735)--(8.536,4.750)--(8.547,4.766)--(8.557,4.782)%
  --(8.567,4.784)--(8.578,4.800)--(8.588,4.816)--(8.598,4.818)--(8.609,4.834)--(8.619,4.850)%
  --(8.629,4.853)--(8.640,4.869)--(8.650,4.885)--(8.660,4.888)--(8.671,4.904)--(8.681,4.921)%
  --(8.691,4.924)--(8.702,4.940)--(8.712,4.957)--(8.722,4.960)--(8.732,4.977)--(8.743,4.994)%
  --(8.753,4.997)--(8.763,5.014)--(8.774,5.031)--(8.784,5.034)--(8.794,5.052)--(8.805,5.055)%
  --(8.815,5.072)--(8.825,5.090)--(8.836,5.094)--(8.846,5.111)--(8.856,5.115)--(8.867,5.133)%
  --(8.877,5.150)--(8.887,5.155)--(8.898,5.172)--(8.908,5.177)--(8.918,5.195)--(8.929,5.199)%
  --(8.939,5.217)--(8.949,5.235)--(8.959,5.240)--(8.970,5.258)--(8.980,5.263)--(8.990,5.281)%
  --(9.001,5.286)--(9.011,5.305)--(9.021,5.310)--(9.032,5.328)--(9.042,5.334)--(9.052,5.352)%
  --(9.063,5.358)--(9.073,5.377)--(9.083,5.382)--(9.094,5.401)--(9.104,5.407)--(9.114,5.413)%
  --(9.125,5.432)--(9.135,5.438)--(9.145,5.458)--(9.156,5.464)--(9.166,5.483)--(9.176,5.490)%
  --(9.186,5.509)--(9.197,5.516)--(9.207,5.522)--(9.217,5.542)--(9.228,5.549)--(9.238,5.568)%
  --(9.248,5.575)--(9.259,5.582)--(9.269,5.602)--(9.279,5.610)--(9.290,5.617)--(9.300,5.637)%
  --(9.310,5.645)--(9.321,5.652)--(9.331,5.672)--(9.341,5.680)--(9.352,5.688)--(9.362,5.709)%
  --(9.372,5.716)--(9.383,5.725)--(9.393,5.745)--(9.403,5.753)--(9.413,5.762)--(9.424,5.770)%
  --(9.434,5.791)--(9.444,5.800)--(9.455,5.808)--(9.465,5.817)--(9.475,5.838)--(9.486,5.847)%
  --(9.496,5.856)--(9.506,5.865)--(9.517,5.887)--(9.527,5.896)--(9.537,5.905)--(9.548,5.915)%
  --(9.558,5.924)--(9.568,5.946)--(9.579,5.956)--(9.589,5.966)--(9.599,5.976)--(9.610,5.986)%
  --(9.620,5.996)--(9.630,6.006)--(9.640,6.028)--(9.651,6.038)--(9.661,6.049)--(9.671,6.059)%
  --(9.682,6.070)--(9.692,6.081)--(9.702,6.091)--(9.713,6.102)--(9.723,6.113)--(9.733,6.124)%
  --(9.744,6.135)--(9.754,6.147)--(9.764,6.158)--(9.775,6.169)--(9.785,6.181)--(9.795,6.192)%
  --(9.806,6.204)--(9.816,6.216)--(9.826,6.228)--(9.837,6.240)--(9.847,6.252)--(9.857,6.264)%
  --(9.867,6.276)--(9.878,6.289)--(9.888,6.301)--(9.898,6.314)--(9.909,6.327)--(9.919,6.328)%
  --(9.929,6.340)--(9.940,6.353)--(9.950,6.367)--(9.960,6.380)--(9.971,6.393)--(9.981,6.406)%
  --(9.991,6.408)--(10.002,6.422)--(10.012,6.436)--(10.022,6.450)--(10.033,6.452)--(10.043,6.466)%
  --(10.053,6.480)--(10.064,6.494)--(10.074,6.497)--(10.084,6.512)--(10.094,6.526)--(10.105,6.541)%
  --(10.115,6.544)--(10.125,6.559)--(10.136,6.574)--(10.146,6.578)--(10.156,6.593)--(10.167,6.597)%
  --(10.177,6.613)--(10.187,6.628)--(10.198,6.633)--(10.208,6.649)--(10.218,6.654)--(10.229,6.670)%
  --(10.239,6.675)--(10.249,6.691)--(10.260,6.696)--(10.270,6.713)--(10.280,6.718)--(10.291,6.735)%
  --(10.301,6.741)--(10.311,6.758)--(10.321,6.764)--(10.332,6.781)--(10.342,6.788)--(10.352,6.795)%
  --(10.363,6.812)--(10.373,6.819)--(10.383,6.837)--(10.394,6.844)--(10.404,6.852)--(10.414,6.859)%
  --(10.425,6.877)--(10.435,6.885)--(10.445,6.893)--(10.456,6.901)--(10.466,6.920)--(10.476,6.929)%
  --(10.487,6.937)--(10.497,6.946)--(10.507,6.955)--(10.518,6.975)--(10.528,6.984)--(10.538,6.993)%
  --(10.548,7.003)--(10.559,7.013)--(10.569,7.023)--(10.579,7.033)--(10.590,7.043)--(10.600,7.054)%
  --(10.610,7.064)--(10.621,7.075)--(10.631,7.086)--(10.641,7.097)--(10.652,7.106)--(10.662,7.117)%
  --(10.672,7.124)--(10.683,7.136)--(10.693,7.148)--(10.703,7.155)--(10.714,7.168)--(10.724,7.175)%
  --(10.734,7.188)--(10.745,7.196)--(10.755,7.204)--(10.765,7.218)--(10.775,7.226)--(10.786,7.235)%
  --(10.796,7.244)--(10.806,7.258)--(10.817,7.268)--(10.827,7.278)--(10.837,7.288)--(10.848,7.298)%
  --(10.858,7.304)--(10.868,7.314)--(10.879,7.325)--(10.889,7.336)--(10.899,7.343)--(10.910,7.354)%
  --(10.920,7.366)--(10.930,7.373)--(10.941,7.386)--(10.951,7.393)--(10.961,7.402)--(10.972,7.415)%
  --(10.982,7.423)--(10.992,7.432)--(11.002,7.441)--(11.013,7.451)--(11.023,7.461)--(11.033,7.471)%
  --(11.044,7.481)--(11.054,7.487)--(11.064,7.498)--(11.075,7.509)--(11.085,7.516)--(11.095,7.524)%
  --(11.106,7.536)--(11.116,7.544)--(11.126,7.552)--(11.137,7.561)--(11.147,7.574)--(11.157,7.579)%
  --(11.168,7.589)--(11.178,7.599)--(11.188,7.609)--(11.199,7.615)--(11.209,7.626)--(11.219,7.634)%
  --(11.229,7.645)--(11.240,7.654)--(11.250,7.662)--(11.260,7.671)--(11.271,7.680)--(11.281,7.690)%
  --(11.291,7.696)--(11.302,7.706)--(11.312,7.713)--(11.322,7.725)--(11.333,7.732)--(11.343,7.741)%
  --(11.353,7.750)--(11.364,7.759)--(11.374,7.769)--(11.384,7.775)--(11.395,7.782)--(11.405,7.793)%
  --(11.415,7.801)--(11.426,7.810)--(11.436,7.819)--(11.446,7.826)--(11.456,7.836)--(11.467,7.843)%
  --(11.477,7.851)--(11.487,7.860)--(11.498,7.866)--(11.508,7.876)--(11.518,7.883)--(11.529,7.891)%
  --(11.539,7.900)--(11.549,7.909)--(11.560,7.916)--(11.570,7.923)--(11.580,7.932)--(11.591,7.941)%
  --(11.601,7.948)--(11.611,7.956)--(11.622,7.965)--(11.632,7.972)--(11.642,7.980)--(11.653,7.988)%
  --(11.663,7.995)--(11.673,8.004)--(11.683,8.010)--(11.694,8.018)--(11.704,8.027)--(11.714,8.034)%
  --(11.725,8.041)--(11.735,8.048)--(11.745,8.055)--(11.756,8.064)--(11.766,8.071)--(11.776,8.078)%
  --(11.787,8.085)--(11.797,8.092)--(11.807,8.101)--(11.818,8.107)--(11.828,8.114)--(11.838,8.122)%
  --(11.849,8.129)--(11.859,8.136)--(11.869,8.143)--(11.880,8.150)--(11.890,8.157)--(11.900,8.164)%
  --(11.910,8.171)--(11.921,8.178)--(11.931,8.184)--(11.941,8.192);
\gpcolor{color=gp lt color border}
\node[gp node left] at (2.788,7.989) {$\mu_n$};
\gpcolor{rgb color={0.000,0.620,0.451}}
\draw[gp path] (1.688,7.989)--(2.604,7.989);
\draw[gp path] (1.644,3.494)--(1.654,3.478)--(1.664,3.462)--(1.675,3.446)--(1.685,3.431)%
  --(1.695,3.415)--(1.706,3.399)--(1.716,3.383)--(1.726,3.367)--(1.737,3.351)--(1.747,3.335)%
  --(1.757,3.319)--(1.768,3.303)--(1.778,3.321)--(1.788,3.305)--(1.799,3.289)--(1.809,3.273)%
  --(1.819,3.257)--(1.830,3.241)--(1.840,3.225)--(1.850,3.209)--(1.860,3.193)--(1.871,3.177)%
  --(1.881,3.160)--(1.891,3.144)--(1.902,3.163)--(1.912,3.147)--(1.922,3.131)--(1.933,3.115)%
  --(1.943,3.099)--(1.953,3.083)--(1.964,3.067)--(1.974,3.051)--(1.984,3.035)--(1.995,3.019)%
  --(2.005,3.003)--(2.015,3.022)--(2.026,3.006)--(2.036,2.990)--(2.046,2.975)--(2.057,2.959)%
  --(2.067,2.943)--(2.077,2.928)--(2.087,2.912)--(2.098,2.896)--(2.108,2.881)--(2.118,2.865)%
  --(2.129,2.884)--(2.139,2.869)--(2.149,2.853)--(2.160,2.838)--(2.170,2.822)--(2.180,2.807)%
  --(2.191,2.792)--(2.201,2.776)--(2.211,2.761)--(2.222,2.746)--(2.232,2.765)--(2.242,2.750)%
  --(2.253,2.735)--(2.263,2.720)--(2.273,2.704)--(2.284,2.689)--(2.294,2.675)--(2.304,2.660)%
  --(2.314,2.645)--(2.325,2.664)--(2.335,2.649)--(2.345,2.635)--(2.356,2.620)--(2.366,2.605)%
  --(2.376,2.590)--(2.387,2.576)--(2.397,2.561)--(2.407,2.547)--(2.418,2.567)--(2.428,2.552)%
  --(2.438,2.538)--(2.449,2.523)--(2.459,2.509)--(2.469,2.495)--(2.480,2.481)--(2.490,2.466)%
  --(2.500,2.486)--(2.511,2.472)--(2.521,2.458)--(2.531,2.444)--(2.542,2.430)--(2.552,2.416)%
  --(2.562,2.403)--(2.572,2.389)--(2.583,2.375)--(2.593,2.395)--(2.603,2.382)--(2.614,2.368)%
  --(2.624,2.354)--(2.634,2.341)--(2.645,2.327)--(2.655,2.314)--(2.665,2.301)--(2.676,2.321)%
  --(2.686,2.308)--(2.696,2.295)--(2.707,2.281)--(2.717,2.268)--(2.727,2.255)--(2.738,2.242)%
  --(2.748,2.229)--(2.758,2.250)--(2.769,2.237)--(2.779,2.224)--(2.789,2.211)--(2.799,2.199)%
  --(2.810,2.186)--(2.820,2.173)--(2.830,2.195)--(2.841,2.182)--(2.851,2.169)--(2.861,2.157)%
  --(2.872,2.145)--(2.882,2.132)--(2.892,2.120)--(2.903,2.107)--(2.913,2.129)--(2.923,2.117)%
  --(2.934,2.105)--(2.944,2.093)--(2.954,2.081)--(2.965,2.069)--(2.975,2.057)--(2.985,2.079)%
  --(2.996,2.067)--(3.006,2.055)--(3.016,2.043)--(3.026,2.032)--(3.037,2.020)--(3.047,2.008)%
  --(3.057,2.031)--(3.068,2.019)--(3.078,2.008)--(3.088,1.996)--(3.099,1.985)--(3.109,1.974)%
  --(3.119,1.962)--(3.130,1.985)--(3.140,1.974)--(3.150,1.963)--(3.161,1.952)--(3.171,1.941)%
  --(3.181,1.930)--(3.192,1.919)--(3.202,1.942)--(3.212,1.931)--(3.223,1.920)--(3.233,1.910)%
  --(3.243,1.899)--(3.253,1.888)--(3.264,1.878)--(3.274,1.901)--(3.284,1.891)--(3.295,1.880)%
  --(3.305,1.870)--(3.315,1.860)--(3.326,1.849)--(3.336,1.839)--(3.346,1.863)--(3.357,1.853)%
  --(3.367,1.843)--(3.377,1.833)--(3.388,1.823)--(3.398,1.813)--(3.408,1.803)--(3.419,1.827)%
  --(3.429,1.817)--(3.439,1.807)--(3.450,1.798)--(3.460,1.788)--(3.470,1.779)--(3.480,1.803)%
  --(3.491,1.793)--(3.501,1.784)--(3.511,1.775)--(3.522,1.766)--(3.532,1.756)--(3.542,1.747)%
  --(3.553,1.772)--(3.563,1.763)--(3.573,1.754)--(3.584,1.745)--(3.594,1.736)--(3.604,1.727)%
  --(3.615,1.752)--(3.625,1.743)--(3.635,1.734)--(3.646,1.726)--(3.656,1.717)--(3.666,1.709)%
  --(3.677,1.700)--(3.687,1.725)--(3.697,1.717)--(3.707,1.708)--(3.718,1.700)--(3.728,1.692)%
  --(3.738,1.684)--(3.749,1.709)--(3.759,1.701)--(3.769,1.693)--(3.780,1.685)--(3.790,1.677)%
  --(3.800,1.669)--(3.811,1.695)--(3.821,1.687)--(3.831,1.679)--(3.842,1.672)--(3.852,1.664)%
  --(3.862,1.657)--(3.873,1.649)--(3.883,1.675)--(3.893,1.668)--(3.904,1.660)--(3.914,1.653)%
  --(3.924,1.646)--(3.934,1.639)--(3.945,1.665)--(3.955,1.658)--(3.965,1.651)--(3.976,1.644)%
  --(3.986,1.637)--(3.996,1.630)--(4.007,1.656)--(4.017,1.649)--(4.027,1.643)--(4.038,1.636)%
  --(4.048,1.629)--(4.058,1.623)--(4.069,1.650)--(4.079,1.643)--(4.089,1.637)--(4.100,1.630)%
  --(4.110,1.624)--(4.120,1.618)--(4.131,1.645)--(4.141,1.639)--(4.151,1.633)--(4.161,1.627)%
  --(4.172,1.621)--(4.182,1.615)--(4.192,1.642)--(4.203,1.636)--(4.213,1.630)--(4.223,1.624)%
  --(4.234,1.619)--(4.244,1.613)--(4.254,1.641)--(4.265,1.635)--(4.275,1.630)--(4.285,1.624)%
  --(4.296,1.619)--(4.306,1.614)--(4.316,1.608)--(4.327,1.636)--(4.337,1.631)--(4.347,1.626)%
  --(4.358,1.621)--(4.368,1.616)--(4.378,1.611)--(4.388,1.639)--(4.399,1.634)--(4.409,1.629)%
  --(4.419,1.624)--(4.430,1.620)--(4.440,1.615)--(4.450,1.643)--(4.461,1.639)--(4.471,1.634)%
  --(4.481,1.630)--(4.492,1.626)--(4.502,1.621)--(4.512,1.650)--(4.523,1.646)--(4.533,1.641)%
  --(4.543,1.637)--(4.554,1.633)--(4.564,1.629)--(4.574,1.658)--(4.585,1.654)--(4.595,1.650)%
  --(4.605,1.646)--(4.615,1.643)--(4.626,1.639)--(4.636,1.668)--(4.646,1.664)--(4.657,1.661)%
  --(4.667,1.657)--(4.677,1.654)--(4.688,1.650)--(4.698,1.680)--(4.708,1.676)--(4.719,1.673)%
  --(4.729,1.670)--(4.739,1.667)--(4.750,1.664)--(4.760,1.693)--(4.770,1.690)--(4.781,1.687)%
  --(4.791,1.685)--(4.801,1.682)--(4.812,1.679)--(4.822,1.709)--(4.832,1.706)--(4.842,1.703)%
  --(4.853,1.701)--(4.863,1.698)--(4.873,1.696)--(4.884,1.726)--(4.894,1.723)--(4.904,1.721)%
  --(4.915,1.719)--(4.925,1.717)--(4.935,1.715)--(4.946,1.712)--(4.956,1.743)--(4.966,1.741)%
  --(4.977,1.739)--(4.987,1.737)--(4.997,1.735)--(5.008,1.733)--(5.018,1.764)--(5.028,1.762)%
  --(5.039,1.760)--(5.049,1.759)--(5.059,1.757)--(5.069,1.756)--(5.080,1.787)--(5.090,1.785)%
  --(5.100,1.784)--(5.111,1.782)--(5.121,1.781)--(5.131,1.780)--(5.142,1.811)--(5.152,1.810)%
  --(5.162,1.809)--(5.173,1.808)--(5.183,1.807)--(5.193,1.806)--(5.204,1.805)--(5.214,1.837)%
  --(5.224,1.836)--(5.235,1.835)--(5.245,1.835)--(5.255,1.834)--(5.266,1.833)--(5.276,1.865)%
  --(5.286,1.865)--(5.296,1.864)--(5.307,1.864)--(5.317,1.864)--(5.327,1.863)--(5.338,1.863)%
  --(5.348,1.895)--(5.358,1.895)--(5.369,1.895)--(5.379,1.895)--(5.389,1.895)--(5.400,1.895)%
  --(5.410,1.927)--(5.420,1.927)--(5.431,1.928)--(5.441,1.928)--(5.451,1.928)--(5.462,1.929)%
  --(5.472,1.929)--(5.482,1.962)--(5.493,1.962)--(5.503,1.963)--(5.513,1.964)--(5.523,1.964)%
  --(5.534,1.965)--(5.544,1.998)--(5.554,1.999)--(5.565,2.000)--(5.575,2.001)--(5.585,2.002)%
  --(5.596,2.003)--(5.606,2.004)--(5.616,2.037)--(5.627,2.038)--(5.637,2.039)--(5.647,2.040)%
  --(5.658,2.042)--(5.668,2.043)--(5.678,2.045)--(5.689,2.078)--(5.699,2.080)--(5.709,2.081)%
  --(5.720,2.083)--(5.730,2.085)--(5.740,2.086)--(5.750,2.088)--(5.761,2.122)--(5.771,2.124)%
  --(5.781,2.126)--(5.792,2.128)--(5.802,2.130)--(5.812,2.132)--(5.823,2.134)--(5.833,2.168)%
  --(5.843,2.170)--(5.854,2.172)--(5.864,2.175)--(5.874,2.177)--(5.885,2.180)--(5.895,2.182)%
  --(5.905,2.216)--(5.916,2.219)--(5.926,2.221)--(5.936,2.224)--(5.947,2.227)--(5.957,2.238)%
  --(5.967,2.240)--(5.977,2.243)--(5.988,2.262)--(5.998,2.265)--(6.008,2.268)--(6.019,2.271)%
  --(6.029,2.290)--(6.039,2.293)--(6.050,2.297)--(6.060,2.300)--(6.070,2.319)--(6.081,2.322)%
  --(6.091,2.326)--(6.101,2.345)--(6.112,2.348)--(6.122,2.352)--(6.132,2.356)--(6.143,2.375)%
  --(6.153,2.379)--(6.163,2.383)--(6.174,2.387)--(6.184,2.390)--(6.194,2.410)--(6.204,2.414)%
  --(6.215,2.418)--(6.225,2.422)--(6.235,2.442)--(6.246,2.446)--(6.256,2.450)--(6.266,2.455)%
  --(6.277,2.475)--(6.287,2.479)--(6.297,2.483)--(6.308,2.488)--(6.318,2.508)--(6.328,2.513)%
  --(6.339,2.517)--(6.349,2.522)--(6.359,2.542)--(6.370,2.547)--(6.380,2.552)--(6.390,2.557)%
  --(6.401,2.562)--(6.411,2.582)--(6.421,2.587)--(6.431,2.592)--(6.442,2.598)--(6.452,2.603)%
  --(6.462,2.623)--(6.473,2.629)--(6.483,2.634)--(6.493,2.640)--(6.504,2.660)--(6.514,2.666)%
  --(6.524,2.671)--(6.535,2.677)--(6.545,2.683)--(6.555,2.704)--(6.566,2.709)--(6.576,2.715)%
  --(6.586,2.721)--(6.597,2.727)--(6.607,2.748)--(6.617,2.754)--(6.628,2.760)--(6.638,2.766)%
  --(6.648,2.773)--(6.658,2.779)--(6.669,2.800)--(6.679,2.807)--(6.689,2.813)--(6.700,2.819)%
  --(6.710,2.826)--(6.720,2.848)--(6.731,2.854)--(6.741,2.861)--(6.751,2.867)--(6.762,2.874)%
  --(6.772,2.881)--(6.782,2.903)--(6.793,2.910)--(6.803,2.917)--(6.813,2.924)--(6.824,2.931)%
  --(6.834,2.938)--(6.844,2.960)--(6.855,2.967)--(6.865,2.974)--(6.875,2.981)--(6.885,2.989)%
  --(6.896,2.996)--(6.906,3.019)--(6.916,3.026)--(6.927,3.034)--(6.937,3.041)--(6.947,3.049)%
  --(6.958,3.056)--(6.968,3.064)--(6.978,3.087)--(6.989,3.095)--(6.999,3.103)--(7.009,3.110)%
  --(7.020,3.118)--(7.030,3.126)--(7.040,3.134)--(7.051,3.143)--(7.061,3.166)--(7.071,3.174)%
  --(7.082,3.182)--(7.092,3.190)--(7.102,3.199)--(7.112,3.207)--(7.123,3.216)--(7.133,3.224)%
  --(7.143,3.233)--(7.154,3.256)--(7.164,3.265)--(7.174,3.274)--(7.185,3.282)--(7.195,3.291)%
  --(7.205,3.300)--(7.216,3.309)--(7.226,3.318)--(7.236,3.327)--(7.247,3.336)--(7.257,3.360)%
  --(7.267,3.369)--(7.278,3.378)--(7.288,3.387)--(7.298,3.397)--(7.309,3.406)--(7.319,3.415)%
  --(7.329,3.425)--(7.339,3.434)--(7.350,3.444)--(7.360,3.454)--(7.370,3.463)--(7.381,3.473)%
  --(7.391,3.483)--(7.401,3.507)--(7.412,3.517)--(7.422,3.527)--(7.432,3.537)--(7.443,3.547)%
  --(7.453,3.557)--(7.463,3.567)--(7.474,3.577)--(7.484,3.588)--(7.494,3.598)--(7.505,3.608)%
  --(7.515,3.619)--(7.525,3.629)--(7.536,3.639)--(7.546,3.650)--(7.556,3.660)--(7.566,3.671)%
  --(7.577,3.682)--(7.587,3.693)--(7.597,3.703)--(7.608,3.714)--(7.618,3.725)--(7.628,3.736)%
  --(7.639,3.747)--(7.649,3.758)--(7.659,3.769)--(7.670,3.780)--(7.680,3.791)--(7.690,3.803)%
  --(7.701,3.814)--(7.711,3.825)--(7.721,3.837)--(7.732,3.848)--(7.742,3.860)--(7.752,3.871)%
  --(7.763,3.883)--(7.773,3.894)--(7.783,3.906)--(7.793,3.918)--(7.804,3.930)--(7.814,3.942)%
  --(7.824,3.953)--(7.835,3.965)--(7.845,3.977)--(7.855,3.989)--(7.866,4.002)--(7.876,4.014)%
  --(7.886,4.012)--(7.897,4.024)--(7.907,4.036)--(7.917,4.049)--(7.928,4.061)--(7.938,4.074)%
  --(7.948,4.086)--(7.959,4.099)--(7.969,4.111)--(7.979,4.124)--(7.990,4.137)--(8.000,4.150)%
  --(8.010,4.162)--(8.021,4.161)--(8.031,4.174)--(8.041,4.187)--(8.051,4.200)--(8.062,4.213)%
  --(8.072,4.226)--(8.082,4.240)--(8.093,4.253)--(8.103,4.252)--(8.113,4.266)--(8.124,4.279)%
  --(8.134,4.292)--(8.144,4.306)--(8.155,4.320)--(8.165,4.333)--(8.175,4.347)--(8.186,4.347)%
  --(8.196,4.361)--(8.206,4.374)--(8.217,4.388)--(8.227,4.402)--(8.237,4.416)--(8.248,4.416)%
  --(8.258,4.431)--(8.268,4.445)--(8.278,4.459)--(8.289,4.473)--(8.299,4.474)--(8.309,4.488)%
  --(8.320,4.503)--(8.330,4.517)--(8.340,4.532)--(8.351,4.532)--(8.361,4.547)--(8.371,4.562)%
  --(8.382,4.577)--(8.392,4.592)--(8.402,4.593)--(8.413,4.608)--(8.423,4.623)--(8.433,4.638)%
  --(8.444,4.639)--(8.454,4.654)--(8.464,4.669)--(8.475,4.685)--(8.485,4.686)--(8.495,4.702)%
  --(8.505,4.717)--(8.516,4.733)--(8.526,4.735)--(8.536,4.750)--(8.547,4.766)--(8.557,4.782)%
  --(8.567,4.784)--(8.578,4.800)--(8.588,4.816)--(8.598,4.818)--(8.609,4.834)--(8.619,4.850)%
  --(8.629,4.853)--(8.640,4.869)--(8.650,4.885)--(8.660,4.888)--(8.671,4.904)--(8.681,4.921)%
  --(8.691,4.924)--(8.702,4.940)--(8.712,4.957)--(8.722,4.960)--(8.732,4.977)--(8.743,4.994)%
  --(8.753,4.997)--(8.763,5.014)--(8.774,5.031)--(8.784,5.034)--(8.794,5.052)--(8.805,5.055)%
  --(8.815,5.072)--(8.825,5.090)--(8.836,5.094)--(8.846,5.111)--(8.856,5.115)--(8.867,5.133)%
  --(8.877,5.150)--(8.887,5.155)--(8.898,5.172)--(8.908,5.177)--(8.918,5.195)--(8.929,5.199)%
  --(8.939,5.217)--(8.949,5.235)--(8.959,5.240)--(8.970,5.258)--(8.980,5.263)--(8.990,5.281)%
  --(9.001,5.286)--(9.011,5.305)--(9.021,5.310)--(9.032,5.328)--(9.042,5.334)--(9.052,5.352)%
  --(9.063,5.358)--(9.073,5.377)--(9.083,5.382)--(9.094,5.401)--(9.104,5.407)--(9.114,5.413)%
  --(9.125,5.432)--(9.135,5.438)--(9.145,5.458)--(9.156,5.464)--(9.166,5.483)--(9.176,5.490)%
  --(9.186,5.509)--(9.197,5.516)--(9.207,5.522)--(9.217,5.542)--(9.228,5.549)--(9.238,5.568)%
  --(9.248,5.575)--(9.259,5.582)--(9.269,5.602)--(9.279,5.610)--(9.290,5.617)--(9.300,5.637)%
  --(9.310,5.645)--(9.321,5.652)--(9.331,5.672)--(9.341,5.680)--(9.352,5.688)--(9.362,5.709)%
  --(9.372,5.716)--(9.383,5.725)--(9.393,5.745)--(9.403,5.753)--(9.413,5.762)--(9.424,5.770)%
  --(9.434,5.791)--(9.444,5.800)--(9.455,5.808)--(9.465,5.817)--(9.475,5.838)--(9.486,5.847)%
  --(9.496,5.856)--(9.506,5.865)--(9.517,5.887)--(9.527,5.896)--(9.537,5.905)--(9.548,5.915)%
  --(9.558,5.924)--(9.568,5.946)--(9.579,5.956)--(9.589,5.966)--(9.599,5.976)--(9.610,5.986)%
  --(9.620,5.996)--(9.630,6.006)--(9.640,6.028)--(9.651,6.038)--(9.661,6.049)--(9.671,6.059)%
  --(9.682,6.070)--(9.692,6.081)--(9.702,6.091)--(9.713,6.102)--(9.723,6.113)--(9.733,6.124)%
  --(9.744,6.135)--(9.754,6.147)--(9.764,6.158)--(9.775,6.169)--(9.785,6.181)--(9.795,6.192)%
  --(9.806,6.204)--(9.816,6.216)--(9.826,6.228)--(9.837,6.240)--(9.847,6.252)--(9.857,6.264)%
  --(9.867,6.276)--(9.878,6.289)--(9.888,6.301)--(9.898,6.314)--(9.909,6.327)--(9.919,6.328)%
  --(9.929,6.340)--(9.940,6.353)--(9.950,6.367)--(9.960,6.380)--(9.971,6.393)--(9.981,6.406)%
  --(9.991,6.408)--(10.002,6.422)--(10.012,6.436)--(10.022,6.450)--(10.033,6.452)--(10.043,6.466)%
  --(10.053,6.480)--(10.064,6.494)--(10.074,6.497)--(10.084,6.512)--(10.094,6.526)--(10.105,6.541)%
  --(10.115,6.544)--(10.125,6.559)--(10.136,6.574)--(10.146,6.578)--(10.156,6.593)--(10.167,6.597)%
  --(10.177,6.613)--(10.187,6.628)--(10.198,6.633)--(10.208,6.649)--(10.218,6.654)--(10.229,6.670)%
  --(10.239,6.675)--(10.249,6.691)--(10.260,6.696)--(10.270,6.713)--(10.280,6.718)--(10.291,6.735)%
  --(10.301,6.741)--(10.311,6.758)--(10.321,6.764)--(10.332,6.781)--(10.342,6.788)--(10.352,6.795)%
  --(10.363,6.812)--(10.373,6.819)--(10.383,6.837)--(10.394,6.844)--(10.404,6.852)--(10.414,6.859)%
  --(10.425,6.877)--(10.435,6.885)--(10.445,6.893)--(10.456,6.901)--(10.466,6.920)--(10.476,6.929)%
  --(10.487,6.937)--(10.497,6.946)--(10.507,6.955)--(10.518,6.975)--(10.528,6.984)--(10.538,6.993)%
  --(10.548,7.003)--(10.559,7.013)--(10.569,7.023)--(10.579,7.033)--(10.590,7.043)--(10.600,7.054)%
  --(10.610,7.064)--(10.621,7.075)--(10.631,7.086)--(10.641,7.097)--(10.652,7.106)--(10.662,7.117)%
  --(10.672,7.124)--(10.683,7.136)--(10.693,7.148)--(10.703,7.155)--(10.714,7.168)--(10.724,7.175)%
  --(10.734,7.188)--(10.745,7.196)--(10.755,7.204)--(10.765,7.218)--(10.775,7.226)--(10.786,7.235)%
  --(10.796,7.244)--(10.806,7.258)--(10.817,7.268)--(10.827,7.278)--(10.837,7.288)--(10.848,7.298)%
  --(10.858,7.304)--(10.868,7.314)--(10.879,7.325)--(10.889,7.336)--(10.899,7.343)--(10.910,7.354)%
  --(10.920,7.366)--(10.930,7.373)--(10.941,7.386)--(10.951,7.393)--(10.961,7.402)--(10.972,7.415)%
  --(10.982,7.423)--(10.992,7.432)--(11.002,7.441)--(11.013,7.451)--(11.023,7.461)--(11.033,7.471)%
  --(11.044,7.481)--(11.054,7.487)--(11.064,7.498)--(11.075,7.509)--(11.085,7.516)--(11.095,7.524)%
  --(11.106,7.536)--(11.116,7.544)--(11.126,7.552)--(11.137,7.561)--(11.147,7.574)--(11.157,7.579)%
  --(11.168,7.589)--(11.178,7.599)--(11.188,7.609)--(11.199,7.615)--(11.209,7.626)--(11.219,7.634)%
  --(11.229,7.645)--(11.240,7.654)--(11.250,7.662)--(11.260,7.671)--(11.271,7.680)--(11.281,7.690)%
  --(11.291,7.696)--(11.302,7.706)--(11.312,7.713)--(11.322,7.725)--(11.333,7.732)--(11.343,7.741)%
  --(11.353,7.750)--(11.364,7.759)--(11.374,7.769)--(11.384,7.775)--(11.395,7.782)--(11.405,7.793)%
  --(11.415,7.801)--(11.426,7.810)--(11.436,7.819)--(11.446,7.826)--(11.456,7.836)--(11.467,7.843)%
  --(11.477,7.851)--(11.487,7.860)--(11.498,7.866)--(11.508,7.876)--(11.518,7.883)--(11.529,7.891)%
  --(11.539,7.900)--(11.549,7.909)--(11.560,7.916)--(11.570,7.923)--(11.580,7.932)--(11.591,7.941)%
  --(11.601,7.948)--(11.611,7.956)--(11.622,7.965)--(11.632,7.972)--(11.642,7.980)--(11.653,7.988)%
  --(11.663,7.995)--(11.673,8.004)--(11.683,8.010)--(11.694,8.018)--(11.704,8.027)--(11.714,8.034)%
  --(11.725,8.041)--(11.735,8.048)--(11.745,8.055)--(11.756,8.064)--(11.766,8.071)--(11.776,8.078)%
  --(11.787,8.085)--(11.797,8.092)--(11.807,8.101)--(11.818,8.107)--(11.828,8.114)--(11.838,8.122)%
  --(11.849,8.129)--(11.859,8.136)--(11.869,8.143)--(11.880,8.150)--(11.890,8.157)--(11.900,8.164)%
  --(11.910,8.171)--(11.921,8.178)--(11.931,8.184)--(11.941,8.192);
\gpcolor{color=gp lt color border}
\draw[gp path] (1.320,8.631)--(1.320,0.985)--(13.447,0.985)--(13.447,8.631)--cycle;
%% coordinates of the plot area
\gpdefrectangularnode{gp plot 1}{\pgfpoint{1.320cm}{0.985cm}}{\pgfpoint{13.447cm}{8.631cm}}
\end{tikzpicture}
%% gnuplot variables

	\caption{Gráfico dos potenciais químicos de próton e nêutron calculados a partir da Equação~\eqref{Eq:Potenciais_Quimicos}. Como a fração de próton é 1/2, ambos os potenciais tem o mesmo valor.}
	\label{Fig:chemical_potential_graph}
\end{figure*}

\begin{figure*}
	\begin{tikzpicture}[gnuplot]
%% generated with GNUPLOT 5.0p2 (Lua 5.2; terminal rev. 99, script rev. 100)
%% Mon Feb 29 16:15:38 2016
\path (0.000,0.000) rectangle (14.000,9.000);
\gpcolor{color=gp lt color border}
\gpsetlinetype{gp lt border}
\gpsetdashtype{gp dt solid}
\gpsetlinewidth{1.00}
\draw[gp path] (1.504,0.985)--(1.684,0.985);
\draw[gp path] (13.447,0.985)--(13.267,0.985);
\node[gp node right] at (1.320,0.985) {$-295$};
\draw[gp path] (1.504,2.897)--(1.684,2.897);
\draw[gp path] (13.447,2.897)--(13.267,2.897);
\node[gp node right] at (1.320,2.897) {$-290$};
\draw[gp path] (1.504,4.808)--(1.684,4.808);
\draw[gp path] (13.447,4.808)--(13.267,4.808);
\node[gp node right] at (1.320,4.808) {$-285$};
\draw[gp path] (1.504,6.720)--(1.684,6.720);
\draw[gp path] (13.447,6.720)--(13.267,6.720);
\node[gp node right] at (1.320,6.720) {$-280$};
\draw[gp path] (1.504,8.631)--(1.684,8.631);
\draw[gp path] (13.447,8.631)--(13.267,8.631);
\node[gp node right] at (1.320,8.631) {$-275$};
\draw[gp path] (1.504,0.985)--(1.504,1.165);
\draw[gp path] (1.504,8.631)--(1.504,8.451);
\node[gp node center] at (1.504,0.677) {$0$};
\draw[gp path] (2.997,0.985)--(2.997,1.165);
\draw[gp path] (2.997,8.631)--(2.997,8.451);
\node[gp node center] at (2.997,0.677) {$0.05$};
\draw[gp path] (4.490,0.985)--(4.490,1.165);
\draw[gp path] (4.490,8.631)--(4.490,8.451);
\node[gp node center] at (4.490,0.677) {$0.1$};
\draw[gp path] (5.983,0.985)--(5.983,1.165);
\draw[gp path] (5.983,8.631)--(5.983,8.451);
\node[gp node center] at (5.983,0.677) {$0.15$};
\draw[gp path] (7.476,0.985)--(7.476,1.165);
\draw[gp path] (7.476,8.631)--(7.476,8.451);
\node[gp node center] at (7.476,0.677) {$0.2$};
\draw[gp path] (8.968,0.985)--(8.968,1.165);
\draw[gp path] (8.968,8.631)--(8.968,8.451);
\node[gp node center] at (8.968,0.677) {$0.25$};
\draw[gp path] (10.461,0.985)--(10.461,1.165);
\draw[gp path] (10.461,8.631)--(10.461,8.451);
\node[gp node center] at (10.461,0.677) {$0.3$};
\draw[gp path] (11.954,0.985)--(11.954,1.165);
\draw[gp path] (11.954,8.631)--(11.954,8.451);
\node[gp node center] at (11.954,0.677) {$0.35$};
\draw[gp path] (13.447,0.985)--(13.447,1.165);
\draw[gp path] (13.447,8.631)--(13.447,8.451);
\node[gp node center] at (13.447,0.677) {$0.4$};
\draw[gp path] (1.504,8.631)--(1.504,0.985)--(13.447,0.985)--(13.447,8.631)--cycle;
\node[gp node center,rotate=-270] at (0.246,4.808) {$\omega$ ($\rm{MeV}/\rm{fm}^{3}$)};
\node[gp node center] at (7.475,0.215) {$\rho$ ($\rm{fm}^{-3}$)};
\gpcolor{rgb color={0.580,0.000,0.827}}
\draw[gp path] (1.813,7.929)--(1.823,7.930)--(1.833,7.931)--(1.843,7.931)--(1.853,7.932)%
  --(1.864,7.933)--(1.874,7.934)--(1.884,7.934)--(1.894,7.935)--(1.904,7.936)--(1.914,7.937)%
  --(1.925,7.938)--(1.935,7.939)--(1.945,7.940)--(1.955,7.939)--(1.965,7.940)--(1.975,7.941)%
  --(1.985,7.942)--(1.996,7.943)--(2.006,7.944)--(2.016,7.945)--(2.026,7.946)--(2.036,7.947)%
  --(2.046,7.948)--(2.057,7.949)--(2.067,7.951)--(2.077,7.949)--(2.087,7.951)--(2.097,7.952)%
  --(2.107,7.953)--(2.118,7.954)--(2.128,7.956)--(2.138,7.957)--(2.148,7.958)--(2.158,7.960)%
  --(2.168,7.961)--(2.179,7.963)--(2.189,7.961)--(2.199,7.962)--(2.209,7.964)--(2.219,7.965)%
  --(2.229,7.967)--(2.240,7.968)--(2.250,7.970)--(2.260,7.971)--(2.270,7.973)--(2.280,7.975)%
  --(2.290,7.976)--(2.300,7.974)--(2.311,7.976)--(2.321,7.978)--(2.331,7.979)--(2.341,7.981)%
  --(2.351,7.983)--(2.361,7.985)--(2.372,7.986)--(2.382,7.988)--(2.392,7.990)--(2.402,7.988)%
  --(2.412,7.990)--(2.422,7.991)--(2.433,7.993)--(2.443,7.995)--(2.453,7.997)--(2.463,7.999)%
  --(2.473,8.001)--(2.483,8.003)--(2.494,8.000)--(2.504,8.002)--(2.514,8.004)--(2.524,8.006)%
  --(2.534,8.008)--(2.544,8.010)--(2.555,8.012)--(2.565,8.014)--(2.575,8.016)--(2.585,8.014)%
  --(2.595,8.016)--(2.605,8.018)--(2.616,8.020)--(2.626,8.022)--(2.636,8.024)--(2.646,8.026)%
  --(2.656,8.029)--(2.666,8.025)--(2.676,8.028)--(2.687,8.030)--(2.697,8.032)--(2.707,8.034)%
  --(2.717,8.037)--(2.727,8.039)--(2.737,8.041)--(2.748,8.043)--(2.758,8.040)--(2.768,8.042)%
  --(2.778,8.045)--(2.788,8.047)--(2.798,8.049)--(2.809,8.052)--(2.819,8.054)--(2.829,8.056)%
  --(2.839,8.053)--(2.849,8.055)--(2.859,8.057)--(2.870,8.060)--(2.880,8.062)--(2.890,8.065)%
  --(2.900,8.067)--(2.910,8.070)--(2.920,8.066)--(2.931,8.068)--(2.941,8.071)--(2.951,8.073)%
  --(2.961,8.076)--(2.971,8.078)--(2.981,8.081)--(2.991,8.076)--(3.002,8.079)--(3.012,8.081)%
  --(3.022,8.084)--(3.032,8.086)--(3.042,8.089)--(3.052,8.092)--(3.063,8.094)--(3.073,8.090)%
  --(3.083,8.092)--(3.093,8.095)--(3.103,8.097)--(3.113,8.100)--(3.124,8.102)--(3.134,8.105)%
  --(3.144,8.100)--(3.154,8.103)--(3.164,8.105)--(3.174,8.108)--(3.185,8.111)--(3.195,8.113)%
  --(3.205,8.116)--(3.215,8.111)--(3.225,8.114)--(3.235,8.116)--(3.246,8.119)--(3.256,8.121)%
  --(3.266,8.124)--(3.276,8.127)--(3.286,8.121)--(3.296,8.124)--(3.307,8.127)--(3.317,8.129)%
  --(3.327,8.132)--(3.337,8.135)--(3.347,8.137)--(3.357,8.132)--(3.367,8.134)--(3.378,8.137)%
  --(3.388,8.140)--(3.398,8.143)--(3.408,8.145)--(3.418,8.148)--(3.428,8.142)--(3.439,8.145)%
  --(3.449,8.147)--(3.459,8.150)--(3.469,8.153)--(3.479,8.155)--(3.489,8.158)--(3.500,8.152)%
  --(3.510,8.155)--(3.520,8.157)--(3.530,8.160)--(3.540,8.163)--(3.550,8.165)--(3.561,8.168)%
  --(3.571,8.161)--(3.581,8.164)--(3.591,8.167)--(3.601,8.170)--(3.611,8.172)--(3.622,8.175)%
  --(3.632,8.168)--(3.642,8.171)--(3.652,8.173)--(3.662,8.176)--(3.672,8.179)--(3.682,8.181)%
  --(3.693,8.184)--(3.703,8.177)--(3.713,8.180)--(3.723,8.182)--(3.733,8.185)--(3.743,8.188)%
  --(3.754,8.190)--(3.764,8.183)--(3.774,8.186)--(3.784,8.188)--(3.794,8.191)--(3.804,8.193)%
  --(3.815,8.196)--(3.825,8.199)--(3.835,8.191)--(3.845,8.194)--(3.855,8.196)--(3.865,8.199)%
  --(3.876,8.201)--(3.886,8.204)--(3.896,8.196)--(3.906,8.198)--(3.916,8.201)--(3.926,8.204)%
  --(3.937,8.206)--(3.947,8.209)--(3.957,8.200)--(3.967,8.203)--(3.977,8.205)--(3.987,8.208)%
  --(3.998,8.211)--(4.008,8.213)--(4.018,8.216)--(4.028,8.207)--(4.038,8.209)--(4.048,8.212)%
  --(4.058,8.214)--(4.069,8.217)--(4.079,8.219)--(4.089,8.210)--(4.099,8.213)--(4.109,8.215)%
  --(4.119,8.218)--(4.130,8.220)--(4.140,8.222)--(4.150,8.213)--(4.160,8.216)--(4.170,8.218)%
  --(4.180,8.220)--(4.191,8.223)--(4.201,8.225)--(4.211,8.215)--(4.221,8.218)--(4.231,8.220)%
  --(4.241,8.222)--(4.252,8.225)--(4.262,8.227)--(4.272,8.217)--(4.282,8.219)--(4.292,8.222)%
  --(4.302,8.224)--(4.313,8.226)--(4.323,8.228)--(4.333,8.218)--(4.343,8.220)--(4.353,8.222)%
  --(4.363,8.225)--(4.373,8.227)--(4.384,8.229)--(4.394,8.218)--(4.404,8.220)--(4.414,8.223)%
  --(4.424,8.225)--(4.434,8.227)--(4.445,8.229)--(4.455,8.231)--(4.465,8.220)--(4.475,8.222)%
  --(4.485,8.224)--(4.495,8.226)--(4.506,8.228)--(4.516,8.230)--(4.526,8.219)--(4.536,8.221)%
  --(4.546,8.223)--(4.556,8.225)--(4.567,8.227)--(4.577,8.228)--(4.587,8.217)--(4.597,8.219)%
  --(4.607,8.221)--(4.617,8.222)--(4.628,8.224)--(4.638,8.226)--(4.648,8.214)--(4.658,8.216)%
  --(4.668,8.218)--(4.678,8.219)--(4.689,8.221)--(4.699,8.223)--(4.709,8.210)--(4.719,8.212)%
  --(4.729,8.214)--(4.739,8.215)--(4.749,8.217)--(4.760,8.219)--(4.770,8.206)--(4.780,8.208)%
  --(4.790,8.209)--(4.800,8.211)--(4.810,8.212)--(4.821,8.214)--(4.831,8.201)--(4.841,8.202)%
  --(4.851,8.204)--(4.861,8.205)--(4.871,8.206)--(4.882,8.208)--(4.892,8.194)--(4.902,8.196)%
  --(4.912,8.197)--(4.922,8.198)--(4.932,8.200)--(4.943,8.201)--(4.953,8.187)--(4.963,8.189)%
  --(4.973,8.190)--(4.983,8.191)--(4.993,8.192)--(5.004,8.193)--(5.014,8.179)--(5.024,8.180)%
  --(5.034,8.181)--(5.044,8.182)--(5.054,8.183)--(5.064,8.185)--(5.075,8.186)--(5.085,8.171)%
  --(5.095,8.172)--(5.105,8.173)--(5.115,8.174)--(5.125,8.175)--(5.136,8.176)--(5.146,8.161)%
  --(5.156,8.162)--(5.166,8.162)--(5.176,8.163)--(5.186,8.164)--(5.197,8.165)--(5.207,8.149)%
  --(5.217,8.150)--(5.227,8.151)--(5.237,8.151)--(5.247,8.152)--(5.258,8.153)--(5.268,8.137)%
  --(5.278,8.138)--(5.288,8.138)--(5.298,8.139)--(5.308,8.139)--(5.319,8.140)--(5.329,8.140)%
  --(5.339,8.124)--(5.349,8.124)--(5.359,8.125)--(5.369,8.125)--(5.379,8.125)--(5.390,8.126)%
  --(5.400,8.109)--(5.410,8.109)--(5.420,8.109)--(5.430,8.110)--(5.440,8.110)--(5.451,8.110)%
  --(5.461,8.110)--(5.471,8.093)--(5.481,8.093)--(5.491,8.093)--(5.501,8.093)--(5.512,8.093)%
  --(5.522,8.093)--(5.532,8.076)--(5.542,8.076)--(5.552,8.076)--(5.562,8.075)--(5.573,8.075)%
  --(5.583,8.075)--(5.593,8.075)--(5.603,8.057)--(5.613,8.057)--(5.623,8.056)--(5.634,8.056)%
  --(5.644,8.055)--(5.654,8.055)--(5.664,8.037)--(5.674,8.036)--(5.684,8.036)--(5.695,8.035)%
  --(5.705,8.035)--(5.715,8.034)--(5.725,8.034)--(5.735,8.015)--(5.745,8.014)--(5.755,8.013)%
  --(5.766,8.013)--(5.776,8.012)--(5.786,8.011)--(5.796,8.010)--(5.806,7.991)--(5.816,7.990)%
  --(5.827,7.989)--(5.837,7.988)--(5.847,7.987)--(5.857,7.986)--(5.867,7.985)--(5.877,7.966)%
  --(5.888,7.964)--(5.898,7.963)--(5.908,7.962)--(5.918,7.961)--(5.928,7.960)--(5.938,7.958)%
  --(5.949,7.938)--(5.959,7.937)--(5.969,7.936)--(5.979,7.934)--(5.989,7.933)--(5.999,7.931)%
  --(6.010,7.930)--(6.020,7.909)--(6.030,7.907)--(6.040,7.906)--(6.050,7.904)--(6.060,7.902)%
  --(6.070,7.896)--(6.081,7.894)--(6.091,7.892)--(6.101,7.881)--(6.111,7.879)--(6.121,7.877)%
  --(6.131,7.875)--(6.142,7.864)--(6.152,7.862)--(6.162,7.860)--(6.172,7.857)--(6.182,7.846)%
  --(6.192,7.843)--(6.203,7.841)--(6.213,7.829)--(6.223,7.827)--(6.233,7.825)--(6.243,7.822)%
  --(6.253,7.810)--(6.264,7.808)--(6.274,7.805)--(6.284,7.803)--(6.294,7.800)--(6.304,7.788)%
  --(6.314,7.785)--(6.325,7.782)--(6.335,7.780)--(6.345,7.767)--(6.355,7.764)--(6.365,7.761)%
  --(6.375,7.759)--(6.386,7.746)--(6.396,7.743)--(6.406,7.740)--(6.416,7.737)--(6.426,7.724)%
  --(6.436,7.721)--(6.446,7.718)--(6.457,7.714)--(6.467,7.701)--(6.477,7.698)--(6.487,7.695)%
  --(6.497,7.691)--(6.507,7.688)--(6.518,7.674)--(6.528,7.671)--(6.538,7.667)--(6.548,7.664)%
  --(6.558,7.660)--(6.568,7.646)--(6.579,7.643)--(6.589,7.639)--(6.599,7.635)--(6.609,7.621)%
  --(6.619,7.617)--(6.629,7.614)--(6.640,7.610)--(6.650,7.606)--(6.660,7.591)--(6.670,7.587)%
  --(6.680,7.583)--(6.690,7.579)--(6.701,7.575)--(6.711,7.560)--(6.721,7.556)--(6.731,7.552)%
  --(6.741,7.548)--(6.751,7.543)--(6.761,7.539)--(6.772,7.524)--(6.782,7.519)--(6.792,7.515)%
  --(6.802,7.510)--(6.812,7.506)--(6.822,7.490)--(6.833,7.486)--(6.843,7.481)--(6.853,7.476)%
  --(6.863,7.471)--(6.873,7.466)--(6.883,7.451)--(6.894,7.446)--(6.904,7.441)--(6.914,7.436)%
  --(6.924,7.431)--(6.934,7.425)--(6.944,7.409)--(6.955,7.404)--(6.965,7.399)--(6.975,7.393)%
  --(6.985,7.388)--(6.995,7.383)--(7.005,7.366)--(7.016,7.361)--(7.026,7.355)--(7.036,7.349)%
  --(7.046,7.344)--(7.056,7.338)--(7.066,7.332)--(7.077,7.315)--(7.087,7.310)--(7.097,7.304)%
  --(7.107,7.298)--(7.117,7.292)--(7.127,7.286)--(7.137,7.280)--(7.148,7.274)--(7.158,7.256)%
  --(7.168,7.250)--(7.178,7.244)--(7.188,7.237)--(7.198,7.231)--(7.209,7.225)--(7.219,7.218)%
  --(7.229,7.212)--(7.239,7.205)--(7.249,7.187)--(7.259,7.180)--(7.270,7.173)--(7.280,7.167)%
  --(7.290,7.160)--(7.300,7.153)--(7.310,7.146)--(7.320,7.139)--(7.331,7.132)--(7.341,7.125)%
  --(7.351,7.106)--(7.361,7.099)--(7.371,7.092)--(7.381,7.085)--(7.392,7.077)--(7.402,7.070)%
  --(7.412,7.062)--(7.422,7.055)--(7.432,7.047)--(7.442,7.040)--(7.452,7.032)--(7.463,7.025)%
  --(7.473,7.017)--(7.483,7.009)--(7.493,6.989)--(7.503,6.981)--(7.513,6.973)--(7.524,6.965)%
  --(7.534,6.957)--(7.544,6.949)--(7.554,6.941)--(7.564,6.933)--(7.574,6.924)--(7.585,6.916)%
  --(7.595,6.908)--(7.605,6.899)--(7.615,6.891)--(7.625,6.882)--(7.635,6.873)--(7.646,6.865)%
  --(7.656,6.856)--(7.666,6.847)--(7.676,6.838)--(7.686,6.829)--(7.696,6.820)--(7.707,6.811)%
  --(7.717,6.802)--(7.727,6.793)--(7.737,6.784)--(7.747,6.775)--(7.757,6.765)--(7.767,6.756)%
  --(7.778,6.747)--(7.788,6.737)--(7.798,6.728)--(7.808,6.718)--(7.818,6.708)--(7.828,6.699)%
  --(7.839,6.689)--(7.849,6.679)--(7.859,6.669)--(7.869,6.659)--(7.879,6.649)--(7.889,6.639)%
  --(7.900,6.629)--(7.910,6.618)--(7.920,6.608)--(7.930,6.598)--(7.940,6.587)--(7.950,6.577)%
  --(7.961,6.566)--(7.971,6.568)--(7.981,6.558)--(7.991,6.547)--(8.001,6.536)--(8.011,6.525)%
  --(8.022,6.514)--(8.032,6.504)--(8.042,6.493)--(8.052,6.481)--(8.062,6.470)--(8.072,6.459)%
  --(8.083,6.448)--(8.093,6.436)--(8.103,6.438)--(8.113,6.426)--(8.123,6.415)--(8.133,6.403)%
  --(8.143,6.391)--(8.154,6.380)--(8.164,6.368)--(8.174,6.356)--(8.184,6.357)--(8.194,6.345)%
  --(8.204,6.333)--(8.215,6.321)--(8.225,6.308)--(8.235,6.296)--(8.245,6.284)--(8.255,6.271)%
  --(8.265,6.272)--(8.276,6.259)--(8.286,6.247)--(8.296,6.234)--(8.306,6.221)--(8.316,6.208)%
  --(8.326,6.208)--(8.337,6.195)--(8.347,6.182)--(8.357,6.169)--(8.367,6.156)--(8.377,6.156)%
  --(8.387,6.142)--(8.398,6.129)--(8.408,6.116)--(8.418,6.102)--(8.428,6.101)--(8.438,6.088)%
  --(8.448,6.074)--(8.458,6.060)--(8.469,6.046)--(8.479,6.045)--(8.489,6.031)--(8.499,6.017)%
  --(8.509,6.003)--(8.519,6.002)--(8.530,5.988)--(8.540,5.973)--(8.550,5.959)--(8.560,5.957)%
  --(8.570,5.943)--(8.580,5.928)--(8.591,5.913)--(8.601,5.912)--(8.611,5.897)--(8.621,5.882)%
  --(8.631,5.867)--(8.641,5.865)--(8.652,5.850)--(8.662,5.834)--(8.672,5.832)--(8.682,5.817)%
  --(8.692,5.801)--(8.702,5.799)--(8.713,5.783)--(8.723,5.767)--(8.733,5.765)--(8.743,5.749)%
  --(8.753,5.733)--(8.763,5.730)--(8.774,5.714)--(8.784,5.698)--(8.794,5.695)--(8.804,5.678)%
  --(8.814,5.662)--(8.824,5.659)--(8.834,5.642)--(8.845,5.625)--(8.855,5.622)--(8.865,5.605)%
  --(8.875,5.601)--(8.885,5.584)--(8.895,5.567)--(8.906,5.563)--(8.916,5.546)--(8.926,5.542)%
  --(8.936,5.524)--(8.946,5.507)--(8.956,5.503)--(8.967,5.485)--(8.977,5.481)--(8.987,5.463)%
  --(8.997,5.458)--(9.007,5.440)--(9.017,5.422)--(9.028,5.417)--(9.038,5.399)--(9.048,5.394)%
  --(9.058,5.375)--(9.068,5.370)--(9.078,5.352)--(9.089,5.346)--(9.099,5.327)--(9.109,5.322)%
  --(9.119,5.303)--(9.129,5.297)--(9.139,5.278)--(9.149,5.272)--(9.160,5.253)--(9.170,5.247)%
  --(9.180,5.241)--(9.190,5.221)--(9.200,5.215)--(9.210,5.195)--(9.221,5.189)--(9.231,5.169)%
  --(9.241,5.162)--(9.251,5.142)--(9.261,5.135)--(9.271,5.128)--(9.282,5.108)--(9.292,5.100)%
  --(9.302,5.080)--(9.312,5.073)--(9.322,5.065)--(9.332,5.044)--(9.343,5.037)--(9.353,5.029)%
  --(9.363,5.008)--(9.373,5.000)--(9.383,4.992)--(9.393,4.970)--(9.404,4.962)--(9.414,4.954)%
  --(9.424,4.932)--(9.434,4.924)--(9.444,4.915)--(9.454,4.893)--(9.465,4.884)--(9.475,4.876)%
  --(9.485,4.867)--(9.495,4.844)--(9.505,4.835)--(9.515,4.826)--(9.525,4.816)--(9.536,4.793)%
  --(9.546,4.784)--(9.556,4.774)--(9.566,4.764)--(9.576,4.741)--(9.586,4.731)--(9.597,4.721)%
  --(9.607,4.711)--(9.617,4.700)--(9.627,4.677)--(9.637,4.666)--(9.647,4.655)--(9.658,4.645)%
  --(9.668,4.634)--(9.678,4.623)--(9.688,4.612)--(9.698,4.587)--(9.708,4.576)--(9.719,4.564)%
  --(9.729,4.553)--(9.739,4.541)--(9.749,4.529)--(9.759,4.517)--(9.769,4.505)--(9.780,4.493)%
  --(9.790,4.481)--(9.800,4.469)--(9.810,4.456)--(9.820,4.444)--(9.830,4.431)--(9.840,4.418)%
  --(9.851,4.405)--(9.861,4.392)--(9.871,4.379)--(9.881,4.365)--(9.891,4.352)--(9.901,4.338)%
  --(9.912,4.325)--(9.922,4.311)--(9.932,4.297)--(9.942,4.283)--(9.952,4.268)--(9.962,4.254)%
  --(9.973,4.253)--(9.983,4.238)--(9.993,4.223)--(10.003,4.209)--(10.013,4.194)--(10.023,4.178)%
  --(10.034,4.163)--(10.044,4.161)--(10.054,4.145)--(10.064,4.130)--(10.074,4.114)--(10.084,4.111)%
  --(10.095,4.095)--(10.105,4.079)--(10.115,4.062)--(10.125,4.059)--(10.135,4.042)--(10.145,4.025)%
  --(10.156,4.008)--(10.166,4.004)--(10.176,3.987)--(10.186,3.969)--(10.196,3.965)--(10.206,3.947)%
  --(10.216,3.942)--(10.227,3.924)--(10.237,3.906)--(10.247,3.901)--(10.257,3.882)--(10.267,3.877)%
  --(10.277,3.858)--(10.288,3.852)--(10.298,3.833)--(10.308,3.827)--(10.318,3.807)--(10.328,3.800)%
  --(10.338,3.781)--(10.349,3.774)--(10.359,3.754)--(10.369,3.746)--(10.379,3.726)--(10.389,3.718)%
  --(10.399,3.710)--(10.410,3.689)--(10.420,3.681)--(10.430,3.660)--(10.440,3.651)--(10.450,3.642)%
  --(10.460,3.633)--(10.471,3.611)--(10.481,3.602)--(10.491,3.592)--(10.501,3.582)--(10.511,3.559)%
  --(10.521,3.549)--(10.531,3.539)--(10.542,3.528)--(10.552,3.517)--(10.562,3.493)--(10.572,3.482)%
  --(10.582,3.471)--(10.592,3.459)--(10.603,3.447)--(10.613,3.435)--(10.623,3.423)--(10.633,3.410)%
  --(10.643,3.397)--(10.653,3.384)--(10.664,3.371)--(10.674,3.358)--(10.684,3.344)--(10.694,3.333)%
  --(10.704,3.319)--(10.714,3.311)--(10.725,3.296)--(10.735,3.281)--(10.745,3.272)--(10.755,3.257)%
  --(10.765,3.247)--(10.775,3.231)--(10.786,3.221)--(10.796,3.211)--(10.806,3.195)--(10.816,3.184)%
  --(10.826,3.173)--(10.836,3.161)--(10.846,3.144)--(10.857,3.132)--(10.867,3.119)--(10.877,3.107)%
  --(10.887,3.094)--(10.897,3.087)--(10.907,3.074)--(10.918,3.060)--(10.928,3.046)--(10.938,3.038)%
  --(10.948,3.023)--(10.958,3.008)--(10.968,2.999)--(10.979,2.983)--(10.989,2.973)--(10.999,2.963)%
  --(11.009,2.947)--(11.019,2.935)--(11.029,2.924)--(11.040,2.912)--(11.050,2.900)--(11.060,2.888)%
  --(11.070,2.875)--(11.080,2.862)--(11.090,2.854)--(11.101,2.840)--(11.111,2.826)--(11.121,2.816)%
  --(11.131,2.807)--(11.141,2.791)--(11.151,2.781)--(11.162,2.770)--(11.172,2.759)--(11.182,2.742)%
  --(11.192,2.735)--(11.202,2.723)--(11.212,2.710)--(11.222,2.696)--(11.233,2.688)--(11.243,2.673)%
  --(11.253,2.664)--(11.263,2.649)--(11.273,2.638)--(11.283,2.627)--(11.294,2.615)--(11.304,2.603)%
  --(11.314,2.591)--(11.324,2.582)--(11.334,2.569)--(11.344,2.559)--(11.355,2.545)--(11.365,2.534)%
  --(11.375,2.523)--(11.385,2.512)--(11.395,2.499)--(11.405,2.486)--(11.416,2.478)--(11.426,2.468)%
  --(11.436,2.453)--(11.446,2.443)--(11.456,2.431)--(11.466,2.419)--(11.477,2.411)--(11.487,2.397)%
  --(11.497,2.387)--(11.507,2.376)--(11.517,2.365)--(11.527,2.357)--(11.537,2.343)--(11.548,2.334)%
  --(11.558,2.323)--(11.568,2.311)--(11.578,2.299)--(11.588,2.289)--(11.598,2.279)--(11.609,2.268)%
  --(11.619,2.255)--(11.629,2.245)--(11.639,2.235)--(11.649,2.223)--(11.659,2.213)--(11.670,2.203)%
  --(11.680,2.191)--(11.690,2.181)--(11.700,2.170)--(11.710,2.161)--(11.720,2.151)--(11.731,2.139)%
  --(11.741,2.128)--(11.751,2.119)--(11.761,2.109)--(11.771,2.099)--(11.781,2.088)--(11.792,2.077)%
  --(11.802,2.067)--(11.812,2.058)--(11.822,2.049)--(11.832,2.037)--(11.842,2.027)--(11.853,2.018)%
  --(11.863,2.008)--(11.873,1.997)--(11.883,1.988)--(11.893,1.978)--(11.903,1.969)--(11.913,1.958)%
  --(11.924,1.948)--(11.934,1.939)--(11.944,1.929)--(11.954,1.920)--(11.964,1.910);
\gpcolor{color=gp lt color border}
\draw[gp path] (1.504,8.631)--(1.504,0.985)--(13.447,0.985)--(13.447,8.631)--cycle;
%% coordinates of the plot area
\gpdefrectangularnode{gp plot 1}{\pgfpoint{1.504cm}{0.985cm}}{\pgfpoint{13.447cm}{8.631cm}}
\end{tikzpicture}
%% gnuplot variables

	\caption{Gráfico do potencial grand canônico por unidade de volume obtido através da Equação~\eqref{Eq:potencial_termodinamico}.}
	\label{Fig:thermodynamic_potential_graph}
\end{figure*}

\begin{figure*}
	\begin{tikzpicture}[gnuplot]
%% generated with GNUPLOT 5.0p2 (Lua 5.2; terminal rev. 99, script rev. 100)
%% Mon Feb 29 17:01:32 2016
\path (0.000,0.000) rectangle (14.000,9.000);
\gpcolor{color=gp lt color border}
\gpsetlinetype{gp lt border}
\gpsetdashtype{gp dt solid}
\gpsetlinewidth{1.00}
\draw[gp path] (1.320,0.985)--(1.500,0.985);
\draw[gp path] (13.447,0.985)--(13.267,0.985);
\node[gp node right] at (1.136,0.985) {$275$};
\draw[gp path] (1.320,2.897)--(1.500,2.897);
\draw[gp path] (13.447,2.897)--(13.267,2.897);
\node[gp node right] at (1.136,2.897) {$280$};
\draw[gp path] (1.320,4.808)--(1.500,4.808);
\draw[gp path] (13.447,4.808)--(13.267,4.808);
\node[gp node right] at (1.136,4.808) {$285$};
\draw[gp path] (1.320,6.720)--(1.500,6.720);
\draw[gp path] (13.447,6.720)--(13.267,6.720);
\node[gp node right] at (1.136,6.720) {$290$};
\draw[gp path] (1.320,8.631)--(1.500,8.631);
\draw[gp path] (13.447,8.631)--(13.267,8.631);
\node[gp node right] at (1.136,8.631) {$295$};
\draw[gp path] (1.320,0.985)--(1.320,1.165);
\draw[gp path] (1.320,8.631)--(1.320,8.451);
\node[gp node center] at (1.320,0.677) {$0$};
\draw[gp path] (2.836,0.985)--(2.836,1.165);
\draw[gp path] (2.836,8.631)--(2.836,8.451);
\node[gp node center] at (2.836,0.677) {$0.05$};
\draw[gp path] (4.352,0.985)--(4.352,1.165);
\draw[gp path] (4.352,8.631)--(4.352,8.451);
\node[gp node center] at (4.352,0.677) {$0.1$};
\draw[gp path] (5.868,0.985)--(5.868,1.165);
\draw[gp path] (5.868,8.631)--(5.868,8.451);
\node[gp node center] at (5.868,0.677) {$0.15$};
\draw[gp path] (7.384,0.985)--(7.384,1.165);
\draw[gp path] (7.384,8.631)--(7.384,8.451);
\node[gp node center] at (7.384,0.677) {$0.2$};
\draw[gp path] (8.899,0.985)--(8.899,1.165);
\draw[gp path] (8.899,8.631)--(8.899,8.451);
\node[gp node center] at (8.899,0.677) {$0.25$};
\draw[gp path] (10.415,0.985)--(10.415,1.165);
\draw[gp path] (10.415,8.631)--(10.415,8.451);
\node[gp node center] at (10.415,0.677) {$0.3$};
\draw[gp path] (11.931,0.985)--(11.931,1.165);
\draw[gp path] (11.931,8.631)--(11.931,8.451);
\node[gp node center] at (11.931,0.677) {$0.35$};
\draw[gp path] (13.447,0.985)--(13.447,1.165);
\draw[gp path] (13.447,8.631)--(13.447,8.451);
\node[gp node center] at (13.447,0.677) {$0.4$};
\draw[gp path] (1.320,8.631)--(1.320,0.985)--(13.447,0.985)--(13.447,8.631)--cycle;
\node[gp node center,rotate=-270] at (0.246,4.808) {$P$ ($\rm{MeV}/\rm{fm}^3$)};
\node[gp node center] at (7.383,0.215) {$\rho$ ($\rm{fm}^{-3}$)};
\gpcolor{rgb color={0.580,0.000,0.827}}
\draw[gp path] (1.633,1.687)--(1.644,1.686)--(1.654,1.685)--(1.664,1.685)--(1.675,1.684)%
  --(1.685,1.683)--(1.695,1.682)--(1.706,1.682)--(1.716,1.681)--(1.726,1.680)--(1.737,1.679)%
  --(1.747,1.678)--(1.757,1.677)--(1.768,1.676)--(1.778,1.677)--(1.788,1.676)--(1.799,1.675)%
  --(1.809,1.674)--(1.819,1.673)--(1.830,1.672)--(1.840,1.671)--(1.850,1.670)--(1.860,1.669)%
  --(1.871,1.668)--(1.881,1.667)--(1.891,1.665)--(1.902,1.667)--(1.912,1.665)--(1.922,1.664)%
  --(1.933,1.663)--(1.943,1.662)--(1.953,1.660)--(1.964,1.659)--(1.974,1.658)--(1.984,1.656)%
  --(1.995,1.655)--(2.005,1.653)--(2.015,1.655)--(2.026,1.654)--(2.036,1.652)--(2.046,1.651)%
  --(2.057,1.649)--(2.067,1.648)--(2.077,1.646)--(2.087,1.645)--(2.098,1.643)--(2.108,1.641)%
  --(2.118,1.640)--(2.129,1.642)--(2.139,1.640)--(2.149,1.638)--(2.160,1.637)--(2.170,1.635)%
  --(2.180,1.633)--(2.191,1.631)--(2.201,1.630)--(2.211,1.628)--(2.222,1.626)--(2.232,1.628)%
  --(2.242,1.626)--(2.253,1.625)--(2.263,1.623)--(2.273,1.621)--(2.284,1.619)--(2.294,1.617)%
  --(2.304,1.615)--(2.314,1.613)--(2.325,1.616)--(2.335,1.614)--(2.345,1.612)--(2.356,1.610)%
  --(2.366,1.608)--(2.376,1.606)--(2.387,1.604)--(2.397,1.602)--(2.407,1.600)--(2.418,1.602)%
  --(2.428,1.600)--(2.438,1.598)--(2.449,1.596)--(2.459,1.594)--(2.469,1.592)--(2.480,1.590)%
  --(2.490,1.587)--(2.500,1.591)--(2.511,1.588)--(2.521,1.586)--(2.531,1.584)--(2.542,1.582)%
  --(2.552,1.579)--(2.562,1.577)--(2.572,1.575)--(2.583,1.573)--(2.593,1.576)--(2.603,1.574)%
  --(2.614,1.571)--(2.624,1.569)--(2.634,1.567)--(2.645,1.564)--(2.655,1.562)--(2.665,1.560)%
  --(2.676,1.563)--(2.686,1.561)--(2.696,1.559)--(2.707,1.556)--(2.717,1.554)--(2.727,1.551)%
  --(2.738,1.549)--(2.748,1.546)--(2.758,1.550)--(2.769,1.548)--(2.779,1.545)--(2.789,1.543)%
  --(2.799,1.540)--(2.810,1.538)--(2.820,1.535)--(2.830,1.540)--(2.841,1.537)--(2.851,1.535)%
  --(2.861,1.532)--(2.872,1.530)--(2.882,1.527)--(2.892,1.524)--(2.903,1.522)--(2.913,1.526)%
  --(2.923,1.524)--(2.934,1.521)--(2.944,1.519)--(2.954,1.516)--(2.965,1.514)--(2.975,1.511)%
  --(2.985,1.516)--(2.996,1.513)--(3.006,1.511)--(3.016,1.508)--(3.026,1.505)--(3.037,1.503)%
  --(3.047,1.500)--(3.057,1.505)--(3.068,1.502)--(3.078,1.500)--(3.088,1.497)--(3.099,1.495)%
  --(3.109,1.492)--(3.119,1.489)--(3.130,1.495)--(3.140,1.492)--(3.150,1.489)--(3.161,1.487)%
  --(3.171,1.484)--(3.181,1.481)--(3.192,1.479)--(3.202,1.484)--(3.212,1.482)--(3.223,1.479)%
  --(3.233,1.476)--(3.243,1.473)--(3.253,1.471)--(3.264,1.468)--(3.274,1.474)--(3.284,1.471)%
  --(3.295,1.469)--(3.305,1.466)--(3.315,1.463)--(3.326,1.461)--(3.336,1.458)--(3.346,1.464)%
  --(3.357,1.461)--(3.367,1.459)--(3.377,1.456)--(3.388,1.453)--(3.398,1.451)--(3.408,1.448)%
  --(3.419,1.455)--(3.429,1.452)--(3.439,1.449)--(3.450,1.446)--(3.460,1.444)--(3.470,1.441)%
  --(3.480,1.448)--(3.491,1.445)--(3.501,1.443)--(3.511,1.440)--(3.522,1.437)--(3.532,1.435)%
  --(3.542,1.432)--(3.553,1.439)--(3.563,1.436)--(3.573,1.434)--(3.584,1.431)--(3.594,1.428)%
  --(3.604,1.426)--(3.615,1.433)--(3.625,1.430)--(3.635,1.428)--(3.646,1.425)--(3.656,1.423)%
  --(3.666,1.420)--(3.677,1.417)--(3.687,1.425)--(3.697,1.422)--(3.707,1.420)--(3.718,1.417)%
  --(3.728,1.415)--(3.738,1.412)--(3.749,1.420)--(3.759,1.418)--(3.769,1.415)--(3.780,1.412)%
  --(3.790,1.410)--(3.800,1.407)--(3.811,1.416)--(3.821,1.413)--(3.831,1.411)--(3.842,1.408)%
  --(3.852,1.405)--(3.862,1.403)--(3.873,1.400)--(3.883,1.409)--(3.893,1.407)--(3.904,1.404)%
  --(3.914,1.402)--(3.924,1.399)--(3.934,1.397)--(3.945,1.406)--(3.955,1.403)--(3.965,1.401)%
  --(3.976,1.398)--(3.986,1.396)--(3.996,1.394)--(4.007,1.403)--(4.017,1.400)--(4.027,1.398)%
  --(4.038,1.396)--(4.048,1.393)--(4.058,1.391)--(4.069,1.401)--(4.079,1.398)--(4.089,1.396)%
  --(4.100,1.394)--(4.110,1.391)--(4.120,1.389)--(4.131,1.399)--(4.141,1.397)--(4.151,1.394)%
  --(4.161,1.392)--(4.172,1.390)--(4.182,1.388)--(4.192,1.398)--(4.203,1.396)--(4.213,1.394)%
  --(4.223,1.391)--(4.234,1.389)--(4.244,1.387)--(4.254,1.398)--(4.265,1.396)--(4.275,1.393)%
  --(4.285,1.391)--(4.296,1.389)--(4.306,1.387)--(4.316,1.385)--(4.327,1.396)--(4.337,1.394)%
  --(4.347,1.392)--(4.358,1.390)--(4.368,1.388)--(4.378,1.386)--(4.388,1.397)--(4.399,1.395)%
  --(4.409,1.393)--(4.419,1.391)--(4.430,1.389)--(4.440,1.388)--(4.450,1.399)--(4.461,1.397)%
  --(4.471,1.395)--(4.481,1.394)--(4.492,1.392)--(4.502,1.390)--(4.512,1.402)--(4.523,1.400)%
  --(4.533,1.398)--(4.543,1.397)--(4.554,1.395)--(4.564,1.393)--(4.574,1.406)--(4.585,1.404)%
  --(4.595,1.402)--(4.605,1.401)--(4.615,1.399)--(4.626,1.397)--(4.636,1.410)--(4.646,1.408)%
  --(4.657,1.407)--(4.667,1.405)--(4.677,1.404)--(4.688,1.402)--(4.698,1.415)--(4.708,1.414)%
  --(4.719,1.412)--(4.729,1.411)--(4.739,1.410)--(4.750,1.408)--(4.760,1.422)--(4.770,1.420)%
  --(4.781,1.419)--(4.791,1.418)--(4.801,1.416)--(4.812,1.415)--(4.822,1.429)--(4.832,1.427)%
  --(4.842,1.426)--(4.853,1.425)--(4.863,1.424)--(4.873,1.423)--(4.884,1.437)--(4.894,1.436)%
  --(4.904,1.435)--(4.915,1.434)--(4.925,1.433)--(4.935,1.431)--(4.946,1.430)--(4.956,1.445)%
  --(4.966,1.444)--(4.977,1.443)--(4.987,1.442)--(4.997,1.441)--(5.008,1.440)--(5.018,1.455)%
  --(5.028,1.454)--(5.039,1.454)--(5.049,1.453)--(5.059,1.452)--(5.069,1.451)--(5.080,1.467)%
  --(5.090,1.466)--(5.100,1.465)--(5.111,1.465)--(5.121,1.464)--(5.131,1.463)--(5.142,1.479)%
  --(5.152,1.478)--(5.162,1.478)--(5.173,1.477)--(5.183,1.477)--(5.193,1.476)--(5.204,1.476)%
  --(5.214,1.492)--(5.224,1.492)--(5.235,1.491)--(5.245,1.491)--(5.255,1.491)--(5.266,1.490)%
  --(5.276,1.507)--(5.286,1.507)--(5.296,1.507)--(5.307,1.506)--(5.317,1.506)--(5.327,1.506)%
  --(5.338,1.506)--(5.348,1.523)--(5.358,1.523)--(5.369,1.523)--(5.379,1.523)--(5.389,1.523)%
  --(5.400,1.523)--(5.410,1.540)--(5.420,1.540)--(5.431,1.540)--(5.441,1.541)--(5.451,1.541)%
  --(5.462,1.541)--(5.472,1.541)--(5.482,1.559)--(5.493,1.559)--(5.503,1.560)--(5.513,1.560)%
  --(5.523,1.561)--(5.534,1.561)--(5.544,1.579)--(5.554,1.580)--(5.565,1.580)--(5.575,1.581)%
  --(5.585,1.581)--(5.596,1.582)--(5.606,1.582)--(5.616,1.601)--(5.627,1.602)--(5.637,1.603)%
  --(5.647,1.603)--(5.658,1.604)--(5.668,1.605)--(5.678,1.606)--(5.689,1.625)--(5.699,1.626)%
  --(5.709,1.627)--(5.720,1.628)--(5.730,1.629)--(5.740,1.630)--(5.750,1.631)--(5.761,1.650)%
  --(5.771,1.652)--(5.781,1.653)--(5.792,1.654)--(5.802,1.655)--(5.812,1.656)--(5.823,1.658)%
  --(5.833,1.678)--(5.843,1.679)--(5.854,1.680)--(5.864,1.682)--(5.874,1.683)--(5.885,1.685)%
  --(5.895,1.686)--(5.905,1.707)--(5.916,1.709)--(5.926,1.710)--(5.936,1.712)--(5.947,1.714)%
  --(5.957,1.720)--(5.967,1.722)--(5.977,1.724)--(5.988,1.735)--(5.998,1.737)--(6.008,1.739)%
  --(6.019,1.741)--(6.029,1.752)--(6.039,1.754)--(6.050,1.756)--(6.060,1.759)--(6.070,1.770)%
  --(6.081,1.773)--(6.091,1.775)--(6.101,1.787)--(6.112,1.789)--(6.122,1.791)--(6.132,1.794)%
  --(6.143,1.806)--(6.153,1.808)--(6.163,1.811)--(6.174,1.813)--(6.184,1.816)--(6.194,1.828)%
  --(6.204,1.831)--(6.215,1.834)--(6.225,1.836)--(6.235,1.849)--(6.246,1.852)--(6.256,1.855)%
  --(6.266,1.857)--(6.277,1.870)--(6.287,1.873)--(6.297,1.876)--(6.308,1.879)--(6.318,1.892)%
  --(6.328,1.895)--(6.339,1.898)--(6.349,1.902)--(6.359,1.915)--(6.370,1.918)--(6.380,1.921)%
  --(6.390,1.925)--(6.401,1.928)--(6.411,1.942)--(6.421,1.945)--(6.431,1.949)--(6.442,1.952)%
  --(6.452,1.956)--(6.462,1.970)--(6.473,1.973)--(6.483,1.977)--(6.493,1.981)--(6.504,1.995)%
  --(6.514,1.999)--(6.524,2.002)--(6.535,2.006)--(6.545,2.010)--(6.555,2.025)--(6.566,2.029)%
  --(6.576,2.033)--(6.586,2.037)--(6.597,2.041)--(6.607,2.056)--(6.617,2.060)--(6.628,2.064)%
  --(6.638,2.068)--(6.648,2.073)--(6.658,2.077)--(6.669,2.092)--(6.679,2.097)--(6.689,2.101)%
  --(6.700,2.106)--(6.710,2.110)--(6.720,2.126)--(6.731,2.130)--(6.741,2.135)--(6.751,2.140)%
  --(6.762,2.145)--(6.772,2.150)--(6.782,2.165)--(6.793,2.170)--(6.803,2.175)--(6.813,2.180)%
  --(6.824,2.185)--(6.834,2.191)--(6.844,2.207)--(6.855,2.212)--(6.865,2.217)--(6.875,2.223)%
  --(6.885,2.228)--(6.896,2.233)--(6.906,2.250)--(6.916,2.255)--(6.927,2.261)--(6.937,2.267)%
  --(6.947,2.272)--(6.958,2.278)--(6.968,2.284)--(6.978,2.301)--(6.989,2.306)--(6.999,2.312)%
  --(7.009,2.318)--(7.020,2.324)--(7.030,2.330)--(7.040,2.336)--(7.051,2.342)--(7.061,2.360)%
  --(7.071,2.366)--(7.082,2.372)--(7.092,2.379)--(7.102,2.385)--(7.112,2.391)--(7.123,2.398)%
  --(7.133,2.404)--(7.143,2.411)--(7.154,2.429)--(7.164,2.436)--(7.174,2.443)--(7.185,2.449)%
  --(7.195,2.456)--(7.205,2.463)--(7.216,2.470)--(7.226,2.477)--(7.236,2.484)--(7.247,2.491)%
  --(7.257,2.510)--(7.267,2.517)--(7.278,2.524)--(7.288,2.531)--(7.298,2.539)--(7.309,2.546)%
  --(7.319,2.554)--(7.329,2.561)--(7.339,2.569)--(7.350,2.576)--(7.360,2.584)--(7.370,2.591)%
  --(7.381,2.599)--(7.391,2.607)--(7.401,2.627)--(7.412,2.635)--(7.422,2.643)--(7.432,2.651)%
  --(7.443,2.659)--(7.453,2.667)--(7.463,2.675)--(7.474,2.683)--(7.484,2.692)--(7.494,2.700)%
  --(7.505,2.708)--(7.515,2.717)--(7.525,2.725)--(7.536,2.734)--(7.546,2.743)--(7.556,2.751)%
  --(7.566,2.760)--(7.577,2.769)--(7.587,2.778)--(7.597,2.787)--(7.608,2.796)--(7.618,2.805)%
  --(7.628,2.814)--(7.639,2.823)--(7.649,2.832)--(7.659,2.841)--(7.670,2.851)--(7.680,2.860)%
  --(7.690,2.869)--(7.701,2.879)--(7.711,2.888)--(7.721,2.898)--(7.732,2.908)--(7.742,2.917)%
  --(7.752,2.927)--(7.763,2.937)--(7.773,2.947)--(7.783,2.957)--(7.793,2.967)--(7.804,2.977)%
  --(7.814,2.987)--(7.824,2.998)--(7.835,3.008)--(7.845,3.018)--(7.855,3.029)--(7.866,3.039)%
  --(7.876,3.050)--(7.886,3.048)--(7.897,3.058)--(7.907,3.069)--(7.917,3.080)--(7.928,3.091)%
  --(7.938,3.102)--(7.948,3.112)--(7.959,3.123)--(7.969,3.135)--(7.979,3.146)--(7.990,3.157)%
  --(8.000,3.168)--(8.010,3.180)--(8.021,3.178)--(8.031,3.190)--(8.041,3.201)--(8.051,3.213)%
  --(8.062,3.225)--(8.072,3.236)--(8.082,3.248)--(8.093,3.260)--(8.103,3.259)--(8.113,3.271)%
  --(8.124,3.283)--(8.134,3.295)--(8.144,3.308)--(8.155,3.320)--(8.165,3.332)--(8.175,3.345)%
  --(8.186,3.344)--(8.196,3.357)--(8.206,3.369)--(8.217,3.382)--(8.227,3.395)--(8.237,3.408)%
  --(8.248,3.408)--(8.258,3.421)--(8.268,3.434)--(8.278,3.447)--(8.289,3.460)--(8.299,3.460)%
  --(8.309,3.474)--(8.320,3.487)--(8.330,3.500)--(8.340,3.514)--(8.351,3.515)--(8.361,3.528)%
  --(8.371,3.542)--(8.382,3.556)--(8.392,3.570)--(8.402,3.571)--(8.413,3.585)--(8.423,3.599)%
  --(8.433,3.613)--(8.444,3.614)--(8.454,3.628)--(8.464,3.643)--(8.475,3.657)--(8.485,3.659)%
  --(8.495,3.673)--(8.505,3.688)--(8.516,3.703)--(8.526,3.704)--(8.536,3.719)--(8.547,3.734)%
  --(8.557,3.749)--(8.567,3.751)--(8.578,3.766)--(8.588,3.782)--(8.598,3.784)--(8.609,3.799)%
  --(8.619,3.815)--(8.629,3.817)--(8.640,3.833)--(8.650,3.849)--(8.660,3.851)--(8.671,3.867)%
  --(8.681,3.883)--(8.691,3.886)--(8.702,3.902)--(8.712,3.918)--(8.722,3.921)--(8.732,3.938)%
  --(8.743,3.954)--(8.753,3.957)--(8.763,3.974)--(8.774,3.991)--(8.784,3.994)--(8.794,4.011)%
  --(8.805,4.015)--(8.815,4.032)--(8.825,4.049)--(8.836,4.053)--(8.846,4.070)--(8.856,4.074)%
  --(8.867,4.092)--(8.877,4.109)--(8.887,4.113)--(8.898,4.131)--(8.908,4.135)--(8.918,4.153)%
  --(8.929,4.158)--(8.939,4.176)--(8.949,4.194)--(8.959,4.199)--(8.970,4.217)--(8.980,4.222)%
  --(8.990,4.241)--(9.001,4.246)--(9.011,4.264)--(9.021,4.270)--(9.032,4.289)--(9.042,4.294)%
  --(9.052,4.313)--(9.063,4.319)--(9.073,4.338)--(9.083,4.344)--(9.094,4.363)--(9.104,4.369)%
  --(9.114,4.375)--(9.125,4.395)--(9.135,4.401)--(9.145,4.421)--(9.156,4.427)--(9.166,4.447)%
  --(9.176,4.454)--(9.186,4.474)--(9.197,4.481)--(9.207,4.488)--(9.217,4.508)--(9.228,4.516)%
  --(9.238,4.536)--(9.248,4.543)--(9.259,4.551)--(9.269,4.572)--(9.279,4.579)--(9.290,4.587)%
  --(9.300,4.608)--(9.310,4.616)--(9.321,4.624)--(9.331,4.646)--(9.341,4.654)--(9.352,4.662)%
  --(9.362,4.684)--(9.372,4.692)--(9.383,4.701)--(9.393,4.723)--(9.403,4.732)--(9.413,4.740)%
  --(9.424,4.749)--(9.434,4.772)--(9.444,4.781)--(9.455,4.790)--(9.465,4.800)--(9.475,4.823)%
  --(9.486,4.832)--(9.496,4.842)--(9.506,4.852)--(9.517,4.875)--(9.527,4.885)--(9.537,4.895)%
  --(9.548,4.905)--(9.558,4.916)--(9.568,4.939)--(9.579,4.950)--(9.589,4.961)--(9.599,4.971)%
  --(9.610,4.982)--(9.620,4.993)--(9.630,5.004)--(9.640,5.029)--(9.651,5.040)--(9.661,5.052)%
  --(9.671,5.063)--(9.682,5.075)--(9.692,5.087)--(9.702,5.099)--(9.713,5.111)--(9.723,5.123)%
  --(9.733,5.135)--(9.744,5.147)--(9.754,5.160)--(9.764,5.172)--(9.775,5.185)--(9.785,5.198)%
  --(9.795,5.211)--(9.806,5.224)--(9.816,5.237)--(9.826,5.251)--(9.837,5.264)--(9.847,5.278)%
  --(9.857,5.291)--(9.867,5.305)--(9.878,5.319)--(9.888,5.333)--(9.898,5.348)--(9.909,5.362)%
  --(9.919,5.363)--(9.929,5.378)--(9.940,5.393)--(9.950,5.407)--(9.960,5.422)--(9.971,5.438)%
  --(9.981,5.453)--(9.991,5.455)--(10.002,5.471)--(10.012,5.486)--(10.022,5.502)--(10.033,5.505)%
  --(10.043,5.521)--(10.053,5.537)--(10.064,5.554)--(10.074,5.557)--(10.084,5.574)--(10.094,5.591)%
  --(10.105,5.608)--(10.115,5.612)--(10.125,5.629)--(10.136,5.647)--(10.146,5.651)--(10.156,5.669)%
  --(10.167,5.674)--(10.177,5.692)--(10.187,5.710)--(10.198,5.715)--(10.208,5.734)--(10.218,5.739)%
  --(10.229,5.758)--(10.239,5.764)--(10.249,5.783)--(10.260,5.789)--(10.270,5.809)--(10.280,5.816)%
  --(10.291,5.835)--(10.301,5.842)--(10.311,5.862)--(10.321,5.870)--(10.332,5.890)--(10.342,5.898)%
  --(10.352,5.906)--(10.363,5.927)--(10.373,5.935)--(10.383,5.956)--(10.394,5.965)--(10.404,5.974)%
  --(10.414,5.983)--(10.425,6.005)--(10.435,6.014)--(10.445,6.024)--(10.456,6.034)--(10.466,6.057)%
  --(10.476,6.067)--(10.487,6.077)--(10.497,6.088)--(10.507,6.099)--(10.518,6.123)--(10.528,6.134)%
  --(10.538,6.145)--(10.548,6.157)--(10.559,6.169)--(10.569,6.181)--(10.579,6.193)--(10.590,6.206)%
  --(10.600,6.219)--(10.610,6.232)--(10.621,6.245)--(10.631,6.258)--(10.641,6.272)--(10.652,6.283)%
  --(10.662,6.297)--(10.672,6.305)--(10.683,6.320)--(10.693,6.335)--(10.703,6.344)--(10.714,6.359)%
  --(10.724,6.369)--(10.734,6.385)--(10.745,6.395)--(10.755,6.405)--(10.765,6.421)--(10.775,6.432)%
  --(10.786,6.443)--(10.796,6.455)--(10.806,6.472)--(10.817,6.484)--(10.827,6.497)--(10.837,6.509)%
  --(10.848,6.522)--(10.858,6.529)--(10.868,6.542)--(10.879,6.556)--(10.889,6.570)--(10.899,6.578)%
  --(10.910,6.593)--(10.920,6.608)--(10.930,6.617)--(10.941,6.633)--(10.951,6.643)--(10.961,6.653)%
  --(10.972,6.669)--(10.982,6.681)--(10.992,6.692)--(11.002,6.704)--(11.013,6.716)--(11.023,6.728)%
  --(11.033,6.741)--(11.044,6.754)--(11.054,6.762)--(11.064,6.776)--(11.075,6.790)--(11.085,6.800)%
  --(11.095,6.809)--(11.106,6.825)--(11.116,6.835)--(11.126,6.846)--(11.137,6.857)--(11.147,6.874)%
  --(11.157,6.881)--(11.168,6.893)--(11.178,6.906)--(11.188,6.920)--(11.199,6.928)--(11.209,6.943)%
  --(11.219,6.952)--(11.229,6.967)--(11.240,6.978)--(11.250,6.989)--(11.260,7.001)--(11.271,7.013)%
  --(11.281,7.025)--(11.291,7.034)--(11.302,7.047)--(11.312,7.057)--(11.322,7.071)--(11.333,7.082)%
  --(11.343,7.093)--(11.353,7.104)--(11.364,7.117)--(11.374,7.130)--(11.384,7.138)--(11.395,7.148)%
  --(11.405,7.163)--(11.415,7.173)--(11.426,7.185)--(11.436,7.197)--(11.446,7.205)--(11.456,7.219)%
  --(11.467,7.229)--(11.477,7.240)--(11.487,7.251)--(11.498,7.259)--(11.508,7.273)--(11.518,7.282)%
  --(11.529,7.293)--(11.539,7.305)--(11.549,7.317)--(11.560,7.327)--(11.570,7.337)--(11.580,7.348)%
  --(11.591,7.361)--(11.601,7.371)--(11.611,7.381)--(11.622,7.393)--(11.632,7.403)--(11.642,7.413)%
  --(11.653,7.425)--(11.663,7.435)--(11.673,7.446)--(11.683,7.455)--(11.694,7.465)--(11.704,7.477)%
  --(11.714,7.488)--(11.725,7.497)--(11.735,7.507)--(11.745,7.517)--(11.756,7.528)--(11.766,7.539)%
  --(11.776,7.549)--(11.787,7.558)--(11.797,7.567)--(11.807,7.579)--(11.818,7.589)--(11.828,7.598)%
  --(11.838,7.608)--(11.849,7.619)--(11.859,7.628)--(11.869,7.638)--(11.880,7.647)--(11.890,7.658)%
  --(11.900,7.668)--(11.910,7.677)--(11.921,7.687)--(11.931,7.696)--(11.941,7.706);
\gpcolor{color=gp lt color border}
\draw[gp path] (1.320,8.631)--(1.320,0.985)--(13.447,0.985)--(13.447,8.631)--cycle;
%% coordinates of the plot area
\gpdefrectangularnode{gp plot 1}{\pgfpoint{1.320cm}{0.985cm}}{\pgfpoint{13.447cm}{8.631cm}}
\end{tikzpicture}
%% gnuplot variables

	\caption{Gráfico da pressão obtido através da Equação~\eqref{Eq:Pressao}. Devido ao fato de que estamos usando $\varepsilon_o = 0$ por enquanto, a escala vertical do gráfico está deslocada no sentido positivo.}
	\label{Fig:pressure_graph}
\end{figure*}

