%%%%%%%%%%%%%%%%%%%%%%%%%%%%%%%
%%%%%%%%%%%%%%%%%%%%%%%%%%%%%%%
\chapter{Fase de Hadrons: eNJL}
%%%%%%%%%%%%%%%%%%%%%%%%%%%%%%%

%%%%%%%%%%%%%%%%%%%%%%%%%
\section{Simetria quiral}
%%%%%%%%%%%%%%%%%%%%%%%%%

Ver discussão em Ref.\cite{Vogl}
% Vogl, U., and W. Weise. - Progress in Particle and Nuclear Physics 27 (1991): 195-272 - The Nambu and Jona-Lasinio model: its implications for hadrons and nuclei

\begin{description}
	\item[Limite quiral] No limite quiral, a massa constituinte vai a zero:
		\begin{equation}
			m \to 0.
		\end{equation}
\end{description}

%%%%%%%%%%%%%%%%%%%%%%%
\section{Termodinâmica}
%%%%%%%%%%%%%%%%%%%%%%%

Temos que $dS$ ou $dU$ -- deve ser $dU$, pois o potencial químico é a ``quantidade de energia ganha ao se inserir mais uma partícula no sistema'' -- tem um termo $\mu dN$. Vamos trabalhar com densidade bariônica, mas essa densidade é só o número de partículas $N$ dividido pelo volume (as outras variáveis também serão trabalhadas divididas pelo volume). Logo, temos $\mu d\rho$. Por isso, temos que $\mu = d\epsilon/d\rho$, onde $\epsilon$ é a densidade de energia $\epsilon = E/V$. 

Glendenning\cite{Glendenning}:
Degenerate ideal Fermi gas: ideal pois não tem interações entre as partículas, degenerado pois todos os estados até uma certa energia -- a energia de Fermi -- estão ocupados. Nesse caso, a soma sobre todos os estados ocupados (que são autoestados de momento, pois não há interação) deve se dar sobre o momento. Isso pode ser escrito como a integral
\begin{equation}
	\int_0^{k_f} \frac{d^3k}{(2\pi^3)}.
\end{equation}
%
(pelo que lembro, é um cálculo realizado em um octante, contando quantos estados existem entre $p$ e $p+dp$ levando-se em conta que são ondas estacionárias em uma caixa de lado $L$. Nesse caso $k$ (que está associado ao momento) é um inteiro vezes o comprimento de onda dividido por dois. Tentar achar isso Ref [63] do Glendenning.

O gás pode ser considerado degenerado se $T \ll E_F = \sqrt{k_F^2 + m^2}$.

A densidade é obtida simplesmente somando os estados ocupados. A energia é calculada somando a energia de cada estado ocupado. A pressão eu não sei:
\begin{align}
	\rho &= \\
	\epsilon &= \\
	p &=
\end{align}

In thermodynamics, chemical potential, also known as partial molar free energy, is a form of potential energy that can be absorbed or released during a chemical reaction. It may also change during a phase transition. The chemical potential of a species in a mixture can be defined as the slope of the free energy of the system with respect to a change in the number of moles of just that species. Thus, it is the partial derivative of the free energy with respect to the amount of the species, all other species' concentrations in the mixture remaining constant, and at constant temperature. When pressure is constant, chemical potential is the partial molar Gibbs free energy. At chemical equilibrium or in phase equilibrium the total sum of chemical potentials is zero, as the free energy is at a minimum. \url{https://en.wikipedia.org/wiki/Chemical_potential}

%%%%%%%%%%%%%%%%%%%%%%%%%%%%%%%%%%%%%%%
\section{Modelo de Nambu--Jona-Lasinio}
%%%%%%%%%%%%%%%%%%%%%%%%%%%%%%%%%%%%%%%

Sobre a origem do modelo\cite{Buballa}:
\begin{quote}
Historically, it goes back to two papers by Nambu and Jona-Lasinio in 1961 [...] to a time when QCD and even quarks were still unknown. In its original version, the NJL model was therefore a model of interacting \emph{nucleons}, and obviously, confinement [\dots] was not an issue. On the other hand, even in the pre-QCD era there were already indications for the existence of a (partially) conserved axial vector current (PCAC), i.e., chiral symmetry. Since (approximate) chiral symmetry implies (almost) massless fermions on the Lagrangian level, the problem was to find a mechanism which explains the large nucleon mass without destroying the symmetry.

It was the pioneering idea of Nambu and Jona-Lasinio that the gap in the Dirac spectrum of the nucleon\footnote{\emph{the gap in the Dirac spectrum of the nucleon}, do que se trata?} can be generated quite analogously to the energy gap of a superconductor in BCS theory, which has been developed a few years earlier. To that end they introduced a Lagrangian for a nucleon field $\psi$ with a point-like, chirally symmetric four-fermion interaction\footnote{\emph{point-like}: acho que isto está ligado ao fato de que o termo que corresponderia a um potencial é simplesmente uma constante vezes uma coisa que depende do $\psi$, não algo que saia de uma outra equação que dependa do $\psi$, como a Eq. de Klein-Gordon no modelo de Walecka para o núcleo que eu trabalhei. \emph{chirally symmetric}: imagino que seja por que não há diferenciação entre as duas quiralidaes (helicidades para uma partícula sem massa). \emph{four-fermion interaction}: não tenho ideia de como é possivel ver isso nos termos.},
\begin{equation}\label{Eq:LagOriginalNJL}
	\mathcal{L} = \bar{\psi}(i\slashed{\partial} - m)\psi + G\{(\bar{\psi}i\gamma_5\vec{\tau}\psi)^2\}.
\end{equation}
\end{quote}
%
Na lagrangiana acima,
\begin{itemize}
	\item $m$ é a massa constituinte do nucleon, que deve ser pequena devido ao fato de que se não fosse, não haveria possibilidade de haver simetria quiral. Simetria quiral implica em massa pequena.
	\item $\vec{\tau}$ é a matriz de Pauli atuando no espaço de isospin.
	\item $G$ é uma constante de acoplamento que possui alguma dimensão (termos diferentes tem dimensões diferentes).
\end{itemize}

Ainda temos\cite{Buballa}
\begin{quote}
[T]he self-energy induced by the interaction generates an effective mass $M$ which can be considerably larger than $m$ and stays large, even when $m$ is taken to zero (``chiral limit''). At the same time, there are light collective nucleon-antinucleon excitations which become massless in the chiral limit: The pion emerges as the Goldstone boson of the spontaneously broken chiral symmetry. In fact, this discovery was an important milestone on the way to the general derivation of the Goldstone theorem in the same year.\footnote{\emph{large collective excitatations}: o que é isso? \emph{Goldstone boson}: o que é isso? \emph{spontaneously broken}: o que é isso?}

After the development of QCD, the NJL model was reinterpreted as a schematic quark model. At that point, of course, the lack of confinement became a problem\footnote{Se não ter confinamento é um problema para os quarks, pq temos que adicionar termos ao modelo para a fase de hadrons para poder descrevê-la adequadamente? Não deveria descrever direito pois foi para isso que o modelo foi criado? Ou adicionamos termos só para descrever adequadamente as propriedades da fase de hádrons (principalmente: qual é o papel do termo proporcional a $G_{sv}$?). Na fase de quarks, não tem importância o confinamento pois estamos a uma densidade muito alta, então eles desconfinam de qualquer forma. Devemos tratar as duas separadamente e depois descobrir onde fica a transição usando condições de Gibbs}, severely limiting the applicability of the model. On the other hand, there are many situations where chiral symmetry is the relevant feature of QCD, confinement being less important. The most prominent example is again the Goldstone nature of the pion. In this aspect the NJL model is superior to the MIT bag, which [\dots] fails to explain the low pion mass.\footnote{Qual é a relação entre ser o boson de Goldstone (?) e ter massa pequena? É por que de alguma forma ele está ligado à simetria quiral?}

[\dots]

After the reinterpretation of the NJL model as a quark model, many authors kept the original form of the Lagrangian Eq.~\eqref{Eq:LagOriginalNJL}, with $\psi$ now being a quark field with two flavor and three color degrees of freedom. However, this choice is not unique and we can write down many other chirally symmetric interaction terms.
\end{quote}

%%%%%%%%%%%%%%%%%%%%%%%%%%%%%%%%%%%%%%%%%%%%%%%%%%%%%%%%%%%%%
\section{Artigo: Pais, Menezes, Providência; Fase de Hádrons}
%%%%%%%%%%%%%%%%%%%%%%%%%%%%%%%%%%%%%%%%%%%%%%%%%%%%%%%%%%%%%

Sobre o modelo\cite{Pais}:
\begin{quote}
The NJL model can be extended [...] to yield reasonable saturation properties of nuclear matter, the field $\psi$ being the nucleon field. An effective density dependent coupling constant is obtained if the following extended NJL (eNJL) Lagrangian density, which actually pushes chiral symmetry restoration to higher densities, is considered,
\begin{equation}\label{Eq:Lagrangiana_eNLJ_Pais}
\begin{split}
	\mathcal{L} =&~ \bar{\psi}(i\gamma^\mu\partial_\mu)\psi + G_s[(\bar{\psi}\psi)^2 + (\bar{\psi}i\gamma_5\vec{\tau}\psi)^2] \\
	& - G_v(\bar{\psi}\gamma^\mu\psi)^2 - G_{sv}[(\bar{\psi}\psi)^2 + (\bar{\psi}i\gamma_5\vec{\tau}\psi)^2](\bar{\psi}\gamma^\mu\psi)^2 \\
	& - G_\rho[(\bar{\psi}\gamma^\mu\vec{\tau}\psi)^2 + (\bar{\psi}\gamma_5\gamma^\mu\vec{\tau}\psi)^2] \\
	& - G_{v\rho}(\bar{\psi}\gamma^\mu\psi)^2[(\bar{\psi}\gamma^\mu\vec{\tau}\psi)^2 + (\bar{\psi}\gamma_5\gamma^\mu\vec{\tau}\psi)^2] \\
	& - G_{s\rho} [(\bar{\psi}\psi)^2 + (\bar{\psi}i\gamma_5\vec{\tau}\psi)^2][(\bar{\psi}\gamma^\mu\vec{\tau}\psi)^2 + (\bar{\psi}\gamma_5\gamma^\mu\vec{\tau}\psi)^2].
\end{split}
\end{equation}
\end{quote}

Outras informações relevantes (ainda do artigo):
\begin{itemize}
	\item Para a matéria nuclear, a degenerescência é $2 N_f$;
	\item O \emph{cutoff} $\Lambda$ é tal que a massa do nucleon no vácuo seja de 939 MeV, determinada variacionalmente;
	\item O termo proporcional a $G_v$ simula uma repulsão de curto alcance quiralmente invariante entre dois nucleons;
	\item O termo proporcional a $G_s$ simula uma repulsão de curto alcance entre os nucleons (chiral invariant)\footnote{Acredito que seja em $G_v$, pois é o que consta na última versão do artigo. De qq forma, se um é isso, o outro é o que?};
	\item ``The term in $G_{sv}$ accounts for the density dependence of the scalar coupling. For the nuclear matter, the NJL model leads to binding, but the binding energy per particle does no have a minimum except at a rather high density where the nucleon mass is small or vanishing. The introduction of the $G_{sv}$ coupling term is required to correct this.''
	\item O termo proporcional a $G_\rho$ (isovetor-vetor) é incluido para descrever a matéria nuclear assimétrica (em isospin);
	\item ``The terms $G_{v\rho}$ and $G_{s\rho}$ make the symmetry energy softer.''
\end{itemize}

A partir da lagrangiana \eqref{Eq:Lagrangiana_eNLJ_Pais}, é possível determinar\footnote{Como? Através de $\omega(T,\mu) = -\frac{T}{V} \ln \mathcal{Z}$?} o potencial termodinâmico\footnote{Potencial Grand-canônico, ou potencial de Landau.} por unidade de volume, dado por
\begin{equation}\label{Eq:potencial_termodinamico}
\begin{split}
	\omega(\mu) =&~ \varepsilon_{\rm{kin}} + m\rho_s - G_s\rho_s^2 + G_v\rho^2 + G_{sv}\rho_s^2\rho^2 + G_\rho\rho_3^2 \\
	&+ G_{v\rho}\rho^2\rho_3^2 + G_{s\rho}\rho_s^2\rho_3^2 - \mu_p\rho_p - \mu_n\rho_n,
\end{split}
\end{equation}
%
onde
\begin{itemize}
	\item $\rho$ é a densidade bariônica, dada pela soma das densidades de nêutron e próton\footnote{São densidades numéricas de partículas, ou seja, representam o número de partículas por unidade de volume.}:
	\begin{equation}
		\rho = \rho_p + \rho_n.
	\end{equation}

	\item As densidades bariônicas de próton e nêutron são dadas por\footnote{De onde vem essa expressão para $\rho_i$. Explicar, explicar o momento de Fermi também.}
	\begin{equation}
		\rho_i = \int_0^{k_F^i}\frac{dp}{\pi^2}p^2; \qquad i = p,n; \quad k_F^i = \textrm{momento de Fermi},
	\end{equation}
	%
	ou, caso $\rho_i$ sejam conhecidos
	\begin{equation}\label{Eq:Mom_Fermi_a_partir_de_rho}
		p_F^i = \sqrt[3]{3\pi^2\rho_i}.
	\end{equation}
	
	\item $\mu_p$ e $\mu_n$ representam os potenciais químicos de próton e nêutron, respectivamente.
\end{itemize}

O termo cinético na expressão acima pode ser calculado através de (primeiro termo da Eq. (1) em \cite{PRC_68_035804_2003}, o resto é energia potencial)\footnote{Degenerescência: O 2 se refere às duas possibilidades de spin; Podemos ter um $N_f$ que representa o número de sabores.}
\begin{align}
	\varepsilon_{\rm{kin}} &= \langle\bar{\psi}(\vec{\gamma}\cdot\vec{p})\psi\rangle \\
	&= 2 N_c\sum_i \int \frac{d^3p}{(2\pi)^3}\frac{p^2 + m_i M_i}{E_i}(n_{i-}-n_{i+})\theta(\Lambda^2 - p^2),
\end{align}
%
onde
\begin{itemize}
	\item A soma se dá sobre as espécies de partículas;
	\item $N_c$ representa o número de cores\footnote{No nosso caso, 1?};
	\item $\theta$ é a função degrau, $\Lambda$ é o \emph{cutoff};
	\item $n_{i\pm}$ são as funções de distribuição de Fermi para estados de energia positiva e negativa (respectivamente), dados por
	\begin{equation}
		n_{i\pm} = \frac{1}{1 + \exp(\pm[\beta(E_i\mp\mu_i)])}
	\end{equation}
	%
	onde $i = p, n$ (no nosso caso, no artigo é $u, d, s$) e $\beta = T^{-1}$
	\item $M_i$ é a massa efetiva do nucleon em questão (quark, no artigo).
	\item $E_i = \sqrt{p^2 + M_i^2}$
	\item $m_i$ são as massas constituintes -- nuas, \emph{bare} --.
\end{itemize}

Se tomarmos $T \to 0$, temos que $n_{i-} \to 1$ e $n_{i+} \to 0$; Além disso, se o integrando só depende do módulo de $\vec{p}$, então (\cite{Glendenning}, p. 92)
\begin{equation}
	\int\frac{d^3p}{(2\pi)^3} \to \frac{1}{2\pi^2}\int p^2dp.
\end{equation}
%
Logo, temos
\begin{align}
	\varepsilon &= 2 N_c \frac{1}{2\pi^2}\sum_i \int p^2 dp \frac{p^2 + m_i M_i}{\sqrt{p^2 + M_i^2}} \theta(\Lambda^2 - p^2) \\
	&= \frac{N_c}{\pi^2}\sum_i\left[\int \frac{p^4dp}{\sqrt{p^2 + M_i^2}}\theta(\Lambda^2 - p^2) + \int m_i M_i \frac{p^2 dp}{\sqrt{p^2 + M_i^2}}\theta(\Lambda^2 - p^2)\right] \label{Eq:Engergia_cin_separada}
\end{align}
%
Podemos utilizar as relações (\cite{Glendenning} p. 94\footnote{Na Ref. o primeiro termo da segunda expressão aparece sem o $k$ multiplicando, o que dimensionalmente está incorreto.})
\begin{align}
	\int \frac{k^4}{\sqrt{k^2 + m^2}} dk &= \frac{1}{4}\left[k^3\epsilon - \frac{3}{2} m^2k\epsilon + \frac{3}{2}m^4\ln\frac{\epsilon + k}{m} \right]\\
	\int \frac{k^2}{\sqrt{k^2 + m^2}} dk &= \frac{1}{2}\left[k\epsilon - \frac{1}{2}m^2\ln\frac{\epsilon + k}{m}\right] \label{Eq:Integ_momento_quad}
\end{align}
%
onde $\epsilon = \sqrt{k^2+m^2}$. Tomando o caso $m_i \to 0$\footnote{No prog. \texttt{eos\_enjl1-dens-assym- clean-rho-vr.f}: $\varepsilon \propto [F_2(M, k_F^i) - F_2(M, \Lambda)]$ ao invés de $\varepsilon \propto [F_2(M_i, \Lambda) - F_2(M_i, 0)]$; Isso se deve à retirada da contribuição do vácuo. Além disso, aparentemente os $M_i$ podem ser diferentes. No prog. são iguais, imagino que seja por considerarmos $m_n = m_p = m_N$.}, obtemos
\begin{equation}\label{Eq:Energia_kin}
	\varepsilon_{\rm{kin}} = \frac{N_c}{\pi^2}\sum_i \Big[\underbrace{\frac{1}{8}\Big((2p^3 - 3M_i^2p)\sqrt{p^2 + M_i^2} + 3M_i^4\ln\frac{p + \sqrt{p^2 + M_i^2}}{M_i}\Big)}_{F_2(m,p)}\Big]_0^\Lambda
\end{equation}

A densidade escalar $\rho_s$ é dada por\footnote{De onde vem essa expressão?}
\begin{equation}\label{Eq:Dens_Escalar}
	\rho_s^i = \frac{M}{\pi^2}[F_0(M, p_F^i) - F_0(M, \Lambda)], \quad i = p, n,
\end{equation}
%
onde
\begin{equation}
	F_0(M, x) = \int_0^x dp\frac{p^2}{\sqrt{M^2 + p^2}} dp.
\end{equation}
%
Utilizando a Equação~\eqref{Eq:Integ_momento_quad}, podemos reescrever a equação acima como
\begin{equation}
	F_0(M, x) = \frac{1}{2}\left[x\sqrt{x^2+M^2} - M^2 \ln \frac{x + \sqrt{x^2+M^2}}{M}\right].
\end{equation}

A massa efetiva $M$ na equação acima é dada por\footnote{Essa equação é conhecida como \emph{gap equation}}
\begin{equation}\label{Eq:Gap}
	M = m - 2G_s\rho_s + 2G_{sv}\rho_s\rho^2 + 2 G_{s\rho}\rho_s\rho_3^2,
\end{equation}
%
com $\rho_s = \rho_s^p + \rho_s^n$. Temos, portanto, uma interdependência entre as equações. Para que seja possível solucionar tais equações, podemos definir uma função $f(M)$ de tal forma que
\begin{equation}\label{Eq:Gap_zero}
	f(M) = M - m + 2G_s\rho_s - 2G_{sv}\rho_s\rho^2 - 2 G_{s\rho}\rho_s\rho_3^2,.
\end{equation}
%
Para solucionarmos a equação acima, basta utilizarmos uma rotina para encontrar zeros de funções, por exemplo biseção ou Newton-Raphson, encontrando o valor de $M$ para o qual $f(M) = 0$. A densidade escalar $\rho_s$ pode ser calculada através da expressão~\eqref{Eq:Dens_Escalar}\footnote{Na prática é mais fácil salvar o último valor de $\rho_s$ calculado pela rotina que tenta encontrar o zero da função $f(M)$.}.

Os potenciais químicos são dados por\footnote{Como essas expressões são calculadas?}
\begin{equation}\label{Eq:Potenciais_Quimicos}
\begin{split}
	\mu_i =&~ E_{p_F}^i + 2G_v\rho + 2G_{sv}\rho\rho_s^2 \pm 2G_\rho\rho_3+2G_{v\rho}\rho_3^2\rho \\
	& \pm 2G_{v\rho}\rho^2\rho_3 \pm 2 G_{s\rho}\rho_3\rho_s^2,
\end{split}
\end{equation}
%
onde $i = p,n$, os sinais superiores se referem ao caso de prótons, e $E_{p_F}^i = \sqrt{M^2 + (p_F^i)^2}$.

As equações de estado para pressão $P$ e densidade de energia $\varepsilon$ são dadas por\footnote{Como são calculadas?}
\begin{align}
	P &= -\omega(\mu) + \epsilon_0 \label{Eq:Pressao}\\
	\varepsilon &= -P + \mu_p\rho_p + \mu_n\rho_n. \label{Eq:Densidade_energia}
\end{align}

%%%%%%%%%%%%%%%%%%%%%%%%%%%%%
\section{Análise dimensional}
%%%%%%%%%%%%%%%%%%%%%%%%%%%%%

Devido ao fato de que $\hbar = c = 1$, adimensionais, e que não carregamos quaisquer unidades, é comum que algumas equações tenham dimensões discrepantes entre os membros esquerdo e direito, ou mesmo entre termos de um mesmo membro. Podemos acertar as dimensões multiplicando por potências de $\hbar c$. Nas unidades usuais em Física Nuclear, temos que $\hbar c = 197.326\rm{MeV}\cdot\rm{fm}$. Logo, todas as grandezas têm dimensões que envolvem MeV ou fm, ou uma combinação de ambos\sidenote{Note que de qualquer forma as unidades são atípicas, pois --~por exemplo~-- a unidade de massa é o eV, que na realidade tem dimensão de energia. Isso pode ser explicado através de $E^2 = p^2 c^2 + m^2 c^4$, onde assumimos que $c = 1$, adimensional.}:

\begin{itemize}
	\item Como $E^2 = p^2c^2 + m^2c^4$, temos $[E] = [m] = [p]$;
	\item Como $\rho = \bar{\psi}\gamma^0\psi$ é o número de partículas por unidade de volume, temos que $[\rho] = \rm{fm}^{-3}$
	\item O item acima implica que $[\psi] = \rm{fm}^{-3/2}$. Consequentemente, $[\rho_s] = [\bar{\psi}\psi] = \rm{fm}^{-3}$. 
	\item Como o potencial químico esta relacionado à variação de energia ao se adicionar ou retirar partículas do sistema, sua unidade é a de energia (MeV).
	\item O potencial termodinâmico $\omega$ é o potencial grande-canônico $\Omega$\footnote{Potencial de Landau} por unidade de volume, portanto tem dimensão de energia por unidade de volume ($\rm{MeV}\cdot\rm{fm}^{-3}$), assim como a densidade de energia $\varepsilon$ e a pressão $P$.
\end{itemize}

Dessa forma, as equações discutidas acima precisam ter suas unidades checadas e --~quando necessário~-- corrigidas, de forma a ficarem consistentes. Temos então:
\begin{fullwidth}
\begin{itemize}

\item O potencial termodinâmico (por unidade de volume) $\omega$ deve ter dimensão energia por volume (no nosso caso, $\rm{MeV}/\rm{fm}^3$):
\begin{equation}
\begin{split}
	\underbrace{\omega}_{\frac{\rm{MeV}}{\rm{fm}^3}} =&~ \underbrace{\varepsilon_{\rm{kin}}}_{\frac{\rm{MeV}}{\rm{fm}^3}} +\underbrace{m}_{\rm{MeV}}\underbrace{\rho_s}_{\rm{fm}^{-3}} - \underbrace{G_s}_{\rm{fm}^2}\underbrace{\rho_s^2}_{\rm{fm}^{-6}} + \underbrace{G_v}_{\rm{fm}^2}\underbrace{\rho^2}_{\rm{fm}^{-6}} + \underbrace{G_{sv}}_{\rm{fm}^8}\underbrace{\rho_s^2\rho^2}_{\rm{fm}^{-12}} \\
&+ \underbrace{G_\rho}_{\rm{fm}^2}\underbrace{\rho_3^2}_{\rm{fm}^{-6}} + \underbrace{G_{v\rho}}_{\rm{fm}^2}\underbrace{\rho^2\rho_3^2}_{\rm{fm}^{-12}} + \underbrace{G_{s\rho}}_{\rm{fm}^8}\underbrace{\rho_s^2\rho_3^2}_{\rm{fm}^{-12}} - \underbrace{\mu_p}_{\rm{MeV}}\underbrace{\rho_p}_{\rm{fm}^{-3}} - \underbrace{\mu_n}_{\rm{MeV}}\underbrace{\rho_n}_{\rm{fm}^{-3}}.
\end{split}
\end{equation}
%
Os termos envolvendo as constantes $G_i$ necessitam ser multiplicados por $\hbar c$, cuja dimensão é $\rm{MeV}\cdot\rm{fm}$, resultando na dimensão $\rm{MeV}/\rm{fm}^3$ para tais termos.

\item A pressão deve ter unidade de energia por unidade de volume ($\rm{MeV}/\rm{fm}^3$):
\begin{equation}
	\underbrace{P}_{\frac{\rm{MeV}}{\rm{fm}^3}} = - \underbrace{\omega}_{\frac{\rm{MeV}}{\rm{fm}^3}} + \underbrace{\varepsilon_0}_{\frac{\rm{MeV}}{\rm{fm}^3}},
\end{equation}
%
assim como a densidade de energia
\begin{equation}
	\underbrace{\varepsilon}_{\frac{\rm{MeV}}{\rm{fm}^3}} = - \underbrace{P}_{\frac{\rm{MeV}}{\rm{fm}^3}} + \underbrace{\mu_p}_{\rm{MeV}}\underbrace{\rho_p}_{\rm{fm}^{-3}} + \underbrace{\mu_n}_{\rm{MeV}}\underbrace{\rho_n}_{\rm{fm}^{-3}},
\end{equation}
%
e podemos ver que ambas estão com todas as dimensões corretas.

\item A equação para o cálculo da massa
\begin{equation}
	\underbrace{M}_{\rm{MeV}} = m - 2 \underbrace{G_s}_{\rm{fm}^2} \underbrace{\rho_s}_{\rm{fm}^{-3}} + 2 \underbrace{G_{sv}}_{\rm{fm}^8}\underbrace{\rho_s\rho^2}_{\rm{fm}^{-9}} + \underbrace{G_{s\rho}}_{\rm{fm}^8}\underbrace{\rho_s\rho_3^2}_{\rm{fm}^{-9}}
\end{equation}
%
necessita ter seus termos proporcionais a $G_i$ multiplicados por $\hbar c$.

\item A equação para a densidade bariônica
\begin{equation}
	\underbrace{\rho_i}_{\rm{fm}^{-3}} = \underbrace{\int_0^{p_F^i} \frac{p^2 dp}{\pi^2}}_{\rm{MeV}^3} = \underbrace{\frac{1}{3\pi^2}p^3\Big|_0^{p_F^i}}_{\rm{MeV}^3},
\end{equation}
%
assim como a equação para a densidade escalar
\begin{equation}
	\underbrace{\rho_s^i}_{\rm{fm}^{-3}} = \underbrace{\frac{M}{\pi^2}[F_0(M,p_F^i) - F_0(M, \Lambda)]}_{\rm{MeV}^3},
\end{equation}
%
onde usamos (vide Eq.~\eqref{Eq:Integ_momento_quad})
\begin{equation}
	F_0(m,p) = \underbrace{\int_0^x \frac{p^2}{\sqrt{p^2 + m^2}}}_{\rm{MeV}^2} dp,
\end{equation}
%
devem ser multiplicadas por $(\hbar c)^{-3}$.

\item Para o potencial químico temos
\begin{equation}
	\underbrace{\mu_i}_{\rm{MeV}} = \underbrace{E_{p_F}^i}_{\rm{MeV}} +~2 \underbrace{G_v}_{\rm{fm}^2}\underbrace{\rho}_{\rm{fm^{-3}}} +~2 \underbrace{G_{sv}}_{\rm{fm}^8}\underbrace{\rho\rho_s^2}_{\rm{fm^{-9}}} \pm~2 \underbrace{G_\rho}_{\rm{fm}^2}\underbrace{\rho_3}_{\rm{fm^{-3}}} +~2 \underbrace{G_{v\rho}}_{\rm{fm}^8}\underbrace{\rho_3^2\rho}_{\rm{fm^{-9}}} \pm~2 \underbrace{G_{v\rho}}_{\rm{fm}^8}\underbrace{\rho^2\rho_3}_{\rm{fm^{-9}}} \pm~2
\underbrace{G_{s\rho}}_{\rm{fm}^8}\underbrace{\rho_3\rho_s^2}_{\rm{fm^{-9}}},
\end{equation}
%
onde
\begin{equation}
	\underbrace{E_{p_F}^i}_{\rm{MeV}} = \sqrt{M^2 + p_F^2}; \quad [M] = [p_F] = \rm{MeV}.
\end{equation}
%
Portanto, verificamos que os termos proporcionais a $G_i$ devem ser multiplicados por $\hbar c$.

\item O termo cinético da energia é dado pela Equação~\eqref{Eq:Energia_kin}. O termo definido como $F_2(m, p)$ tem dimensão de $\rm{MeV}^4$, já que todos as parcelas são produto de $m$, $p$, e $\epsilon = \sqrt{p^2+m_i^2}$. Além disso, o argumento do logarítimo é adimensional. No entanto, como $\varepsilon_{\rm{kin}}$ é uma densidade de energia, a dimensão correta é $\rm{MeV} / \rm{fm}^3$. Portanto, é necessário multiplicar a expressão para $\varepsilon_{\rm{kin}}$ por $(\hbar c)^{-3}$.
\end{itemize}
\end{fullwidth}

%%%%%%%%%%%%%%%%%%%%%%%%%%%%%%%%%%%%%%%%%%%%%%%
\section{Solução da equação para $M$, $\rho_s$}
%%%%%%%%%%%%%%%%%%%%%%%%%%%%%%%%%%%%%%%%%%%%%%%

A solução da Equação~\eqref{Eq:Gap} é encontrada zerando a Equação~\eqref{Eq:Gap_zero}. No entanto, essa última depende de $\rho$ e dos momentos de Fermi para próton e nêutron (o que é uma dependência indireta de $\rho$ e da fração de prótons). Assim, a forma da função pode se alterar de acordo com os valores de tais variáveis. A Figura~\ref{Fig:Gap_zero_graph} mostra curvas de $f(M)$ para diferentes valores de $\rho$.

\begin{figure*}
	\begin{tikzpicture}[gnuplot]
%% generated with GNUPLOT 5.0p2 (Lua 5.2; terminal rev. 99, script rev. 100)
%% Fri Mar  4 16:19:10 2016
\path (0.000,0.000) rectangle (14.000,9.000);
\gpcolor{color=gp lt color border}
\gpsetlinetype{gp lt border}
\gpsetdashtype{gp dt solid}
\gpsetlinewidth{1.00}
\draw[gp path] (1.504,0.985)--(1.684,0.985);
\draw[gp path] (13.447,0.985)--(13.267,0.985);
\node[gp node right] at (1.320,0.985) {$-500$};
\draw[gp path] (1.504,2.514)--(1.684,2.514);
\draw[gp path] (13.447,2.514)--(13.267,2.514);
\node[gp node right] at (1.320,2.514) {$0$};
\draw[gp path] (1.504,4.043)--(1.684,4.043);
\draw[gp path] (13.447,4.043)--(13.267,4.043);
\node[gp node right] at (1.320,4.043) {$500$};
\draw[gp path] (1.504,5.573)--(1.684,5.573);
\draw[gp path] (13.447,5.573)--(13.267,5.573);
\node[gp node right] at (1.320,5.573) {$1000$};
\draw[gp path] (1.504,7.102)--(1.684,7.102);
\draw[gp path] (13.447,7.102)--(13.267,7.102);
\node[gp node right] at (1.320,7.102) {$1500$};
\draw[gp path] (1.504,8.631)--(1.684,8.631);
\draw[gp path] (13.447,8.631)--(13.267,8.631);
\node[gp node right] at (1.320,8.631) {$2000$};
\draw[gp path] (1.504,0.985)--(1.504,1.165);
\draw[gp path] (1.504,8.631)--(1.504,8.451);
\node[gp node center] at (1.504,0.677) {$0$};
\draw[gp path] (2.698,0.985)--(2.698,1.165);
\draw[gp path] (2.698,8.631)--(2.698,8.451);
\node[gp node center] at (2.698,0.677) {$200$};
\draw[gp path] (3.893,0.985)--(3.893,1.165);
\draw[gp path] (3.893,8.631)--(3.893,8.451);
\node[gp node center] at (3.893,0.677) {$400$};
\draw[gp path] (5.087,0.985)--(5.087,1.165);
\draw[gp path] (5.087,8.631)--(5.087,8.451);
\node[gp node center] at (5.087,0.677) {$600$};
\draw[gp path] (6.281,0.985)--(6.281,1.165);
\draw[gp path] (6.281,8.631)--(6.281,8.451);
\node[gp node center] at (6.281,0.677) {$800$};
\draw[gp path] (7.476,0.985)--(7.476,1.165);
\draw[gp path] (7.476,8.631)--(7.476,8.451);
\node[gp node center] at (7.476,0.677) {$1000$};
\draw[gp path] (8.670,0.985)--(8.670,1.165);
\draw[gp path] (8.670,8.631)--(8.670,8.451);
\node[gp node center] at (8.670,0.677) {$1200$};
\draw[gp path] (9.864,0.985)--(9.864,1.165);
\draw[gp path] (9.864,8.631)--(9.864,8.451);
\node[gp node center] at (9.864,0.677) {$1400$};
\draw[gp path] (11.058,0.985)--(11.058,1.165);
\draw[gp path] (11.058,8.631)--(11.058,8.451);
\node[gp node center] at (11.058,0.677) {$1600$};
\draw[gp path] (12.253,0.985)--(12.253,1.165);
\draw[gp path] (12.253,8.631)--(12.253,8.451);
\node[gp node center] at (12.253,0.677) {$1800$};
\draw[gp path] (13.447,0.985)--(13.447,1.165);
\draw[gp path] (13.447,8.631)--(13.447,8.451);
\node[gp node center] at (13.447,0.677) {$2000$};
\draw[gp path] (1.504,8.631)--(1.504,0.985)--(13.447,0.985)--(13.447,8.631)--cycle;
\node[gp node center,rotate=-270] at (0.246,4.808) {$f(M) = M + G_s\rho_s - 2G_{sv}\rho_s\rho^2$ (MeV)};
\node[gp node center] at (7.475,0.215) {$M$ (MeV)};
\node[gp node left] at (2.972,8.297) {$\rho_{\rm{min}}$};
\gpcolor{rgb color={0.580,0.000,0.827}}
\draw[gp path] (1.872,8.297)--(2.788,8.297);
\draw[gp path] (1.507,2.510)--(1.510,2.506)--(1.513,2.503)--(1.516,2.499)--(1.519,2.495)%
  --(1.522,2.491)--(1.525,2.487)--(1.528,2.483)--(1.531,2.479)--(1.534,2.476)--(1.537,2.472)%
  --(1.540,2.468)--(1.543,2.464)--(1.546,2.460)--(1.549,2.456)--(1.552,2.452)--(1.555,2.449)%
  --(1.558,2.445)--(1.561,2.441)--(1.564,2.437)--(1.567,2.433)--(1.570,2.429)--(1.573,2.425)%
  --(1.576,2.422)--(1.579,2.418)--(1.582,2.414)--(1.585,2.410)--(1.588,2.406)--(1.591,2.402)%
  --(1.594,2.399)--(1.597,2.395)--(1.600,2.391)--(1.603,2.387)--(1.606,2.383)--(1.609,2.380)%
  --(1.611,2.376)--(1.614,2.372)--(1.617,2.368)--(1.620,2.364)--(1.623,2.360)--(1.626,2.357)%
  --(1.629,2.353)--(1.632,2.349)--(1.635,2.345)--(1.638,2.341)--(1.641,2.338)--(1.644,2.334)%
  --(1.647,2.330)--(1.650,2.326)--(1.653,2.323)--(1.656,2.319)--(1.659,2.315)--(1.662,2.311)%
  --(1.665,2.308)--(1.668,2.304)--(1.671,2.300)--(1.674,2.296)--(1.677,2.293)--(1.680,2.289)%
  --(1.683,2.285)--(1.686,2.281)--(1.689,2.278)--(1.692,2.274)--(1.695,2.270)--(1.698,2.266)%
  --(1.701,2.263)--(1.704,2.259)--(1.707,2.255)--(1.710,2.252)--(1.713,2.248)--(1.716,2.244)%
  --(1.719,2.241)--(1.722,2.237)--(1.725,2.233)--(1.728,2.230)--(1.731,2.226)--(1.734,2.222)%
  --(1.737,2.219)--(1.740,2.215)--(1.743,2.211)--(1.746,2.208)--(1.749,2.204)--(1.752,2.200)%
  --(1.755,2.197)--(1.758,2.193)--(1.761,2.190)--(1.764,2.186)--(1.767,2.182)--(1.770,2.179)%
  --(1.773,2.175)--(1.776,2.172)--(1.779,2.168)--(1.782,2.165)--(1.785,2.161)--(1.788,2.157)%
  --(1.791,2.154)--(1.794,2.150)--(1.797,2.147)--(1.800,2.143)--(1.803,2.140)--(1.806,2.136)%
  --(1.809,2.133)--(1.812,2.129)--(1.815,2.126)--(1.818,2.122)--(1.820,2.119)--(1.823,2.115)%
  --(1.826,2.112)--(1.829,2.108)--(1.832,2.105)--(1.835,2.101)--(1.838,2.098)--(1.841,2.095)%
  --(1.844,2.091)--(1.847,2.088)--(1.850,2.084)--(1.853,2.081)--(1.856,2.077)--(1.859,2.074)%
  --(1.862,2.071)--(1.865,2.067)--(1.868,2.064)--(1.871,2.061)--(1.874,2.057)--(1.877,2.054)%
  --(1.880,2.050)--(1.883,2.047)--(1.886,2.044)--(1.889,2.041)--(1.892,2.037)--(1.895,2.034)%
  --(1.898,2.031)--(1.901,2.027)--(1.904,2.024)--(1.907,2.021)--(1.910,2.017)--(1.913,2.014)%
  --(1.916,2.011)--(1.919,2.008)--(1.922,2.004)--(1.925,2.001)--(1.928,1.998)--(1.931,1.995)%
  --(1.934,1.992)--(1.937,1.988)--(1.940,1.985)--(1.943,1.982)--(1.946,1.979)--(1.949,1.976)%
  --(1.952,1.972)--(1.955,1.969)--(1.958,1.966)--(1.961,1.963)--(1.964,1.960)--(1.967,1.957)%
  --(1.970,1.954)--(1.973,1.950)--(1.976,1.947)--(1.979,1.944)--(1.982,1.941)--(1.985,1.938)%
  --(1.988,1.935)--(1.991,1.932)--(1.994,1.929)--(1.997,1.926)--(2.000,1.923)--(2.003,1.920)%
  --(2.006,1.917)--(2.009,1.914)--(2.012,1.911)--(2.015,1.908)--(2.018,1.905)--(2.021,1.902)%
  --(2.024,1.899)--(2.027,1.896)--(2.029,1.893)--(2.032,1.890)--(2.035,1.887)--(2.038,1.884)%
  --(2.041,1.881)--(2.044,1.878)--(2.047,1.875)--(2.050,1.873)--(2.053,1.870)--(2.056,1.867)%
  --(2.059,1.864)--(2.062,1.861)--(2.065,1.858)--(2.068,1.855)--(2.071,1.852)--(2.074,1.850)%
  --(2.077,1.847)--(2.080,1.844)--(2.083,1.841)--(2.086,1.838)--(2.089,1.836)--(2.092,1.833)%
  --(2.095,1.830)--(2.098,1.827)--(2.101,1.825)--(2.104,1.822)--(2.107,1.819)--(2.110,1.816)%
  --(2.113,1.814)--(2.116,1.811)--(2.119,1.808)--(2.122,1.805)--(2.125,1.803)--(2.128,1.800)%
  --(2.131,1.797)--(2.134,1.795)--(2.137,1.792)--(2.140,1.789)--(2.143,1.787)--(2.146,1.784)%
  --(2.149,1.782)--(2.152,1.779)--(2.155,1.776)--(2.158,1.774)--(2.161,1.771)--(2.164,1.769)%
  --(2.167,1.766)--(2.170,1.764)--(2.173,1.761)--(2.176,1.758)--(2.179,1.756)--(2.182,1.753)%
  --(2.185,1.751)--(2.188,1.748)--(2.191,1.746)--(2.194,1.743)--(2.197,1.741)--(2.200,1.738)%
  --(2.203,1.736)--(2.206,1.733)--(2.209,1.731)--(2.212,1.729)--(2.215,1.726)--(2.218,1.724)%
  --(2.221,1.721)--(2.224,1.719)--(2.227,1.716)--(2.230,1.714)--(2.233,1.712)--(2.236,1.709)%
  --(2.238,1.707)--(2.241,1.705)--(2.244,1.702)--(2.247,1.700)--(2.250,1.698)--(2.253,1.695)%
  --(2.256,1.693)--(2.259,1.691)--(2.262,1.688)--(2.265,1.686)--(2.268,1.684)--(2.271,1.682)%
  --(2.274,1.679)--(2.277,1.677)--(2.280,1.675)--(2.283,1.673)--(2.286,1.670)--(2.289,1.668)%
  --(2.292,1.666)--(2.295,1.664)--(2.298,1.661)--(2.301,1.659)--(2.304,1.657)--(2.307,1.655)%
  --(2.310,1.653)--(2.313,1.651)--(2.316,1.648)--(2.319,1.646)--(2.322,1.644)--(2.325,1.642)%
  --(2.328,1.640)--(2.331,1.638)--(2.334,1.636)--(2.337,1.634)--(2.340,1.632)--(2.343,1.629)%
  --(2.346,1.627)--(2.349,1.625)--(2.352,1.623)--(2.355,1.621)--(2.358,1.619)--(2.361,1.617)%
  --(2.364,1.615)--(2.367,1.613)--(2.370,1.611)--(2.373,1.609)--(2.376,1.607)--(2.379,1.605)%
  --(2.382,1.603)--(2.385,1.601)--(2.388,1.599)--(2.391,1.597)--(2.394,1.595)--(2.397,1.594)%
  --(2.400,1.592)--(2.403,1.590)--(2.406,1.588)--(2.409,1.586)--(2.412,1.584)--(2.415,1.582)%
  --(2.418,1.580)--(2.421,1.578)--(2.424,1.577)--(2.427,1.575)--(2.430,1.573)--(2.433,1.571)%
  --(2.436,1.569)--(2.439,1.567)--(2.442,1.566)--(2.445,1.564)--(2.447,1.562)--(2.450,1.560)%
  --(2.453,1.558)--(2.456,1.557)--(2.459,1.555)--(2.462,1.553)--(2.465,1.551)--(2.468,1.550)%
  --(2.471,1.548)--(2.474,1.546)--(2.477,1.544)--(2.480,1.543)--(2.483,1.541)--(2.486,1.539)%
  --(2.489,1.538)--(2.492,1.536)--(2.495,1.534)--(2.498,1.533)--(2.501,1.531)--(2.504,1.529)%
  --(2.507,1.528)--(2.510,1.526)--(2.513,1.524)--(2.516,1.523)--(2.519,1.521)--(2.522,1.520)%
  --(2.525,1.518)--(2.528,1.516)--(2.531,1.515)--(2.534,1.513)--(2.537,1.512)--(2.540,1.510)%
  --(2.543,1.509)--(2.546,1.507)--(2.549,1.506)--(2.552,1.504)--(2.555,1.502)--(2.558,1.501)%
  --(2.561,1.499)--(2.564,1.498)--(2.567,1.496)--(2.570,1.495)--(2.573,1.493)--(2.576,1.492)%
  --(2.579,1.491)--(2.582,1.489)--(2.585,1.488)--(2.588,1.486)--(2.591,1.485)--(2.594,1.483)%
  --(2.597,1.482)--(2.600,1.481)--(2.603,1.479)--(2.606,1.478)--(2.609,1.476)--(2.612,1.475)%
  --(2.615,1.474)--(2.618,1.472)--(2.621,1.471)--(2.624,1.469)--(2.627,1.468)--(2.630,1.467)%
  --(2.633,1.465)--(2.636,1.464)--(2.639,1.463)--(2.642,1.462)--(2.645,1.460)--(2.648,1.459)%
  --(2.651,1.458)--(2.654,1.456)--(2.656,1.455)--(2.659,1.454)--(2.662,1.453)--(2.665,1.451)%
  --(2.668,1.450)--(2.671,1.449)--(2.674,1.448)--(2.677,1.446)--(2.680,1.445)--(2.683,1.444)%
  --(2.686,1.443)--(2.689,1.441)--(2.692,1.440)--(2.695,1.439)--(2.698,1.438)--(2.701,1.437)%
  --(2.704,1.436)--(2.707,1.434)--(2.710,1.433)--(2.713,1.432)--(2.716,1.431)--(2.719,1.430)%
  --(2.722,1.429)--(2.725,1.428)--(2.728,1.426)--(2.731,1.425)--(2.734,1.424)--(2.737,1.423)%
  --(2.740,1.422)--(2.743,1.421)--(2.746,1.420)--(2.749,1.419)--(2.752,1.418)--(2.755,1.417)%
  --(2.758,1.416)--(2.761,1.415)--(2.764,1.414)--(2.767,1.412)--(2.770,1.411)--(2.773,1.410)%
  --(2.776,1.409)--(2.779,1.408)--(2.782,1.407)--(2.785,1.406)--(2.788,1.405)--(2.791,1.404)%
  --(2.794,1.403)--(2.797,1.402)--(2.800,1.402)--(2.803,1.401)--(2.806,1.400)--(2.809,1.399)%
  --(2.812,1.398)--(2.815,1.397)--(2.818,1.396)--(2.821,1.395)--(2.824,1.394)--(2.827,1.393)%
  --(2.830,1.392)--(2.833,1.391)--(2.836,1.390)--(2.839,1.390)--(2.842,1.389)--(2.845,1.388)%
  --(2.848,1.387)--(2.851,1.386)--(2.854,1.385)--(2.857,1.384)--(2.860,1.384)--(2.863,1.383)%
  --(2.866,1.382)--(2.868,1.381)--(2.871,1.380)--(2.874,1.379)--(2.877,1.379)--(2.880,1.378)%
  --(2.883,1.377)--(2.886,1.376)--(2.889,1.375)--(2.892,1.375)--(2.895,1.374)--(2.898,1.373)%
  --(2.901,1.372)--(2.904,1.372)--(2.907,1.371)--(2.910,1.370)--(2.913,1.369)--(2.916,1.369)%
  --(2.919,1.368)--(2.922,1.367)--(2.925,1.366)--(2.928,1.366)--(2.931,1.365)--(2.934,1.364)%
  --(2.937,1.364)--(2.940,1.363)--(2.943,1.362)--(2.946,1.362)--(2.949,1.361)--(2.952,1.360)%
  --(2.955,1.360)--(2.958,1.359)--(2.961,1.358)--(2.964,1.358)--(2.967,1.357)--(2.970,1.356)%
  --(2.973,1.356)--(2.976,1.355)--(2.979,1.355)--(2.982,1.354)--(2.985,1.353)--(2.988,1.353)%
  --(2.991,1.352)--(2.994,1.352)--(2.997,1.351)--(3.000,1.350)--(3.003,1.350)--(3.006,1.349)%
  --(3.009,1.349)--(3.012,1.348)--(3.015,1.348)--(3.018,1.347)--(3.021,1.347)--(3.024,1.346)%
  --(3.027,1.345)--(3.030,1.345)--(3.033,1.344)--(3.036,1.344)--(3.039,1.343)--(3.042,1.343)%
  --(3.045,1.342)--(3.048,1.342)--(3.051,1.341)--(3.054,1.341)--(3.057,1.340)--(3.060,1.340)%
  --(3.063,1.339)--(3.066,1.339)--(3.069,1.339)--(3.072,1.338)--(3.075,1.338)--(3.077,1.337)%
  --(3.080,1.337)--(3.083,1.336)--(3.086,1.336)--(3.089,1.335)--(3.092,1.335)--(3.095,1.335)%
  --(3.098,1.334)--(3.101,1.334)--(3.104,1.333)--(3.107,1.333)--(3.110,1.333)--(3.113,1.332)%
  --(3.116,1.332)--(3.119,1.331)--(3.122,1.331)--(3.125,1.331)--(3.128,1.330)--(3.131,1.330)%
  --(3.134,1.330)--(3.137,1.329)--(3.140,1.329)--(3.143,1.329)--(3.146,1.328)--(3.149,1.328)%
  --(3.152,1.328)--(3.155,1.327)--(3.158,1.327)--(3.161,1.327)--(3.164,1.326)--(3.167,1.326)%
  --(3.170,1.326)--(3.173,1.325)--(3.176,1.325)--(3.179,1.325)--(3.182,1.325)--(3.185,1.324)%
  --(3.188,1.324)--(3.191,1.324)--(3.194,1.324)--(3.197,1.323)--(3.200,1.323)--(3.203,1.323)%
  --(3.206,1.323)--(3.209,1.322)--(3.212,1.322)--(3.215,1.322)--(3.218,1.322)--(3.221,1.321)%
  --(3.224,1.321)--(3.227,1.321)--(3.230,1.321)--(3.233,1.321)--(3.236,1.320)--(3.239,1.320)%
  --(3.242,1.320)--(3.245,1.320)--(3.248,1.320)--(3.251,1.319)--(3.254,1.319)--(3.257,1.319)%
  --(3.260,1.319)--(3.263,1.319)--(3.266,1.319)--(3.269,1.318)--(3.272,1.318)--(3.275,1.318)%
  --(3.278,1.318)--(3.281,1.318)--(3.284,1.318)--(3.286,1.318)--(3.289,1.317)--(3.292,1.317)%
  --(3.295,1.317)--(3.298,1.317)--(3.301,1.317)--(3.304,1.317)--(3.307,1.317)--(3.310,1.317)%
  --(3.313,1.317)--(3.316,1.317)--(3.319,1.316)--(3.322,1.316)--(3.325,1.316)--(3.328,1.316)%
  --(3.331,1.316)--(3.334,1.316)--(3.337,1.316)--(3.340,1.316)--(3.343,1.316)--(3.346,1.316)%
  --(3.349,1.316)--(3.352,1.316)--(3.355,1.316)--(3.358,1.316)--(3.361,1.316)--(3.364,1.316)%
  --(3.367,1.316)--(3.370,1.316)--(3.373,1.316)--(3.376,1.316)--(3.379,1.316)--(3.382,1.316)%
  --(3.385,1.316)--(3.388,1.316)--(3.391,1.316)--(3.394,1.316)--(3.397,1.316)--(3.400,1.316)%
  --(3.403,1.316)--(3.406,1.316)--(3.409,1.316)--(3.412,1.316)--(3.415,1.316)--(3.418,1.316)%
  --(3.421,1.316)--(3.424,1.316)--(3.427,1.316)--(3.430,1.316)--(3.433,1.316)--(3.436,1.316)%
  --(3.439,1.316)--(3.442,1.316)--(3.445,1.316)--(3.448,1.317)--(3.451,1.317)--(3.454,1.317)%
  --(3.457,1.317)--(3.460,1.317)--(3.463,1.317)--(3.466,1.317)--(3.469,1.317)--(3.472,1.317)%
  --(3.475,1.317)--(3.478,1.318)--(3.481,1.318)--(3.484,1.318)--(3.487,1.318)--(3.490,1.318)%
  --(3.493,1.318)--(3.495,1.318)--(3.498,1.318)--(3.501,1.319)--(3.504,1.319)--(3.507,1.319)%
  --(3.510,1.319)--(3.513,1.319)--(3.516,1.319)--(3.519,1.319)--(3.522,1.320)--(3.525,1.320)%
  --(3.528,1.320)--(3.531,1.320)--(3.534,1.320)--(3.537,1.321)--(3.540,1.321)--(3.543,1.321)%
  --(3.546,1.321)--(3.549,1.321)--(3.552,1.321)--(3.555,1.322)--(3.558,1.322)--(3.561,1.322)%
  --(3.564,1.322)--(3.567,1.323)--(3.570,1.323)--(3.573,1.323)--(3.576,1.323)--(3.579,1.323)%
  --(3.582,1.324)--(3.585,1.324)--(3.588,1.324)--(3.591,1.324)--(3.594,1.325)--(3.597,1.325)%
  --(3.600,1.325)--(3.603,1.325)--(3.606,1.326)--(3.609,1.326)--(3.612,1.326)--(3.615,1.326)%
  --(3.618,1.327)--(3.621,1.327)--(3.624,1.327)--(3.627,1.327)--(3.630,1.328)--(3.633,1.328)%
  --(3.636,1.328)--(3.639,1.329)--(3.642,1.329)--(3.645,1.329)--(3.648,1.330)--(3.651,1.330)%
  --(3.654,1.330)--(3.657,1.330)--(3.660,1.331)--(3.663,1.331)--(3.666,1.331)--(3.669,1.332)%
  --(3.672,1.332)--(3.675,1.332)--(3.678,1.333)--(3.681,1.333)--(3.684,1.333)--(3.687,1.334)%
  --(3.690,1.334)--(3.693,1.334)--(3.696,1.335)--(3.699,1.335)--(3.702,1.335)--(3.704,1.336)%
  --(3.707,1.336)--(3.710,1.336)--(3.713,1.337)--(3.716,1.337)--(3.719,1.338)--(3.722,1.338)%
  --(3.725,1.338)--(3.728,1.339)--(3.731,1.339)--(3.734,1.339)--(3.737,1.340)--(3.740,1.340)%
  --(3.743,1.341)--(3.746,1.341)--(3.749,1.341)--(3.752,1.342)--(3.755,1.342)--(3.758,1.343)%
  --(3.761,1.343)--(3.764,1.343)--(3.767,1.344)--(3.770,1.344)--(3.773,1.345)--(3.776,1.345)%
  --(3.779,1.345)--(3.782,1.346)--(3.785,1.346)--(3.788,1.347)--(3.791,1.347)--(3.794,1.348)%
  --(3.797,1.348)--(3.800,1.348)--(3.803,1.349)--(3.806,1.349)--(3.809,1.350)--(3.812,1.350)%
  --(3.815,1.351)--(3.818,1.351)--(3.821,1.352)--(3.824,1.352)--(3.827,1.353)--(3.830,1.353)%
  --(3.833,1.353)--(3.836,1.354)--(3.839,1.354)--(3.842,1.355)--(3.845,1.355)--(3.848,1.356)%
  --(3.851,1.356)--(3.854,1.357)--(3.857,1.357)--(3.860,1.358)--(3.863,1.358)--(3.866,1.359)%
  --(3.869,1.359)--(3.872,1.360)--(3.875,1.360)--(3.878,1.361)--(3.881,1.361)--(3.884,1.362)%
  --(3.887,1.362)--(3.890,1.363)--(3.893,1.363)--(3.896,1.364)--(3.899,1.364)--(3.902,1.365)%
  --(3.905,1.365)--(3.908,1.366)--(3.911,1.366)--(3.914,1.367)--(3.916,1.368)--(3.919,1.368)%
  --(3.922,1.369)--(3.925,1.369)--(3.928,1.370)--(3.931,1.370)--(3.934,1.371)--(3.937,1.371)%
  --(3.940,1.372)--(3.943,1.372)--(3.946,1.373)--(3.949,1.374)--(3.952,1.374)--(3.955,1.375)%
  --(3.958,1.375)--(3.961,1.376)--(3.964,1.376)--(3.967,1.377)--(3.970,1.378)--(3.973,1.378)%
  --(3.976,1.379)--(3.979,1.379)--(3.982,1.380)--(3.985,1.380)--(3.988,1.381)--(3.991,1.382)%
  --(3.994,1.382)--(3.997,1.383)--(4.000,1.383)--(4.003,1.384)--(4.006,1.385)--(4.009,1.385)%
  --(4.012,1.386)--(4.015,1.386)--(4.018,1.387)--(4.021,1.388)--(4.024,1.388)--(4.027,1.389)%
  --(4.030,1.389)--(4.033,1.390)--(4.036,1.391)--(4.039,1.391)--(4.042,1.392)--(4.045,1.393)%
  --(4.048,1.393)--(4.051,1.394)--(4.054,1.395)--(4.057,1.395)--(4.060,1.396)--(4.063,1.396)%
  --(4.066,1.397)--(4.069,1.398)--(4.072,1.398)--(4.075,1.399)--(4.078,1.400)--(4.081,1.400)%
  --(4.084,1.401)--(4.087,1.402)--(4.090,1.402)--(4.093,1.403)--(4.096,1.404)--(4.099,1.404)%
  --(4.102,1.405)--(4.105,1.406)--(4.108,1.406)--(4.111,1.407)--(4.114,1.408)--(4.117,1.408)%
  --(4.120,1.409)--(4.123,1.410)--(4.125,1.410)--(4.128,1.411)--(4.131,1.412)--(4.134,1.412)%
  --(4.137,1.413)--(4.140,1.414)--(4.143,1.414)--(4.146,1.415)--(4.149,1.416)--(4.152,1.417)%
  --(4.155,1.417)--(4.158,1.418)--(4.161,1.419)--(4.164,1.419)--(4.167,1.420)--(4.170,1.421)%
  --(4.173,1.422)--(4.176,1.422)--(4.179,1.423)--(4.182,1.424)--(4.185,1.424)--(4.188,1.425)%
  --(4.191,1.426)--(4.194,1.427)--(4.197,1.427)--(4.200,1.428)--(4.203,1.429)--(4.206,1.429)%
  --(4.209,1.430)--(4.212,1.431)--(4.215,1.432)--(4.218,1.432)--(4.221,1.433)--(4.224,1.434)%
  --(4.227,1.435)--(4.230,1.435)--(4.233,1.436)--(4.236,1.437)--(4.239,1.438)--(4.242,1.438)%
  --(4.245,1.439)--(4.248,1.440)--(4.251,1.441)--(4.254,1.441)--(4.257,1.442)--(4.260,1.443)%
  --(4.263,1.444)--(4.266,1.444)--(4.269,1.445)--(4.272,1.446)--(4.275,1.447)--(4.278,1.448)%
  --(4.281,1.448)--(4.284,1.449)--(4.287,1.450)--(4.290,1.451)--(4.293,1.451)--(4.296,1.452)%
  --(4.299,1.453)--(4.302,1.454)--(4.305,1.455)--(4.308,1.455)--(4.311,1.456)--(4.314,1.457)%
  --(4.317,1.458)--(4.320,1.459)--(4.323,1.459)--(4.326,1.460)--(4.329,1.461)--(4.332,1.462)%
  --(4.334,1.463)--(4.337,1.463)--(4.340,1.464)--(4.343,1.465)--(4.346,1.466)--(4.349,1.467)%
  --(4.352,1.467)--(4.355,1.468)--(4.358,1.469)--(4.361,1.470)--(4.364,1.471)--(4.367,1.472)%
  --(4.370,1.472)--(4.373,1.473)--(4.376,1.474)--(4.379,1.475)--(4.382,1.476)--(4.385,1.477)%
  --(4.388,1.477)--(4.391,1.478)--(4.394,1.479)--(4.397,1.480)--(4.400,1.481)--(4.403,1.482)%
  --(4.406,1.482)--(4.409,1.483)--(4.412,1.484)--(4.415,1.485)--(4.418,1.486)--(4.421,1.487)%
  --(4.424,1.487)--(4.427,1.488)--(4.430,1.489)--(4.433,1.490)--(4.436,1.491)--(4.439,1.492)%
  --(4.442,1.493)--(4.445,1.493)--(4.448,1.494)--(4.451,1.495)--(4.454,1.496)--(4.457,1.497)%
  --(4.460,1.498)--(4.463,1.499)--(4.466,1.500)--(4.469,1.500)--(4.472,1.501)--(4.475,1.502)%
  --(4.478,1.503)--(4.481,1.504)--(4.484,1.505)--(4.487,1.506)--(4.490,1.507)--(4.493,1.507)%
  --(4.496,1.508)--(4.499,1.509)--(4.502,1.510)--(4.505,1.511)--(4.508,1.512)--(4.511,1.513)%
  --(4.514,1.514)--(4.517,1.515)--(4.520,1.515)--(4.523,1.516)--(4.526,1.517)--(4.529,1.518)%
  --(4.532,1.519)--(4.535,1.520)--(4.538,1.521)--(4.541,1.522)--(4.543,1.523)--(4.546,1.524)%
  --(4.549,1.524)--(4.552,1.525)--(4.555,1.526)--(4.558,1.527)--(4.561,1.528)--(4.564,1.529)%
  --(4.567,1.530)--(4.570,1.531)--(4.573,1.532)--(4.576,1.533)--(4.579,1.534)--(4.582,1.535)%
  --(4.585,1.535)--(4.588,1.536)--(4.591,1.537)--(4.594,1.538)--(4.597,1.539)--(4.600,1.540)%
  --(4.603,1.541)--(4.606,1.542)--(4.609,1.543)--(4.612,1.544)--(4.615,1.545)--(4.618,1.546)%
  --(4.621,1.547)--(4.624,1.548)--(4.627,1.549)--(4.630,1.550)--(4.633,1.550)--(4.636,1.551)%
  --(4.639,1.552)--(4.642,1.553)--(4.645,1.554)--(4.648,1.555)--(4.651,1.556)--(4.654,1.557)%
  --(4.657,1.558)--(4.660,1.559)--(4.663,1.560)--(4.666,1.561)--(4.669,1.562)--(4.672,1.563)%
  --(4.675,1.564)--(4.678,1.565)--(4.681,1.566)--(4.684,1.567)--(4.687,1.568)--(4.690,1.569)%
  --(4.693,1.570)--(4.696,1.571)--(4.699,1.572)--(4.702,1.573)--(4.705,1.573)--(4.708,1.574)%
  --(4.711,1.575)--(4.714,1.576)--(4.717,1.577)--(4.720,1.578)--(4.723,1.579)--(4.726,1.580)%
  --(4.729,1.581)--(4.732,1.582)--(4.735,1.583)--(4.738,1.584)--(4.741,1.585)--(4.744,1.586)%
  --(4.747,1.587)--(4.750,1.588)--(4.752,1.589)--(4.755,1.590)--(4.758,1.591)--(4.761,1.592)%
  --(4.764,1.593)--(4.767,1.594)--(4.770,1.595)--(4.773,1.596)--(4.776,1.597)--(4.779,1.598)%
  --(4.782,1.599)--(4.785,1.600)--(4.788,1.601)--(4.791,1.602)--(4.794,1.603)--(4.797,1.604)%
  --(4.800,1.605)--(4.803,1.606)--(4.806,1.607)--(4.809,1.608)--(4.812,1.609)--(4.815,1.610)%
  --(4.818,1.611)--(4.821,1.612)--(4.824,1.613)--(4.827,1.614)--(4.830,1.615)--(4.833,1.616)%
  --(4.836,1.617)--(4.839,1.618)--(4.842,1.619)--(4.845,1.620)--(4.848,1.622)--(4.851,1.623)%
  --(4.854,1.624)--(4.857,1.625)--(4.860,1.626)--(4.863,1.627)--(4.866,1.628)--(4.869,1.629)%
  --(4.872,1.630)--(4.875,1.631)--(4.878,1.632)--(4.881,1.633)--(4.884,1.634)--(4.887,1.635)%
  --(4.890,1.636)--(4.893,1.637)--(4.896,1.638)--(4.899,1.639)--(4.902,1.640)--(4.905,1.641)%
  --(4.908,1.642)--(4.911,1.643)--(4.914,1.644)--(4.917,1.645)--(4.920,1.646)--(4.923,1.647)%
  --(4.926,1.649)--(4.929,1.650)--(4.932,1.651)--(4.935,1.652)--(4.938,1.653)--(4.941,1.654)%
  --(4.944,1.655)--(4.947,1.656)--(4.950,1.657)--(4.953,1.658)--(4.956,1.659)--(4.959,1.660)%
  --(4.961,1.661)--(4.964,1.662)--(4.967,1.663)--(4.970,1.664)--(4.973,1.665)--(4.976,1.667)%
  --(4.979,1.668)--(4.982,1.669)--(4.985,1.670)--(4.988,1.671)--(4.991,1.672)--(4.994,1.673)%
  --(4.997,1.674)--(5.000,1.675)--(5.003,1.676)--(5.006,1.677)--(5.009,1.678)--(5.012,1.679)%
  --(5.015,1.681)--(5.018,1.682)--(5.021,1.683)--(5.024,1.684)--(5.027,1.685)--(5.030,1.686)%
  --(5.033,1.687)--(5.036,1.688)--(5.039,1.689)--(5.042,1.690)--(5.045,1.691)--(5.048,1.692)%
  --(5.051,1.694)--(5.054,1.695)--(5.057,1.696)--(5.060,1.697)--(5.063,1.698)--(5.066,1.699)%
  --(5.069,1.700)--(5.072,1.701)--(5.075,1.702)--(5.078,1.703)--(5.081,1.704)--(5.084,1.706)%
  --(5.087,1.707)--(5.090,1.708)--(5.093,1.709)--(5.096,1.710)--(5.099,1.711)--(5.102,1.712)%
  --(5.105,1.713)--(5.108,1.714)--(5.111,1.715)--(5.114,1.717)--(5.117,1.718)--(5.120,1.719)%
  --(5.123,1.720)--(5.126,1.721)--(5.129,1.722)--(5.132,1.723)--(5.135,1.724)--(5.138,1.725)%
  --(5.141,1.727)--(5.144,1.728)--(5.147,1.729)--(5.150,1.730)--(5.153,1.731)--(5.156,1.732)%
  --(5.159,1.733)--(5.162,1.734)--(5.165,1.736)--(5.168,1.737)--(5.171,1.738)--(5.173,1.739)%
  --(5.176,1.740)--(5.179,1.741)--(5.182,1.742)--(5.185,1.743)--(5.188,1.745)--(5.191,1.746)%
  --(5.194,1.747)--(5.197,1.748)--(5.200,1.749)--(5.203,1.750)--(5.206,1.751)--(5.209,1.752)%
  --(5.212,1.754)--(5.215,1.755)--(5.218,1.756)--(5.221,1.757)--(5.224,1.758)--(5.227,1.759)%
  --(5.230,1.760)--(5.233,1.762)--(5.236,1.763)--(5.239,1.764)--(5.242,1.765)--(5.245,1.766)%
  --(5.248,1.767)--(5.251,1.768)--(5.254,1.770)--(5.257,1.771)--(5.260,1.772)--(5.263,1.773)%
  --(5.266,1.774)--(5.269,1.775)--(5.272,1.776)--(5.275,1.778)--(5.278,1.779)--(5.281,1.780)%
  --(5.284,1.781)--(5.287,1.782)--(5.290,1.783)--(5.293,1.784)--(5.296,1.786)--(5.299,1.787)%
  --(5.302,1.788)--(5.305,1.789)--(5.308,1.790)--(5.311,1.791)--(5.314,1.793)--(5.317,1.794)%
  --(5.320,1.795)--(5.323,1.796)--(5.326,1.797)--(5.329,1.798)--(5.332,1.799)--(5.335,1.801)%
  --(5.338,1.802)--(5.341,1.803)--(5.344,1.804)--(5.347,1.805)--(5.350,1.806)--(5.353,1.808)%
  --(5.356,1.809)--(5.359,1.810)--(5.362,1.811)--(5.365,1.812)--(5.368,1.813)--(5.371,1.815)%
  --(5.374,1.816)--(5.377,1.817)--(5.380,1.818)--(5.382,1.819)--(5.385,1.821)--(5.388,1.822)%
  --(5.391,1.823)--(5.394,1.824)--(5.397,1.825)--(5.400,1.826)--(5.403,1.828)--(5.406,1.829)%
  --(5.409,1.830)--(5.412,1.831)--(5.415,1.832)--(5.418,1.833)--(5.421,1.835)--(5.424,1.836)%
  --(5.427,1.837)--(5.430,1.838)--(5.433,1.839)--(5.436,1.841)--(5.439,1.842)--(5.442,1.843)%
  --(5.445,1.844)--(5.448,1.845)--(5.451,1.847)--(5.454,1.848)--(5.457,1.849)--(5.460,1.850)%
  --(5.463,1.851)--(5.466,1.852)--(5.469,1.854)--(5.472,1.855)--(5.475,1.856)--(5.478,1.857)%
  --(5.481,1.858)--(5.484,1.860)--(5.487,1.861)--(5.490,1.862)--(5.493,1.863)--(5.496,1.864)%
  --(5.499,1.866)--(5.502,1.867)--(5.505,1.868)--(5.508,1.869)--(5.511,1.870)--(5.514,1.872)%
  --(5.517,1.873)--(5.520,1.874)--(5.523,1.875)--(5.526,1.876)--(5.529,1.878)--(5.532,1.879)%
  --(5.535,1.880)--(5.538,1.881)--(5.541,1.882)--(5.544,1.884)--(5.547,1.885)--(5.550,1.886)%
  --(5.553,1.887)--(5.556,1.888)--(5.559,1.890)--(5.562,1.891)--(5.565,1.892)--(5.568,1.893)%
  --(5.571,1.895)--(5.574,1.896)--(5.577,1.897)--(5.580,1.898)--(5.583,1.899)--(5.586,1.901)%
  --(5.589,1.902)--(5.591,1.903)--(5.594,1.904)--(5.597,1.905)--(5.600,1.907)--(5.603,1.908)%
  --(5.606,1.909)--(5.609,1.910)--(5.612,1.912)--(5.615,1.913)--(5.618,1.914)--(5.621,1.915)%
  --(5.624,1.916)--(5.627,1.918)--(5.630,1.919)--(5.633,1.920)--(5.636,1.921)--(5.639,1.923)%
  --(5.642,1.924)--(5.645,1.925)--(5.648,1.926)--(5.651,1.927)--(5.654,1.929)--(5.657,1.930)%
  --(5.660,1.931)--(5.663,1.932)--(5.666,1.934)--(5.669,1.935)--(5.672,1.936)--(5.675,1.937)%
  --(5.678,1.939)--(5.681,1.940)--(5.684,1.941)--(5.687,1.942)--(5.690,1.943)--(5.693,1.945)%
  --(5.696,1.946)--(5.699,1.947)--(5.702,1.948)--(5.705,1.950)--(5.708,1.951)--(5.711,1.952)%
  --(5.714,1.953)--(5.717,1.955)--(5.720,1.956)--(5.723,1.957)--(5.726,1.958)--(5.729,1.960)%
  --(5.732,1.961)--(5.735,1.962)--(5.738,1.963)--(5.741,1.965)--(5.744,1.966)--(5.747,1.967)%
  --(5.750,1.968)--(5.753,1.970)--(5.756,1.971)--(5.759,1.972)--(5.762,1.973)--(5.765,1.974)%
  --(5.768,1.976)--(5.771,1.977)--(5.774,1.978)--(5.777,1.979)--(5.780,1.981)--(5.783,1.982)%
  --(5.786,1.983)--(5.789,1.984)--(5.792,1.986)--(5.795,1.987)--(5.798,1.988)--(5.800,1.989)%
  --(5.803,1.991)--(5.806,1.992)--(5.809,1.993)--(5.812,1.995)--(5.815,1.996)--(5.818,1.997)%
  --(5.821,1.998)--(5.824,2.000)--(5.827,2.001)--(5.830,2.002)--(5.833,2.003)--(5.836,2.005)%
  --(5.839,2.006)--(5.842,2.007)--(5.845,2.008)--(5.848,2.010)--(5.851,2.011)--(5.854,2.012)%
  --(5.857,2.013)--(5.860,2.015)--(5.863,2.016)--(5.866,2.017)--(5.869,2.018)--(5.872,2.020)%
  --(5.875,2.021)--(5.878,2.022)--(5.881,2.023)--(5.884,2.025)--(5.887,2.026)--(5.890,2.027)%
  --(5.893,2.029)--(5.896,2.030)--(5.899,2.031)--(5.902,2.032)--(5.905,2.034)--(5.908,2.035)%
  --(5.911,2.036)--(5.914,2.037)--(5.917,2.039)--(5.920,2.040)--(5.923,2.041)--(5.926,2.043)%
  --(5.929,2.044)--(5.932,2.045)--(5.935,2.046)--(5.938,2.048)--(5.941,2.049)--(5.944,2.050)%
  --(5.947,2.051)--(5.950,2.053)--(5.953,2.054)--(5.956,2.055)--(5.959,2.057)--(5.962,2.058)%
  --(5.965,2.059)--(5.968,2.060)--(5.971,2.062)--(5.974,2.063)--(5.977,2.064)--(5.980,2.065)%
  --(5.983,2.067)--(5.986,2.068)--(5.989,2.069)--(5.992,2.071)--(5.995,2.072)--(5.998,2.073)%
  --(6.001,2.074)--(6.004,2.076)--(6.007,2.077)--(6.009,2.078)--(6.012,2.080)--(6.015,2.081)%
  --(6.018,2.082)--(6.021,2.083)--(6.024,2.085)--(6.027,2.086)--(6.030,2.087)--(6.033,2.089)%
  --(6.036,2.090)--(6.039,2.091)--(6.042,2.092)--(6.045,2.094)--(6.048,2.095)--(6.051,2.096)%
  --(6.054,2.098)--(6.057,2.099)--(6.060,2.100)--(6.063,2.101)--(6.066,2.103)--(6.069,2.104)%
  --(6.072,2.105)--(6.075,2.107)--(6.078,2.108)--(6.081,2.109)--(6.084,2.111)--(6.087,2.112)%
  --(6.090,2.113)--(6.093,2.114)--(6.096,2.116)--(6.099,2.117)--(6.102,2.118)--(6.105,2.120)%
  --(6.108,2.121)--(6.111,2.122)--(6.114,2.123)--(6.117,2.125)--(6.120,2.126)--(6.123,2.127)%
  --(6.126,2.129)--(6.129,2.130)--(6.132,2.131)--(6.135,2.133)--(6.138,2.134)--(6.141,2.135)%
  --(6.144,2.136)--(6.147,2.138)--(6.150,2.139)--(6.153,2.140)--(6.156,2.142)--(6.159,2.143)%
  --(6.162,2.144)--(6.165,2.146)--(6.168,2.147)--(6.171,2.148)--(6.174,2.150)--(6.177,2.151)%
  --(6.180,2.152)--(6.183,2.153)--(6.186,2.155)--(6.189,2.156)--(6.192,2.157)--(6.195,2.159)%
  --(6.198,2.160)--(6.201,2.161)--(6.204,2.163)--(6.207,2.164)--(6.210,2.165)--(6.213,2.167)%
  --(6.216,2.168)--(6.218,2.169)--(6.221,2.170)--(6.224,2.172)--(6.227,2.173)--(6.230,2.174)%
  --(6.233,2.176)--(6.236,2.177)--(6.239,2.178)--(6.242,2.180)--(6.245,2.181)--(6.248,2.182)%
  --(6.251,2.184)--(6.254,2.185)--(6.257,2.186)--(6.260,2.188)--(6.263,2.189)--(6.266,2.190)%
  --(6.269,2.191)--(6.272,2.193)--(6.275,2.194)--(6.278,2.195)--(6.281,2.197)--(6.284,2.198)%
  --(6.287,2.199)--(6.290,2.201)--(6.293,2.202)--(6.296,2.203)--(6.299,2.205)--(6.302,2.206)%
  --(6.305,2.207)--(6.308,2.209)--(6.311,2.210)--(6.314,2.211)--(6.317,2.213)--(6.320,2.214)%
  --(6.323,2.215)--(6.326,2.217)--(6.329,2.218)--(6.332,2.219)--(6.335,2.221)--(6.338,2.222)%
  --(6.341,2.223)--(6.344,2.224)--(6.347,2.226)--(6.350,2.227)--(6.353,2.228)--(6.356,2.230)%
  --(6.359,2.231)--(6.362,2.232)--(6.365,2.234)--(6.368,2.235)--(6.371,2.236)--(6.374,2.238)%
  --(6.377,2.239)--(6.380,2.240)--(6.383,2.242)--(6.386,2.243)--(6.389,2.244)--(6.392,2.246)%
  --(6.395,2.247)--(6.398,2.248)--(6.401,2.250)--(6.404,2.251)--(6.407,2.252)--(6.410,2.254)%
  --(6.413,2.255)--(6.416,2.256)--(6.419,2.258)--(6.422,2.259)--(6.425,2.260)--(6.428,2.262)%
  --(6.430,2.263)--(6.433,2.264)--(6.436,2.266)--(6.439,2.267)--(6.442,2.268)--(6.445,2.270)%
  --(6.448,2.271)--(6.451,2.272)--(6.454,2.274)--(6.457,2.275)--(6.460,2.276)--(6.463,2.278)%
  --(6.466,2.279)--(6.469,2.280)--(6.472,2.282)--(6.475,2.283)--(6.478,2.284)--(6.481,2.286)%
  --(6.484,2.287)--(6.487,2.288)--(6.490,2.290)--(6.493,2.291)--(6.496,2.292)--(6.499,2.294)%
  --(6.502,2.295)--(6.505,2.297)--(6.508,2.298)--(6.511,2.299)--(6.514,2.301)--(6.517,2.302)%
  --(6.520,2.303)--(6.523,2.305)--(6.526,2.306)--(6.529,2.307)--(6.532,2.309)--(6.535,2.310)%
  --(6.538,2.311)--(6.541,2.313)--(6.544,2.314)--(6.547,2.315)--(6.550,2.317)--(6.553,2.318)%
  --(6.556,2.319)--(6.559,2.321)--(6.562,2.322)--(6.565,2.323)--(6.568,2.325)--(6.571,2.326)%
  --(6.574,2.327)--(6.577,2.329)--(6.580,2.330)--(6.583,2.331)--(6.586,2.333)--(6.589,2.334)%
  --(6.592,2.336)--(6.595,2.337)--(6.598,2.338)--(6.601,2.340)--(6.604,2.341)--(6.607,2.342)%
  --(6.610,2.344)--(6.613,2.345)--(6.616,2.346)--(6.619,2.348)--(6.622,2.349)--(6.625,2.350)%
  --(6.628,2.352)--(6.631,2.353)--(6.634,2.354)--(6.637,2.356)--(6.639,2.357)--(6.642,2.359)%
  --(6.645,2.360)--(6.648,2.361)--(6.651,2.363)--(6.654,2.364)--(6.657,2.365)--(6.660,2.367)%
  --(6.663,2.368)--(6.666,2.369)--(6.669,2.371)--(6.672,2.372)--(6.675,2.373)--(6.678,2.375)%
  --(6.681,2.376)--(6.684,2.378)--(6.687,2.379)--(6.690,2.380)--(6.693,2.382)--(6.696,2.383)%
  --(6.699,2.384)--(6.702,2.386)--(6.705,2.387)--(6.708,2.388)--(6.711,2.390)--(6.714,2.391)%
  --(6.717,2.392)--(6.720,2.394)--(6.723,2.395)--(6.726,2.397)--(6.729,2.398)--(6.732,2.399)%
  --(6.735,2.401)--(6.738,2.402)--(6.741,2.403)--(6.744,2.405)--(6.747,2.406)--(6.750,2.407)%
  --(6.753,2.409)--(6.756,2.410)--(6.759,2.412)--(6.762,2.413)--(6.765,2.414)--(6.768,2.416)%
  --(6.771,2.417)--(6.774,2.418)--(6.777,2.420)--(6.780,2.421)--(6.783,2.422)--(6.786,2.424)%
  --(6.789,2.425)--(6.792,2.427)--(6.795,2.428)--(6.798,2.429)--(6.801,2.431)--(6.804,2.432)%
  --(6.807,2.433)--(6.810,2.435)--(6.813,2.436)--(6.816,2.438)--(6.819,2.439)--(6.822,2.440)%
  --(6.825,2.442)--(6.828,2.443)--(6.831,2.444)--(6.834,2.446)--(6.837,2.447)--(6.840,2.448)%
  --(6.843,2.450)--(6.846,2.451)--(6.848,2.453)--(6.851,2.454)--(6.854,2.455)--(6.857,2.457)%
  --(6.860,2.458)--(6.863,2.459)--(6.866,2.461)--(6.869,2.462)--(6.872,2.464)--(6.875,2.465)%
  --(6.878,2.466)--(6.881,2.468)--(6.884,2.469)--(6.887,2.470)--(6.890,2.472)--(6.893,2.473)%
  --(6.896,2.475)--(6.899,2.476)--(6.902,2.477)--(6.905,2.479)--(6.908,2.480)--(6.911,2.481)%
  --(6.914,2.483)--(6.917,2.484)--(6.920,2.486)--(6.923,2.487)--(6.926,2.488)--(6.929,2.490)%
  --(6.932,2.491)--(6.935,2.492)--(6.938,2.494)--(6.941,2.495)--(6.944,2.497)--(6.947,2.498)%
  --(6.950,2.499)--(6.953,2.501)--(6.956,2.502)--(6.959,2.504)--(6.962,2.505)--(6.965,2.506)%
  --(6.968,2.508)--(6.971,2.509)--(6.974,2.510)--(6.977,2.512)--(6.980,2.513)--(6.983,2.515)%
  --(6.986,2.516)--(6.989,2.517)--(6.992,2.519)--(6.995,2.520)--(6.998,2.521)--(7.001,2.523)%
  --(7.004,2.524)--(7.007,2.526)--(7.010,2.527)--(7.013,2.528)--(7.016,2.530)--(7.019,2.531)%
  --(7.022,2.533)--(7.025,2.534)--(7.028,2.535)--(7.031,2.537)--(7.034,2.538)--(7.037,2.539)%
  --(7.040,2.541)--(7.043,2.542)--(7.046,2.544)--(7.049,2.545)--(7.052,2.546)--(7.055,2.548)%
  --(7.057,2.549)--(7.060,2.551)--(7.063,2.552)--(7.066,2.553)--(7.069,2.555)--(7.072,2.556)%
  --(7.075,2.557)--(7.078,2.559)--(7.081,2.560)--(7.084,2.562)--(7.087,2.563)--(7.090,2.564)%
  --(7.093,2.566)--(7.096,2.567)--(7.099,2.569)--(7.102,2.570)--(7.105,2.571)--(7.108,2.573)%
  --(7.111,2.574)--(7.114,2.576)--(7.117,2.577)--(7.120,2.578)--(7.123,2.580)--(7.126,2.581)%
  --(7.129,2.583)--(7.132,2.584)--(7.135,2.585)--(7.138,2.587)--(7.141,2.588)--(7.144,2.589)%
  --(7.147,2.591)--(7.150,2.592)--(7.153,2.594)--(7.156,2.595)--(7.159,2.596)--(7.162,2.598)%
  --(7.165,2.599)--(7.168,2.601)--(7.171,2.602)--(7.174,2.603)--(7.177,2.605)--(7.180,2.606)%
  --(7.183,2.608)--(7.186,2.609)--(7.189,2.610)--(7.192,2.612)--(7.195,2.613)--(7.198,2.615)%
  --(7.201,2.616)--(7.204,2.617)--(7.207,2.619)--(7.210,2.620)--(7.213,2.622)--(7.216,2.623)%
  --(7.219,2.624)--(7.222,2.626)--(7.225,2.627)--(7.228,2.629)--(7.231,2.630)--(7.234,2.631)%
  --(7.237,2.633)--(7.240,2.634)--(7.243,2.636)--(7.246,2.637)--(7.249,2.638)--(7.252,2.640)%
  --(7.255,2.641)--(7.258,2.643)--(7.261,2.644)--(7.264,2.645)--(7.266,2.647)--(7.269,2.648)%
  --(7.272,2.650)--(7.275,2.651)--(7.278,2.652)--(7.281,2.654)--(7.284,2.655)--(7.287,2.657)%
  --(7.290,2.658)--(7.293,2.659)--(7.296,2.661)--(7.299,2.662)--(7.302,2.664)--(7.305,2.665)%
  --(7.308,2.666)--(7.311,2.668)--(7.314,2.669)--(7.317,2.671)--(7.320,2.672)--(7.323,2.673)%
  --(7.326,2.675)--(7.329,2.676)--(7.332,2.678)--(7.335,2.679)--(7.338,2.680)--(7.341,2.682)%
  --(7.344,2.683)--(7.347,2.685)--(7.350,2.686)--(7.353,2.687)--(7.356,2.689)--(7.359,2.690)%
  --(7.362,2.692)--(7.365,2.693)--(7.368,2.694)--(7.371,2.696)--(7.374,2.697)--(7.377,2.699)%
  --(7.380,2.700)--(7.383,2.702)--(7.386,2.703)--(7.389,2.704)--(7.392,2.706)--(7.395,2.707)%
  --(7.398,2.709)--(7.401,2.710)--(7.404,2.711)--(7.407,2.713)--(7.410,2.714)--(7.413,2.716)%
  --(7.416,2.717)--(7.419,2.718)--(7.422,2.720)--(7.425,2.721)--(7.428,2.723)--(7.431,2.724)%
  --(7.434,2.725)--(7.437,2.727)--(7.440,2.728)--(7.443,2.730)--(7.446,2.731)--(7.449,2.733)%
  --(7.452,2.734)--(7.455,2.735)--(7.458,2.737)--(7.461,2.738)--(7.464,2.740)--(7.467,2.741)%
  --(7.470,2.742)--(7.473,2.744)--(7.476,2.745)--(7.478,2.747)--(7.481,2.748)--(7.484,2.749)%
  --(7.487,2.751)--(7.490,2.752)--(7.493,2.754)--(7.496,2.755)--(7.499,2.757)--(7.502,2.758)%
  --(7.505,2.759)--(7.508,2.761)--(7.511,2.762)--(7.514,2.764)--(7.517,2.765)--(7.520,2.766)%
  --(7.523,2.768)--(7.526,2.769)--(7.529,2.771)--(7.532,2.772)--(7.535,2.773)--(7.538,2.775)%
  --(7.541,2.776)--(7.544,2.778)--(7.547,2.779)--(7.550,2.781)--(7.553,2.782)--(7.556,2.783)%
  --(7.559,2.785)--(7.562,2.786)--(7.565,2.788)--(7.568,2.789)--(7.571,2.790)--(7.574,2.792)%
  --(7.577,2.793)--(7.580,2.795)--(7.583,2.796)--(7.586,2.798)--(7.589,2.799)--(7.592,2.800)%
  --(7.595,2.802)--(7.598,2.803)--(7.601,2.805)--(7.604,2.806)--(7.607,2.807)--(7.610,2.809)%
  --(7.613,2.810)--(7.616,2.812)--(7.619,2.813)--(7.622,2.815)--(7.625,2.816)--(7.628,2.817)%
  --(7.631,2.819)--(7.634,2.820)--(7.637,2.822)--(7.640,2.823)--(7.643,2.825)--(7.646,2.826)%
  --(7.649,2.827)--(7.652,2.829)--(7.655,2.830)--(7.658,2.832)--(7.661,2.833)--(7.664,2.834)%
  --(7.667,2.836)--(7.670,2.837)--(7.673,2.839)--(7.676,2.840)--(7.679,2.842)--(7.682,2.843)%
  --(7.685,2.844)--(7.687,2.846)--(7.690,2.847)--(7.693,2.849)--(7.696,2.850)--(7.699,2.852)%
  --(7.702,2.853)--(7.705,2.854)--(7.708,2.856)--(7.711,2.857)--(7.714,2.859)--(7.717,2.860)%
  --(7.720,2.862)--(7.723,2.863)--(7.726,2.864)--(7.729,2.866)--(7.732,2.867)--(7.735,2.869)%
  --(7.738,2.870)--(7.741,2.871)--(7.744,2.873)--(7.747,2.874)--(7.750,2.876)--(7.753,2.877)%
  --(7.756,2.879)--(7.759,2.880)--(7.762,2.881)--(7.765,2.883)--(7.768,2.884)--(7.771,2.886)%
  --(7.774,2.887)--(7.777,2.889)--(7.780,2.890)--(7.783,2.891)--(7.786,2.893)--(7.789,2.894)%
  --(7.792,2.896)--(7.795,2.897)--(7.798,2.899)--(7.801,2.900)--(7.804,2.901)--(7.807,2.903)%
  --(7.810,2.904)--(7.813,2.906)--(7.816,2.907)--(7.819,2.909)--(7.822,2.910)--(7.825,2.911)%
  --(7.828,2.913)--(7.831,2.914)--(7.834,2.916)--(7.837,2.917)--(7.840,2.919)--(7.843,2.920)%
  --(7.846,2.921)--(7.849,2.923)--(7.852,2.924)--(7.855,2.926)--(7.858,2.927)--(7.861,2.929)%
  --(7.864,2.930)--(7.867,2.931)--(7.870,2.933)--(7.873,2.934)--(7.876,2.936)--(7.879,2.937)%
  --(7.882,2.939)--(7.885,2.940)--(7.888,2.942)--(7.891,2.943)--(7.894,2.944)--(7.896,2.946)%
  --(7.899,2.947)--(7.902,2.949)--(7.905,2.950)--(7.908,2.952)--(7.911,2.953)--(7.914,2.954)%
  --(7.917,2.956)--(7.920,2.957)--(7.923,2.959)--(7.926,2.960)--(7.929,2.962)--(7.932,2.963)%
  --(7.935,2.964)--(7.938,2.966)--(7.941,2.967)--(7.944,2.969)--(7.947,2.970)--(7.950,2.972)%
  --(7.953,2.973)--(7.956,2.974)--(7.959,2.976)--(7.962,2.977)--(7.965,2.979)--(7.968,2.980)%
  --(7.971,2.982)--(7.974,2.983)--(7.977,2.985)--(7.980,2.986)--(7.983,2.987)--(7.986,2.989)%
  --(7.989,2.990)--(7.992,2.992)--(7.995,2.993)--(7.998,2.995)--(8.001,2.996)--(8.004,2.997)%
  --(8.007,2.999)--(8.010,3.000)--(8.013,3.002)--(8.016,3.003)--(8.019,3.005)--(8.022,3.006)%
  --(8.025,3.007)--(8.028,3.009)--(8.031,3.010)--(8.034,3.012)--(8.037,3.013)--(8.040,3.015)%
  --(8.043,3.016)--(8.046,3.018)--(8.049,3.019)--(8.052,3.020)--(8.055,3.022)--(8.058,3.023)%
  --(8.061,3.025)--(8.064,3.026)--(8.067,3.028)--(8.070,3.029)--(8.073,3.031)--(8.076,3.032)%
  --(8.079,3.033)--(8.082,3.035)--(8.085,3.036)--(8.088,3.038)--(8.091,3.039)--(8.094,3.041)%
  --(8.097,3.042)--(8.100,3.043)--(8.103,3.045)--(8.105,3.046)--(8.108,3.048)--(8.111,3.049)%
  --(8.114,3.051)--(8.117,3.052)--(8.120,3.054)--(8.123,3.055)--(8.126,3.056)--(8.129,3.058)%
  --(8.132,3.059)--(8.135,3.061)--(8.138,3.062)--(8.141,3.064)--(8.144,3.065)--(8.147,3.067)%
  --(8.150,3.068)--(8.153,3.069)--(8.156,3.071)--(8.159,3.072)--(8.162,3.074)--(8.165,3.075)%
  --(8.168,3.077)--(8.171,3.078)--(8.174,3.080)--(8.177,3.081)--(8.180,3.082)--(8.183,3.084)%
  --(8.186,3.085)--(8.189,3.087)--(8.192,3.088)--(8.195,3.090)--(8.198,3.091)--(8.201,3.092)%
  --(8.204,3.094)--(8.207,3.095)--(8.210,3.097)--(8.213,3.098)--(8.216,3.100)--(8.219,3.101)%
  --(8.222,3.103)--(8.225,3.104)--(8.228,3.105)--(8.231,3.107)--(8.234,3.108)--(8.237,3.110)%
  --(8.240,3.111)--(8.243,3.113)--(8.246,3.114)--(8.249,3.116)--(8.252,3.117)--(8.255,3.118)%
  --(8.258,3.120)--(8.261,3.121)--(8.264,3.123)--(8.267,3.124)--(8.270,3.126)--(8.273,3.127)%
  --(8.276,3.129)--(8.279,3.130)--(8.282,3.132)--(8.285,3.133)--(8.288,3.134)--(8.291,3.136)%
  --(8.294,3.137)--(8.297,3.139)--(8.300,3.140)--(8.303,3.142)--(8.306,3.143)--(8.309,3.145)%
  --(8.312,3.146)--(8.314,3.147)--(8.317,3.149)--(8.320,3.150)--(8.323,3.152)--(8.326,3.153)%
  --(8.329,3.155)--(8.332,3.156)--(8.335,3.158)--(8.338,3.159)--(8.341,3.160)--(8.344,3.162)%
  --(8.347,3.163)--(8.350,3.165)--(8.353,3.166)--(8.356,3.168)--(8.359,3.169)--(8.362,3.171)%
  --(8.365,3.172)--(8.368,3.174)--(8.371,3.175)--(8.374,3.176)--(8.377,3.178)--(8.380,3.179)%
  --(8.383,3.181)--(8.386,3.182)--(8.389,3.184)--(8.392,3.185)--(8.395,3.187)--(8.398,3.188)%
  --(8.401,3.189)--(8.404,3.191)--(8.407,3.192)--(8.410,3.194)--(8.413,3.195)--(8.416,3.197)%
  --(8.419,3.198)--(8.422,3.200)--(8.425,3.201)--(8.428,3.203)--(8.431,3.204)--(8.434,3.205)%
  --(8.437,3.207)--(8.440,3.208)--(8.443,3.210)--(8.446,3.211)--(8.449,3.213)--(8.452,3.214)%
  --(8.455,3.216)--(8.458,3.217)--(8.461,3.218)--(8.464,3.220)--(8.467,3.221)--(8.470,3.223)%
  --(8.473,3.224)--(8.476,3.226)--(8.479,3.227)--(8.482,3.229)--(8.485,3.230)--(8.488,3.232)%
  --(8.491,3.233)--(8.494,3.234)--(8.497,3.236)--(8.500,3.237)--(8.503,3.239)--(8.506,3.240)%
  --(8.509,3.242)--(8.512,3.243)--(8.515,3.245)--(8.518,3.246)--(8.521,3.248)--(8.523,3.249)%
  --(8.526,3.250)--(8.529,3.252)--(8.532,3.253)--(8.535,3.255)--(8.538,3.256)--(8.541,3.258)%
  --(8.544,3.259)--(8.547,3.261)--(8.550,3.262)--(8.553,3.264)--(8.556,3.265)--(8.559,3.266)%
  --(8.562,3.268)--(8.565,3.269)--(8.568,3.271)--(8.571,3.272)--(8.574,3.274)--(8.577,3.275)%
  --(8.580,3.277)--(8.583,3.278)--(8.586,3.280)--(8.589,3.281)--(8.592,3.282)--(8.595,3.284)%
  --(8.598,3.285)--(8.601,3.287)--(8.604,3.288)--(8.607,3.290)--(8.610,3.291)--(8.613,3.293)%
  --(8.616,3.294)--(8.619,3.296)--(8.622,3.297)--(8.625,3.299)--(8.628,3.300)--(8.631,3.301)%
  --(8.634,3.303)--(8.637,3.304)--(8.640,3.306)--(8.643,3.307)--(8.646,3.309)--(8.649,3.310)%
  --(8.652,3.312)--(8.655,3.313)--(8.658,3.315)--(8.661,3.316)--(8.664,3.317)--(8.667,3.319)%
  --(8.670,3.320)--(8.673,3.322)--(8.676,3.323)--(8.679,3.325)--(8.682,3.326)--(8.685,3.328)%
  --(8.688,3.329)--(8.691,3.331)--(8.694,3.332)--(8.697,3.334)--(8.700,3.335)--(8.703,3.336)%
  --(8.706,3.338)--(8.709,3.339)--(8.712,3.341)--(8.715,3.342)--(8.718,3.344)--(8.721,3.345)%
  --(8.724,3.347)--(8.727,3.348)--(8.730,3.350)--(8.733,3.351)--(8.735,3.352)--(8.738,3.354)%
  --(8.741,3.355)--(8.744,3.357)--(8.747,3.358)--(8.750,3.360)--(8.753,3.361)--(8.756,3.363)%
  --(8.759,3.364)--(8.762,3.366)--(8.765,3.367)--(8.768,3.369)--(8.771,3.370)--(8.774,3.371)%
  --(8.777,3.373)--(8.780,3.374)--(8.783,3.376)--(8.786,3.377)--(8.789,3.379)--(8.792,3.380)%
  --(8.795,3.382)--(8.798,3.383)--(8.801,3.385)--(8.804,3.386)--(8.807,3.388)--(8.810,3.389)%
  --(8.813,3.390)--(8.816,3.392)--(8.819,3.393)--(8.822,3.395)--(8.825,3.396)--(8.828,3.398)%
  --(8.831,3.399)--(8.834,3.401)--(8.837,3.402)--(8.840,3.404)--(8.843,3.405)--(8.846,3.407)%
  --(8.849,3.408)--(8.852,3.409)--(8.855,3.411)--(8.858,3.412)--(8.861,3.414)--(8.864,3.415)%
  --(8.867,3.417)--(8.870,3.418)--(8.873,3.420)--(8.876,3.421)--(8.879,3.423)--(8.882,3.424)%
  --(8.885,3.426)--(8.888,3.427)--(8.891,3.429)--(8.894,3.430)--(8.897,3.431)--(8.900,3.433)%
  --(8.903,3.434)--(8.906,3.436)--(8.909,3.437)--(8.912,3.439)--(8.915,3.440)--(8.918,3.442)%
  --(8.921,3.443)--(8.924,3.445)--(8.927,3.446)--(8.930,3.448)--(8.933,3.449)--(8.936,3.450)%
  --(8.939,3.452)--(8.942,3.453)--(8.944,3.455)--(8.947,3.456)--(8.950,3.458)--(8.953,3.459)%
  --(8.956,3.461)--(8.959,3.462)--(8.962,3.464)--(8.965,3.465)--(8.968,3.467)--(8.971,3.468)%
  --(8.974,3.470)--(8.977,3.471)--(8.980,3.472)--(8.983,3.474)--(8.986,3.475)--(8.989,3.477)%
  --(8.992,3.478)--(8.995,3.480)--(8.998,3.481)--(9.001,3.483)--(9.004,3.484)--(9.007,3.486)%
  --(9.010,3.487)--(9.013,3.489)--(9.016,3.490)--(9.019,3.492)--(9.022,3.493)--(9.025,3.494)%
  --(9.028,3.496)--(9.031,3.497)--(9.034,3.499)--(9.037,3.500)--(9.040,3.502)--(9.043,3.503)%
  --(9.046,3.505)--(9.049,3.506)--(9.052,3.508)--(9.055,3.509)--(9.058,3.511)--(9.061,3.512)%
  --(9.064,3.514)--(9.067,3.515)--(9.070,3.517)--(9.073,3.518)--(9.076,3.519)--(9.079,3.521)%
  --(9.082,3.522)--(9.085,3.524)--(9.088,3.525)--(9.091,3.527)--(9.094,3.528)--(9.097,3.530)%
  --(9.100,3.531)--(9.103,3.533)--(9.106,3.534)--(9.109,3.536)--(9.112,3.537)--(9.115,3.539)%
  --(9.118,3.540)--(9.121,3.541)--(9.124,3.543)--(9.127,3.544)--(9.130,3.546)--(9.133,3.547)%
  --(9.136,3.549)--(9.139,3.550)--(9.142,3.552)--(9.145,3.553)--(9.148,3.555)--(9.151,3.556)%
  --(9.153,3.558)--(9.156,3.559)--(9.159,3.561)--(9.162,3.562)--(9.165,3.564)--(9.168,3.565)%
  --(9.171,3.566)--(9.174,3.568)--(9.177,3.569)--(9.180,3.571)--(9.183,3.572)--(9.186,3.574)%
  --(9.189,3.575)--(9.192,3.577)--(9.195,3.578)--(9.198,3.580)--(9.201,3.581)--(9.204,3.583)%
  --(9.207,3.584)--(9.210,3.586)--(9.213,3.587)--(9.216,3.589)--(9.219,3.590)--(9.222,3.591)%
  --(9.225,3.593)--(9.228,3.594)--(9.231,3.596)--(9.234,3.597)--(9.237,3.599)--(9.240,3.600)%
  --(9.243,3.602)--(9.246,3.603)--(9.249,3.605)--(9.252,3.606)--(9.255,3.608)--(9.258,3.609)%
  --(9.261,3.611)--(9.264,3.612)--(9.267,3.614)--(9.270,3.615)--(9.273,3.617)--(9.276,3.618)%
  --(9.279,3.619)--(9.282,3.621)--(9.285,3.622)--(9.288,3.624)--(9.291,3.625)--(9.294,3.627)%
  --(9.297,3.628)--(9.300,3.630)--(9.303,3.631)--(9.306,3.633)--(9.309,3.634)--(9.312,3.636)%
  --(9.315,3.637)--(9.318,3.639)--(9.321,3.640)--(9.324,3.642)--(9.327,3.643)--(9.330,3.645)%
  --(9.333,3.646)--(9.336,3.647)--(9.339,3.649)--(9.342,3.650)--(9.345,3.652)--(9.348,3.653)%
  --(9.351,3.655)--(9.354,3.656)--(9.357,3.658)--(9.360,3.659)--(9.362,3.661)--(9.365,3.662)%
  --(9.368,3.664)--(9.371,3.665)--(9.374,3.667)--(9.377,3.668)--(9.380,3.670)--(9.383,3.671)%
  --(9.386,3.673)--(9.389,3.674)--(9.392,3.675)--(9.395,3.677)--(9.398,3.678)--(9.401,3.680)%
  --(9.404,3.681)--(9.407,3.683)--(9.410,3.684)--(9.413,3.686)--(9.416,3.687)--(9.419,3.689)%
  --(9.422,3.690)--(9.425,3.692)--(9.428,3.693)--(9.431,3.695)--(9.434,3.696)--(9.437,3.698)%
  --(9.440,3.699)--(9.443,3.701)--(9.446,3.702)--(9.449,3.704)--(9.452,3.705)--(9.455,3.706)%
  --(9.458,3.708)--(9.461,3.709)--(9.464,3.711)--(9.467,3.712)--(9.470,3.714)--(9.473,3.715)%
  --(9.476,3.717)--(9.479,3.718)--(9.482,3.720)--(9.485,3.721)--(9.488,3.723)--(9.491,3.724)%
  --(9.494,3.726)--(9.497,3.727)--(9.500,3.729)--(9.503,3.730)--(9.506,3.732)--(9.509,3.733)%
  --(9.512,3.735)--(9.515,3.736)--(9.518,3.737)--(9.521,3.739)--(9.524,3.740)--(9.527,3.742)%
  --(9.530,3.743)--(9.533,3.745)--(9.536,3.746)--(9.539,3.748)--(9.542,3.749)--(9.545,3.751)%
  --(9.548,3.752)--(9.551,3.754)--(9.554,3.755)--(9.557,3.757)--(9.560,3.758)--(9.563,3.760)%
  --(9.566,3.761)--(9.569,3.763)--(9.571,3.764)--(9.574,3.766)--(9.577,3.767)--(9.580,3.769)%
  --(9.583,3.770)--(9.586,3.771)--(9.589,3.773)--(9.592,3.774)--(9.595,3.776)--(9.598,3.777)%
  --(9.601,3.779)--(9.604,3.780)--(9.607,3.782)--(9.610,3.783)--(9.613,3.785)--(9.616,3.786)%
  --(9.619,3.788)--(9.622,3.789)--(9.625,3.791)--(9.628,3.792)--(9.631,3.794)--(9.634,3.795)%
  --(9.637,3.797)--(9.640,3.798)--(9.643,3.800)--(9.646,3.801)--(9.649,3.803)--(9.652,3.804)%
  --(9.655,3.806)--(9.658,3.807)--(9.661,3.808)--(9.664,3.810)--(9.667,3.811)--(9.670,3.813)%
  --(9.673,3.814)--(9.676,3.816)--(9.679,3.817)--(9.682,3.819)--(9.685,3.820)--(9.688,3.822)%
  --(9.691,3.823)--(9.694,3.825)--(9.697,3.826)--(9.700,3.828)--(9.703,3.829)--(9.706,3.831)%
  --(9.709,3.832)--(9.712,3.834)--(9.715,3.835)--(9.718,3.837)--(9.721,3.838)--(9.724,3.840)%
  --(9.727,3.841)--(9.730,3.843)--(9.733,3.844)--(9.736,3.845)--(9.739,3.847)--(9.742,3.848)%
  --(9.745,3.850)--(9.748,3.851)--(9.751,3.853)--(9.754,3.854)--(9.757,3.856)--(9.760,3.857)%
  --(9.763,3.859)--(9.766,3.860)--(9.769,3.862)--(9.772,3.863)--(9.775,3.865)--(9.778,3.866)%
  --(9.780,3.868)--(9.783,3.869)--(9.786,3.871)--(9.789,3.872)--(9.792,3.874)--(9.795,3.875)%
  --(9.798,3.877)--(9.801,3.878)--(9.804,3.880)--(9.807,3.881)--(9.810,3.883)--(9.813,3.884)%
  --(9.816,3.885)--(9.819,3.887)--(9.822,3.888)--(9.825,3.890)--(9.828,3.891)--(9.831,3.893)%
  --(9.834,3.894)--(9.837,3.896)--(9.840,3.897)--(9.843,3.899)--(9.846,3.900)--(9.849,3.902)%
  --(9.852,3.903)--(9.855,3.905)--(9.858,3.906)--(9.861,3.908)--(9.864,3.909)--(9.867,3.911)%
  --(9.870,3.912)--(9.873,3.914)--(9.876,3.915)--(9.879,3.917)--(9.882,3.918)--(9.885,3.920)%
  --(9.888,3.921)--(9.891,3.923)--(9.894,3.924)--(9.897,3.926)--(9.900,3.927)--(9.903,3.929)%
  --(9.906,3.930)--(9.909,3.931)--(9.912,3.933)--(9.915,3.934)--(9.918,3.936)--(9.921,3.937)%
  --(9.924,3.939)--(9.927,3.940)--(9.930,3.942)--(9.933,3.943)--(9.936,3.945)--(9.939,3.946)%
  --(9.942,3.948)--(9.945,3.949)--(9.948,3.951)--(9.951,3.952)--(9.954,3.954)--(9.957,3.955)%
  --(9.960,3.957)--(9.963,3.958)--(9.966,3.960)--(9.969,3.961)--(9.972,3.963)--(9.975,3.964)%
  --(9.978,3.966)--(9.981,3.967)--(9.984,3.969)--(9.987,3.970)--(9.990,3.972)--(9.992,3.973)%
  --(9.995,3.975)--(9.998,3.976)--(10.001,3.978)--(10.004,3.979)--(10.007,3.980)--(10.010,3.982)%
  --(10.013,3.983)--(10.016,3.985)--(10.019,3.986)--(10.022,3.988)--(10.025,3.989)--(10.028,3.991)%
  --(10.031,3.992)--(10.034,3.994)--(10.037,3.995)--(10.040,3.997)--(10.043,3.998)--(10.046,4.000)%
  --(10.049,4.001)--(10.052,4.003)--(10.055,4.004)--(10.058,4.006)--(10.061,4.007)--(10.064,4.009)%
  --(10.067,4.010)--(10.070,4.012)--(10.073,4.013)--(10.076,4.015)--(10.079,4.016)--(10.082,4.018)%
  --(10.085,4.019)--(10.088,4.021)--(10.091,4.022)--(10.094,4.024)--(10.097,4.025)--(10.100,4.027)%
  --(10.103,4.028)--(10.106,4.030)--(10.109,4.031)--(10.112,4.033)--(10.115,4.034)--(10.118,4.035)%
  --(10.121,4.037)--(10.124,4.038)--(10.127,4.040)--(10.130,4.041)--(10.133,4.043)--(10.136,4.044)%
  --(10.139,4.046)--(10.142,4.047)--(10.145,4.049)--(10.148,4.050)--(10.151,4.052)--(10.154,4.053)%
  --(10.157,4.055)--(10.160,4.056)--(10.163,4.058)--(10.166,4.059)--(10.169,4.061)--(10.172,4.062)%
  --(10.175,4.064)--(10.178,4.065)--(10.181,4.067)--(10.184,4.068)--(10.187,4.070)--(10.190,4.071)%
  --(10.193,4.073)--(10.196,4.074)--(10.199,4.076)--(10.201,4.077)--(10.204,4.079)--(10.207,4.080)%
  --(10.210,4.082)--(10.213,4.083)--(10.216,4.085)--(10.219,4.086)--(10.222,4.088)--(10.225,4.089)%
  --(10.228,4.091)--(10.231,4.092)--(10.234,4.094)--(10.237,4.095)--(10.240,4.096)--(10.243,4.098)%
  --(10.246,4.099)--(10.249,4.101)--(10.252,4.102)--(10.255,4.104)--(10.258,4.105)--(10.261,4.107)%
  --(10.264,4.108)--(10.267,4.110)--(10.270,4.111)--(10.273,4.113)--(10.276,4.114)--(10.279,4.116)%
  --(10.282,4.117)--(10.285,4.119)--(10.288,4.120)--(10.291,4.122)--(10.294,4.123)--(10.297,4.125)%
  --(10.300,4.126)--(10.303,4.128)--(10.306,4.129)--(10.309,4.131)--(10.312,4.132)--(10.315,4.134)%
  --(10.318,4.135)--(10.321,4.137)--(10.324,4.138)--(10.327,4.140)--(10.330,4.141)--(10.333,4.143)%
  --(10.336,4.144)--(10.339,4.146)--(10.342,4.147)--(10.345,4.149)--(10.348,4.150)--(10.351,4.152)%
  --(10.354,4.153)--(10.357,4.155)--(10.360,4.156)--(10.363,4.158)--(10.366,4.159)--(10.369,4.161)%
  --(10.372,4.162)--(10.375,4.164)--(10.378,4.165)--(10.381,4.167)--(10.384,4.168)--(10.387,4.169)%
  --(10.390,4.171)--(10.393,4.172)--(10.396,4.174)--(10.399,4.175)--(10.402,4.177)--(10.405,4.178)%
  --(10.408,4.180)--(10.410,4.181)--(10.413,4.183)--(10.416,4.184)--(10.419,4.186)--(10.422,4.187)%
  --(10.425,4.189)--(10.428,4.190)--(10.431,4.192)--(10.434,4.193)--(10.437,4.195)--(10.440,4.196)%
  --(10.443,4.198)--(10.446,4.199)--(10.449,4.201)--(10.452,4.202)--(10.455,4.204)--(10.458,4.205)%
  --(10.461,4.207)--(10.464,4.208)--(10.467,4.210)--(10.470,4.211)--(10.473,4.213)--(10.476,4.214)%
  --(10.479,4.216)--(10.482,4.217)--(10.485,4.219)--(10.488,4.220)--(10.491,4.222)--(10.494,4.223)%
  --(10.497,4.225)--(10.500,4.226)--(10.503,4.228)--(10.506,4.229)--(10.509,4.231)--(10.512,4.232)%
  --(10.515,4.234)--(10.518,4.235)--(10.521,4.237)--(10.524,4.238)--(10.527,4.240)--(10.530,4.241)%
  --(10.533,4.243)--(10.536,4.244)--(10.539,4.246)--(10.542,4.247)--(10.545,4.249)--(10.548,4.250)%
  --(10.551,4.252)--(10.554,4.253)--(10.557,4.255)--(10.560,4.256)--(10.563,4.258)--(10.566,4.259)%
  --(10.569,4.260)--(10.572,4.262)--(10.575,4.263)--(10.578,4.265)--(10.581,4.266)--(10.584,4.268)%
  --(10.587,4.269)--(10.590,4.271)--(10.593,4.272)--(10.596,4.274)--(10.599,4.275)--(10.602,4.277)%
  --(10.605,4.278)--(10.608,4.280)--(10.611,4.281)--(10.614,4.283)--(10.617,4.284)--(10.619,4.286)%
  --(10.622,4.287)--(10.625,4.289)--(10.628,4.290)--(10.631,4.292)--(10.634,4.293)--(10.637,4.295)%
  --(10.640,4.296)--(10.643,4.298)--(10.646,4.299)--(10.649,4.301)--(10.652,4.302)--(10.655,4.304)%
  --(10.658,4.305)--(10.661,4.307)--(10.664,4.308)--(10.667,4.310)--(10.670,4.311)--(10.673,4.313)%
  --(10.676,4.314)--(10.679,4.316)--(10.682,4.317)--(10.685,4.319)--(10.688,4.320)--(10.691,4.322)%
  --(10.694,4.323)--(10.697,4.325)--(10.700,4.326)--(10.703,4.328)--(10.706,4.329)--(10.709,4.331)%
  --(10.712,4.332)--(10.715,4.334)--(10.718,4.335)--(10.721,4.337)--(10.724,4.338)--(10.727,4.340)%
  --(10.730,4.341)--(10.733,4.343)--(10.736,4.344)--(10.739,4.346)--(10.742,4.347)--(10.745,4.349)%
  --(10.748,4.350)--(10.751,4.352)--(10.754,4.353)--(10.757,4.355)--(10.760,4.356)--(10.763,4.358)%
  --(10.766,4.359)--(10.769,4.361)--(10.772,4.362)--(10.775,4.364)--(10.778,4.365)--(10.781,4.367)%
  --(10.784,4.368)--(10.787,4.370)--(10.790,4.371)--(10.793,4.373)--(10.796,4.374)--(10.799,4.376)%
  --(10.802,4.377)--(10.805,4.379)--(10.808,4.380)--(10.811,4.382)--(10.814,4.383)--(10.817,4.385)%
  --(10.820,4.386)--(10.823,4.388)--(10.826,4.389)--(10.828,4.390)--(10.831,4.392)--(10.834,4.393)%
  --(10.837,4.395)--(10.840,4.396)--(10.843,4.398)--(10.846,4.399)--(10.849,4.401)--(10.852,4.402)%
  --(10.855,4.404)--(10.858,4.405)--(10.861,4.407)--(10.864,4.408)--(10.867,4.410)--(10.870,4.411)%
  --(10.873,4.413)--(10.876,4.414)--(10.879,4.416)--(10.882,4.417)--(10.885,4.419)--(10.888,4.420)%
  --(10.891,4.422)--(10.894,4.423)--(10.897,4.425)--(10.900,4.426)--(10.903,4.428)--(10.906,4.429)%
  --(10.909,4.431)--(10.912,4.432)--(10.915,4.434)--(10.918,4.435)--(10.921,4.437)--(10.924,4.438)%
  --(10.927,4.440)--(10.930,4.441)--(10.933,4.443)--(10.936,4.444)--(10.939,4.446)--(10.942,4.447)%
  --(10.945,4.449)--(10.948,4.450)--(10.951,4.452)--(10.954,4.453)--(10.957,4.455)--(10.960,4.456)%
  --(10.963,4.458)--(10.966,4.459)--(10.969,4.461)--(10.972,4.462)--(10.975,4.464)--(10.978,4.465)%
  --(10.981,4.467)--(10.984,4.468)--(10.987,4.470)--(10.990,4.471)--(10.993,4.473)--(10.996,4.474)%
  --(10.999,4.476)--(11.002,4.477)--(11.005,4.479)--(11.008,4.480)--(11.011,4.482)--(11.014,4.483)%
  --(11.017,4.485)--(11.020,4.486)--(11.023,4.488)--(11.026,4.489)--(11.029,4.491)--(11.032,4.492)%
  --(11.035,4.494)--(11.037,4.495)--(11.040,4.497)--(11.043,4.498)--(11.046,4.500)--(11.049,4.501)%
  --(11.052,4.503)--(11.055,4.504)--(11.058,4.506)--(11.061,4.507)--(11.064,4.509)--(11.067,4.510)%
  --(11.070,4.512)--(11.073,4.513)--(11.076,4.515)--(11.079,4.516)--(11.082,4.518)--(11.085,4.519)%
  --(11.088,4.521)--(11.091,4.522)--(11.094,4.524)--(11.097,4.525)--(11.100,4.527)--(11.103,4.528)%
  --(11.106,4.530)--(11.109,4.531)--(11.112,4.533)--(11.115,4.534)--(11.118,4.536)--(11.121,4.537)%
  --(11.124,4.539)--(11.127,4.540)--(11.130,4.542)--(11.133,4.543)--(11.136,4.545)--(11.139,4.546)%
  --(11.142,4.548)--(11.145,4.549)--(11.148,4.551)--(11.151,4.552)--(11.154,4.554)--(11.157,4.555)%
  --(11.160,4.557)--(11.163,4.558)--(11.166,4.560)--(11.169,4.561)--(11.172,4.563)--(11.175,4.564)%
  --(11.178,4.566)--(11.181,4.567)--(11.184,4.569)--(11.187,4.570)--(11.190,4.572)--(11.193,4.573)%
  --(11.196,4.575)--(11.199,4.576)--(11.202,4.578)--(11.205,4.579)--(11.208,4.581)--(11.211,4.582)%
  --(11.214,4.584)--(11.217,4.585)--(11.220,4.587)--(11.223,4.588)--(11.226,4.590)--(11.229,4.591)%
  --(11.232,4.593)--(11.235,4.594)--(11.238,4.596)--(11.241,4.597)--(11.244,4.599)--(11.247,4.600)%
  --(11.249,4.602)--(11.252,4.603)--(11.255,4.605)--(11.258,4.606)--(11.261,4.608)--(11.264,4.609)%
  --(11.267,4.611)--(11.270,4.612)--(11.273,4.614)--(11.276,4.615)--(11.279,4.617)--(11.282,4.618)%
  --(11.285,4.620)--(11.288,4.621)--(11.291,4.623)--(11.294,4.624)--(11.297,4.626)--(11.300,4.627)%
  --(11.303,4.629)--(11.306,4.630)--(11.309,4.632)--(11.312,4.633)--(11.315,4.635)--(11.318,4.636)%
  --(11.321,4.638)--(11.324,4.639)--(11.327,4.641)--(11.330,4.642)--(11.333,4.644)--(11.336,4.645)%
  --(11.339,4.647)--(11.342,4.648)--(11.345,4.650)--(11.348,4.651)--(11.351,4.653)--(11.354,4.654)%
  --(11.357,4.656)--(11.360,4.657)--(11.363,4.659)--(11.366,4.660)--(11.369,4.662)--(11.372,4.663)%
  --(11.375,4.665)--(11.378,4.666)--(11.381,4.668)--(11.384,4.669)--(11.387,4.671)--(11.390,4.672)%
  --(11.393,4.674)--(11.396,4.675)--(11.399,4.677)--(11.402,4.678)--(11.405,4.680)--(11.408,4.681)%
  --(11.411,4.683)--(11.414,4.684)--(11.417,4.686)--(11.420,4.687)--(11.423,4.689)--(11.426,4.690)%
  --(11.429,4.692)--(11.432,4.693)--(11.435,4.695)--(11.438,4.696)--(11.441,4.698)--(11.444,4.699)%
  --(11.447,4.701)--(11.450,4.702)--(11.453,4.704)--(11.456,4.705)--(11.458,4.707)--(11.461,4.708)%
  --(11.464,4.710)--(11.467,4.711)--(11.470,4.713)--(11.473,4.714)--(11.476,4.716)--(11.479,4.717)%
  --(11.482,4.719)--(11.485,4.720)--(11.488,4.722)--(11.491,4.723)--(11.494,4.725)--(11.497,4.726)%
  --(11.500,4.728)--(11.503,4.729)--(11.506,4.731)--(11.509,4.732)--(11.512,4.734)--(11.515,4.735)%
  --(11.518,4.737)--(11.521,4.738)--(11.524,4.740)--(11.527,4.741)--(11.530,4.743)--(11.533,4.744)%
  --(11.536,4.746)--(11.539,4.747)--(11.542,4.749)--(11.545,4.750)--(11.548,4.752)--(11.551,4.753)%
  --(11.554,4.755)--(11.557,4.756)--(11.560,4.758)--(11.563,4.759)--(11.566,4.761)--(11.569,4.762)%
  --(11.572,4.764)--(11.575,4.765)--(11.578,4.767)--(11.581,4.768)--(11.584,4.770)--(11.587,4.771)%
  --(11.590,4.773)--(11.593,4.774)--(11.596,4.776)--(11.599,4.777)--(11.602,4.779)--(11.605,4.780)%
  --(11.608,4.782)--(11.611,4.783)--(11.614,4.785)--(11.617,4.786)--(11.620,4.788)--(11.623,4.789)%
  --(11.626,4.791)--(11.629,4.792)--(11.632,4.794)--(11.635,4.795)--(11.638,4.797)--(11.641,4.798)%
  --(11.644,4.800)--(11.647,4.801)--(11.650,4.803)--(11.653,4.804)--(11.656,4.806)--(11.659,4.807)%
  --(11.662,4.809)--(11.665,4.810)--(11.667,4.812)--(11.670,4.813)--(11.673,4.815)--(11.676,4.816)%
  --(11.679,4.818)--(11.682,4.819)--(11.685,4.821)--(11.688,4.822)--(11.691,4.824)--(11.694,4.825)%
  --(11.697,4.827)--(11.700,4.828)--(11.703,4.830)--(11.706,4.831)--(11.709,4.833)--(11.712,4.834)%
  --(11.715,4.836)--(11.718,4.837)--(11.721,4.839)--(11.724,4.840)--(11.727,4.842)--(11.730,4.843)%
  --(11.733,4.845)--(11.736,4.846)--(11.739,4.848)--(11.742,4.849)--(11.745,4.851)--(11.748,4.852)%
  --(11.751,4.854)--(11.754,4.855)--(11.757,4.857)--(11.760,4.858)--(11.763,4.860)--(11.766,4.861)%
  --(11.769,4.863)--(11.772,4.864)--(11.775,4.866)--(11.778,4.868)--(11.781,4.869)--(11.784,4.871)%
  --(11.787,4.872)--(11.790,4.874)--(11.793,4.875)--(11.796,4.877)--(11.799,4.878)--(11.802,4.880)%
  --(11.805,4.881)--(11.808,4.883)--(11.811,4.884)--(11.814,4.886)--(11.817,4.887)--(11.820,4.889)%
  --(11.823,4.890)--(11.826,4.892)--(11.829,4.893)--(11.832,4.895)--(11.835,4.896)--(11.838,4.898)%
  --(11.841,4.899)--(11.844,4.901)--(11.847,4.902)--(11.850,4.904)--(11.853,4.905)--(11.856,4.907)%
  --(11.859,4.908)--(11.862,4.910)--(11.865,4.911)--(11.868,4.913)--(11.871,4.914)--(11.874,4.916)%
  --(11.876,4.917)--(11.879,4.919)--(11.882,4.920)--(11.885,4.922)--(11.888,4.923)--(11.891,4.925)%
  --(11.894,4.926)--(11.897,4.928)--(11.900,4.929)--(11.903,4.931)--(11.906,4.932)--(11.909,4.934)%
  --(11.912,4.935)--(11.915,4.937)--(11.918,4.938)--(11.921,4.940)--(11.924,4.941)--(11.927,4.943)%
  --(11.930,4.944)--(11.933,4.946)--(11.936,4.947)--(11.939,4.949)--(11.942,4.950)--(11.945,4.952)%
  --(11.948,4.953)--(11.951,4.955)--(11.954,4.956)--(11.957,4.958)--(11.960,4.959)--(11.963,4.961)%
  --(11.966,4.962)--(11.969,4.964)--(11.972,4.965)--(11.975,4.967)--(11.978,4.968)--(11.981,4.970)%
  --(11.984,4.971)--(11.987,4.973)--(11.990,4.974)--(11.993,4.976)--(11.996,4.977)--(11.999,4.979)%
  --(12.002,4.980)--(12.005,4.982)--(12.008,4.983)--(12.011,4.985)--(12.014,4.986)--(12.017,4.988)%
  --(12.020,4.989)--(12.023,4.991)--(12.026,4.992)--(12.029,4.994)--(12.032,4.995)--(12.035,4.997)%
  --(12.038,4.998)--(12.041,5.000)--(12.044,5.001)--(12.047,5.003)--(12.050,5.004)--(12.053,5.006)%
  --(12.056,5.007)--(12.059,5.009)--(12.062,5.010)--(12.065,5.012)--(12.068,5.013)--(12.071,5.015)%
  --(12.074,5.016)--(12.077,5.018)--(12.080,5.020)--(12.083,5.021)--(12.085,5.023)--(12.088,5.024)%
  --(12.091,5.026)--(12.094,5.027)--(12.097,5.029)--(12.100,5.030)--(12.103,5.032)--(12.106,5.033)%
  --(12.109,5.035)--(12.112,5.036)--(12.115,5.038)--(12.118,5.039)--(12.121,5.041)--(12.124,5.042)%
  --(12.127,5.044)--(12.130,5.045)--(12.133,5.047)--(12.136,5.048)--(12.139,5.050)--(12.142,5.051)%
  --(12.145,5.053)--(12.148,5.054)--(12.151,5.056)--(12.154,5.057)--(12.157,5.059)--(12.160,5.060)%
  --(12.163,5.062)--(12.166,5.063)--(12.169,5.065)--(12.172,5.066)--(12.175,5.068)--(12.178,5.069)%
  --(12.181,5.071)--(12.184,5.072)--(12.187,5.074)--(12.190,5.075)--(12.193,5.077)--(12.196,5.078)%
  --(12.199,5.080)--(12.202,5.081)--(12.205,5.083)--(12.208,5.084)--(12.211,5.086)--(12.214,5.087)%
  --(12.217,5.089)--(12.220,5.090)--(12.223,5.092)--(12.226,5.093)--(12.229,5.095)--(12.232,5.096)%
  --(12.235,5.098)--(12.238,5.099)--(12.241,5.101)--(12.244,5.102)--(12.247,5.104)--(12.250,5.105)%
  --(12.253,5.107)--(12.256,5.108)--(12.259,5.110)--(12.262,5.111)--(12.265,5.113)--(12.268,5.114)%
  --(12.271,5.116)--(12.274,5.117)--(12.277,5.119)--(12.280,5.120)--(12.283,5.122)--(12.286,5.123)%
  --(12.289,5.125)--(12.292,5.126)--(12.295,5.128)--(12.297,5.129)--(12.300,5.131)--(12.303,5.132)%
  --(12.306,5.134)--(12.309,5.136)--(12.312,5.137)--(12.315,5.139)--(12.318,5.140)--(12.321,5.142)%
  --(12.324,5.143)--(12.327,5.145)--(12.330,5.146)--(12.333,5.148)--(12.336,5.149)--(12.339,5.151)%
  --(12.342,5.152)--(12.345,5.154)--(12.348,5.155)--(12.351,5.157)--(12.354,5.158)--(12.357,5.160)%
  --(12.360,5.161)--(12.363,5.163)--(12.366,5.164)--(12.369,5.166)--(12.372,5.167)--(12.375,5.169)%
  --(12.378,5.170)--(12.381,5.172)--(12.384,5.173)--(12.387,5.175)--(12.390,5.176)--(12.393,5.178)%
  --(12.396,5.179)--(12.399,5.181)--(12.402,5.182)--(12.405,5.184)--(12.408,5.185)--(12.411,5.187)%
  --(12.414,5.188)--(12.417,5.190)--(12.420,5.191)--(12.423,5.193)--(12.426,5.194)--(12.429,5.196)%
  --(12.432,5.197)--(12.435,5.199)--(12.438,5.200)--(12.441,5.202)--(12.444,5.203)--(12.447,5.205)%
  --(12.450,5.206)--(12.453,5.208)--(12.456,5.209)--(12.459,5.211)--(12.462,5.212)--(12.465,5.214)%
  --(12.468,5.215)--(12.471,5.217)--(12.474,5.218)--(12.477,5.220)--(12.480,5.221)--(12.483,5.223)%
  --(12.486,5.224)--(12.489,5.226)--(12.492,5.227)--(12.495,5.229)--(12.498,5.231)--(12.501,5.232)%
  --(12.504,5.234)--(12.506,5.235)--(12.509,5.237)--(12.512,5.238)--(12.515,5.240)--(12.518,5.241)%
  --(12.521,5.243)--(12.524,5.244)--(12.527,5.246)--(12.530,5.247)--(12.533,5.249)--(12.536,5.250)%
  --(12.539,5.252)--(12.542,5.253)--(12.545,5.255)--(12.548,5.256)--(12.551,5.258)--(12.554,5.259)%
  --(12.557,5.261)--(12.560,5.262)--(12.563,5.264)--(12.566,5.265)--(12.569,5.267)--(12.572,5.268)%
  --(12.575,5.270)--(12.578,5.271)--(12.581,5.273)--(12.584,5.274)--(12.587,5.276)--(12.590,5.277)%
  --(12.593,5.279)--(12.596,5.280)--(12.599,5.282)--(12.602,5.283)--(12.605,5.285)--(12.608,5.286)%
  --(12.611,5.288)--(12.614,5.289)--(12.617,5.291)--(12.620,5.292)--(12.623,5.294)--(12.626,5.295)%
  --(12.629,5.297)--(12.632,5.298)--(12.635,5.300)--(12.638,5.301)--(12.641,5.303)--(12.644,5.304)%
  --(12.647,5.306)--(12.650,5.307)--(12.653,5.309)--(12.656,5.310)--(12.659,5.312)--(12.662,5.313)%
  --(12.665,5.315)--(12.668,5.317)--(12.671,5.318)--(12.674,5.320)--(12.677,5.321)--(12.680,5.323)%
  --(12.683,5.324)--(12.686,5.326)--(12.689,5.327)--(12.692,5.329)--(12.695,5.330)--(12.698,5.332)%
  --(12.701,5.333)--(12.704,5.335)--(12.707,5.336)--(12.710,5.338)--(12.713,5.339)--(12.715,5.341)%
  --(12.718,5.342)--(12.721,5.344)--(12.724,5.345)--(12.727,5.347)--(12.730,5.348)--(12.733,5.350)%
  --(12.736,5.351)--(12.739,5.353)--(12.742,5.354)--(12.745,5.356)--(12.748,5.357)--(12.751,5.359)%
  --(12.754,5.360)--(12.757,5.362)--(12.760,5.363)--(12.763,5.365)--(12.766,5.366)--(12.769,5.368)%
  --(12.772,5.369)--(12.775,5.371)--(12.778,5.372)--(12.781,5.374)--(12.784,5.375)--(12.787,5.377)%
  --(12.790,5.378)--(12.793,5.380)--(12.796,5.381)--(12.799,5.383)--(12.802,5.384)--(12.805,5.386)%
  --(12.808,5.387)--(12.811,5.389)--(12.814,5.390)--(12.817,5.392)--(12.820,5.394)--(12.823,5.395)%
  --(12.826,5.397)--(12.829,5.398)--(12.832,5.400)--(12.835,5.401)--(12.838,5.403)--(12.841,5.404)%
  --(12.844,5.406)--(12.847,5.407)--(12.850,5.409)--(12.853,5.410)--(12.856,5.412)--(12.859,5.413)%
  --(12.862,5.415)--(12.865,5.416)--(12.868,5.418)--(12.871,5.419)--(12.874,5.421)--(12.877,5.422)%
  --(12.880,5.424)--(12.883,5.425)--(12.886,5.427)--(12.889,5.428)--(12.892,5.430)--(12.895,5.431)%
  --(12.898,5.433)--(12.901,5.434)--(12.904,5.436)--(12.907,5.437)--(12.910,5.439)--(12.913,5.440)%
  --(12.916,5.442)--(12.919,5.443)--(12.922,5.445)--(12.924,5.446)--(12.927,5.448)--(12.930,5.449)%
  --(12.933,5.451)--(12.936,5.452)--(12.939,5.454)--(12.942,5.455)--(12.945,5.457)--(12.948,5.458)%
  --(12.951,5.460)--(12.954,5.461)--(12.957,5.463)--(12.960,5.464)--(12.963,5.466)--(12.966,5.468)%
  --(12.969,5.469)--(12.972,5.471)--(12.975,5.472)--(12.978,5.474)--(12.981,5.475)--(12.984,5.477)%
  --(12.987,5.478)--(12.990,5.480)--(12.993,5.481)--(12.996,5.483)--(12.999,5.484)--(13.002,5.486)%
  --(13.005,5.487)--(13.008,5.489)--(13.011,5.490)--(13.014,5.492)--(13.017,5.493)--(13.020,5.495)%
  --(13.023,5.496)--(13.026,5.498)--(13.029,5.499)--(13.032,5.501)--(13.035,5.502)--(13.038,5.504)%
  --(13.041,5.505)--(13.044,5.507)--(13.047,5.508)--(13.050,5.510)--(13.053,5.511)--(13.056,5.513)%
  --(13.059,5.514)--(13.062,5.516)--(13.065,5.517)--(13.068,5.519)--(13.071,5.520)--(13.074,5.522)%
  --(13.077,5.523)--(13.080,5.525)--(13.083,5.526)--(13.086,5.528)--(13.089,5.529)--(13.092,5.531)%
  --(13.095,5.532)--(13.098,5.534)--(13.101,5.536)--(13.104,5.537)--(13.107,5.539)--(13.110,5.540)%
  --(13.113,5.542)--(13.116,5.543)--(13.119,5.545)--(13.122,5.546)--(13.125,5.548)--(13.128,5.549)%
  --(13.131,5.551)--(13.133,5.552)--(13.136,5.554)--(13.139,5.555)--(13.142,5.557)--(13.145,5.558)%
  --(13.148,5.560)--(13.151,5.561)--(13.154,5.563)--(13.157,5.564)--(13.160,5.566)--(13.163,5.567)%
  --(13.166,5.569)--(13.169,5.570)--(13.172,5.572)--(13.175,5.573)--(13.178,5.575)--(13.181,5.576)%
  --(13.184,5.578)--(13.187,5.579)--(13.190,5.581)--(13.193,5.582)--(13.196,5.584)--(13.199,5.585)%
  --(13.202,5.587)--(13.205,5.588)--(13.208,5.590)--(13.211,5.591)--(13.214,5.593)--(13.217,5.594)%
  --(13.220,5.596)--(13.223,5.597)--(13.226,5.599)--(13.229,5.601)--(13.232,5.602)--(13.235,5.604)%
  --(13.238,5.605)--(13.241,5.607)--(13.244,5.608)--(13.247,5.610)--(13.250,5.611)--(13.253,5.613)%
  --(13.256,5.614)--(13.259,5.616)--(13.262,5.617)--(13.265,5.619)--(13.268,5.620)--(13.271,5.622)%
  --(13.274,5.623)--(13.277,5.625)--(13.280,5.626)--(13.283,5.628)--(13.286,5.629)--(13.289,5.631)%
  --(13.292,5.632)--(13.295,5.634)--(13.298,5.635)--(13.301,5.637)--(13.304,5.638)--(13.307,5.640)%
  --(13.310,5.641)--(13.313,5.643)--(13.316,5.644)--(13.319,5.646)--(13.322,5.647)--(13.325,5.649)%
  --(13.328,5.650)--(13.331,5.652)--(13.334,5.653)--(13.337,5.655)--(13.340,5.656)--(13.342,5.658)%
  --(13.345,5.659)--(13.348,5.661)--(13.351,5.663)--(13.354,5.664)--(13.357,5.666)--(13.360,5.667)%
  --(13.363,5.669)--(13.366,5.670)--(13.369,5.672)--(13.372,5.673)--(13.375,5.675)--(13.378,5.676)%
  --(13.381,5.678)--(13.384,5.679)--(13.387,5.681)--(13.390,5.682)--(13.393,5.684)--(13.396,5.685)%
  --(13.399,5.687)--(13.402,5.688)--(13.405,5.690)--(13.408,5.691)--(13.411,5.693)--(13.414,5.694)%
  --(13.417,5.696)--(13.420,5.697)--(13.423,5.699)--(13.426,5.700)--(13.429,5.702)--(13.432,5.703)%
  --(13.435,5.705)--(13.438,5.706)--(13.441,5.708)--(13.444,5.709);
\gpcolor{color=gp lt color border}
\node[gp node left] at (2.972,7.989) {$\rho \approx \nicefrac{1}{3} \cdot \rho_{\rm{max}}$};
\gpcolor{rgb color={0.000,0.620,0.451}}
\draw[gp path] (1.872,7.989)--(2.788,7.989);
\draw[gp path] (1.507,2.512)--(1.510,2.510)--(1.513,2.508)--(1.516,2.506)--(1.519,2.504)%
  --(1.522,2.502)--(1.525,2.499)--(1.528,2.497)--(1.531,2.495)--(1.534,2.493)--(1.537,2.491)%
  --(1.540,2.489)--(1.543,2.487)--(1.546,2.485)--(1.549,2.482)--(1.552,2.480)--(1.555,2.478)%
  --(1.558,2.476)--(1.561,2.474)--(1.564,2.472)--(1.567,2.470)--(1.570,2.468)--(1.573,2.466)%
  --(1.576,2.463)--(1.579,2.461)--(1.582,2.459)--(1.585,2.457)--(1.588,2.455)--(1.591,2.453)%
  --(1.594,2.451)--(1.597,2.449)--(1.600,2.447)--(1.603,2.445)--(1.606,2.442)--(1.609,2.440)%
  --(1.611,2.438)--(1.614,2.436)--(1.617,2.434)--(1.620,2.432)--(1.623,2.430)--(1.626,2.428)%
  --(1.629,2.426)--(1.632,2.424)--(1.635,2.422)--(1.638,2.419)--(1.641,2.417)--(1.644,2.415)%
  --(1.647,2.413)--(1.650,2.411)--(1.653,2.409)--(1.656,2.407)--(1.659,2.405)--(1.662,2.403)%
  --(1.665,2.401)--(1.668,2.399)--(1.671,2.397)--(1.674,2.394)--(1.677,2.392)--(1.680,2.390)%
  --(1.683,2.388)--(1.686,2.386)--(1.689,2.384)--(1.692,2.382)--(1.695,2.380)--(1.698,2.378)%
  --(1.701,2.376)--(1.704,2.374)--(1.707,2.372)--(1.710,2.370)--(1.713,2.368)--(1.716,2.366)%
  --(1.719,2.364)--(1.722,2.362)--(1.725,2.360)--(1.728,2.357)--(1.731,2.355)--(1.734,2.353)%
  --(1.737,2.351)--(1.740,2.349)--(1.743,2.347)--(1.746,2.345)--(1.749,2.343)--(1.752,2.341)%
  --(1.755,2.339)--(1.758,2.337)--(1.761,2.335)--(1.764,2.333)--(1.767,2.331)--(1.770,2.329)%
  --(1.773,2.327)--(1.776,2.325)--(1.779,2.323)--(1.782,2.321)--(1.785,2.319)--(1.788,2.317)%
  --(1.791,2.315)--(1.794,2.313)--(1.797,2.311)--(1.800,2.309)--(1.803,2.307)--(1.806,2.305)%
  --(1.809,2.303)--(1.812,2.301)--(1.815,2.299)--(1.818,2.297)--(1.820,2.295)--(1.823,2.293)%
  --(1.826,2.291)--(1.829,2.290)--(1.832,2.288)--(1.835,2.286)--(1.838,2.284)--(1.841,2.282)%
  --(1.844,2.280)--(1.847,2.278)--(1.850,2.276)--(1.853,2.274)--(1.856,2.272)--(1.859,2.270)%
  --(1.862,2.268)--(1.865,2.266)--(1.868,2.264)--(1.871,2.262)--(1.874,2.261)--(1.877,2.259)%
  --(1.880,2.257)--(1.883,2.255)--(1.886,2.253)--(1.889,2.251)--(1.892,2.249)--(1.895,2.247)%
  --(1.898,2.245)--(1.901,2.243)--(1.904,2.242)--(1.907,2.240)--(1.910,2.238)--(1.913,2.236)%
  --(1.916,2.234)--(1.919,2.232)--(1.922,2.230)--(1.925,2.229)--(1.928,2.227)--(1.931,2.225)%
  --(1.934,2.223)--(1.937,2.221)--(1.940,2.219)--(1.943,2.217)--(1.946,2.216)--(1.949,2.214)%
  --(1.952,2.212)--(1.955,2.210)--(1.958,2.208)--(1.961,2.207)--(1.964,2.205)--(1.967,2.203)%
  --(1.970,2.201)--(1.973,2.199)--(1.976,2.198)--(1.979,2.196)--(1.982,2.194)--(1.985,2.192)%
  --(1.988,2.190)--(1.991,2.189)--(1.994,2.187)--(1.997,2.185)--(2.000,2.183)--(2.003,2.182)%
  --(2.006,2.180)--(2.009,2.178)--(2.012,2.176)--(2.015,2.175)--(2.018,2.173)--(2.021,2.171)%
  --(2.024,2.169)--(2.027,2.168)--(2.029,2.166)--(2.032,2.164)--(2.035,2.162)--(2.038,2.161)%
  --(2.041,2.159)--(2.044,2.157)--(2.047,2.156)--(2.050,2.154)--(2.053,2.152)--(2.056,2.151)%
  --(2.059,2.149)--(2.062,2.147)--(2.065,2.145)--(2.068,2.144)--(2.071,2.142)--(2.074,2.140)%
  --(2.077,2.139)--(2.080,2.137)--(2.083,2.136)--(2.086,2.134)--(2.089,2.132)--(2.092,2.131)%
  --(2.095,2.129)--(2.098,2.127)--(2.101,2.126)--(2.104,2.124)--(2.107,2.122)--(2.110,2.121)%
  --(2.113,2.119)--(2.116,2.118)--(2.119,2.116)--(2.122,2.114)--(2.125,2.113)--(2.128,2.111)%
  --(2.131,2.110)--(2.134,2.108)--(2.137,2.107)--(2.140,2.105)--(2.143,2.103)--(2.146,2.102)%
  --(2.149,2.100)--(2.152,2.099)--(2.155,2.097)--(2.158,2.096)--(2.161,2.094)--(2.164,2.093)%
  --(2.167,2.091)--(2.170,2.090)--(2.173,2.088)--(2.176,2.087)--(2.179,2.085)--(2.182,2.084)%
  --(2.185,2.082)--(2.188,2.081)--(2.191,2.079)--(2.194,2.078)--(2.197,2.076)--(2.200,2.075)%
  --(2.203,2.073)--(2.206,2.072)--(2.209,2.070)--(2.212,2.069)--(2.215,2.067)--(2.218,2.066)%
  --(2.221,2.064)--(2.224,2.063)--(2.227,2.062)--(2.230,2.060)--(2.233,2.059)--(2.236,2.057)%
  --(2.238,2.056)--(2.241,2.054)--(2.244,2.053)--(2.247,2.052)--(2.250,2.050)--(2.253,2.049)%
  --(2.256,2.047)--(2.259,2.046)--(2.262,2.045)--(2.265,2.043)--(2.268,2.042)--(2.271,2.041)%
  --(2.274,2.039)--(2.277,2.038)--(2.280,2.037)--(2.283,2.035)--(2.286,2.034)--(2.289,2.033)%
  --(2.292,2.031)--(2.295,2.030)--(2.298,2.029)--(2.301,2.027)--(2.304,2.026)--(2.307,2.025)%
  --(2.310,2.023)--(2.313,2.022)--(2.316,2.021)--(2.319,2.019)--(2.322,2.018)--(2.325,2.017)%
  --(2.328,2.016)--(2.331,2.014)--(2.334,2.013)--(2.337,2.012)--(2.340,2.011)--(2.343,2.009)%
  --(2.346,2.008)--(2.349,2.007)--(2.352,2.006)--(2.355,2.004)--(2.358,2.003)--(2.361,2.002)%
  --(2.364,2.001)--(2.367,2.000)--(2.370,1.998)--(2.373,1.997)--(2.376,1.996)--(2.379,1.995)%
  --(2.382,1.994)--(2.385,1.992)--(2.388,1.991)--(2.391,1.990)--(2.394,1.989)--(2.397,1.988)%
  --(2.400,1.987)--(2.403,1.985)--(2.406,1.984)--(2.409,1.983)--(2.412,1.982)--(2.415,1.981)%
  --(2.418,1.980)--(2.421,1.979)--(2.424,1.978)--(2.427,1.976)--(2.430,1.975)--(2.433,1.974)%
  --(2.436,1.973)--(2.439,1.972)--(2.442,1.971)--(2.445,1.970)--(2.447,1.969)--(2.450,1.968)%
  --(2.453,1.967)--(2.456,1.966)--(2.459,1.965)--(2.462,1.963)--(2.465,1.962)--(2.468,1.961)%
  --(2.471,1.960)--(2.474,1.959)--(2.477,1.958)--(2.480,1.957)--(2.483,1.956)--(2.486,1.955)%
  --(2.489,1.954)--(2.492,1.953)--(2.495,1.952)--(2.498,1.951)--(2.501,1.950)--(2.504,1.949)%
  --(2.507,1.948)--(2.510,1.947)--(2.513,1.946)--(2.516,1.945)--(2.519,1.944)--(2.522,1.943)%
  --(2.525,1.943)--(2.528,1.942)--(2.531,1.941)--(2.534,1.940)--(2.537,1.939)--(2.540,1.938)%
  --(2.543,1.937)--(2.546,1.936)--(2.549,1.935)--(2.552,1.934)--(2.555,1.933)--(2.558,1.932)%
  --(2.561,1.932)--(2.564,1.931)--(2.567,1.930)--(2.570,1.929)--(2.573,1.928)--(2.576,1.927)%
  --(2.579,1.926)--(2.582,1.925)--(2.585,1.925)--(2.588,1.924)--(2.591,1.923)--(2.594,1.922)%
  --(2.597,1.921)--(2.600,1.920)--(2.603,1.920)--(2.606,1.919)--(2.609,1.918)--(2.612,1.917)%
  --(2.615,1.916)--(2.618,1.915)--(2.621,1.915)--(2.624,1.914)--(2.627,1.913)--(2.630,1.912)%
  --(2.633,1.912)--(2.636,1.911)--(2.639,1.910)--(2.642,1.909)--(2.645,1.909)--(2.648,1.908)%
  --(2.651,1.907)--(2.654,1.906)--(2.656,1.906)--(2.659,1.905)--(2.662,1.904)--(2.665,1.903)%
  --(2.668,1.903)--(2.671,1.902)--(2.674,1.901)--(2.677,1.900)--(2.680,1.900)--(2.683,1.899)%
  --(2.686,1.898)--(2.689,1.898)--(2.692,1.897)--(2.695,1.896)--(2.698,1.896)--(2.701,1.895)%
  --(2.704,1.894)--(2.707,1.894)--(2.710,1.893)--(2.713,1.892)--(2.716,1.892)--(2.719,1.891)%
  --(2.722,1.890)--(2.725,1.890)--(2.728,1.889)--(2.731,1.889)--(2.734,1.888)--(2.737,1.887)%
  --(2.740,1.887)--(2.743,1.886)--(2.746,1.886)--(2.749,1.885)--(2.752,1.884)--(2.755,1.884)%
  --(2.758,1.883)--(2.761,1.883)--(2.764,1.882)--(2.767,1.881)--(2.770,1.881)--(2.773,1.880)%
  --(2.776,1.880)--(2.779,1.879)--(2.782,1.879)--(2.785,1.878)--(2.788,1.878)--(2.791,1.877)%
  --(2.794,1.877)--(2.797,1.876)--(2.800,1.876)--(2.803,1.875)--(2.806,1.875)--(2.809,1.874)%
  --(2.812,1.873)--(2.815,1.873)--(2.818,1.873)--(2.821,1.872)--(2.824,1.872)--(2.827,1.871)%
  --(2.830,1.871)--(2.833,1.870)--(2.836,1.870)--(2.839,1.869)--(2.842,1.869)--(2.845,1.868)%
  --(2.848,1.868)--(2.851,1.867)--(2.854,1.867)--(2.857,1.867)--(2.860,1.866)--(2.863,1.866)%
  --(2.866,1.865)--(2.868,1.865)--(2.871,1.864)--(2.874,1.864)--(2.877,1.864)--(2.880,1.863)%
  --(2.883,1.863)--(2.886,1.862)--(2.889,1.862)--(2.892,1.862)--(2.895,1.861)--(2.898,1.861)%
  --(2.901,1.860)--(2.904,1.860)--(2.907,1.860)--(2.910,1.859)--(2.913,1.859)--(2.916,1.859)%
  --(2.919,1.858)--(2.922,1.858)--(2.925,1.858)--(2.928,1.857)--(2.931,1.857)--(2.934,1.857)%
  --(2.937,1.856)--(2.940,1.856)--(2.943,1.856)--(2.946,1.855)--(2.949,1.855)--(2.952,1.855)%
  --(2.955,1.855)--(2.958,1.854)--(2.961,1.854)--(2.964,1.854)--(2.967,1.853)--(2.970,1.853)%
  --(2.973,1.853)--(2.976,1.853)--(2.979,1.852)--(2.982,1.852)--(2.985,1.852)--(2.988,1.852)%
  --(2.991,1.851)--(2.994,1.851)--(2.997,1.851)--(3.000,1.851)--(3.003,1.850)--(3.006,1.850)%
  --(3.009,1.850)--(3.012,1.850)--(3.015,1.849)--(3.018,1.849)--(3.021,1.849)--(3.024,1.849)%
  --(3.027,1.849)--(3.030,1.848)--(3.033,1.848)--(3.036,1.848)--(3.039,1.848)--(3.042,1.848)%
  --(3.045,1.847)--(3.048,1.847)--(3.051,1.847)--(3.054,1.847)--(3.057,1.847)--(3.060,1.847)%
  --(3.063,1.846)--(3.066,1.846)--(3.069,1.846)--(3.072,1.846)--(3.075,1.846)--(3.077,1.846)%
  --(3.080,1.846)--(3.083,1.845)--(3.086,1.845)--(3.089,1.845)--(3.092,1.845)--(3.095,1.845)%
  --(3.098,1.845)--(3.101,1.845)--(3.104,1.845)--(3.107,1.845)--(3.110,1.845)--(3.113,1.844)%
  --(3.116,1.844)--(3.119,1.844)--(3.122,1.844)--(3.125,1.844)--(3.128,1.844)--(3.131,1.844)%
  --(3.134,1.844)--(3.137,1.844)--(3.140,1.844)--(3.143,1.844)--(3.146,1.844)--(3.149,1.844)%
  --(3.152,1.844)--(3.155,1.844)--(3.158,1.843)--(3.161,1.843)--(3.164,1.843)--(3.167,1.843)%
  --(3.170,1.843)--(3.173,1.843)--(3.176,1.843)--(3.179,1.843)--(3.182,1.843)--(3.185,1.843)%
  --(3.188,1.843)--(3.191,1.843)--(3.194,1.843)--(3.197,1.843)--(3.200,1.843)--(3.203,1.843)%
  --(3.206,1.843)--(3.209,1.843)--(3.212,1.843)--(3.215,1.843)--(3.218,1.843)--(3.221,1.844)%
  --(3.224,1.844)--(3.227,1.844)--(3.230,1.844)--(3.233,1.844)--(3.236,1.844)--(3.239,1.844)%
  --(3.242,1.844)--(3.245,1.844)--(3.248,1.844)--(3.251,1.844)--(3.254,1.844)--(3.257,1.844)%
  --(3.260,1.844)--(3.263,1.844)--(3.266,1.844)--(3.269,1.845)--(3.272,1.845)--(3.275,1.845)%
  --(3.278,1.845)--(3.281,1.845)--(3.284,1.845)--(3.286,1.845)--(3.289,1.845)--(3.292,1.845)%
  --(3.295,1.846)--(3.298,1.846)--(3.301,1.846)--(3.304,1.846)--(3.307,1.846)--(3.310,1.846)%
  --(3.313,1.846)--(3.316,1.846)--(3.319,1.847)--(3.322,1.847)--(3.325,1.847)--(3.328,1.847)%
  --(3.331,1.847)--(3.334,1.847)--(3.337,1.848)--(3.340,1.848)--(3.343,1.848)--(3.346,1.848)%
  --(3.349,1.848)--(3.352,1.848)--(3.355,1.849)--(3.358,1.849)--(3.361,1.849)--(3.364,1.849)%
  --(3.367,1.849)--(3.370,1.850)--(3.373,1.850)--(3.376,1.850)--(3.379,1.850)--(3.382,1.850)%
  --(3.385,1.851)--(3.388,1.851)--(3.391,1.851)--(3.394,1.851)--(3.397,1.851)--(3.400,1.852)%
  --(3.403,1.852)--(3.406,1.852)--(3.409,1.852)--(3.412,1.853)--(3.415,1.853)--(3.418,1.853)%
  --(3.421,1.853)--(3.424,1.854)--(3.427,1.854)--(3.430,1.854)--(3.433,1.854)--(3.436,1.855)%
  --(3.439,1.855)--(3.442,1.855)--(3.445,1.856)--(3.448,1.856)--(3.451,1.856)--(3.454,1.856)%
  --(3.457,1.857)--(3.460,1.857)--(3.463,1.857)--(3.466,1.858)--(3.469,1.858)--(3.472,1.858)%
  --(3.475,1.858)--(3.478,1.859)--(3.481,1.859)--(3.484,1.859)--(3.487,1.860)--(3.490,1.860)%
  --(3.493,1.860)--(3.495,1.861)--(3.498,1.861)--(3.501,1.861)--(3.504,1.862)--(3.507,1.862)%
  --(3.510,1.862)--(3.513,1.863)--(3.516,1.863)--(3.519,1.863)--(3.522,1.864)--(3.525,1.864)%
  --(3.528,1.864)--(3.531,1.865)--(3.534,1.865)--(3.537,1.865)--(3.540,1.866)--(3.543,1.866)%
  --(3.546,1.866)--(3.549,1.867)--(3.552,1.867)--(3.555,1.868)--(3.558,1.868)--(3.561,1.868)%
  --(3.564,1.869)--(3.567,1.869)--(3.570,1.870)--(3.573,1.870)--(3.576,1.870)--(3.579,1.871)%
  --(3.582,1.871)--(3.585,1.871)--(3.588,1.872)--(3.591,1.872)--(3.594,1.873)--(3.597,1.873)%
  --(3.600,1.874)--(3.603,1.874)--(3.606,1.874)--(3.609,1.875)--(3.612,1.875)--(3.615,1.876)%
  --(3.618,1.876)--(3.621,1.876)--(3.624,1.877)--(3.627,1.877)--(3.630,1.878)--(3.633,1.878)%
  --(3.636,1.879)--(3.639,1.879)--(3.642,1.880)--(3.645,1.880)--(3.648,1.880)--(3.651,1.881)%
  --(3.654,1.881)--(3.657,1.882)--(3.660,1.882)--(3.663,1.883)--(3.666,1.883)--(3.669,1.884)%
  --(3.672,1.884)--(3.675,1.885)--(3.678,1.885)--(3.681,1.886)--(3.684,1.886)--(3.687,1.887)%
  --(3.690,1.887)--(3.693,1.888)--(3.696,1.888)--(3.699,1.889)--(3.702,1.889)--(3.704,1.890)%
  --(3.707,1.890)--(3.710,1.891)--(3.713,1.891)--(3.716,1.892)--(3.719,1.892)--(3.722,1.893)%
  --(3.725,1.893)--(3.728,1.894)--(3.731,1.894)--(3.734,1.895)--(3.737,1.895)--(3.740,1.896)%
  --(3.743,1.896)--(3.746,1.897)--(3.749,1.897)--(3.752,1.898)--(3.755,1.898)--(3.758,1.899)%
  --(3.761,1.900)--(3.764,1.900)--(3.767,1.901)--(3.770,1.901)--(3.773,1.902)--(3.776,1.902)%
  --(3.779,1.903)--(3.782,1.903)--(3.785,1.904)--(3.788,1.905)--(3.791,1.905)--(3.794,1.906)%
  --(3.797,1.906)--(3.800,1.907)--(3.803,1.907)--(3.806,1.908)--(3.809,1.909)--(3.812,1.909)%
  --(3.815,1.910)--(3.818,1.910)--(3.821,1.911)--(3.824,1.911)--(3.827,1.912)--(3.830,1.913)%
  --(3.833,1.913)--(3.836,1.914)--(3.839,1.914)--(3.842,1.915)--(3.845,1.916)--(3.848,1.916)%
  --(3.851,1.917)--(3.854,1.917)--(3.857,1.918)--(3.860,1.919)--(3.863,1.919)--(3.866,1.920)%
  --(3.869,1.921)--(3.872,1.921)--(3.875,1.922)--(3.878,1.922)--(3.881,1.923)--(3.884,1.924)%
  --(3.887,1.924)--(3.890,1.925)--(3.893,1.926)--(3.896,1.926)--(3.899,1.927)--(3.902,1.927)%
  --(3.905,1.928)--(3.908,1.929)--(3.911,1.929)--(3.914,1.930)--(3.916,1.931)--(3.919,1.931)%
  --(3.922,1.932)--(3.925,1.933)--(3.928,1.933)--(3.931,1.934)--(3.934,1.935)--(3.937,1.935)%
  --(3.940,1.936)--(3.943,1.937)--(3.946,1.937)--(3.949,1.938)--(3.952,1.939)--(3.955,1.939)%
  --(3.958,1.940)--(3.961,1.941)--(3.964,1.941)--(3.967,1.942)--(3.970,1.943)--(3.973,1.943)%
  --(3.976,1.944)--(3.979,1.945)--(3.982,1.946)--(3.985,1.946)--(3.988,1.947)--(3.991,1.948)%
  --(3.994,1.948)--(3.997,1.949)--(4.000,1.950)--(4.003,1.950)--(4.006,1.951)--(4.009,1.952)%
  --(4.012,1.953)--(4.015,1.953)--(4.018,1.954)--(4.021,1.955)--(4.024,1.955)--(4.027,1.956)%
  --(4.030,1.957)--(4.033,1.958)--(4.036,1.958)--(4.039,1.959)--(4.042,1.960)--(4.045,1.961)%
  --(4.048,1.961)--(4.051,1.962)--(4.054,1.963)--(4.057,1.963)--(4.060,1.964)--(4.063,1.965)%
  --(4.066,1.966)--(4.069,1.966)--(4.072,1.967)--(4.075,1.968)--(4.078,1.969)--(4.081,1.969)%
  --(4.084,1.970)--(4.087,1.971)--(4.090,1.972)--(4.093,1.972)--(4.096,1.973)--(4.099,1.974)%
  --(4.102,1.975)--(4.105,1.976)--(4.108,1.976)--(4.111,1.977)--(4.114,1.978)--(4.117,1.979)%
  --(4.120,1.979)--(4.123,1.980)--(4.125,1.981)--(4.128,1.982)--(4.131,1.982)--(4.134,1.983)%
  --(4.137,1.984)--(4.140,1.985)--(4.143,1.986)--(4.146,1.986)--(4.149,1.987)--(4.152,1.988)%
  --(4.155,1.989)--(4.158,1.990)--(4.161,1.990)--(4.164,1.991)--(4.167,1.992)--(4.170,1.993)%
  --(4.173,1.994)--(4.176,1.994)--(4.179,1.995)--(4.182,1.996)--(4.185,1.997)--(4.188,1.998)%
  --(4.191,1.998)--(4.194,1.999)--(4.197,2.000)--(4.200,2.001)--(4.203,2.002)--(4.206,2.003)%
  --(4.209,2.003)--(4.212,2.004)--(4.215,2.005)--(4.218,2.006)--(4.221,2.007)--(4.224,2.007)%
  --(4.227,2.008)--(4.230,2.009)--(4.233,2.010)--(4.236,2.011)--(4.239,2.012)--(4.242,2.012)%
  --(4.245,2.013)--(4.248,2.014)--(4.251,2.015)--(4.254,2.016)--(4.257,2.017)--(4.260,2.018)%
  --(4.263,2.018)--(4.266,2.019)--(4.269,2.020)--(4.272,2.021)--(4.275,2.022)--(4.278,2.023)%
  --(4.281,2.024)--(4.284,2.024)--(4.287,2.025)--(4.290,2.026)--(4.293,2.027)--(4.296,2.028)%
  --(4.299,2.029)--(4.302,2.030)--(4.305,2.030)--(4.308,2.031)--(4.311,2.032)--(4.314,2.033)%
  --(4.317,2.034)--(4.320,2.035)--(4.323,2.036)--(4.326,2.037)--(4.329,2.037)--(4.332,2.038)%
  --(4.334,2.039)--(4.337,2.040)--(4.340,2.041)--(4.343,2.042)--(4.346,2.043)--(4.349,2.044)%
  --(4.352,2.045)--(4.355,2.045)--(4.358,2.046)--(4.361,2.047)--(4.364,2.048)--(4.367,2.049)%
  --(4.370,2.050)--(4.373,2.051)--(4.376,2.052)--(4.379,2.053)--(4.382,2.054)--(4.385,2.054)%
  --(4.388,2.055)--(4.391,2.056)--(4.394,2.057)--(4.397,2.058)--(4.400,2.059)--(4.403,2.060)%
  --(4.406,2.061)--(4.409,2.062)--(4.412,2.063)--(4.415,2.064)--(4.418,2.064)--(4.421,2.065)%
  --(4.424,2.066)--(4.427,2.067)--(4.430,2.068)--(4.433,2.069)--(4.436,2.070)--(4.439,2.071)%
  --(4.442,2.072)--(4.445,2.073)--(4.448,2.074)--(4.451,2.075)--(4.454,2.076)--(4.457,2.077)%
  --(4.460,2.077)--(4.463,2.078)--(4.466,2.079)--(4.469,2.080)--(4.472,2.081)--(4.475,2.082)%
  --(4.478,2.083)--(4.481,2.084)--(4.484,2.085)--(4.487,2.086)--(4.490,2.087)--(4.493,2.088)%
  --(4.496,2.089)--(4.499,2.090)--(4.502,2.091)--(4.505,2.092)--(4.508,2.093)--(4.511,2.094)%
  --(4.514,2.095)--(4.517,2.096)--(4.520,2.096)--(4.523,2.097)--(4.526,2.098)--(4.529,2.099)%
  --(4.532,2.100)--(4.535,2.101)--(4.538,2.102)--(4.541,2.103)--(4.543,2.104)--(4.546,2.105)%
  --(4.549,2.106)--(4.552,2.107)--(4.555,2.108)--(4.558,2.109)--(4.561,2.110)--(4.564,2.111)%
  --(4.567,2.112)--(4.570,2.113)--(4.573,2.114)--(4.576,2.115)--(4.579,2.116)--(4.582,2.117)%
  --(4.585,2.118)--(4.588,2.119)--(4.591,2.120)--(4.594,2.121)--(4.597,2.122)--(4.600,2.123)%
  --(4.603,2.124)--(4.606,2.125)--(4.609,2.126)--(4.612,2.127)--(4.615,2.128)--(4.618,2.129)%
  --(4.621,2.130)--(4.624,2.131)--(4.627,2.132)--(4.630,2.133)--(4.633,2.134)--(4.636,2.135)%
  --(4.639,2.136)--(4.642,2.137)--(4.645,2.138)--(4.648,2.139)--(4.651,2.140)--(4.654,2.141)%
  --(4.657,2.142)--(4.660,2.143)--(4.663,2.144)--(4.666,2.145)--(4.669,2.146)--(4.672,2.147)%
  --(4.675,2.148)--(4.678,2.149)--(4.681,2.150)--(4.684,2.151)--(4.687,2.152)--(4.690,2.153)%
  --(4.693,2.154)--(4.696,2.155)--(4.699,2.156)--(4.702,2.157)--(4.705,2.158)--(4.708,2.159)%
  --(4.711,2.160)--(4.714,2.161)--(4.717,2.162)--(4.720,2.163)--(4.723,2.164)--(4.726,2.166)%
  --(4.729,2.167)--(4.732,2.168)--(4.735,2.169)--(4.738,2.170)--(4.741,2.171)--(4.744,2.172)%
  --(4.747,2.173)--(4.750,2.174)--(4.752,2.175)--(4.755,2.176)--(4.758,2.177)--(4.761,2.178)%
  --(4.764,2.179)--(4.767,2.180)--(4.770,2.181)--(4.773,2.182)--(4.776,2.183)--(4.779,2.184)%
  --(4.782,2.185)--(4.785,2.186)--(4.788,2.188)--(4.791,2.189)--(4.794,2.190)--(4.797,2.191)%
  --(4.800,2.192)--(4.803,2.193)--(4.806,2.194)--(4.809,2.195)--(4.812,2.196)--(4.815,2.197)%
  --(4.818,2.198)--(4.821,2.199)--(4.824,2.200)--(4.827,2.201)--(4.830,2.202)--(4.833,2.203)%
  --(4.836,2.205)--(4.839,2.206)--(4.842,2.207)--(4.845,2.208)--(4.848,2.209)--(4.851,2.210)%
  --(4.854,2.211)--(4.857,2.212)--(4.860,2.213)--(4.863,2.214)--(4.866,2.215)--(4.869,2.216)%
  --(4.872,2.217)--(4.875,2.219)--(4.878,2.220)--(4.881,2.221)--(4.884,2.222)--(4.887,2.223)%
  --(4.890,2.224)--(4.893,2.225)--(4.896,2.226)--(4.899,2.227)--(4.902,2.228)--(4.905,2.229)%
  --(4.908,2.231)--(4.911,2.232)--(4.914,2.233)--(4.917,2.234)--(4.920,2.235)--(4.923,2.236)%
  --(4.926,2.237)--(4.929,2.238)--(4.932,2.239)--(4.935,2.240)--(4.938,2.242)--(4.941,2.243)%
  --(4.944,2.244)--(4.947,2.245)--(4.950,2.246)--(4.953,2.247)--(4.956,2.248)--(4.959,2.249)%
  --(4.961,2.250)--(4.964,2.251)--(4.967,2.253)--(4.970,2.254)--(4.973,2.255)--(4.976,2.256)%
  --(4.979,2.257)--(4.982,2.258)--(4.985,2.259)--(4.988,2.260)--(4.991,2.261)--(4.994,2.263)%
  --(4.997,2.264)--(5.000,2.265)--(5.003,2.266)--(5.006,2.267)--(5.009,2.268)--(5.012,2.269)%
  --(5.015,2.270)--(5.018,2.272)--(5.021,2.273)--(5.024,2.274)--(5.027,2.275)--(5.030,2.276)%
  --(5.033,2.277)--(5.036,2.278)--(5.039,2.279)--(5.042,2.281)--(5.045,2.282)--(5.048,2.283)%
  --(5.051,2.284)--(5.054,2.285)--(5.057,2.286)--(5.060,2.287)--(5.063,2.289)--(5.066,2.290)%
  --(5.069,2.291)--(5.072,2.292)--(5.075,2.293)--(5.078,2.294)--(5.081,2.295)--(5.084,2.296)%
  --(5.087,2.298)--(5.090,2.299)--(5.093,2.300)--(5.096,2.301)--(5.099,2.302)--(5.102,2.303)%
  --(5.105,2.304)--(5.108,2.306)--(5.111,2.307)--(5.114,2.308)--(5.117,2.309)--(5.120,2.310)%
  --(5.123,2.311)--(5.126,2.313)--(5.129,2.314)--(5.132,2.315)--(5.135,2.316)--(5.138,2.317)%
  --(5.141,2.318)--(5.144,2.319)--(5.147,2.321)--(5.150,2.322)--(5.153,2.323)--(5.156,2.324)%
  --(5.159,2.325)--(5.162,2.326)--(5.165,2.328)--(5.168,2.329)--(5.171,2.330)--(5.173,2.331)%
  --(5.176,2.332)--(5.179,2.333)--(5.182,2.335)--(5.185,2.336)--(5.188,2.337)--(5.191,2.338)%
  --(5.194,2.339)--(5.197,2.340)--(5.200,2.342)--(5.203,2.343)--(5.206,2.344)--(5.209,2.345)%
  --(5.212,2.346)--(5.215,2.347)--(5.218,2.349)--(5.221,2.350)--(5.224,2.351)--(5.227,2.352)%
  --(5.230,2.353)--(5.233,2.354)--(5.236,2.356)--(5.239,2.357)--(5.242,2.358)--(5.245,2.359)%
  --(5.248,2.360)--(5.251,2.361)--(5.254,2.363)--(5.257,2.364)--(5.260,2.365)--(5.263,2.366)%
  --(5.266,2.367)--(5.269,2.369)--(5.272,2.370)--(5.275,2.371)--(5.278,2.372)--(5.281,2.373)%
  --(5.284,2.374)--(5.287,2.376)--(5.290,2.377)--(5.293,2.378)--(5.296,2.379)--(5.299,2.380)%
  --(5.302,2.382)--(5.305,2.383)--(5.308,2.384)--(5.311,2.385)--(5.314,2.386)--(5.317,2.388)%
  --(5.320,2.389)--(5.323,2.390)--(5.326,2.391)--(5.329,2.392)--(5.332,2.394)--(5.335,2.395)%
  --(5.338,2.396)--(5.341,2.397)--(5.344,2.398)--(5.347,2.400)--(5.350,2.401)--(5.353,2.402)%
  --(5.356,2.403)--(5.359,2.404)--(5.362,2.406)--(5.365,2.407)--(5.368,2.408)--(5.371,2.409)%
  --(5.374,2.410)--(5.377,2.412)--(5.380,2.413)--(5.382,2.414)--(5.385,2.415)--(5.388,2.416)%
  --(5.391,2.418)--(5.394,2.419)--(5.397,2.420)--(5.400,2.421)--(5.403,2.422)--(5.406,2.424)%
  --(5.409,2.425)--(5.412,2.426)--(5.415,2.427)--(5.418,2.428)--(5.421,2.430)--(5.424,2.431)%
  --(5.427,2.432)--(5.430,2.433)--(5.433,2.435)--(5.436,2.436)--(5.439,2.437)--(5.442,2.438)%
  --(5.445,2.439)--(5.448,2.441)--(5.451,2.442)--(5.454,2.443)--(5.457,2.444)--(5.460,2.446)%
  --(5.463,2.447)--(5.466,2.448)--(5.469,2.449)--(5.472,2.450)--(5.475,2.452)--(5.478,2.453)%
  --(5.481,2.454)--(5.484,2.455)--(5.487,2.457)--(5.490,2.458)--(5.493,2.459)--(5.496,2.460)%
  --(5.499,2.461)--(5.502,2.463)--(5.505,2.464)--(5.508,2.465)--(5.511,2.466)--(5.514,2.468)%
  --(5.517,2.469)--(5.520,2.470)--(5.523,2.471)--(5.526,2.473)--(5.529,2.474)--(5.532,2.475)%
  --(5.535,2.476)--(5.538,2.477)--(5.541,2.479)--(5.544,2.480)--(5.547,2.481)--(5.550,2.482)%
  --(5.553,2.484)--(5.556,2.485)--(5.559,2.486)--(5.562,2.487)--(5.565,2.489)--(5.568,2.490)%
  --(5.571,2.491)--(5.574,2.492)--(5.577,2.494)--(5.580,2.495)--(5.583,2.496)--(5.586,2.497)%
  --(5.589,2.499)--(5.591,2.500)--(5.594,2.501)--(5.597,2.502)--(5.600,2.504)--(5.603,2.505)%
  --(5.606,2.506)--(5.609,2.507)--(5.612,2.509)--(5.615,2.510)--(5.618,2.511)--(5.621,2.512)%
  --(5.624,2.514)--(5.627,2.515)--(5.630,2.516)--(5.633,2.517)--(5.636,2.519)--(5.639,2.520)%
  --(5.642,2.521)--(5.645,2.522)--(5.648,2.524)--(5.651,2.525)--(5.654,2.526)--(5.657,2.527)%
  --(5.660,2.529)--(5.663,2.530)--(5.666,2.531)--(5.669,2.532)--(5.672,2.534)--(5.675,2.535)%
  --(5.678,2.536)--(5.681,2.537)--(5.684,2.539)--(5.687,2.540)--(5.690,2.541)--(5.693,2.542)%
  --(5.696,2.544)--(5.699,2.545)--(5.702,2.546)--(5.705,2.547)--(5.708,2.549)--(5.711,2.550)%
  --(5.714,2.551)--(5.717,2.553)--(5.720,2.554)--(5.723,2.555)--(5.726,2.556)--(5.729,2.558)%
  --(5.732,2.559)--(5.735,2.560)--(5.738,2.561)--(5.741,2.563)--(5.744,2.564)--(5.747,2.565)%
  --(5.750,2.566)--(5.753,2.568)--(5.756,2.569)--(5.759,2.570)--(5.762,2.572)--(5.765,2.573)%
  --(5.768,2.574)--(5.771,2.575)--(5.774,2.577)--(5.777,2.578)--(5.780,2.579)--(5.783,2.580)%
  --(5.786,2.582)--(5.789,2.583)--(5.792,2.584)--(5.795,2.586)--(5.798,2.587)--(5.800,2.588)%
  --(5.803,2.589)--(5.806,2.591)--(5.809,2.592)--(5.812,2.593)--(5.815,2.594)--(5.818,2.596)%
  --(5.821,2.597)--(5.824,2.598)--(5.827,2.600)--(5.830,2.601)--(5.833,2.602)--(5.836,2.603)%
  --(5.839,2.605)--(5.842,2.606)--(5.845,2.607)--(5.848,2.609)--(5.851,2.610)--(5.854,2.611)%
  --(5.857,2.612)--(5.860,2.614)--(5.863,2.615)--(5.866,2.616)--(5.869,2.618)--(5.872,2.619)%
  --(5.875,2.620)--(5.878,2.621)--(5.881,2.623)--(5.884,2.624)--(5.887,2.625)--(5.890,2.627)%
  --(5.893,2.628)--(5.896,2.629)--(5.899,2.630)--(5.902,2.632)--(5.905,2.633)--(5.908,2.634)%
  --(5.911,2.636)--(5.914,2.637)--(5.917,2.638)--(5.920,2.640)--(5.923,2.641)--(5.926,2.642)%
  --(5.929,2.643)--(5.932,2.645)--(5.935,2.646)--(5.938,2.647)--(5.941,2.649)--(5.944,2.650)%
  --(5.947,2.651)--(5.950,2.652)--(5.953,2.654)--(5.956,2.655)--(5.959,2.656)--(5.962,2.658)%
  --(5.965,2.659)--(5.968,2.660)--(5.971,2.662)--(5.974,2.663)--(5.977,2.664)--(5.980,2.665)%
  --(5.983,2.667)--(5.986,2.668)--(5.989,2.669)--(5.992,2.671)--(5.995,2.672)--(5.998,2.673)%
  --(6.001,2.675)--(6.004,2.676)--(6.007,2.677)--(6.009,2.679)--(6.012,2.680)--(6.015,2.681)%
  --(6.018,2.682)--(6.021,2.684)--(6.024,2.685)--(6.027,2.686)--(6.030,2.688)--(6.033,2.689)%
  --(6.036,2.690)--(6.039,2.692)--(6.042,2.693)--(6.045,2.694)--(6.048,2.696)--(6.051,2.697)%
  --(6.054,2.698)--(6.057,2.699)--(6.060,2.701)--(6.063,2.702)--(6.066,2.703)--(6.069,2.705)%
  --(6.072,2.706)--(6.075,2.707)--(6.078,2.709)--(6.081,2.710)--(6.084,2.711)--(6.087,2.713)%
  --(6.090,2.714)--(6.093,2.715)--(6.096,2.717)--(6.099,2.718)--(6.102,2.719)--(6.105,2.720)%
  --(6.108,2.722)--(6.111,2.723)--(6.114,2.724)--(6.117,2.726)--(6.120,2.727)--(6.123,2.728)%
  --(6.126,2.730)--(6.129,2.731)--(6.132,2.732)--(6.135,2.734)--(6.138,2.735)--(6.141,2.736)%
  --(6.144,2.738)--(6.147,2.739)--(6.150,2.740)--(6.153,2.742)--(6.156,2.743)--(6.159,2.744)%
  --(6.162,2.746)--(6.165,2.747)--(6.168,2.748)--(6.171,2.750)--(6.174,2.751)--(6.177,2.752)%
  --(6.180,2.754)--(6.183,2.755)--(6.186,2.756)--(6.189,2.758)--(6.192,2.759)--(6.195,2.760)%
  --(6.198,2.761)--(6.201,2.763)--(6.204,2.764)--(6.207,2.765)--(6.210,2.767)--(6.213,2.768)%
  --(6.216,2.769)--(6.218,2.771)--(6.221,2.772)--(6.224,2.773)--(6.227,2.775)--(6.230,2.776)%
  --(6.233,2.777)--(6.236,2.779)--(6.239,2.780)--(6.242,2.781)--(6.245,2.783)--(6.248,2.784)%
  --(6.251,2.785)--(6.254,2.787)--(6.257,2.788)--(6.260,2.789)--(6.263,2.791)--(6.266,2.792)%
  --(6.269,2.793)--(6.272,2.795)--(6.275,2.796)--(6.278,2.797)--(6.281,2.799)--(6.284,2.800)%
  --(6.287,2.801)--(6.290,2.803)--(6.293,2.804)--(6.296,2.805)--(6.299,2.807)--(6.302,2.808)%
  --(6.305,2.809)--(6.308,2.811)--(6.311,2.812)--(6.314,2.813)--(6.317,2.815)--(6.320,2.816)%
  --(6.323,2.818)--(6.326,2.819)--(6.329,2.820)--(6.332,2.822)--(6.335,2.823)--(6.338,2.824)%
  --(6.341,2.826)--(6.344,2.827)--(6.347,2.828)--(6.350,2.830)--(6.353,2.831)--(6.356,2.832)%
  --(6.359,2.834)--(6.362,2.835)--(6.365,2.836)--(6.368,2.838)--(6.371,2.839)--(6.374,2.840)%
  --(6.377,2.842)--(6.380,2.843)--(6.383,2.844)--(6.386,2.846)--(6.389,2.847)--(6.392,2.848)%
  --(6.395,2.850)--(6.398,2.851)--(6.401,2.852)--(6.404,2.854)--(6.407,2.855)--(6.410,2.857)%
  --(6.413,2.858)--(6.416,2.859)--(6.419,2.861)--(6.422,2.862)--(6.425,2.863)--(6.428,2.865)%
  --(6.430,2.866)--(6.433,2.867)--(6.436,2.869)--(6.439,2.870)--(6.442,2.871)--(6.445,2.873)%
  --(6.448,2.874)--(6.451,2.875)--(6.454,2.877)--(6.457,2.878)--(6.460,2.879)--(6.463,2.881)%
  --(6.466,2.882)--(6.469,2.884)--(6.472,2.885)--(6.475,2.886)--(6.478,2.888)--(6.481,2.889)%
  --(6.484,2.890)--(6.487,2.892)--(6.490,2.893)--(6.493,2.894)--(6.496,2.896)--(6.499,2.897)%
  --(6.502,2.898)--(6.505,2.900)--(6.508,2.901)--(6.511,2.903)--(6.514,2.904)--(6.517,2.905)%
  --(6.520,2.907)--(6.523,2.908)--(6.526,2.909)--(6.529,2.911)--(6.532,2.912)--(6.535,2.913)%
  --(6.538,2.915)--(6.541,2.916)--(6.544,2.917)--(6.547,2.919)--(6.550,2.920)--(6.553,2.922)%
  --(6.556,2.923)--(6.559,2.924)--(6.562,2.926)--(6.565,2.927)--(6.568,2.928)--(6.571,2.930)%
  --(6.574,2.931)--(6.577,2.932)--(6.580,2.934)--(6.583,2.935)--(6.586,2.937)--(6.589,2.938)%
  --(6.592,2.939)--(6.595,2.941)--(6.598,2.942)--(6.601,2.943)--(6.604,2.945)--(6.607,2.946)%
  --(6.610,2.948)--(6.613,2.949)--(6.616,2.950)--(6.619,2.952)--(6.622,2.953)--(6.625,2.954)%
  --(6.628,2.956)--(6.631,2.957)--(6.634,2.958)--(6.637,2.960)--(6.639,2.961)--(6.642,2.963)%
  --(6.645,2.964)--(6.648,2.965)--(6.651,2.967)--(6.654,2.968)--(6.657,2.969)--(6.660,2.971)%
  --(6.663,2.972)--(6.666,2.974)--(6.669,2.975)--(6.672,2.976)--(6.675,2.978)--(6.678,2.979)%
  --(6.681,2.980)--(6.684,2.982)--(6.687,2.983)--(6.690,2.984)--(6.693,2.986)--(6.696,2.987)%
  --(6.699,2.989)--(6.702,2.990)--(6.705,2.991)--(6.708,2.993)--(6.711,2.994)--(6.714,2.995)%
  --(6.717,2.997)--(6.720,2.998)--(6.723,3.000)--(6.726,3.001)--(6.729,3.002)--(6.732,3.004)%
  --(6.735,3.005)--(6.738,3.006)--(6.741,3.008)--(6.744,3.009)--(6.747,3.011)--(6.750,3.012)%
  --(6.753,3.013)--(6.756,3.015)--(6.759,3.016)--(6.762,3.018)--(6.765,3.019)--(6.768,3.020)%
  --(6.771,3.022)--(6.774,3.023)--(6.777,3.024)--(6.780,3.026)--(6.783,3.027)--(6.786,3.029)%
  --(6.789,3.030)--(6.792,3.031)--(6.795,3.033)--(6.798,3.034)--(6.801,3.035)--(6.804,3.037)%
  --(6.807,3.038)--(6.810,3.040)--(6.813,3.041)--(6.816,3.042)--(6.819,3.044)--(6.822,3.045)%
  --(6.825,3.047)--(6.828,3.048)--(6.831,3.049)--(6.834,3.051)--(6.837,3.052)--(6.840,3.053)%
  --(6.843,3.055)--(6.846,3.056)--(6.848,3.058)--(6.851,3.059)--(6.854,3.060)--(6.857,3.062)%
  --(6.860,3.063)--(6.863,3.065)--(6.866,3.066)--(6.869,3.067)--(6.872,3.069)--(6.875,3.070)%
  --(6.878,3.071)--(6.881,3.073)--(6.884,3.074)--(6.887,3.076)--(6.890,3.077)--(6.893,3.078)%
  --(6.896,3.080)--(6.899,3.081)--(6.902,3.083)--(6.905,3.084)--(6.908,3.085)--(6.911,3.087)%
  --(6.914,3.088)--(6.917,3.089)--(6.920,3.091)--(6.923,3.092)--(6.926,3.094)--(6.929,3.095)%
  --(6.932,3.096)--(6.935,3.098)--(6.938,3.099)--(6.941,3.101)--(6.944,3.102)--(6.947,3.103)%
  --(6.950,3.105)--(6.953,3.106)--(6.956,3.108)--(6.959,3.109)--(6.962,3.110)--(6.965,3.112)%
  --(6.968,3.113)--(6.971,3.115)--(6.974,3.116)--(6.977,3.117)--(6.980,3.119)--(6.983,3.120)%
  --(6.986,3.122)--(6.989,3.123)--(6.992,3.124)--(6.995,3.126)--(6.998,3.127)--(7.001,3.128)%
  --(7.004,3.130)--(7.007,3.131)--(7.010,3.133)--(7.013,3.134)--(7.016,3.135)--(7.019,3.137)%
  --(7.022,3.138)--(7.025,3.140)--(7.028,3.141)--(7.031,3.142)--(7.034,3.144)--(7.037,3.145)%
  --(7.040,3.147)--(7.043,3.148)--(7.046,3.149)--(7.049,3.151)--(7.052,3.152)--(7.055,3.154)%
  --(7.057,3.155)--(7.060,3.156)--(7.063,3.158)--(7.066,3.159)--(7.069,3.161)--(7.072,3.162)%
  --(7.075,3.163)--(7.078,3.165)--(7.081,3.166)--(7.084,3.168)--(7.087,3.169)--(7.090,3.170)%
  --(7.093,3.172)--(7.096,3.173)--(7.099,3.175)--(7.102,3.176)--(7.105,3.177)--(7.108,3.179)%
  --(7.111,3.180)--(7.114,3.182)--(7.117,3.183)--(7.120,3.184)--(7.123,3.186)--(7.126,3.187)%
  --(7.129,3.189)--(7.132,3.190)--(7.135,3.191)--(7.138,3.193)--(7.141,3.194)--(7.144,3.196)%
  --(7.147,3.197)--(7.150,3.198)--(7.153,3.200)--(7.156,3.201)--(7.159,3.203)--(7.162,3.204)%
  --(7.165,3.206)--(7.168,3.207)--(7.171,3.208)--(7.174,3.210)--(7.177,3.211)--(7.180,3.213)%
  --(7.183,3.214)--(7.186,3.215)--(7.189,3.217)--(7.192,3.218)--(7.195,3.220)--(7.198,3.221)%
  --(7.201,3.222)--(7.204,3.224)--(7.207,3.225)--(7.210,3.227)--(7.213,3.228)--(7.216,3.229)%
  --(7.219,3.231)--(7.222,3.232)--(7.225,3.234)--(7.228,3.235)--(7.231,3.236)--(7.234,3.238)%
  --(7.237,3.239)--(7.240,3.241)--(7.243,3.242)--(7.246,3.244)--(7.249,3.245)--(7.252,3.246)%
  --(7.255,3.248)--(7.258,3.249)--(7.261,3.251)--(7.264,3.252)--(7.266,3.253)--(7.269,3.255)%
  --(7.272,3.256)--(7.275,3.258)--(7.278,3.259)--(7.281,3.260)--(7.284,3.262)--(7.287,3.263)%
  --(7.290,3.265)--(7.293,3.266)--(7.296,3.268)--(7.299,3.269)--(7.302,3.270)--(7.305,3.272)%
  --(7.308,3.273)--(7.311,3.275)--(7.314,3.276)--(7.317,3.277)--(7.320,3.279)--(7.323,3.280)%
  --(7.326,3.282)--(7.329,3.283)--(7.332,3.284)--(7.335,3.286)--(7.338,3.287)--(7.341,3.289)%
  --(7.344,3.290)--(7.347,3.292)--(7.350,3.293)--(7.353,3.294)--(7.356,3.296)--(7.359,3.297)%
  --(7.362,3.299)--(7.365,3.300)--(7.368,3.301)--(7.371,3.303)--(7.374,3.304)--(7.377,3.306)%
  --(7.380,3.307)--(7.383,3.309)--(7.386,3.310)--(7.389,3.311)--(7.392,3.313)--(7.395,3.314)%
  --(7.398,3.316)--(7.401,3.317)--(7.404,3.318)--(7.407,3.320)--(7.410,3.321)--(7.413,3.323)%
  --(7.416,3.324)--(7.419,3.326)--(7.422,3.327)--(7.425,3.328)--(7.428,3.330)--(7.431,3.331)%
  --(7.434,3.333)--(7.437,3.334)--(7.440,3.336)--(7.443,3.337)--(7.446,3.338)--(7.449,3.340)%
  --(7.452,3.341)--(7.455,3.343)--(7.458,3.344)--(7.461,3.345)--(7.464,3.347)--(7.467,3.348)%
  --(7.470,3.350)--(7.473,3.351)--(7.476,3.353)--(7.478,3.354)--(7.481,3.355)--(7.484,3.357)%
  --(7.487,3.358)--(7.490,3.360)--(7.493,3.361)--(7.496,3.363)--(7.499,3.364)--(7.502,3.365)%
  --(7.505,3.367)--(7.508,3.368)--(7.511,3.370)--(7.514,3.371)--(7.517,3.372)--(7.520,3.374)%
  --(7.523,3.375)--(7.526,3.377)--(7.529,3.378)--(7.532,3.380)--(7.535,3.381)--(7.538,3.382)%
  --(7.541,3.384)--(7.544,3.385)--(7.547,3.387)--(7.550,3.388)--(7.553,3.390)--(7.556,3.391)%
  --(7.559,3.392)--(7.562,3.394)--(7.565,3.395)--(7.568,3.397)--(7.571,3.398)--(7.574,3.400)%
  --(7.577,3.401)--(7.580,3.402)--(7.583,3.404)--(7.586,3.405)--(7.589,3.407)--(7.592,3.408)%
  --(7.595,3.410)--(7.598,3.411)--(7.601,3.412)--(7.604,3.414)--(7.607,3.415)--(7.610,3.417)%
  --(7.613,3.418)--(7.616,3.420)--(7.619,3.421)--(7.622,3.422)--(7.625,3.424)--(7.628,3.425)%
  --(7.631,3.427)--(7.634,3.428)--(7.637,3.430)--(7.640,3.431)--(7.643,3.432)--(7.646,3.434)%
  --(7.649,3.435)--(7.652,3.437)--(7.655,3.438)--(7.658,3.440)--(7.661,3.441)--(7.664,3.442)%
  --(7.667,3.444)--(7.670,3.445)--(7.673,3.447)--(7.676,3.448)--(7.679,3.450)--(7.682,3.451)%
  --(7.685,3.452)--(7.687,3.454)--(7.690,3.455)--(7.693,3.457)--(7.696,3.458)--(7.699,3.460)%
  --(7.702,3.461)--(7.705,3.462)--(7.708,3.464)--(7.711,3.465)--(7.714,3.467)--(7.717,3.468)%
  --(7.720,3.470)--(7.723,3.471)--(7.726,3.472)--(7.729,3.474)--(7.732,3.475)--(7.735,3.477)%
  --(7.738,3.478)--(7.741,3.480)--(7.744,3.481)--(7.747,3.483)--(7.750,3.484)--(7.753,3.485)%
  --(7.756,3.487)--(7.759,3.488)--(7.762,3.490)--(7.765,3.491)--(7.768,3.493)--(7.771,3.494)%
  --(7.774,3.495)--(7.777,3.497)--(7.780,3.498)--(7.783,3.500)--(7.786,3.501)--(7.789,3.503)%
  --(7.792,3.504)--(7.795,3.505)--(7.798,3.507)--(7.801,3.508)--(7.804,3.510)--(7.807,3.511)%
  --(7.810,3.513)--(7.813,3.514)--(7.816,3.516)--(7.819,3.517)--(7.822,3.518)--(7.825,3.520)%
  --(7.828,3.521)--(7.831,3.523)--(7.834,3.524)--(7.837,3.526)--(7.840,3.527)--(7.843,3.528)%
  --(7.846,3.530)--(7.849,3.531)--(7.852,3.533)--(7.855,3.534)--(7.858,3.536)--(7.861,3.537)%
  --(7.864,3.539)--(7.867,3.540)--(7.870,3.541)--(7.873,3.543)--(7.876,3.544)--(7.879,3.546)%
  --(7.882,3.547)--(7.885,3.549)--(7.888,3.550)--(7.891,3.551)--(7.894,3.553)--(7.896,3.554)%
  --(7.899,3.556)--(7.902,3.557)--(7.905,3.559)--(7.908,3.560)--(7.911,3.562)--(7.914,3.563)%
  --(7.917,3.564)--(7.920,3.566)--(7.923,3.567)--(7.926,3.569)--(7.929,3.570)--(7.932,3.572)%
  --(7.935,3.573)--(7.938,3.575)--(7.941,3.576)--(7.944,3.577)--(7.947,3.579)--(7.950,3.580)%
  --(7.953,3.582)--(7.956,3.583)--(7.959,3.585)--(7.962,3.586)--(7.965,3.588)--(7.968,3.589)%
  --(7.971,3.590)--(7.974,3.592)--(7.977,3.593)--(7.980,3.595)--(7.983,3.596)--(7.986,3.598)%
  --(7.989,3.599)--(7.992,3.601)--(7.995,3.602)--(7.998,3.603)--(8.001,3.605)--(8.004,3.606)%
  --(8.007,3.608)--(8.010,3.609)--(8.013,3.611)--(8.016,3.612)--(8.019,3.614)--(8.022,3.615)%
  --(8.025,3.616)--(8.028,3.618)--(8.031,3.619)--(8.034,3.621)--(8.037,3.622)--(8.040,3.624)%
  --(8.043,3.625)--(8.046,3.627)--(8.049,3.628)--(8.052,3.629)--(8.055,3.631)--(8.058,3.632)%
  --(8.061,3.634)--(8.064,3.635)--(8.067,3.637)--(8.070,3.638)--(8.073,3.640)--(8.076,3.641)%
  --(8.079,3.642)--(8.082,3.644)--(8.085,3.645)--(8.088,3.647)--(8.091,3.648)--(8.094,3.650)%
  --(8.097,3.651)--(8.100,3.653)--(8.103,3.654)--(8.105,3.655)--(8.108,3.657)--(8.111,3.658)%
  --(8.114,3.660)--(8.117,3.661)--(8.120,3.663)--(8.123,3.664)--(8.126,3.666)--(8.129,3.667)%
  --(8.132,3.668)--(8.135,3.670)--(8.138,3.671)--(8.141,3.673)--(8.144,3.674)--(8.147,3.676)%
  --(8.150,3.677)--(8.153,3.679)--(8.156,3.680)--(8.159,3.682)--(8.162,3.683)--(8.165,3.684)%
  --(8.168,3.686)--(8.171,3.687)--(8.174,3.689)--(8.177,3.690)--(8.180,3.692)--(8.183,3.693)%
  --(8.186,3.695)--(8.189,3.696)--(8.192,3.697)--(8.195,3.699)--(8.198,3.700)--(8.201,3.702)%
  --(8.204,3.703)--(8.207,3.705)--(8.210,3.706)--(8.213,3.708)--(8.216,3.709)--(8.219,3.711)%
  --(8.222,3.712)--(8.225,3.713)--(8.228,3.715)--(8.231,3.716)--(8.234,3.718)--(8.237,3.719)%
  --(8.240,3.721)--(8.243,3.722)--(8.246,3.724)--(8.249,3.725)--(8.252,3.726)--(8.255,3.728)%
  --(8.258,3.729)--(8.261,3.731)--(8.264,3.732)--(8.267,3.734)--(8.270,3.735)--(8.273,3.737)%
  --(8.276,3.738)--(8.279,3.740)--(8.282,3.741)--(8.285,3.742)--(8.288,3.744)--(8.291,3.745)%
  --(8.294,3.747)--(8.297,3.748)--(8.300,3.750)--(8.303,3.751)--(8.306,3.753)--(8.309,3.754)%
  --(8.312,3.756)--(8.314,3.757)--(8.317,3.758)--(8.320,3.760)--(8.323,3.761)--(8.326,3.763)%
  --(8.329,3.764)--(8.332,3.766)--(8.335,3.767)--(8.338,3.769)--(8.341,3.770)--(8.344,3.772)%
  --(8.347,3.773)--(8.350,3.774)--(8.353,3.776)--(8.356,3.777)--(8.359,3.779)--(8.362,3.780)%
  --(8.365,3.782)--(8.368,3.783)--(8.371,3.785)--(8.374,3.786)--(8.377,3.788)--(8.380,3.789)%
  --(8.383,3.790)--(8.386,3.792)--(8.389,3.793)--(8.392,3.795)--(8.395,3.796)--(8.398,3.798)%
  --(8.401,3.799)--(8.404,3.801)--(8.407,3.802)--(8.410,3.804)--(8.413,3.805)--(8.416,3.806)%
  --(8.419,3.808)--(8.422,3.809)--(8.425,3.811)--(8.428,3.812)--(8.431,3.814)--(8.434,3.815)%
  --(8.437,3.817)--(8.440,3.818)--(8.443,3.820)--(8.446,3.821)--(8.449,3.823)--(8.452,3.824)%
  --(8.455,3.825)--(8.458,3.827)--(8.461,3.828)--(8.464,3.830)--(8.467,3.831)--(8.470,3.833)%
  --(8.473,3.834)--(8.476,3.836)--(8.479,3.837)--(8.482,3.839)--(8.485,3.840)--(8.488,3.841)%
  --(8.491,3.843)--(8.494,3.844)--(8.497,3.846)--(8.500,3.847)--(8.503,3.849)--(8.506,3.850)%
  --(8.509,3.852)--(8.512,3.853)--(8.515,3.855)--(8.518,3.856)--(8.521,3.858)--(8.523,3.859)%
  --(8.526,3.860)--(8.529,3.862)--(8.532,3.863)--(8.535,3.865)--(8.538,3.866)--(8.541,3.868)%
  --(8.544,3.869)--(8.547,3.871)--(8.550,3.872)--(8.553,3.874)--(8.556,3.875)--(8.559,3.877)%
  --(8.562,3.878)--(8.565,3.879)--(8.568,3.881)--(8.571,3.882)--(8.574,3.884)--(8.577,3.885)%
  --(8.580,3.887)--(8.583,3.888)--(8.586,3.890)--(8.589,3.891)--(8.592,3.893)--(8.595,3.894)%
  --(8.598,3.896)--(8.601,3.897)--(8.604,3.898)--(8.607,3.900)--(8.610,3.901)--(8.613,3.903)%
  --(8.616,3.904)--(8.619,3.906)--(8.622,3.907)--(8.625,3.909)--(8.628,3.910)--(8.631,3.912)%
  --(8.634,3.913)--(8.637,3.915)--(8.640,3.916)--(8.643,3.917)--(8.646,3.919)--(8.649,3.920)%
  --(8.652,3.922)--(8.655,3.923)--(8.658,3.925)--(8.661,3.926)--(8.664,3.928)--(8.667,3.929)%
  --(8.670,3.931)--(8.673,3.932)--(8.676,3.934)--(8.679,3.935)--(8.682,3.937)--(8.685,3.938)%
  --(8.688,3.939)--(8.691,3.941)--(8.694,3.942)--(8.697,3.944)--(8.700,3.945)--(8.703,3.947)%
  --(8.706,3.948)--(8.709,3.950)--(8.712,3.951)--(8.715,3.953)--(8.718,3.954)--(8.721,3.956)%
  --(8.724,3.957)--(8.727,3.958)--(8.730,3.960)--(8.733,3.961)--(8.735,3.963)--(8.738,3.964)%
  --(8.741,3.966)--(8.744,3.967)--(8.747,3.969)--(8.750,3.970)--(8.753,3.972)--(8.756,3.973)%
  --(8.759,3.975)--(8.762,3.976)--(8.765,3.978)--(8.768,3.979)--(8.771,3.980)--(8.774,3.982)%
  --(8.777,3.983)--(8.780,3.985)--(8.783,3.986)--(8.786,3.988)--(8.789,3.989)--(8.792,3.991)%
  --(8.795,3.992)--(8.798,3.994)--(8.801,3.995)--(8.804,3.997)--(8.807,3.998)--(8.810,4.000)%
  --(8.813,4.001)--(8.816,4.002)--(8.819,4.004)--(8.822,4.005)--(8.825,4.007)--(8.828,4.008)%
  --(8.831,4.010)--(8.834,4.011)--(8.837,4.013)--(8.840,4.014)--(8.843,4.016)--(8.846,4.017)%
  --(8.849,4.019)--(8.852,4.020)--(8.855,4.022)--(8.858,4.023)--(8.861,4.025)--(8.864,4.026)%
  --(8.867,4.027)--(8.870,4.029)--(8.873,4.030)--(8.876,4.032)--(8.879,4.033)--(8.882,4.035)%
  --(8.885,4.036)--(8.888,4.038)--(8.891,4.039)--(8.894,4.041)--(8.897,4.042)--(8.900,4.044)%
  --(8.903,4.045)--(8.906,4.047)--(8.909,4.048)--(8.912,4.049)--(8.915,4.051)--(8.918,4.052)%
  --(8.921,4.054)--(8.924,4.055)--(8.927,4.057)--(8.930,4.058)--(8.933,4.060)--(8.936,4.061)%
  --(8.939,4.063)--(8.942,4.064)--(8.944,4.066)--(8.947,4.067)--(8.950,4.069)--(8.953,4.070)%
  --(8.956,4.072)--(8.959,4.073)--(8.962,4.074)--(8.965,4.076)--(8.968,4.077)--(8.971,4.079)%
  --(8.974,4.080)--(8.977,4.082)--(8.980,4.083)--(8.983,4.085)--(8.986,4.086)--(8.989,4.088)%
  --(8.992,4.089)--(8.995,4.091)--(8.998,4.092)--(9.001,4.094)--(9.004,4.095)--(9.007,4.097)%
  --(9.010,4.098)--(9.013,4.100)--(9.016,4.101)--(9.019,4.102)--(9.022,4.104)--(9.025,4.105)%
  --(9.028,4.107)--(9.031,4.108)--(9.034,4.110)--(9.037,4.111)--(9.040,4.113)--(9.043,4.114)%
  --(9.046,4.116)--(9.049,4.117)--(9.052,4.119)--(9.055,4.120)--(9.058,4.122)--(9.061,4.123)%
  --(9.064,4.125)--(9.067,4.126)--(9.070,4.127)--(9.073,4.129)--(9.076,4.130)--(9.079,4.132)%
  --(9.082,4.133)--(9.085,4.135)--(9.088,4.136)--(9.091,4.138)--(9.094,4.139)--(9.097,4.141)%
  --(9.100,4.142)--(9.103,4.144)--(9.106,4.145)--(9.109,4.147)--(9.112,4.148)--(9.115,4.150)%
  --(9.118,4.151)--(9.121,4.153)--(9.124,4.154)--(9.127,4.155)--(9.130,4.157)--(9.133,4.158)%
  --(9.136,4.160)--(9.139,4.161)--(9.142,4.163)--(9.145,4.164)--(9.148,4.166)--(9.151,4.167)%
  --(9.153,4.169)--(9.156,4.170)--(9.159,4.172)--(9.162,4.173)--(9.165,4.175)--(9.168,4.176)%
  --(9.171,4.178)--(9.174,4.179)--(9.177,4.181)--(9.180,4.182)--(9.183,4.184)--(9.186,4.185)%
  --(9.189,4.186)--(9.192,4.188)--(9.195,4.189)--(9.198,4.191)--(9.201,4.192)--(9.204,4.194)%
  --(9.207,4.195)--(9.210,4.197)--(9.213,4.198)--(9.216,4.200)--(9.219,4.201)--(9.222,4.203)%
  --(9.225,4.204)--(9.228,4.206)--(9.231,4.207)--(9.234,4.209)--(9.237,4.210)--(9.240,4.212)%
  --(9.243,4.213)--(9.246,4.215)--(9.249,4.216)--(9.252,4.217)--(9.255,4.219)--(9.258,4.220)%
  --(9.261,4.222)--(9.264,4.223)--(9.267,4.225)--(9.270,4.226)--(9.273,4.228)--(9.276,4.229)%
  --(9.279,4.231)--(9.282,4.232)--(9.285,4.234)--(9.288,4.235)--(9.291,4.237)--(9.294,4.238)%
  --(9.297,4.240)--(9.300,4.241)--(9.303,4.243)--(9.306,4.244)--(9.309,4.246)--(9.312,4.247)%
  --(9.315,4.248)--(9.318,4.250)--(9.321,4.251)--(9.324,4.253)--(9.327,4.254)--(9.330,4.256)%
  --(9.333,4.257)--(9.336,4.259)--(9.339,4.260)--(9.342,4.262)--(9.345,4.263)--(9.348,4.265)%
  --(9.351,4.266)--(9.354,4.268)--(9.357,4.269)--(9.360,4.271)--(9.362,4.272)--(9.365,4.274)%
  --(9.368,4.275)--(9.371,4.277)--(9.374,4.278)--(9.377,4.280)--(9.380,4.281)--(9.383,4.282)%
  --(9.386,4.284)--(9.389,4.285)--(9.392,4.287)--(9.395,4.288)--(9.398,4.290)--(9.401,4.291)%
  --(9.404,4.293)--(9.407,4.294)--(9.410,4.296)--(9.413,4.297)--(9.416,4.299)--(9.419,4.300)%
  --(9.422,4.302)--(9.425,4.303)--(9.428,4.305)--(9.431,4.306)--(9.434,4.308)--(9.437,4.309)%
  --(9.440,4.311)--(9.443,4.312)--(9.446,4.314)--(9.449,4.315)--(9.452,4.317)--(9.455,4.318)%
  --(9.458,4.319)--(9.461,4.321)--(9.464,4.322)--(9.467,4.324)--(9.470,4.325)--(9.473,4.327)%
  --(9.476,4.328)--(9.479,4.330)--(9.482,4.331)--(9.485,4.333)--(9.488,4.334)--(9.491,4.336)%
  --(9.494,4.337)--(9.497,4.339)--(9.500,4.340)--(9.503,4.342)--(9.506,4.343)--(9.509,4.345)%
  --(9.512,4.346)--(9.515,4.348)--(9.518,4.349)--(9.521,4.351)--(9.524,4.352)--(9.527,4.354)%
  --(9.530,4.355)--(9.533,4.357)--(9.536,4.358)--(9.539,4.359)--(9.542,4.361)--(9.545,4.362)%
  --(9.548,4.364)--(9.551,4.365)--(9.554,4.367)--(9.557,4.368)--(9.560,4.370)--(9.563,4.371)%
  --(9.566,4.373)--(9.569,4.374)--(9.571,4.376)--(9.574,4.377)--(9.577,4.379)--(9.580,4.380)%
  --(9.583,4.382)--(9.586,4.383)--(9.589,4.385)--(9.592,4.386)--(9.595,4.388)--(9.598,4.389)%
  --(9.601,4.391)--(9.604,4.392)--(9.607,4.394)--(9.610,4.395)--(9.613,4.397)--(9.616,4.398)%
  --(9.619,4.400)--(9.622,4.401)--(9.625,4.402)--(9.628,4.404)--(9.631,4.405)--(9.634,4.407)%
  --(9.637,4.408)--(9.640,4.410)--(9.643,4.411)--(9.646,4.413)--(9.649,4.414)--(9.652,4.416)%
  --(9.655,4.417)--(9.658,4.419)--(9.661,4.420)--(9.664,4.422)--(9.667,4.423)--(9.670,4.425)%
  --(9.673,4.426)--(9.676,4.428)--(9.679,4.429)--(9.682,4.431)--(9.685,4.432)--(9.688,4.434)%
  --(9.691,4.435)--(9.694,4.437)--(9.697,4.438)--(9.700,4.440)--(9.703,4.441)--(9.706,4.443)%
  --(9.709,4.444)--(9.712,4.446)--(9.715,4.447)--(9.718,4.448)--(9.721,4.450)--(9.724,4.451)%
  --(9.727,4.453)--(9.730,4.454)--(9.733,4.456)--(9.736,4.457)--(9.739,4.459)--(9.742,4.460)%
  --(9.745,4.462)--(9.748,4.463)--(9.751,4.465)--(9.754,4.466)--(9.757,4.468)--(9.760,4.469)%
  --(9.763,4.471)--(9.766,4.472)--(9.769,4.474)--(9.772,4.475)--(9.775,4.477)--(9.778,4.478)%
  --(9.780,4.480)--(9.783,4.481)--(9.786,4.483)--(9.789,4.484)--(9.792,4.486)--(9.795,4.487)%
  --(9.798,4.489)--(9.801,4.490)--(9.804,4.492)--(9.807,4.493)--(9.810,4.495)--(9.813,4.496)%
  --(9.816,4.498)--(9.819,4.499)--(9.822,4.500)--(9.825,4.502)--(9.828,4.503)--(9.831,4.505)%
  --(9.834,4.506)--(9.837,4.508)--(9.840,4.509)--(9.843,4.511)--(9.846,4.512)--(9.849,4.514)%
  --(9.852,4.515)--(9.855,4.517)--(9.858,4.518)--(9.861,4.520)--(9.864,4.521)--(9.867,4.523)%
  --(9.870,4.524)--(9.873,4.526)--(9.876,4.527)--(9.879,4.529)--(9.882,4.530)--(9.885,4.532)%
  --(9.888,4.533)--(9.891,4.535)--(9.894,4.536)--(9.897,4.538)--(9.900,4.539)--(9.903,4.541)%
  --(9.906,4.542)--(9.909,4.544)--(9.912,4.545)--(9.915,4.547)--(9.918,4.548)--(9.921,4.550)%
  --(9.924,4.551)--(9.927,4.553)--(9.930,4.554)--(9.933,4.556)--(9.936,4.557)--(9.939,4.558)%
  --(9.942,4.560)--(9.945,4.561)--(9.948,4.563)--(9.951,4.564)--(9.954,4.566)--(9.957,4.567)%
  --(9.960,4.569)--(9.963,4.570)--(9.966,4.572)--(9.969,4.573)--(9.972,4.575)--(9.975,4.576)%
  --(9.978,4.578)--(9.981,4.579)--(9.984,4.581)--(9.987,4.582)--(9.990,4.584)--(9.992,4.585)%
  --(9.995,4.587)--(9.998,4.588)--(10.001,4.590)--(10.004,4.591)--(10.007,4.593)--(10.010,4.594)%
  --(10.013,4.596)--(10.016,4.597)--(10.019,4.599)--(10.022,4.600)--(10.025,4.602)--(10.028,4.603)%
  --(10.031,4.605)--(10.034,4.606)--(10.037,4.608)--(10.040,4.609)--(10.043,4.611)--(10.046,4.612)%
  --(10.049,4.614)--(10.052,4.615)--(10.055,4.617)--(10.058,4.618)--(10.061,4.620)--(10.064,4.621)%
  --(10.067,4.623)--(10.070,4.624)--(10.073,4.625)--(10.076,4.627)--(10.079,4.628)--(10.082,4.630)%
  --(10.085,4.631)--(10.088,4.633)--(10.091,4.634)--(10.094,4.636)--(10.097,4.637)--(10.100,4.639)%
  --(10.103,4.640)--(10.106,4.642)--(10.109,4.643)--(10.112,4.645)--(10.115,4.646)--(10.118,4.648)%
  --(10.121,4.649)--(10.124,4.651)--(10.127,4.652)--(10.130,4.654)--(10.133,4.655)--(10.136,4.657)%
  --(10.139,4.658)--(10.142,4.660)--(10.145,4.661)--(10.148,4.663)--(10.151,4.664)--(10.154,4.666)%
  --(10.157,4.667)--(10.160,4.669)--(10.163,4.670)--(10.166,4.672)--(10.169,4.673)--(10.172,4.675)%
  --(10.175,4.676)--(10.178,4.678)--(10.181,4.679)--(10.184,4.681)--(10.187,4.682)--(10.190,4.684)%
  --(10.193,4.685)--(10.196,4.687)--(10.199,4.688)--(10.201,4.690)--(10.204,4.691)--(10.207,4.693)%
  --(10.210,4.694)--(10.213,4.696)--(10.216,4.697)--(10.219,4.699)--(10.222,4.700)--(10.225,4.702)%
  --(10.228,4.703)--(10.231,4.705)--(10.234,4.706)--(10.237,4.707)--(10.240,4.709)--(10.243,4.710)%
  --(10.246,4.712)--(10.249,4.713)--(10.252,4.715)--(10.255,4.716)--(10.258,4.718)--(10.261,4.719)%
  --(10.264,4.721)--(10.267,4.722)--(10.270,4.724)--(10.273,4.725)--(10.276,4.727)--(10.279,4.728)%
  --(10.282,4.730)--(10.285,4.731)--(10.288,4.733)--(10.291,4.734)--(10.294,4.736)--(10.297,4.737)%
  --(10.300,4.739)--(10.303,4.740)--(10.306,4.742)--(10.309,4.743)--(10.312,4.745)--(10.315,4.746)%
  --(10.318,4.748)--(10.321,4.749)--(10.324,4.751)--(10.327,4.752)--(10.330,4.754)--(10.333,4.755)%
  --(10.336,4.757)--(10.339,4.758)--(10.342,4.760)--(10.345,4.761)--(10.348,4.763)--(10.351,4.764)%
  --(10.354,4.766)--(10.357,4.767)--(10.360,4.769)--(10.363,4.770)--(10.366,4.772)--(10.369,4.773)%
  --(10.372,4.775)--(10.375,4.776)--(10.378,4.778)--(10.381,4.779)--(10.384,4.781)--(10.387,4.782)%
  --(10.390,4.784)--(10.393,4.785)--(10.396,4.787)--(10.399,4.788)--(10.402,4.790)--(10.405,4.791)%
  --(10.408,4.793)--(10.410,4.794)--(10.413,4.796)--(10.416,4.797)--(10.419,4.799)--(10.422,4.800)%
  --(10.425,4.802)--(10.428,4.803)--(10.431,4.805)--(10.434,4.806)--(10.437,4.808)--(10.440,4.809)%
  --(10.443,4.811)--(10.446,4.812)--(10.449,4.814)--(10.452,4.815)--(10.455,4.816)--(10.458,4.818)%
  --(10.461,4.819)--(10.464,4.821)--(10.467,4.822)--(10.470,4.824)--(10.473,4.825)--(10.476,4.827)%
  --(10.479,4.828)--(10.482,4.830)--(10.485,4.831)--(10.488,4.833)--(10.491,4.834)--(10.494,4.836)%
  --(10.497,4.837)--(10.500,4.839)--(10.503,4.840)--(10.506,4.842)--(10.509,4.843)--(10.512,4.845)%
  --(10.515,4.846)--(10.518,4.848)--(10.521,4.849)--(10.524,4.851)--(10.527,4.852)--(10.530,4.854)%
  --(10.533,4.855)--(10.536,4.857)--(10.539,4.858)--(10.542,4.860)--(10.545,4.861)--(10.548,4.863)%
  --(10.551,4.864)--(10.554,4.866)--(10.557,4.867)--(10.560,4.869)--(10.563,4.870)--(10.566,4.872)%
  --(10.569,4.873)--(10.572,4.875)--(10.575,4.876)--(10.578,4.878)--(10.581,4.879)--(10.584,4.881)%
  --(10.587,4.882)--(10.590,4.884)--(10.593,4.885)--(10.596,4.887)--(10.599,4.888)--(10.602,4.890)%
  --(10.605,4.891)--(10.608,4.893)--(10.611,4.894)--(10.614,4.896)--(10.617,4.897)--(10.619,4.899)%
  --(10.622,4.900)--(10.625,4.902)--(10.628,4.903)--(10.631,4.905)--(10.634,4.906)--(10.637,4.908)%
  --(10.640,4.909)--(10.643,4.911)--(10.646,4.912)--(10.649,4.914)--(10.652,4.915)--(10.655,4.917)%
  --(10.658,4.918)--(10.661,4.920)--(10.664,4.921)--(10.667,4.923)--(10.670,4.924)--(10.673,4.926)%
  --(10.676,4.927)--(10.679,4.929)--(10.682,4.930)--(10.685,4.932)--(10.688,4.933)--(10.691,4.935)%
  --(10.694,4.936)--(10.697,4.938)--(10.700,4.939)--(10.703,4.941)--(10.706,4.942)--(10.709,4.944)%
  --(10.712,4.945)--(10.715,4.947)--(10.718,4.948)--(10.721,4.950)--(10.724,4.951)--(10.727,4.953)%
  --(10.730,4.954)--(10.733,4.956)--(10.736,4.957)--(10.739,4.959)--(10.742,4.960)--(10.745,4.962)%
  --(10.748,4.963)--(10.751,4.965)--(10.754,4.966)--(10.757,4.968)--(10.760,4.969)--(10.763,4.971)%
  --(10.766,4.972)--(10.769,4.974)--(10.772,4.975)--(10.775,4.977)--(10.778,4.978)--(10.781,4.980)%
  --(10.784,4.981)--(10.787,4.983)--(10.790,4.984)--(10.793,4.986)--(10.796,4.987)--(10.799,4.989)%
  --(10.802,4.990)--(10.805,4.992)--(10.808,4.993)--(10.811,4.995)--(10.814,4.996)--(10.817,4.998)%
  --(10.820,4.999)--(10.823,5.001)--(10.826,5.002)--(10.828,5.004)--(10.831,5.005)--(10.834,5.007)%
  --(10.837,5.008)--(10.840,5.010)--(10.843,5.011)--(10.846,5.013)--(10.849,5.014)--(10.852,5.016)%
  --(10.855,5.017)--(10.858,5.019)--(10.861,5.020)--(10.864,5.022)--(10.867,5.023)--(10.870,5.025)%
  --(10.873,5.026)--(10.876,5.028)--(10.879,5.029)--(10.882,5.031)--(10.885,5.032)--(10.888,5.034)%
  --(10.891,5.035)--(10.894,5.037)--(10.897,5.038)--(10.900,5.040)--(10.903,5.041)--(10.906,5.043)%
  --(10.909,5.044)--(10.912,5.046)--(10.915,5.047)--(10.918,5.049)--(10.921,5.050)--(10.924,5.052)%
  --(10.927,5.053)--(10.930,5.055)--(10.933,5.056)--(10.936,5.058)--(10.939,5.059)--(10.942,5.060)%
  --(10.945,5.062)--(10.948,5.063)--(10.951,5.065)--(10.954,5.066)--(10.957,5.068)--(10.960,5.069)%
  --(10.963,5.071)--(10.966,5.072)--(10.969,5.074)--(10.972,5.075)--(10.975,5.077)--(10.978,5.078)%
  --(10.981,5.080)--(10.984,5.081)--(10.987,5.083)--(10.990,5.084)--(10.993,5.086)--(10.996,5.087)%
  --(10.999,5.089)--(11.002,5.090)--(11.005,5.092)--(11.008,5.093)--(11.011,5.095)--(11.014,5.096)%
  --(11.017,5.098)--(11.020,5.099)--(11.023,5.101)--(11.026,5.102)--(11.029,5.104)--(11.032,5.105)%
  --(11.035,5.107)--(11.037,5.108)--(11.040,5.110)--(11.043,5.111)--(11.046,5.113)--(11.049,5.114)%
  --(11.052,5.116)--(11.055,5.117)--(11.058,5.119)--(11.061,5.120)--(11.064,5.122)--(11.067,5.123)%
  --(11.070,5.125)--(11.073,5.126)--(11.076,5.128)--(11.079,5.129)--(11.082,5.131)--(11.085,5.132)%
  --(11.088,5.134)--(11.091,5.135)--(11.094,5.137)--(11.097,5.138)--(11.100,5.140)--(11.103,5.142)%
  --(11.106,5.143)--(11.109,5.145)--(11.112,5.146)--(11.115,5.148)--(11.118,5.149)--(11.121,5.151)%
  --(11.124,5.152)--(11.127,5.154)--(11.130,5.155)--(11.133,5.157)--(11.136,5.158)--(11.139,5.160)%
  --(11.142,5.161)--(11.145,5.163)--(11.148,5.164)--(11.151,5.166)--(11.154,5.167)--(11.157,5.169)%
  --(11.160,5.170)--(11.163,5.172)--(11.166,5.173)--(11.169,5.175)--(11.172,5.176)--(11.175,5.178)%
  --(11.178,5.179)--(11.181,5.181)--(11.184,5.182)--(11.187,5.184)--(11.190,5.185)--(11.193,5.187)%
  --(11.196,5.188)--(11.199,5.190)--(11.202,5.191)--(11.205,5.193)--(11.208,5.194)--(11.211,5.196)%
  --(11.214,5.197)--(11.217,5.199)--(11.220,5.200)--(11.223,5.202)--(11.226,5.203)--(11.229,5.205)%
  --(11.232,5.206)--(11.235,5.208)--(11.238,5.209)--(11.241,5.211)--(11.244,5.212)--(11.247,5.214)%
  --(11.249,5.215)--(11.252,5.217)--(11.255,5.218)--(11.258,5.220)--(11.261,5.221)--(11.264,5.223)%
  --(11.267,5.224)--(11.270,5.226)--(11.273,5.227)--(11.276,5.229)--(11.279,5.230)--(11.282,5.232)%
  --(11.285,5.233)--(11.288,5.235)--(11.291,5.236)--(11.294,5.238)--(11.297,5.239)--(11.300,5.241)%
  --(11.303,5.242)--(11.306,5.244)--(11.309,5.245)--(11.312,5.247)--(11.315,5.248)--(11.318,5.250)%
  --(11.321,5.251)--(11.324,5.253)--(11.327,5.254)--(11.330,5.256)--(11.333,5.257)--(11.336,5.259)%
  --(11.339,5.260)--(11.342,5.262)--(11.345,5.263)--(11.348,5.265)--(11.351,5.266)--(11.354,5.268)%
  --(11.357,5.269)--(11.360,5.271)--(11.363,5.272)--(11.366,5.274)--(11.369,5.275)--(11.372,5.277)%
  --(11.375,5.278)--(11.378,5.280)--(11.381,5.281)--(11.384,5.283)--(11.387,5.284)--(11.390,5.286)%
  --(11.393,5.287)--(11.396,5.289)--(11.399,5.290)--(11.402,5.292)--(11.405,5.293)--(11.408,5.295)%
  --(11.411,5.296)--(11.414,5.298)--(11.417,5.299)--(11.420,5.301)--(11.423,5.302)--(11.426,5.304)%
  --(11.429,5.305)--(11.432,5.307)--(11.435,5.308)--(11.438,5.310)--(11.441,5.311)--(11.444,5.313)%
  --(11.447,5.314)--(11.450,5.316)--(11.453,5.317)--(11.456,5.319)--(11.458,5.320)--(11.461,5.322)%
  --(11.464,5.323)--(11.467,5.325)--(11.470,5.326)--(11.473,5.328)--(11.476,5.329)--(11.479,5.331)%
  --(11.482,5.332)--(11.485,5.334)--(11.488,5.335)--(11.491,5.337)--(11.494,5.338)--(11.497,5.340)%
  --(11.500,5.341)--(11.503,5.343)--(11.506,5.344)--(11.509,5.346)--(11.512,5.347)--(11.515,5.349)%
  --(11.518,5.350)--(11.521,5.352)--(11.524,5.353)--(11.527,5.355)--(11.530,5.356)--(11.533,5.358)%
  --(11.536,5.359)--(11.539,5.361)--(11.542,5.362)--(11.545,5.364)--(11.548,5.365)--(11.551,5.367)%
  --(11.554,5.368)--(11.557,5.370)--(11.560,5.371)--(11.563,5.373)--(11.566,5.374)--(11.569,5.376)%
  --(11.572,5.377)--(11.575,5.379)--(11.578,5.380)--(11.581,5.382)--(11.584,5.383)--(11.587,5.385)%
  --(11.590,5.386)--(11.593,5.388)--(11.596,5.389)--(11.599,5.391)--(11.602,5.392)--(11.605,5.394)%
  --(11.608,5.395)--(11.611,5.397)--(11.614,5.398)--(11.617,5.400)--(11.620,5.401)--(11.623,5.403)%
  --(11.626,5.404)--(11.629,5.406)--(11.632,5.408)--(11.635,5.409)--(11.638,5.411)--(11.641,5.412)%
  --(11.644,5.414)--(11.647,5.415)--(11.650,5.417)--(11.653,5.418)--(11.656,5.420)--(11.659,5.421)%
  --(11.662,5.423)--(11.665,5.424)--(11.667,5.426)--(11.670,5.427)--(11.673,5.429)--(11.676,5.430)%
  --(11.679,5.432)--(11.682,5.433)--(11.685,5.435)--(11.688,5.436)--(11.691,5.438)--(11.694,5.439)%
  --(11.697,5.441)--(11.700,5.442)--(11.703,5.444)--(11.706,5.445)--(11.709,5.447)--(11.712,5.448)%
  --(11.715,5.450)--(11.718,5.451)--(11.721,5.453)--(11.724,5.454)--(11.727,5.456)--(11.730,5.457)%
  --(11.733,5.459)--(11.736,5.460)--(11.739,5.462)--(11.742,5.463)--(11.745,5.465)--(11.748,5.466)%
  --(11.751,5.468)--(11.754,5.469)--(11.757,5.471)--(11.760,5.472)--(11.763,5.474)--(11.766,5.475)%
  --(11.769,5.477)--(11.772,5.478)--(11.775,5.480)--(11.778,5.481)--(11.781,5.483)--(11.784,5.484)%
  --(11.787,5.486)--(11.790,5.487)--(11.793,5.489)--(11.796,5.490)--(11.799,5.492)--(11.802,5.493)%
  --(11.805,5.495)--(11.808,5.496)--(11.811,5.498)--(11.814,5.499)--(11.817,5.501)--(11.820,5.502)%
  --(11.823,5.504)--(11.826,5.505)--(11.829,5.507)--(11.832,5.508)--(11.835,5.510)--(11.838,5.511)%
  --(11.841,5.513)--(11.844,5.514)--(11.847,5.516)--(11.850,5.517)--(11.853,5.519)--(11.856,5.520)%
  --(11.859,5.522)--(11.862,5.523)--(11.865,5.525)--(11.868,5.526)--(11.871,5.528)--(11.874,5.529)%
  --(11.876,5.531)--(11.879,5.532)--(11.882,5.534)--(11.885,5.535)--(11.888,5.537)--(11.891,5.538)%
  --(11.894,5.540)--(11.897,5.542)--(11.900,5.543)--(11.903,5.545)--(11.906,5.546)--(11.909,5.548)%
  --(11.912,5.549)--(11.915,5.551)--(11.918,5.552)--(11.921,5.554)--(11.924,5.555)--(11.927,5.557)%
  --(11.930,5.558)--(11.933,5.560)--(11.936,5.561)--(11.939,5.563)--(11.942,5.564)--(11.945,5.566)%
  --(11.948,5.567)--(11.951,5.569)--(11.954,5.570)--(11.957,5.572)--(11.960,5.573)--(11.963,5.575)%
  --(11.966,5.576)--(11.969,5.578)--(11.972,5.579)--(11.975,5.581)--(11.978,5.582)--(11.981,5.584)%
  --(11.984,5.585)--(11.987,5.587)--(11.990,5.588)--(11.993,5.590)--(11.996,5.591)--(11.999,5.593)%
  --(12.002,5.594)--(12.005,5.596)--(12.008,5.597)--(12.011,5.599)--(12.014,5.600)--(12.017,5.602)%
  --(12.020,5.603)--(12.023,5.605)--(12.026,5.606)--(12.029,5.608)--(12.032,5.609)--(12.035,5.611)%
  --(12.038,5.612)--(12.041,5.614)--(12.044,5.615)--(12.047,5.617)--(12.050,5.618)--(12.053,5.620)%
  --(12.056,5.621)--(12.059,5.623)--(12.062,5.624)--(12.065,5.626)--(12.068,5.627)--(12.071,5.629)%
  --(12.074,5.630)--(12.077,5.632)--(12.080,5.633)--(12.083,5.635)--(12.085,5.636)--(12.088,5.638)%
  --(12.091,5.639)--(12.094,5.641)--(12.097,5.642)--(12.100,5.644)--(12.103,5.646)--(12.106,5.647)%
  --(12.109,5.649)--(12.112,5.650)--(12.115,5.652)--(12.118,5.653)--(12.121,5.655)--(12.124,5.656)%
  --(12.127,5.658)--(12.130,5.659)--(12.133,5.661)--(12.136,5.662)--(12.139,5.664)--(12.142,5.665)%
  --(12.145,5.667)--(12.148,5.668)--(12.151,5.670)--(12.154,5.671)--(12.157,5.673)--(12.160,5.674)%
  --(12.163,5.676)--(12.166,5.677)--(12.169,5.679)--(12.172,5.680)--(12.175,5.682)--(12.178,5.683)%
  --(12.181,5.685)--(12.184,5.686)--(12.187,5.688)--(12.190,5.689)--(12.193,5.691)--(12.196,5.692)%
  --(12.199,5.694)--(12.202,5.695)--(12.205,5.697)--(12.208,5.698)--(12.211,5.700)--(12.214,5.701)%
  --(12.217,5.703)--(12.220,5.704)--(12.223,5.706)--(12.226,5.707)--(12.229,5.709)--(12.232,5.710)%
  --(12.235,5.712)--(12.238,5.713)--(12.241,5.715)--(12.244,5.716)--(12.247,5.718)--(12.250,5.719)%
  --(12.253,5.721)--(12.256,5.722)--(12.259,5.724)--(12.262,5.725)--(12.265,5.727)--(12.268,5.728)%
  --(12.271,5.730)--(12.274,5.731)--(12.277,5.733)--(12.280,5.735)--(12.283,5.736)--(12.286,5.738)%
  --(12.289,5.739)--(12.292,5.741)--(12.295,5.742)--(12.297,5.744)--(12.300,5.745)--(12.303,5.747)%
  --(12.306,5.748)--(12.309,5.750)--(12.312,5.751)--(12.315,5.753)--(12.318,5.754)--(12.321,5.756)%
  --(12.324,5.757)--(12.327,5.759)--(12.330,5.760)--(12.333,5.762)--(12.336,5.763)--(12.339,5.765)%
  --(12.342,5.766)--(12.345,5.768)--(12.348,5.769)--(12.351,5.771)--(12.354,5.772)--(12.357,5.774)%
  --(12.360,5.775)--(12.363,5.777)--(12.366,5.778)--(12.369,5.780)--(12.372,5.781)--(12.375,5.783)%
  --(12.378,5.784)--(12.381,5.786)--(12.384,5.787)--(12.387,5.789)--(12.390,5.790)--(12.393,5.792)%
  --(12.396,5.793)--(12.399,5.795)--(12.402,5.796)--(12.405,5.798)--(12.408,5.799)--(12.411,5.801)%
  --(12.414,5.802)--(12.417,5.804)--(12.420,5.805)--(12.423,5.807)--(12.426,5.808)--(12.429,5.810)%
  --(12.432,5.811)--(12.435,5.813)--(12.438,5.814)--(12.441,5.816)--(12.444,5.818)--(12.447,5.819)%
  --(12.450,5.821)--(12.453,5.822)--(12.456,5.824)--(12.459,5.825)--(12.462,5.827)--(12.465,5.828)%
  --(12.468,5.830)--(12.471,5.831)--(12.474,5.833)--(12.477,5.834)--(12.480,5.836)--(12.483,5.837)%
  --(12.486,5.839)--(12.489,5.840)--(12.492,5.842)--(12.495,5.843)--(12.498,5.845)--(12.501,5.846)%
  --(12.504,5.848)--(12.506,5.849)--(12.509,5.851)--(12.512,5.852)--(12.515,5.854)--(12.518,5.855)%
  --(12.521,5.857)--(12.524,5.858)--(12.527,5.860)--(12.530,5.861)--(12.533,5.863)--(12.536,5.864)%
  --(12.539,5.866)--(12.542,5.867)--(12.545,5.869)--(12.548,5.870)--(12.551,5.872)--(12.554,5.873)%
  --(12.557,5.875)--(12.560,5.876)--(12.563,5.878)--(12.566,5.879)--(12.569,5.881)--(12.572,5.882)%
  --(12.575,5.884)--(12.578,5.885)--(12.581,5.887)--(12.584,5.888)--(12.587,5.890)--(12.590,5.892)%
  --(12.593,5.893)--(12.596,5.895)--(12.599,5.896)--(12.602,5.898)--(12.605,5.899)--(12.608,5.901)%
  --(12.611,5.902)--(12.614,5.904)--(12.617,5.905)--(12.620,5.907)--(12.623,5.908)--(12.626,5.910)%
  --(12.629,5.911)--(12.632,5.913)--(12.635,5.914)--(12.638,5.916)--(12.641,5.917)--(12.644,5.919)%
  --(12.647,5.920)--(12.650,5.922)--(12.653,5.923)--(12.656,5.925)--(12.659,5.926)--(12.662,5.928)%
  --(12.665,5.929)--(12.668,5.931)--(12.671,5.932)--(12.674,5.934)--(12.677,5.935)--(12.680,5.937)%
  --(12.683,5.938)--(12.686,5.940)--(12.689,5.941)--(12.692,5.943)--(12.695,5.944)--(12.698,5.946)%
  --(12.701,5.947)--(12.704,5.949)--(12.707,5.950)--(12.710,5.952)--(12.713,5.953)--(12.715,5.955)%
  --(12.718,5.956)--(12.721,5.958)--(12.724,5.959)--(12.727,5.961)--(12.730,5.963)--(12.733,5.964)%
  --(12.736,5.966)--(12.739,5.967)--(12.742,5.969)--(12.745,5.970)--(12.748,5.972)--(12.751,5.973)%
  --(12.754,5.975)--(12.757,5.976)--(12.760,5.978)--(12.763,5.979)--(12.766,5.981)--(12.769,5.982)%
  --(12.772,5.984)--(12.775,5.985)--(12.778,5.987)--(12.781,5.988)--(12.784,5.990)--(12.787,5.991)%
  --(12.790,5.993)--(12.793,5.994)--(12.796,5.996)--(12.799,5.997)--(12.802,5.999)--(12.805,6.000)%
  --(12.808,6.002)--(12.811,6.003)--(12.814,6.005)--(12.817,6.006)--(12.820,6.008)--(12.823,6.009)%
  --(12.826,6.011)--(12.829,6.012)--(12.832,6.014)--(12.835,6.015)--(12.838,6.017)--(12.841,6.018)%
  --(12.844,6.020)--(12.847,6.021)--(12.850,6.023)--(12.853,6.024)--(12.856,6.026)--(12.859,6.027)%
  --(12.862,6.029)--(12.865,6.031)--(12.868,6.032)--(12.871,6.034)--(12.874,6.035)--(12.877,6.037)%
  --(12.880,6.038)--(12.883,6.040)--(12.886,6.041)--(12.889,6.043)--(12.892,6.044)--(12.895,6.046)%
  --(12.898,6.047)--(12.901,6.049)--(12.904,6.050)--(12.907,6.052)--(12.910,6.053)--(12.913,6.055)%
  --(12.916,6.056)--(12.919,6.058)--(12.922,6.059)--(12.924,6.061)--(12.927,6.062)--(12.930,6.064)%
  --(12.933,6.065)--(12.936,6.067)--(12.939,6.068)--(12.942,6.070)--(12.945,6.071)--(12.948,6.073)%
  --(12.951,6.074)--(12.954,6.076)--(12.957,6.077)--(12.960,6.079)--(12.963,6.080)--(12.966,6.082)%
  --(12.969,6.083)--(12.972,6.085)--(12.975,6.086)--(12.978,6.088)--(12.981,6.089)--(12.984,6.091)%
  --(12.987,6.093)--(12.990,6.094)--(12.993,6.096)--(12.996,6.097)--(12.999,6.099)--(13.002,6.100)%
  --(13.005,6.102)--(13.008,6.103)--(13.011,6.105)--(13.014,6.106)--(13.017,6.108)--(13.020,6.109)%
  --(13.023,6.111)--(13.026,6.112)--(13.029,6.114)--(13.032,6.115)--(13.035,6.117)--(13.038,6.118)%
  --(13.041,6.120)--(13.044,6.121)--(13.047,6.123)--(13.050,6.124)--(13.053,6.126)--(13.056,6.127)%
  --(13.059,6.129)--(13.062,6.130)--(13.065,6.132)--(13.068,6.133)--(13.071,6.135)--(13.074,6.136)%
  --(13.077,6.138)--(13.080,6.139)--(13.083,6.141)--(13.086,6.142)--(13.089,6.144)--(13.092,6.145)%
  --(13.095,6.147)--(13.098,6.148)--(13.101,6.150)--(13.104,6.152)--(13.107,6.153)--(13.110,6.155)%
  --(13.113,6.156)--(13.116,6.158)--(13.119,6.159)--(13.122,6.161)--(13.125,6.162)--(13.128,6.164)%
  --(13.131,6.165)--(13.133,6.167)--(13.136,6.168)--(13.139,6.170)--(13.142,6.171)--(13.145,6.173)%
  --(13.148,6.174)--(13.151,6.176)--(13.154,6.177)--(13.157,6.179)--(13.160,6.180)--(13.163,6.182)%
  --(13.166,6.183)--(13.169,6.185)--(13.172,6.186)--(13.175,6.188)--(13.178,6.189)--(13.181,6.191)%
  --(13.184,6.192)--(13.187,6.194)--(13.190,6.195)--(13.193,6.197)--(13.196,6.198)--(13.199,6.200)%
  --(13.202,6.201)--(13.205,6.203)--(13.208,6.204)--(13.211,6.206)--(13.214,6.207)--(13.217,6.209)%
  --(13.220,6.211)--(13.223,6.212)--(13.226,6.214)--(13.229,6.215)--(13.232,6.217)--(13.235,6.218)%
  --(13.238,6.220)--(13.241,6.221)--(13.244,6.223)--(13.247,6.224)--(13.250,6.226)--(13.253,6.227)%
  --(13.256,6.229)--(13.259,6.230)--(13.262,6.232)--(13.265,6.233)--(13.268,6.235)--(13.271,6.236)%
  --(13.274,6.238)--(13.277,6.239)--(13.280,6.241)--(13.283,6.242)--(13.286,6.244)--(13.289,6.245)%
  --(13.292,6.247)--(13.295,6.248)--(13.298,6.250)--(13.301,6.251)--(13.304,6.253)--(13.307,6.254)%
  --(13.310,6.256)--(13.313,6.257)--(13.316,6.259)--(13.319,6.260)--(13.322,6.262)--(13.325,6.263)%
  --(13.328,6.265)--(13.331,6.267)--(13.334,6.268)--(13.337,6.270)--(13.340,6.271)--(13.342,6.273)%
  --(13.345,6.274)--(13.348,6.276)--(13.351,6.277)--(13.354,6.279)--(13.357,6.280)--(13.360,6.282)%
  --(13.363,6.283)--(13.366,6.285)--(13.369,6.286)--(13.372,6.288)--(13.375,6.289)--(13.378,6.291)%
  --(13.381,6.292)--(13.384,6.294)--(13.387,6.295)--(13.390,6.297)--(13.393,6.298)--(13.396,6.300)%
  --(13.399,6.301)--(13.402,6.303)--(13.405,6.304)--(13.408,6.306)--(13.411,6.307)--(13.414,6.309)%
  --(13.417,6.310)--(13.420,6.312)--(13.423,6.313)--(13.426,6.315)--(13.429,6.316)--(13.432,6.318)%
  --(13.435,6.319)--(13.438,6.321)--(13.441,6.323)--(13.444,6.324);
\gpcolor{color=gp lt color border}
\node[gp node left] at (2.972,7.681) {$\rho \approx \nicefrac{2}{3} \cdot \rho_{\rm{max}}$};
\gpcolor{rgb color={0.337,0.706,0.914}}
\draw[gp path] (1.872,7.681)--(2.788,7.681);
\draw[gp path] (1.507,2.513)--(1.510,2.512)--(1.513,2.511)--(1.516,2.510)--(1.519,2.509)%
  --(1.522,2.508)--(1.525,2.507)--(1.528,2.506)--(1.531,2.505)--(1.534,2.504)--(1.537,2.503)%
  --(1.540,2.502)--(1.543,2.501)--(1.546,2.500)--(1.549,2.499)--(1.552,2.498)--(1.555,2.497)%
  --(1.558,2.496)--(1.561,2.495)--(1.564,2.494)--(1.567,2.493)--(1.570,2.492)--(1.573,2.491)%
  --(1.576,2.490)--(1.579,2.489)--(1.582,2.488)--(1.585,2.487)--(1.588,2.486)--(1.591,2.485)%
  --(1.594,2.484)--(1.597,2.483)--(1.600,2.482)--(1.603,2.481)--(1.606,2.480)--(1.609,2.479)%
  --(1.611,2.478)--(1.614,2.477)--(1.617,2.476)--(1.620,2.475)--(1.623,2.474)--(1.626,2.473)%
  --(1.629,2.472)--(1.632,2.471)--(1.635,2.471)--(1.638,2.470)--(1.641,2.469)--(1.644,2.468)%
  --(1.647,2.467)--(1.650,2.466)--(1.653,2.465)--(1.656,2.464)--(1.659,2.463)--(1.662,2.462)%
  --(1.665,2.461)--(1.668,2.460)--(1.671,2.459)--(1.674,2.458)--(1.677,2.457)--(1.680,2.456)%
  --(1.683,2.455)--(1.686,2.454)--(1.689,2.453)--(1.692,2.452)--(1.695,2.451)--(1.698,2.450)%
  --(1.701,2.449)--(1.704,2.448)--(1.707,2.447)--(1.710,2.446)--(1.713,2.445)--(1.716,2.444)%
  --(1.719,2.443)--(1.722,2.442)--(1.725,2.441)--(1.728,2.440)--(1.731,2.440)--(1.734,2.439)%
  --(1.737,2.438)--(1.740,2.437)--(1.743,2.436)--(1.746,2.435)--(1.749,2.434)--(1.752,2.433)%
  --(1.755,2.432)--(1.758,2.431)--(1.761,2.430)--(1.764,2.429)--(1.767,2.428)--(1.770,2.427)%
  --(1.773,2.426)--(1.776,2.425)--(1.779,2.424)--(1.782,2.424)--(1.785,2.423)--(1.788,2.422)%
  --(1.791,2.421)--(1.794,2.420)--(1.797,2.419)--(1.800,2.418)--(1.803,2.417)--(1.806,2.416)%
  --(1.809,2.415)--(1.812,2.414)--(1.815,2.413)--(1.818,2.412)--(1.820,2.412)--(1.823,2.411)%
  --(1.826,2.410)--(1.829,2.409)--(1.832,2.408)--(1.835,2.407)--(1.838,2.406)--(1.841,2.405)%
  --(1.844,2.404)--(1.847,2.403)--(1.850,2.403)--(1.853,2.402)--(1.856,2.401)--(1.859,2.400)%
  --(1.862,2.399)--(1.865,2.398)--(1.868,2.397)--(1.871,2.396)--(1.874,2.395)--(1.877,2.395)%
  --(1.880,2.394)--(1.883,2.393)--(1.886,2.392)--(1.889,2.391)--(1.892,2.390)--(1.895,2.389)%
  --(1.898,2.388)--(1.901,2.388)--(1.904,2.387)--(1.907,2.386)--(1.910,2.385)--(1.913,2.384)%
  --(1.916,2.383)--(1.919,2.382)--(1.922,2.382)--(1.925,2.381)--(1.928,2.380)--(1.931,2.379)%
  --(1.934,2.378)--(1.937,2.377)--(1.940,2.377)--(1.943,2.376)--(1.946,2.375)--(1.949,2.374)%
  --(1.952,2.373)--(1.955,2.372)--(1.958,2.372)--(1.961,2.371)--(1.964,2.370)--(1.967,2.369)%
  --(1.970,2.368)--(1.973,2.367)--(1.976,2.367)--(1.979,2.366)--(1.982,2.365)--(1.985,2.364)%
  --(1.988,2.363)--(1.991,2.363)--(1.994,2.362)--(1.997,2.361)--(2.000,2.360)--(2.003,2.359)%
  --(2.006,2.359)--(2.009,2.358)--(2.012,2.357)--(2.015,2.356)--(2.018,2.356)--(2.021,2.355)%
  --(2.024,2.354)--(2.027,2.353)--(2.029,2.352)--(2.032,2.352)--(2.035,2.351)--(2.038,2.350)%
  --(2.041,2.349)--(2.044,2.349)--(2.047,2.348)--(2.050,2.347)--(2.053,2.346)--(2.056,2.346)%
  --(2.059,2.345)--(2.062,2.344)--(2.065,2.343)--(2.068,2.343)--(2.071,2.342)--(2.074,2.341)%
  --(2.077,2.340)--(2.080,2.340)--(2.083,2.339)--(2.086,2.338)--(2.089,2.338)--(2.092,2.337)%
  --(2.095,2.336)--(2.098,2.335)--(2.101,2.335)--(2.104,2.334)--(2.107,2.333)--(2.110,2.333)%
  --(2.113,2.332)--(2.116,2.331)--(2.119,2.331)--(2.122,2.330)--(2.125,2.329)--(2.128,2.328)%
  --(2.131,2.328)--(2.134,2.327)--(2.137,2.326)--(2.140,2.326)--(2.143,2.325)--(2.146,2.324)%
  --(2.149,2.324)--(2.152,2.323)--(2.155,2.322)--(2.158,2.322)--(2.161,2.321)--(2.164,2.320)%
  --(2.167,2.320)--(2.170,2.319)--(2.173,2.319)--(2.176,2.318)--(2.179,2.317)--(2.182,2.317)%
  --(2.185,2.316)--(2.188,2.315)--(2.191,2.315)--(2.194,2.314)--(2.197,2.313)--(2.200,2.313)%
  --(2.203,2.312)--(2.206,2.312)--(2.209,2.311)--(2.212,2.310)--(2.215,2.310)--(2.218,2.309)%
  --(2.221,2.309)--(2.224,2.308)--(2.227,2.307)--(2.230,2.307)--(2.233,2.306)--(2.236,2.306)%
  --(2.238,2.305)--(2.241,2.304)--(2.244,2.304)--(2.247,2.303)--(2.250,2.303)--(2.253,2.302)%
  --(2.256,2.302)--(2.259,2.301)--(2.262,2.301)--(2.265,2.300)--(2.268,2.299)--(2.271,2.299)%
  --(2.274,2.298)--(2.277,2.298)--(2.280,2.297)--(2.283,2.297)--(2.286,2.296)--(2.289,2.296)%
  --(2.292,2.295)--(2.295,2.295)--(2.298,2.294)--(2.301,2.294)--(2.304,2.293)--(2.307,2.293)%
  --(2.310,2.292)--(2.313,2.291)--(2.316,2.291)--(2.319,2.290)--(2.322,2.290)--(2.325,2.289)%
  --(2.328,2.289)--(2.331,2.289)--(2.334,2.288)--(2.337,2.288)--(2.340,2.287)--(2.343,2.287)%
  --(2.346,2.286)--(2.349,2.286)--(2.352,2.285)--(2.355,2.285)--(2.358,2.284)--(2.361,2.284)%
  --(2.364,2.283)--(2.367,2.283)--(2.370,2.282)--(2.373,2.282)--(2.376,2.282)--(2.379,2.281)%
  --(2.382,2.281)--(2.385,2.280)--(2.388,2.280)--(2.391,2.279)--(2.394,2.279)--(2.397,2.279)%
  --(2.400,2.278)--(2.403,2.278)--(2.406,2.277)--(2.409,2.277)--(2.412,2.276)--(2.415,2.276)%
  --(2.418,2.276)--(2.421,2.275)--(2.424,2.275)--(2.427,2.274)--(2.430,2.274)--(2.433,2.274)%
  --(2.436,2.273)--(2.439,2.273)--(2.442,2.273)--(2.445,2.272)--(2.447,2.272)--(2.450,2.271)%
  --(2.453,2.271)--(2.456,2.271)--(2.459,2.270)--(2.462,2.270)--(2.465,2.270)--(2.468,2.269)%
  --(2.471,2.269)--(2.474,2.269)--(2.477,2.268)--(2.480,2.268)--(2.483,2.268)--(2.486,2.267)%
  --(2.489,2.267)--(2.492,2.267)--(2.495,2.266)--(2.498,2.266)--(2.501,2.266)--(2.504,2.265)%
  --(2.507,2.265)--(2.510,2.265)--(2.513,2.265)--(2.516,2.264)--(2.519,2.264)--(2.522,2.264)%
  --(2.525,2.263)--(2.528,2.263)--(2.531,2.263)--(2.534,2.263)--(2.537,2.262)--(2.540,2.262)%
  --(2.543,2.262)--(2.546,2.261)--(2.549,2.261)--(2.552,2.261)--(2.555,2.261)--(2.558,2.260)%
  --(2.561,2.260)--(2.564,2.260)--(2.567,2.260)--(2.570,2.259)--(2.573,2.259)--(2.576,2.259)%
  --(2.579,2.259)--(2.582,2.258)--(2.585,2.258)--(2.588,2.258)--(2.591,2.258)--(2.594,2.258)%
  --(2.597,2.257)--(2.600,2.257)--(2.603,2.257)--(2.606,2.257)--(2.609,2.257)--(2.612,2.256)%
  --(2.615,2.256)--(2.618,2.256)--(2.621,2.256)--(2.624,2.256)--(2.627,2.255)--(2.630,2.255)%
  --(2.633,2.255)--(2.636,2.255)--(2.639,2.255)--(2.642,2.255)--(2.645,2.254)--(2.648,2.254)%
  --(2.651,2.254)--(2.654,2.254)--(2.656,2.254)--(2.659,2.254)--(2.662,2.254)--(2.665,2.253)%
  --(2.668,2.253)--(2.671,2.253)--(2.674,2.253)--(2.677,2.253)--(2.680,2.253)--(2.683,2.253)%
  --(2.686,2.252)--(2.689,2.252)--(2.692,2.252)--(2.695,2.252)--(2.698,2.252)--(2.701,2.252)%
  --(2.704,2.252)--(2.707,2.252)--(2.710,2.252)--(2.713,2.252)--(2.716,2.251)--(2.719,2.251)%
  --(2.722,2.251)--(2.725,2.251)--(2.728,2.251)--(2.731,2.251)--(2.734,2.251)--(2.737,2.251)%
  --(2.740,2.251)--(2.743,2.251)--(2.746,2.251)--(2.749,2.251)--(2.752,2.251)--(2.755,2.251)%
  --(2.758,2.251)--(2.761,2.251)--(2.764,2.250)--(2.767,2.250)--(2.770,2.250)--(2.773,2.250)%
  --(2.776,2.250)--(2.779,2.250)--(2.782,2.250)--(2.785,2.250)--(2.788,2.250)--(2.791,2.250)%
  --(2.794,2.250)--(2.797,2.250)--(2.800,2.250)--(2.803,2.250)--(2.806,2.250)--(2.809,2.250)%
  --(2.812,2.250)--(2.815,2.250)--(2.818,2.250)--(2.821,2.250)--(2.824,2.250)--(2.827,2.250)%
  --(2.830,2.250)--(2.833,2.250)--(2.836,2.250)--(2.839,2.250)--(2.842,2.251)--(2.845,2.251)%
  --(2.848,2.251)--(2.851,2.251)--(2.854,2.251)--(2.857,2.251)--(2.860,2.251)--(2.863,2.251)%
  --(2.866,2.251)--(2.868,2.251)--(2.871,2.251)--(2.874,2.251)--(2.877,2.251)--(2.880,2.251)%
  --(2.883,2.251)--(2.886,2.251)--(2.889,2.252)--(2.892,2.252)--(2.895,2.252)--(2.898,2.252)%
  --(2.901,2.252)--(2.904,2.252)--(2.907,2.252)--(2.910,2.252)--(2.913,2.252)--(2.916,2.252)%
  --(2.919,2.253)--(2.922,2.253)--(2.925,2.253)--(2.928,2.253)--(2.931,2.253)--(2.934,2.253)%
  --(2.937,2.253)--(2.940,2.253)--(2.943,2.254)--(2.946,2.254)--(2.949,2.254)--(2.952,2.254)%
  --(2.955,2.254)--(2.958,2.254)--(2.961,2.254)--(2.964,2.255)--(2.967,2.255)--(2.970,2.255)%
  --(2.973,2.255)--(2.976,2.255)--(2.979,2.255)--(2.982,2.256)--(2.985,2.256)--(2.988,2.256)%
  --(2.991,2.256)--(2.994,2.256)--(2.997,2.256)--(3.000,2.257)--(3.003,2.257)--(3.006,2.257)%
  --(3.009,2.257)--(3.012,2.257)--(3.015,2.258)--(3.018,2.258)--(3.021,2.258)--(3.024,2.258)%
  --(3.027,2.259)--(3.030,2.259)--(3.033,2.259)--(3.036,2.259)--(3.039,2.259)--(3.042,2.260)%
  --(3.045,2.260)--(3.048,2.260)--(3.051,2.260)--(3.054,2.261)--(3.057,2.261)--(3.060,2.261)%
  --(3.063,2.261)--(3.066,2.262)--(3.069,2.262)--(3.072,2.262)--(3.075,2.262)--(3.077,2.263)%
  --(3.080,2.263)--(3.083,2.263)--(3.086,2.263)--(3.089,2.264)--(3.092,2.264)--(3.095,2.264)%
  --(3.098,2.264)--(3.101,2.265)--(3.104,2.265)--(3.107,2.265)--(3.110,2.266)--(3.113,2.266)%
  --(3.116,2.266)--(3.119,2.266)--(3.122,2.267)--(3.125,2.267)--(3.128,2.267)--(3.131,2.268)%
  --(3.134,2.268)--(3.137,2.268)--(3.140,2.269)--(3.143,2.269)--(3.146,2.269)--(3.149,2.270)%
  --(3.152,2.270)--(3.155,2.270)--(3.158,2.271)--(3.161,2.271)--(3.164,2.271)--(3.167,2.272)%
  --(3.170,2.272)--(3.173,2.272)--(3.176,2.273)--(3.179,2.273)--(3.182,2.273)--(3.185,2.274)%
  --(3.188,2.274)--(3.191,2.274)--(3.194,2.275)--(3.197,2.275)--(3.200,2.275)--(3.203,2.276)%
  --(3.206,2.276)--(3.209,2.277)--(3.212,2.277)--(3.215,2.277)--(3.218,2.278)--(3.221,2.278)%
  --(3.224,2.278)--(3.227,2.279)--(3.230,2.279)--(3.233,2.280)--(3.236,2.280)--(3.239,2.280)%
  --(3.242,2.281)--(3.245,2.281)--(3.248,2.282)--(3.251,2.282)--(3.254,2.282)--(3.257,2.283)%
  --(3.260,2.283)--(3.263,2.284)--(3.266,2.284)--(3.269,2.285)--(3.272,2.285)--(3.275,2.285)%
  --(3.278,2.286)--(3.281,2.286)--(3.284,2.287)--(3.286,2.287)--(3.289,2.288)--(3.292,2.288)%
  --(3.295,2.288)--(3.298,2.289)--(3.301,2.289)--(3.304,2.290)--(3.307,2.290)--(3.310,2.291)%
  --(3.313,2.291)--(3.316,2.292)--(3.319,2.292)--(3.322,2.292)--(3.325,2.293)--(3.328,2.293)%
  --(3.331,2.294)--(3.334,2.294)--(3.337,2.295)--(3.340,2.295)--(3.343,2.296)--(3.346,2.296)%
  --(3.349,2.297)--(3.352,2.297)--(3.355,2.298)--(3.358,2.298)--(3.361,2.299)--(3.364,2.299)%
  --(3.367,2.300)--(3.370,2.300)--(3.373,2.301)--(3.376,2.301)--(3.379,2.302)--(3.382,2.302)%
  --(3.385,2.303)--(3.388,2.303)--(3.391,2.304)--(3.394,2.304)--(3.397,2.305)--(3.400,2.305)%
  --(3.403,2.306)--(3.406,2.306)--(3.409,2.307)--(3.412,2.307)--(3.415,2.308)--(3.418,2.308)%
  --(3.421,2.309)--(3.424,2.310)--(3.427,2.310)--(3.430,2.311)--(3.433,2.311)--(3.436,2.312)%
  --(3.439,2.312)--(3.442,2.313)--(3.445,2.313)--(3.448,2.314)--(3.451,2.314)--(3.454,2.315)%
  --(3.457,2.316)--(3.460,2.316)--(3.463,2.317)--(3.466,2.317)--(3.469,2.318)--(3.472,2.318)%
  --(3.475,2.319)--(3.478,2.320)--(3.481,2.320)--(3.484,2.321)--(3.487,2.321)--(3.490,2.322)%
  --(3.493,2.322)--(3.495,2.323)--(3.498,2.324)--(3.501,2.324)--(3.504,2.325)--(3.507,2.325)%
  --(3.510,2.326)--(3.513,2.327)--(3.516,2.327)--(3.519,2.328)--(3.522,2.328)--(3.525,2.329)%
  --(3.528,2.330)--(3.531,2.330)--(3.534,2.331)--(3.537,2.332)--(3.540,2.332)--(3.543,2.333)%
  --(3.546,2.333)--(3.549,2.334)--(3.552,2.335)--(3.555,2.335)--(3.558,2.336)--(3.561,2.337)%
  --(3.564,2.337)--(3.567,2.338)--(3.570,2.338)--(3.573,2.339)--(3.576,2.340)--(3.579,2.340)%
  --(3.582,2.341)--(3.585,2.342)--(3.588,2.342)--(3.591,2.343)--(3.594,2.344)--(3.597,2.344)%
  --(3.600,2.345)--(3.603,2.346)--(3.606,2.346)--(3.609,2.347)--(3.612,2.348)--(3.615,2.348)%
  --(3.618,2.349)--(3.621,2.350)--(3.624,2.350)--(3.627,2.351)--(3.630,2.352)--(3.633,2.352)%
  --(3.636,2.353)--(3.639,2.354)--(3.642,2.354)--(3.645,2.355)--(3.648,2.356)--(3.651,2.356)%
  --(3.654,2.357)--(3.657,2.358)--(3.660,2.358)--(3.663,2.359)--(3.666,2.360)--(3.669,2.360)%
  --(3.672,2.361)--(3.675,2.362)--(3.678,2.363)--(3.681,2.363)--(3.684,2.364)--(3.687,2.365)%
  --(3.690,2.365)--(3.693,2.366)--(3.696,2.367)--(3.699,2.368)--(3.702,2.368)--(3.704,2.369)%
  --(3.707,2.370)--(3.710,2.370)--(3.713,2.371)--(3.716,2.372)--(3.719,2.373)--(3.722,2.373)%
  --(3.725,2.374)--(3.728,2.375)--(3.731,2.376)--(3.734,2.376)--(3.737,2.377)--(3.740,2.378)%
  --(3.743,2.378)--(3.746,2.379)--(3.749,2.380)--(3.752,2.381)--(3.755,2.381)--(3.758,2.382)%
  --(3.761,2.383)--(3.764,2.384)--(3.767,2.384)--(3.770,2.385)--(3.773,2.386)--(3.776,2.387)%
  --(3.779,2.388)--(3.782,2.388)--(3.785,2.389)--(3.788,2.390)--(3.791,2.391)--(3.794,2.391)%
  --(3.797,2.392)--(3.800,2.393)--(3.803,2.394)--(3.806,2.394)--(3.809,2.395)--(3.812,2.396)%
  --(3.815,2.397)--(3.818,2.398)--(3.821,2.398)--(3.824,2.399)--(3.827,2.400)--(3.830,2.401)%
  --(3.833,2.402)--(3.836,2.402)--(3.839,2.403)--(3.842,2.404)--(3.845,2.405)--(3.848,2.405)%
  --(3.851,2.406)--(3.854,2.407)--(3.857,2.408)--(3.860,2.409)--(3.863,2.410)--(3.866,2.410)%
  --(3.869,2.411)--(3.872,2.412)--(3.875,2.413)--(3.878,2.414)--(3.881,2.414)--(3.884,2.415)%
  --(3.887,2.416)--(3.890,2.417)--(3.893,2.418)--(3.896,2.418)--(3.899,2.419)--(3.902,2.420)%
  --(3.905,2.421)--(3.908,2.422)--(3.911,2.423)--(3.914,2.423)--(3.916,2.424)--(3.919,2.425)%
  --(3.922,2.426)--(3.925,2.427)--(3.928,2.428)--(3.931,2.428)--(3.934,2.429)--(3.937,2.430)%
  --(3.940,2.431)--(3.943,2.432)--(3.946,2.433)--(3.949,2.434)--(3.952,2.434)--(3.955,2.435)%
  --(3.958,2.436)--(3.961,2.437)--(3.964,2.438)--(3.967,2.439)--(3.970,2.440)--(3.973,2.440)%
  --(3.976,2.441)--(3.979,2.442)--(3.982,2.443)--(3.985,2.444)--(3.988,2.445)--(3.991,2.446)%
  --(3.994,2.446)--(3.997,2.447)--(4.000,2.448)--(4.003,2.449)--(4.006,2.450)--(4.009,2.451)%
  --(4.012,2.452)--(4.015,2.453)--(4.018,2.453)--(4.021,2.454)--(4.024,2.455)--(4.027,2.456)%
  --(4.030,2.457)--(4.033,2.458)--(4.036,2.459)--(4.039,2.460)--(4.042,2.461)--(4.045,2.461)%
  --(4.048,2.462)--(4.051,2.463)--(4.054,2.464)--(4.057,2.465)--(4.060,2.466)--(4.063,2.467)%
  --(4.066,2.468)--(4.069,2.469)--(4.072,2.470)--(4.075,2.470)--(4.078,2.471)--(4.081,2.472)%
  --(4.084,2.473)--(4.087,2.474)--(4.090,2.475)--(4.093,2.476)--(4.096,2.477)--(4.099,2.478)%
  --(4.102,2.479)--(4.105,2.480)--(4.108,2.481)--(4.111,2.481)--(4.114,2.482)--(4.117,2.483)%
  --(4.120,2.484)--(4.123,2.485)--(4.125,2.486)--(4.128,2.487)--(4.131,2.488)--(4.134,2.489)%
  --(4.137,2.490)--(4.140,2.491)--(4.143,2.492)--(4.146,2.493)--(4.149,2.494)--(4.152,2.495)%
  --(4.155,2.495)--(4.158,2.496)--(4.161,2.497)--(4.164,2.498)--(4.167,2.499)--(4.170,2.500)%
  --(4.173,2.501)--(4.176,2.502)--(4.179,2.503)--(4.182,2.504)--(4.185,2.505)--(4.188,2.506)%
  --(4.191,2.507)--(4.194,2.508)--(4.197,2.509)--(4.200,2.510)--(4.203,2.511)--(4.206,2.512)%
  --(4.209,2.513)--(4.212,2.514)--(4.215,2.515)--(4.218,2.516)--(4.221,2.516)--(4.224,2.517)%
  --(4.227,2.518)--(4.230,2.519)--(4.233,2.520)--(4.236,2.521)--(4.239,2.522)--(4.242,2.523)%
  --(4.245,2.524)--(4.248,2.525)--(4.251,2.526)--(4.254,2.527)--(4.257,2.528)--(4.260,2.529)%
  --(4.263,2.530)--(4.266,2.531)--(4.269,2.532)--(4.272,2.533)--(4.275,2.534)--(4.278,2.535)%
  --(4.281,2.536)--(4.284,2.537)--(4.287,2.538)--(4.290,2.539)--(4.293,2.540)--(4.296,2.541)%
  --(4.299,2.542)--(4.302,2.543)--(4.305,2.544)--(4.308,2.545)--(4.311,2.546)--(4.314,2.547)%
  --(4.317,2.548)--(4.320,2.549)--(4.323,2.550)--(4.326,2.551)--(4.329,2.552)--(4.332,2.553)%
  --(4.334,2.554)--(4.337,2.555)--(4.340,2.556)--(4.343,2.557)--(4.346,2.558)--(4.349,2.559)%
  --(4.352,2.560)--(4.355,2.561)--(4.358,2.562)--(4.361,2.563)--(4.364,2.564)--(4.367,2.565)%
  --(4.370,2.566)--(4.373,2.567)--(4.376,2.568)--(4.379,2.569)--(4.382,2.570)--(4.385,2.571)%
  --(4.388,2.572)--(4.391,2.573)--(4.394,2.575)--(4.397,2.576)--(4.400,2.577)--(4.403,2.578)%
  --(4.406,2.579)--(4.409,2.580)--(4.412,2.581)--(4.415,2.582)--(4.418,2.583)--(4.421,2.584)%
  --(4.424,2.585)--(4.427,2.586)--(4.430,2.587)--(4.433,2.588)--(4.436,2.589)--(4.439,2.590)%
  --(4.442,2.591)--(4.445,2.592)--(4.448,2.593)--(4.451,2.594)--(4.454,2.595)--(4.457,2.596)%
  --(4.460,2.597)--(4.463,2.599)--(4.466,2.600)--(4.469,2.601)--(4.472,2.602)--(4.475,2.603)%
  --(4.478,2.604)--(4.481,2.605)--(4.484,2.606)--(4.487,2.607)--(4.490,2.608)--(4.493,2.609)%
  --(4.496,2.610)--(4.499,2.611)--(4.502,2.612)--(4.505,2.613)--(4.508,2.614)--(4.511,2.616)%
  --(4.514,2.617)--(4.517,2.618)--(4.520,2.619)--(4.523,2.620)--(4.526,2.621)--(4.529,2.622)%
  --(4.532,2.623)--(4.535,2.624)--(4.538,2.625)--(4.541,2.626)--(4.543,2.627)--(4.546,2.628)%
  --(4.549,2.629)--(4.552,2.631)--(4.555,2.632)--(4.558,2.633)--(4.561,2.634)--(4.564,2.635)%
  --(4.567,2.636)--(4.570,2.637)--(4.573,2.638)--(4.576,2.639)--(4.579,2.640)--(4.582,2.641)%
  --(4.585,2.643)--(4.588,2.644)--(4.591,2.645)--(4.594,2.646)--(4.597,2.647)--(4.600,2.648)%
  --(4.603,2.649)--(4.606,2.650)--(4.609,2.651)--(4.612,2.652)--(4.615,2.653)--(4.618,2.655)%
  --(4.621,2.656)--(4.624,2.657)--(4.627,2.658)--(4.630,2.659)--(4.633,2.660)--(4.636,2.661)%
  --(4.639,2.662)--(4.642,2.663)--(4.645,2.665)--(4.648,2.666)--(4.651,2.667)--(4.654,2.668)%
  --(4.657,2.669)--(4.660,2.670)--(4.663,2.671)--(4.666,2.672)--(4.669,2.673)--(4.672,2.675)%
  --(4.675,2.676)--(4.678,2.677)--(4.681,2.678)--(4.684,2.679)--(4.687,2.680)--(4.690,2.681)%
  --(4.693,2.682)--(4.696,2.683)--(4.699,2.685)--(4.702,2.686)--(4.705,2.687)--(4.708,2.688)%
  --(4.711,2.689)--(4.714,2.690)--(4.717,2.691)--(4.720,2.693)--(4.723,2.694)--(4.726,2.695)%
  --(4.729,2.696)--(4.732,2.697)--(4.735,2.698)--(4.738,2.699)--(4.741,2.700)--(4.744,2.702)%
  --(4.747,2.703)--(4.750,2.704)--(4.752,2.705)--(4.755,2.706)--(4.758,2.707)--(4.761,2.708)%
  --(4.764,2.710)--(4.767,2.711)--(4.770,2.712)--(4.773,2.713)--(4.776,2.714)--(4.779,2.715)%
  --(4.782,2.716)--(4.785,2.718)--(4.788,2.719)--(4.791,2.720)--(4.794,2.721)--(4.797,2.722)%
  --(4.800,2.723)--(4.803,2.724)--(4.806,2.726)--(4.809,2.727)--(4.812,2.728)--(4.815,2.729)%
  --(4.818,2.730)--(4.821,2.731)--(4.824,2.732)--(4.827,2.734)--(4.830,2.735)--(4.833,2.736)%
  --(4.836,2.737)--(4.839,2.738)--(4.842,2.739)--(4.845,2.741)--(4.848,2.742)--(4.851,2.743)%
  --(4.854,2.744)--(4.857,2.745)--(4.860,2.746)--(4.863,2.748)--(4.866,2.749)--(4.869,2.750)%
  --(4.872,2.751)--(4.875,2.752)--(4.878,2.753)--(4.881,2.755)--(4.884,2.756)--(4.887,2.757)%
  --(4.890,2.758)--(4.893,2.759)--(4.896,2.760)--(4.899,2.762)--(4.902,2.763)--(4.905,2.764)%
  --(4.908,2.765)--(4.911,2.766)--(4.914,2.767)--(4.917,2.769)--(4.920,2.770)--(4.923,2.771)%
  --(4.926,2.772)--(4.929,2.773)--(4.932,2.775)--(4.935,2.776)--(4.938,2.777)--(4.941,2.778)%
  --(4.944,2.779)--(4.947,2.780)--(4.950,2.782)--(4.953,2.783)--(4.956,2.784)--(4.959,2.785)%
  --(4.961,2.786)--(4.964,2.788)--(4.967,2.789)--(4.970,2.790)--(4.973,2.791)--(4.976,2.792)%
  --(4.979,2.793)--(4.982,2.795)--(4.985,2.796)--(4.988,2.797)--(4.991,2.798)--(4.994,2.799)%
  --(4.997,2.801)--(5.000,2.802)--(5.003,2.803)--(5.006,2.804)--(5.009,2.805)--(5.012,2.807)%
  --(5.015,2.808)--(5.018,2.809)--(5.021,2.810)--(5.024,2.811)--(5.027,2.813)--(5.030,2.814)%
  --(5.033,2.815)--(5.036,2.816)--(5.039,2.817)--(5.042,2.819)--(5.045,2.820)--(5.048,2.821)%
  --(5.051,2.822)--(5.054,2.823)--(5.057,2.825)--(5.060,2.826)--(5.063,2.827)--(5.066,2.828)%
  --(5.069,2.830)--(5.072,2.831)--(5.075,2.832)--(5.078,2.833)--(5.081,2.834)--(5.084,2.836)%
  --(5.087,2.837)--(5.090,2.838)--(5.093,2.839)--(5.096,2.840)--(5.099,2.842)--(5.102,2.843)%
  --(5.105,2.844)--(5.108,2.845)--(5.111,2.846)--(5.114,2.848)--(5.117,2.849)--(5.120,2.850)%
  --(5.123,2.851)--(5.126,2.853)--(5.129,2.854)--(5.132,2.855)--(5.135,2.856)--(5.138,2.857)%
  --(5.141,2.859)--(5.144,2.860)--(5.147,2.861)--(5.150,2.862)--(5.153,2.864)--(5.156,2.865)%
  --(5.159,2.866)--(5.162,2.867)--(5.165,2.868)--(5.168,2.870)--(5.171,2.871)--(5.173,2.872)%
  --(5.176,2.873)--(5.179,2.875)--(5.182,2.876)--(5.185,2.877)--(5.188,2.878)--(5.191,2.880)%
  --(5.194,2.881)--(5.197,2.882)--(5.200,2.883)--(5.203,2.885)--(5.206,2.886)--(5.209,2.887)%
  --(5.212,2.888)--(5.215,2.889)--(5.218,2.891)--(5.221,2.892)--(5.224,2.893)--(5.227,2.894)%
  --(5.230,2.896)--(5.233,2.897)--(5.236,2.898)--(5.239,2.899)--(5.242,2.901)--(5.245,2.902)%
  --(5.248,2.903)--(5.251,2.904)--(5.254,2.906)--(5.257,2.907)--(5.260,2.908)--(5.263,2.909)%
  --(5.266,2.911)--(5.269,2.912)--(5.272,2.913)--(5.275,2.914)--(5.278,2.916)--(5.281,2.917)%
  --(5.284,2.918)--(5.287,2.919)--(5.290,2.921)--(5.293,2.922)--(5.296,2.923)--(5.299,2.924)%
  --(5.302,2.926)--(5.305,2.927)--(5.308,2.928)--(5.311,2.929)--(5.314,2.931)--(5.317,2.932)%
  --(5.320,2.933)--(5.323,2.934)--(5.326,2.936)--(5.329,2.937)--(5.332,2.938)--(5.335,2.939)%
  --(5.338,2.941)--(5.341,2.942)--(5.344,2.943)--(5.347,2.944)--(5.350,2.946)--(5.353,2.947)%
  --(5.356,2.948)--(5.359,2.949)--(5.362,2.951)--(5.365,2.952)--(5.368,2.953)--(5.371,2.954)%
  --(5.374,2.956)--(5.377,2.957)--(5.380,2.958)--(5.382,2.960)--(5.385,2.961)--(5.388,2.962)%
  --(5.391,2.963)--(5.394,2.965)--(5.397,2.966)--(5.400,2.967)--(5.403,2.968)--(5.406,2.970)%
  --(5.409,2.971)--(5.412,2.972)--(5.415,2.973)--(5.418,2.975)--(5.421,2.976)--(5.424,2.977)%
  --(5.427,2.979)--(5.430,2.980)--(5.433,2.981)--(5.436,2.982)--(5.439,2.984)--(5.442,2.985)%
  --(5.445,2.986)--(5.448,2.987)--(5.451,2.989)--(5.454,2.990)--(5.457,2.991)--(5.460,2.993)%
  --(5.463,2.994)--(5.466,2.995)--(5.469,2.996)--(5.472,2.998)--(5.475,2.999)--(5.478,3.000)%
  --(5.481,3.002)--(5.484,3.003)--(5.487,3.004)--(5.490,3.005)--(5.493,3.007)--(5.496,3.008)%
  --(5.499,3.009)--(5.502,3.011)--(5.505,3.012)--(5.508,3.013)--(5.511,3.014)--(5.514,3.016)%
  --(5.517,3.017)--(5.520,3.018)--(5.523,3.019)--(5.526,3.021)--(5.529,3.022)--(5.532,3.023)%
  --(5.535,3.025)--(5.538,3.026)--(5.541,3.027)--(5.544,3.029)--(5.547,3.030)--(5.550,3.031)%
  --(5.553,3.032)--(5.556,3.034)--(5.559,3.035)--(5.562,3.036)--(5.565,3.038)--(5.568,3.039)%
  --(5.571,3.040)--(5.574,3.041)--(5.577,3.043)--(5.580,3.044)--(5.583,3.045)--(5.586,3.047)%
  --(5.589,3.048)--(5.591,3.049)--(5.594,3.050)--(5.597,3.052)--(5.600,3.053)--(5.603,3.054)%
  --(5.606,3.056)--(5.609,3.057)--(5.612,3.058)--(5.615,3.060)--(5.618,3.061)--(5.621,3.062)%
  --(5.624,3.063)--(5.627,3.065)--(5.630,3.066)--(5.633,3.067)--(5.636,3.069)--(5.639,3.070)%
  --(5.642,3.071)--(5.645,3.073)--(5.648,3.074)--(5.651,3.075)--(5.654,3.076)--(5.657,3.078)%
  --(5.660,3.079)--(5.663,3.080)--(5.666,3.082)--(5.669,3.083)--(5.672,3.084)--(5.675,3.086)%
  --(5.678,3.087)--(5.681,3.088)--(5.684,3.090)--(5.687,3.091)--(5.690,3.092)--(5.693,3.093)%
  --(5.696,3.095)--(5.699,3.096)--(5.702,3.097)--(5.705,3.099)--(5.708,3.100)--(5.711,3.101)%
  --(5.714,3.103)--(5.717,3.104)--(5.720,3.105)--(5.723,3.107)--(5.726,3.108)--(5.729,3.109)%
  --(5.732,3.111)--(5.735,3.112)--(5.738,3.113)--(5.741,3.114)--(5.744,3.116)--(5.747,3.117)%
  --(5.750,3.118)--(5.753,3.120)--(5.756,3.121)--(5.759,3.122)--(5.762,3.124)--(5.765,3.125)%
  --(5.768,3.126)--(5.771,3.128)--(5.774,3.129)--(5.777,3.130)--(5.780,3.132)--(5.783,3.133)%
  --(5.786,3.134)--(5.789,3.136)--(5.792,3.137)--(5.795,3.138)--(5.798,3.140)--(5.800,3.141)%
  --(5.803,3.142)--(5.806,3.143)--(5.809,3.145)--(5.812,3.146)--(5.815,3.147)--(5.818,3.149)%
  --(5.821,3.150)--(5.824,3.151)--(5.827,3.153)--(5.830,3.154)--(5.833,3.155)--(5.836,3.157)%
  --(5.839,3.158)--(5.842,3.159)--(5.845,3.161)--(5.848,3.162)--(5.851,3.163)--(5.854,3.165)%
  --(5.857,3.166)--(5.860,3.167)--(5.863,3.169)--(5.866,3.170)--(5.869,3.171)--(5.872,3.173)%
  --(5.875,3.174)--(5.878,3.175)--(5.881,3.177)--(5.884,3.178)--(5.887,3.179)--(5.890,3.181)%
  --(5.893,3.182)--(5.896,3.183)--(5.899,3.185)--(5.902,3.186)--(5.905,3.187)--(5.908,3.189)%
  --(5.911,3.190)--(5.914,3.191)--(5.917,3.193)--(5.920,3.194)--(5.923,3.195)--(5.926,3.197)%
  --(5.929,3.198)--(5.932,3.199)--(5.935,3.201)--(5.938,3.202)--(5.941,3.203)--(5.944,3.205)%
  --(5.947,3.206)--(5.950,3.207)--(5.953,3.209)--(5.956,3.210)--(5.959,3.211)--(5.962,3.213)%
  --(5.965,3.214)--(5.968,3.215)--(5.971,3.217)--(5.974,3.218)--(5.977,3.219)--(5.980,3.221)%
  --(5.983,3.222)--(5.986,3.223)--(5.989,3.225)--(5.992,3.226)--(5.995,3.227)--(5.998,3.229)%
  --(6.001,3.230)--(6.004,3.232)--(6.007,3.233)--(6.009,3.234)--(6.012,3.236)--(6.015,3.237)%
  --(6.018,3.238)--(6.021,3.240)--(6.024,3.241)--(6.027,3.242)--(6.030,3.244)--(6.033,3.245)%
  --(6.036,3.246)--(6.039,3.248)--(6.042,3.249)--(6.045,3.250)--(6.048,3.252)--(6.051,3.253)%
  --(6.054,3.254)--(6.057,3.256)--(6.060,3.257)--(6.063,3.258)--(6.066,3.260)--(6.069,3.261)%
  --(6.072,3.263)--(6.075,3.264)--(6.078,3.265)--(6.081,3.267)--(6.084,3.268)--(6.087,3.269)%
  --(6.090,3.271)--(6.093,3.272)--(6.096,3.273)--(6.099,3.275)--(6.102,3.276)--(6.105,3.277)%
  --(6.108,3.279)--(6.111,3.280)--(6.114,3.281)--(6.117,3.283)--(6.120,3.284)--(6.123,3.286)%
  --(6.126,3.287)--(6.129,3.288)--(6.132,3.290)--(6.135,3.291)--(6.138,3.292)--(6.141,3.294)%
  --(6.144,3.295)--(6.147,3.296)--(6.150,3.298)--(6.153,3.299)--(6.156,3.300)--(6.159,3.302)%
  --(6.162,3.303)--(6.165,3.305)--(6.168,3.306)--(6.171,3.307)--(6.174,3.309)--(6.177,3.310)%
  --(6.180,3.311)--(6.183,3.313)--(6.186,3.314)--(6.189,3.315)--(6.192,3.317)--(6.195,3.318)%
  --(6.198,3.320)--(6.201,3.321)--(6.204,3.322)--(6.207,3.324)--(6.210,3.325)--(6.213,3.326)%
  --(6.216,3.328)--(6.218,3.329)--(6.221,3.330)--(6.224,3.332)--(6.227,3.333)--(6.230,3.335)%
  --(6.233,3.336)--(6.236,3.337)--(6.239,3.339)--(6.242,3.340)--(6.245,3.341)--(6.248,3.343)%
  --(6.251,3.344)--(6.254,3.345)--(6.257,3.347)--(6.260,3.348)--(6.263,3.350)--(6.266,3.351)%
  --(6.269,3.352)--(6.272,3.354)--(6.275,3.355)--(6.278,3.356)--(6.281,3.358)--(6.284,3.359)%
  --(6.287,3.361)--(6.290,3.362)--(6.293,3.363)--(6.296,3.365)--(6.299,3.366)--(6.302,3.367)%
  --(6.305,3.369)--(6.308,3.370)--(6.311,3.371)--(6.314,3.373)--(6.317,3.374)--(6.320,3.376)%
  --(6.323,3.377)--(6.326,3.378)--(6.329,3.380)--(6.332,3.381)--(6.335,3.382)--(6.338,3.384)%
  --(6.341,3.385)--(6.344,3.387)--(6.347,3.388)--(6.350,3.389)--(6.353,3.391)--(6.356,3.392)%
  --(6.359,3.393)--(6.362,3.395)--(6.365,3.396)--(6.368,3.398)--(6.371,3.399)--(6.374,3.400)%
  --(6.377,3.402)--(6.380,3.403)--(6.383,3.404)--(6.386,3.406)--(6.389,3.407)--(6.392,3.409)%
  --(6.395,3.410)--(6.398,3.411)--(6.401,3.413)--(6.404,3.414)--(6.407,3.416)--(6.410,3.417)%
  --(6.413,3.418)--(6.416,3.420)--(6.419,3.421)--(6.422,3.422)--(6.425,3.424)--(6.428,3.425)%
  --(6.430,3.427)--(6.433,3.428)--(6.436,3.429)--(6.439,3.431)--(6.442,3.432)--(6.445,3.433)%
  --(6.448,3.435)--(6.451,3.436)--(6.454,3.438)--(6.457,3.439)--(6.460,3.440)--(6.463,3.442)%
  --(6.466,3.443)--(6.469,3.445)--(6.472,3.446)--(6.475,3.447)--(6.478,3.449)--(6.481,3.450)%
  --(6.484,3.451)--(6.487,3.453)--(6.490,3.454)--(6.493,3.456)--(6.496,3.457)--(6.499,3.458)%
  --(6.502,3.460)--(6.505,3.461)--(6.508,3.463)--(6.511,3.464)--(6.514,3.465)--(6.517,3.467)%
  --(6.520,3.468)--(6.523,3.470)--(6.526,3.471)--(6.529,3.472)--(6.532,3.474)--(6.535,3.475)%
  --(6.538,3.476)--(6.541,3.478)--(6.544,3.479)--(6.547,3.481)--(6.550,3.482)--(6.553,3.483)%
  --(6.556,3.485)--(6.559,3.486)--(6.562,3.488)--(6.565,3.489)--(6.568,3.490)--(6.571,3.492)%
  --(6.574,3.493)--(6.577,3.495)--(6.580,3.496)--(6.583,3.497)--(6.586,3.499)--(6.589,3.500)%
  --(6.592,3.502)--(6.595,3.503)--(6.598,3.504)--(6.601,3.506)--(6.604,3.507)--(6.607,3.508)%
  --(6.610,3.510)--(6.613,3.511)--(6.616,3.513)--(6.619,3.514)--(6.622,3.515)--(6.625,3.517)%
  --(6.628,3.518)--(6.631,3.520)--(6.634,3.521)--(6.637,3.522)--(6.639,3.524)--(6.642,3.525)%
  --(6.645,3.527)--(6.648,3.528)--(6.651,3.529)--(6.654,3.531)--(6.657,3.532)--(6.660,3.534)%
  --(6.663,3.535)--(6.666,3.536)--(6.669,3.538)--(6.672,3.539)--(6.675,3.541)--(6.678,3.542)%
  --(6.681,3.543)--(6.684,3.545)--(6.687,3.546)--(6.690,3.548)--(6.693,3.549)--(6.696,3.550)%
  --(6.699,3.552)--(6.702,3.553)--(6.705,3.555)--(6.708,3.556)--(6.711,3.557)--(6.714,3.559)%
  --(6.717,3.560)--(6.720,3.562)--(6.723,3.563)--(6.726,3.564)--(6.729,3.566)--(6.732,3.567)%
  --(6.735,3.569)--(6.738,3.570)--(6.741,3.571)--(6.744,3.573)--(6.747,3.574)--(6.750,3.576)%
  --(6.753,3.577)--(6.756,3.578)--(6.759,3.580)--(6.762,3.581)--(6.765,3.583)--(6.768,3.584)%
  --(6.771,3.585)--(6.774,3.587)--(6.777,3.588)--(6.780,3.590)--(6.783,3.591)--(6.786,3.592)%
  --(6.789,3.594)--(6.792,3.595)--(6.795,3.597)--(6.798,3.598)--(6.801,3.600)--(6.804,3.601)%
  --(6.807,3.602)--(6.810,3.604)--(6.813,3.605)--(6.816,3.607)--(6.819,3.608)--(6.822,3.609)%
  --(6.825,3.611)--(6.828,3.612)--(6.831,3.614)--(6.834,3.615)--(6.837,3.616)--(6.840,3.618)%
  --(6.843,3.619)--(6.846,3.621)--(6.848,3.622)--(6.851,3.623)--(6.854,3.625)--(6.857,3.626)%
  --(6.860,3.628)--(6.863,3.629)--(6.866,3.631)--(6.869,3.632)--(6.872,3.633)--(6.875,3.635)%
  --(6.878,3.636)--(6.881,3.638)--(6.884,3.639)--(6.887,3.640)--(6.890,3.642)--(6.893,3.643)%
  --(6.896,3.645)--(6.899,3.646)--(6.902,3.647)--(6.905,3.649)--(6.908,3.650)--(6.911,3.652)%
  --(6.914,3.653)--(6.917,3.655)--(6.920,3.656)--(6.923,3.657)--(6.926,3.659)--(6.929,3.660)%
  --(6.932,3.662)--(6.935,3.663)--(6.938,3.664)--(6.941,3.666)--(6.944,3.667)--(6.947,3.669)%
  --(6.950,3.670)--(6.953,3.672)--(6.956,3.673)--(6.959,3.674)--(6.962,3.676)--(6.965,3.677)%
  --(6.968,3.679)--(6.971,3.680)--(6.974,3.681)--(6.977,3.683)--(6.980,3.684)--(6.983,3.686)%
  --(6.986,3.687)--(6.989,3.688)--(6.992,3.690)--(6.995,3.691)--(6.998,3.693)--(7.001,3.694)%
  --(7.004,3.696)--(7.007,3.697)--(7.010,3.698)--(7.013,3.700)--(7.016,3.701)--(7.019,3.703)%
  --(7.022,3.704)--(7.025,3.706)--(7.028,3.707)--(7.031,3.708)--(7.034,3.710)--(7.037,3.711)%
  --(7.040,3.713)--(7.043,3.714)--(7.046,3.715)--(7.049,3.717)--(7.052,3.718)--(7.055,3.720)%
  --(7.057,3.721)--(7.060,3.723)--(7.063,3.724)--(7.066,3.725)--(7.069,3.727)--(7.072,3.728)%
  --(7.075,3.730)--(7.078,3.731)--(7.081,3.733)--(7.084,3.734)--(7.087,3.735)--(7.090,3.737)%
  --(7.093,3.738)--(7.096,3.740)--(7.099,3.741)--(7.102,3.742)--(7.105,3.744)--(7.108,3.745)%
  --(7.111,3.747)--(7.114,3.748)--(7.117,3.750)--(7.120,3.751)--(7.123,3.752)--(7.126,3.754)%
  --(7.129,3.755)--(7.132,3.757)--(7.135,3.758)--(7.138,3.760)--(7.141,3.761)--(7.144,3.762)%
  --(7.147,3.764)--(7.150,3.765)--(7.153,3.767)--(7.156,3.768)--(7.159,3.770)--(7.162,3.771)%
  --(7.165,3.772)--(7.168,3.774)--(7.171,3.775)--(7.174,3.777)--(7.177,3.778)--(7.180,3.780)%
  --(7.183,3.781)--(7.186,3.782)--(7.189,3.784)--(7.192,3.785)--(7.195,3.787)--(7.198,3.788)%
  --(7.201,3.790)--(7.204,3.791)--(7.207,3.792)--(7.210,3.794)--(7.213,3.795)--(7.216,3.797)%
  --(7.219,3.798)--(7.222,3.800)--(7.225,3.801)--(7.228,3.802)--(7.231,3.804)--(7.234,3.805)%
  --(7.237,3.807)--(7.240,3.808)--(7.243,3.810)--(7.246,3.811)--(7.249,3.812)--(7.252,3.814)%
  --(7.255,3.815)--(7.258,3.817)--(7.261,3.818)--(7.264,3.820)--(7.266,3.821)--(7.269,3.822)%
  --(7.272,3.824)--(7.275,3.825)--(7.278,3.827)--(7.281,3.828)--(7.284,3.830)--(7.287,3.831)%
  --(7.290,3.832)--(7.293,3.834)--(7.296,3.835)--(7.299,3.837)--(7.302,3.838)--(7.305,3.840)%
  --(7.308,3.841)--(7.311,3.842)--(7.314,3.844)--(7.317,3.845)--(7.320,3.847)--(7.323,3.848)%
  --(7.326,3.850)--(7.329,3.851)--(7.332,3.852)--(7.335,3.854)--(7.338,3.855)--(7.341,3.857)%
  --(7.344,3.858)--(7.347,3.860)--(7.350,3.861)--(7.353,3.863)--(7.356,3.864)--(7.359,3.865)%
  --(7.362,3.867)--(7.365,3.868)--(7.368,3.870)--(7.371,3.871)--(7.374,3.873)--(7.377,3.874)%
  --(7.380,3.875)--(7.383,3.877)--(7.386,3.878)--(7.389,3.880)--(7.392,3.881)--(7.395,3.883)%
  --(7.398,3.884)--(7.401,3.886)--(7.404,3.887)--(7.407,3.888)--(7.410,3.890)--(7.413,3.891)%
  --(7.416,3.893)--(7.419,3.894)--(7.422,3.896)--(7.425,3.897)--(7.428,3.898)--(7.431,3.900)%
  --(7.434,3.901)--(7.437,3.903)--(7.440,3.904)--(7.443,3.906)--(7.446,3.907)--(7.449,3.909)%
  --(7.452,3.910)--(7.455,3.911)--(7.458,3.913)--(7.461,3.914)--(7.464,3.916)--(7.467,3.917)%
  --(7.470,3.919)--(7.473,3.920)--(7.476,3.921)--(7.478,3.923)--(7.481,3.924)--(7.484,3.926)%
  --(7.487,3.927)--(7.490,3.929)--(7.493,3.930)--(7.496,3.932)--(7.499,3.933)--(7.502,3.934)%
  --(7.505,3.936)--(7.508,3.937)--(7.511,3.939)--(7.514,3.940)--(7.517,3.942)--(7.520,3.943)%
  --(7.523,3.945)--(7.526,3.946)--(7.529,3.947)--(7.532,3.949)--(7.535,3.950)--(7.538,3.952)%
  --(7.541,3.953)--(7.544,3.955)--(7.547,3.956)--(7.550,3.958)--(7.553,3.959)--(7.556,3.960)%
  --(7.559,3.962)--(7.562,3.963)--(7.565,3.965)--(7.568,3.966)--(7.571,3.968)--(7.574,3.969)%
  --(7.577,3.970)--(7.580,3.972)--(7.583,3.973)--(7.586,3.975)--(7.589,3.976)--(7.592,3.978)%
  --(7.595,3.979)--(7.598,3.981)--(7.601,3.982)--(7.604,3.983)--(7.607,3.985)--(7.610,3.986)%
  --(7.613,3.988)--(7.616,3.989)--(7.619,3.991)--(7.622,3.992)--(7.625,3.994)--(7.628,3.995)%
  --(7.631,3.997)--(7.634,3.998)--(7.637,3.999)--(7.640,4.001)--(7.643,4.002)--(7.646,4.004)%
  --(7.649,4.005)--(7.652,4.007)--(7.655,4.008)--(7.658,4.010)--(7.661,4.011)--(7.664,4.012)%
  --(7.667,4.014)--(7.670,4.015)--(7.673,4.017)--(7.676,4.018)--(7.679,4.020)--(7.682,4.021)%
  --(7.685,4.023)--(7.687,4.024)--(7.690,4.025)--(7.693,4.027)--(7.696,4.028)--(7.699,4.030)%
  --(7.702,4.031)--(7.705,4.033)--(7.708,4.034)--(7.711,4.036)--(7.714,4.037)--(7.717,4.038)%
  --(7.720,4.040)--(7.723,4.041)--(7.726,4.043)--(7.729,4.044)--(7.732,4.046)--(7.735,4.047)%
  --(7.738,4.049)--(7.741,4.050)--(7.744,4.052)--(7.747,4.053)--(7.750,4.054)--(7.753,4.056)%
  --(7.756,4.057)--(7.759,4.059)--(7.762,4.060)--(7.765,4.062)--(7.768,4.063)--(7.771,4.065)%
  --(7.774,4.066)--(7.777,4.067)--(7.780,4.069)--(7.783,4.070)--(7.786,4.072)--(7.789,4.073)%
  --(7.792,4.075)--(7.795,4.076)--(7.798,4.078)--(7.801,4.079)--(7.804,4.081)--(7.807,4.082)%
  --(7.810,4.083)--(7.813,4.085)--(7.816,4.086)--(7.819,4.088)--(7.822,4.089)--(7.825,4.091)%
  --(7.828,4.092)--(7.831,4.094)--(7.834,4.095)--(7.837,4.097)--(7.840,4.098)--(7.843,4.099)%
  --(7.846,4.101)--(7.849,4.102)--(7.852,4.104)--(7.855,4.105)--(7.858,4.107)--(7.861,4.108)%
  --(7.864,4.110)--(7.867,4.111)--(7.870,4.113)--(7.873,4.114)--(7.876,4.115)--(7.879,4.117)%
  --(7.882,4.118)--(7.885,4.120)--(7.888,4.121)--(7.891,4.123)--(7.894,4.124)--(7.896,4.126)%
  --(7.899,4.127)--(7.902,4.129)--(7.905,4.130)--(7.908,4.131)--(7.911,4.133)--(7.914,4.134)%
  --(7.917,4.136)--(7.920,4.137)--(7.923,4.139)--(7.926,4.140)--(7.929,4.142)--(7.932,4.143)%
  --(7.935,4.145)--(7.938,4.146)--(7.941,4.147)--(7.944,4.149)--(7.947,4.150)--(7.950,4.152)%
  --(7.953,4.153)--(7.956,4.155)--(7.959,4.156)--(7.962,4.158)--(7.965,4.159)--(7.968,4.161)%
  --(7.971,4.162)--(7.974,4.163)--(7.977,4.165)--(7.980,4.166)--(7.983,4.168)--(7.986,4.169)%
  --(7.989,4.171)--(7.992,4.172)--(7.995,4.174)--(7.998,4.175)--(8.001,4.177)--(8.004,4.178)%
  --(8.007,4.179)--(8.010,4.181)--(8.013,4.182)--(8.016,4.184)--(8.019,4.185)--(8.022,4.187)%
  --(8.025,4.188)--(8.028,4.190)--(8.031,4.191)--(8.034,4.193)--(8.037,4.194)--(8.040,4.196)%
  --(8.043,4.197)--(8.046,4.198)--(8.049,4.200)--(8.052,4.201)--(8.055,4.203)--(8.058,4.204)%
  --(8.061,4.206)--(8.064,4.207)--(8.067,4.209)--(8.070,4.210)--(8.073,4.212)--(8.076,4.213)%
  --(8.079,4.215)--(8.082,4.216)--(8.085,4.217)--(8.088,4.219)--(8.091,4.220)--(8.094,4.222)%
  --(8.097,4.223)--(8.100,4.225)--(8.103,4.226)--(8.105,4.228)--(8.108,4.229)--(8.111,4.231)%
  --(8.114,4.232)--(8.117,4.234)--(8.120,4.235)--(8.123,4.236)--(8.126,4.238)--(8.129,4.239)%
  --(8.132,4.241)--(8.135,4.242)--(8.138,4.244)--(8.141,4.245)--(8.144,4.247)--(8.147,4.248)%
  --(8.150,4.250)--(8.153,4.251)--(8.156,4.253)--(8.159,4.254)--(8.162,4.255)--(8.165,4.257)%
  --(8.168,4.258)--(8.171,4.260)--(8.174,4.261)--(8.177,4.263)--(8.180,4.264)--(8.183,4.266)%
  --(8.186,4.267)--(8.189,4.269)--(8.192,4.270)--(8.195,4.272)--(8.198,4.273)--(8.201,4.274)%
  --(8.204,4.276)--(8.207,4.277)--(8.210,4.279)--(8.213,4.280)--(8.216,4.282)--(8.219,4.283)%
  --(8.222,4.285)--(8.225,4.286)--(8.228,4.288)--(8.231,4.289)--(8.234,4.291)--(8.237,4.292)%
  --(8.240,4.293)--(8.243,4.295)--(8.246,4.296)--(8.249,4.298)--(8.252,4.299)--(8.255,4.301)%
  --(8.258,4.302)--(8.261,4.304)--(8.264,4.305)--(8.267,4.307)--(8.270,4.308)--(8.273,4.310)%
  --(8.276,4.311)--(8.279,4.313)--(8.282,4.314)--(8.285,4.315)--(8.288,4.317)--(8.291,4.318)%
  --(8.294,4.320)--(8.297,4.321)--(8.300,4.323)--(8.303,4.324)--(8.306,4.326)--(8.309,4.327)%
  --(8.312,4.329)--(8.314,4.330)--(8.317,4.332)--(8.320,4.333)--(8.323,4.335)--(8.326,4.336)%
  --(8.329,4.337)--(8.332,4.339)--(8.335,4.340)--(8.338,4.342)--(8.341,4.343)--(8.344,4.345)%
  --(8.347,4.346)--(8.350,4.348)--(8.353,4.349)--(8.356,4.351)--(8.359,4.352)--(8.362,4.354)%
  --(8.365,4.355)--(8.368,4.357)--(8.371,4.358)--(8.374,4.359)--(8.377,4.361)--(8.380,4.362)%
  --(8.383,4.364)--(8.386,4.365)--(8.389,4.367)--(8.392,4.368)--(8.395,4.370)--(8.398,4.371)%
  --(8.401,4.373)--(8.404,4.374)--(8.407,4.376)--(8.410,4.377)--(8.413,4.379)--(8.416,4.380)%
  --(8.419,4.382)--(8.422,4.383)--(8.425,4.384)--(8.428,4.386)--(8.431,4.387)--(8.434,4.389)%
  --(8.437,4.390)--(8.440,4.392)--(8.443,4.393)--(8.446,4.395)--(8.449,4.396)--(8.452,4.398)%
  --(8.455,4.399)--(8.458,4.401)--(8.461,4.402)--(8.464,4.404)--(8.467,4.405)--(8.470,4.407)%
  --(8.473,4.408)--(8.476,4.409)--(8.479,4.411)--(8.482,4.412)--(8.485,4.414)--(8.488,4.415)%
  --(8.491,4.417)--(8.494,4.418)--(8.497,4.420)--(8.500,4.421)--(8.503,4.423)--(8.506,4.424)%
  --(8.509,4.426)--(8.512,4.427)--(8.515,4.429)--(8.518,4.430)--(8.521,4.432)--(8.523,4.433)%
  --(8.526,4.434)--(8.529,4.436)--(8.532,4.437)--(8.535,4.439)--(8.538,4.440)--(8.541,4.442)%
  --(8.544,4.443)--(8.547,4.445)--(8.550,4.446)--(8.553,4.448)--(8.556,4.449)--(8.559,4.451)%
  --(8.562,4.452)--(8.565,4.454)--(8.568,4.455)--(8.571,4.457)--(8.574,4.458)--(8.577,4.459)%
  --(8.580,4.461)--(8.583,4.462)--(8.586,4.464)--(8.589,4.465)--(8.592,4.467)--(8.595,4.468)%
  --(8.598,4.470)--(8.601,4.471)--(8.604,4.473)--(8.607,4.474)--(8.610,4.476)--(8.613,4.477)%
  --(8.616,4.479)--(8.619,4.480)--(8.622,4.482)--(8.625,4.483)--(8.628,4.485)--(8.631,4.486)%
  --(8.634,4.487)--(8.637,4.489)--(8.640,4.490)--(8.643,4.492)--(8.646,4.493)--(8.649,4.495)%
  --(8.652,4.496)--(8.655,4.498)--(8.658,4.499)--(8.661,4.501)--(8.664,4.502)--(8.667,4.504)%
  --(8.670,4.505)--(8.673,4.507)--(8.676,4.508)--(8.679,4.510)--(8.682,4.511)--(8.685,4.513)%
  --(8.688,4.514)--(8.691,4.516)--(8.694,4.517)--(8.697,4.518)--(8.700,4.520)--(8.703,4.521)%
  --(8.706,4.523)--(8.709,4.524)--(8.712,4.526)--(8.715,4.527)--(8.718,4.529)--(8.721,4.530)%
  --(8.724,4.532)--(8.727,4.533)--(8.730,4.535)--(8.733,4.536)--(8.735,4.538)--(8.738,4.539)%
  --(8.741,4.541)--(8.744,4.542)--(8.747,4.544)--(8.750,4.545)--(8.753,4.546)--(8.756,4.548)%
  --(8.759,4.549)--(8.762,4.551)--(8.765,4.552)--(8.768,4.554)--(8.771,4.555)--(8.774,4.557)%
  --(8.777,4.558)--(8.780,4.560)--(8.783,4.561)--(8.786,4.563)--(8.789,4.564)--(8.792,4.566)%
  --(8.795,4.567)--(8.798,4.569)--(8.801,4.570)--(8.804,4.572)--(8.807,4.573)--(8.810,4.575)%
  --(8.813,4.576)--(8.816,4.578)--(8.819,4.579)--(8.822,4.580)--(8.825,4.582)--(8.828,4.583)%
  --(8.831,4.585)--(8.834,4.586)--(8.837,4.588)--(8.840,4.589)--(8.843,4.591)--(8.846,4.592)%
  --(8.849,4.594)--(8.852,4.595)--(8.855,4.597)--(8.858,4.598)--(8.861,4.600)--(8.864,4.601)%
  --(8.867,4.603)--(8.870,4.604)--(8.873,4.606)--(8.876,4.607)--(8.879,4.609)--(8.882,4.610)%
  --(8.885,4.612)--(8.888,4.613)--(8.891,4.614)--(8.894,4.616)--(8.897,4.617)--(8.900,4.619)%
  --(8.903,4.620)--(8.906,4.622)--(8.909,4.623)--(8.912,4.625)--(8.915,4.626)--(8.918,4.628)%
  --(8.921,4.629)--(8.924,4.631)--(8.927,4.632)--(8.930,4.634)--(8.933,4.635)--(8.936,4.637)%
  --(8.939,4.638)--(8.942,4.640)--(8.944,4.641)--(8.947,4.643)--(8.950,4.644)--(8.953,4.646)%
  --(8.956,4.647)--(8.959,4.649)--(8.962,4.650)--(8.965,4.651)--(8.968,4.653)--(8.971,4.654)%
  --(8.974,4.656)--(8.977,4.657)--(8.980,4.659)--(8.983,4.660)--(8.986,4.662)--(8.989,4.663)%
  --(8.992,4.665)--(8.995,4.666)--(8.998,4.668)--(9.001,4.669)--(9.004,4.671)--(9.007,4.672)%
  --(9.010,4.674)--(9.013,4.675)--(9.016,4.677)--(9.019,4.678)--(9.022,4.680)--(9.025,4.681)%
  --(9.028,4.683)--(9.031,4.684)--(9.034,4.686)--(9.037,4.687)--(9.040,4.689)--(9.043,4.690)%
  --(9.046,4.691)--(9.049,4.693)--(9.052,4.694)--(9.055,4.696)--(9.058,4.697)--(9.061,4.699)%
  --(9.064,4.700)--(9.067,4.702)--(9.070,4.703)--(9.073,4.705)--(9.076,4.706)--(9.079,4.708)%
  --(9.082,4.709)--(9.085,4.711)--(9.088,4.712)--(9.091,4.714)--(9.094,4.715)--(9.097,4.717)%
  --(9.100,4.718)--(9.103,4.720)--(9.106,4.721)--(9.109,4.723)--(9.112,4.724)--(9.115,4.726)%
  --(9.118,4.727)--(9.121,4.729)--(9.124,4.730)--(9.127,4.732)--(9.130,4.733)--(9.133,4.734)%
  --(9.136,4.736)--(9.139,4.737)--(9.142,4.739)--(9.145,4.740)--(9.148,4.742)--(9.151,4.743)%
  --(9.153,4.745)--(9.156,4.746)--(9.159,4.748)--(9.162,4.749)--(9.165,4.751)--(9.168,4.752)%
  --(9.171,4.754)--(9.174,4.755)--(9.177,4.757)--(9.180,4.758)--(9.183,4.760)--(9.186,4.761)%
  --(9.189,4.763)--(9.192,4.764)--(9.195,4.766)--(9.198,4.767)--(9.201,4.769)--(9.204,4.770)%
  --(9.207,4.772)--(9.210,4.773)--(9.213,4.775)--(9.216,4.776)--(9.219,4.778)--(9.222,4.779)%
  --(9.225,4.780)--(9.228,4.782)--(9.231,4.783)--(9.234,4.785)--(9.237,4.786)--(9.240,4.788)%
  --(9.243,4.789)--(9.246,4.791)--(9.249,4.792)--(9.252,4.794)--(9.255,4.795)--(9.258,4.797)%
  --(9.261,4.798)--(9.264,4.800)--(9.267,4.801)--(9.270,4.803)--(9.273,4.804)--(9.276,4.806)%
  --(9.279,4.807)--(9.282,4.809)--(9.285,4.810)--(9.288,4.812)--(9.291,4.813)--(9.294,4.815)%
  --(9.297,4.816)--(9.300,4.818)--(9.303,4.819)--(9.306,4.821)--(9.309,4.822)--(9.312,4.824)%
  --(9.315,4.825)--(9.318,4.827)--(9.321,4.828)--(9.324,4.830)--(9.327,4.831)--(9.330,4.832)%
  --(9.333,4.834)--(9.336,4.835)--(9.339,4.837)--(9.342,4.838)--(9.345,4.840)--(9.348,4.841)%
  --(9.351,4.843)--(9.354,4.844)--(9.357,4.846)--(9.360,4.847)--(9.362,4.849)--(9.365,4.850)%
  --(9.368,4.852)--(9.371,4.853)--(9.374,4.855)--(9.377,4.856)--(9.380,4.858)--(9.383,4.859)%
  --(9.386,4.861)--(9.389,4.862)--(9.392,4.864)--(9.395,4.865)--(9.398,4.867)--(9.401,4.868)%
  --(9.404,4.870)--(9.407,4.871)--(9.410,4.873)--(9.413,4.874)--(9.416,4.876)--(9.419,4.877)%
  --(9.422,4.879)--(9.425,4.880)--(9.428,4.882)--(9.431,4.883)--(9.434,4.885)--(9.437,4.886)%
  --(9.440,4.888)--(9.443,4.889)--(9.446,4.891)--(9.449,4.892)--(9.452,4.893)--(9.455,4.895)%
  --(9.458,4.896)--(9.461,4.898)--(9.464,4.899)--(9.467,4.901)--(9.470,4.902)--(9.473,4.904)%
  --(9.476,4.905)--(9.479,4.907)--(9.482,4.908)--(9.485,4.910)--(9.488,4.911)--(9.491,4.913)%
  --(9.494,4.914)--(9.497,4.916)--(9.500,4.917)--(9.503,4.919)--(9.506,4.920)--(9.509,4.922)%
  --(9.512,4.923)--(9.515,4.925)--(9.518,4.926)--(9.521,4.928)--(9.524,4.929)--(9.527,4.931)%
  --(9.530,4.932)--(9.533,4.934)--(9.536,4.935)--(9.539,4.937)--(9.542,4.938)--(9.545,4.940)%
  --(9.548,4.941)--(9.551,4.943)--(9.554,4.944)--(9.557,4.946)--(9.560,4.947)--(9.563,4.949)%
  --(9.566,4.950)--(9.569,4.952)--(9.571,4.953)--(9.574,4.955)--(9.577,4.956)--(9.580,4.958)%
  --(9.583,4.959)--(9.586,4.961)--(9.589,4.962)--(9.592,4.963)--(9.595,4.965)--(9.598,4.966)%
  --(9.601,4.968)--(9.604,4.969)--(9.607,4.971)--(9.610,4.972)--(9.613,4.974)--(9.616,4.975)%
  --(9.619,4.977)--(9.622,4.978)--(9.625,4.980)--(9.628,4.981)--(9.631,4.983)--(9.634,4.984)%
  --(9.637,4.986)--(9.640,4.987)--(9.643,4.989)--(9.646,4.990)--(9.649,4.992)--(9.652,4.993)%
  --(9.655,4.995)--(9.658,4.996)--(9.661,4.998)--(9.664,4.999)--(9.667,5.001)--(9.670,5.002)%
  --(9.673,5.004)--(9.676,5.005)--(9.679,5.007)--(9.682,5.008)--(9.685,5.010)--(9.688,5.011)%
  --(9.691,5.013)--(9.694,5.014)--(9.697,5.016)--(9.700,5.017)--(9.703,5.019)--(9.706,5.020)%
  --(9.709,5.022)--(9.712,5.023)--(9.715,5.025)--(9.718,5.026)--(9.721,5.028)--(9.724,5.029)%
  --(9.727,5.031)--(9.730,5.032)--(9.733,5.034)--(9.736,5.035)--(9.739,5.037)--(9.742,5.038)%
  --(9.745,5.040)--(9.748,5.041)--(9.751,5.043)--(9.754,5.044)--(9.757,5.046)--(9.760,5.047)%
  --(9.763,5.049)--(9.766,5.050)--(9.769,5.051)--(9.772,5.053)--(9.775,5.054)--(9.778,5.056)%
  --(9.780,5.057)--(9.783,5.059)--(9.786,5.060)--(9.789,5.062)--(9.792,5.063)--(9.795,5.065)%
  --(9.798,5.066)--(9.801,5.068)--(9.804,5.069)--(9.807,5.071)--(9.810,5.072)--(9.813,5.074)%
  --(9.816,5.075)--(9.819,5.077)--(9.822,5.078)--(9.825,5.080)--(9.828,5.081)--(9.831,5.083)%
  --(9.834,5.084)--(9.837,5.086)--(9.840,5.087)--(9.843,5.089)--(9.846,5.090)--(9.849,5.092)%
  --(9.852,5.093)--(9.855,5.095)--(9.858,5.096)--(9.861,5.098)--(9.864,5.099)--(9.867,5.101)%
  --(9.870,5.102)--(9.873,5.104)--(9.876,5.105)--(9.879,5.107)--(9.882,5.108)--(9.885,5.110)%
  --(9.888,5.111)--(9.891,5.113)--(9.894,5.114)--(9.897,5.116)--(9.900,5.117)--(9.903,5.119)%
  --(9.906,5.120)--(9.909,5.122)--(9.912,5.123)--(9.915,5.125)--(9.918,5.126)--(9.921,5.128)%
  --(9.924,5.129)--(9.927,5.131)--(9.930,5.132)--(9.933,5.134)--(9.936,5.135)--(9.939,5.137)%
  --(9.942,5.138)--(9.945,5.140)--(9.948,5.141)--(9.951,5.143)--(9.954,5.144)--(9.957,5.146)%
  --(9.960,5.147)--(9.963,5.149)--(9.966,5.150)--(9.969,5.152)--(9.972,5.153)--(9.975,5.155)%
  --(9.978,5.156)--(9.981,5.158)--(9.984,5.159)--(9.987,5.161)--(9.990,5.162)--(9.992,5.164)%
  --(9.995,5.165)--(9.998,5.167)--(10.001,5.168)--(10.004,5.170)--(10.007,5.171)--(10.010,5.173)%
  --(10.013,5.174)--(10.016,5.176)--(10.019,5.177)--(10.022,5.179)--(10.025,5.180)--(10.028,5.181)%
  --(10.031,5.183)--(10.034,5.184)--(10.037,5.186)--(10.040,5.187)--(10.043,5.189)--(10.046,5.190)%
  --(10.049,5.192)--(10.052,5.193)--(10.055,5.195)--(10.058,5.196)--(10.061,5.198)--(10.064,5.199)%
  --(10.067,5.201)--(10.070,5.202)--(10.073,5.204)--(10.076,5.205)--(10.079,5.207)--(10.082,5.208)%
  --(10.085,5.210)--(10.088,5.211)--(10.091,5.213)--(10.094,5.214)--(10.097,5.216)--(10.100,5.217)%
  --(10.103,5.219)--(10.106,5.220)--(10.109,5.222)--(10.112,5.223)--(10.115,5.225)--(10.118,5.226)%
  --(10.121,5.228)--(10.124,5.229)--(10.127,5.231)--(10.130,5.232)--(10.133,5.234)--(10.136,5.235)%
  --(10.139,5.237)--(10.142,5.238)--(10.145,5.240)--(10.148,5.241)--(10.151,5.243)--(10.154,5.244)%
  --(10.157,5.246)--(10.160,5.247)--(10.163,5.249)--(10.166,5.250)--(10.169,5.252)--(10.172,5.253)%
  --(10.175,5.255)--(10.178,5.256)--(10.181,5.258)--(10.184,5.259)--(10.187,5.261)--(10.190,5.262)%
  --(10.193,5.264)--(10.196,5.265)--(10.199,5.267)--(10.201,5.268)--(10.204,5.270)--(10.207,5.271)%
  --(10.210,5.273)--(10.213,5.274)--(10.216,5.276)--(10.219,5.277)--(10.222,5.279)--(10.225,5.280)%
  --(10.228,5.282)--(10.231,5.283)--(10.234,5.285)--(10.237,5.286)--(10.240,5.288)--(10.243,5.289)%
  --(10.246,5.291)--(10.249,5.292)--(10.252,5.294)--(10.255,5.295)--(10.258,5.297)--(10.261,5.298)%
  --(10.264,5.300)--(10.267,5.301)--(10.270,5.303)--(10.273,5.304)--(10.276,5.306)--(10.279,5.307)%
  --(10.282,5.309)--(10.285,5.310)--(10.288,5.312)--(10.291,5.313)--(10.294,5.315)--(10.297,5.316)%
  --(10.300,5.318)--(10.303,5.319)--(10.306,5.321)--(10.309,5.322)--(10.312,5.324)--(10.315,5.325)%
  --(10.318,5.327)--(10.321,5.328)--(10.324,5.330)--(10.327,5.331)--(10.330,5.333)--(10.333,5.334)%
  --(10.336,5.336)--(10.339,5.337)--(10.342,5.339)--(10.345,5.340)--(10.348,5.342)--(10.351,5.343)%
  --(10.354,5.345)--(10.357,5.346)--(10.360,5.348)--(10.363,5.349)--(10.366,5.351)--(10.369,5.352)%
  --(10.372,5.354)--(10.375,5.355)--(10.378,5.357)--(10.381,5.358)--(10.384,5.360)--(10.387,5.361)%
  --(10.390,5.363)--(10.393,5.364)--(10.396,5.366)--(10.399,5.367)--(10.402,5.369)--(10.405,5.370)%
  --(10.408,5.372)--(10.410,5.373)--(10.413,5.375)--(10.416,5.376)--(10.419,5.378)--(10.422,5.379)%
  --(10.425,5.381)--(10.428,5.382)--(10.431,5.384)--(10.434,5.385)--(10.437,5.387)--(10.440,5.388)%
  --(10.443,5.390)--(10.446,5.391)--(10.449,5.393)--(10.452,5.394)--(10.455,5.396)--(10.458,5.397)%
  --(10.461,5.399)--(10.464,5.400)--(10.467,5.402)--(10.470,5.403)--(10.473,5.405)--(10.476,5.406)%
  --(10.479,5.408)--(10.482,5.409)--(10.485,5.411)--(10.488,5.412)--(10.491,5.414)--(10.494,5.415)%
  --(10.497,5.417)--(10.500,5.418)--(10.503,5.420)--(10.506,5.421)--(10.509,5.423)--(10.512,5.424)%
  --(10.515,5.426)--(10.518,5.427)--(10.521,5.429)--(10.524,5.430)--(10.527,5.432)--(10.530,5.433)%
  --(10.533,5.435)--(10.536,5.436)--(10.539,5.438)--(10.542,5.439)--(10.545,5.441)--(10.548,5.442)%
  --(10.551,5.444)--(10.554,5.445)--(10.557,5.447)--(10.560,5.448)--(10.563,5.450)--(10.566,5.451)%
  --(10.569,5.453)--(10.572,5.454)--(10.575,5.456)--(10.578,5.457)--(10.581,5.459)--(10.584,5.460)%
  --(10.587,5.462)--(10.590,5.463)--(10.593,5.465)--(10.596,5.466)--(10.599,5.468)--(10.602,5.469)%
  --(10.605,5.471)--(10.608,5.472)--(10.611,5.474)--(10.614,5.475)--(10.617,5.477)--(10.619,5.478)%
  --(10.622,5.480)--(10.625,5.481)--(10.628,5.483)--(10.631,5.484)--(10.634,5.486)--(10.637,5.487)%
  --(10.640,5.489)--(10.643,5.490)--(10.646,5.492)--(10.649,5.493)--(10.652,5.495)--(10.655,5.496)%
  --(10.658,5.498)--(10.661,5.499)--(10.664,5.501)--(10.667,5.502)--(10.670,5.504)--(10.673,5.505)%
  --(10.676,5.507)--(10.679,5.508)--(10.682,5.510)--(10.685,5.511)--(10.688,5.513)--(10.691,5.514)%
  --(10.694,5.516)--(10.697,5.517)--(10.700,5.519)--(10.703,5.520)--(10.706,5.522)--(10.709,5.523)%
  --(10.712,5.525)--(10.715,5.526)--(10.718,5.528)--(10.721,5.529)--(10.724,5.531)--(10.727,5.532)%
  --(10.730,5.534)--(10.733,5.535)--(10.736,5.537)--(10.739,5.538)--(10.742,5.540)--(10.745,5.541)%
  --(10.748,5.543)--(10.751,5.544)--(10.754,5.546)--(10.757,5.547)--(10.760,5.549)--(10.763,5.550)%
  --(10.766,5.552)--(10.769,5.553)--(10.772,5.555)--(10.775,5.556)--(10.778,5.558)--(10.781,5.559)%
  --(10.784,5.561)--(10.787,5.562)--(10.790,5.564)--(10.793,5.565)--(10.796,5.567)--(10.799,5.568)%
  --(10.802,5.570)--(10.805,5.571)--(10.808,5.573)--(10.811,5.574)--(10.814,5.576)--(10.817,5.577)%
  --(10.820,5.579)--(10.823,5.580)--(10.826,5.582)--(10.828,5.583)--(10.831,5.585)--(10.834,5.586)%
  --(10.837,5.588)--(10.840,5.589)--(10.843,5.591)--(10.846,5.592)--(10.849,5.594)--(10.852,5.595)%
  --(10.855,5.597)--(10.858,5.598)--(10.861,5.600)--(10.864,5.601)--(10.867,5.603)--(10.870,5.604)%
  --(10.873,5.606)--(10.876,5.607)--(10.879,5.609)--(10.882,5.610)--(10.885,5.612)--(10.888,5.613)%
  --(10.891,5.615)--(10.894,5.616)--(10.897,5.618)--(10.900,5.620)--(10.903,5.621)--(10.906,5.623)%
  --(10.909,5.624)--(10.912,5.626)--(10.915,5.627)--(10.918,5.629)--(10.921,5.630)--(10.924,5.632)%
  --(10.927,5.633)--(10.930,5.635)--(10.933,5.636)--(10.936,5.638)--(10.939,5.639)--(10.942,5.641)%
  --(10.945,5.642)--(10.948,5.644)--(10.951,5.645)--(10.954,5.647)--(10.957,5.648)--(10.960,5.650)%
  --(10.963,5.651)--(10.966,5.653)--(10.969,5.654)--(10.972,5.656)--(10.975,5.657)--(10.978,5.659)%
  --(10.981,5.660)--(10.984,5.662)--(10.987,5.663)--(10.990,5.665)--(10.993,5.666)--(10.996,5.668)%
  --(10.999,5.669)--(11.002,5.671)--(11.005,5.672)--(11.008,5.674)--(11.011,5.675)--(11.014,5.677)%
  --(11.017,5.678)--(11.020,5.680)--(11.023,5.681)--(11.026,5.683)--(11.029,5.684)--(11.032,5.686)%
  --(11.035,5.687)--(11.037,5.689)--(11.040,5.690)--(11.043,5.692)--(11.046,5.693)--(11.049,5.695)%
  --(11.052,5.696)--(11.055,5.698)--(11.058,5.699)--(11.061,5.701)--(11.064,5.702)--(11.067,5.704)%
  --(11.070,5.705)--(11.073,5.707)--(11.076,5.708)--(11.079,5.710)--(11.082,5.711)--(11.085,5.713)%
  --(11.088,5.714)--(11.091,5.716)--(11.094,5.717)--(11.097,5.719)--(11.100,5.720)--(11.103,5.722)%
  --(11.106,5.723)--(11.109,5.725)--(11.112,5.726)--(11.115,5.728)--(11.118,5.729)--(11.121,5.731)%
  --(11.124,5.732)--(11.127,5.734)--(11.130,5.735)--(11.133,5.737)--(11.136,5.738)--(11.139,5.740)%
  --(11.142,5.741)--(11.145,5.743)--(11.148,5.744)--(11.151,5.746)--(11.154,5.747)--(11.157,5.749)%
  --(11.160,5.750)--(11.163,5.752)--(11.166,5.753)--(11.169,5.755)--(11.172,5.756)--(11.175,5.758)%
  --(11.178,5.759)--(11.181,5.761)--(11.184,5.762)--(11.187,5.764)--(11.190,5.765)--(11.193,5.767)%
  --(11.196,5.768)--(11.199,5.770)--(11.202,5.772)--(11.205,5.773)--(11.208,5.775)--(11.211,5.776)%
  --(11.214,5.778)--(11.217,5.779)--(11.220,5.781)--(11.223,5.782)--(11.226,5.784)--(11.229,5.785)%
  --(11.232,5.787)--(11.235,5.788)--(11.238,5.790)--(11.241,5.791)--(11.244,5.793)--(11.247,5.794)%
  --(11.249,5.796)--(11.252,5.797)--(11.255,5.799)--(11.258,5.800)--(11.261,5.802)--(11.264,5.803)%
  --(11.267,5.805)--(11.270,5.806)--(11.273,5.808)--(11.276,5.809)--(11.279,5.811)--(11.282,5.812)%
  --(11.285,5.814)--(11.288,5.815)--(11.291,5.817)--(11.294,5.818)--(11.297,5.820)--(11.300,5.821)%
  --(11.303,5.823)--(11.306,5.824)--(11.309,5.826)--(11.312,5.827)--(11.315,5.829)--(11.318,5.830)%
  --(11.321,5.832)--(11.324,5.833)--(11.327,5.835)--(11.330,5.836)--(11.333,5.838)--(11.336,5.839)%
  --(11.339,5.841)--(11.342,5.842)--(11.345,5.844)--(11.348,5.845)--(11.351,5.847)--(11.354,5.848)%
  --(11.357,5.850)--(11.360,5.851)--(11.363,5.853)--(11.366,5.854)--(11.369,5.856)--(11.372,5.857)%
  --(11.375,5.859)--(11.378,5.860)--(11.381,5.862)--(11.384,5.863)--(11.387,5.865)--(11.390,5.866)%
  --(11.393,5.868)--(11.396,5.869)--(11.399,5.871)--(11.402,5.872)--(11.405,5.874)--(11.408,5.875)%
  --(11.411,5.877)--(11.414,5.878)--(11.417,5.880)--(11.420,5.881)--(11.423,5.883)--(11.426,5.885)%
  --(11.429,5.886)--(11.432,5.888)--(11.435,5.889)--(11.438,5.891)--(11.441,5.892)--(11.444,5.894)%
  --(11.447,5.895)--(11.450,5.897)--(11.453,5.898)--(11.456,5.900)--(11.458,5.901)--(11.461,5.903)%
  --(11.464,5.904)--(11.467,5.906)--(11.470,5.907)--(11.473,5.909)--(11.476,5.910)--(11.479,5.912)%
  --(11.482,5.913)--(11.485,5.915)--(11.488,5.916)--(11.491,5.918)--(11.494,5.919)--(11.497,5.921)%
  --(11.500,5.922)--(11.503,5.924)--(11.506,5.925)--(11.509,5.927)--(11.512,5.928)--(11.515,5.930)%
  --(11.518,5.931)--(11.521,5.933)--(11.524,5.934)--(11.527,5.936)--(11.530,5.937)--(11.533,5.939)%
  --(11.536,5.940)--(11.539,5.942)--(11.542,5.943)--(11.545,5.945)--(11.548,5.946)--(11.551,5.948)%
  --(11.554,5.949)--(11.557,5.951)--(11.560,5.952)--(11.563,5.954)--(11.566,5.955)--(11.569,5.957)%
  --(11.572,5.958)--(11.575,5.960)--(11.578,5.961)--(11.581,5.963)--(11.584,5.964)--(11.587,5.966)%
  --(11.590,5.967)--(11.593,5.969)--(11.596,5.970)--(11.599,5.972)--(11.602,5.973)--(11.605,5.975)%
  --(11.608,5.977)--(11.611,5.978)--(11.614,5.980)--(11.617,5.981)--(11.620,5.983)--(11.623,5.984)%
  --(11.626,5.986)--(11.629,5.987)--(11.632,5.989)--(11.635,5.990)--(11.638,5.992)--(11.641,5.993)%
  --(11.644,5.995)--(11.647,5.996)--(11.650,5.998)--(11.653,5.999)--(11.656,6.001)--(11.659,6.002)%
  --(11.662,6.004)--(11.665,6.005)--(11.667,6.007)--(11.670,6.008)--(11.673,6.010)--(11.676,6.011)%
  --(11.679,6.013)--(11.682,6.014)--(11.685,6.016)--(11.688,6.017)--(11.691,6.019)--(11.694,6.020)%
  --(11.697,6.022)--(11.700,6.023)--(11.703,6.025)--(11.706,6.026)--(11.709,6.028)--(11.712,6.029)%
  --(11.715,6.031)--(11.718,6.032)--(11.721,6.034)--(11.724,6.035)--(11.727,6.037)--(11.730,6.038)%
  --(11.733,6.040)--(11.736,6.041)--(11.739,6.043)--(11.742,6.044)--(11.745,6.046)--(11.748,6.047)%
  --(11.751,6.049)--(11.754,6.050)--(11.757,6.052)--(11.760,6.053)--(11.763,6.055)--(11.766,6.056)%
  --(11.769,6.058)--(11.772,6.060)--(11.775,6.061)--(11.778,6.063)--(11.781,6.064)--(11.784,6.066)%
  --(11.787,6.067)--(11.790,6.069)--(11.793,6.070)--(11.796,6.072)--(11.799,6.073)--(11.802,6.075)%
  --(11.805,6.076)--(11.808,6.078)--(11.811,6.079)--(11.814,6.081)--(11.817,6.082)--(11.820,6.084)%
  --(11.823,6.085)--(11.826,6.087)--(11.829,6.088)--(11.832,6.090)--(11.835,6.091)--(11.838,6.093)%
  --(11.841,6.094)--(11.844,6.096)--(11.847,6.097)--(11.850,6.099)--(11.853,6.100)--(11.856,6.102)%
  --(11.859,6.103)--(11.862,6.105)--(11.865,6.106)--(11.868,6.108)--(11.871,6.109)--(11.874,6.111)%
  --(11.876,6.112)--(11.879,6.114)--(11.882,6.115)--(11.885,6.117)--(11.888,6.118)--(11.891,6.120)%
  --(11.894,6.121)--(11.897,6.123)--(11.900,6.124)--(11.903,6.126)--(11.906,6.127)--(11.909,6.129)%
  --(11.912,6.130)--(11.915,6.132)--(11.918,6.133)--(11.921,6.135)--(11.924,6.137)--(11.927,6.138)%
  --(11.930,6.140)--(11.933,6.141)--(11.936,6.143)--(11.939,6.144)--(11.942,6.146)--(11.945,6.147)%
  --(11.948,6.149)--(11.951,6.150)--(11.954,6.152)--(11.957,6.153)--(11.960,6.155)--(11.963,6.156)%
  --(11.966,6.158)--(11.969,6.159)--(11.972,6.161)--(11.975,6.162)--(11.978,6.164)--(11.981,6.165)%
  --(11.984,6.167)--(11.987,6.168)--(11.990,6.170)--(11.993,6.171)--(11.996,6.173)--(11.999,6.174)%
  --(12.002,6.176)--(12.005,6.177)--(12.008,6.179)--(12.011,6.180)--(12.014,6.182)--(12.017,6.183)%
  --(12.020,6.185)--(12.023,6.186)--(12.026,6.188)--(12.029,6.189)--(12.032,6.191)--(12.035,6.192)%
  --(12.038,6.194)--(12.041,6.195)--(12.044,6.197)--(12.047,6.198)--(12.050,6.200)--(12.053,6.201)%
  --(12.056,6.203)--(12.059,6.204)--(12.062,6.206)--(12.065,6.208)--(12.068,6.209)--(12.071,6.211)%
  --(12.074,6.212)--(12.077,6.214)--(12.080,6.215)--(12.083,6.217)--(12.085,6.218)--(12.088,6.220)%
  --(12.091,6.221)--(12.094,6.223)--(12.097,6.224)--(12.100,6.226)--(12.103,6.227)--(12.106,6.229)%
  --(12.109,6.230)--(12.112,6.232)--(12.115,6.233)--(12.118,6.235)--(12.121,6.236)--(12.124,6.238)%
  --(12.127,6.239)--(12.130,6.241)--(12.133,6.242)--(12.136,6.244)--(12.139,6.245)--(12.142,6.247)%
  --(12.145,6.248)--(12.148,6.250)--(12.151,6.251)--(12.154,6.253)--(12.157,6.254)--(12.160,6.256)%
  --(12.163,6.257)--(12.166,6.259)--(12.169,6.260)--(12.172,6.262)--(12.175,6.263)--(12.178,6.265)%
  --(12.181,6.266)--(12.184,6.268)--(12.187,6.269)--(12.190,6.271)--(12.193,6.273)--(12.196,6.274)%
  --(12.199,6.276)--(12.202,6.277)--(12.205,6.279)--(12.208,6.280)--(12.211,6.282)--(12.214,6.283)%
  --(12.217,6.285)--(12.220,6.286)--(12.223,6.288)--(12.226,6.289)--(12.229,6.291)--(12.232,6.292)%
  --(12.235,6.294)--(12.238,6.295)--(12.241,6.297)--(12.244,6.298)--(12.247,6.300)--(12.250,6.301)%
  --(12.253,6.303)--(12.256,6.304)--(12.259,6.306)--(12.262,6.307)--(12.265,6.309)--(12.268,6.310)%
  --(12.271,6.312)--(12.274,6.313)--(12.277,6.315)--(12.280,6.316)--(12.283,6.318)--(12.286,6.319)%
  --(12.289,6.321)--(12.292,6.322)--(12.295,6.324)--(12.297,6.325)--(12.300,6.327)--(12.303,6.328)%
  --(12.306,6.330)--(12.309,6.331)--(12.312,6.333)--(12.315,6.334)--(12.318,6.336)--(12.321,6.338)%
  --(12.324,6.339)--(12.327,6.341)--(12.330,6.342)--(12.333,6.344)--(12.336,6.345)--(12.339,6.347)%
  --(12.342,6.348)--(12.345,6.350)--(12.348,6.351)--(12.351,6.353)--(12.354,6.354)--(12.357,6.356)%
  --(12.360,6.357)--(12.363,6.359)--(12.366,6.360)--(12.369,6.362)--(12.372,6.363)--(12.375,6.365)%
  --(12.378,6.366)--(12.381,6.368)--(12.384,6.369)--(12.387,6.371)--(12.390,6.372)--(12.393,6.374)%
  --(12.396,6.375)--(12.399,6.377)--(12.402,6.378)--(12.405,6.380)--(12.408,6.381)--(12.411,6.383)%
  --(12.414,6.384)--(12.417,6.386)--(12.420,6.387)--(12.423,6.389)--(12.426,6.390)--(12.429,6.392)%
  --(12.432,6.393)--(12.435,6.395)--(12.438,6.397)--(12.441,6.398)--(12.444,6.400)--(12.447,6.401)%
  --(12.450,6.403)--(12.453,6.404)--(12.456,6.406)--(12.459,6.407)--(12.462,6.409)--(12.465,6.410)%
  --(12.468,6.412)--(12.471,6.413)--(12.474,6.415)--(12.477,6.416)--(12.480,6.418)--(12.483,6.419)%
  --(12.486,6.421)--(12.489,6.422)--(12.492,6.424)--(12.495,6.425)--(12.498,6.427)--(12.501,6.428)%
  --(12.504,6.430)--(12.506,6.431)--(12.509,6.433)--(12.512,6.434)--(12.515,6.436)--(12.518,6.437)%
  --(12.521,6.439)--(12.524,6.440)--(12.527,6.442)--(12.530,6.443)--(12.533,6.445)--(12.536,6.446)%
  --(12.539,6.448)--(12.542,6.449)--(12.545,6.451)--(12.548,6.452)--(12.551,6.454)--(12.554,6.456)%
  --(12.557,6.457)--(12.560,6.459)--(12.563,6.460)--(12.566,6.462)--(12.569,6.463)--(12.572,6.465)%
  --(12.575,6.466)--(12.578,6.468)--(12.581,6.469)--(12.584,6.471)--(12.587,6.472)--(12.590,6.474)%
  --(12.593,6.475)--(12.596,6.477)--(12.599,6.478)--(12.602,6.480)--(12.605,6.481)--(12.608,6.483)%
  --(12.611,6.484)--(12.614,6.486)--(12.617,6.487)--(12.620,6.489)--(12.623,6.490)--(12.626,6.492)%
  --(12.629,6.493)--(12.632,6.495)--(12.635,6.496)--(12.638,6.498)--(12.641,6.499)--(12.644,6.501)%
  --(12.647,6.502)--(12.650,6.504)--(12.653,6.505)--(12.656,6.507)--(12.659,6.508)--(12.662,6.510)%
  --(12.665,6.512)--(12.668,6.513)--(12.671,6.515)--(12.674,6.516)--(12.677,6.518)--(12.680,6.519)%
  --(12.683,6.521)--(12.686,6.522)--(12.689,6.524)--(12.692,6.525)--(12.695,6.527)--(12.698,6.528)%
  --(12.701,6.530)--(12.704,6.531)--(12.707,6.533)--(12.710,6.534)--(12.713,6.536)--(12.715,6.537)%
  --(12.718,6.539)--(12.721,6.540)--(12.724,6.542)--(12.727,6.543)--(12.730,6.545)--(12.733,6.546)%
  --(12.736,6.548)--(12.739,6.549)--(12.742,6.551)--(12.745,6.552)--(12.748,6.554)--(12.751,6.555)%
  --(12.754,6.557)--(12.757,6.558)--(12.760,6.560)--(12.763,6.561)--(12.766,6.563)--(12.769,6.565)%
  --(12.772,6.566)--(12.775,6.568)--(12.778,6.569)--(12.781,6.571)--(12.784,6.572)--(12.787,6.574)%
  --(12.790,6.575)--(12.793,6.577)--(12.796,6.578)--(12.799,6.580)--(12.802,6.581)--(12.805,6.583)%
  --(12.808,6.584)--(12.811,6.586)--(12.814,6.587)--(12.817,6.589)--(12.820,6.590)--(12.823,6.592)%
  --(12.826,6.593)--(12.829,6.595)--(12.832,6.596)--(12.835,6.598)--(12.838,6.599)--(12.841,6.601)%
  --(12.844,6.602)--(12.847,6.604)--(12.850,6.605)--(12.853,6.607)--(12.856,6.608)--(12.859,6.610)%
  --(12.862,6.611)--(12.865,6.613)--(12.868,6.614)--(12.871,6.616)--(12.874,6.618)--(12.877,6.619)%
  --(12.880,6.621)--(12.883,6.622)--(12.886,6.624)--(12.889,6.625)--(12.892,6.627)--(12.895,6.628)%
  --(12.898,6.630)--(12.901,6.631)--(12.904,6.633)--(12.907,6.634)--(12.910,6.636)--(12.913,6.637)%
  --(12.916,6.639)--(12.919,6.640)--(12.922,6.642)--(12.924,6.643)--(12.927,6.645)--(12.930,6.646)%
  --(12.933,6.648)--(12.936,6.649)--(12.939,6.651)--(12.942,6.652)--(12.945,6.654)--(12.948,6.655)%
  --(12.951,6.657)--(12.954,6.658)--(12.957,6.660)--(12.960,6.661)--(12.963,6.663)--(12.966,6.664)%
  --(12.969,6.666)--(12.972,6.667)--(12.975,6.669)--(12.978,6.671)--(12.981,6.672)--(12.984,6.674)%
  --(12.987,6.675)--(12.990,6.677)--(12.993,6.678)--(12.996,6.680)--(12.999,6.681)--(13.002,6.683)%
  --(13.005,6.684)--(13.008,6.686)--(13.011,6.687)--(13.014,6.689)--(13.017,6.690)--(13.020,6.692)%
  --(13.023,6.693)--(13.026,6.695)--(13.029,6.696)--(13.032,6.698)--(13.035,6.699)--(13.038,6.701)%
  --(13.041,6.702)--(13.044,6.704)--(13.047,6.705)--(13.050,6.707)--(13.053,6.708)--(13.056,6.710)%
  --(13.059,6.711)--(13.062,6.713)--(13.065,6.714)--(13.068,6.716)--(13.071,6.717)--(13.074,6.719)%
  --(13.077,6.721)--(13.080,6.722)--(13.083,6.724)--(13.086,6.725)--(13.089,6.727)--(13.092,6.728)%
  --(13.095,6.730)--(13.098,6.731)--(13.101,6.733)--(13.104,6.734)--(13.107,6.736)--(13.110,6.737)%
  --(13.113,6.739)--(13.116,6.740)--(13.119,6.742)--(13.122,6.743)--(13.125,6.745)--(13.128,6.746)%
  --(13.131,6.748)--(13.133,6.749)--(13.136,6.751)--(13.139,6.752)--(13.142,6.754)--(13.145,6.755)%
  --(13.148,6.757)--(13.151,6.758)--(13.154,6.760)--(13.157,6.761)--(13.160,6.763)--(13.163,6.764)%
  --(13.166,6.766)--(13.169,6.768)--(13.172,6.769)--(13.175,6.771)--(13.178,6.772)--(13.181,6.774)%
  --(13.184,6.775)--(13.187,6.777)--(13.190,6.778)--(13.193,6.780)--(13.196,6.781)--(13.199,6.783)%
  --(13.202,6.784)--(13.205,6.786)--(13.208,6.787)--(13.211,6.789)--(13.214,6.790)--(13.217,6.792)%
  --(13.220,6.793)--(13.223,6.795)--(13.226,6.796)--(13.229,6.798)--(13.232,6.799)--(13.235,6.801)%
  --(13.238,6.802)--(13.241,6.804)--(13.244,6.805)--(13.247,6.807)--(13.250,6.808)--(13.253,6.810)%
  --(13.256,6.811)--(13.259,6.813)--(13.262,6.814)--(13.265,6.816)--(13.268,6.818)--(13.271,6.819)%
  --(13.274,6.821)--(13.277,6.822)--(13.280,6.824)--(13.283,6.825)--(13.286,6.827)--(13.289,6.828)%
  --(13.292,6.830)--(13.295,6.831)--(13.298,6.833)--(13.301,6.834)--(13.304,6.836)--(13.307,6.837)%
  --(13.310,6.839)--(13.313,6.840)--(13.316,6.842)--(13.319,6.843)--(13.322,6.845)--(13.325,6.846)%
  --(13.328,6.848)--(13.331,6.849)--(13.334,6.851)--(13.337,6.852)--(13.340,6.854)--(13.342,6.855)%
  --(13.345,6.857)--(13.348,6.858)--(13.351,6.860)--(13.354,6.861)--(13.357,6.863)--(13.360,6.865)%
  --(13.363,6.866)--(13.366,6.868)--(13.369,6.869)--(13.372,6.871)--(13.375,6.872)--(13.378,6.874)%
  --(13.381,6.875)--(13.384,6.877)--(13.387,6.878)--(13.390,6.880)--(13.393,6.881)--(13.396,6.883)%
  --(13.399,6.884)--(13.402,6.886)--(13.405,6.887)--(13.408,6.889)--(13.411,6.890)--(13.414,6.892)%
  --(13.417,6.893)--(13.420,6.895)--(13.423,6.896)--(13.426,6.898)--(13.429,6.899)--(13.432,6.901)%
  --(13.435,6.902)--(13.438,6.904)--(13.441,6.905)--(13.444,6.907);
\gpcolor{color=gp lt color border}
\node[gp node left] at (2.972,7.373) {$\rho \approx \rho_{\rm{max}}$};
\gpcolor{rgb color={0.902,0.624,0.000}}
\draw[gp path] (1.872,7.373)--(2.788,7.373);
\draw[gp path] (1.507,2.514)--(1.510,2.514)--(1.513,2.514)--(1.516,2.514)--(1.519,2.514)%
  --(1.522,2.514)--(1.525,2.514)--(1.528,2.514)--(1.531,2.514)--(1.534,2.514)--(1.537,2.514)%
  --(1.540,2.514)--(1.543,2.514)--(1.546,2.514)--(1.549,2.514)--(1.552,2.514)--(1.555,2.514)%
  --(1.558,2.514)--(1.561,2.514)--(1.564,2.514)--(1.567,2.514)--(1.570,2.514)--(1.573,2.514)%
  --(1.576,2.514)--(1.579,2.514)--(1.582,2.514)--(1.585,2.514)--(1.588,2.514)--(1.591,2.514)%
  --(1.594,2.514)--(1.597,2.514)--(1.600,2.514)--(1.603,2.514)--(1.606,2.514)--(1.609,2.514)%
  --(1.611,2.514)--(1.614,2.514)--(1.617,2.514)--(1.620,2.514)--(1.623,2.514)--(1.626,2.514)%
  --(1.629,2.514)--(1.632,2.514)--(1.635,2.514)--(1.638,2.514)--(1.641,2.514)--(1.644,2.514)%
  --(1.647,2.514)--(1.650,2.514)--(1.653,2.514)--(1.656,2.514)--(1.659,2.514)--(1.662,2.514)%
  --(1.665,2.514)--(1.668,2.514)--(1.671,2.514)--(1.674,2.514)--(1.677,2.514)--(1.680,2.514)%
  --(1.683,2.514)--(1.686,2.514)--(1.689,2.514)--(1.692,2.514)--(1.695,2.514)--(1.698,2.514)%
  --(1.701,2.514)--(1.704,2.514)--(1.707,2.514)--(1.710,2.514)--(1.713,2.514)--(1.716,2.514)%
  --(1.719,2.514)--(1.722,2.514)--(1.725,2.514)--(1.728,2.514)--(1.731,2.515)--(1.734,2.515)%
  --(1.737,2.515)--(1.740,2.515)--(1.743,2.515)--(1.746,2.515)--(1.749,2.515)--(1.752,2.515)%
  --(1.755,2.515)--(1.758,2.515)--(1.761,2.515)--(1.764,2.515)--(1.767,2.515)--(1.770,2.515)%
  --(1.773,2.515)--(1.776,2.515)--(1.779,2.515)--(1.782,2.515)--(1.785,2.515)--(1.788,2.515)%
  --(1.791,2.515)--(1.794,2.515)--(1.797,2.515)--(1.800,2.515)--(1.803,2.515)--(1.806,2.515)%
  --(1.809,2.515)--(1.812,2.515)--(1.815,2.515)--(1.818,2.515)--(1.820,2.515)--(1.823,2.515)%
  --(1.826,2.516)--(1.829,2.516)--(1.832,2.516)--(1.835,2.516)--(1.838,2.516)--(1.841,2.516)%
  --(1.844,2.516)--(1.847,2.516)--(1.850,2.516)--(1.853,2.516)--(1.856,2.516)--(1.859,2.516)%
  --(1.862,2.516)--(1.865,2.516)--(1.868,2.516)--(1.871,2.516)--(1.874,2.516)--(1.877,2.516)%
  --(1.880,2.517)--(1.883,2.517)--(1.886,2.517)--(1.889,2.517)--(1.892,2.517)--(1.895,2.517)%
  --(1.898,2.517)--(1.901,2.517)--(1.904,2.517)--(1.907,2.517)--(1.910,2.517)--(1.913,2.517)%
  --(1.916,2.517)--(1.919,2.517)--(1.922,2.517)--(1.925,2.518)--(1.928,2.518)--(1.931,2.518)%
  --(1.934,2.518)--(1.937,2.518)--(1.940,2.518)--(1.943,2.518)--(1.946,2.518)--(1.949,2.518)%
  --(1.952,2.518)--(1.955,2.518)--(1.958,2.518)--(1.961,2.519)--(1.964,2.519)--(1.967,2.519)%
  --(1.970,2.519)--(1.973,2.519)--(1.976,2.519)--(1.979,2.519)--(1.982,2.519)--(1.985,2.519)%
  --(1.988,2.519)--(1.991,2.520)--(1.994,2.520)--(1.997,2.520)--(2.000,2.520)--(2.003,2.520)%
  --(2.006,2.520)--(2.009,2.520)--(2.012,2.520)--(2.015,2.520)--(2.018,2.521)--(2.021,2.521)%
  --(2.024,2.521)--(2.027,2.521)--(2.029,2.521)--(2.032,2.521)--(2.035,2.521)--(2.038,2.521)%
  --(2.041,2.522)--(2.044,2.522)--(2.047,2.522)--(2.050,2.522)--(2.053,2.522)--(2.056,2.522)%
  --(2.059,2.522)--(2.062,2.522)--(2.065,2.523)--(2.068,2.523)--(2.071,2.523)--(2.074,2.523)%
  --(2.077,2.523)--(2.080,2.523)--(2.083,2.523)--(2.086,2.524)--(2.089,2.524)--(2.092,2.524)%
  --(2.095,2.524)--(2.098,2.524)--(2.101,2.524)--(2.104,2.524)--(2.107,2.525)--(2.110,2.525)%
  --(2.113,2.525)--(2.116,2.525)--(2.119,2.525)--(2.122,2.525)--(2.125,2.526)--(2.128,2.526)%
  --(2.131,2.526)--(2.134,2.526)--(2.137,2.526)--(2.140,2.526)--(2.143,2.527)--(2.146,2.527)%
  --(2.149,2.527)--(2.152,2.527)--(2.155,2.527)--(2.158,2.527)--(2.161,2.528)--(2.164,2.528)%
  --(2.167,2.528)--(2.170,2.528)--(2.173,2.528)--(2.176,2.529)--(2.179,2.529)--(2.182,2.529)%
  --(2.185,2.529)--(2.188,2.529)--(2.191,2.530)--(2.194,2.530)--(2.197,2.530)--(2.200,2.530)%
  --(2.203,2.530)--(2.206,2.531)--(2.209,2.531)--(2.212,2.531)--(2.215,2.531)--(2.218,2.531)%
  --(2.221,2.532)--(2.224,2.532)--(2.227,2.532)--(2.230,2.532)--(2.233,2.533)--(2.236,2.533)%
  --(2.238,2.533)--(2.241,2.533)--(2.244,2.533)--(2.247,2.534)--(2.250,2.534)--(2.253,2.534)%
  --(2.256,2.534)--(2.259,2.535)--(2.262,2.535)--(2.265,2.535)--(2.268,2.535)--(2.271,2.536)%
  --(2.274,2.536)--(2.277,2.536)--(2.280,2.536)--(2.283,2.537)--(2.286,2.537)--(2.289,2.537)%
  --(2.292,2.537)--(2.295,2.538)--(2.298,2.538)--(2.301,2.538)--(2.304,2.538)--(2.307,2.539)%
  --(2.310,2.539)--(2.313,2.539)--(2.316,2.539)--(2.319,2.540)--(2.322,2.540)--(2.325,2.540)%
  --(2.328,2.540)--(2.331,2.541)--(2.334,2.541)--(2.337,2.541)--(2.340,2.542)--(2.343,2.542)%
  --(2.346,2.542)--(2.349,2.542)--(2.352,2.543)--(2.355,2.543)--(2.358,2.543)--(2.361,2.544)%
  --(2.364,2.544)--(2.367,2.544)--(2.370,2.544)--(2.373,2.545)--(2.376,2.545)--(2.379,2.545)%
  --(2.382,2.546)--(2.385,2.546)--(2.388,2.546)--(2.391,2.547)--(2.394,2.547)--(2.397,2.547)%
  --(2.400,2.548)--(2.403,2.548)--(2.406,2.548)--(2.409,2.548)--(2.412,2.549)--(2.415,2.549)%
  --(2.418,2.549)--(2.421,2.550)--(2.424,2.550)--(2.427,2.550)--(2.430,2.551)--(2.433,2.551)%
  --(2.436,2.551)--(2.439,2.552)--(2.442,2.552)--(2.445,2.552)--(2.447,2.553)--(2.450,2.553)%
  --(2.453,2.554)--(2.456,2.554)--(2.459,2.554)--(2.462,2.555)--(2.465,2.555)--(2.468,2.555)%
  --(2.471,2.556)--(2.474,2.556)--(2.477,2.556)--(2.480,2.557)--(2.483,2.557)--(2.486,2.557)%
  --(2.489,2.558)--(2.492,2.558)--(2.495,2.559)--(2.498,2.559)--(2.501,2.559)--(2.504,2.560)%
  --(2.507,2.560)--(2.510,2.560)--(2.513,2.561)--(2.516,2.561)--(2.519,2.562)--(2.522,2.562)%
  --(2.525,2.562)--(2.528,2.563)--(2.531,2.563)--(2.534,2.564)--(2.537,2.564)--(2.540,2.564)%
  --(2.543,2.565)--(2.546,2.565)--(2.549,2.566)--(2.552,2.566)--(2.555,2.566)--(2.558,2.567)%
  --(2.561,2.567)--(2.564,2.568)--(2.567,2.568)--(2.570,2.568)--(2.573,2.569)--(2.576,2.569)%
  --(2.579,2.570)--(2.582,2.570)--(2.585,2.571)--(2.588,2.571)--(2.591,2.571)--(2.594,2.572)%
  --(2.597,2.572)--(2.600,2.573)--(2.603,2.573)--(2.606,2.574)--(2.609,2.574)--(2.612,2.574)%
  --(2.615,2.575)--(2.618,2.575)--(2.621,2.576)--(2.624,2.576)--(2.627,2.577)--(2.630,2.577)%
  --(2.633,2.578)--(2.636,2.578)--(2.639,2.579)--(2.642,2.579)--(2.645,2.579)--(2.648,2.580)%
  --(2.651,2.580)--(2.654,2.581)--(2.656,2.581)--(2.659,2.582)--(2.662,2.582)--(2.665,2.583)%
  --(2.668,2.583)--(2.671,2.584)--(2.674,2.584)--(2.677,2.585)--(2.680,2.585)--(2.683,2.586)%
  --(2.686,2.586)--(2.689,2.587)--(2.692,2.587)--(2.695,2.588)--(2.698,2.588)--(2.701,2.589)%
  --(2.704,2.589)--(2.707,2.590)--(2.710,2.590)--(2.713,2.591)--(2.716,2.591)--(2.719,2.592)%
  --(2.722,2.592)--(2.725,2.593)--(2.728,2.593)--(2.731,2.594)--(2.734,2.594)--(2.737,2.595)%
  --(2.740,2.595)--(2.743,2.596)--(2.746,2.596)--(2.749,2.597)--(2.752,2.597)--(2.755,2.598)%
  --(2.758,2.598)--(2.761,2.599)--(2.764,2.599)--(2.767,2.600)--(2.770,2.600)--(2.773,2.601)%
  --(2.776,2.602)--(2.779,2.602)--(2.782,2.603)--(2.785,2.603)--(2.788,2.604)--(2.791,2.604)%
  --(2.794,2.605)--(2.797,2.605)--(2.800,2.606)--(2.803,2.607)--(2.806,2.607)--(2.809,2.608)%
  --(2.812,2.608)--(2.815,2.609)--(2.818,2.609)--(2.821,2.610)--(2.824,2.610)--(2.827,2.611)%
  --(2.830,2.612)--(2.833,2.612)--(2.836,2.613)--(2.839,2.613)--(2.842,2.614)--(2.845,2.614)%
  --(2.848,2.615)--(2.851,2.616)--(2.854,2.616)--(2.857,2.617)--(2.860,2.617)--(2.863,2.618)%
  --(2.866,2.619)--(2.868,2.619)--(2.871,2.620)--(2.874,2.620)--(2.877,2.621)--(2.880,2.622)%
  --(2.883,2.622)--(2.886,2.623)--(2.889,2.623)--(2.892,2.624)--(2.895,2.625)--(2.898,2.625)%
  --(2.901,2.626)--(2.904,2.626)--(2.907,2.627)--(2.910,2.628)--(2.913,2.628)--(2.916,2.629)%
  --(2.919,2.630)--(2.922,2.630)--(2.925,2.631)--(2.928,2.631)--(2.931,2.632)--(2.934,2.633)%
  --(2.937,2.633)--(2.940,2.634)--(2.943,2.635)--(2.946,2.635)--(2.949,2.636)--(2.952,2.637)%
  --(2.955,2.637)--(2.958,2.638)--(2.961,2.638)--(2.964,2.639)--(2.967,2.640)--(2.970,2.640)%
  --(2.973,2.641)--(2.976,2.642)--(2.979,2.642)--(2.982,2.643)--(2.985,2.644)--(2.988,2.644)%
  --(2.991,2.645)--(2.994,2.646)--(2.997,2.646)--(3.000,2.647)--(3.003,2.648)--(3.006,2.648)%
  --(3.009,2.649)--(3.012,2.650)--(3.015,2.650)--(3.018,2.651)--(3.021,2.652)--(3.024,2.652)%
  --(3.027,2.653)--(3.030,2.654)--(3.033,2.655)--(3.036,2.655)--(3.039,2.656)--(3.042,2.657)%
  --(3.045,2.657)--(3.048,2.658)--(3.051,2.659)--(3.054,2.659)--(3.057,2.660)--(3.060,2.661)%
  --(3.063,2.661)--(3.066,2.662)--(3.069,2.663)--(3.072,2.664)--(3.075,2.664)--(3.077,2.665)%
  --(3.080,2.666)--(3.083,2.666)--(3.086,2.667)--(3.089,2.668)--(3.092,2.669)--(3.095,2.669)%
  --(3.098,2.670)--(3.101,2.671)--(3.104,2.671)--(3.107,2.672)--(3.110,2.673)--(3.113,2.674)%
  --(3.116,2.674)--(3.119,2.675)--(3.122,2.676)--(3.125,2.677)--(3.128,2.677)--(3.131,2.678)%
  --(3.134,2.679)--(3.137,2.680)--(3.140,2.680)--(3.143,2.681)--(3.146,2.682)--(3.149,2.683)%
  --(3.152,2.683)--(3.155,2.684)--(3.158,2.685)--(3.161,2.686)--(3.164,2.686)--(3.167,2.687)%
  --(3.170,2.688)--(3.173,2.689)--(3.176,2.689)--(3.179,2.690)--(3.182,2.691)--(3.185,2.692)%
  --(3.188,2.692)--(3.191,2.693)--(3.194,2.694)--(3.197,2.695)--(3.200,2.695)--(3.203,2.696)%
  --(3.206,2.697)--(3.209,2.698)--(3.212,2.699)--(3.215,2.699)--(3.218,2.700)--(3.221,2.701)%
  --(3.224,2.702)--(3.227,2.703)--(3.230,2.703)--(3.233,2.704)--(3.236,2.705)--(3.239,2.706)%
  --(3.242,2.706)--(3.245,2.707)--(3.248,2.708)--(3.251,2.709)--(3.254,2.710)--(3.257,2.710)%
  --(3.260,2.711)--(3.263,2.712)--(3.266,2.713)--(3.269,2.714)--(3.272,2.715)--(3.275,2.715)%
  --(3.278,2.716)--(3.281,2.717)--(3.284,2.718)--(3.286,2.719)--(3.289,2.719)--(3.292,2.720)%
  --(3.295,2.721)--(3.298,2.722)--(3.301,2.723)--(3.304,2.724)--(3.307,2.724)--(3.310,2.725)%
  --(3.313,2.726)--(3.316,2.727)--(3.319,2.728)--(3.322,2.729)--(3.325,2.729)--(3.328,2.730)%
  --(3.331,2.731)--(3.334,2.732)--(3.337,2.733)--(3.340,2.734)--(3.343,2.734)--(3.346,2.735)%
  --(3.349,2.736)--(3.352,2.737)--(3.355,2.738)--(3.358,2.739)--(3.361,2.739)--(3.364,2.740)%
  --(3.367,2.741)--(3.370,2.742)--(3.373,2.743)--(3.376,2.744)--(3.379,2.745)--(3.382,2.745)%
  --(3.385,2.746)--(3.388,2.747)--(3.391,2.748)--(3.394,2.749)--(3.397,2.750)--(3.400,2.751)%
  --(3.403,2.752)--(3.406,2.752)--(3.409,2.753)--(3.412,2.754)--(3.415,2.755)--(3.418,2.756)%
  --(3.421,2.757)--(3.424,2.758)--(3.427,2.759)--(3.430,2.759)--(3.433,2.760)--(3.436,2.761)%
  --(3.439,2.762)--(3.442,2.763)--(3.445,2.764)--(3.448,2.765)--(3.451,2.766)--(3.454,2.767)%
  --(3.457,2.767)--(3.460,2.768)--(3.463,2.769)--(3.466,2.770)--(3.469,2.771)--(3.472,2.772)%
  --(3.475,2.773)--(3.478,2.774)--(3.481,2.775)--(3.484,2.776)--(3.487,2.776)--(3.490,2.777)%
  --(3.493,2.778)--(3.495,2.779)--(3.498,2.780)--(3.501,2.781)--(3.504,2.782)--(3.507,2.783)%
  --(3.510,2.784)--(3.513,2.785)--(3.516,2.786)--(3.519,2.787)--(3.522,2.787)--(3.525,2.788)%
  --(3.528,2.789)--(3.531,2.790)--(3.534,2.791)--(3.537,2.792)--(3.540,2.793)--(3.543,2.794)%
  --(3.546,2.795)--(3.549,2.796)--(3.552,2.797)--(3.555,2.798)--(3.558,2.799)--(3.561,2.800)%
  --(3.564,2.800)--(3.567,2.801)--(3.570,2.802)--(3.573,2.803)--(3.576,2.804)--(3.579,2.805)%
  --(3.582,2.806)--(3.585,2.807)--(3.588,2.808)--(3.591,2.809)--(3.594,2.810)--(3.597,2.811)%
  --(3.600,2.812)--(3.603,2.813)--(3.606,2.814)--(3.609,2.815)--(3.612,2.816)--(3.615,2.817)%
  --(3.618,2.818)--(3.621,2.819)--(3.624,2.820)--(3.627,2.821)--(3.630,2.821)--(3.633,2.822)%
  --(3.636,2.823)--(3.639,2.824)--(3.642,2.825)--(3.645,2.826)--(3.648,2.827)--(3.651,2.828)%
  --(3.654,2.829)--(3.657,2.830)--(3.660,2.831)--(3.663,2.832)--(3.666,2.833)--(3.669,2.834)%
  --(3.672,2.835)--(3.675,2.836)--(3.678,2.837)--(3.681,2.838)--(3.684,2.839)--(3.687,2.840)%
  --(3.690,2.841)--(3.693,2.842)--(3.696,2.843)--(3.699,2.844)--(3.702,2.845)--(3.704,2.846)%
  --(3.707,2.847)--(3.710,2.848)--(3.713,2.849)--(3.716,2.850)--(3.719,2.851)--(3.722,2.852)%
  --(3.725,2.853)--(3.728,2.854)--(3.731,2.855)--(3.734,2.856)--(3.737,2.857)--(3.740,2.858)%
  --(3.743,2.859)--(3.746,2.860)--(3.749,2.861)--(3.752,2.862)--(3.755,2.863)--(3.758,2.864)%
  --(3.761,2.865)--(3.764,2.866)--(3.767,2.867)--(3.770,2.868)--(3.773,2.869)--(3.776,2.870)%
  --(3.779,2.871)--(3.782,2.872)--(3.785,2.873)--(3.788,2.874)--(3.791,2.875)--(3.794,2.876)%
  --(3.797,2.877)--(3.800,2.878)--(3.803,2.879)--(3.806,2.880)--(3.809,2.881)--(3.812,2.882)%
  --(3.815,2.883)--(3.818,2.885)--(3.821,2.886)--(3.824,2.887)--(3.827,2.888)--(3.830,2.889)%
  --(3.833,2.890)--(3.836,2.891)--(3.839,2.892)--(3.842,2.893)--(3.845,2.894)--(3.848,2.895)%
  --(3.851,2.896)--(3.854,2.897)--(3.857,2.898)--(3.860,2.899)--(3.863,2.900)--(3.866,2.901)%
  --(3.869,2.902)--(3.872,2.903)--(3.875,2.904)--(3.878,2.905)--(3.881,2.906)--(3.884,2.908)%
  --(3.887,2.909)--(3.890,2.910)--(3.893,2.911)--(3.896,2.912)--(3.899,2.913)--(3.902,2.914)%
  --(3.905,2.915)--(3.908,2.916)--(3.911,2.917)--(3.914,2.918)--(3.916,2.919)--(3.919,2.920)%
  --(3.922,2.921)--(3.925,2.922)--(3.928,2.923)--(3.931,2.925)--(3.934,2.926)--(3.937,2.927)%
  --(3.940,2.928)--(3.943,2.929)--(3.946,2.930)--(3.949,2.931)--(3.952,2.932)--(3.955,2.933)%
  --(3.958,2.934)--(3.961,2.935)--(3.964,2.936)--(3.967,2.937)--(3.970,2.939)--(3.973,2.940)%
  --(3.976,2.941)--(3.979,2.942)--(3.982,2.943)--(3.985,2.944)--(3.988,2.945)--(3.991,2.946)%
  --(3.994,2.947)--(3.997,2.948)--(4.000,2.949)--(4.003,2.951)--(4.006,2.952)--(4.009,2.953)%
  --(4.012,2.954)--(4.015,2.955)--(4.018,2.956)--(4.021,2.957)--(4.024,2.958)--(4.027,2.959)%
  --(4.030,2.960)--(4.033,2.961)--(4.036,2.963)--(4.039,2.964)--(4.042,2.965)--(4.045,2.966)%
  --(4.048,2.967)--(4.051,2.968)--(4.054,2.969)--(4.057,2.970)--(4.060,2.971)--(4.063,2.973)%
  --(4.066,2.974)--(4.069,2.975)--(4.072,2.976)--(4.075,2.977)--(4.078,2.978)--(4.081,2.979)%
  --(4.084,2.980)--(4.087,2.981)--(4.090,2.983)--(4.093,2.984)--(4.096,2.985)--(4.099,2.986)%
  --(4.102,2.987)--(4.105,2.988)--(4.108,2.989)--(4.111,2.990)--(4.114,2.992)--(4.117,2.993)%
  --(4.120,2.994)--(4.123,2.995)--(4.125,2.996)--(4.128,2.997)--(4.131,2.998)--(4.134,2.999)%
  --(4.137,3.001)--(4.140,3.002)--(4.143,3.003)--(4.146,3.004)--(4.149,3.005)--(4.152,3.006)%
  --(4.155,3.007)--(4.158,3.009)--(4.161,3.010)--(4.164,3.011)--(4.167,3.012)--(4.170,3.013)%
  --(4.173,3.014)--(4.176,3.015)--(4.179,3.016)--(4.182,3.018)--(4.185,3.019)--(4.188,3.020)%
  --(4.191,3.021)--(4.194,3.022)--(4.197,3.023)--(4.200,3.025)--(4.203,3.026)--(4.206,3.027)%
  --(4.209,3.028)--(4.212,3.029)--(4.215,3.030)--(4.218,3.031)--(4.221,3.033)--(4.224,3.034)%
  --(4.227,3.035)--(4.230,3.036)--(4.233,3.037)--(4.236,3.038)--(4.239,3.039)--(4.242,3.041)%
  --(4.245,3.042)--(4.248,3.043)--(4.251,3.044)--(4.254,3.045)--(4.257,3.046)--(4.260,3.048)%
  --(4.263,3.049)--(4.266,3.050)--(4.269,3.051)--(4.272,3.052)--(4.275,3.053)--(4.278,3.055)%
  --(4.281,3.056)--(4.284,3.057)--(4.287,3.058)--(4.290,3.059)--(4.293,3.060)--(4.296,3.062)%
  --(4.299,3.063)--(4.302,3.064)--(4.305,3.065)--(4.308,3.066)--(4.311,3.067)--(4.314,3.069)%
  --(4.317,3.070)--(4.320,3.071)--(4.323,3.072)--(4.326,3.073)--(4.329,3.075)--(4.332,3.076)%
  --(4.334,3.077)--(4.337,3.078)--(4.340,3.079)--(4.343,3.080)--(4.346,3.082)--(4.349,3.083)%
  --(4.352,3.084)--(4.355,3.085)--(4.358,3.086)--(4.361,3.088)--(4.364,3.089)--(4.367,3.090)%
  --(4.370,3.091)--(4.373,3.092)--(4.376,3.093)--(4.379,3.095)--(4.382,3.096)--(4.385,3.097)%
  --(4.388,3.098)--(4.391,3.099)--(4.394,3.101)--(4.397,3.102)--(4.400,3.103)--(4.403,3.104)%
  --(4.406,3.105)--(4.409,3.107)--(4.412,3.108)--(4.415,3.109)--(4.418,3.110)--(4.421,3.111)%
  --(4.424,3.113)--(4.427,3.114)--(4.430,3.115)--(4.433,3.116)--(4.436,3.117)--(4.439,3.119)%
  --(4.442,3.120)--(4.445,3.121)--(4.448,3.122)--(4.451,3.123)--(4.454,3.125)--(4.457,3.126)%
  --(4.460,3.127)--(4.463,3.128)--(4.466,3.129)--(4.469,3.131)--(4.472,3.132)--(4.475,3.133)%
  --(4.478,3.134)--(4.481,3.136)--(4.484,3.137)--(4.487,3.138)--(4.490,3.139)--(4.493,3.140)%
  --(4.496,3.142)--(4.499,3.143)--(4.502,3.144)--(4.505,3.145)--(4.508,3.146)--(4.511,3.148)%
  --(4.514,3.149)--(4.517,3.150)--(4.520,3.151)--(4.523,3.153)--(4.526,3.154)--(4.529,3.155)%
  --(4.532,3.156)--(4.535,3.157)--(4.538,3.159)--(4.541,3.160)--(4.543,3.161)--(4.546,3.162)%
  --(4.549,3.164)--(4.552,3.165)--(4.555,3.166)--(4.558,3.167)--(4.561,3.168)--(4.564,3.170)%
  --(4.567,3.171)--(4.570,3.172)--(4.573,3.173)--(4.576,3.175)--(4.579,3.176)--(4.582,3.177)%
  --(4.585,3.178)--(4.588,3.180)--(4.591,3.181)--(4.594,3.182)--(4.597,3.183)--(4.600,3.184)%
  --(4.603,3.186)--(4.606,3.187)--(4.609,3.188)--(4.612,3.189)--(4.615,3.191)--(4.618,3.192)%
  --(4.621,3.193)--(4.624,3.194)--(4.627,3.196)--(4.630,3.197)--(4.633,3.198)--(4.636,3.199)%
  --(4.639,3.201)--(4.642,3.202)--(4.645,3.203)--(4.648,3.204)--(4.651,3.206)--(4.654,3.207)%
  --(4.657,3.208)--(4.660,3.209)--(4.663,3.211)--(4.666,3.212)--(4.669,3.213)--(4.672,3.214)%
  --(4.675,3.216)--(4.678,3.217)--(4.681,3.218)--(4.684,3.219)--(4.687,3.221)--(4.690,3.222)%
  --(4.693,3.223)--(4.696,3.224)--(4.699,3.226)--(4.702,3.227)--(4.705,3.228)--(4.708,3.229)%
  --(4.711,3.231)--(4.714,3.232)--(4.717,3.233)--(4.720,3.234)--(4.723,3.236)--(4.726,3.237)%
  --(4.729,3.238)--(4.732,3.239)--(4.735,3.241)--(4.738,3.242)--(4.741,3.243)--(4.744,3.244)%
  --(4.747,3.246)--(4.750,3.247)--(4.752,3.248)--(4.755,3.250)--(4.758,3.251)--(4.761,3.252)%
  --(4.764,3.253)--(4.767,3.255)--(4.770,3.256)--(4.773,3.257)--(4.776,3.258)--(4.779,3.260)%
  --(4.782,3.261)--(4.785,3.262)--(4.788,3.263)--(4.791,3.265)--(4.794,3.266)--(4.797,3.267)%
  --(4.800,3.269)--(4.803,3.270)--(4.806,3.271)--(4.809,3.272)--(4.812,3.274)--(4.815,3.275)%
  --(4.818,3.276)--(4.821,3.277)--(4.824,3.279)--(4.827,3.280)--(4.830,3.281)--(4.833,3.283)%
  --(4.836,3.284)--(4.839,3.285)--(4.842,3.286)--(4.845,3.288)--(4.848,3.289)--(4.851,3.290)%
  --(4.854,3.291)--(4.857,3.293)--(4.860,3.294)--(4.863,3.295)--(4.866,3.297)--(4.869,3.298)%
  --(4.872,3.299)--(4.875,3.300)--(4.878,3.302)--(4.881,3.303)--(4.884,3.304)--(4.887,3.306)%
  --(4.890,3.307)--(4.893,3.308)--(4.896,3.309)--(4.899,3.311)--(4.902,3.312)--(4.905,3.313)%
  --(4.908,3.315)--(4.911,3.316)--(4.914,3.317)--(4.917,3.318)--(4.920,3.320)--(4.923,3.321)%
  --(4.926,3.322)--(4.929,3.324)--(4.932,3.325)--(4.935,3.326)--(4.938,3.327)--(4.941,3.329)%
  --(4.944,3.330)--(4.947,3.331)--(4.950,3.333)--(4.953,3.334)--(4.956,3.335)--(4.959,3.337)%
  --(4.961,3.338)--(4.964,3.339)--(4.967,3.340)--(4.970,3.342)--(4.973,3.343)--(4.976,3.344)%
  --(4.979,3.346)--(4.982,3.347)--(4.985,3.348)--(4.988,3.350)--(4.991,3.351)--(4.994,3.352)%
  --(4.997,3.353)--(5.000,3.355)--(5.003,3.356)--(5.006,3.357)--(5.009,3.359)--(5.012,3.360)%
  --(5.015,3.361)--(5.018,3.363)--(5.021,3.364)--(5.024,3.365)--(5.027,3.366)--(5.030,3.368)%
  --(5.033,3.369)--(5.036,3.370)--(5.039,3.372)--(5.042,3.373)--(5.045,3.374)--(5.048,3.376)%
  --(5.051,3.377)--(5.054,3.378)--(5.057,3.380)--(5.060,3.381)--(5.063,3.382)--(5.066,3.383)%
  --(5.069,3.385)--(5.072,3.386)--(5.075,3.387)--(5.078,3.389)--(5.081,3.390)--(5.084,3.391)%
  --(5.087,3.393)--(5.090,3.394)--(5.093,3.395)--(5.096,3.397)--(5.099,3.398)--(5.102,3.399)%
  --(5.105,3.401)--(5.108,3.402)--(5.111,3.403)--(5.114,3.404)--(5.117,3.406)--(5.120,3.407)%
  --(5.123,3.408)--(5.126,3.410)--(5.129,3.411)--(5.132,3.412)--(5.135,3.414)--(5.138,3.415)%
  --(5.141,3.416)--(5.144,3.418)--(5.147,3.419)--(5.150,3.420)--(5.153,3.422)--(5.156,3.423)%
  --(5.159,3.424)--(5.162,3.426)--(5.165,3.427)--(5.168,3.428)--(5.171,3.430)--(5.173,3.431)%
  --(5.176,3.432)--(5.179,3.434)--(5.182,3.435)--(5.185,3.436)--(5.188,3.438)--(5.191,3.439)%
  --(5.194,3.440)--(5.197,3.441)--(5.200,3.443)--(5.203,3.444)--(5.206,3.445)--(5.209,3.447)%
  --(5.212,3.448)--(5.215,3.449)--(5.218,3.451)--(5.221,3.452)--(5.224,3.453)--(5.227,3.455)%
  --(5.230,3.456)--(5.233,3.457)--(5.236,3.459)--(5.239,3.460)--(5.242,3.461)--(5.245,3.463)%
  --(5.248,3.464)--(5.251,3.465)--(5.254,3.467)--(5.257,3.468)--(5.260,3.469)--(5.263,3.471)%
  --(5.266,3.472)--(5.269,3.473)--(5.272,3.475)--(5.275,3.476)--(5.278,3.477)--(5.281,3.479)%
  --(5.284,3.480)--(5.287,3.481)--(5.290,3.483)--(5.293,3.484)--(5.296,3.485)--(5.299,3.487)%
  --(5.302,3.488)--(5.305,3.490)--(5.308,3.491)--(5.311,3.492)--(5.314,3.494)--(5.317,3.495)%
  --(5.320,3.496)--(5.323,3.498)--(5.326,3.499)--(5.329,3.500)--(5.332,3.502)--(5.335,3.503)%
  --(5.338,3.504)--(5.341,3.506)--(5.344,3.507)--(5.347,3.508)--(5.350,3.510)--(5.353,3.511)%
  --(5.356,3.512)--(5.359,3.514)--(5.362,3.515)--(5.365,3.516)--(5.368,3.518)--(5.371,3.519)%
  --(5.374,3.520)--(5.377,3.522)--(5.380,3.523)--(5.382,3.524)--(5.385,3.526)--(5.388,3.527)%
  --(5.391,3.528)--(5.394,3.530)--(5.397,3.531)--(5.400,3.533)--(5.403,3.534)--(5.406,3.535)%
  --(5.409,3.537)--(5.412,3.538)--(5.415,3.539)--(5.418,3.541)--(5.421,3.542)--(5.424,3.543)%
  --(5.427,3.545)--(5.430,3.546)--(5.433,3.547)--(5.436,3.549)--(5.439,3.550)--(5.442,3.551)%
  --(5.445,3.553)--(5.448,3.554)--(5.451,3.556)--(5.454,3.557)--(5.457,3.558)--(5.460,3.560)%
  --(5.463,3.561)--(5.466,3.562)--(5.469,3.564)--(5.472,3.565)--(5.475,3.566)--(5.478,3.568)%
  --(5.481,3.569)--(5.484,3.570)--(5.487,3.572)--(5.490,3.573)--(5.493,3.575)--(5.496,3.576)%
  --(5.499,3.577)--(5.502,3.579)--(5.505,3.580)--(5.508,3.581)--(5.511,3.583)--(5.514,3.584)%
  --(5.517,3.585)--(5.520,3.587)--(5.523,3.588)--(5.526,3.590)--(5.529,3.591)--(5.532,3.592)%
  --(5.535,3.594)--(5.538,3.595)--(5.541,3.596)--(5.544,3.598)--(5.547,3.599)--(5.550,3.600)%
  --(5.553,3.602)--(5.556,3.603)--(5.559,3.605)--(5.562,3.606)--(5.565,3.607)--(5.568,3.609)%
  --(5.571,3.610)--(5.574,3.611)--(5.577,3.613)--(5.580,3.614)--(5.583,3.615)--(5.586,3.617)%
  --(5.589,3.618)--(5.591,3.620)--(5.594,3.621)--(5.597,3.622)--(5.600,3.624)--(5.603,3.625)%
  --(5.606,3.626)--(5.609,3.628)--(5.612,3.629)--(5.615,3.631)--(5.618,3.632)--(5.621,3.633)%
  --(5.624,3.635)--(5.627,3.636)--(5.630,3.637)--(5.633,3.639)--(5.636,3.640)--(5.639,3.642)%
  --(5.642,3.643)--(5.645,3.644)--(5.648,3.646)--(5.651,3.647)--(5.654,3.648)--(5.657,3.650)%
  --(5.660,3.651)--(5.663,3.652)--(5.666,3.654)--(5.669,3.655)--(5.672,3.657)--(5.675,3.658)%
  --(5.678,3.659)--(5.681,3.661)--(5.684,3.662)--(5.687,3.664)--(5.690,3.665)--(5.693,3.666)%
  --(5.696,3.668)--(5.699,3.669)--(5.702,3.670)--(5.705,3.672)--(5.708,3.673)--(5.711,3.675)%
  --(5.714,3.676)--(5.717,3.677)--(5.720,3.679)--(5.723,3.680)--(5.726,3.681)--(5.729,3.683)%
  --(5.732,3.684)--(5.735,3.686)--(5.738,3.687)--(5.741,3.688)--(5.744,3.690)--(5.747,3.691)%
  --(5.750,3.692)--(5.753,3.694)--(5.756,3.695)--(5.759,3.697)--(5.762,3.698)--(5.765,3.699)%
  --(5.768,3.701)--(5.771,3.702)--(5.774,3.704)--(5.777,3.705)--(5.780,3.706)--(5.783,3.708)%
  --(5.786,3.709)--(5.789,3.710)--(5.792,3.712)--(5.795,3.713)--(5.798,3.715)--(5.800,3.716)%
  --(5.803,3.717)--(5.806,3.719)--(5.809,3.720)--(5.812,3.722)--(5.815,3.723)--(5.818,3.724)%
  --(5.821,3.726)--(5.824,3.727)--(5.827,3.729)--(5.830,3.730)--(5.833,3.731)--(5.836,3.733)%
  --(5.839,3.734)--(5.842,3.735)--(5.845,3.737)--(5.848,3.738)--(5.851,3.740)--(5.854,3.741)%
  --(5.857,3.742)--(5.860,3.744)--(5.863,3.745)--(5.866,3.747)--(5.869,3.748)--(5.872,3.749)%
  --(5.875,3.751)--(5.878,3.752)--(5.881,3.754)--(5.884,3.755)--(5.887,3.756)--(5.890,3.758)%
  --(5.893,3.759)--(5.896,3.761)--(5.899,3.762)--(5.902,3.763)--(5.905,3.765)--(5.908,3.766)%
  --(5.911,3.767)--(5.914,3.769)--(5.917,3.770)--(5.920,3.772)--(5.923,3.773)--(5.926,3.774)%
  --(5.929,3.776)--(5.932,3.777)--(5.935,3.779)--(5.938,3.780)--(5.941,3.781)--(5.944,3.783)%
  --(5.947,3.784)--(5.950,3.786)--(5.953,3.787)--(5.956,3.788)--(5.959,3.790)--(5.962,3.791)%
  --(5.965,3.793)--(5.968,3.794)--(5.971,3.795)--(5.974,3.797)--(5.977,3.798)--(5.980,3.800)%
  --(5.983,3.801)--(5.986,3.802)--(5.989,3.804)--(5.992,3.805)--(5.995,3.807)--(5.998,3.808)%
  --(6.001,3.809)--(6.004,3.811)--(6.007,3.812)--(6.009,3.814)--(6.012,3.815)--(6.015,3.816)%
  --(6.018,3.818)--(6.021,3.819)--(6.024,3.821)--(6.027,3.822)--(6.030,3.823)--(6.033,3.825)%
  --(6.036,3.826)--(6.039,3.828)--(6.042,3.829)--(6.045,3.830)--(6.048,3.832)--(6.051,3.833)%
  --(6.054,3.835)--(6.057,3.836)--(6.060,3.838)--(6.063,3.839)--(6.066,3.840)--(6.069,3.842)%
  --(6.072,3.843)--(6.075,3.845)--(6.078,3.846)--(6.081,3.847)--(6.084,3.849)--(6.087,3.850)%
  --(6.090,3.852)--(6.093,3.853)--(6.096,3.854)--(6.099,3.856)--(6.102,3.857)--(6.105,3.859)%
  --(6.108,3.860)--(6.111,3.861)--(6.114,3.863)--(6.117,3.864)--(6.120,3.866)--(6.123,3.867)%
  --(6.126,3.868)--(6.129,3.870)--(6.132,3.871)--(6.135,3.873)--(6.138,3.874)--(6.141,3.876)%
  --(6.144,3.877)--(6.147,3.878)--(6.150,3.880)--(6.153,3.881)--(6.156,3.883)--(6.159,3.884)%
  --(6.162,3.885)--(6.165,3.887)--(6.168,3.888)--(6.171,3.890)--(6.174,3.891)--(6.177,3.892)%
  --(6.180,3.894)--(6.183,3.895)--(6.186,3.897)--(6.189,3.898)--(6.192,3.900)--(6.195,3.901)%
  --(6.198,3.902)--(6.201,3.904)--(6.204,3.905)--(6.207,3.907)--(6.210,3.908)--(6.213,3.909)%
  --(6.216,3.911)--(6.218,3.912)--(6.221,3.914)--(6.224,3.915)--(6.227,3.917)--(6.230,3.918)%
  --(6.233,3.919)--(6.236,3.921)--(6.239,3.922)--(6.242,3.924)--(6.245,3.925)--(6.248,3.926)%
  --(6.251,3.928)--(6.254,3.929)--(6.257,3.931)--(6.260,3.932)--(6.263,3.934)--(6.266,3.935)%
  --(6.269,3.936)--(6.272,3.938)--(6.275,3.939)--(6.278,3.941)--(6.281,3.942)--(6.284,3.943)%
  --(6.287,3.945)--(6.290,3.946)--(6.293,3.948)--(6.296,3.949)--(6.299,3.951)--(6.302,3.952)%
  --(6.305,3.953)--(6.308,3.955)--(6.311,3.956)--(6.314,3.958)--(6.317,3.959)--(6.320,3.960)%
  --(6.323,3.962)--(6.326,3.963)--(6.329,3.965)--(6.332,3.966)--(6.335,3.968)--(6.338,3.969)%
  --(6.341,3.970)--(6.344,3.972)--(6.347,3.973)--(6.350,3.975)--(6.353,3.976)--(6.356,3.978)%
  --(6.359,3.979)--(6.362,3.980)--(6.365,3.982)--(6.368,3.983)--(6.371,3.985)--(6.374,3.986)%
  --(6.377,3.988)--(6.380,3.989)--(6.383,3.990)--(6.386,3.992)--(6.389,3.993)--(6.392,3.995)%
  --(6.395,3.996)--(6.398,3.997)--(6.401,3.999)--(6.404,4.000)--(6.407,4.002)--(6.410,4.003)%
  --(6.413,4.005)--(6.416,4.006)--(6.419,4.007)--(6.422,4.009)--(6.425,4.010)--(6.428,4.012)%
  --(6.430,4.013)--(6.433,4.015)--(6.436,4.016)--(6.439,4.017)--(6.442,4.019)--(6.445,4.020)%
  --(6.448,4.022)--(6.451,4.023)--(6.454,4.025)--(6.457,4.026)--(6.460,4.027)--(6.463,4.029)%
  --(6.466,4.030)--(6.469,4.032)--(6.472,4.033)--(6.475,4.035)--(6.478,4.036)--(6.481,4.037)%
  --(6.484,4.039)--(6.487,4.040)--(6.490,4.042)--(6.493,4.043)--(6.496,4.045)--(6.499,4.046)%
  --(6.502,4.047)--(6.505,4.049)--(6.508,4.050)--(6.511,4.052)--(6.514,4.053)--(6.517,4.055)%
  --(6.520,4.056)--(6.523,4.058)--(6.526,4.059)--(6.529,4.060)--(6.532,4.062)--(6.535,4.063)%
  --(6.538,4.065)--(6.541,4.066)--(6.544,4.068)--(6.547,4.069)--(6.550,4.070)--(6.553,4.072)%
  --(6.556,4.073)--(6.559,4.075)--(6.562,4.076)--(6.565,4.078)--(6.568,4.079)--(6.571,4.080)%
  --(6.574,4.082)--(6.577,4.083)--(6.580,4.085)--(6.583,4.086)--(6.586,4.088)--(6.589,4.089)%
  --(6.592,4.090)--(6.595,4.092)--(6.598,4.093)--(6.601,4.095)--(6.604,4.096)--(6.607,4.098)%
  --(6.610,4.099)--(6.613,4.101)--(6.616,4.102)--(6.619,4.103)--(6.622,4.105)--(6.625,4.106)%
  --(6.628,4.108)--(6.631,4.109)--(6.634,4.111)--(6.637,4.112)--(6.639,4.113)--(6.642,4.115)%
  --(6.645,4.116)--(6.648,4.118)--(6.651,4.119)--(6.654,4.121)--(6.657,4.122)--(6.660,4.124)%
  --(6.663,4.125)--(6.666,4.126)--(6.669,4.128)--(6.672,4.129)--(6.675,4.131)--(6.678,4.132)%
  --(6.681,4.134)--(6.684,4.135)--(6.687,4.136)--(6.690,4.138)--(6.693,4.139)--(6.696,4.141)%
  --(6.699,4.142)--(6.702,4.144)--(6.705,4.145)--(6.708,4.147)--(6.711,4.148)--(6.714,4.149)%
  --(6.717,4.151)--(6.720,4.152)--(6.723,4.154)--(6.726,4.155)--(6.729,4.157)--(6.732,4.158)%
  --(6.735,4.160)--(6.738,4.161)--(6.741,4.162)--(6.744,4.164)--(6.747,4.165)--(6.750,4.167)%
  --(6.753,4.168)--(6.756,4.170)--(6.759,4.171)--(6.762,4.172)--(6.765,4.174)--(6.768,4.175)%
  --(6.771,4.177)--(6.774,4.178)--(6.777,4.180)--(6.780,4.181)--(6.783,4.183)--(6.786,4.184)%
  --(6.789,4.185)--(6.792,4.187)--(6.795,4.188)--(6.798,4.190)--(6.801,4.191)--(6.804,4.193)%
  --(6.807,4.194)--(6.810,4.196)--(6.813,4.197)--(6.816,4.198)--(6.819,4.200)--(6.822,4.201)%
  --(6.825,4.203)--(6.828,4.204)--(6.831,4.206)--(6.834,4.207)--(6.837,4.209)--(6.840,4.210)%
  --(6.843,4.211)--(6.846,4.213)--(6.848,4.214)--(6.851,4.216)--(6.854,4.217)--(6.857,4.219)%
  --(6.860,4.220)--(6.863,4.222)--(6.866,4.223)--(6.869,4.225)--(6.872,4.226)--(6.875,4.227)%
  --(6.878,4.229)--(6.881,4.230)--(6.884,4.232)--(6.887,4.233)--(6.890,4.235)--(6.893,4.236)%
  --(6.896,4.238)--(6.899,4.239)--(6.902,4.240)--(6.905,4.242)--(6.908,4.243)--(6.911,4.245)%
  --(6.914,4.246)--(6.917,4.248)--(6.920,4.249)--(6.923,4.251)--(6.926,4.252)--(6.929,4.253)%
  --(6.932,4.255)--(6.935,4.256)--(6.938,4.258)--(6.941,4.259)--(6.944,4.261)--(6.947,4.262)%
  --(6.950,4.264)--(6.953,4.265)--(6.956,4.267)--(6.959,4.268)--(6.962,4.269)--(6.965,4.271)%
  --(6.968,4.272)--(6.971,4.274)--(6.974,4.275)--(6.977,4.277)--(6.980,4.278)--(6.983,4.280)%
  --(6.986,4.281)--(6.989,4.282)--(6.992,4.284)--(6.995,4.285)--(6.998,4.287)--(7.001,4.288)%
  --(7.004,4.290)--(7.007,4.291)--(7.010,4.293)--(7.013,4.294)--(7.016,4.296)--(7.019,4.297)%
  --(7.022,4.298)--(7.025,4.300)--(7.028,4.301)--(7.031,4.303)--(7.034,4.304)--(7.037,4.306)%
  --(7.040,4.307)--(7.043,4.309)--(7.046,4.310)--(7.049,4.312)--(7.052,4.313)--(7.055,4.314)%
  --(7.057,4.316)--(7.060,4.317)--(7.063,4.319)--(7.066,4.320)--(7.069,4.322)--(7.072,4.323)%
  --(7.075,4.325)--(7.078,4.326)--(7.081,4.328)--(7.084,4.329)--(7.087,4.330)--(7.090,4.332)%
  --(7.093,4.333)--(7.096,4.335)--(7.099,4.336)--(7.102,4.338)--(7.105,4.339)--(7.108,4.341)%
  --(7.111,4.342)--(7.114,4.344)--(7.117,4.345)--(7.120,4.346)--(7.123,4.348)--(7.126,4.349)%
  --(7.129,4.351)--(7.132,4.352)--(7.135,4.354)--(7.138,4.355)--(7.141,4.357)--(7.144,4.358)%
  --(7.147,4.360)--(7.150,4.361)--(7.153,4.362)--(7.156,4.364)--(7.159,4.365)--(7.162,4.367)%
  --(7.165,4.368)--(7.168,4.370)--(7.171,4.371)--(7.174,4.373)--(7.177,4.374)--(7.180,4.376)%
  --(7.183,4.377)--(7.186,4.379)--(7.189,4.380)--(7.192,4.381)--(7.195,4.383)--(7.198,4.384)%
  --(7.201,4.386)--(7.204,4.387)--(7.207,4.389)--(7.210,4.390)--(7.213,4.392)--(7.216,4.393)%
  --(7.219,4.395)--(7.222,4.396)--(7.225,4.397)--(7.228,4.399)--(7.231,4.400)--(7.234,4.402)%
  --(7.237,4.403)--(7.240,4.405)--(7.243,4.406)--(7.246,4.408)--(7.249,4.409)--(7.252,4.411)%
  --(7.255,4.412)--(7.258,4.414)--(7.261,4.415)--(7.264,4.416)--(7.266,4.418)--(7.269,4.419)%
  --(7.272,4.421)--(7.275,4.422)--(7.278,4.424)--(7.281,4.425)--(7.284,4.427)--(7.287,4.428)%
  --(7.290,4.430)--(7.293,4.431)--(7.296,4.433)--(7.299,4.434)--(7.302,4.435)--(7.305,4.437)%
  --(7.308,4.438)--(7.311,4.440)--(7.314,4.441)--(7.317,4.443)--(7.320,4.444)--(7.323,4.446)%
  --(7.326,4.447)--(7.329,4.449)--(7.332,4.450)--(7.335,4.452)--(7.338,4.453)--(7.341,4.454)%
  --(7.344,4.456)--(7.347,4.457)--(7.350,4.459)--(7.353,4.460)--(7.356,4.462)--(7.359,4.463)%
  --(7.362,4.465)--(7.365,4.466)--(7.368,4.468)--(7.371,4.469)--(7.374,4.471)--(7.377,4.472)%
  --(7.380,4.474)--(7.383,4.475)--(7.386,4.476)--(7.389,4.478)--(7.392,4.479)--(7.395,4.481)%
  --(7.398,4.482)--(7.401,4.484)--(7.404,4.485)--(7.407,4.487)--(7.410,4.488)--(7.413,4.490)%
  --(7.416,4.491)--(7.419,4.493)--(7.422,4.494)--(7.425,4.495)--(7.428,4.497)--(7.431,4.498)%
  --(7.434,4.500)--(7.437,4.501)--(7.440,4.503)--(7.443,4.504)--(7.446,4.506)--(7.449,4.507)%
  --(7.452,4.509)--(7.455,4.510)--(7.458,4.512)--(7.461,4.513)--(7.464,4.515)--(7.467,4.516)%
  --(7.470,4.517)--(7.473,4.519)--(7.476,4.520)--(7.478,4.522)--(7.481,4.523)--(7.484,4.525)%
  --(7.487,4.526)--(7.490,4.528)--(7.493,4.529)--(7.496,4.531)--(7.499,4.532)--(7.502,4.534)%
  --(7.505,4.535)--(7.508,4.537)--(7.511,4.538)--(7.514,4.539)--(7.517,4.541)--(7.520,4.542)%
  --(7.523,4.544)--(7.526,4.545)--(7.529,4.547)--(7.532,4.548)--(7.535,4.550)--(7.538,4.551)%
  --(7.541,4.553)--(7.544,4.554)--(7.547,4.556)--(7.550,4.557)--(7.553,4.559)--(7.556,4.560)%
  --(7.559,4.562)--(7.562,4.563)--(7.565,4.564)--(7.568,4.566)--(7.571,4.567)--(7.574,4.569)%
  --(7.577,4.570)--(7.580,4.572)--(7.583,4.573)--(7.586,4.575)--(7.589,4.576)--(7.592,4.578)%
  --(7.595,4.579)--(7.598,4.581)--(7.601,4.582)--(7.604,4.584)--(7.607,4.585)--(7.610,4.587)%
  --(7.613,4.588)--(7.616,4.589)--(7.619,4.591)--(7.622,4.592)--(7.625,4.594)--(7.628,4.595)%
  --(7.631,4.597)--(7.634,4.598)--(7.637,4.600)--(7.640,4.601)--(7.643,4.603)--(7.646,4.604)%
  --(7.649,4.606)--(7.652,4.607)--(7.655,4.609)--(7.658,4.610)--(7.661,4.612)--(7.664,4.613)%
  --(7.667,4.614)--(7.670,4.616)--(7.673,4.617)--(7.676,4.619)--(7.679,4.620)--(7.682,4.622)%
  --(7.685,4.623)--(7.687,4.625)--(7.690,4.626)--(7.693,4.628)--(7.696,4.629)--(7.699,4.631)%
  --(7.702,4.632)--(7.705,4.634)--(7.708,4.635)--(7.711,4.637)--(7.714,4.638)--(7.717,4.640)%
  --(7.720,4.641)--(7.723,4.642)--(7.726,4.644)--(7.729,4.645)--(7.732,4.647)--(7.735,4.648)%
  --(7.738,4.650)--(7.741,4.651)--(7.744,4.653)--(7.747,4.654)--(7.750,4.656)--(7.753,4.657)%
  --(7.756,4.659)--(7.759,4.660)--(7.762,4.662)--(7.765,4.663)--(7.768,4.665)--(7.771,4.666)%
  --(7.774,4.668)--(7.777,4.669)--(7.780,4.670)--(7.783,4.672)--(7.786,4.673)--(7.789,4.675)%
  --(7.792,4.676)--(7.795,4.678)--(7.798,4.679)--(7.801,4.681)--(7.804,4.682)--(7.807,4.684)%
  --(7.810,4.685)--(7.813,4.687)--(7.816,4.688)--(7.819,4.690)--(7.822,4.691)--(7.825,4.693)%
  --(7.828,4.694)--(7.831,4.696)--(7.834,4.697)--(7.837,4.698)--(7.840,4.700)--(7.843,4.701)%
  --(7.846,4.703)--(7.849,4.704)--(7.852,4.706)--(7.855,4.707)--(7.858,4.709)--(7.861,4.710)%
  --(7.864,4.712)--(7.867,4.713)--(7.870,4.715)--(7.873,4.716)--(7.876,4.718)--(7.879,4.719)%
  --(7.882,4.721)--(7.885,4.722)--(7.888,4.724)--(7.891,4.725)--(7.894,4.727)--(7.896,4.728)%
  --(7.899,4.730)--(7.902,4.731)--(7.905,4.732)--(7.908,4.734)--(7.911,4.735)--(7.914,4.737)%
  --(7.917,4.738)--(7.920,4.740)--(7.923,4.741)--(7.926,4.743)--(7.929,4.744)--(7.932,4.746)%
  --(7.935,4.747)--(7.938,4.749)--(7.941,4.750)--(7.944,4.752)--(7.947,4.753)--(7.950,4.755)%
  --(7.953,4.756)--(7.956,4.758)--(7.959,4.759)--(7.962,4.761)--(7.965,4.762)--(7.968,4.764)%
  --(7.971,4.765)--(7.974,4.766)--(7.977,4.768)--(7.980,4.769)--(7.983,4.771)--(7.986,4.772)%
  --(7.989,4.774)--(7.992,4.775)--(7.995,4.777)--(7.998,4.778)--(8.001,4.780)--(8.004,4.781)%
  --(8.007,4.783)--(8.010,4.784)--(8.013,4.786)--(8.016,4.787)--(8.019,4.789)--(8.022,4.790)%
  --(8.025,4.792)--(8.028,4.793)--(8.031,4.795)--(8.034,4.796)--(8.037,4.798)--(8.040,4.799)%
  --(8.043,4.800)--(8.046,4.802)--(8.049,4.803)--(8.052,4.805)--(8.055,4.806)--(8.058,4.808)%
  --(8.061,4.809)--(8.064,4.811)--(8.067,4.812)--(8.070,4.814)--(8.073,4.815)--(8.076,4.817)%
  --(8.079,4.818)--(8.082,4.820)--(8.085,4.821)--(8.088,4.823)--(8.091,4.824)--(8.094,4.826)%
  --(8.097,4.827)--(8.100,4.829)--(8.103,4.830)--(8.105,4.832)--(8.108,4.833)--(8.111,4.835)%
  --(8.114,4.836)--(8.117,4.838)--(8.120,4.839)--(8.123,4.840)--(8.126,4.842)--(8.129,4.843)%
  --(8.132,4.845)--(8.135,4.846)--(8.138,4.848)--(8.141,4.849)--(8.144,4.851)--(8.147,4.852)%
  --(8.150,4.854)--(8.153,4.855)--(8.156,4.857)--(8.159,4.858)--(8.162,4.860)--(8.165,4.861)%
  --(8.168,4.863)--(8.171,4.864)--(8.174,4.866)--(8.177,4.867)--(8.180,4.869)--(8.183,4.870)%
  --(8.186,4.872)--(8.189,4.873)--(8.192,4.875)--(8.195,4.876)--(8.198,4.878)--(8.201,4.879)%
  --(8.204,4.881)--(8.207,4.882)--(8.210,4.883)--(8.213,4.885)--(8.216,4.886)--(8.219,4.888)%
  --(8.222,4.889)--(8.225,4.891)--(8.228,4.892)--(8.231,4.894)--(8.234,4.895)--(8.237,4.897)%
  --(8.240,4.898)--(8.243,4.900)--(8.246,4.901)--(8.249,4.903)--(8.252,4.904)--(8.255,4.906)%
  --(8.258,4.907)--(8.261,4.909)--(8.264,4.910)--(8.267,4.912)--(8.270,4.913)--(8.273,4.915)%
  --(8.276,4.916)--(8.279,4.918)--(8.282,4.919)--(8.285,4.921)--(8.288,4.922)--(8.291,4.924)%
  --(8.294,4.925)--(8.297,4.927)--(8.300,4.928)--(8.303,4.930)--(8.306,4.931)--(8.309,4.932)%
  --(8.312,4.934)--(8.314,4.935)--(8.317,4.937)--(8.320,4.938)--(8.323,4.940)--(8.326,4.941)%
  --(8.329,4.943)--(8.332,4.944)--(8.335,4.946)--(8.338,4.947)--(8.341,4.949)--(8.344,4.950)%
  --(8.347,4.952)--(8.350,4.953)--(8.353,4.955)--(8.356,4.956)--(8.359,4.958)--(8.362,4.959)%
  --(8.365,4.961)--(8.368,4.962)--(8.371,4.964)--(8.374,4.965)--(8.377,4.967)--(8.380,4.968)%
  --(8.383,4.970)--(8.386,4.971)--(8.389,4.973)--(8.392,4.974)--(8.395,4.976)--(8.398,4.977)%
  --(8.401,4.979)--(8.404,4.980)--(8.407,4.982)--(8.410,4.983)--(8.413,4.985)--(8.416,4.986)%
  --(8.419,4.987)--(8.422,4.989)--(8.425,4.990)--(8.428,4.992)--(8.431,4.993)--(8.434,4.995)%
  --(8.437,4.996)--(8.440,4.998)--(8.443,4.999)--(8.446,5.001)--(8.449,5.002)--(8.452,5.004)%
  --(8.455,5.005)--(8.458,5.007)--(8.461,5.008)--(8.464,5.010)--(8.467,5.011)--(8.470,5.013)%
  --(8.473,5.014)--(8.476,5.016)--(8.479,5.017)--(8.482,5.019)--(8.485,5.020)--(8.488,5.022)%
  --(8.491,5.023)--(8.494,5.025)--(8.497,5.026)--(8.500,5.028)--(8.503,5.029)--(8.506,5.031)%
  --(8.509,5.032)--(8.512,5.034)--(8.515,5.035)--(8.518,5.037)--(8.521,5.038)--(8.523,5.040)%
  --(8.526,5.041)--(8.529,5.043)--(8.532,5.044)--(8.535,5.046)--(8.538,5.047)--(8.541,5.048)%
  --(8.544,5.050)--(8.547,5.051)--(8.550,5.053)--(8.553,5.054)--(8.556,5.056)--(8.559,5.057)%
  --(8.562,5.059)--(8.565,5.060)--(8.568,5.062)--(8.571,5.063)--(8.574,5.065)--(8.577,5.066)%
  --(8.580,5.068)--(8.583,5.069)--(8.586,5.071)--(8.589,5.072)--(8.592,5.074)--(8.595,5.075)%
  --(8.598,5.077)--(8.601,5.078)--(8.604,5.080)--(8.607,5.081)--(8.610,5.083)--(8.613,5.084)%
  --(8.616,5.086)--(8.619,5.087)--(8.622,5.089)--(8.625,5.090)--(8.628,5.092)--(8.631,5.093)%
  --(8.634,5.095)--(8.637,5.096)--(8.640,5.098)--(8.643,5.099)--(8.646,5.101)--(8.649,5.102)%
  --(8.652,5.104)--(8.655,5.105)--(8.658,5.107)--(8.661,5.108)--(8.664,5.110)--(8.667,5.111)%
  --(8.670,5.113)--(8.673,5.114)--(8.676,5.116)--(8.679,5.117)--(8.682,5.119)--(8.685,5.120)%
  --(8.688,5.122)--(8.691,5.123)--(8.694,5.125)--(8.697,5.126)--(8.700,5.127)--(8.703,5.129)%
  --(8.706,5.130)--(8.709,5.132)--(8.712,5.133)--(8.715,5.135)--(8.718,5.136)--(8.721,5.138)%
  --(8.724,5.139)--(8.727,5.141)--(8.730,5.142)--(8.733,5.144)--(8.735,5.145)--(8.738,5.147)%
  --(8.741,5.148)--(8.744,5.150)--(8.747,5.151)--(8.750,5.153)--(8.753,5.154)--(8.756,5.156)%
  --(8.759,5.157)--(8.762,5.159)--(8.765,5.160)--(8.768,5.162)--(8.771,5.163)--(8.774,5.165)%
  --(8.777,5.166)--(8.780,5.168)--(8.783,5.169)--(8.786,5.171)--(8.789,5.172)--(8.792,5.174)%
  --(8.795,5.175)--(8.798,5.177)--(8.801,5.178)--(8.804,5.180)--(8.807,5.181)--(8.810,5.183)%
  --(8.813,5.184)--(8.816,5.186)--(8.819,5.187)--(8.822,5.189)--(8.825,5.190)--(8.828,5.192)%
  --(8.831,5.193)--(8.834,5.195)--(8.837,5.196)--(8.840,5.198)--(8.843,5.199)--(8.846,5.201)%
  --(8.849,5.202)--(8.852,5.204)--(8.855,5.205)--(8.858,5.207)--(8.861,5.208)--(8.864,5.210)%
  --(8.867,5.211)--(8.870,5.213)--(8.873,5.214)--(8.876,5.216)--(8.879,5.217)--(8.882,5.219)%
  --(8.885,5.220)--(8.888,5.222)--(8.891,5.223)--(8.894,5.225)--(8.897,5.226)--(8.900,5.228)%
  --(8.903,5.229)--(8.906,5.231)--(8.909,5.232)--(8.912,5.234)--(8.915,5.235)--(8.918,5.236)%
  --(8.921,5.238)--(8.924,5.239)--(8.927,5.241)--(8.930,5.242)--(8.933,5.244)--(8.936,5.245)%
  --(8.939,5.247)--(8.942,5.248)--(8.944,5.250)--(8.947,5.251)--(8.950,5.253)--(8.953,5.254)%
  --(8.956,5.256)--(8.959,5.257)--(8.962,5.259)--(8.965,5.260)--(8.968,5.262)--(8.971,5.263)%
  --(8.974,5.265)--(8.977,5.266)--(8.980,5.268)--(8.983,5.269)--(8.986,5.271)--(8.989,5.272)%
  --(8.992,5.274)--(8.995,5.275)--(8.998,5.277)--(9.001,5.278)--(9.004,5.280)--(9.007,5.281)%
  --(9.010,5.283)--(9.013,5.284)--(9.016,5.286)--(9.019,5.287)--(9.022,5.289)--(9.025,5.290)%
  --(9.028,5.292)--(9.031,5.293)--(9.034,5.295)--(9.037,5.296)--(9.040,5.298)--(9.043,5.299)%
  --(9.046,5.301)--(9.049,5.302)--(9.052,5.304)--(9.055,5.305)--(9.058,5.307)--(9.061,5.308)%
  --(9.064,5.310)--(9.067,5.311)--(9.070,5.313)--(9.073,5.314)--(9.076,5.316)--(9.079,5.317)%
  --(9.082,5.319)--(9.085,5.320)--(9.088,5.322)--(9.091,5.323)--(9.094,5.325)--(9.097,5.326)%
  --(9.100,5.328)--(9.103,5.329)--(9.106,5.331)--(9.109,5.332)--(9.112,5.334)--(9.115,5.335)%
  --(9.118,5.337)--(9.121,5.338)--(9.124,5.340)--(9.127,5.341)--(9.130,5.343)--(9.133,5.344)%
  --(9.136,5.346)--(9.139,5.347)--(9.142,5.349)--(9.145,5.350)--(9.148,5.352)--(9.151,5.353)%
  --(9.153,5.355)--(9.156,5.356)--(9.159,5.358)--(9.162,5.359)--(9.165,5.361)--(9.168,5.362)%
  --(9.171,5.364)--(9.174,5.365)--(9.177,5.367)--(9.180,5.368)--(9.183,5.370)--(9.186,5.371)%
  --(9.189,5.373)--(9.192,5.374)--(9.195,5.376)--(9.198,5.377)--(9.201,5.379)--(9.204,5.380)%
  --(9.207,5.382)--(9.210,5.383)--(9.213,5.385)--(9.216,5.386)--(9.219,5.388)--(9.222,5.389)%
  --(9.225,5.391)--(9.228,5.392)--(9.231,5.394)--(9.234,5.395)--(9.237,5.397)--(9.240,5.398)%
  --(9.243,5.400)--(9.246,5.401)--(9.249,5.403)--(9.252,5.404)--(9.255,5.406)--(9.258,5.407)%
  --(9.261,5.409)--(9.264,5.410)--(9.267,5.412)--(9.270,5.413)--(9.273,5.415)--(9.276,5.416)%
  --(9.279,5.418)--(9.282,5.419)--(9.285,5.421)--(9.288,5.422)--(9.291,5.424)--(9.294,5.425)%
  --(9.297,5.427)--(9.300,5.428)--(9.303,5.430)--(9.306,5.431)--(9.309,5.433)--(9.312,5.434)%
  --(9.315,5.436)--(9.318,5.437)--(9.321,5.439)--(9.324,5.440)--(9.327,5.442)--(9.330,5.443)%
  --(9.333,5.445)--(9.336,5.446)--(9.339,5.448)--(9.342,5.449)--(9.345,5.451)--(9.348,5.452)%
  --(9.351,5.454)--(9.354,5.455)--(9.357,5.457)--(9.360,5.458)--(9.362,5.460)--(9.365,5.461)%
  --(9.368,5.463)--(9.371,5.464)--(9.374,5.466)--(9.377,5.467)--(9.380,5.469)--(9.383,5.470)%
  --(9.386,5.472)--(9.389,5.473)--(9.392,5.475)--(9.395,5.476)--(9.398,5.478)--(9.401,5.479)%
  --(9.404,5.481)--(9.407,5.482)--(9.410,5.484)--(9.413,5.485)--(9.416,5.487)--(9.419,5.488)%
  --(9.422,5.490)--(9.425,5.491)--(9.428,5.493)--(9.431,5.494)--(9.434,5.496)--(9.437,5.497)%
  --(9.440,5.499)--(9.443,5.500)--(9.446,5.502)--(9.449,5.503)--(9.452,5.505)--(9.455,5.506)%
  --(9.458,5.508)--(9.461,5.509)--(9.464,5.511)--(9.467,5.512)--(9.470,5.514)--(9.473,5.515)%
  --(9.476,5.517)--(9.479,5.518)--(9.482,5.520)--(9.485,5.521)--(9.488,5.523)--(9.491,5.524)%
  --(9.494,5.526)--(9.497,5.527)--(9.500,5.529)--(9.503,5.530)--(9.506,5.532)--(9.509,5.533)%
  --(9.512,5.535)--(9.515,5.536)--(9.518,5.538)--(9.521,5.539)--(9.524,5.541)--(9.527,5.542)%
  --(9.530,5.544)--(9.533,5.545)--(9.536,5.547)--(9.539,5.548)--(9.542,5.550)--(9.545,5.551)%
  --(9.548,5.553)--(9.551,5.554)--(9.554,5.556)--(9.557,5.557)--(9.560,5.559)--(9.563,5.560)%
  --(9.566,5.562)--(9.569,5.563)--(9.571,5.565)--(9.574,5.566)--(9.577,5.568)--(9.580,5.569)%
  --(9.583,5.571)--(9.586,5.572)--(9.589,5.574)--(9.592,5.575)--(9.595,5.577)--(9.598,5.578)%
  --(9.601,5.580)--(9.604,5.581)--(9.607,5.583)--(9.610,5.584)--(9.613,5.586)--(9.616,5.587)%
  --(9.619,5.589)--(9.622,5.590)--(9.625,5.592)--(9.628,5.593)--(9.631,5.595)--(9.634,5.596)%
  --(9.637,5.598)--(9.640,5.599)--(9.643,5.601)--(9.646,5.602)--(9.649,5.604)--(9.652,5.605)%
  --(9.655,5.607)--(9.658,5.608)--(9.661,5.610)--(9.664,5.611)--(9.667,5.613)--(9.670,5.614)%
  --(9.673,5.616)--(9.676,5.617)--(9.679,5.619)--(9.682,5.620)--(9.685,5.622)--(9.688,5.623)%
  --(9.691,5.625)--(9.694,5.626)--(9.697,5.628)--(9.700,5.629)--(9.703,5.631)--(9.706,5.632)%
  --(9.709,5.634)--(9.712,5.635)--(9.715,5.637)--(9.718,5.638)--(9.721,5.640)--(9.724,5.641)%
  --(9.727,5.643)--(9.730,5.644)--(9.733,5.646)--(9.736,5.647)--(9.739,5.649)--(9.742,5.650)%
  --(9.745,5.652)--(9.748,5.653)--(9.751,5.655)--(9.754,5.656)--(9.757,5.658)--(9.760,5.659)%
  --(9.763,5.661)--(9.766,5.662)--(9.769,5.664)--(9.772,5.665)--(9.775,5.667)--(9.778,5.668)%
  --(9.780,5.670)--(9.783,5.671)--(9.786,5.673)--(9.789,5.674)--(9.792,5.676)--(9.795,5.677)%
  --(9.798,5.679)--(9.801,5.680)--(9.804,5.682)--(9.807,5.683)--(9.810,5.685)--(9.813,5.686)%
  --(9.816,5.688)--(9.819,5.689)--(9.822,5.691)--(9.825,5.692)--(9.828,5.694)--(9.831,5.695)%
  --(9.834,5.697)--(9.837,5.698)--(9.840,5.700)--(9.843,5.701)--(9.846,5.703)--(9.849,5.704)%
  --(9.852,5.706)--(9.855,5.707)--(9.858,5.709)--(9.861,5.710)--(9.864,5.712)--(9.867,5.713)%
  --(9.870,5.715)--(9.873,5.716)--(9.876,5.718)--(9.879,5.719)--(9.882,5.721)--(9.885,5.722)%
  --(9.888,5.724)--(9.891,5.725)--(9.894,5.727)--(9.897,5.728)--(9.900,5.730)--(9.903,5.732)%
  --(9.906,5.733)--(9.909,5.735)--(9.912,5.736)--(9.915,5.738)--(9.918,5.739)--(9.921,5.741)%
  --(9.924,5.742)--(9.927,5.744)--(9.930,5.745)--(9.933,5.747)--(9.936,5.748)--(9.939,5.750)%
  --(9.942,5.751)--(9.945,5.753)--(9.948,5.754)--(9.951,5.756)--(9.954,5.757)--(9.957,5.759)%
  --(9.960,5.760)--(9.963,5.762)--(9.966,5.763)--(9.969,5.765)--(9.972,5.766)--(9.975,5.768)%
  --(9.978,5.769)--(9.981,5.771)--(9.984,5.772)--(9.987,5.774)--(9.990,5.775)--(9.992,5.777)%
  --(9.995,5.778)--(9.998,5.780)--(10.001,5.781)--(10.004,5.783)--(10.007,5.784)--(10.010,5.786)%
  --(10.013,5.787)--(10.016,5.789)--(10.019,5.790)--(10.022,5.792)--(10.025,5.793)--(10.028,5.795)%
  --(10.031,5.796)--(10.034,5.798)--(10.037,5.799)--(10.040,5.801)--(10.043,5.802)--(10.046,5.804)%
  --(10.049,5.805)--(10.052,5.807)--(10.055,5.808)--(10.058,5.810)--(10.061,5.811)--(10.064,5.813)%
  --(10.067,5.814)--(10.070,5.816)--(10.073,5.817)--(10.076,5.819)--(10.079,5.820)--(10.082,5.822)%
  --(10.085,5.823)--(10.088,5.825)--(10.091,5.826)--(10.094,5.828)--(10.097,5.829)--(10.100,5.831)%
  --(10.103,5.832)--(10.106,5.834)--(10.109,5.835)--(10.112,5.837)--(10.115,5.838)--(10.118,5.840)%
  --(10.121,5.841)--(10.124,5.843)--(10.127,5.844)--(10.130,5.846)--(10.133,5.847)--(10.136,5.849)%
  --(10.139,5.850)--(10.142,5.852)--(10.145,5.853)--(10.148,5.855)--(10.151,5.856)--(10.154,5.858)%
  --(10.157,5.859)--(10.160,5.861)--(10.163,5.863)--(10.166,5.864)--(10.169,5.866)--(10.172,5.867)%
  --(10.175,5.869)--(10.178,5.870)--(10.181,5.872)--(10.184,5.873)--(10.187,5.875)--(10.190,5.876)%
  --(10.193,5.878)--(10.196,5.879)--(10.199,5.881)--(10.201,5.882)--(10.204,5.884)--(10.207,5.885)%
  --(10.210,5.887)--(10.213,5.888)--(10.216,5.890)--(10.219,5.891)--(10.222,5.893)--(10.225,5.894)%
  --(10.228,5.896)--(10.231,5.897)--(10.234,5.899)--(10.237,5.900)--(10.240,5.902)--(10.243,5.903)%
  --(10.246,5.905)--(10.249,5.906)--(10.252,5.908)--(10.255,5.909)--(10.258,5.911)--(10.261,5.912)%
  --(10.264,5.914)--(10.267,5.915)--(10.270,5.917)--(10.273,5.918)--(10.276,5.920)--(10.279,5.921)%
  --(10.282,5.923)--(10.285,5.924)--(10.288,5.926)--(10.291,5.927)--(10.294,5.929)--(10.297,5.930)%
  --(10.300,5.932)--(10.303,5.933)--(10.306,5.935)--(10.309,5.936)--(10.312,5.938)--(10.315,5.939)%
  --(10.318,5.941)--(10.321,5.942)--(10.324,5.944)--(10.327,5.945)--(10.330,5.947)--(10.333,5.948)%
  --(10.336,5.950)--(10.339,5.951)--(10.342,5.953)--(10.345,5.954)--(10.348,5.956)--(10.351,5.957)%
  --(10.354,5.959)--(10.357,5.960)--(10.360,5.962)--(10.363,5.964)--(10.366,5.965)--(10.369,5.967)%
  --(10.372,5.968)--(10.375,5.970)--(10.378,5.971)--(10.381,5.973)--(10.384,5.974)--(10.387,5.976)%
  --(10.390,5.977)--(10.393,5.979)--(10.396,5.980)--(10.399,5.982)--(10.402,5.983)--(10.405,5.985)%
  --(10.408,5.986)--(10.410,5.988)--(10.413,5.989)--(10.416,5.991)--(10.419,5.992)--(10.422,5.994)%
  --(10.425,5.995)--(10.428,5.997)--(10.431,5.998)--(10.434,6.000)--(10.437,6.001)--(10.440,6.003)%
  --(10.443,6.004)--(10.446,6.006)--(10.449,6.007)--(10.452,6.009)--(10.455,6.010)--(10.458,6.012)%
  --(10.461,6.013)--(10.464,6.015)--(10.467,6.016)--(10.470,6.018)--(10.473,6.019)--(10.476,6.021)%
  --(10.479,6.022)--(10.482,6.024)--(10.485,6.025)--(10.488,6.027)--(10.491,6.028)--(10.494,6.030)%
  --(10.497,6.031)--(10.500,6.033)--(10.503,6.034)--(10.506,6.036)--(10.509,6.037)--(10.512,6.039)%
  --(10.515,6.040)--(10.518,6.042)--(10.521,6.043)--(10.524,6.045)--(10.527,6.046)--(10.530,6.048)%
  --(10.533,6.050)--(10.536,6.051)--(10.539,6.053)--(10.542,6.054)--(10.545,6.056)--(10.548,6.057)%
  --(10.551,6.059)--(10.554,6.060)--(10.557,6.062)--(10.560,6.063)--(10.563,6.065)--(10.566,6.066)%
  --(10.569,6.068)--(10.572,6.069)--(10.575,6.071)--(10.578,6.072)--(10.581,6.074)--(10.584,6.075)%
  --(10.587,6.077)--(10.590,6.078)--(10.593,6.080)--(10.596,6.081)--(10.599,6.083)--(10.602,6.084)%
  --(10.605,6.086)--(10.608,6.087)--(10.611,6.089)--(10.614,6.090)--(10.617,6.092)--(10.619,6.093)%
  --(10.622,6.095)--(10.625,6.096)--(10.628,6.098)--(10.631,6.099)--(10.634,6.101)--(10.637,6.102)%
  --(10.640,6.104)--(10.643,6.105)--(10.646,6.107)--(10.649,6.108)--(10.652,6.110)--(10.655,6.111)%
  --(10.658,6.113)--(10.661,6.114)--(10.664,6.116)--(10.667,6.117)--(10.670,6.119)--(10.673,6.120)%
  --(10.676,6.122)--(10.679,6.123)--(10.682,6.125)--(10.685,6.127)--(10.688,6.128)--(10.691,6.130)%
  --(10.694,6.131)--(10.697,6.133)--(10.700,6.134)--(10.703,6.136)--(10.706,6.137)--(10.709,6.139)%
  --(10.712,6.140)--(10.715,6.142)--(10.718,6.143)--(10.721,6.145)--(10.724,6.146)--(10.727,6.148)%
  --(10.730,6.149)--(10.733,6.151)--(10.736,6.152)--(10.739,6.154)--(10.742,6.155)--(10.745,6.157)%
  --(10.748,6.158)--(10.751,6.160)--(10.754,6.161)--(10.757,6.163)--(10.760,6.164)--(10.763,6.166)%
  --(10.766,6.167)--(10.769,6.169)--(10.772,6.170)--(10.775,6.172)--(10.778,6.173)--(10.781,6.175)%
  --(10.784,6.176)--(10.787,6.178)--(10.790,6.179)--(10.793,6.181)--(10.796,6.182)--(10.799,6.184)%
  --(10.802,6.185)--(10.805,6.187)--(10.808,6.188)--(10.811,6.190)--(10.814,6.191)--(10.817,6.193)%
  --(10.820,6.194)--(10.823,6.196)--(10.826,6.198)--(10.828,6.199)--(10.831,6.201)--(10.834,6.202)%
  --(10.837,6.204)--(10.840,6.205)--(10.843,6.207)--(10.846,6.208)--(10.849,6.210)--(10.852,6.211)%
  --(10.855,6.213)--(10.858,6.214)--(10.861,6.216)--(10.864,6.217)--(10.867,6.219)--(10.870,6.220)%
  --(10.873,6.222)--(10.876,6.223)--(10.879,6.225)--(10.882,6.226)--(10.885,6.228)--(10.888,6.229)%
  --(10.891,6.231)--(10.894,6.232)--(10.897,6.234)--(10.900,6.235)--(10.903,6.237)--(10.906,6.238)%
  --(10.909,6.240)--(10.912,6.241)--(10.915,6.243)--(10.918,6.244)--(10.921,6.246)--(10.924,6.247)%
  --(10.927,6.249)--(10.930,6.250)--(10.933,6.252)--(10.936,6.253)--(10.939,6.255)--(10.942,6.256)%
  --(10.945,6.258)--(10.948,6.259)--(10.951,6.261)--(10.954,6.262)--(10.957,6.264)--(10.960,6.266)%
  --(10.963,6.267)--(10.966,6.269)--(10.969,6.270)--(10.972,6.272)--(10.975,6.273)--(10.978,6.275)%
  --(10.981,6.276)--(10.984,6.278)--(10.987,6.279)--(10.990,6.281)--(10.993,6.282)--(10.996,6.284)%
  --(10.999,6.285)--(11.002,6.287)--(11.005,6.288)--(11.008,6.290)--(11.011,6.291)--(11.014,6.293)%
  --(11.017,6.294)--(11.020,6.296)--(11.023,6.297)--(11.026,6.299)--(11.029,6.300)--(11.032,6.302)%
  --(11.035,6.303)--(11.037,6.305)--(11.040,6.306)--(11.043,6.308)--(11.046,6.309)--(11.049,6.311)%
  --(11.052,6.312)--(11.055,6.314)--(11.058,6.315)--(11.061,6.317)--(11.064,6.318)--(11.067,6.320)%
  --(11.070,6.321)--(11.073,6.323)--(11.076,6.324)--(11.079,6.326)--(11.082,6.328)--(11.085,6.329)%
  --(11.088,6.331)--(11.091,6.332)--(11.094,6.334)--(11.097,6.335)--(11.100,6.337)--(11.103,6.338)%
  --(11.106,6.340)--(11.109,6.341)--(11.112,6.343)--(11.115,6.344)--(11.118,6.346)--(11.121,6.347)%
  --(11.124,6.349)--(11.127,6.350)--(11.130,6.352)--(11.133,6.353)--(11.136,6.355)--(11.139,6.356)%
  --(11.142,6.358)--(11.145,6.359)--(11.148,6.361)--(11.151,6.362)--(11.154,6.364)--(11.157,6.365)%
  --(11.160,6.367)--(11.163,6.368)--(11.166,6.370)--(11.169,6.371)--(11.172,6.373)--(11.175,6.374)%
  --(11.178,6.376)--(11.181,6.377)--(11.184,6.379)--(11.187,6.380)--(11.190,6.382)--(11.193,6.383)%
  --(11.196,6.385)--(11.199,6.386)--(11.202,6.388)--(11.205,6.390)--(11.208,6.391)--(11.211,6.393)%
  --(11.214,6.394)--(11.217,6.396)--(11.220,6.397)--(11.223,6.399)--(11.226,6.400)--(11.229,6.402)%
  --(11.232,6.403)--(11.235,6.405)--(11.238,6.406)--(11.241,6.408)--(11.244,6.409)--(11.247,6.411)%
  --(11.249,6.412)--(11.252,6.414)--(11.255,6.415)--(11.258,6.417)--(11.261,6.418)--(11.264,6.420)%
  --(11.267,6.421)--(11.270,6.423)--(11.273,6.424)--(11.276,6.426)--(11.279,6.427)--(11.282,6.429)%
  --(11.285,6.430)--(11.288,6.432)--(11.291,6.433)--(11.294,6.435)--(11.297,6.436)--(11.300,6.438)%
  --(11.303,6.439)--(11.306,6.441)--(11.309,6.442)--(11.312,6.444)--(11.315,6.446)--(11.318,6.447)%
  --(11.321,6.449)--(11.324,6.450)--(11.327,6.452)--(11.330,6.453)--(11.333,6.455)--(11.336,6.456)%
  --(11.339,6.458)--(11.342,6.459)--(11.345,6.461)--(11.348,6.462)--(11.351,6.464)--(11.354,6.465)%
  --(11.357,6.467)--(11.360,6.468)--(11.363,6.470)--(11.366,6.471)--(11.369,6.473)--(11.372,6.474)%
  --(11.375,6.476)--(11.378,6.477)--(11.381,6.479)--(11.384,6.480)--(11.387,6.482)--(11.390,6.483)%
  --(11.393,6.485)--(11.396,6.486)--(11.399,6.488)--(11.402,6.489)--(11.405,6.491)--(11.408,6.492)%
  --(11.411,6.494)--(11.414,6.495)--(11.417,6.497)--(11.420,6.498)--(11.423,6.500)--(11.426,6.502)%
  --(11.429,6.503)--(11.432,6.505)--(11.435,6.506)--(11.438,6.508)--(11.441,6.509)--(11.444,6.511)%
  --(11.447,6.512)--(11.450,6.514)--(11.453,6.515)--(11.456,6.517)--(11.458,6.518)--(11.461,6.520)%
  --(11.464,6.521)--(11.467,6.523)--(11.470,6.524)--(11.473,6.526)--(11.476,6.527)--(11.479,6.529)%
  --(11.482,6.530)--(11.485,6.532)--(11.488,6.533)--(11.491,6.535)--(11.494,6.536)--(11.497,6.538)%
  --(11.500,6.539)--(11.503,6.541)--(11.506,6.542)--(11.509,6.544)--(11.512,6.545)--(11.515,6.547)%
  --(11.518,6.548)--(11.521,6.550)--(11.524,6.551)--(11.527,6.553)--(11.530,6.555)--(11.533,6.556)%
  --(11.536,6.558)--(11.539,6.559)--(11.542,6.561)--(11.545,6.562)--(11.548,6.564)--(11.551,6.565)%
  --(11.554,6.567)--(11.557,6.568)--(11.560,6.570)--(11.563,6.571)--(11.566,6.573)--(11.569,6.574)%
  --(11.572,6.576)--(11.575,6.577)--(11.578,6.579)--(11.581,6.580)--(11.584,6.582)--(11.587,6.583)%
  --(11.590,6.585)--(11.593,6.586)--(11.596,6.588)--(11.599,6.589)--(11.602,6.591)--(11.605,6.592)%
  --(11.608,6.594)--(11.611,6.595)--(11.614,6.597)--(11.617,6.598)--(11.620,6.600)--(11.623,6.601)%
  --(11.626,6.603)--(11.629,6.605)--(11.632,6.606)--(11.635,6.608)--(11.638,6.609)--(11.641,6.611)%
  --(11.644,6.612)--(11.647,6.614)--(11.650,6.615)--(11.653,6.617)--(11.656,6.618)--(11.659,6.620)%
  --(11.662,6.621)--(11.665,6.623)--(11.667,6.624)--(11.670,6.626)--(11.673,6.627)--(11.676,6.629)%
  --(11.679,6.630)--(11.682,6.632)--(11.685,6.633)--(11.688,6.635)--(11.691,6.636)--(11.694,6.638)%
  --(11.697,6.639)--(11.700,6.641)--(11.703,6.642)--(11.706,6.644)--(11.709,6.645)--(11.712,6.647)%
  --(11.715,6.648)--(11.718,6.650)--(11.721,6.651)--(11.724,6.653)--(11.727,6.654)--(11.730,6.656)%
  --(11.733,6.658)--(11.736,6.659)--(11.739,6.661)--(11.742,6.662)--(11.745,6.664)--(11.748,6.665)%
  --(11.751,6.667)--(11.754,6.668)--(11.757,6.670)--(11.760,6.671)--(11.763,6.673)--(11.766,6.674)%
  --(11.769,6.676)--(11.772,6.677)--(11.775,6.679)--(11.778,6.680)--(11.781,6.682)--(11.784,6.683)%
  --(11.787,6.685)--(11.790,6.686)--(11.793,6.688)--(11.796,6.689)--(11.799,6.691)--(11.802,6.692)%
  --(11.805,6.694)--(11.808,6.695)--(11.811,6.697)--(11.814,6.698)--(11.817,6.700)--(11.820,6.701)%
  --(11.823,6.703)--(11.826,6.705)--(11.829,6.706)--(11.832,6.708)--(11.835,6.709)--(11.838,6.711)%
  --(11.841,6.712)--(11.844,6.714)--(11.847,6.715)--(11.850,6.717)--(11.853,6.718)--(11.856,6.720)%
  --(11.859,6.721)--(11.862,6.723)--(11.865,6.724)--(11.868,6.726)--(11.871,6.727)--(11.874,6.729)%
  --(11.876,6.730)--(11.879,6.732)--(11.882,6.733)--(11.885,6.735)--(11.888,6.736)--(11.891,6.738)%
  --(11.894,6.739)--(11.897,6.741)--(11.900,6.742)--(11.903,6.744)--(11.906,6.745)--(11.909,6.747)%
  --(11.912,6.748)--(11.915,6.750)--(11.918,6.751)--(11.921,6.753)--(11.924,6.755)--(11.927,6.756)%
  --(11.930,6.758)--(11.933,6.759)--(11.936,6.761)--(11.939,6.762)--(11.942,6.764)--(11.945,6.765)%
  --(11.948,6.767)--(11.951,6.768)--(11.954,6.770)--(11.957,6.771)--(11.960,6.773)--(11.963,6.774)%
  --(11.966,6.776)--(11.969,6.777)--(11.972,6.779)--(11.975,6.780)--(11.978,6.782)--(11.981,6.783)%
  --(11.984,6.785)--(11.987,6.786)--(11.990,6.788)--(11.993,6.789)--(11.996,6.791)--(11.999,6.792)%
  --(12.002,6.794)--(12.005,6.795)--(12.008,6.797)--(12.011,6.798)--(12.014,6.800)--(12.017,6.802)%
  --(12.020,6.803)--(12.023,6.805)--(12.026,6.806)--(12.029,6.808)--(12.032,6.809)--(12.035,6.811)%
  --(12.038,6.812)--(12.041,6.814)--(12.044,6.815)--(12.047,6.817)--(12.050,6.818)--(12.053,6.820)%
  --(12.056,6.821)--(12.059,6.823)--(12.062,6.824)--(12.065,6.826)--(12.068,6.827)--(12.071,6.829)%
  --(12.074,6.830)--(12.077,6.832)--(12.080,6.833)--(12.083,6.835)--(12.085,6.836)--(12.088,6.838)%
  --(12.091,6.839)--(12.094,6.841)--(12.097,6.842)--(12.100,6.844)--(12.103,6.846)--(12.106,6.847)%
  --(12.109,6.849)--(12.112,6.850)--(12.115,6.852)--(12.118,6.853)--(12.121,6.855)--(12.124,6.856)%
  --(12.127,6.858)--(12.130,6.859)--(12.133,6.861)--(12.136,6.862)--(12.139,6.864)--(12.142,6.865)%
  --(12.145,6.867)--(12.148,6.868)--(12.151,6.870)--(12.154,6.871)--(12.157,6.873)--(12.160,6.874)%
  --(12.163,6.876)--(12.166,6.877)--(12.169,6.879)--(12.172,6.880)--(12.175,6.882)--(12.178,6.883)%
  --(12.181,6.885)--(12.184,6.886)--(12.187,6.888)--(12.190,6.889)--(12.193,6.891)--(12.196,6.893)%
  --(12.199,6.894)--(12.202,6.896)--(12.205,6.897)--(12.208,6.899)--(12.211,6.900)--(12.214,6.902)%
  --(12.217,6.903)--(12.220,6.905)--(12.223,6.906)--(12.226,6.908)--(12.229,6.909)--(12.232,6.911)%
  --(12.235,6.912)--(12.238,6.914)--(12.241,6.915)--(12.244,6.917)--(12.247,6.918)--(12.250,6.920)%
  --(12.253,6.921)--(12.256,6.923)--(12.259,6.924)--(12.262,6.926)--(12.265,6.927)--(12.268,6.929)%
  --(12.271,6.930)--(12.274,6.932)--(12.277,6.933)--(12.280,6.935)--(12.283,6.937)--(12.286,6.938)%
  --(12.289,6.940)--(12.292,6.941)--(12.295,6.943)--(12.297,6.944)--(12.300,6.946)--(12.303,6.947)%
  --(12.306,6.949)--(12.309,6.950)--(12.312,6.952)--(12.315,6.953)--(12.318,6.955)--(12.321,6.956)%
  --(12.324,6.958)--(12.327,6.959)--(12.330,6.961)--(12.333,6.962)--(12.336,6.964)--(12.339,6.965)%
  --(12.342,6.967)--(12.345,6.968)--(12.348,6.970)--(12.351,6.971)--(12.354,6.973)--(12.357,6.974)%
  --(12.360,6.976)--(12.363,6.977)--(12.366,6.979)--(12.369,6.981)--(12.372,6.982)--(12.375,6.984)%
  --(12.378,6.985)--(12.381,6.987)--(12.384,6.988)--(12.387,6.990)--(12.390,6.991)--(12.393,6.993)%
  --(12.396,6.994)--(12.399,6.996)--(12.402,6.997)--(12.405,6.999)--(12.408,7.000)--(12.411,7.002)%
  --(12.414,7.003)--(12.417,7.005)--(12.420,7.006)--(12.423,7.008)--(12.426,7.009)--(12.429,7.011)%
  --(12.432,7.012)--(12.435,7.014)--(12.438,7.015)--(12.441,7.017)--(12.444,7.018)--(12.447,7.020)%
  --(12.450,7.021)--(12.453,7.023)--(12.456,7.025)--(12.459,7.026)--(12.462,7.028)--(12.465,7.029)%
  --(12.468,7.031)--(12.471,7.032)--(12.474,7.034)--(12.477,7.035)--(12.480,7.037)--(12.483,7.038)%
  --(12.486,7.040)--(12.489,7.041)--(12.492,7.043)--(12.495,7.044)--(12.498,7.046)--(12.501,7.047)%
  --(12.504,7.049)--(12.506,7.050)--(12.509,7.052)--(12.512,7.053)--(12.515,7.055)--(12.518,7.056)%
  --(12.521,7.058)--(12.524,7.059)--(12.527,7.061)--(12.530,7.062)--(12.533,7.064)--(12.536,7.066)%
  --(12.539,7.067)--(12.542,7.069)--(12.545,7.070)--(12.548,7.072)--(12.551,7.073)--(12.554,7.075)%
  --(12.557,7.076)--(12.560,7.078)--(12.563,7.079)--(12.566,7.081)--(12.569,7.082)--(12.572,7.084)%
  --(12.575,7.085)--(12.578,7.087)--(12.581,7.088)--(12.584,7.090)--(12.587,7.091)--(12.590,7.093)%
  --(12.593,7.094)--(12.596,7.096)--(12.599,7.097)--(12.602,7.099)--(12.605,7.100)--(12.608,7.102)%
  --(12.611,7.103)--(12.614,7.105)--(12.617,7.106)--(12.620,7.108)--(12.623,7.110)--(12.626,7.111)%
  --(12.629,7.113)--(12.632,7.114)--(12.635,7.116)--(12.638,7.117)--(12.641,7.119)--(12.644,7.120)%
  --(12.647,7.122)--(12.650,7.123)--(12.653,7.125)--(12.656,7.126)--(12.659,7.128)--(12.662,7.129)%
  --(12.665,7.131)--(12.668,7.132)--(12.671,7.134)--(12.674,7.135)--(12.677,7.137)--(12.680,7.138)%
  --(12.683,7.140)--(12.686,7.141)--(12.689,7.143)--(12.692,7.144)--(12.695,7.146)--(12.698,7.147)%
  --(12.701,7.149)--(12.704,7.151)--(12.707,7.152)--(12.710,7.154)--(12.713,7.155)--(12.715,7.157)%
  --(12.718,7.158)--(12.721,7.160)--(12.724,7.161)--(12.727,7.163)--(12.730,7.164)--(12.733,7.166)%
  --(12.736,7.167)--(12.739,7.169)--(12.742,7.170)--(12.745,7.172)--(12.748,7.173)--(12.751,7.175)%
  --(12.754,7.176)--(12.757,7.178)--(12.760,7.179)--(12.763,7.181)--(12.766,7.182)--(12.769,7.184)%
  --(12.772,7.185)--(12.775,7.187)--(12.778,7.188)--(12.781,7.190)--(12.784,7.192)--(12.787,7.193)%
  --(12.790,7.195)--(12.793,7.196)--(12.796,7.198)--(12.799,7.199)--(12.802,7.201)--(12.805,7.202)%
  --(12.808,7.204)--(12.811,7.205)--(12.814,7.207)--(12.817,7.208)--(12.820,7.210)--(12.823,7.211)%
  --(12.826,7.213)--(12.829,7.214)--(12.832,7.216)--(12.835,7.217)--(12.838,7.219)--(12.841,7.220)%
  --(12.844,7.222)--(12.847,7.223)--(12.850,7.225)--(12.853,7.226)--(12.856,7.228)--(12.859,7.229)%
  --(12.862,7.231)--(12.865,7.233)--(12.868,7.234)--(12.871,7.236)--(12.874,7.237)--(12.877,7.239)%
  --(12.880,7.240)--(12.883,7.242)--(12.886,7.243)--(12.889,7.245)--(12.892,7.246)--(12.895,7.248)%
  --(12.898,7.249)--(12.901,7.251)--(12.904,7.252)--(12.907,7.254)--(12.910,7.255)--(12.913,7.257)%
  --(12.916,7.258)--(12.919,7.260)--(12.922,7.261)--(12.924,7.263)--(12.927,7.264)--(12.930,7.266)%
  --(12.933,7.267)--(12.936,7.269)--(12.939,7.271)--(12.942,7.272)--(12.945,7.274)--(12.948,7.275)%
  --(12.951,7.277)--(12.954,7.278)--(12.957,7.280)--(12.960,7.281)--(12.963,7.283)--(12.966,7.284)%
  --(12.969,7.286)--(12.972,7.287)--(12.975,7.289)--(12.978,7.290)--(12.981,7.292)--(12.984,7.293)%
  --(12.987,7.295)--(12.990,7.296)--(12.993,7.298)--(12.996,7.299)--(12.999,7.301)--(13.002,7.302)%
  --(13.005,7.304)--(13.008,7.305)--(13.011,7.307)--(13.014,7.308)--(13.017,7.310)--(13.020,7.312)%
  --(13.023,7.313)--(13.026,7.315)--(13.029,7.316)--(13.032,7.318)--(13.035,7.319)--(13.038,7.321)%
  --(13.041,7.322)--(13.044,7.324)--(13.047,7.325)--(13.050,7.327)--(13.053,7.328)--(13.056,7.330)%
  --(13.059,7.331)--(13.062,7.333)--(13.065,7.334)--(13.068,7.336)--(13.071,7.337)--(13.074,7.339)%
  --(13.077,7.340)--(13.080,7.342)--(13.083,7.343)--(13.086,7.345)--(13.089,7.346)--(13.092,7.348)%
  --(13.095,7.350)--(13.098,7.351)--(13.101,7.353)--(13.104,7.354)--(13.107,7.356)--(13.110,7.357)%
  --(13.113,7.359)--(13.116,7.360)--(13.119,7.362)--(13.122,7.363)--(13.125,7.365)--(13.128,7.366)%
  --(13.131,7.368)--(13.133,7.369)--(13.136,7.371)--(13.139,7.372)--(13.142,7.374)--(13.145,7.375)%
  --(13.148,7.377)--(13.151,7.378)--(13.154,7.380)--(13.157,7.381)--(13.160,7.383)--(13.163,7.384)%
  --(13.166,7.386)--(13.169,7.388)--(13.172,7.389)--(13.175,7.391)--(13.178,7.392)--(13.181,7.394)%
  --(13.184,7.395)--(13.187,7.397)--(13.190,7.398)--(13.193,7.400)--(13.196,7.401)--(13.199,7.403)%
  --(13.202,7.404)--(13.205,7.406)--(13.208,7.407)--(13.211,7.409)--(13.214,7.410)--(13.217,7.412)%
  --(13.220,7.413)--(13.223,7.415)--(13.226,7.416)--(13.229,7.418)--(13.232,7.419)--(13.235,7.421)%
  --(13.238,7.422)--(13.241,7.424)--(13.244,7.425)--(13.247,7.427)--(13.250,7.429)--(13.253,7.430)%
  --(13.256,7.432)--(13.259,7.433)--(13.262,7.435)--(13.265,7.436)--(13.268,7.438)--(13.271,7.439)%
  --(13.274,7.441)--(13.277,7.442)--(13.280,7.444)--(13.283,7.445)--(13.286,7.447)--(13.289,7.448)%
  --(13.292,7.450)--(13.295,7.451)--(13.298,7.453)--(13.301,7.454)--(13.304,7.456)--(13.307,7.457)%
  --(13.310,7.459)--(13.313,7.460)--(13.316,7.462)--(13.319,7.463)--(13.322,7.465)--(13.325,7.467)%
  --(13.328,7.468)--(13.331,7.470)--(13.334,7.471)--(13.337,7.473)--(13.340,7.474)--(13.342,7.476)%
  --(13.345,7.477)--(13.348,7.479)--(13.351,7.480)--(13.354,7.482)--(13.357,7.483)--(13.360,7.485)%
  --(13.363,7.486)--(13.366,7.488)--(13.369,7.489)--(13.372,7.491)--(13.375,7.492)--(13.378,7.494)%
  --(13.381,7.495)--(13.384,7.497)--(13.387,7.498)--(13.390,7.500)--(13.393,7.501)--(13.396,7.503)%
  --(13.399,7.505)--(13.402,7.506)--(13.405,7.508)--(13.408,7.509)--(13.411,7.511)--(13.414,7.512)%
  --(13.417,7.514)--(13.420,7.515)--(13.423,7.517)--(13.426,7.518)--(13.429,7.520)--(13.432,7.521)%
  --(13.435,7.523)--(13.438,7.524)--(13.441,7.526)--(13.444,7.527);
\gpcolor{rgb color={0.941,0.894,0.259}}
\draw[gp path] (1.504,2.514)--(1.625,2.514)--(1.745,2.514)--(1.866,2.514)--(1.986,2.514)%
  --(2.107,2.514)--(2.228,2.514)--(2.348,2.514)--(2.469,2.514)--(2.589,2.514)--(2.710,2.514)%
  --(2.831,2.514)--(2.951,2.514)--(3.072,2.514)--(3.192,2.514)--(3.313,2.514)--(3.434,2.514)%
  --(3.554,2.514)--(3.675,2.514)--(3.796,2.514)--(3.916,2.514)--(4.037,2.514)--(4.157,2.514)%
  --(4.278,2.514)--(4.399,2.514)--(4.519,2.514)--(4.640,2.514)--(4.760,2.514)--(4.881,2.514)%
  --(5.002,2.514)--(5.122,2.514)--(5.243,2.514)--(5.363,2.514)--(5.484,2.514)--(5.605,2.514)%
  --(5.725,2.514)--(5.846,2.514)--(5.966,2.514)--(6.087,2.514)--(6.208,2.514)--(6.328,2.514)%
  --(6.449,2.514)--(6.569,2.514)--(6.690,2.514)--(6.811,2.514)--(6.931,2.514)--(7.052,2.514)%
  --(7.172,2.514)--(7.293,2.514)--(7.414,2.514)--(7.534,2.514)--(7.655,2.514)--(7.776,2.514)%
  --(7.896,2.514)--(8.017,2.514)--(8.137,2.514)--(8.258,2.514)--(8.379,2.514)--(8.499,2.514)%
  --(8.620,2.514)--(8.740,2.514)--(8.861,2.514)--(8.982,2.514)--(9.102,2.514)--(9.223,2.514)%
  --(9.343,2.514)--(9.464,2.514)--(9.585,2.514)--(9.705,2.514)--(9.826,2.514)--(9.946,2.514)%
  --(10.067,2.514)--(10.188,2.514)--(10.308,2.514)--(10.429,2.514)--(10.549,2.514)--(10.670,2.514)%
  --(10.791,2.514)--(10.911,2.514)--(11.032,2.514)--(11.152,2.514)--(11.273,2.514)--(11.394,2.514)%
  --(11.514,2.514)--(11.635,2.514)--(11.756,2.514)--(11.876,2.514)--(11.997,2.514)--(12.117,2.514)%
  --(12.238,2.514)--(12.359,2.514)--(12.479,2.514)--(12.600,2.514)--(12.720,2.514)--(12.841,2.514)%
  --(12.962,2.514)--(13.082,2.514)--(13.203,2.514)--(13.323,2.514)--(13.444,2.514);
\gpcolor{color=gp lt color border}
\draw[gp path] (1.504,8.631)--(1.504,0.985)--(13.447,0.985)--(13.447,8.631)--cycle;
%% coordinates of the plot area
\gpdefrectangularnode{gp plot 1}{\pgfpoint{1.504cm}{0.985cm}}{\pgfpoint{13.447cm}{8.631cm}}
\end{tikzpicture}
%% gnuplot variables

	\caption{Exemplos de formas diferentes para $f(M)$ para valores diferentes de $\rho$. Os momentos de Fermi foram determinados a partir de \eqref{Eq:Mom_Fermi_a_partir_de_rho} com fração de prótons 1/2. \protect[Parameters: eNJL1; Proton fraction: 1/2] }
	\label{Fig:Gap_zero_graph}
\end{figure*}

Portanto, ao se utilizar um método qualquer para se determinar o zero da função, são necessários parâmetros que tornem possível encontrar o zero da função. Como a função só tem um zero\footnote[][1cm]{Desprezando a solução trivial $M = 0$.}, temos uma situação mais simples, pois não há risco de que tenhamos que encontrar mais que um mínimo (se esse fosse o caso, teríamos que encontrar aquele que minimiza alguma energia\cite{Buballa}. Verificar.). Adotando o método da biseção, que é o mais simples, só precisamos de dois pontos que devem estar à esquerda e à direita do zero. Adotamos os valores \np{1.0e-3} e 1500 (em MeV), sendo o primeiro bastante próximo de zero e o segundo aparentemente se posiciona à direita do zero para vários valores de densidade. Não podemos adotar zero para o valor à esquerda pois $f(M=0) = 0$. A Fig.~\ref{Fig:mass_graph} mostra os valores de $M$ obtidos de acordo com o acima exposto, enquanto a Fig.~\ref{Fig:scalar_density_graph} mostra a curva das densidades escalares correspondentes, ambos para o caso $m = 0$.

\begin{figure*}
	\begin{tikzpicture}[gnuplot]
%% generated with GNUPLOT 5.0p2 (Lua 5.2; terminal rev. 99, script rev. 100)
%% Fri Mar  4 16:19:10 2016
\path (0.000,0.000) rectangle (14.000,9.000);
\gpcolor{color=gp lt color border}
\gpsetlinetype{gp lt border}
\gpsetdashtype{gp dt solid}
\gpsetlinewidth{1.00}
\draw[gp path] (1.504,0.985)--(1.684,0.985);
\draw[gp path] (13.447,0.985)--(13.267,0.985);
\node[gp node right] at (1.320,0.985) {$0$};
\draw[gp path] (1.504,1.750)--(1.684,1.750);
\draw[gp path] (13.447,1.750)--(13.267,1.750);
\node[gp node right] at (1.320,1.750) {$100$};
\draw[gp path] (1.504,2.514)--(1.684,2.514);
\draw[gp path] (13.447,2.514)--(13.267,2.514);
\node[gp node right] at (1.320,2.514) {$200$};
\draw[gp path] (1.504,3.279)--(1.684,3.279);
\draw[gp path] (13.447,3.279)--(13.267,3.279);
\node[gp node right] at (1.320,3.279) {$300$};
\draw[gp path] (1.504,4.043)--(1.684,4.043);
\draw[gp path] (13.447,4.043)--(13.267,4.043);
\node[gp node right] at (1.320,4.043) {$400$};
\draw[gp path] (1.504,4.808)--(1.684,4.808);
\draw[gp path] (13.447,4.808)--(13.267,4.808);
\node[gp node right] at (1.320,4.808) {$500$};
\draw[gp path] (1.504,5.573)--(1.684,5.573);
\draw[gp path] (13.447,5.573)--(13.267,5.573);
\node[gp node right] at (1.320,5.573) {$600$};
\draw[gp path] (1.504,6.337)--(1.684,6.337);
\draw[gp path] (13.447,6.337)--(13.267,6.337);
\node[gp node right] at (1.320,6.337) {$700$};
\draw[gp path] (1.504,7.102)--(1.684,7.102);
\draw[gp path] (13.447,7.102)--(13.267,7.102);
\node[gp node right] at (1.320,7.102) {$800$};
\draw[gp path] (1.504,7.866)--(1.684,7.866);
\draw[gp path] (13.447,7.866)--(13.267,7.866);
\node[gp node right] at (1.320,7.866) {$900$};
\draw[gp path] (1.504,8.631)--(1.684,8.631);
\draw[gp path] (13.447,8.631)--(13.267,8.631);
\node[gp node right] at (1.320,8.631) {$1000$};
\draw[gp path] (1.504,0.985)--(1.504,1.165);
\draw[gp path] (1.504,8.631)--(1.504,8.451);
\node[gp node center] at (1.504,0.677) {$0$};
\draw[gp path] (2.997,0.985)--(2.997,1.165);
\draw[gp path] (2.997,8.631)--(2.997,8.451);
\node[gp node center] at (2.997,0.677) {$0.05$};
\draw[gp path] (4.490,0.985)--(4.490,1.165);
\draw[gp path] (4.490,8.631)--(4.490,8.451);
\node[gp node center] at (4.490,0.677) {$0.1$};
\draw[gp path] (5.983,0.985)--(5.983,1.165);
\draw[gp path] (5.983,8.631)--(5.983,8.451);
\node[gp node center] at (5.983,0.677) {$0.15$};
\draw[gp path] (7.476,0.985)--(7.476,1.165);
\draw[gp path] (7.476,8.631)--(7.476,8.451);
\node[gp node center] at (7.476,0.677) {$0.2$};
\draw[gp path] (8.968,0.985)--(8.968,1.165);
\draw[gp path] (8.968,8.631)--(8.968,8.451);
\node[gp node center] at (8.968,0.677) {$0.25$};
\draw[gp path] (10.461,0.985)--(10.461,1.165);
\draw[gp path] (10.461,8.631)--(10.461,8.451);
\node[gp node center] at (10.461,0.677) {$0.3$};
\draw[gp path] (11.954,0.985)--(11.954,1.165);
\draw[gp path] (11.954,8.631)--(11.954,8.451);
\node[gp node center] at (11.954,0.677) {$0.35$};
\draw[gp path] (13.447,0.985)--(13.447,1.165);
\draw[gp path] (13.447,8.631)--(13.447,8.451);
\node[gp node center] at (13.447,0.677) {$0.4$};
\draw[gp path] (1.504,8.631)--(1.504,0.985)--(13.447,0.985)--(13.447,8.631)--cycle;
\node[gp node center,rotate=-270] at (0.246,4.808) {$M$ (MeV)};
\node[gp node center] at (7.475,0.215) {$\rho$ ($\rm{fm}^{-3}$)};
\gpcolor{rgb color={0.580,0.000,0.827}}
\draw[gp path] (1.813,8.001)--(1.823,7.995)--(1.833,7.989)--(1.843,7.984)--(1.853,7.978)%
  --(1.864,7.973)--(1.874,7.967)--(1.884,7.961)--(1.894,7.956)--(1.904,7.950)--(1.914,7.945)%
  --(1.925,7.939)--(1.935,7.933)--(1.945,7.928)--(1.955,7.925)--(1.965,7.919)--(1.975,7.914)%
  --(1.985,7.908)--(1.996,7.903)--(2.006,7.897)--(2.016,7.891)--(2.026,7.886)--(2.036,7.880)%
  --(2.046,7.875)--(2.057,7.869)--(2.067,7.863)--(2.077,7.861)--(2.087,7.855)--(2.097,7.849)%
  --(2.107,7.844)--(2.118,7.838)--(2.128,7.833)--(2.138,7.827)--(2.148,7.821)--(2.158,7.816)%
  --(2.168,7.810)--(2.179,7.805)--(2.189,7.802)--(2.199,7.796)--(2.209,7.791)--(2.219,7.785)%
  --(2.229,7.779)--(2.240,7.774)--(2.250,7.768)--(2.260,7.763)--(2.270,7.757)--(2.280,7.751)%
  --(2.290,7.746)--(2.300,7.743)--(2.311,7.737)--(2.321,7.732)--(2.331,7.726)--(2.341,7.721)%
  --(2.351,7.715)--(2.361,7.709)--(2.372,7.704)--(2.382,7.698)--(2.392,7.693)--(2.402,7.690)%
  --(2.412,7.684)--(2.422,7.679)--(2.433,7.673)--(2.443,7.667)--(2.453,7.662)--(2.463,7.656)%
  --(2.473,7.651)--(2.483,7.645)--(2.494,7.642)--(2.504,7.637)--(2.514,7.631)--(2.524,7.625)%
  --(2.534,7.620)--(2.544,7.614)--(2.555,7.609)--(2.565,7.603)--(2.575,7.597)--(2.585,7.595)%
  --(2.595,7.589)--(2.605,7.583)--(2.616,7.578)--(2.626,7.572)--(2.636,7.567)--(2.646,7.561)%
  --(2.656,7.555)--(2.666,7.553)--(2.676,7.547)--(2.687,7.541)--(2.697,7.536)--(2.707,7.530)%
  --(2.717,7.525)--(2.727,7.519)--(2.737,7.513)--(2.748,7.508)--(2.758,7.505)--(2.768,7.499)%
  --(2.778,7.494)--(2.788,7.488)--(2.798,7.483)--(2.809,7.477)--(2.819,7.471)--(2.829,7.466)%
  --(2.839,7.463)--(2.849,7.457)--(2.859,7.452)--(2.870,7.446)--(2.880,7.441)--(2.890,7.435)%
  --(2.900,7.429)--(2.910,7.424)--(2.920,7.421)--(2.931,7.415)--(2.941,7.410)--(2.951,7.404)%
  --(2.961,7.399)--(2.971,7.393)--(2.981,7.387)--(2.991,7.385)--(3.002,7.379)--(3.012,7.373)%
  --(3.022,7.368)--(3.032,7.362)--(3.042,7.357)--(3.052,7.351)--(3.063,7.345)--(3.073,7.343)%
  --(3.083,7.337)--(3.093,7.331)--(3.103,7.326)--(3.113,7.320)--(3.124,7.315)--(3.134,7.309)%
  --(3.144,7.306)--(3.154,7.301)--(3.164,7.295)--(3.174,7.289)--(3.185,7.284)--(3.195,7.278)%
  --(3.205,7.273)--(3.215,7.270)--(3.225,7.264)--(3.235,7.259)--(3.246,7.253)--(3.256,7.247)%
  --(3.266,7.242)--(3.276,7.236)--(3.286,7.233)--(3.296,7.228)--(3.307,7.222)--(3.317,7.217)%
  --(3.327,7.211)--(3.337,7.205)--(3.347,7.200)--(3.357,7.197)--(3.367,7.191)--(3.378,7.186)%
  --(3.388,7.180)--(3.398,7.175)--(3.408,7.169)--(3.418,7.163)--(3.428,7.161)--(3.439,7.155)%
  --(3.449,7.149)--(3.459,7.144)--(3.469,7.138)--(3.479,7.133)--(3.489,7.127)--(3.500,7.124)%
  --(3.510,7.119)--(3.520,7.113)--(3.530,7.107)--(3.540,7.102)--(3.550,7.096)--(3.561,7.091)%
  --(3.571,7.088)--(3.581,7.082)--(3.591,7.077)--(3.601,7.071)--(3.611,7.065)--(3.622,7.060)%
  --(3.632,7.057)--(3.642,7.051)--(3.652,7.046)--(3.662,7.040)--(3.672,7.035)--(3.682,7.029)%
  --(3.693,7.023)--(3.703,7.021)--(3.713,7.015)--(3.723,7.009)--(3.733,7.004)--(3.743,6.998)%
  --(3.754,6.993)--(3.764,6.990)--(3.774,6.984)--(3.784,6.979)--(3.794,6.973)--(3.804,6.967)%
  --(3.815,6.962)--(3.825,6.956)--(3.835,6.953)--(3.845,6.948)--(3.855,6.942)--(3.865,6.937)%
  --(3.876,6.931)--(3.886,6.925)--(3.896,6.923)--(3.906,6.917)--(3.916,6.911)--(3.926,6.906)%
  --(3.937,6.900)--(3.947,6.895)--(3.957,6.892)--(3.967,6.886)--(3.977,6.881)--(3.987,6.875)%
  --(3.998,6.869)--(4.008,6.864)--(4.018,6.858)--(4.028,6.855)--(4.038,6.850)--(4.048,6.844)%
  --(4.058,6.839)--(4.069,6.833)--(4.079,6.827)--(4.089,6.825)--(4.099,6.819)--(4.109,6.813)%
  --(4.119,6.808)--(4.130,6.802)--(4.140,6.797)--(4.150,6.794)--(4.160,6.788)--(4.170,6.783)%
  --(4.180,6.777)--(4.191,6.771)--(4.201,6.766)--(4.211,6.763)--(4.221,6.757)--(4.231,6.752)%
  --(4.241,6.746)--(4.252,6.741)--(4.262,6.735)--(4.272,6.732)--(4.282,6.727)--(4.292,6.721)%
  --(4.302,6.715)--(4.313,6.710)--(4.323,6.704)--(4.333,6.701)--(4.343,6.696)--(4.353,6.690)%
  --(4.363,6.685)--(4.373,6.679)--(4.384,6.673)--(4.394,6.668)--(4.404,6.665)--(4.414,6.659)%
  --(4.424,6.654)--(4.434,6.648)--(4.445,6.643)--(4.455,6.637)--(4.465,6.634)--(4.475,6.629)%
  --(4.485,6.623)--(4.495,6.617)--(4.506,6.612)--(4.516,6.606)--(4.526,6.603)--(4.536,6.598)%
  --(4.546,6.592)--(4.556,6.587)--(4.567,6.581)--(4.577,6.575)--(4.587,6.573)--(4.597,6.567)%
  --(4.607,6.561)--(4.617,6.556)--(4.628,6.550)--(4.638,6.545)--(4.648,6.542)--(4.658,6.536)%
  --(4.668,6.531)--(4.678,6.525)--(4.689,6.519)--(4.699,6.514)--(4.709,6.511)--(4.719,6.505)%
  --(4.729,6.500)--(4.739,6.494)--(4.749,6.489)--(4.760,6.483)--(4.770,6.480)--(4.780,6.475)%
  --(4.790,6.469)--(4.800,6.463)--(4.810,6.458)--(4.821,6.452)--(4.831,6.449)--(4.841,6.444)%
  --(4.851,6.438)--(4.861,6.433)--(4.871,6.427)--(4.882,6.421)--(4.892,6.419)--(4.902,6.413)%
  --(4.912,6.407)--(4.922,6.402)--(4.932,6.396)--(4.943,6.391)--(4.953,6.388)--(4.963,6.382)%
  --(4.973,6.376)--(4.983,6.371)--(4.993,6.365)--(5.004,6.360)--(5.014,6.357)--(5.024,6.351)%
  --(5.034,6.346)--(5.044,6.340)--(5.054,6.334)--(5.064,6.329)--(5.075,6.323)--(5.085,6.320)%
  --(5.095,6.315)--(5.105,6.309)--(5.115,6.304)--(5.125,6.298)--(5.136,6.292)--(5.146,6.290)%
  --(5.156,6.284)--(5.166,6.278)--(5.176,6.273)--(5.186,6.267)--(5.197,6.262)--(5.207,6.259)%
  --(5.217,6.253)--(5.227,6.248)--(5.237,6.242)--(5.247,6.236)--(5.258,6.231)--(5.268,6.228)%
  --(5.278,6.222)--(5.288,6.217)--(5.298,6.211)--(5.308,6.206)--(5.319,6.200)--(5.329,6.194)%
  --(5.339,6.192)--(5.349,6.186)--(5.359,6.180)--(5.369,6.175)--(5.379,6.169)--(5.390,6.164)%
  --(5.400,6.161)--(5.410,6.155)--(5.420,6.150)--(5.430,6.144)--(5.440,6.138)--(5.451,6.133)%
  --(5.461,6.127)--(5.471,6.124)--(5.481,6.119)--(5.491,6.113)--(5.501,6.108)--(5.512,6.102)%
  --(5.522,6.096)--(5.532,6.094)--(5.542,6.088)--(5.552,6.082)--(5.562,6.077)--(5.573,6.071)%
  --(5.583,6.066)--(5.593,6.060)--(5.603,6.057)--(5.613,6.052)--(5.623,6.046)--(5.634,6.040)%
  --(5.644,6.035)--(5.654,6.029)--(5.664,6.026)--(5.674,6.021)--(5.684,6.015)--(5.695,6.010)%
  --(5.705,6.004)--(5.715,5.998)--(5.725,5.993)--(5.735,5.990)--(5.745,5.984)--(5.755,5.979)%
  --(5.766,5.973)--(5.776,5.968)--(5.786,5.962)--(5.796,5.956)--(5.806,5.954)--(5.816,5.948)%
  --(5.827,5.942)--(5.837,5.937)--(5.847,5.931)--(5.857,5.926)--(5.867,5.920)--(5.877,5.917)%
  --(5.888,5.912)--(5.898,5.906)--(5.908,5.900)--(5.918,5.895)--(5.928,5.889)--(5.938,5.884)%
  --(5.949,5.881)--(5.959,5.875)--(5.969,5.870)--(5.979,5.864)--(5.989,5.858)--(5.999,5.853)%
  --(6.010,5.847)--(6.020,5.844)--(6.030,5.839)--(6.040,5.833)--(6.050,5.828)--(6.060,5.822)%
  --(6.070,5.817)--(6.081,5.812)--(6.091,5.806)--(6.101,5.802)--(6.111,5.796)--(6.121,5.791)%
  --(6.131,5.785)--(6.142,5.781)--(6.152,5.775)--(6.162,5.770)--(6.172,5.764)--(6.182,5.760)%
  --(6.192,5.754)--(6.203,5.749)--(6.213,5.744)--(6.223,5.739)--(6.233,5.733)--(6.243,5.728)%
  --(6.253,5.723)--(6.264,5.718)--(6.274,5.712)--(6.284,5.707)--(6.294,5.701)--(6.304,5.697)%
  --(6.314,5.691)--(6.325,5.686)--(6.335,5.680)--(6.345,5.676)--(6.355,5.670)--(6.365,5.665)%
  --(6.375,5.659)--(6.386,5.655)--(6.396,5.649)--(6.406,5.644)--(6.416,5.638)--(6.426,5.634)%
  --(6.436,5.628)--(6.446,5.623)--(6.457,5.617)--(6.467,5.611)--(6.477,5.607)--(6.487,5.602)%
  --(6.497,5.596)--(6.507,5.590)--(6.518,5.586)--(6.528,5.581)--(6.538,5.575)--(6.548,5.569)%
  --(6.558,5.564)--(6.568,5.560)--(6.579,5.554)--(6.589,5.548)--(6.599,5.543)--(6.609,5.539)%
  --(6.619,5.533)--(6.629,5.527)--(6.640,5.522)--(6.650,5.516)--(6.660,5.512)--(6.670,5.506)%
  --(6.680,5.501)--(6.690,5.495)--(6.701,5.490)--(6.711,5.485)--(6.721,5.480)--(6.731,5.474)%
  --(6.741,5.469)--(6.751,5.463)--(6.761,5.457)--(6.772,5.453)--(6.782,5.448)--(6.792,5.442)%
  --(6.802,5.436)--(6.812,5.431)--(6.822,5.427)--(6.833,5.421)--(6.843,5.415)--(6.853,5.410)%
  --(6.863,5.404)--(6.873,5.399)--(6.883,5.394)--(6.894,5.389)--(6.904,5.383)--(6.914,5.378)%
  --(6.924,5.372)--(6.934,5.366)--(6.944,5.362)--(6.955,5.357)--(6.965,5.351)--(6.975,5.345)%
  --(6.985,5.340)--(6.995,5.334)--(7.005,5.330)--(7.016,5.324)--(7.026,5.319)--(7.036,5.313)%
  --(7.046,5.308)--(7.056,5.302)--(7.066,5.296)--(7.077,5.292)--(7.087,5.287)--(7.097,5.281)%
  --(7.107,5.275)--(7.117,5.270)--(7.127,5.264)--(7.137,5.259)--(7.148,5.253)--(7.158,5.249)%
  --(7.168,5.243)--(7.178,5.238)--(7.188,5.232)--(7.198,5.226)--(7.209,5.221)--(7.219,5.215)%
  --(7.229,5.210)--(7.239,5.204)--(7.249,5.200)--(7.259,5.194)--(7.270,5.189)--(7.280,5.183)%
  --(7.290,5.177)--(7.300,5.172)--(7.310,5.166)--(7.320,5.161)--(7.331,5.155)--(7.341,5.149)%
  --(7.351,5.145)--(7.361,5.140)--(7.371,5.134)--(7.381,5.128)--(7.392,5.123)--(7.402,5.117)%
  --(7.412,5.112)--(7.422,5.106)--(7.432,5.100)--(7.442,5.095)--(7.452,5.089)--(7.463,5.084)%
  --(7.473,5.078)--(7.483,5.072)--(7.493,5.068)--(7.503,5.063)--(7.513,5.057)--(7.524,5.051)%
  --(7.534,5.046)--(7.544,5.040)--(7.554,5.035)--(7.564,5.029)--(7.574,5.023)--(7.585,5.018)%
  --(7.595,5.012)--(7.605,5.007)--(7.615,5.001)--(7.625,4.995)--(7.635,4.990)--(7.646,4.984)%
  --(7.656,4.979)--(7.666,4.973)--(7.676,4.967)--(7.686,4.962)--(7.696,4.956)--(7.707,4.951)%
  --(7.717,4.945)--(7.727,4.939)--(7.737,4.934)--(7.747,4.928)--(7.757,4.923)--(7.767,4.917)%
  --(7.778,4.911)--(7.788,4.906)--(7.798,4.900)--(7.808,4.895)--(7.818,4.889)--(7.828,4.883)%
  --(7.839,4.878)--(7.849,4.872)--(7.859,4.867)--(7.869,4.861)--(7.879,4.855)--(7.889,4.850)%
  --(7.900,4.844)--(7.910,4.839)--(7.920,4.833)--(7.930,4.827)--(7.940,4.822)--(7.950,4.816)%
  --(7.961,4.811)--(7.971,4.804)--(7.981,4.798)--(7.991,4.792)--(8.001,4.787)--(8.011,4.781)%
  --(8.022,4.776)--(8.032,4.770)--(8.042,4.764)--(8.052,4.759)--(8.062,4.753)--(8.072,4.748)%
  --(8.083,4.742)--(8.093,4.736)--(8.103,4.729)--(8.113,4.724)--(8.123,4.718)--(8.133,4.713)%
  --(8.143,4.707)--(8.154,4.701)--(8.164,4.696)--(8.174,4.690)--(8.184,4.683)--(8.194,4.678)%
  --(8.204,4.672)--(8.215,4.666)--(8.225,4.661)--(8.235,4.655)--(8.245,4.650)--(8.255,4.644)%
  --(8.265,4.637)--(8.276,4.631)--(8.286,4.626)--(8.296,4.620)--(8.306,4.615)--(8.316,4.609)%
  --(8.326,4.602)--(8.337,4.596)--(8.347,4.591)--(8.357,4.585)--(8.367,4.580)--(8.377,4.573)%
  --(8.387,4.567)--(8.398,4.561)--(8.408,4.556)--(8.418,4.550)--(8.428,4.543)--(8.438,4.538)%
  --(8.448,4.532)--(8.458,4.526)--(8.469,4.521)--(8.479,4.514)--(8.489,4.508)--(8.499,4.503)%
  --(8.509,4.497)--(8.519,4.490)--(8.530,4.484)--(8.540,4.479)--(8.550,4.473)--(8.560,4.466)%
  --(8.570,4.461)--(8.580,4.455)--(8.591,4.449)--(8.601,4.442)--(8.611,4.437)--(8.621,4.431)%
  --(8.631,4.426)--(8.641,4.419)--(8.652,4.413)--(8.662,4.407)--(8.672,4.400)--(8.682,4.395)%
  --(8.692,4.389)--(8.702,4.382)--(8.713,4.377)--(8.723,4.371)--(8.733,4.364)--(8.743,4.358)%
  --(8.753,4.353)--(8.763,4.346)--(8.774,4.340)--(8.784,4.335)--(8.794,4.328)--(8.804,4.322)%
  --(8.814,4.316)--(8.824,4.309)--(8.834,4.304)--(8.845,4.298)--(8.855,4.291)--(8.865,4.286)%
  --(8.875,4.279)--(8.885,4.273)--(8.895,4.267)--(8.906,4.260)--(8.916,4.255)--(8.926,4.248)%
  --(8.936,4.242)--(8.946,4.237)--(8.956,4.230)--(8.967,4.224)--(8.977,4.217)--(8.987,4.211)%
  --(8.997,4.204)--(9.007,4.199)--(9.017,4.193)--(9.028,4.186)--(9.038,4.181)--(9.048,4.174)%
  --(9.058,4.168)--(9.068,4.161)--(9.078,4.155)--(9.089,4.148)--(9.099,4.143)--(9.109,4.136)%
  --(9.119,4.130)--(9.129,4.123)--(9.139,4.118)--(9.149,4.111)--(9.160,4.105)--(9.170,4.098)%
  --(9.180,4.091)--(9.190,4.085)--(9.200,4.078)--(9.210,4.073)--(9.221,4.066)--(9.231,4.060)%
  --(9.241,4.053)--(9.251,4.048)--(9.261,4.041)--(9.271,4.034)--(9.282,4.028)--(9.292,4.021)%
  --(9.302,4.015)--(9.312,4.008)--(9.322,4.001)--(9.332,3.996)--(9.343,3.989)--(9.353,3.982)%
  --(9.363,3.976)--(9.373,3.969)--(9.383,3.962)--(9.393,3.957)--(9.404,3.950)--(9.414,3.943)%
  --(9.424,3.937)--(9.434,3.930)--(9.444,3.923)--(9.454,3.917)--(9.465,3.910)--(9.475,3.903)%
  --(9.485,3.896)--(9.495,3.891)--(9.505,3.884)--(9.515,3.877)--(9.525,3.870)--(9.536,3.864)%
  --(9.546,3.857)--(9.556,3.850)--(9.566,3.843)--(9.576,3.838)--(9.586,3.831)--(9.597,3.824)%
  --(9.607,3.817)--(9.617,3.810)--(9.627,3.804)--(9.637,3.797)--(9.647,3.790)--(9.658,3.783)%
  --(9.668,3.776)--(9.678,3.769)--(9.688,3.762)--(9.698,3.756)--(9.708,3.749)--(9.719,3.742)%
  --(9.729,3.735)--(9.739,3.728)--(9.749,3.721)--(9.759,3.714)--(9.769,3.707)--(9.780,3.700)%
  --(9.790,3.693)--(9.800,3.686)--(9.810,3.679)--(9.820,3.672)--(9.830,3.665)--(9.840,3.658)%
  --(9.851,3.651)--(9.861,3.644)--(9.871,3.637)--(9.881,3.630)--(9.891,3.623)--(9.901,3.616)%
  --(9.912,3.609)--(9.922,3.602)--(9.932,3.595)--(9.942,3.588)--(9.952,3.581)--(9.962,3.574)%
  --(9.973,3.566)--(9.983,3.559)--(9.993,3.552)--(10.003,3.545)--(10.013,3.538)--(10.023,3.531)%
  --(10.034,3.524)--(10.044,3.516)--(10.054,3.509)--(10.064,3.502)--(10.074,3.495)--(10.084,3.486)%
  --(10.095,3.479)--(10.105,3.472)--(10.115,3.465)--(10.125,3.457)--(10.135,3.450)--(10.145,3.443)%
  --(10.156,3.436)--(10.166,3.427)--(10.176,3.420)--(10.186,3.413)--(10.196,3.405)--(10.206,3.398)%
  --(10.216,3.390)--(10.227,3.383)--(10.237,3.376)--(10.247,3.367)--(10.257,3.360)--(10.267,3.352)%
  --(10.277,3.345)--(10.288,3.336)--(10.298,3.329)--(10.308,3.321)--(10.318,3.314)--(10.328,3.306)%
  --(10.338,3.299)--(10.349,3.290)--(10.359,3.283)--(10.369,3.275)--(10.379,3.268)--(10.389,3.259)%
  --(10.399,3.251)--(10.410,3.244)--(10.420,3.236)--(10.430,3.229)--(10.440,3.220)--(10.450,3.212)%
  --(10.460,3.203)--(10.471,3.196)--(10.481,3.188)--(10.491,3.180)--(10.501,3.171)--(10.511,3.164)%
  --(10.521,3.156)--(10.531,3.147)--(10.542,3.139)--(10.552,3.131)--(10.562,3.124)--(10.572,3.115)%
  --(10.582,3.107)--(10.592,3.098)--(10.603,3.090)--(10.613,3.082)--(10.623,3.073)--(10.633,3.065)%
  --(10.643,3.056)--(10.653,3.048)--(10.664,3.040)--(10.674,3.031)--(10.684,3.023)--(10.694,3.014)%
  --(10.704,3.006)--(10.714,2.996)--(10.725,2.988)--(10.735,2.980)--(10.745,2.971)--(10.755,2.962)%
  --(10.765,2.953)--(10.775,2.945)--(10.786,2.936)--(10.796,2.926)--(10.806,2.918)--(10.816,2.909)%
  --(10.826,2.900)--(10.836,2.891)--(10.846,2.882)--(10.857,2.873)--(10.867,2.864)--(10.877,2.855)%
  --(10.887,2.846)--(10.897,2.836)--(10.907,2.827)--(10.918,2.818)--(10.928,2.809)--(10.938,2.799)%
  --(10.948,2.790)--(10.958,2.781)--(10.968,2.771)--(10.979,2.762)--(10.989,2.752)--(10.999,2.742)%
  --(11.009,2.733)--(11.019,2.723)--(11.029,2.714)--(11.040,2.704)--(11.050,2.694)--(11.060,2.684)%
  --(11.070,2.674)--(11.080,2.665)--(11.090,2.654)--(11.101,2.644)--(11.111,2.635)--(11.121,2.624)%
  --(11.131,2.614)--(11.141,2.604)--(11.151,2.593)--(11.162,2.583)--(11.172,2.572)--(11.182,2.562)%
  --(11.192,2.551)--(11.202,2.541)--(11.212,2.530)--(11.222,2.520)--(11.233,2.509)--(11.243,2.498)%
  --(11.253,2.487)--(11.263,2.476)--(11.273,2.465)--(11.283,2.454)--(11.294,2.443)--(11.304,2.432)%
  --(11.314,2.420)--(11.324,2.408)--(11.334,2.397)--(11.344,2.385)--(11.355,2.374)--(11.365,2.362)%
  --(11.375,2.350)--(11.385,2.338)--(11.395,2.327)--(11.405,2.315)--(11.416,2.302)--(11.426,2.289)%
  --(11.436,2.278)--(11.446,2.265)--(11.456,2.252)--(11.466,2.240)--(11.477,2.226)--(11.487,2.214)%
  --(11.497,2.201)--(11.507,2.187)--(11.517,2.174)--(11.527,2.160)--(11.537,2.147)--(11.548,2.133)%
  --(11.558,2.119)--(11.568,2.105)--(11.578,2.091)--(11.588,2.076)--(11.598,2.061)--(11.609,2.047)%
  --(11.619,2.032)--(11.629,2.016)--(11.639,2.001)--(11.649,1.986)--(11.659,1.970)--(11.670,1.953)%
  --(11.680,1.937)--(11.690,1.921)--(11.700,1.904)--(11.710,1.886)--(11.720,1.869)--(11.731,1.851)%
  --(11.741,1.833)--(11.751,1.814)--(11.761,1.795)--(11.771,1.776)--(11.781,1.756)--(11.792,1.736)%
  --(11.802,1.715)--(11.812,1.693)--(11.822,1.671)--(11.832,1.648)--(11.842,1.624)--(11.853,1.600)%
  --(11.863,1.574)--(11.873,1.548)--(11.883,1.520)--(11.893,1.490)--(11.903,1.459)--(11.913,1.425)%
  --(11.924,1.389)--(11.934,1.350)--(11.944,1.305)--(11.954,1.253)--(11.964,1.189);
\gpcolor{color=gp lt color border}
\draw[gp path] (1.504,8.631)--(1.504,0.985)--(13.447,0.985)--(13.447,8.631)--cycle;
%% coordinates of the plot area
\gpdefrectangularnode{gp plot 1}{\pgfpoint{1.504cm}{0.985cm}}{\pgfpoint{13.447cm}{8.631cm}}
\end{tikzpicture}
%% gnuplot variables

	\caption{Gráfico mostrando a massa em função da densidade bariônica para fração de prótons 1/2. Note que $M$ diminui até zero em $\rho \approx 0.35$. Nesse ponto ocorre a restauração da simetria quiral. \protect[Parameters: eNJL1; Proton fraction: 1/2] }
	\label{Fig:mass_graph}
\end{figure*}

\begin{figure*}
	\begin{tikzpicture}[gnuplot]
%% generated with GNUPLOT 5.0p2 (Lua 5.2; terminal rev. 99, script rev. 100)
%% Fri Mar  4 16:19:10 2016
\path (0.000,0.000) rectangle (14.000,9.000);
\gpcolor{color=gp lt color border}
\gpsetlinetype{gp lt border}
\gpsetdashtype{gp dt solid}
\gpsetlinewidth{1.00}
\draw[gp path] (1.688,0.985)--(1.868,0.985);
\draw[gp path] (13.447,0.985)--(13.267,0.985);
\node[gp node right] at (1.504,0.985) {$-0.5$};
\draw[gp path] (1.688,1.750)--(1.868,1.750);
\draw[gp path] (13.447,1.750)--(13.267,1.750);
\node[gp node right] at (1.504,1.750) {$-0.45$};
\draw[gp path] (1.688,2.514)--(1.868,2.514);
\draw[gp path] (13.447,2.514)--(13.267,2.514);
\node[gp node right] at (1.504,2.514) {$-0.4$};
\draw[gp path] (1.688,3.279)--(1.868,3.279);
\draw[gp path] (13.447,3.279)--(13.267,3.279);
\node[gp node right] at (1.504,3.279) {$-0.35$};
\draw[gp path] (1.688,4.043)--(1.868,4.043);
\draw[gp path] (13.447,4.043)--(13.267,4.043);
\node[gp node right] at (1.504,4.043) {$-0.3$};
\draw[gp path] (1.688,4.808)--(1.868,4.808);
\draw[gp path] (13.447,4.808)--(13.267,4.808);
\node[gp node right] at (1.504,4.808) {$-0.25$};
\draw[gp path] (1.688,5.573)--(1.868,5.573);
\draw[gp path] (13.447,5.573)--(13.267,5.573);
\node[gp node right] at (1.504,5.573) {$-0.2$};
\draw[gp path] (1.688,6.337)--(1.868,6.337);
\draw[gp path] (13.447,6.337)--(13.267,6.337);
\node[gp node right] at (1.504,6.337) {$-0.15$};
\draw[gp path] (1.688,7.102)--(1.868,7.102);
\draw[gp path] (13.447,7.102)--(13.267,7.102);
\node[gp node right] at (1.504,7.102) {$-0.1$};
\draw[gp path] (1.688,7.866)--(1.868,7.866);
\draw[gp path] (13.447,7.866)--(13.267,7.866);
\node[gp node right] at (1.504,7.866) {$-0.05$};
\draw[gp path] (1.688,8.631)--(1.868,8.631);
\draw[gp path] (13.447,8.631)--(13.267,8.631);
\node[gp node right] at (1.504,8.631) {$0$};
\draw[gp path] (1.688,0.985)--(1.688,1.165);
\draw[gp path] (1.688,8.631)--(1.688,8.451);
\node[gp node center] at (1.688,0.677) {$0$};
\draw[gp path] (3.158,0.985)--(3.158,1.165);
\draw[gp path] (3.158,8.631)--(3.158,8.451);
\node[gp node center] at (3.158,0.677) {$0.05$};
\draw[gp path] (4.628,0.985)--(4.628,1.165);
\draw[gp path] (4.628,8.631)--(4.628,8.451);
\node[gp node center] at (4.628,0.677) {$0.1$};
\draw[gp path] (6.098,0.985)--(6.098,1.165);
\draw[gp path] (6.098,8.631)--(6.098,8.451);
\node[gp node center] at (6.098,0.677) {$0.15$};
\draw[gp path] (7.568,0.985)--(7.568,1.165);
\draw[gp path] (7.568,8.631)--(7.568,8.451);
\node[gp node center] at (7.568,0.677) {$0.2$};
\draw[gp path] (9.037,0.985)--(9.037,1.165);
\draw[gp path] (9.037,8.631)--(9.037,8.451);
\node[gp node center] at (9.037,0.677) {$0.25$};
\draw[gp path] (10.507,0.985)--(10.507,1.165);
\draw[gp path] (10.507,8.631)--(10.507,8.451);
\node[gp node center] at (10.507,0.677) {$0.3$};
\draw[gp path] (11.977,0.985)--(11.977,1.165);
\draw[gp path] (11.977,8.631)--(11.977,8.451);
\node[gp node center] at (11.977,0.677) {$0.35$};
\draw[gp path] (13.447,0.985)--(13.447,1.165);
\draw[gp path] (13.447,8.631)--(13.447,8.451);
\node[gp node center] at (13.447,0.677) {$0.4$};
\draw[gp path] (1.688,8.631)--(1.688,0.985)--(13.447,0.985)--(13.447,8.631)--cycle;
\node[gp node center,rotate=-270] at (0.246,4.808) {$\rho_s$ ($\rm{fm}^{-3}$)};
\node[gp node center] at (7.567,0.215) {$\rho$ ($\rm{fm}^{-3}$)};
\gpcolor{rgb color={0.580,0.000,0.827}}
\draw[gp path] (1.992,1.310)--(2.002,1.316)--(2.012,1.322)--(2.022,1.328)--(2.032,1.333)%
  --(2.042,1.339)--(2.052,1.345)--(2.062,1.350)--(2.072,1.356)--(2.082,1.362)--(2.092,1.368)%
  --(2.102,1.373)--(2.112,1.379)--(2.122,1.385)--(2.132,1.390)--(2.142,1.396)--(2.152,1.402)%
  --(2.162,1.408)--(2.172,1.413)--(2.182,1.419)--(2.192,1.425)--(2.202,1.431)--(2.212,1.436)%
  --(2.222,1.442)--(2.232,1.448)--(2.242,1.454)--(2.252,1.459)--(2.262,1.465)--(2.272,1.470)%
  --(2.282,1.476)--(2.292,1.482)--(2.302,1.488)--(2.312,1.493)--(2.322,1.499)--(2.332,1.505)%
  --(2.342,1.511)--(2.352,1.516)--(2.362,1.522)--(2.372,1.528)--(2.382,1.533)--(2.392,1.539)%
  --(2.402,1.545)--(2.412,1.551)--(2.422,1.556)--(2.432,1.562)--(2.442,1.568)--(2.452,1.574)%
  --(2.462,1.579)--(2.472,1.585)--(2.482,1.591)--(2.492,1.596)--(2.502,1.602)--(2.512,1.608)%
  --(2.522,1.614)--(2.532,1.619)--(2.542,1.625)--(2.552,1.631)--(2.562,1.637)--(2.572,1.642)%
  --(2.582,1.648)--(2.592,1.654)--(2.602,1.659)--(2.612,1.665)--(2.622,1.671)--(2.632,1.677)%
  --(2.642,1.682)--(2.652,1.688)--(2.662,1.694)--(2.672,1.699)--(2.682,1.705)--(2.692,1.711)%
  --(2.702,1.717)--(2.712,1.722)--(2.722,1.728)--(2.732,1.734)--(2.742,1.740)--(2.752,1.745)%
  --(2.762,1.751)--(2.772,1.757)--(2.782,1.762)--(2.792,1.768)--(2.802,1.774)--(2.812,1.780)%
  --(2.822,1.786)--(2.832,1.791)--(2.842,1.797)--(2.852,1.803)--(2.862,1.808)--(2.872,1.814)%
  --(2.882,1.820)--(2.892,1.826)--(2.902,1.831)--(2.912,1.837)--(2.922,1.843)--(2.932,1.848)%
  --(2.942,1.854)--(2.952,1.860)--(2.962,1.866)--(2.972,1.872)--(2.982,1.877)--(2.992,1.883)%
  --(3.003,1.889)--(3.013,1.894)--(3.023,1.900)--(3.033,1.906)--(3.043,1.912)--(3.053,1.917)%
  --(3.063,1.923)--(3.073,1.929)--(3.083,1.935)--(3.093,1.940)--(3.103,1.946)--(3.113,1.952)%
  --(3.123,1.958)--(3.133,1.963)--(3.143,1.969)--(3.153,1.975)--(3.163,1.980)--(3.173,1.986)%
  --(3.183,1.992)--(3.193,1.998)--(3.203,2.004)--(3.213,2.009)--(3.223,2.015)--(3.233,2.021)%
  --(3.243,2.026)--(3.253,2.032)--(3.263,2.038)--(3.273,2.044)--(3.283,2.050)--(3.293,2.055)%
  --(3.303,2.061)--(3.313,2.067)--(3.323,2.073)--(3.333,2.078)--(3.343,2.084)--(3.353,2.090)%
  --(3.363,2.096)--(3.373,2.101)--(3.383,2.107)--(3.393,2.113)--(3.403,2.119)--(3.413,2.124)%
  --(3.423,2.130)--(3.433,2.136)--(3.443,2.141)--(3.453,2.147)--(3.463,2.153)--(3.473,2.159)%
  --(3.483,2.165)--(3.493,2.171)--(3.503,2.176)--(3.513,2.182)--(3.523,2.188)--(3.533,2.193)%
  --(3.543,2.199)--(3.553,2.205)--(3.563,2.211)--(3.573,2.217)--(3.583,2.222)--(3.593,2.228)%
  --(3.603,2.234)--(3.613,2.240)--(3.623,2.245)--(3.633,2.251)--(3.643,2.257)--(3.653,2.263)%
  --(3.663,2.268)--(3.673,2.274)--(3.683,2.280)--(3.693,2.286)--(3.703,2.292)--(3.713,2.297)%
  --(3.723,2.303)--(3.733,2.309)--(3.743,2.315)--(3.753,2.320)--(3.763,2.326)--(3.773,2.332)%
  --(3.783,2.338)--(3.793,2.343)--(3.803,2.349)--(3.813,2.355)--(3.823,2.361)--(3.833,2.367)%
  --(3.843,2.373)--(3.853,2.378)--(3.863,2.384)--(3.873,2.390)--(3.883,2.395)--(3.893,2.401)%
  --(3.903,2.407)--(3.913,2.413)--(3.923,2.418)--(3.933,2.424)--(3.943,2.430)--(3.953,2.436)%
  --(3.963,2.442)--(3.973,2.448)--(3.983,2.453)--(3.993,2.459)--(4.003,2.465)--(4.013,2.471)%
  --(4.023,2.477)--(4.033,2.482)--(4.043,2.488)--(4.053,2.494)--(4.063,2.500)--(4.073,2.505)%
  --(4.083,2.511)--(4.093,2.517)--(4.103,2.523)--(4.113,2.528)--(4.123,2.534)--(4.133,2.540)%
  --(4.143,2.546)--(4.153,2.552)--(4.163,2.558)--(4.173,2.563)--(4.183,2.569)--(4.193,2.575)%
  --(4.203,2.581)--(4.213,2.587)--(4.223,2.592)--(4.233,2.598)--(4.243,2.604)--(4.253,2.610)%
  --(4.263,2.615)--(4.273,2.621)--(4.283,2.627)--(4.293,2.633)--(4.303,2.639)--(4.313,2.644)%
  --(4.323,2.650)--(4.333,2.656)--(4.343,2.662)--(4.353,2.667)--(4.363,2.673)--(4.373,2.679)%
  --(4.383,2.685)--(4.393,2.691)--(4.403,2.697)--(4.413,2.702)--(4.423,2.708)--(4.433,2.714)%
  --(4.443,2.720)--(4.453,2.726)--(4.463,2.732)--(4.473,2.737)--(4.483,2.743)--(4.493,2.749)%
  --(4.503,2.755)--(4.513,2.761)--(4.523,2.767)--(4.533,2.772)--(4.543,2.778)--(4.553,2.784)%
  --(4.563,2.790)--(4.573,2.796)--(4.583,2.801)--(4.593,2.807)--(4.603,2.813)--(4.613,2.819)%
  --(4.623,2.825)--(4.633,2.830)--(4.643,2.836)--(4.653,2.842)--(4.663,2.848)--(4.673,2.854)%
  --(4.683,2.860)--(4.693,2.865)--(4.703,2.871)--(4.713,2.877)--(4.723,2.883)--(4.733,2.889)%
  --(4.743,2.894)--(4.753,2.900)--(4.763,2.906)--(4.773,2.912)--(4.783,2.918)--(4.793,2.924)%
  --(4.803,2.929)--(4.813,2.935)--(4.823,2.941)--(4.833,2.947)--(4.843,2.953)--(4.853,2.959)%
  --(4.863,2.964)--(4.873,2.970)--(4.883,2.976)--(4.893,2.982)--(4.903,2.988)--(4.913,2.994)%
  --(4.923,2.999)--(4.933,3.005)--(4.944,3.011)--(4.954,3.017)--(4.964,3.023)--(4.974,3.029)%
  --(4.984,3.035)--(4.994,3.040)--(5.004,3.046)--(5.014,3.052)--(5.024,3.058)--(5.034,3.064)%
  --(5.044,3.070)--(5.054,3.076)--(5.064,3.081)--(5.074,3.087)--(5.084,3.093)--(5.094,3.099)%
  --(5.104,3.105)--(5.114,3.111)--(5.124,3.117)--(5.134,3.123)--(5.144,3.128)--(5.154,3.134)%
  --(5.164,3.140)--(5.174,3.146)--(5.184,3.152)--(5.194,3.158)--(5.204,3.164)--(5.214,3.169)%
  --(5.224,3.175)--(5.234,3.181)--(5.244,3.187)--(5.254,3.193)--(5.264,3.199)--(5.274,3.204)%
  --(5.284,3.210)--(5.294,3.216)--(5.304,3.222)--(5.314,3.228)--(5.324,3.234)--(5.334,3.239)%
  --(5.344,3.245)--(5.354,3.251)--(5.364,3.257)--(5.374,3.263)--(5.384,3.269)--(5.394,3.275)%
  --(5.404,3.281)--(5.414,3.287)--(5.424,3.293)--(5.434,3.299)--(5.444,3.305)--(5.454,3.310)%
  --(5.464,3.316)--(5.474,3.322)--(5.484,3.328)--(5.494,3.334)--(5.504,3.340)--(5.514,3.346)%
  --(5.524,3.351)--(5.534,3.357)--(5.544,3.363)--(5.554,3.369)--(5.564,3.375)--(5.574,3.381)%
  --(5.584,3.387)--(5.594,3.393)--(5.604,3.398)--(5.614,3.404)--(5.624,3.410)--(5.634,3.416)%
  --(5.644,3.422)--(5.654,3.428)--(5.664,3.434)--(5.674,3.440)--(5.684,3.446)--(5.694,3.452)%
  --(5.704,3.458)--(5.714,3.464)--(5.724,3.469)--(5.734,3.475)--(5.744,3.481)--(5.754,3.487)%
  --(5.764,3.493)--(5.774,3.499)--(5.784,3.505)--(5.794,3.511)--(5.804,3.517)--(5.814,3.523)%
  --(5.824,3.529)--(5.834,3.535)--(5.844,3.541)--(5.854,3.546)--(5.864,3.552)--(5.874,3.558)%
  --(5.884,3.564)--(5.894,3.570)--(5.904,3.576)--(5.914,3.582)--(5.924,3.588)--(5.934,3.594)%
  --(5.944,3.600)--(5.954,3.606)--(5.964,3.612)--(5.974,3.618)--(5.984,3.624)--(5.994,3.629)%
  --(6.004,3.635)--(6.014,3.641)--(6.024,3.647)--(6.034,3.653)--(6.044,3.659)--(6.054,3.665)%
  --(6.064,3.671)--(6.074,3.677)--(6.084,3.683)--(6.094,3.689)--(6.104,3.695)--(6.114,3.701)%
  --(6.124,3.707)--(6.134,3.712)--(6.144,3.718)--(6.154,3.724)--(6.164,3.730)--(6.174,3.736)%
  --(6.184,3.742)--(6.194,3.748)--(6.204,3.754)--(6.214,3.760)--(6.224,3.766)--(6.234,3.772)%
  --(6.244,3.778)--(6.254,3.784)--(6.264,3.790)--(6.274,3.796)--(6.284,3.802)--(6.294,3.808)%
  --(6.304,3.814)--(6.314,3.820)--(6.324,3.826)--(6.334,3.832)--(6.344,3.838)--(6.354,3.844)%
  --(6.364,3.850)--(6.374,3.856)--(6.384,3.862)--(6.394,3.868)--(6.404,3.874)--(6.414,3.880)%
  --(6.424,3.886)--(6.434,3.892)--(6.444,3.898)--(6.454,3.903)--(6.464,3.910)--(6.474,3.916)%
  --(6.484,3.922)--(6.494,3.927)--(6.504,3.934)--(6.514,3.940)--(6.524,3.946)--(6.534,3.951)%
  --(6.544,3.957)--(6.554,3.964)--(6.564,3.970)--(6.574,3.976)--(6.584,3.981)--(6.594,3.988)%
  --(6.604,3.994)--(6.614,4.000)--(6.624,4.005)--(6.634,4.012)--(6.644,4.018)--(6.654,4.024)%
  --(6.664,4.030)--(6.674,4.036)--(6.684,4.042)--(6.694,4.048)--(6.704,4.054)--(6.714,4.060)%
  --(6.724,4.066)--(6.734,4.072)--(6.744,4.078)--(6.754,4.084)--(6.764,4.090)--(6.774,4.096)%
  --(6.784,4.102)--(6.794,4.108)--(6.804,4.114)--(6.814,4.120)--(6.824,4.126)--(6.834,4.132)%
  --(6.844,4.138)--(6.854,4.144)--(6.864,4.150)--(6.874,4.156)--(6.885,4.162)--(6.895,4.168)%
  --(6.905,4.175)--(6.915,4.181)--(6.925,4.186)--(6.935,4.193)--(6.945,4.199)--(6.955,4.205)%
  --(6.965,4.211)--(6.975,4.217)--(6.985,4.223)--(6.995,4.229)--(7.005,4.235)--(7.015,4.241)%
  --(7.025,4.247)--(7.035,4.253)--(7.045,4.259)--(7.055,4.265)--(7.065,4.272)--(7.075,4.278)%
  --(7.085,4.284)--(7.095,4.290)--(7.105,4.296)--(7.115,4.302)--(7.125,4.308)--(7.135,4.314)%
  --(7.145,4.320)--(7.155,4.326)--(7.165,4.333)--(7.175,4.338)--(7.185,4.344)--(7.195,4.351)%
  --(7.205,4.357)--(7.215,4.363)--(7.225,4.369)--(7.235,4.375)--(7.245,4.381)--(7.255,4.387)%
  --(7.265,4.393)--(7.275,4.399)--(7.285,4.406)--(7.295,4.412)--(7.305,4.418)--(7.315,4.424)%
  --(7.325,4.430)--(7.335,4.436)--(7.345,4.442)--(7.355,4.448)--(7.365,4.455)--(7.375,4.461)%
  --(7.385,4.467)--(7.395,4.473)--(7.405,4.479)--(7.415,4.485)--(7.425,4.492)--(7.435,4.498)%
  --(7.445,4.504)--(7.455,4.510)--(7.465,4.516)--(7.475,4.522)--(7.485,4.528)--(7.495,4.535)%
  --(7.505,4.541)--(7.515,4.547)--(7.525,4.553)--(7.535,4.559)--(7.545,4.565)--(7.555,4.572)%
  --(7.565,4.578)--(7.575,4.584)--(7.585,4.590)--(7.595,4.596)--(7.605,4.602)--(7.615,4.608)%
  --(7.625,4.615)--(7.635,4.621)--(7.645,4.627)--(7.655,4.633)--(7.665,4.639)--(7.675,4.646)%
  --(7.685,4.652)--(7.695,4.658)--(7.705,4.664)--(7.715,4.671)--(7.725,4.677)--(7.735,4.683)%
  --(7.745,4.689)--(7.755,4.695)--(7.765,4.702)--(7.775,4.708)--(7.785,4.714)--(7.795,4.720)%
  --(7.805,4.727)--(7.815,4.733)--(7.825,4.739)--(7.835,4.745)--(7.845,4.751)--(7.855,4.758)%
  --(7.865,4.764)--(7.875,4.770)--(7.885,4.776)--(7.895,4.783)--(7.905,4.789)--(7.915,4.795)%
  --(7.925,4.801)--(7.935,4.808)--(7.945,4.814)--(7.955,4.820)--(7.965,4.826)--(7.975,4.833)%
  --(7.985,4.839)--(7.995,4.845)--(8.005,4.851)--(8.015,4.858)--(8.025,4.864)--(8.035,4.870)%
  --(8.045,4.876)--(8.055,4.883)--(8.065,4.889)--(8.075,4.896)--(8.085,4.902)--(8.095,4.908)%
  --(8.105,4.914)--(8.115,4.921)--(8.125,4.927)--(8.135,4.933)--(8.145,4.939)--(8.155,4.946)%
  --(8.165,4.952)--(8.175,4.958)--(8.185,4.965)--(8.195,4.971)--(8.205,4.978)--(8.215,4.984)%
  --(8.225,4.990)--(8.235,4.996)--(8.245,5.003)--(8.255,5.009)--(8.265,5.016)--(8.275,5.022)%
  --(8.285,5.028)--(8.295,5.035)--(8.305,5.041)--(8.315,5.047)--(8.325,5.053)--(8.335,5.060)%
  --(8.345,5.066)--(8.355,5.073)--(8.365,5.079)--(8.375,5.085)--(8.385,5.092)--(8.395,5.098)%
  --(8.405,5.105)--(8.415,5.111)--(8.425,5.117)--(8.435,5.124)--(8.445,5.130)--(8.455,5.137)%
  --(8.465,5.143)--(8.475,5.149)--(8.485,5.156)--(8.495,5.162)--(8.505,5.169)--(8.515,5.175)%
  --(8.525,5.182)--(8.535,5.188)--(8.545,5.194)--(8.555,5.201)--(8.565,5.207)--(8.575,5.214)%
  --(8.585,5.220)--(8.595,5.227)--(8.605,5.233)--(8.615,5.240)--(8.625,5.246)--(8.635,5.253)%
  --(8.645,5.259)--(8.655,5.265)--(8.665,5.272)--(8.675,5.279)--(8.685,5.285)--(8.695,5.291)%
  --(8.705,5.298)--(8.715,5.304)--(8.725,5.311)--(8.735,5.317)--(8.745,5.324)--(8.755,5.330)%
  --(8.765,5.337)--(8.775,5.344)--(8.785,5.350)--(8.795,5.356)--(8.805,5.363)--(8.815,5.370)%
  --(8.826,5.376)--(8.836,5.383)--(8.846,5.389)--(8.856,5.396)--(8.866,5.403)--(8.876,5.409)%
  --(8.886,5.415)--(8.896,5.422)--(8.906,5.429)--(8.916,5.435)--(8.926,5.442)--(8.936,5.448)%
  --(8.946,5.455)--(8.956,5.462)--(8.966,5.468)--(8.976,5.475)--(8.986,5.481)--(8.996,5.488)%
  --(9.006,5.495)--(9.016,5.501)--(9.026,5.508)--(9.036,5.514)--(9.046,5.521)--(9.056,5.528)%
  --(9.066,5.535)--(9.076,5.541)--(9.086,5.547)--(9.096,5.554)--(9.106,5.561)--(9.116,5.568)%
  --(9.126,5.574)--(9.136,5.581)--(9.146,5.588)--(9.156,5.595)--(9.166,5.601)--(9.176,5.608)%
  --(9.186,5.614)--(9.196,5.621)--(9.206,5.628)--(9.216,5.635)--(9.226,5.641)--(9.236,5.648)%
  --(9.246,5.655)--(9.256,5.662)--(9.266,5.669)--(9.276,5.675)--(9.286,5.682)--(9.296,5.689)%
  --(9.306,5.696)--(9.316,5.702)--(9.326,5.709)--(9.336,5.716)--(9.346,5.722)--(9.356,5.730)%
  --(9.366,5.736)--(9.376,5.743)--(9.386,5.750)--(9.396,5.757)--(9.406,5.764)--(9.416,5.771)%
  --(9.426,5.777)--(9.436,5.784)--(9.446,5.791)--(9.456,5.798)--(9.466,5.805)--(9.476,5.812)%
  --(9.486,5.818)--(9.496,5.825)--(9.506,5.832)--(9.516,5.839)--(9.526,5.846)--(9.536,5.853)%
  --(9.546,5.860)--(9.556,5.867)--(9.566,5.874)--(9.576,5.881)--(9.586,5.888)--(9.596,5.894)%
  --(9.606,5.902)--(9.616,5.909)--(9.626,5.916)--(9.636,5.922)--(9.646,5.929)--(9.656,5.936)%
  --(9.666,5.944)--(9.676,5.951)--(9.686,5.957)--(9.696,5.964)--(9.706,5.972)--(9.716,5.979)%
  --(9.726,5.986)--(9.736,5.993)--(9.746,6.000)--(9.756,6.007)--(9.766,6.014)--(9.776,6.021)%
  --(9.786,6.028)--(9.796,6.035)--(9.806,6.042)--(9.816,6.050)--(9.826,6.057)--(9.836,6.064)%
  --(9.846,6.071)--(9.856,6.078)--(9.866,6.085)--(9.876,6.093)--(9.886,6.100)--(9.896,6.107)%
  --(9.906,6.114)--(9.916,6.121)--(9.926,6.129)--(9.936,6.136)--(9.946,6.143)--(9.956,6.150)%
  --(9.966,6.158)--(9.976,6.165)--(9.986,6.172)--(9.996,6.179)--(10.006,6.186)--(10.016,6.194)%
  --(10.026,6.202)--(10.036,6.209)--(10.046,6.216)--(10.056,6.223)--(10.066,6.231)--(10.076,6.238)%
  --(10.086,6.245)--(10.096,6.253)--(10.106,6.260)--(10.116,6.268)--(10.126,6.275)--(10.136,6.283)%
  --(10.146,6.290)--(10.156,6.297)--(10.166,6.305)--(10.176,6.313)--(10.186,6.320)--(10.196,6.327)%
  --(10.206,6.335)--(10.216,6.343)--(10.226,6.350)--(10.236,6.357)--(10.246,6.365)--(10.256,6.373)%
  --(10.266,6.381)--(10.276,6.388)--(10.286,6.395)--(10.296,6.403)--(10.306,6.411)--(10.316,6.419)%
  --(10.326,6.426)--(10.336,6.434)--(10.346,6.441)--(10.356,6.450)--(10.366,6.457)--(10.376,6.465)%
  --(10.386,6.472)--(10.396,6.480)--(10.406,6.488)--(10.416,6.496)--(10.426,6.503)--(10.436,6.511)%
  --(10.446,6.520)--(10.456,6.527)--(10.466,6.535)--(10.476,6.543)--(10.486,6.551)--(10.496,6.559)%
  --(10.506,6.567)--(10.516,6.574)--(10.526,6.583)--(10.536,6.591)--(10.546,6.599)--(10.556,6.606)%
  --(10.566,6.615)--(10.576,6.623)--(10.586,6.631)--(10.596,6.639)--(10.606,6.647)--(10.616,6.655)%
  --(10.626,6.663)--(10.636,6.671)--(10.646,6.680)--(10.656,6.688)--(10.666,6.696)--(10.676,6.704)%
  --(10.686,6.713)--(10.696,6.721)--(10.706,6.729)--(10.716,6.737)--(10.726,6.746)--(10.736,6.754)%
  --(10.746,6.763)--(10.756,6.771)--(10.767,6.780)--(10.777,6.788)--(10.787,6.797)--(10.797,6.805)%
  --(10.807,6.814)--(10.817,6.822)--(10.827,6.831)--(10.837,6.840)--(10.847,6.848)--(10.857,6.857)%
  --(10.867,6.866)--(10.877,6.874)--(10.887,6.883)--(10.897,6.891)--(10.907,6.900)--(10.917,6.909)%
  --(10.927,6.918)--(10.937,6.927)--(10.947,6.936)--(10.957,6.945)--(10.967,6.954)--(10.977,6.963)%
  --(10.987,6.972)--(10.997,6.981)--(11.007,6.990)--(11.017,6.999)--(11.027,7.008)--(11.037,7.018)%
  --(11.047,7.027)--(11.057,7.036)--(11.067,7.045)--(11.077,7.055)--(11.087,7.064)--(11.097,7.074)%
  --(11.107,7.083)--(11.117,7.092)--(11.127,7.102)--(11.137,7.112)--(11.147,7.121)--(11.157,7.131)%
  --(11.167,7.141)--(11.177,7.150)--(11.187,7.160)--(11.197,7.170)--(11.207,7.180)--(11.217,7.190)%
  --(11.227,7.200)--(11.237,7.210)--(11.247,7.220)--(11.257,7.230)--(11.267,7.241)--(11.277,7.251)%
  --(11.287,7.261)--(11.297,7.271)--(11.307,7.282)--(11.317,7.292)--(11.327,7.303)--(11.337,7.313)%
  --(11.347,7.324)--(11.357,7.335)--(11.367,7.346)--(11.377,7.357)--(11.387,7.367)--(11.397,7.379)%
  --(11.407,7.390)--(11.417,7.401)--(11.427,7.412)--(11.437,7.423)--(11.447,7.435)--(11.457,7.447)%
  --(11.467,7.458)--(11.477,7.470)--(11.487,7.481)--(11.497,7.493)--(11.507,7.505)--(11.517,7.517)%
  --(11.527,7.529)--(11.537,7.542)--(11.547,7.554)--(11.557,7.567)--(11.567,7.579)--(11.577,7.592)%
  --(11.587,7.605)--(11.597,7.618)--(11.607,7.631)--(11.617,7.645)--(11.627,7.658)--(11.637,7.672)%
  --(11.647,7.685)--(11.657,7.700)--(11.667,7.714)--(11.677,7.728)--(11.687,7.743)--(11.697,7.757)%
  --(11.707,7.772)--(11.717,7.788)--(11.727,7.803)--(11.737,7.819)--(11.747,7.835)--(11.757,7.851)%
  --(11.767,7.867)--(11.777,7.885)--(11.787,7.902)--(11.797,7.920)--(11.807,7.938)--(11.817,7.956)%
  --(11.827,7.975)--(11.837,7.995)--(11.847,8.015)--(11.857,8.035)--(11.867,8.057)--(11.877,8.079)%
  --(11.887,8.102)--(11.897,8.126)--(11.907,8.151)--(11.917,8.178)--(11.927,8.206)--(11.937,8.236)%
  --(11.947,8.269)--(11.957,8.305)--(11.967,8.344)--(11.977,8.391)--(11.987,8.448);
\gpcolor{color=gp lt color border}
\draw[gp path] (1.688,8.631)--(1.688,0.985)--(13.447,0.985)--(13.447,8.631)--cycle;
%% coordinates of the plot area
\gpdefrectangularnode{gp plot 1}{\pgfpoint{1.688cm}{0.985cm}}{\pgfpoint{13.447cm}{8.631cm}}
\end{tikzpicture}
%% gnuplot variables

	\caption{Gráfico da densidade escalar em função da densidade bariônica para fração de prótons 1/2. O resultado mostrado nesse gráfico é obtido juntamente com os resultados mostrados na Fig.~\ref{Fig:mass_graph} através da solução da Equação~\ref{Eq:Gap_zero}. \protect[Parameters: eNJL1; Proton fraction: 1/2] }
	\label{Fig:scalar_density_graph}
\end{figure*}

\FloatBarrier
%%%%%%%%%%%%%%%%%%%%%%%%%%%%%%%%%%%%%%%%%%%%%%%%%%%%%%%%%%%%%%%%%%%%%%%%%%%%%
\section{Potenciais químicos, termodinâmico, pressão, e densidade de energia}
%%%%%%%%%%%%%%%%%%%%%%%%%%%%%%%%%%%%%%%%%%%%%%%%%%%%%%%%%%%%%%%%%%%%%%%%%%%%%

As figuras abaixo mostram os resultados obtidos através das massas e densidades escalares obtidas na seção anterior, utilizando as Equações~\eqref{Eq:Potenciais_Quimicos}, \eqref{Eq:potencial_termodinamico}, \eqref{Eq:Pressao} e~\eqref{Eq:Densidade_energia}. Para a correta reprodução dos resultados para a pressão e para a densidade de energia, é essencial que o valor de $\varepsilon_0$ seja calculado. Podemos fazê-lo através de\footnote{Não sei de onde isso vem. Tem algum outro termo nessa equação ($\propto G_{s\rho},~G_{v\rho},~\dots$? }
\begin{equation}\label{Eq:Def_dens_energia_vacuo}
	\varepsilon_0 = (\varepsilon_{\rm{kin}})_0 - G_s [(\rho_s)_0]^2
\end{equation}
%
onde\footnote{As somas são sobre prótons e nêutrons, mas como os resultados são idênticos, podemos só multiplicar por 2.}
\begin{align}
	(\varepsilon_{\rm{kin}})_0 &= \sum_i \frac{N_c}{\pi^2} (F_2(m_N,0) - F_2(m_N, \Lambda)) \label{Eq:Def_dens_energia_cin_vacuo}\\
	(\rho_s)_0 &= \sum_i \frac{m_N}{\pi^2} (F_0(m_N, 0) - F_0(m_N, \Lambda)) \label{Eq:Def_dens_escalar_vacuo}
\end{align}

Para que as dimensões fiquem corretas
\begin{itemize}
	\item O segundo termo da Equação~\eqref{Eq:Def_dens_energia_vacuo} deve ser multiplicado por $\hbar c$.
	\item O lado direito das Equações~\eqref{Eq:Def_dens_energia_cin_vacuo} e \eqref{Eq:Def_dens_escalar_vacuo} deve ser multiplicado por $(\hbar c)^{-3}$
\end{itemize}

\begin{figure*}
	\begin{tikzpicture}[gnuplot]
%% generated with GNUPLOT 5.0p2 (Lua 5.2; terminal rev. 99, script rev. 100)
%% Fri Mar  4 16:19:10 2016
\path (0.000,0.000) rectangle (14.000,9.000);
\gpcolor{color=gp lt color border}
\gpsetlinetype{gp lt border}
\gpsetdashtype{gp dt solid}
\gpsetlinewidth{1.00}
\draw[gp path] (1.320,0.985)--(1.500,0.985);
\draw[gp path] (13.447,0.985)--(13.267,0.985);
\node[gp node right] at (1.136,0.985) {$910$};
\draw[gp path] (1.320,1.941)--(1.500,1.941);
\draw[gp path] (13.447,1.941)--(13.267,1.941);
\node[gp node right] at (1.136,1.941) {$920$};
\draw[gp path] (1.320,2.897)--(1.500,2.897);
\draw[gp path] (13.447,2.897)--(13.267,2.897);
\node[gp node right] at (1.136,2.897) {$930$};
\draw[gp path] (1.320,3.852)--(1.500,3.852);
\draw[gp path] (13.447,3.852)--(13.267,3.852);
\node[gp node right] at (1.136,3.852) {$940$};
\draw[gp path] (1.320,4.808)--(1.500,4.808);
\draw[gp path] (13.447,4.808)--(13.267,4.808);
\node[gp node right] at (1.136,4.808) {$950$};
\draw[gp path] (1.320,5.764)--(1.500,5.764);
\draw[gp path] (13.447,5.764)--(13.267,5.764);
\node[gp node right] at (1.136,5.764) {$960$};
\draw[gp path] (1.320,6.720)--(1.500,6.720);
\draw[gp path] (13.447,6.720)--(13.267,6.720);
\node[gp node right] at (1.136,6.720) {$970$};
\draw[gp path] (1.320,7.675)--(1.500,7.675);
\draw[gp path] (13.447,7.675)--(13.267,7.675);
\node[gp node right] at (1.136,7.675) {$980$};
\draw[gp path] (1.320,8.631)--(1.500,8.631);
\draw[gp path] (13.447,8.631)--(13.267,8.631);
\node[gp node right] at (1.136,8.631) {$990$};
\draw[gp path] (1.320,0.985)--(1.320,1.165);
\draw[gp path] (1.320,8.631)--(1.320,8.451);
\node[gp node center] at (1.320,0.677) {$0$};
\draw[gp path] (2.836,0.985)--(2.836,1.165);
\draw[gp path] (2.836,8.631)--(2.836,8.451);
\node[gp node center] at (2.836,0.677) {$0.05$};
\draw[gp path] (4.352,0.985)--(4.352,1.165);
\draw[gp path] (4.352,8.631)--(4.352,8.451);
\node[gp node center] at (4.352,0.677) {$0.1$};
\draw[gp path] (5.868,0.985)--(5.868,1.165);
\draw[gp path] (5.868,8.631)--(5.868,8.451);
\node[gp node center] at (5.868,0.677) {$0.15$};
\draw[gp path] (7.384,0.985)--(7.384,1.165);
\draw[gp path] (7.384,8.631)--(7.384,8.451);
\node[gp node center] at (7.384,0.677) {$0.2$};
\draw[gp path] (8.899,0.985)--(8.899,1.165);
\draw[gp path] (8.899,8.631)--(8.899,8.451);
\node[gp node center] at (8.899,0.677) {$0.25$};
\draw[gp path] (10.415,0.985)--(10.415,1.165);
\draw[gp path] (10.415,8.631)--(10.415,8.451);
\node[gp node center] at (10.415,0.677) {$0.3$};
\draw[gp path] (11.931,0.985)--(11.931,1.165);
\draw[gp path] (11.931,8.631)--(11.931,8.451);
\node[gp node center] at (11.931,0.677) {$0.35$};
\draw[gp path] (13.447,0.985)--(13.447,1.165);
\draw[gp path] (13.447,8.631)--(13.447,8.451);
\node[gp node center] at (13.447,0.677) {$0.4$};
\draw[gp path] (1.320,8.631)--(1.320,0.985)--(13.447,0.985)--(13.447,8.631)--cycle;
\node[gp node center,rotate=-270] at (0.246,4.808) {$\mu$ (MeV)};
\node[gp node center] at (7.383,0.215) {$\rho$ ($\rm{fm}^{-3}$)};
\node[gp node left] at (2.788,8.297) {$\mu_p$};
\gpcolor{rgb color={0.580,0.000,0.827}}
\draw[gp path] (1.688,8.297)--(2.604,8.297);
\draw[gp path] (1.644,3.494)--(1.654,3.478)--(1.664,3.462)--(1.675,3.446)--(1.685,3.431)%
  --(1.695,3.415)--(1.706,3.399)--(1.716,3.383)--(1.726,3.367)--(1.737,3.351)--(1.747,3.335)%
  --(1.757,3.319)--(1.768,3.303)--(1.778,3.321)--(1.788,3.305)--(1.799,3.289)--(1.809,3.273)%
  --(1.819,3.257)--(1.830,3.241)--(1.840,3.225)--(1.850,3.209)--(1.860,3.193)--(1.871,3.177)%
  --(1.881,3.161)--(1.891,3.144)--(1.902,3.163)--(1.912,3.147)--(1.922,3.131)--(1.933,3.115)%
  --(1.943,3.099)--(1.953,3.083)--(1.964,3.067)--(1.974,3.051)--(1.984,3.035)--(1.995,3.019)%
  --(2.005,3.003)--(2.015,3.022)--(2.026,3.006)--(2.036,2.991)--(2.046,2.975)--(2.057,2.959)%
  --(2.067,2.943)--(2.077,2.928)--(2.087,2.912)--(2.098,2.896)--(2.108,2.881)--(2.118,2.865)%
  --(2.129,2.884)--(2.139,2.869)--(2.149,2.853)--(2.160,2.838)--(2.170,2.822)--(2.180,2.807)%
  --(2.191,2.792)--(2.201,2.776)--(2.211,2.761)--(2.222,2.746)--(2.232,2.765)--(2.242,2.750)%
  --(2.253,2.735)--(2.263,2.720)--(2.273,2.705)--(2.284,2.689)--(2.294,2.675)--(2.304,2.660)%
  --(2.314,2.645)--(2.325,2.664)--(2.335,2.649)--(2.345,2.635)--(2.356,2.620)--(2.366,2.605)%
  --(2.376,2.591)--(2.387,2.576)--(2.397,2.561)--(2.407,2.547)--(2.418,2.567)--(2.428,2.552)%
  --(2.438,2.538)--(2.449,2.523)--(2.459,2.509)--(2.469,2.495)--(2.480,2.481)--(2.490,2.466)%
  --(2.500,2.486)--(2.511,2.472)--(2.521,2.458)--(2.531,2.444)--(2.542,2.430)--(2.552,2.416)%
  --(2.562,2.403)--(2.572,2.389)--(2.583,2.375)--(2.593,2.395)--(2.603,2.382)--(2.614,2.368)%
  --(2.624,2.354)--(2.634,2.341)--(2.645,2.327)--(2.655,2.314)--(2.665,2.301)--(2.676,2.321)%
  --(2.686,2.308)--(2.696,2.295)--(2.707,2.281)--(2.717,2.268)--(2.727,2.255)--(2.738,2.242)%
  --(2.748,2.229)--(2.758,2.250)--(2.769,2.237)--(2.779,2.224)--(2.789,2.211)--(2.799,2.199)%
  --(2.810,2.186)--(2.820,2.173)--(2.830,2.195)--(2.841,2.182)--(2.851,2.169)--(2.861,2.157)%
  --(2.872,2.145)--(2.882,2.132)--(2.892,2.120)--(2.903,2.107)--(2.913,2.129)--(2.923,2.117)%
  --(2.934,2.105)--(2.944,2.093)--(2.954,2.081)--(2.965,2.069)--(2.975,2.057)--(2.985,2.079)%
  --(2.996,2.067)--(3.006,2.055)--(3.016,2.043)--(3.026,2.032)--(3.037,2.020)--(3.047,2.008)%
  --(3.057,2.031)--(3.068,2.019)--(3.078,2.008)--(3.088,1.996)--(3.099,1.985)--(3.109,1.974)%
  --(3.119,1.962)--(3.130,1.985)--(3.140,1.974)--(3.150,1.963)--(3.161,1.952)--(3.171,1.941)%
  --(3.181,1.930)--(3.192,1.919)--(3.202,1.942)--(3.212,1.931)--(3.223,1.920)--(3.233,1.910)%
  --(3.243,1.899)--(3.253,1.888)--(3.264,1.878)--(3.274,1.901)--(3.284,1.891)--(3.295,1.880)%
  --(3.305,1.870)--(3.315,1.860)--(3.326,1.849)--(3.336,1.839)--(3.346,1.863)--(3.357,1.853)%
  --(3.367,1.843)--(3.377,1.833)--(3.388,1.823)--(3.398,1.813)--(3.408,1.803)--(3.419,1.827)%
  --(3.429,1.817)--(3.439,1.807)--(3.450,1.798)--(3.460,1.788)--(3.470,1.779)--(3.480,1.803)%
  --(3.491,1.793)--(3.501,1.784)--(3.511,1.775)--(3.522,1.766)--(3.532,1.756)--(3.542,1.747)%
  --(3.553,1.772)--(3.563,1.763)--(3.573,1.754)--(3.584,1.745)--(3.594,1.736)--(3.604,1.727)%
  --(3.615,1.752)--(3.625,1.743)--(3.635,1.734)--(3.646,1.726)--(3.656,1.717)--(3.666,1.709)%
  --(3.677,1.700)--(3.687,1.725)--(3.697,1.717)--(3.707,1.708)--(3.718,1.700)--(3.728,1.692)%
  --(3.738,1.684)--(3.749,1.709)--(3.759,1.701)--(3.769,1.693)--(3.780,1.685)--(3.790,1.677)%
  --(3.800,1.669)--(3.811,1.695)--(3.821,1.687)--(3.831,1.679)--(3.842,1.672)--(3.852,1.664)%
  --(3.862,1.657)--(3.873,1.649)--(3.883,1.675)--(3.893,1.668)--(3.904,1.660)--(3.914,1.653)%
  --(3.924,1.646)--(3.934,1.639)--(3.945,1.665)--(3.955,1.658)--(3.965,1.651)--(3.976,1.644)%
  --(3.986,1.637)--(3.996,1.630)--(4.007,1.656)--(4.017,1.649)--(4.027,1.643)--(4.038,1.636)%
  --(4.048,1.630)--(4.058,1.623)--(4.069,1.650)--(4.079,1.643)--(4.089,1.637)--(4.100,1.630)%
  --(4.110,1.624)--(4.120,1.618)--(4.131,1.645)--(4.141,1.639)--(4.151,1.633)--(4.161,1.627)%
  --(4.172,1.621)--(4.182,1.615)--(4.192,1.642)--(4.203,1.636)--(4.213,1.630)--(4.223,1.624)%
  --(4.234,1.619)--(4.244,1.613)--(4.254,1.608)--(4.265,1.635)--(4.275,1.630)--(4.285,1.624)%
  --(4.296,1.619)--(4.306,1.614)--(4.316,1.608)--(4.327,1.636)--(4.337,1.631)--(4.347,1.626)%
  --(4.358,1.621)--(4.368,1.616)--(4.378,1.611)--(4.388,1.639)--(4.399,1.634)--(4.409,1.629)%
  --(4.419,1.624)--(4.430,1.620)--(4.440,1.615)--(4.450,1.643)--(4.461,1.639)--(4.471,1.634)%
  --(4.481,1.630)--(4.492,1.626)--(4.502,1.621)--(4.512,1.650)--(4.523,1.646)--(4.533,1.641)%
  --(4.543,1.637)--(4.554,1.633)--(4.564,1.629)--(4.574,1.658)--(4.585,1.654)--(4.595,1.650)%
  --(4.605,1.646)--(4.615,1.643)--(4.626,1.639)--(4.636,1.668)--(4.646,1.664)--(4.657,1.661)%
  --(4.667,1.657)--(4.677,1.654)--(4.688,1.650)--(4.698,1.680)--(4.708,1.676)--(4.719,1.673)%
  --(4.729,1.670)--(4.739,1.667)--(4.750,1.664)--(4.760,1.693)--(4.770,1.690)--(4.781,1.687)%
  --(4.791,1.685)--(4.801,1.682)--(4.812,1.679)--(4.822,1.709)--(4.832,1.706)--(4.842,1.703)%
  --(4.853,1.701)--(4.863,1.698)--(4.873,1.696)--(4.884,1.726)--(4.894,1.724)--(4.904,1.721)%
  --(4.915,1.719)--(4.925,1.717)--(4.935,1.715)--(4.946,1.712)--(4.956,1.743)--(4.966,1.741)%
  --(4.977,1.739)--(4.987,1.737)--(4.997,1.735)--(5.008,1.733)--(5.018,1.764)--(5.028,1.762)%
  --(5.039,1.760)--(5.049,1.759)--(5.059,1.757)--(5.069,1.756)--(5.080,1.787)--(5.090,1.785)%
  --(5.100,1.784)--(5.111,1.783)--(5.121,1.781)--(5.131,1.780)--(5.142,1.811)--(5.152,1.810)%
  --(5.162,1.809)--(5.173,1.808)--(5.183,1.807)--(5.193,1.806)--(5.204,1.805)--(5.214,1.837)%
  --(5.224,1.836)--(5.235,1.835)--(5.245,1.835)--(5.255,1.834)--(5.266,1.833)--(5.276,1.865)%
  --(5.286,1.865)--(5.296,1.864)--(5.307,1.864)--(5.317,1.864)--(5.327,1.863)--(5.338,1.863)%
  --(5.348,1.895)--(5.358,1.895)--(5.369,1.895)--(5.379,1.895)--(5.389,1.895)--(5.400,1.895)%
  --(5.410,1.927)--(5.420,1.928)--(5.431,1.928)--(5.441,1.928)--(5.451,1.928)--(5.462,1.929)%
  --(5.472,1.929)--(5.482,1.962)--(5.493,1.962)--(5.503,1.963)--(5.513,1.964)--(5.523,1.964)%
  --(5.534,1.965)--(5.544,1.998)--(5.554,1.999)--(5.565,2.000)--(5.575,2.001)--(5.585,2.002)%
  --(5.596,2.003)--(5.606,2.004)--(5.616,2.037)--(5.627,2.038)--(5.637,2.039)--(5.647,2.040)%
  --(5.658,2.042)--(5.668,2.043)--(5.678,2.045)--(5.689,2.078)--(5.699,2.080)--(5.709,2.081)%
  --(5.720,2.083)--(5.730,2.085)--(5.740,2.086)--(5.750,2.088)--(5.761,2.122)--(5.771,2.124)%
  --(5.781,2.126)--(5.792,2.128)--(5.802,2.130)--(5.812,2.132)--(5.823,2.134)--(5.833,2.168)%
  --(5.843,2.170)--(5.854,2.172)--(5.864,2.175)--(5.874,2.177)--(5.885,2.180)--(5.895,2.182)%
  --(5.905,2.216)--(5.916,2.219)--(5.926,2.221)--(5.936,2.224)--(5.947,2.227)--(5.957,2.238)%
  --(5.967,2.240)--(5.977,2.243)--(5.988,2.262)--(5.998,2.265)--(6.008,2.268)--(6.019,2.271)%
  --(6.029,2.290)--(6.039,2.293)--(6.050,2.297)--(6.060,2.300)--(6.070,2.319)--(6.081,2.322)%
  --(6.091,2.326)--(6.101,2.345)--(6.112,2.348)--(6.122,2.352)--(6.132,2.356)--(6.143,2.375)%
  --(6.153,2.379)--(6.163,2.383)--(6.174,2.387)--(6.184,2.390)--(6.194,2.410)--(6.204,2.414)%
  --(6.215,2.418)--(6.225,2.422)--(6.235,2.442)--(6.246,2.446)--(6.256,2.450)--(6.266,2.455)%
  --(6.277,2.475)--(6.287,2.479)--(6.297,2.483)--(6.308,2.488)--(6.318,2.508)--(6.328,2.513)%
  --(6.339,2.517)--(6.349,2.522)--(6.359,2.527)--(6.370,2.547)--(6.380,2.552)--(6.390,2.557)%
  --(6.401,2.562)--(6.411,2.582)--(6.421,2.587)--(6.431,2.592)--(6.442,2.598)--(6.452,2.603)%
  --(6.462,2.623)--(6.473,2.629)--(6.483,2.634)--(6.493,2.640)--(6.504,2.660)--(6.514,2.666)%
  --(6.524,2.671)--(6.535,2.677)--(6.545,2.683)--(6.555,2.704)--(6.566,2.710)--(6.576,2.715)%
  --(6.586,2.721)--(6.597,2.727)--(6.607,2.748)--(6.617,2.754)--(6.628,2.760)--(6.638,2.767)%
  --(6.648,2.773)--(6.658,2.779)--(6.669,2.800)--(6.679,2.807)--(6.689,2.813)--(6.700,2.819)%
  --(6.710,2.826)--(6.720,2.848)--(6.731,2.854)--(6.741,2.861)--(6.751,2.867)--(6.762,2.874)%
  --(6.772,2.881)--(6.782,2.903)--(6.793,2.910)--(6.803,2.917)--(6.813,2.924)--(6.824,2.931)%
  --(6.834,2.938)--(6.844,2.960)--(6.855,2.967)--(6.865,2.974)--(6.875,2.981)--(6.885,2.989)%
  --(6.896,2.996)--(6.906,3.019)--(6.916,3.026)--(6.927,3.034)--(6.937,3.041)--(6.947,3.049)%
  --(6.958,3.056)--(6.968,3.064)--(6.978,3.087)--(6.989,3.095)--(6.999,3.103)--(7.009,3.110)%
  --(7.020,3.118)--(7.030,3.126)--(7.040,3.134)--(7.051,3.143)--(7.061,3.166)--(7.071,3.174)%
  --(7.082,3.182)--(7.092,3.190)--(7.102,3.199)--(7.112,3.207)--(7.123,3.216)--(7.133,3.224)%
  --(7.143,3.233)--(7.154,3.256)--(7.164,3.265)--(7.174,3.274)--(7.185,3.282)--(7.195,3.291)%
  --(7.205,3.300)--(7.216,3.309)--(7.226,3.318)--(7.236,3.327)--(7.247,3.336)--(7.257,3.360)%
  --(7.267,3.369)--(7.278,3.378)--(7.288,3.387)--(7.298,3.397)--(7.309,3.406)--(7.319,3.415)%
  --(7.329,3.425)--(7.339,3.434)--(7.350,3.444)--(7.360,3.454)--(7.370,3.463)--(7.381,3.473)%
  --(7.391,3.483)--(7.401,3.507)--(7.412,3.517)--(7.422,3.527)--(7.432,3.537)--(7.443,3.547)%
  --(7.453,3.557)--(7.463,3.567)--(7.474,3.577)--(7.484,3.588)--(7.494,3.598)--(7.505,3.608)%
  --(7.515,3.619)--(7.525,3.629)--(7.536,3.639)--(7.546,3.650)--(7.556,3.660)--(7.566,3.671)%
  --(7.577,3.682)--(7.587,3.693)--(7.597,3.703)--(7.608,3.714)--(7.618,3.725)--(7.628,3.736)%
  --(7.639,3.747)--(7.649,3.758)--(7.659,3.769)--(7.670,3.780)--(7.680,3.791)--(7.690,3.803)%
  --(7.701,3.814)--(7.711,3.825)--(7.721,3.837)--(7.732,3.848)--(7.742,3.860)--(7.752,3.871)%
  --(7.763,3.883)--(7.773,3.894)--(7.783,3.906)--(7.793,3.918)--(7.804,3.930)--(7.814,3.942)%
  --(7.824,3.953)--(7.835,3.965)--(7.845,3.977)--(7.855,3.989)--(7.866,4.002)--(7.876,4.014)%
  --(7.886,4.012)--(7.897,4.024)--(7.907,4.036)--(7.917,4.049)--(7.928,4.061)--(7.938,4.074)%
  --(7.948,4.086)--(7.959,4.099)--(7.969,4.111)--(7.979,4.124)--(7.990,4.137)--(8.000,4.150)%
  --(8.010,4.162)--(8.021,4.161)--(8.031,4.174)--(8.041,4.187)--(8.051,4.200)--(8.062,4.213)%
  --(8.072,4.226)--(8.082,4.240)--(8.093,4.253)--(8.103,4.252)--(8.113,4.266)--(8.124,4.279)%
  --(8.134,4.292)--(8.144,4.306)--(8.155,4.320)--(8.165,4.333)--(8.175,4.347)--(8.186,4.347)%
  --(8.196,4.361)--(8.206,4.374)--(8.217,4.388)--(8.227,4.402)--(8.237,4.416)--(8.248,4.416)%
  --(8.258,4.431)--(8.268,4.445)--(8.278,4.459)--(8.289,4.473)--(8.299,4.474)--(8.309,4.488)%
  --(8.320,4.503)--(8.330,4.517)--(8.340,4.532)--(8.351,4.532)--(8.361,4.547)--(8.371,4.562)%
  --(8.382,4.577)--(8.392,4.592)--(8.402,4.593)--(8.413,4.608)--(8.423,4.623)--(8.433,4.638)%
  --(8.444,4.639)--(8.454,4.654)--(8.464,4.669)--(8.475,4.685)--(8.485,4.686)--(8.495,4.702)%
  --(8.505,4.717)--(8.516,4.733)--(8.526,4.735)--(8.536,4.750)--(8.547,4.766)--(8.557,4.782)%
  --(8.567,4.784)--(8.578,4.800)--(8.588,4.816)--(8.598,4.818)--(8.609,4.834)--(8.619,4.850)%
  --(8.629,4.853)--(8.640,4.869)--(8.650,4.885)--(8.660,4.888)--(8.671,4.904)--(8.681,4.921)%
  --(8.691,4.924)--(8.702,4.940)--(8.712,4.957)--(8.722,4.960)--(8.732,4.977)--(8.743,4.994)%
  --(8.753,4.997)--(8.763,5.014)--(8.774,5.031)--(8.784,5.034)--(8.794,5.052)--(8.805,5.055)%
  --(8.815,5.073)--(8.825,5.090)--(8.836,5.094)--(8.846,5.111)--(8.856,5.115)--(8.867,5.133)%
  --(8.877,5.150)--(8.887,5.155)--(8.898,5.172)--(8.908,5.177)--(8.918,5.195)--(8.929,5.199)%
  --(8.939,5.217)--(8.949,5.235)--(8.959,5.240)--(8.970,5.258)--(8.980,5.263)--(8.990,5.281)%
  --(9.001,5.286)--(9.011,5.305)--(9.021,5.310)--(9.032,5.328)--(9.042,5.334)--(9.052,5.352)%
  --(9.063,5.358)--(9.073,5.377)--(9.083,5.383)--(9.094,5.401)--(9.104,5.407)--(9.114,5.413)%
  --(9.125,5.432)--(9.135,5.438)--(9.145,5.458)--(9.156,5.464)--(9.166,5.483)--(9.176,5.490)%
  --(9.186,5.509)--(9.197,5.516)--(9.207,5.522)--(9.217,5.542)--(9.228,5.549)--(9.238,5.568)%
  --(9.248,5.575)--(9.259,5.582)--(9.269,5.602)--(9.279,5.610)--(9.290,5.617)--(9.300,5.637)%
  --(9.310,5.645)--(9.321,5.652)--(9.331,5.672)--(9.341,5.680)--(9.352,5.688)--(9.362,5.709)%
  --(9.372,5.716)--(9.383,5.725)--(9.393,5.745)--(9.403,5.753)--(9.413,5.762)--(9.424,5.770)%
  --(9.434,5.791)--(9.444,5.800)--(9.455,5.808)--(9.465,5.817)--(9.475,5.838)--(9.486,5.847)%
  --(9.496,5.856)--(9.506,5.865)--(9.517,5.887)--(9.527,5.896)--(9.537,5.905)--(9.548,5.915)%
  --(9.558,5.924)--(9.568,5.946)--(9.579,5.956)--(9.589,5.966)--(9.599,5.976)--(9.610,5.986)%
  --(9.620,5.996)--(9.630,6.006)--(9.640,6.028)--(9.651,6.038)--(9.661,6.049)--(9.671,6.059)%
  --(9.682,6.070)--(9.692,6.081)--(9.702,6.091)--(9.713,6.102)--(9.723,6.113)--(9.733,6.124)%
  --(9.744,6.135)--(9.754,6.147)--(9.764,6.158)--(9.775,6.169)--(9.785,6.181)--(9.795,6.192)%
  --(9.806,6.204)--(9.816,6.216)--(9.826,6.228)--(9.837,6.240)--(9.847,6.252)--(9.857,6.264)%
  --(9.867,6.276)--(9.878,6.289)--(9.888,6.301)--(9.898,6.314)--(9.909,6.327)--(9.919,6.328)%
  --(9.929,6.340)--(9.940,6.353)--(9.950,6.367)--(9.960,6.380)--(9.971,6.393)--(9.981,6.406)%
  --(9.991,6.408)--(10.002,6.422)--(10.012,6.436)--(10.022,6.450)--(10.033,6.452)--(10.043,6.466)%
  --(10.053,6.480)--(10.064,6.494)--(10.074,6.497)--(10.084,6.512)--(10.094,6.526)--(10.105,6.541)%
  --(10.115,6.544)--(10.125,6.559)--(10.136,6.574)--(10.146,6.578)--(10.156,6.593)--(10.167,6.597)%
  --(10.177,6.613)--(10.187,6.628)--(10.198,6.633)--(10.208,6.649)--(10.218,6.654)--(10.229,6.670)%
  --(10.239,6.675)--(10.249,6.691)--(10.260,6.696)--(10.270,6.713)--(10.280,6.718)--(10.291,6.735)%
  --(10.301,6.741)--(10.311,6.758)--(10.321,6.764)--(10.332,6.781)--(10.342,6.788)--(10.352,6.795)%
  --(10.363,6.812)--(10.373,6.819)--(10.383,6.837)--(10.394,6.844)--(10.404,6.852)--(10.414,6.859)%
  --(10.425,6.877)--(10.435,6.885)--(10.445,6.893)--(10.456,6.901)--(10.466,6.920)--(10.476,6.929)%
  --(10.487,6.937)--(10.497,6.946)--(10.507,6.955)--(10.518,6.975)--(10.528,6.984)--(10.538,6.993)%
  --(10.548,7.003)--(10.559,7.013)--(10.569,7.023)--(10.579,7.033)--(10.590,7.043)--(10.600,7.054)%
  --(10.610,7.064)--(10.621,7.075)--(10.631,7.086)--(10.641,7.097)--(10.652,7.106)--(10.662,7.117)%
  --(10.672,7.124)--(10.683,7.136)--(10.693,7.148)--(10.703,7.155)--(10.714,7.168)--(10.724,7.175)%
  --(10.734,7.188)--(10.745,7.196)--(10.755,7.204)--(10.765,7.218)--(10.775,7.226)--(10.786,7.235)%
  --(10.796,7.244)--(10.806,7.258)--(10.817,7.268)--(10.827,7.278)--(10.837,7.288)--(10.848,7.298)%
  --(10.858,7.304)--(10.868,7.314)--(10.879,7.325)--(10.889,7.336)--(10.899,7.343)--(10.910,7.354)%
  --(10.920,7.366)--(10.930,7.373)--(10.941,7.386)--(10.951,7.393)--(10.961,7.402)--(10.972,7.415)%
  --(10.982,7.423)--(10.992,7.432)--(11.002,7.441)--(11.013,7.451)--(11.023,7.461)--(11.033,7.471)%
  --(11.044,7.481)--(11.054,7.487)--(11.064,7.498)--(11.075,7.509)--(11.085,7.516)--(11.095,7.524)%
  --(11.106,7.536)--(11.116,7.544)--(11.126,7.552)--(11.137,7.561)--(11.147,7.574)--(11.157,7.579)%
  --(11.168,7.589)--(11.178,7.599)--(11.188,7.609)--(11.199,7.615)--(11.209,7.626)--(11.219,7.634)%
  --(11.229,7.645)--(11.240,7.654)--(11.250,7.662)--(11.260,7.671)--(11.271,7.680)--(11.281,7.690)%
  --(11.291,7.696)--(11.302,7.706)--(11.312,7.713)--(11.322,7.725)--(11.333,7.732)--(11.343,7.741)%
  --(11.353,7.750)--(11.364,7.759)--(11.374,7.769)--(11.384,7.775)--(11.395,7.782)--(11.405,7.793)%
  --(11.415,7.801)--(11.426,7.810)--(11.436,7.819)--(11.446,7.826)--(11.456,7.836)--(11.467,7.843)%
  --(11.477,7.851)--(11.487,7.860)--(11.498,7.866)--(11.508,7.876)--(11.518,7.883)--(11.529,7.891)%
  --(11.539,7.900)--(11.549,7.909)--(11.560,7.916)--(11.570,7.923)--(11.580,7.932)--(11.591,7.941)%
  --(11.601,7.948)--(11.611,7.956)--(11.622,7.965)--(11.632,7.972)--(11.642,7.980)--(11.653,7.988)%
  --(11.663,7.995)--(11.673,8.004)--(11.683,8.010)--(11.694,8.018)--(11.704,8.027)--(11.714,8.034)%
  --(11.725,8.041)--(11.735,8.048)--(11.745,8.055)--(11.756,8.064)--(11.766,8.071)--(11.776,8.078)%
  --(11.787,8.085)--(11.797,8.092)--(11.807,8.101)--(11.818,8.107)--(11.828,8.114)--(11.838,8.122)%
  --(11.849,8.129)--(11.859,8.136)--(11.869,8.143)--(11.880,8.150)--(11.890,8.157)--(11.900,8.164)%
  --(11.910,8.171)--(11.921,8.178)--(11.931,8.184)--(11.941,8.192);
\gpcolor{color=gp lt color border}
\node[gp node left] at (2.788,7.989) {$\mu_n$};
\gpcolor{rgb color={0.000,0.620,0.451}}
\draw[gp path] (1.688,7.989)--(2.604,7.989);
\draw[gp path] (1.644,3.494)--(1.654,3.478)--(1.664,3.462)--(1.675,3.446)--(1.685,3.431)%
  --(1.695,3.415)--(1.706,3.399)--(1.716,3.383)--(1.726,3.367)--(1.737,3.351)--(1.747,3.335)%
  --(1.757,3.319)--(1.768,3.303)--(1.778,3.321)--(1.788,3.305)--(1.799,3.289)--(1.809,3.273)%
  --(1.819,3.257)--(1.830,3.241)--(1.840,3.225)--(1.850,3.209)--(1.860,3.193)--(1.871,3.177)%
  --(1.881,3.161)--(1.891,3.144)--(1.902,3.163)--(1.912,3.147)--(1.922,3.131)--(1.933,3.115)%
  --(1.943,3.099)--(1.953,3.083)--(1.964,3.067)--(1.974,3.051)--(1.984,3.035)--(1.995,3.019)%
  --(2.005,3.003)--(2.015,3.022)--(2.026,3.006)--(2.036,2.991)--(2.046,2.975)--(2.057,2.959)%
  --(2.067,2.943)--(2.077,2.928)--(2.087,2.912)--(2.098,2.896)--(2.108,2.881)--(2.118,2.865)%
  --(2.129,2.884)--(2.139,2.869)--(2.149,2.853)--(2.160,2.838)--(2.170,2.822)--(2.180,2.807)%
  --(2.191,2.792)--(2.201,2.776)--(2.211,2.761)--(2.222,2.746)--(2.232,2.765)--(2.242,2.750)%
  --(2.253,2.735)--(2.263,2.720)--(2.273,2.705)--(2.284,2.689)--(2.294,2.675)--(2.304,2.660)%
  --(2.314,2.645)--(2.325,2.664)--(2.335,2.649)--(2.345,2.635)--(2.356,2.620)--(2.366,2.605)%
  --(2.376,2.591)--(2.387,2.576)--(2.397,2.561)--(2.407,2.547)--(2.418,2.567)--(2.428,2.552)%
  --(2.438,2.538)--(2.449,2.523)--(2.459,2.509)--(2.469,2.495)--(2.480,2.481)--(2.490,2.466)%
  --(2.500,2.486)--(2.511,2.472)--(2.521,2.458)--(2.531,2.444)--(2.542,2.430)--(2.552,2.416)%
  --(2.562,2.403)--(2.572,2.389)--(2.583,2.375)--(2.593,2.395)--(2.603,2.382)--(2.614,2.368)%
  --(2.624,2.354)--(2.634,2.341)--(2.645,2.327)--(2.655,2.314)--(2.665,2.301)--(2.676,2.321)%
  --(2.686,2.308)--(2.696,2.295)--(2.707,2.281)--(2.717,2.268)--(2.727,2.255)--(2.738,2.242)%
  --(2.748,2.229)--(2.758,2.250)--(2.769,2.237)--(2.779,2.224)--(2.789,2.211)--(2.799,2.199)%
  --(2.810,2.186)--(2.820,2.173)--(2.830,2.195)--(2.841,2.182)--(2.851,2.169)--(2.861,2.157)%
  --(2.872,2.145)--(2.882,2.132)--(2.892,2.120)--(2.903,2.107)--(2.913,2.129)--(2.923,2.117)%
  --(2.934,2.105)--(2.944,2.093)--(2.954,2.081)--(2.965,2.069)--(2.975,2.057)--(2.985,2.079)%
  --(2.996,2.067)--(3.006,2.055)--(3.016,2.043)--(3.026,2.032)--(3.037,2.020)--(3.047,2.008)%
  --(3.057,2.031)--(3.068,2.019)--(3.078,2.008)--(3.088,1.996)--(3.099,1.985)--(3.109,1.974)%
  --(3.119,1.962)--(3.130,1.985)--(3.140,1.974)--(3.150,1.963)--(3.161,1.952)--(3.171,1.941)%
  --(3.181,1.930)--(3.192,1.919)--(3.202,1.942)--(3.212,1.931)--(3.223,1.920)--(3.233,1.910)%
  --(3.243,1.899)--(3.253,1.888)--(3.264,1.878)--(3.274,1.901)--(3.284,1.891)--(3.295,1.880)%
  --(3.305,1.870)--(3.315,1.860)--(3.326,1.849)--(3.336,1.839)--(3.346,1.863)--(3.357,1.853)%
  --(3.367,1.843)--(3.377,1.833)--(3.388,1.823)--(3.398,1.813)--(3.408,1.803)--(3.419,1.827)%
  --(3.429,1.817)--(3.439,1.807)--(3.450,1.798)--(3.460,1.788)--(3.470,1.779)--(3.480,1.803)%
  --(3.491,1.793)--(3.501,1.784)--(3.511,1.775)--(3.522,1.766)--(3.532,1.756)--(3.542,1.747)%
  --(3.553,1.772)--(3.563,1.763)--(3.573,1.754)--(3.584,1.745)--(3.594,1.736)--(3.604,1.727)%
  --(3.615,1.752)--(3.625,1.743)--(3.635,1.734)--(3.646,1.726)--(3.656,1.717)--(3.666,1.709)%
  --(3.677,1.700)--(3.687,1.725)--(3.697,1.717)--(3.707,1.708)--(3.718,1.700)--(3.728,1.692)%
  --(3.738,1.684)--(3.749,1.709)--(3.759,1.701)--(3.769,1.693)--(3.780,1.685)--(3.790,1.677)%
  --(3.800,1.669)--(3.811,1.695)--(3.821,1.687)--(3.831,1.679)--(3.842,1.672)--(3.852,1.664)%
  --(3.862,1.657)--(3.873,1.649)--(3.883,1.675)--(3.893,1.668)--(3.904,1.660)--(3.914,1.653)%
  --(3.924,1.646)--(3.934,1.639)--(3.945,1.665)--(3.955,1.658)--(3.965,1.651)--(3.976,1.644)%
  --(3.986,1.637)--(3.996,1.630)--(4.007,1.656)--(4.017,1.649)--(4.027,1.643)--(4.038,1.636)%
  --(4.048,1.630)--(4.058,1.623)--(4.069,1.650)--(4.079,1.643)--(4.089,1.637)--(4.100,1.630)%
  --(4.110,1.624)--(4.120,1.618)--(4.131,1.645)--(4.141,1.639)--(4.151,1.633)--(4.161,1.627)%
  --(4.172,1.621)--(4.182,1.615)--(4.192,1.642)--(4.203,1.636)--(4.213,1.630)--(4.223,1.624)%
  --(4.234,1.619)--(4.244,1.613)--(4.254,1.608)--(4.265,1.635)--(4.275,1.630)--(4.285,1.624)%
  --(4.296,1.619)--(4.306,1.614)--(4.316,1.608)--(4.327,1.636)--(4.337,1.631)--(4.347,1.626)%
  --(4.358,1.621)--(4.368,1.616)--(4.378,1.611)--(4.388,1.639)--(4.399,1.634)--(4.409,1.629)%
  --(4.419,1.624)--(4.430,1.620)--(4.440,1.615)--(4.450,1.643)--(4.461,1.639)--(4.471,1.634)%
  --(4.481,1.630)--(4.492,1.626)--(4.502,1.621)--(4.512,1.650)--(4.523,1.646)--(4.533,1.641)%
  --(4.543,1.637)--(4.554,1.633)--(4.564,1.629)--(4.574,1.658)--(4.585,1.654)--(4.595,1.650)%
  --(4.605,1.646)--(4.615,1.643)--(4.626,1.639)--(4.636,1.668)--(4.646,1.664)--(4.657,1.661)%
  --(4.667,1.657)--(4.677,1.654)--(4.688,1.650)--(4.698,1.680)--(4.708,1.676)--(4.719,1.673)%
  --(4.729,1.670)--(4.739,1.667)--(4.750,1.664)--(4.760,1.693)--(4.770,1.690)--(4.781,1.687)%
  --(4.791,1.685)--(4.801,1.682)--(4.812,1.679)--(4.822,1.709)--(4.832,1.706)--(4.842,1.703)%
  --(4.853,1.701)--(4.863,1.698)--(4.873,1.696)--(4.884,1.726)--(4.894,1.724)--(4.904,1.721)%
  --(4.915,1.719)--(4.925,1.717)--(4.935,1.715)--(4.946,1.712)--(4.956,1.743)--(4.966,1.741)%
  --(4.977,1.739)--(4.987,1.737)--(4.997,1.735)--(5.008,1.733)--(5.018,1.764)--(5.028,1.762)%
  --(5.039,1.760)--(5.049,1.759)--(5.059,1.757)--(5.069,1.756)--(5.080,1.787)--(5.090,1.785)%
  --(5.100,1.784)--(5.111,1.783)--(5.121,1.781)--(5.131,1.780)--(5.142,1.811)--(5.152,1.810)%
  --(5.162,1.809)--(5.173,1.808)--(5.183,1.807)--(5.193,1.806)--(5.204,1.805)--(5.214,1.837)%
  --(5.224,1.836)--(5.235,1.835)--(5.245,1.835)--(5.255,1.834)--(5.266,1.833)--(5.276,1.865)%
  --(5.286,1.865)--(5.296,1.864)--(5.307,1.864)--(5.317,1.864)--(5.327,1.863)--(5.338,1.863)%
  --(5.348,1.895)--(5.358,1.895)--(5.369,1.895)--(5.379,1.895)--(5.389,1.895)--(5.400,1.895)%
  --(5.410,1.927)--(5.420,1.928)--(5.431,1.928)--(5.441,1.928)--(5.451,1.928)--(5.462,1.929)%
  --(5.472,1.929)--(5.482,1.962)--(5.493,1.962)--(5.503,1.963)--(5.513,1.964)--(5.523,1.964)%
  --(5.534,1.965)--(5.544,1.998)--(5.554,1.999)--(5.565,2.000)--(5.575,2.001)--(5.585,2.002)%
  --(5.596,2.003)--(5.606,2.004)--(5.616,2.037)--(5.627,2.038)--(5.637,2.039)--(5.647,2.040)%
  --(5.658,2.042)--(5.668,2.043)--(5.678,2.045)--(5.689,2.078)--(5.699,2.080)--(5.709,2.081)%
  --(5.720,2.083)--(5.730,2.085)--(5.740,2.086)--(5.750,2.088)--(5.761,2.122)--(5.771,2.124)%
  --(5.781,2.126)--(5.792,2.128)--(5.802,2.130)--(5.812,2.132)--(5.823,2.134)--(5.833,2.168)%
  --(5.843,2.170)--(5.854,2.172)--(5.864,2.175)--(5.874,2.177)--(5.885,2.180)--(5.895,2.182)%
  --(5.905,2.216)--(5.916,2.219)--(5.926,2.221)--(5.936,2.224)--(5.947,2.227)--(5.957,2.238)%
  --(5.967,2.240)--(5.977,2.243)--(5.988,2.262)--(5.998,2.265)--(6.008,2.268)--(6.019,2.271)%
  --(6.029,2.290)--(6.039,2.293)--(6.050,2.297)--(6.060,2.300)--(6.070,2.319)--(6.081,2.322)%
  --(6.091,2.326)--(6.101,2.345)--(6.112,2.348)--(6.122,2.352)--(6.132,2.356)--(6.143,2.375)%
  --(6.153,2.379)--(6.163,2.383)--(6.174,2.387)--(6.184,2.390)--(6.194,2.410)--(6.204,2.414)%
  --(6.215,2.418)--(6.225,2.422)--(6.235,2.442)--(6.246,2.446)--(6.256,2.450)--(6.266,2.455)%
  --(6.277,2.475)--(6.287,2.479)--(6.297,2.483)--(6.308,2.488)--(6.318,2.508)--(6.328,2.513)%
  --(6.339,2.517)--(6.349,2.522)--(6.359,2.527)--(6.370,2.547)--(6.380,2.552)--(6.390,2.557)%
  --(6.401,2.562)--(6.411,2.582)--(6.421,2.587)--(6.431,2.592)--(6.442,2.598)--(6.452,2.603)%
  --(6.462,2.623)--(6.473,2.629)--(6.483,2.634)--(6.493,2.640)--(6.504,2.660)--(6.514,2.666)%
  --(6.524,2.671)--(6.535,2.677)--(6.545,2.683)--(6.555,2.704)--(6.566,2.710)--(6.576,2.715)%
  --(6.586,2.721)--(6.597,2.727)--(6.607,2.748)--(6.617,2.754)--(6.628,2.760)--(6.638,2.767)%
  --(6.648,2.773)--(6.658,2.779)--(6.669,2.800)--(6.679,2.807)--(6.689,2.813)--(6.700,2.819)%
  --(6.710,2.826)--(6.720,2.848)--(6.731,2.854)--(6.741,2.861)--(6.751,2.867)--(6.762,2.874)%
  --(6.772,2.881)--(6.782,2.903)--(6.793,2.910)--(6.803,2.917)--(6.813,2.924)--(6.824,2.931)%
  --(6.834,2.938)--(6.844,2.960)--(6.855,2.967)--(6.865,2.974)--(6.875,2.981)--(6.885,2.989)%
  --(6.896,2.996)--(6.906,3.019)--(6.916,3.026)--(6.927,3.034)--(6.937,3.041)--(6.947,3.049)%
  --(6.958,3.056)--(6.968,3.064)--(6.978,3.087)--(6.989,3.095)--(6.999,3.103)--(7.009,3.110)%
  --(7.020,3.118)--(7.030,3.126)--(7.040,3.134)--(7.051,3.143)--(7.061,3.166)--(7.071,3.174)%
  --(7.082,3.182)--(7.092,3.190)--(7.102,3.199)--(7.112,3.207)--(7.123,3.216)--(7.133,3.224)%
  --(7.143,3.233)--(7.154,3.256)--(7.164,3.265)--(7.174,3.274)--(7.185,3.282)--(7.195,3.291)%
  --(7.205,3.300)--(7.216,3.309)--(7.226,3.318)--(7.236,3.327)--(7.247,3.336)--(7.257,3.360)%
  --(7.267,3.369)--(7.278,3.378)--(7.288,3.387)--(7.298,3.397)--(7.309,3.406)--(7.319,3.415)%
  --(7.329,3.425)--(7.339,3.434)--(7.350,3.444)--(7.360,3.454)--(7.370,3.463)--(7.381,3.473)%
  --(7.391,3.483)--(7.401,3.507)--(7.412,3.517)--(7.422,3.527)--(7.432,3.537)--(7.443,3.547)%
  --(7.453,3.557)--(7.463,3.567)--(7.474,3.577)--(7.484,3.588)--(7.494,3.598)--(7.505,3.608)%
  --(7.515,3.619)--(7.525,3.629)--(7.536,3.639)--(7.546,3.650)--(7.556,3.660)--(7.566,3.671)%
  --(7.577,3.682)--(7.587,3.693)--(7.597,3.703)--(7.608,3.714)--(7.618,3.725)--(7.628,3.736)%
  --(7.639,3.747)--(7.649,3.758)--(7.659,3.769)--(7.670,3.780)--(7.680,3.791)--(7.690,3.803)%
  --(7.701,3.814)--(7.711,3.825)--(7.721,3.837)--(7.732,3.848)--(7.742,3.860)--(7.752,3.871)%
  --(7.763,3.883)--(7.773,3.894)--(7.783,3.906)--(7.793,3.918)--(7.804,3.930)--(7.814,3.942)%
  --(7.824,3.953)--(7.835,3.965)--(7.845,3.977)--(7.855,3.989)--(7.866,4.002)--(7.876,4.014)%
  --(7.886,4.012)--(7.897,4.024)--(7.907,4.036)--(7.917,4.049)--(7.928,4.061)--(7.938,4.074)%
  --(7.948,4.086)--(7.959,4.099)--(7.969,4.111)--(7.979,4.124)--(7.990,4.137)--(8.000,4.150)%
  --(8.010,4.162)--(8.021,4.161)--(8.031,4.174)--(8.041,4.187)--(8.051,4.200)--(8.062,4.213)%
  --(8.072,4.226)--(8.082,4.240)--(8.093,4.253)--(8.103,4.252)--(8.113,4.266)--(8.124,4.279)%
  --(8.134,4.292)--(8.144,4.306)--(8.155,4.320)--(8.165,4.333)--(8.175,4.347)--(8.186,4.347)%
  --(8.196,4.361)--(8.206,4.374)--(8.217,4.388)--(8.227,4.402)--(8.237,4.416)--(8.248,4.416)%
  --(8.258,4.431)--(8.268,4.445)--(8.278,4.459)--(8.289,4.473)--(8.299,4.474)--(8.309,4.488)%
  --(8.320,4.503)--(8.330,4.517)--(8.340,4.532)--(8.351,4.532)--(8.361,4.547)--(8.371,4.562)%
  --(8.382,4.577)--(8.392,4.592)--(8.402,4.593)--(8.413,4.608)--(8.423,4.623)--(8.433,4.638)%
  --(8.444,4.639)--(8.454,4.654)--(8.464,4.669)--(8.475,4.685)--(8.485,4.686)--(8.495,4.702)%
  --(8.505,4.717)--(8.516,4.733)--(8.526,4.735)--(8.536,4.750)--(8.547,4.766)--(8.557,4.782)%
  --(8.567,4.784)--(8.578,4.800)--(8.588,4.816)--(8.598,4.818)--(8.609,4.834)--(8.619,4.850)%
  --(8.629,4.853)--(8.640,4.869)--(8.650,4.885)--(8.660,4.888)--(8.671,4.904)--(8.681,4.921)%
  --(8.691,4.924)--(8.702,4.940)--(8.712,4.957)--(8.722,4.960)--(8.732,4.977)--(8.743,4.994)%
  --(8.753,4.997)--(8.763,5.014)--(8.774,5.031)--(8.784,5.034)--(8.794,5.052)--(8.805,5.055)%
  --(8.815,5.073)--(8.825,5.090)--(8.836,5.094)--(8.846,5.111)--(8.856,5.115)--(8.867,5.133)%
  --(8.877,5.150)--(8.887,5.155)--(8.898,5.172)--(8.908,5.177)--(8.918,5.195)--(8.929,5.199)%
  --(8.939,5.217)--(8.949,5.235)--(8.959,5.240)--(8.970,5.258)--(8.980,5.263)--(8.990,5.281)%
  --(9.001,5.286)--(9.011,5.305)--(9.021,5.310)--(9.032,5.328)--(9.042,5.334)--(9.052,5.352)%
  --(9.063,5.358)--(9.073,5.377)--(9.083,5.383)--(9.094,5.401)--(9.104,5.407)--(9.114,5.413)%
  --(9.125,5.432)--(9.135,5.438)--(9.145,5.458)--(9.156,5.464)--(9.166,5.483)--(9.176,5.490)%
  --(9.186,5.509)--(9.197,5.516)--(9.207,5.522)--(9.217,5.542)--(9.228,5.549)--(9.238,5.568)%
  --(9.248,5.575)--(9.259,5.582)--(9.269,5.602)--(9.279,5.610)--(9.290,5.617)--(9.300,5.637)%
  --(9.310,5.645)--(9.321,5.652)--(9.331,5.672)--(9.341,5.680)--(9.352,5.688)--(9.362,5.709)%
  --(9.372,5.716)--(9.383,5.725)--(9.393,5.745)--(9.403,5.753)--(9.413,5.762)--(9.424,5.770)%
  --(9.434,5.791)--(9.444,5.800)--(9.455,5.808)--(9.465,5.817)--(9.475,5.838)--(9.486,5.847)%
  --(9.496,5.856)--(9.506,5.865)--(9.517,5.887)--(9.527,5.896)--(9.537,5.905)--(9.548,5.915)%
  --(9.558,5.924)--(9.568,5.946)--(9.579,5.956)--(9.589,5.966)--(9.599,5.976)--(9.610,5.986)%
  --(9.620,5.996)--(9.630,6.006)--(9.640,6.028)--(9.651,6.038)--(9.661,6.049)--(9.671,6.059)%
  --(9.682,6.070)--(9.692,6.081)--(9.702,6.091)--(9.713,6.102)--(9.723,6.113)--(9.733,6.124)%
  --(9.744,6.135)--(9.754,6.147)--(9.764,6.158)--(9.775,6.169)--(9.785,6.181)--(9.795,6.192)%
  --(9.806,6.204)--(9.816,6.216)--(9.826,6.228)--(9.837,6.240)--(9.847,6.252)--(9.857,6.264)%
  --(9.867,6.276)--(9.878,6.289)--(9.888,6.301)--(9.898,6.314)--(9.909,6.327)--(9.919,6.328)%
  --(9.929,6.340)--(9.940,6.353)--(9.950,6.367)--(9.960,6.380)--(9.971,6.393)--(9.981,6.406)%
  --(9.991,6.408)--(10.002,6.422)--(10.012,6.436)--(10.022,6.450)--(10.033,6.452)--(10.043,6.466)%
  --(10.053,6.480)--(10.064,6.494)--(10.074,6.497)--(10.084,6.512)--(10.094,6.526)--(10.105,6.541)%
  --(10.115,6.544)--(10.125,6.559)--(10.136,6.574)--(10.146,6.578)--(10.156,6.593)--(10.167,6.597)%
  --(10.177,6.613)--(10.187,6.628)--(10.198,6.633)--(10.208,6.649)--(10.218,6.654)--(10.229,6.670)%
  --(10.239,6.675)--(10.249,6.691)--(10.260,6.696)--(10.270,6.713)--(10.280,6.718)--(10.291,6.735)%
  --(10.301,6.741)--(10.311,6.758)--(10.321,6.764)--(10.332,6.781)--(10.342,6.788)--(10.352,6.795)%
  --(10.363,6.812)--(10.373,6.819)--(10.383,6.837)--(10.394,6.844)--(10.404,6.852)--(10.414,6.859)%
  --(10.425,6.877)--(10.435,6.885)--(10.445,6.893)--(10.456,6.901)--(10.466,6.920)--(10.476,6.929)%
  --(10.487,6.937)--(10.497,6.946)--(10.507,6.955)--(10.518,6.975)--(10.528,6.984)--(10.538,6.993)%
  --(10.548,7.003)--(10.559,7.013)--(10.569,7.023)--(10.579,7.033)--(10.590,7.043)--(10.600,7.054)%
  --(10.610,7.064)--(10.621,7.075)--(10.631,7.086)--(10.641,7.097)--(10.652,7.106)--(10.662,7.117)%
  --(10.672,7.124)--(10.683,7.136)--(10.693,7.148)--(10.703,7.155)--(10.714,7.168)--(10.724,7.175)%
  --(10.734,7.188)--(10.745,7.196)--(10.755,7.204)--(10.765,7.218)--(10.775,7.226)--(10.786,7.235)%
  --(10.796,7.244)--(10.806,7.258)--(10.817,7.268)--(10.827,7.278)--(10.837,7.288)--(10.848,7.298)%
  --(10.858,7.304)--(10.868,7.314)--(10.879,7.325)--(10.889,7.336)--(10.899,7.343)--(10.910,7.354)%
  --(10.920,7.366)--(10.930,7.373)--(10.941,7.386)--(10.951,7.393)--(10.961,7.402)--(10.972,7.415)%
  --(10.982,7.423)--(10.992,7.432)--(11.002,7.441)--(11.013,7.451)--(11.023,7.461)--(11.033,7.471)%
  --(11.044,7.481)--(11.054,7.487)--(11.064,7.498)--(11.075,7.509)--(11.085,7.516)--(11.095,7.524)%
  --(11.106,7.536)--(11.116,7.544)--(11.126,7.552)--(11.137,7.561)--(11.147,7.574)--(11.157,7.579)%
  --(11.168,7.589)--(11.178,7.599)--(11.188,7.609)--(11.199,7.615)--(11.209,7.626)--(11.219,7.634)%
  --(11.229,7.645)--(11.240,7.654)--(11.250,7.662)--(11.260,7.671)--(11.271,7.680)--(11.281,7.690)%
  --(11.291,7.696)--(11.302,7.706)--(11.312,7.713)--(11.322,7.725)--(11.333,7.732)--(11.343,7.741)%
  --(11.353,7.750)--(11.364,7.759)--(11.374,7.769)--(11.384,7.775)--(11.395,7.782)--(11.405,7.793)%
  --(11.415,7.801)--(11.426,7.810)--(11.436,7.819)--(11.446,7.826)--(11.456,7.836)--(11.467,7.843)%
  --(11.477,7.851)--(11.487,7.860)--(11.498,7.866)--(11.508,7.876)--(11.518,7.883)--(11.529,7.891)%
  --(11.539,7.900)--(11.549,7.909)--(11.560,7.916)--(11.570,7.923)--(11.580,7.932)--(11.591,7.941)%
  --(11.601,7.948)--(11.611,7.956)--(11.622,7.965)--(11.632,7.972)--(11.642,7.980)--(11.653,7.988)%
  --(11.663,7.995)--(11.673,8.004)--(11.683,8.010)--(11.694,8.018)--(11.704,8.027)--(11.714,8.034)%
  --(11.725,8.041)--(11.735,8.048)--(11.745,8.055)--(11.756,8.064)--(11.766,8.071)--(11.776,8.078)%
  --(11.787,8.085)--(11.797,8.092)--(11.807,8.101)--(11.818,8.107)--(11.828,8.114)--(11.838,8.122)%
  --(11.849,8.129)--(11.859,8.136)--(11.869,8.143)--(11.880,8.150)--(11.890,8.157)--(11.900,8.164)%
  --(11.910,8.171)--(11.921,8.178)--(11.931,8.184)--(11.941,8.192);
\gpcolor{color=gp lt color border}
\draw[gp path] (1.320,8.631)--(1.320,0.985)--(13.447,0.985)--(13.447,8.631)--cycle;
%% coordinates of the plot area
\gpdefrectangularnode{gp plot 1}{\pgfpoint{1.320cm}{0.985cm}}{\pgfpoint{13.447cm}{8.631cm}}
\end{tikzpicture}
%% gnuplot variables

	\caption{Gráfico dos potenciais químicos de próton e nêutron calculados a partir da Equação~\eqref{Eq:Potenciais_Quimicos}. Como a fração de próton é 1/2, ambos os potenciais tem o mesmo valor. \protect[Parameters: eNJL1; Proton fraction: 1/2] }
	\label{Fig:chemical_potential_graph}
\end{figure*}

\begin{figure*}
	\begin{tikzpicture}[gnuplot]
%% generated with GNUPLOT 5.0p2 (Lua 5.2; terminal rev. 99, script rev. 100)
%% Fri Mar  4 16:19:10 2016
\path (0.000,0.000) rectangle (14.000,9.000);
\gpcolor{color=gp lt color border}
\gpsetlinetype{gp lt border}
\gpsetdashtype{gp dt solid}
\gpsetlinewidth{1.00}
\draw[gp path] (1.504,0.985)--(1.684,0.985);
\draw[gp path] (13.447,0.985)--(13.267,0.985);
\node[gp node right] at (1.320,0.985) {$-294$};
\draw[gp path] (1.504,1.835)--(1.684,1.835);
\draw[gp path] (13.447,1.835)--(13.267,1.835);
\node[gp node right] at (1.320,1.835) {$-292$};
\draw[gp path] (1.504,2.684)--(1.684,2.684);
\draw[gp path] (13.447,2.684)--(13.267,2.684);
\node[gp node right] at (1.320,2.684) {$-290$};
\draw[gp path] (1.504,3.534)--(1.684,3.534);
\draw[gp path] (13.447,3.534)--(13.267,3.534);
\node[gp node right] at (1.320,3.534) {$-288$};
\draw[gp path] (1.504,4.383)--(1.684,4.383);
\draw[gp path] (13.447,4.383)--(13.267,4.383);
\node[gp node right] at (1.320,4.383) {$-286$};
\draw[gp path] (1.504,5.233)--(1.684,5.233);
\draw[gp path] (13.447,5.233)--(13.267,5.233);
\node[gp node right] at (1.320,5.233) {$-284$};
\draw[gp path] (1.504,6.082)--(1.684,6.082);
\draw[gp path] (13.447,6.082)--(13.267,6.082);
\node[gp node right] at (1.320,6.082) {$-282$};
\draw[gp path] (1.504,6.932)--(1.684,6.932);
\draw[gp path] (13.447,6.932)--(13.267,6.932);
\node[gp node right] at (1.320,6.932) {$-280$};
\draw[gp path] (1.504,7.781)--(1.684,7.781);
\draw[gp path] (13.447,7.781)--(13.267,7.781);
\node[gp node right] at (1.320,7.781) {$-278$};
\draw[gp path] (1.504,8.631)--(1.684,8.631);
\draw[gp path] (13.447,8.631)--(13.267,8.631);
\node[gp node right] at (1.320,8.631) {$-276$};
\draw[gp path] (1.504,0.985)--(1.504,1.165);
\draw[gp path] (1.504,8.631)--(1.504,8.451);
\node[gp node center] at (1.504,0.677) {$0$};
\draw[gp path] (2.997,0.985)--(2.997,1.165);
\draw[gp path] (2.997,8.631)--(2.997,8.451);
\node[gp node center] at (2.997,0.677) {$0.05$};
\draw[gp path] (4.490,0.985)--(4.490,1.165);
\draw[gp path] (4.490,8.631)--(4.490,8.451);
\node[gp node center] at (4.490,0.677) {$0.1$};
\draw[gp path] (5.983,0.985)--(5.983,1.165);
\draw[gp path] (5.983,8.631)--(5.983,8.451);
\node[gp node center] at (5.983,0.677) {$0.15$};
\draw[gp path] (7.476,0.985)--(7.476,1.165);
\draw[gp path] (7.476,8.631)--(7.476,8.451);
\node[gp node center] at (7.476,0.677) {$0.2$};
\draw[gp path] (8.968,0.985)--(8.968,1.165);
\draw[gp path] (8.968,8.631)--(8.968,8.451);
\node[gp node center] at (8.968,0.677) {$0.25$};
\draw[gp path] (10.461,0.985)--(10.461,1.165);
\draw[gp path] (10.461,8.631)--(10.461,8.451);
\node[gp node center] at (10.461,0.677) {$0.3$};
\draw[gp path] (11.954,0.985)--(11.954,1.165);
\draw[gp path] (11.954,8.631)--(11.954,8.451);
\node[gp node center] at (11.954,0.677) {$0.35$};
\draw[gp path] (13.447,0.985)--(13.447,1.165);
\draw[gp path] (13.447,8.631)--(13.447,8.451);
\node[gp node center] at (13.447,0.677) {$0.4$};
\draw[gp path] (1.504,8.631)--(1.504,0.985)--(13.447,0.985)--(13.447,8.631)--cycle;
\node[gp node center,rotate=-270] at (0.246,4.808) {$\omega$ ($\rm{MeV}/\rm{fm}^{3}$)};
\node[gp node center] at (7.475,0.215) {$\rho$ ($\rm{fm}^{-3}$)};
\gpcolor{rgb color={0.580,0.000,0.827}}
\draw[gp path] (1.813,8.276)--(1.823,8.277)--(1.833,8.278)--(1.843,8.278)--(1.853,8.279)%
  --(1.864,8.280)--(1.874,8.281)--(1.884,8.282)--(1.894,8.283)--(1.904,8.284)--(1.914,8.285)%
  --(1.925,8.286)--(1.935,8.287)--(1.945,8.288)--(1.955,8.287)--(1.965,8.288)--(1.975,8.289)%
  --(1.985,8.290)--(1.996,8.291)--(2.006,8.292)--(2.016,8.293)--(2.026,8.295)--(2.036,8.296)%
  --(2.046,8.297)--(2.057,8.299)--(2.067,8.300)--(2.077,8.298)--(2.087,8.300)--(2.097,8.301)%
  --(2.107,8.302)--(2.118,8.304)--(2.128,8.305)--(2.138,8.307)--(2.148,8.308)--(2.158,8.310)%
  --(2.168,8.312)--(2.179,8.313)--(2.189,8.311)--(2.199,8.313)--(2.209,8.314)--(2.219,8.316)%
  --(2.229,8.318)--(2.240,8.320)--(2.250,8.321)--(2.260,8.323)--(2.270,8.325)--(2.280,8.327)%
  --(2.290,8.328)--(2.300,8.326)--(2.311,8.328)--(2.321,8.330)--(2.331,8.332)--(2.341,8.334)%
  --(2.351,8.336)--(2.361,8.338)--(2.372,8.340)--(2.382,8.342)--(2.392,8.344)--(2.402,8.341)%
  --(2.412,8.343)--(2.422,8.345)--(2.433,8.347)--(2.443,8.349)--(2.453,8.351)--(2.463,8.353)%
  --(2.473,8.356)--(2.483,8.358)--(2.494,8.355)--(2.504,8.357)--(2.514,8.359)--(2.524,8.362)%
  --(2.534,8.364)--(2.544,8.366)--(2.555,8.368)--(2.565,8.371)--(2.575,8.373)--(2.585,8.370)%
  --(2.595,8.372)--(2.605,8.374)--(2.616,8.377)--(2.626,8.379)--(2.636,8.382)--(2.646,8.384)%
  --(2.656,8.386)--(2.666,8.383)--(2.676,8.385)--(2.687,8.388)--(2.697,8.390)--(2.707,8.393)%
  --(2.717,8.395)--(2.727,8.398)--(2.737,8.400)--(2.748,8.403)--(2.758,8.399)--(2.768,8.402)%
  --(2.778,8.404)--(2.788,8.407)--(2.798,8.409)--(2.809,8.412)--(2.819,8.415)--(2.829,8.417)%
  --(2.839,8.413)--(2.849,8.416)--(2.859,8.419)--(2.870,8.421)--(2.880,8.424)--(2.890,8.427)%
  --(2.900,8.429)--(2.910,8.432)--(2.920,8.428)--(2.931,8.430)--(2.941,8.433)--(2.951,8.436)%
  --(2.961,8.439)--(2.971,8.441)--(2.981,8.444)--(2.991,8.439)--(3.002,8.442)--(3.012,8.445)%
  --(3.022,8.448)--(3.032,8.451)--(3.042,8.454)--(3.052,8.456)--(3.063,8.459)--(3.073,8.454)%
  --(3.083,8.457)--(3.093,8.460)--(3.103,8.463)--(3.113,8.466)--(3.124,8.469)--(3.134,8.471)%
  --(3.144,8.466)--(3.154,8.469)--(3.164,8.472)--(3.174,8.475)--(3.185,8.478)--(3.195,8.481)%
  --(3.205,8.484)--(3.215,8.478)--(3.225,8.481)--(3.235,8.484)--(3.246,8.487)--(3.256,8.490)%
  --(3.266,8.493)--(3.276,8.496)--(3.286,8.490)--(3.296,8.493)--(3.307,8.495)--(3.317,8.498)%
  --(3.327,8.501)--(3.337,8.504)--(3.347,8.507)--(3.357,8.501)--(3.367,8.504)--(3.378,8.507)%
  --(3.388,8.510)--(3.398,8.513)--(3.408,8.516)--(3.418,8.519)--(3.428,8.512)--(3.439,8.515)%
  --(3.449,8.518)--(3.459,8.521)--(3.469,8.524)--(3.479,8.527)--(3.489,8.530)--(3.500,8.523)%
  --(3.510,8.526)--(3.520,8.529)--(3.530,8.532)--(3.540,8.535)--(3.550,8.538)--(3.561,8.541)%
  --(3.571,8.534)--(3.581,8.537)--(3.591,8.540)--(3.601,8.543)--(3.611,8.546)--(3.622,8.549)%
  --(3.632,8.541)--(3.642,8.544)--(3.652,8.547)--(3.662,8.550)--(3.672,8.553)--(3.682,8.556)%
  --(3.693,8.559)--(3.703,8.551)--(3.713,8.554)--(3.723,8.557)--(3.733,8.560)--(3.743,8.563)%
  --(3.754,8.566)--(3.764,8.558)--(3.774,8.561)--(3.784,8.564)--(3.794,8.567)--(3.804,8.570)%
  --(3.815,8.573)--(3.825,8.575)--(3.835,8.567)--(3.845,8.570)--(3.855,8.573)--(3.865,8.575)%
  --(3.876,8.578)--(3.886,8.581)--(3.896,8.572)--(3.906,8.575)--(3.916,8.578)--(3.926,8.581)%
  --(3.937,8.584)--(3.947,8.587)--(3.957,8.577)--(3.967,8.580)--(3.977,8.583)--(3.987,8.586)%
  --(3.998,8.589)--(4.008,8.591)--(4.018,8.594)--(4.028,8.585)--(4.038,8.587)--(4.048,8.590)%
  --(4.058,8.593)--(4.069,8.596)--(4.079,8.598)--(4.089,8.588)--(4.099,8.591)--(4.109,8.594)%
  --(4.119,8.596)--(4.130,8.599)--(4.140,8.602)--(4.150,8.591)--(4.160,8.594)--(4.170,8.597)%
  --(4.180,8.599)--(4.191,8.602)--(4.201,8.605)--(4.211,8.594)--(4.221,8.597)--(4.231,8.599)%
  --(4.241,8.602)--(4.252,8.604)--(4.262,8.607)--(4.272,8.596)--(4.282,8.598)--(4.292,8.601)%
  --(4.302,8.603)--(4.313,8.606)--(4.323,8.608)--(4.333,8.597)--(4.343,8.599)--(4.353,8.602)%
  --(4.363,8.604)--(4.373,8.607)--(4.384,8.609)--(4.394,8.611)--(4.404,8.600)--(4.414,8.602)%
  --(4.424,8.604)--(4.434,8.607)--(4.445,8.609)--(4.455,8.611)--(4.465,8.599)--(4.475,8.601)%
  --(4.485,8.604)--(4.495,8.606)--(4.506,8.608)--(4.516,8.610)--(4.526,8.598)--(4.536,8.600)%
  --(4.546,8.602)--(4.556,8.604)--(4.567,8.606)--(4.577,8.608)--(4.587,8.596)--(4.597,8.598)%
  --(4.607,8.600)--(4.617,8.602)--(4.628,8.604)--(4.638,8.606)--(4.648,8.592)--(4.658,8.594)%
  --(4.668,8.596)--(4.678,8.598)--(4.689,8.600)--(4.699,8.602)--(4.709,8.588)--(4.719,8.590)%
  --(4.729,8.592)--(4.739,8.594)--(4.749,8.596)--(4.760,8.598)--(4.770,8.584)--(4.780,8.585)%
  --(4.790,8.587)--(4.800,8.589)--(4.810,8.590)--(4.821,8.592)--(4.831,8.578)--(4.841,8.579)%
  --(4.851,8.581)--(4.861,8.582)--(4.871,8.584)--(4.882,8.586)--(4.892,8.571)--(4.902,8.572)%
  --(4.912,8.574)--(4.922,8.575)--(4.932,8.577)--(4.943,8.578)--(4.953,8.563)--(4.963,8.564)%
  --(4.973,8.565)--(4.983,8.567)--(4.993,8.568)--(5.004,8.569)--(5.014,8.554)--(5.024,8.555)%
  --(5.034,8.556)--(5.044,8.557)--(5.054,8.559)--(5.064,8.560)--(5.075,8.561)--(5.085,8.545)%
  --(5.095,8.546)--(5.105,8.547)--(5.115,8.548)--(5.125,8.549)--(5.136,8.550)--(5.146,8.533)%
  --(5.156,8.534)--(5.166,8.535)--(5.176,8.536)--(5.186,8.537)--(5.197,8.538)--(5.207,8.521)%
  --(5.217,8.521)--(5.227,8.522)--(5.237,8.523)--(5.247,8.524)--(5.258,8.524)--(5.268,8.507)%
  --(5.278,8.507)--(5.288,8.508)--(5.298,8.509)--(5.308,8.509)--(5.319,8.510)--(5.329,8.510)%
  --(5.339,8.492)--(5.349,8.493)--(5.359,8.493)--(5.369,8.493)--(5.379,8.494)--(5.390,8.494)%
  --(5.400,8.476)--(5.410,8.476)--(5.420,8.476)--(5.430,8.476)--(5.440,8.477)--(5.451,8.477)%
  --(5.461,8.477)--(5.471,8.458)--(5.481,8.458)--(5.491,8.458)--(5.501,8.458)--(5.512,8.458)%
  --(5.522,8.458)--(5.532,8.439)--(5.542,8.439)--(5.552,8.439)--(5.562,8.438)--(5.573,8.438)%
  --(5.583,8.438)--(5.593,8.438)--(5.603,8.418)--(5.613,8.418)--(5.623,8.417)--(5.634,8.417)%
  --(5.644,8.416)--(5.654,8.416)--(5.664,8.396)--(5.674,8.395)--(5.684,8.394)--(5.695,8.394)%
  --(5.705,8.393)--(5.715,8.393)--(5.725,8.392)--(5.735,8.371)--(5.745,8.370)--(5.755,8.370)%
  --(5.766,8.369)--(5.776,8.368)--(5.786,8.367)--(5.796,8.366)--(5.806,8.345)--(5.816,8.344)%
  --(5.827,8.343)--(5.837,8.342)--(5.847,8.341)--(5.857,8.339)--(5.867,8.338)--(5.877,8.316)%
  --(5.888,8.315)--(5.898,8.314)--(5.908,8.313)--(5.918,8.311)--(5.928,8.310)--(5.938,8.308)%
  --(5.949,8.286)--(5.959,8.285)--(5.969,8.283)--(5.979,8.281)--(5.989,8.280)--(5.999,8.278)%
  --(6.010,8.276)--(6.020,8.254)--(6.030,8.252)--(6.040,8.250)--(6.050,8.248)--(6.060,8.246)%
  --(6.070,8.239)--(6.081,8.237)--(6.091,8.235)--(6.101,8.222)--(6.111,8.220)--(6.121,8.218)%
  --(6.131,8.216)--(6.142,8.203)--(6.152,8.201)--(6.162,8.199)--(6.172,8.196)--(6.182,8.183)%
  --(6.192,8.181)--(6.203,8.178)--(6.213,8.165)--(6.223,8.162)--(6.233,8.160)--(6.243,8.157)%
  --(6.253,8.144)--(6.264,8.141)--(6.274,8.138)--(6.284,8.135)--(6.294,8.133)--(6.304,8.119)%
  --(6.314,8.116)--(6.325,8.113)--(6.335,8.110)--(6.345,8.096)--(6.355,8.093)--(6.365,8.090)%
  --(6.375,8.087)--(6.386,8.072)--(6.396,8.069)--(6.406,8.066)--(6.416,8.062)--(6.426,8.048)%
  --(6.436,8.044)--(6.446,8.041)--(6.457,8.037)--(6.467,8.034)--(6.477,8.019)--(6.487,8.015)%
  --(6.497,8.012)--(6.507,8.008)--(6.518,7.993)--(6.528,7.989)--(6.538,7.985)--(6.548,7.981)%
  --(6.558,7.977)--(6.568,7.962)--(6.579,7.958)--(6.589,7.954)--(6.599,7.950)--(6.609,7.934)%
  --(6.619,7.930)--(6.629,7.925)--(6.640,7.921)--(6.650,7.917)--(6.660,7.901)--(6.670,7.896)%
  --(6.680,7.892)--(6.690,7.887)--(6.701,7.883)--(6.711,7.866)--(6.721,7.861)--(6.731,7.857)%
  --(6.741,7.852)--(6.751,7.847)--(6.761,7.842)--(6.772,7.825)--(6.782,7.821)--(6.792,7.816)%
  --(6.802,7.811)--(6.812,7.805)--(6.822,7.788)--(6.833,7.783)--(6.843,7.778)--(6.853,7.772)%
  --(6.863,7.767)--(6.873,7.762)--(6.883,7.744)--(6.894,7.739)--(6.904,7.733)--(6.914,7.728)%
  --(6.924,7.722)--(6.934,7.716)--(6.944,7.698)--(6.955,7.692)--(6.965,7.687)--(6.975,7.681)%
  --(6.985,7.675)--(6.995,7.669)--(7.005,7.650)--(7.016,7.644)--(7.026,7.638)--(7.036,7.632)%
  --(7.046,7.626)--(7.056,7.619)--(7.066,7.613)--(7.077,7.594)--(7.087,7.587)--(7.097,7.581)%
  --(7.107,7.574)--(7.117,7.568)--(7.127,7.561)--(7.137,7.554)--(7.148,7.548)--(7.158,7.528)%
  --(7.168,7.521)--(7.178,7.514)--(7.188,7.507)--(7.198,7.500)--(7.209,7.493)--(7.219,7.486)%
  --(7.229,7.479)--(7.239,7.471)--(7.249,7.451)--(7.259,7.444)--(7.270,7.436)--(7.280,7.429)%
  --(7.290,7.421)--(7.300,7.414)--(7.310,7.406)--(7.320,7.398)--(7.331,7.390)--(7.341,7.383)%
  --(7.351,7.362)--(7.361,7.354)--(7.371,7.346)--(7.381,7.338)--(7.392,7.329)--(7.402,7.321)%
  --(7.412,7.313)--(7.422,7.305)--(7.432,7.296)--(7.442,7.288)--(7.452,7.279)--(7.463,7.271)%
  --(7.473,7.262)--(7.483,7.254)--(7.493,7.232)--(7.503,7.223)--(7.513,7.214)--(7.524,7.205)%
  --(7.534,7.196)--(7.544,7.187)--(7.554,7.178)--(7.564,7.169)--(7.574,7.160)--(7.585,7.150)%
  --(7.595,7.141)--(7.605,7.131)--(7.615,7.122)--(7.625,7.112)--(7.635,7.103)--(7.646,7.093)%
  --(7.656,7.084)--(7.666,7.074)--(7.676,7.064)--(7.686,7.054)--(7.696,7.044)--(7.707,7.034)%
  --(7.717,7.024)--(7.727,7.014)--(7.737,7.004)--(7.747,6.993)--(7.757,6.983)--(7.767,6.972)%
  --(7.778,6.962)--(7.788,6.951)--(7.798,6.941)--(7.808,6.930)--(7.818,6.919)--(7.828,6.909)%
  --(7.839,6.898)--(7.849,6.887)--(7.859,6.876)--(7.869,6.865)--(7.879,6.853)--(7.889,6.842)%
  --(7.900,6.831)--(7.910,6.820)--(7.920,6.808)--(7.930,6.797)--(7.940,6.785)--(7.950,6.774)%
  --(7.961,6.762)--(7.971,6.764)--(7.981,6.752)--(7.991,6.740)--(8.001,6.728)--(8.011,6.716)%
  --(8.022,6.704)--(8.032,6.692)--(8.042,6.680)--(8.052,6.667)--(8.062,6.655)--(8.072,6.642)%
  --(8.083,6.630)--(8.093,6.617)--(8.103,6.619)--(8.113,6.606)--(8.123,6.593)--(8.133,6.580)%
  --(8.143,6.567)--(8.154,6.554)--(8.164,6.541)--(8.174,6.528)--(8.184,6.529)--(8.194,6.515)%
  --(8.204,6.502)--(8.215,6.489)--(8.225,6.475)--(8.235,6.461)--(8.245,6.448)--(8.255,6.434)%
  --(8.265,6.434)--(8.276,6.420)--(8.286,6.406)--(8.296,6.392)--(8.306,6.378)--(8.316,6.364)%
  --(8.326,6.364)--(8.337,6.349)--(8.347,6.335)--(8.357,6.320)--(8.367,6.306)--(8.377,6.305)%
  --(8.387,6.291)--(8.398,6.276)--(8.408,6.261)--(8.418,6.246)--(8.428,6.245)--(8.438,6.230)%
  --(8.448,6.215)--(8.458,6.199)--(8.469,6.184)--(8.479,6.183)--(8.489,6.167)--(8.499,6.152)%
  --(8.509,6.136)--(8.519,6.135)--(8.530,6.119)--(8.540,6.103)--(8.550,6.087)--(8.560,6.085)%
  --(8.570,6.069)--(8.580,6.053)--(8.591,6.036)--(8.601,6.034)--(8.611,6.018)--(8.621,6.001)%
  --(8.631,5.984)--(8.641,5.982)--(8.652,5.965)--(8.662,5.948)--(8.672,5.946)--(8.682,5.929)%
  --(8.692,5.912)--(8.702,5.909)--(8.713,5.891)--(8.723,5.874)--(8.733,5.871)--(8.743,5.853)%
  --(8.753,5.836)--(8.763,5.832)--(8.774,5.815)--(8.784,5.797)--(8.794,5.793)--(8.804,5.775)%
  --(8.814,5.757)--(8.824,5.753)--(8.834,5.735)--(8.845,5.716)--(8.855,5.712)--(8.865,5.693)%
  --(8.875,5.689)--(8.885,5.670)--(8.895,5.651)--(8.906,5.647)--(8.916,5.628)--(8.926,5.623)%
  --(8.936,5.604)--(8.946,5.585)--(8.956,5.580)--(8.967,5.560)--(8.977,5.555)--(8.987,5.535)%
  --(8.997,5.530)--(9.007,5.510)--(9.017,5.490)--(9.028,5.485)--(9.038,5.464)--(9.048,5.459)%
  --(9.058,5.438)--(9.068,5.433)--(9.078,5.412)--(9.089,5.406)--(9.099,5.385)--(9.109,5.379)%
  --(9.119,5.358)--(9.129,5.352)--(9.139,5.330)--(9.149,5.324)--(9.160,5.302)--(9.170,5.296)%
  --(9.180,5.289)--(9.190,5.267)--(9.200,5.260)--(9.210,5.238)--(9.221,5.231)--(9.231,5.209)%
  --(9.241,5.201)--(9.251,5.179)--(9.261,5.171)--(9.271,5.164)--(9.282,5.141)--(9.292,5.133)%
  --(9.302,5.110)--(9.312,5.102)--(9.322,5.094)--(9.332,5.070)--(9.343,5.062)--(9.353,5.053)%
  --(9.363,5.030)--(9.373,5.021)--(9.383,5.012)--(9.393,4.988)--(9.404,4.979)--(9.414,4.970)%
  --(9.424,4.946)--(9.434,4.937)--(9.444,4.927)--(9.454,4.903)--(9.465,4.893)--(9.475,4.883)%
  --(9.485,4.873)--(9.495,4.848)--(9.505,4.838)--(9.515,4.828)--(9.525,4.817)--(9.536,4.792)%
  --(9.546,4.781)--(9.556,4.770)--(9.566,4.760)--(9.576,4.734)--(9.586,4.723)--(9.597,4.711)%
  --(9.607,4.700)--(9.617,4.688)--(9.627,4.662)--(9.637,4.650)--(9.647,4.638)--(9.658,4.626)%
  --(9.668,4.614)--(9.678,4.602)--(9.688,4.590)--(9.698,4.563)--(9.708,4.550)--(9.719,4.537)%
  --(9.729,4.524)--(9.739,4.511)--(9.749,4.498)--(9.759,4.485)--(9.769,4.472)--(9.780,4.458)%
  --(9.790,4.445)--(9.800,4.431)--(9.810,4.417)--(9.820,4.403)--(9.830,4.389)--(9.840,4.375)%
  --(9.851,4.360)--(9.861,4.346)--(9.871,4.331)--(9.881,4.316)--(9.891,4.301)--(9.901,4.286)%
  --(9.912,4.271)--(9.922,4.255)--(9.932,4.240)--(9.942,4.224)--(9.952,4.208)--(9.962,4.192)%
  --(9.973,4.191)--(9.983,4.175)--(9.993,4.158)--(10.003,4.142)--(10.013,4.125)--(10.023,4.108)%
  --(10.034,4.091)--(10.044,4.089)--(10.054,4.072)--(10.064,4.054)--(10.074,4.037)--(10.084,4.033)%
  --(10.095,4.016)--(10.105,3.997)--(10.115,3.979)--(10.125,3.975)--(10.135,3.957)--(10.145,3.938)%
  --(10.156,3.919)--(10.166,3.915)--(10.176,3.896)--(10.186,3.876)--(10.196,3.871)--(10.206,3.852)%
  --(10.216,3.846)--(10.227,3.826)--(10.237,3.806)--(10.247,3.800)--(10.257,3.780)--(10.267,3.773)%
  --(10.277,3.752)--(10.288,3.746)--(10.298,3.725)--(10.308,3.718)--(10.318,3.696)--(10.328,3.688)%
  --(10.338,3.667)--(10.349,3.659)--(10.359,3.636)--(10.369,3.628)--(10.379,3.605)--(10.389,3.597)%
  --(10.399,3.588)--(10.410,3.565)--(10.420,3.556)--(10.430,3.532)--(10.440,3.522)--(10.450,3.512)%
  --(10.460,3.502)--(10.471,3.478)--(10.481,3.467)--(10.491,3.457)--(10.501,3.446)--(10.511,3.421)%
  --(10.521,3.409)--(10.531,3.398)--(10.542,3.386)--(10.552,3.374)--(10.562,3.347)--(10.572,3.335)%
  --(10.582,3.322)--(10.592,3.309)--(10.603,3.296)--(10.613,3.282)--(10.623,3.269)--(10.633,3.255)%
  --(10.643,3.241)--(10.653,3.226)--(10.664,3.211)--(10.674,3.196)--(10.684,3.181)--(10.694,3.169)%
  --(10.704,3.153)--(10.714,3.144)--(10.725,3.128)--(10.735,3.112)--(10.745,3.102)--(10.755,3.084)%
  --(10.765,3.074)--(10.775,3.056)--(10.786,3.045)--(10.796,3.034)--(10.806,3.015)--(10.816,3.003)%
  --(10.826,2.991)--(10.836,2.978)--(10.846,2.959)--(10.857,2.945)--(10.867,2.932)--(10.877,2.918)%
  --(10.887,2.904)--(10.897,2.896)--(10.907,2.881)--(10.918,2.866)--(10.928,2.850)--(10.938,2.841)%
  --(10.948,2.825)--(10.958,2.808)--(10.968,2.798)--(10.979,2.781)--(10.989,2.769)--(10.999,2.758)%
  --(11.009,2.740)--(11.019,2.727)--(11.029,2.715)--(11.040,2.702)--(11.050,2.688)--(11.060,2.674)%
  --(11.070,2.660)--(11.080,2.646)--(11.090,2.637)--(11.101,2.621)--(11.111,2.605)--(11.121,2.595)%
  --(11.131,2.584)--(11.141,2.567)--(11.151,2.556)--(11.162,2.544)--(11.172,2.531)--(11.182,2.512)%
  --(11.192,2.505)--(11.202,2.491)--(11.212,2.477)--(11.222,2.462)--(11.233,2.452)--(11.243,2.436)%
  --(11.253,2.426)--(11.263,2.409)--(11.273,2.397)--(11.283,2.385)--(11.294,2.372)--(11.304,2.358)%
  --(11.314,2.344)--(11.324,2.335)--(11.334,2.320)--(11.344,2.310)--(11.355,2.293)--(11.365,2.282)%
  --(11.375,2.269)--(11.385,2.256)--(11.395,2.243)--(11.405,2.228)--(11.416,2.219)--(11.426,2.208)%
  --(11.436,2.192)--(11.446,2.180)--(11.456,2.167)--(11.466,2.153)--(11.477,2.144)--(11.487,2.129)%
  --(11.497,2.118)--(11.507,2.106)--(11.517,2.093)--(11.527,2.084)--(11.537,2.070)--(11.548,2.059)%
  --(11.558,2.047)--(11.568,2.034)--(11.578,2.020)--(11.588,2.010)--(11.598,1.998)--(11.609,1.985)%
  --(11.619,1.971)--(11.629,1.961)--(11.639,1.949)--(11.649,1.935)--(11.659,1.925)--(11.670,1.913)%
  --(11.680,1.900)--(11.690,1.889)--(11.700,1.877)--(11.710,1.867)--(11.720,1.855)--(11.731,1.842)%
  --(11.741,1.830)--(11.751,1.821)--(11.761,1.809)--(11.771,1.798)--(11.781,1.785)--(11.792,1.774)%
  --(11.802,1.763)--(11.812,1.752)--(11.822,1.742)--(11.832,1.729)--(11.842,1.718)--(11.853,1.708)%
  --(11.863,1.696)--(11.873,1.685)--(11.883,1.674)--(11.893,1.663)--(11.903,1.653)--(11.913,1.642)%
  --(11.924,1.631)--(11.934,1.620)--(11.944,1.610)--(11.954,1.599)--(11.964,1.588);
\gpcolor{color=gp lt color border}
\draw[gp path] (1.504,8.631)--(1.504,0.985)--(13.447,0.985)--(13.447,8.631)--cycle;
%% coordinates of the plot area
\gpdefrectangularnode{gp plot 1}{\pgfpoint{1.504cm}{0.985cm}}{\pgfpoint{13.447cm}{8.631cm}}
\end{tikzpicture}
%% gnuplot variables

	\caption{Gráfico do potencial grand canônico por unidade de volume obtido através da Equação~\eqref{Eq:potencial_termodinamico}. \protect[Parameters: eNJL1; Proton fraction: 1/2] }
	\label{Fig:thermodynamic_potential_graph}
\end{figure*}

\begin{figure*}
	\begin{tikzpicture}[gnuplot]
%% generated with GNUPLOT 5.0p2 (Lua 5.2; terminal rev. 99, script rev. 100)
%% Fri Mar  4 16:19:10 2016
\path (0.000,0.000) rectangle (14.000,9.000);
\gpcolor{color=gp lt color border}
\gpsetlinetype{gp lt border}
\gpsetdashtype{gp dt solid}
\gpsetlinewidth{1.00}
\draw[gp path] (1.136,0.985)--(1.316,0.985);
\draw[gp path] (13.447,0.985)--(13.267,0.985);
\node[gp node right] at (0.952,0.985) {$-2$};
\draw[gp path] (1.136,1.835)--(1.316,1.835);
\draw[gp path] (13.447,1.835)--(13.267,1.835);
\node[gp node right] at (0.952,1.835) {$0$};
\draw[gp path] (1.136,2.684)--(1.316,2.684);
\draw[gp path] (13.447,2.684)--(13.267,2.684);
\node[gp node right] at (0.952,2.684) {$2$};
\draw[gp path] (1.136,3.534)--(1.316,3.534);
\draw[gp path] (13.447,3.534)--(13.267,3.534);
\node[gp node right] at (0.952,3.534) {$4$};
\draw[gp path] (1.136,4.383)--(1.316,4.383);
\draw[gp path] (13.447,4.383)--(13.267,4.383);
\node[gp node right] at (0.952,4.383) {$6$};
\draw[gp path] (1.136,5.233)--(1.316,5.233);
\draw[gp path] (13.447,5.233)--(13.267,5.233);
\node[gp node right] at (0.952,5.233) {$8$};
\draw[gp path] (1.136,6.082)--(1.316,6.082);
\draw[gp path] (13.447,6.082)--(13.267,6.082);
\node[gp node right] at (0.952,6.082) {$10$};
\draw[gp path] (1.136,6.932)--(1.316,6.932);
\draw[gp path] (13.447,6.932)--(13.267,6.932);
\node[gp node right] at (0.952,6.932) {$12$};
\draw[gp path] (1.136,7.781)--(1.316,7.781);
\draw[gp path] (13.447,7.781)--(13.267,7.781);
\node[gp node right] at (0.952,7.781) {$14$};
\draw[gp path] (1.136,8.631)--(1.316,8.631);
\draw[gp path] (13.447,8.631)--(13.267,8.631);
\node[gp node right] at (0.952,8.631) {$16$};
\draw[gp path] (1.136,0.985)--(1.136,1.165);
\draw[gp path] (1.136,8.631)--(1.136,8.451);
\node[gp node center] at (1.136,0.677) {$0$};
\draw[gp path] (2.675,0.985)--(2.675,1.165);
\draw[gp path] (2.675,8.631)--(2.675,8.451);
\node[gp node center] at (2.675,0.677) {$0.05$};
\draw[gp path] (4.214,0.985)--(4.214,1.165);
\draw[gp path] (4.214,8.631)--(4.214,8.451);
\node[gp node center] at (4.214,0.677) {$0.1$};
\draw[gp path] (5.753,0.985)--(5.753,1.165);
\draw[gp path] (5.753,8.631)--(5.753,8.451);
\node[gp node center] at (5.753,0.677) {$0.15$};
\draw[gp path] (7.292,0.985)--(7.292,1.165);
\draw[gp path] (7.292,8.631)--(7.292,8.451);
\node[gp node center] at (7.292,0.677) {$0.2$};
\draw[gp path] (8.830,0.985)--(8.830,1.165);
\draw[gp path] (8.830,8.631)--(8.830,8.451);
\node[gp node center] at (8.830,0.677) {$0.25$};
\draw[gp path] (10.369,0.985)--(10.369,1.165);
\draw[gp path] (10.369,8.631)--(10.369,8.451);
\node[gp node center] at (10.369,0.677) {$0.3$};
\draw[gp path] (11.908,0.985)--(11.908,1.165);
\draw[gp path] (11.908,8.631)--(11.908,8.451);
\node[gp node center] at (11.908,0.677) {$0.35$};
\draw[gp path] (13.447,0.985)--(13.447,1.165);
\draw[gp path] (13.447,8.631)--(13.447,8.451);
\node[gp node center] at (13.447,0.677) {$0.4$};
\draw[gp path] (1.136,8.631)--(1.136,0.985)--(13.447,0.985)--(13.447,8.631)--cycle;
\node[gp node center,rotate=-270] at (0.246,4.808) {$P$ ($\rm{MeV}/\rm{fm}^3$)};
\node[gp node center] at (7.291,0.215) {$\rho$ ($\rm{fm}^{-3}$)};
\gpcolor{rgb color={0.580,0.000,0.827}}
\draw[gp path] (1.454,1.827)--(1.465,1.826)--(1.475,1.826)--(1.486,1.825)--(1.496,1.824)%
  --(1.507,1.823)--(1.517,1.822)--(1.528,1.821)--(1.538,1.821)--(1.549,1.820)--(1.559,1.819)%
  --(1.569,1.818)--(1.580,1.817)--(1.590,1.816)--(1.601,1.817)--(1.611,1.816)--(1.622,1.815)%
  --(1.632,1.813)--(1.643,1.812)--(1.653,1.811)--(1.664,1.810)--(1.674,1.809)--(1.685,1.807)%
  --(1.695,1.806)--(1.706,1.805)--(1.716,1.803)--(1.727,1.805)--(1.737,1.804)--(1.748,1.802)%
  --(1.758,1.801)--(1.768,1.799)--(1.779,1.798)--(1.789,1.796)--(1.800,1.795)--(1.810,1.793)%
  --(1.821,1.792)--(1.831,1.790)--(1.842,1.792)--(1.852,1.790)--(1.863,1.789)--(1.873,1.787)%
  --(1.884,1.785)--(1.894,1.784)--(1.905,1.782)--(1.915,1.780)--(1.926,1.779)--(1.936,1.777)%
  --(1.947,1.775)--(1.957,1.777)--(1.968,1.775)--(1.978,1.773)--(1.988,1.772)--(1.999,1.770)%
  --(2.009,1.768)--(2.020,1.766)--(2.030,1.764)--(2.041,1.762)--(2.051,1.760)--(2.062,1.762)%
  --(2.072,1.760)--(2.083,1.758)--(2.093,1.756)--(2.104,1.754)--(2.114,1.752)--(2.125,1.750)%
  --(2.135,1.748)--(2.146,1.745)--(2.156,1.748)--(2.167,1.746)--(2.177,1.744)--(2.187,1.742)%
  --(2.198,1.740)--(2.208,1.737)--(2.219,1.735)--(2.229,1.733)--(2.240,1.730)--(2.250,1.733)%
  --(2.261,1.731)--(2.271,1.729)--(2.282,1.726)--(2.292,1.724)--(2.303,1.722)--(2.313,1.719)%
  --(2.324,1.717)--(2.334,1.720)--(2.345,1.718)--(2.355,1.715)--(2.366,1.713)--(2.376,1.710)%
  --(2.387,1.708)--(2.397,1.705)--(2.407,1.703)--(2.418,1.700)--(2.428,1.704)--(2.439,1.702)%
  --(2.449,1.699)--(2.460,1.696)--(2.470,1.694)--(2.481,1.691)--(2.491,1.689)--(2.502,1.686)%
  --(2.512,1.690)--(2.523,1.687)--(2.533,1.685)--(2.544,1.682)--(2.554,1.679)--(2.565,1.677)%
  --(2.575,1.674)--(2.586,1.671)--(2.596,1.676)--(2.606,1.673)--(2.617,1.670)--(2.627,1.667)%
  --(2.638,1.665)--(2.648,1.662)--(2.659,1.659)--(2.669,1.664)--(2.680,1.661)--(2.690,1.658)%
  --(2.701,1.655)--(2.711,1.653)--(2.722,1.650)--(2.732,1.647)--(2.743,1.644)--(2.753,1.649)%
  --(2.764,1.646)--(2.774,1.643)--(2.785,1.641)--(2.795,1.638)--(2.806,1.635)--(2.816,1.632)%
  --(2.826,1.637)--(2.837,1.634)--(2.847,1.631)--(2.858,1.629)--(2.868,1.626)--(2.879,1.623)%
  --(2.889,1.620)--(2.900,1.625)--(2.910,1.622)--(2.921,1.620)--(2.931,1.617)--(2.942,1.614)%
  --(2.952,1.611)--(2.963,1.608)--(2.973,1.614)--(2.984,1.611)--(2.994,1.608)--(3.005,1.605)%
  --(3.015,1.602)--(3.025,1.599)--(3.036,1.596)--(3.046,1.602)--(3.057,1.599)--(3.067,1.596)%
  --(3.078,1.593)--(3.088,1.590)--(3.099,1.587)--(3.109,1.584)--(3.120,1.591)--(3.130,1.588)%
  --(3.141,1.585)--(3.151,1.582)--(3.162,1.579)--(3.172,1.576)--(3.183,1.573)--(3.193,1.580)%
  --(3.204,1.577)--(3.214,1.574)--(3.224,1.571)--(3.235,1.568)--(3.245,1.565)--(3.256,1.562)%
  --(3.266,1.569)--(3.277,1.566)--(3.287,1.563)--(3.298,1.560)--(3.308,1.557)--(3.319,1.554)%
  --(3.329,1.562)--(3.340,1.559)--(3.350,1.556)--(3.361,1.553)--(3.371,1.550)--(3.382,1.547)%
  --(3.392,1.544)--(3.403,1.552)--(3.413,1.549)--(3.424,1.546)--(3.434,1.543)--(3.444,1.540)%
  --(3.455,1.537)--(3.465,1.545)--(3.476,1.543)--(3.486,1.540)--(3.497,1.537)--(3.507,1.534)%
  --(3.518,1.531)--(3.528,1.528)--(3.539,1.536)--(3.549,1.534)--(3.560,1.531)--(3.570,1.528)%
  --(3.581,1.525)--(3.591,1.522)--(3.602,1.531)--(3.612,1.528)--(3.623,1.525)--(3.633,1.522)%
  --(3.643,1.520)--(3.654,1.517)--(3.664,1.526)--(3.675,1.523)--(3.685,1.520)--(3.696,1.518)%
  --(3.706,1.515)--(3.717,1.512)--(3.727,1.509)--(3.738,1.519)--(3.748,1.516)--(3.759,1.513)%
  --(3.769,1.510)--(3.780,1.508)--(3.790,1.505)--(3.801,1.515)--(3.811,1.512)--(3.822,1.510)%
  --(3.832,1.507)--(3.843,1.504)--(3.853,1.501)--(3.863,1.512)--(3.874,1.509)--(3.884,1.507)%
  --(3.895,1.504)--(3.905,1.501)--(3.916,1.499)--(3.926,1.509)--(3.937,1.507)--(3.947,1.504)%
  --(3.958,1.502)--(3.968,1.499)--(3.979,1.496)--(3.989,1.507)--(4.000,1.505)--(4.010,1.502)%
  --(4.021,1.500)--(4.031,1.497)--(4.042,1.495)--(4.052,1.506)--(4.062,1.504)--(4.073,1.502)%
  --(4.083,1.499)--(4.094,1.497)--(4.104,1.494)--(4.115,1.492)--(4.125,1.504)--(4.136,1.501)%
  --(4.146,1.499)--(4.157,1.497)--(4.167,1.494)--(4.178,1.492)--(4.188,1.504)--(4.199,1.502)%
  --(4.209,1.500)--(4.220,1.497)--(4.230,1.495)--(4.241,1.493)--(4.251,1.506)--(4.262,1.503)%
  --(4.272,1.501)--(4.282,1.499)--(4.293,1.497)--(4.303,1.495)--(4.314,1.508)--(4.324,1.506)%
  --(4.335,1.504)--(4.345,1.502)--(4.356,1.500)--(4.366,1.497)--(4.377,1.511)--(4.387,1.509)%
  --(4.398,1.507)--(4.408,1.505)--(4.419,1.503)--(4.429,1.501)--(4.440,1.515)--(4.450,1.513)%
  --(4.461,1.511)--(4.471,1.509)--(4.481,1.507)--(4.492,1.506)--(4.502,1.520)--(4.513,1.518)%
  --(4.523,1.516)--(4.534,1.515)--(4.544,1.513)--(4.555,1.511)--(4.565,1.526)--(4.576,1.524)%
  --(4.586,1.522)--(4.597,1.521)--(4.607,1.519)--(4.618,1.518)--(4.628,1.533)--(4.639,1.531)%
  --(4.649,1.530)--(4.660,1.528)--(4.670,1.527)--(4.680,1.525)--(4.691,1.541)--(4.701,1.539)%
  --(4.712,1.538)--(4.722,1.536)--(4.733,1.535)--(4.743,1.534)--(4.754,1.550)--(4.764,1.548)%
  --(4.775,1.547)--(4.785,1.546)--(4.796,1.545)--(4.806,1.544)--(4.817,1.542)--(4.827,1.559)%
  --(4.838,1.558)--(4.848,1.557)--(4.859,1.555)--(4.869,1.554)--(4.880,1.553)--(4.890,1.570)%
  --(4.900,1.569)--(4.911,1.568)--(4.921,1.567)--(4.932,1.566)--(4.942,1.566)--(4.953,1.583)%
  --(4.963,1.582)--(4.974,1.581)--(4.984,1.580)--(4.995,1.580)--(5.005,1.579)--(5.016,1.596)%
  --(5.026,1.596)--(5.037,1.595)--(5.047,1.595)--(5.058,1.594)--(5.068,1.594)--(5.079,1.593)%
  --(5.089,1.611)--(5.099,1.611)--(5.110,1.610)--(5.120,1.610)--(5.131,1.609)--(5.141,1.609)%
  --(5.152,1.627)--(5.162,1.627)--(5.173,1.627)--(5.183,1.627)--(5.194,1.627)--(5.204,1.626)%
  --(5.215,1.626)--(5.225,1.645)--(5.236,1.645)--(5.246,1.645)--(5.257,1.645)--(5.267,1.645)%
  --(5.278,1.645)--(5.288,1.664)--(5.299,1.665)--(5.309,1.665)--(5.319,1.665)--(5.330,1.665)%
  --(5.340,1.665)--(5.351,1.666)--(5.361,1.685)--(5.372,1.686)--(5.382,1.686)--(5.393,1.687)%
  --(5.403,1.687)--(5.414,1.687)--(5.424,1.708)--(5.435,1.708)--(5.445,1.709)--(5.456,1.709)%
  --(5.466,1.710)--(5.477,1.711)--(5.487,1.711)--(5.498,1.732)--(5.508,1.733)--(5.518,1.734)%
  --(5.529,1.735)--(5.539,1.735)--(5.550,1.736)--(5.560,1.737)--(5.571,1.759)--(5.581,1.760)%
  --(5.592,1.761)--(5.602,1.762)--(5.613,1.763)--(5.623,1.764)--(5.634,1.765)--(5.644,1.787)%
  --(5.655,1.788)--(5.665,1.789)--(5.676,1.791)--(5.686,1.792)--(5.697,1.793)--(5.707,1.795)%
  --(5.718,1.817)--(5.728,1.819)--(5.738,1.820)--(5.749,1.822)--(5.759,1.823)--(5.770,1.825)%
  --(5.780,1.827)--(5.791,1.850)--(5.801,1.851)--(5.812,1.853)--(5.822,1.855)--(5.833,1.857)%
  --(5.843,1.864)--(5.854,1.866)--(5.864,1.868)--(5.875,1.881)--(5.885,1.883)--(5.896,1.885)%
  --(5.906,1.887)--(5.917,1.900)--(5.927,1.902)--(5.937,1.905)--(5.948,1.907)--(5.958,1.920)%
  --(5.969,1.923)--(5.979,1.925)--(5.990,1.938)--(6.000,1.941)--(6.011,1.944)--(6.021,1.946)%
  --(6.032,1.960)--(6.042,1.962)--(6.053,1.965)--(6.063,1.968)--(6.074,1.971)--(6.084,1.985)%
  --(6.095,1.987)--(6.105,1.990)--(6.116,1.993)--(6.126,2.008)--(6.136,2.011)--(6.147,2.014)%
  --(6.157,2.017)--(6.168,2.031)--(6.178,2.034)--(6.189,2.038)--(6.199,2.041)--(6.210,2.056)%
  --(6.220,2.059)--(6.231,2.062)--(6.241,2.066)--(6.252,2.069)--(6.262,2.084)--(6.273,2.088)%
  --(6.283,2.092)--(6.294,2.095)--(6.304,2.111)--(6.315,2.114)--(6.325,2.118)--(6.336,2.122)%
  --(6.346,2.126)--(6.356,2.142)--(6.367,2.146)--(6.377,2.150)--(6.388,2.154)--(6.398,2.170)%
  --(6.409,2.174)--(6.419,2.178)--(6.430,2.182)--(6.440,2.187)--(6.451,2.203)--(6.461,2.207)%
  --(6.472,2.212)--(6.482,2.216)--(6.493,2.221)--(6.503,2.237)--(6.514,2.242)--(6.524,2.247)%
  --(6.535,2.251)--(6.545,2.256)--(6.555,2.261)--(6.566,2.278)--(6.576,2.283)--(6.587,2.288)%
  --(6.597,2.293)--(6.608,2.298)--(6.618,2.315)--(6.629,2.320)--(6.639,2.326)--(6.650,2.331)%
  --(6.660,2.336)--(6.671,2.342)--(6.681,2.359)--(6.692,2.365)--(6.702,2.370)--(6.713,2.376)%
  --(6.723,2.381)--(6.734,2.387)--(6.744,2.405)--(6.755,2.411)--(6.765,2.417)--(6.775,2.423)%
  --(6.786,2.429)--(6.796,2.435)--(6.807,2.453)--(6.817,2.459)--(6.828,2.465)--(6.838,2.471)%
  --(6.849,2.478)--(6.859,2.484)--(6.870,2.490)--(6.880,2.509)--(6.891,2.516)--(6.901,2.522)%
  --(6.912,2.529)--(6.922,2.536)--(6.933,2.542)--(6.943,2.549)--(6.954,2.556)--(6.964,2.575)%
  --(6.974,2.582)--(6.985,2.589)--(6.995,2.596)--(7.006,2.603)--(7.016,2.610)--(7.027,2.617)%
  --(7.037,2.625)--(7.048,2.632)--(7.058,2.652)--(7.069,2.660)--(7.079,2.667)--(7.090,2.674)%
  --(7.100,2.682)--(7.111,2.690)--(7.121,2.697)--(7.132,2.705)--(7.142,2.713)--(7.153,2.721)%
  --(7.163,2.742)--(7.174,2.750)--(7.184,2.758)--(7.194,2.766)--(7.205,2.774)--(7.215,2.782)%
  --(7.226,2.790)--(7.236,2.799)--(7.247,2.807)--(7.257,2.815)--(7.268,2.824)--(7.278,2.832)%
  --(7.289,2.841)--(7.299,2.850)--(7.310,2.872)--(7.320,2.880)--(7.331,2.889)--(7.341,2.898)%
  --(7.352,2.907)--(7.362,2.916)--(7.373,2.925)--(7.383,2.935)--(7.393,2.944)--(7.404,2.953)%
  --(7.414,2.962)--(7.425,2.972)--(7.435,2.981)--(7.446,2.991)--(7.456,3.000)--(7.467,3.010)%
  --(7.477,3.020)--(7.488,3.030)--(7.498,3.039)--(7.509,3.049)--(7.519,3.059)--(7.530,3.069)%
  --(7.540,3.079)--(7.551,3.090)--(7.561,3.100)--(7.572,3.110)--(7.582,3.120)--(7.592,3.131)%
  --(7.603,3.141)--(7.613,3.152)--(7.624,3.162)--(7.634,3.173)--(7.645,3.184)--(7.655,3.195)%
  --(7.666,3.206)--(7.676,3.217)--(7.687,3.228)--(7.697,3.239)--(7.708,3.250)--(7.718,3.261)%
  --(7.729,3.272)--(7.739,3.284)--(7.750,3.295)--(7.760,3.307)--(7.771,3.318)--(7.781,3.330)%
  --(7.792,3.341)--(7.802,3.339)--(7.812,3.351)--(7.823,3.363)--(7.833,3.375)--(7.844,3.387)%
  --(7.854,3.399)--(7.865,3.411)--(7.875,3.424)--(7.886,3.436)--(7.896,3.448)--(7.907,3.461)%
  --(7.917,3.473)--(7.928,3.486)--(7.938,3.485)--(7.949,3.497)--(7.959,3.510)--(7.970,3.523)%
  --(7.980,3.536)--(7.991,3.549)--(8.001,3.562)--(8.011,3.575)--(8.022,3.575)--(8.032,3.588)%
  --(8.043,3.601)--(8.053,3.615)--(8.064,3.628)--(8.074,3.642)--(8.085,3.656)--(8.095,3.669)%
  --(8.106,3.669)--(8.116,3.683)--(8.127,3.697)--(8.137,3.711)--(8.148,3.725)--(8.158,3.739)%
  --(8.169,3.739)--(8.179,3.754)--(8.190,3.768)--(8.200,3.783)--(8.211,3.797)--(8.221,3.798)%
  --(8.231,3.813)--(8.242,3.827)--(8.252,3.842)--(8.263,3.857)--(8.273,3.858)--(8.284,3.873)%
  --(8.294,3.889)--(8.305,3.904)--(8.315,3.919)--(8.326,3.920)--(8.336,3.936)--(8.347,3.951)%
  --(8.357,3.967)--(8.368,3.969)--(8.378,3.984)--(8.389,4.000)--(8.399,4.016)--(8.410,4.018)%
  --(8.420,4.034)--(8.430,4.051)--(8.441,4.067)--(8.451,4.069)--(8.462,4.085)--(8.472,4.102)%
  --(8.483,4.119)--(8.493,4.121)--(8.504,4.138)--(8.514,4.155)--(8.525,4.157)--(8.535,4.175)%
  --(8.546,4.192)--(8.556,4.194)--(8.567,4.212)--(8.577,4.229)--(8.588,4.232)--(8.598,4.250)%
  --(8.609,4.268)--(8.619,4.271)--(8.630,4.289)--(8.640,4.307)--(8.650,4.310)--(8.661,4.328)%
  --(8.671,4.347)--(8.682,4.350)--(8.692,4.369)--(8.703,4.387)--(8.713,4.391)--(8.724,4.410)%
  --(8.734,4.414)--(8.745,4.433)--(8.755,4.452)--(8.766,4.456)--(8.776,4.475)--(8.787,4.480)%
  --(8.797,4.499)--(8.808,4.519)--(8.818,4.523)--(8.829,4.543)--(8.839,4.548)--(8.849,4.568)%
  --(8.860,4.573)--(8.870,4.593)--(8.881,4.613)--(8.891,4.619)--(8.902,4.639)--(8.912,4.644)%
  --(8.923,4.665)--(8.933,4.671)--(8.944,4.691)--(8.954,4.697)--(8.965,4.718)--(8.975,4.724)%
  --(8.986,4.745)--(8.996,4.752)--(9.007,4.773)--(9.017,4.780)--(9.028,4.801)--(9.038,4.808)%
  --(9.048,4.815)--(9.059,4.836)--(9.069,4.843)--(9.080,4.865)--(9.090,4.872)--(9.101,4.895)%
  --(9.111,4.902)--(9.122,4.924)--(9.132,4.932)--(9.143,4.940)--(9.153,4.962)--(9.164,4.970)%
  --(9.174,4.993)--(9.185,5.001)--(9.195,5.010)--(9.206,5.033)--(9.216,5.041)--(9.227,5.050)%
  --(9.237,5.073)--(9.248,5.082)--(9.258,5.091)--(9.268,5.115)--(9.279,5.124)--(9.289,5.133)%
  --(9.300,5.157)--(9.310,5.167)--(9.321,5.176)--(9.331,5.201)--(9.342,5.210)--(9.352,5.220)%
  --(9.363,5.230)--(9.373,5.255)--(9.384,5.265)--(9.394,5.276)--(9.405,5.286)--(9.415,5.312)%
  --(9.426,5.322)--(9.436,5.333)--(9.447,5.344)--(9.457,5.370)--(9.467,5.381)--(9.478,5.392)%
  --(9.488,5.403)--(9.499,5.415)--(9.509,5.441)--(9.520,5.453)--(9.530,5.465)--(9.541,5.477)%
  --(9.551,5.489)--(9.562,5.501)--(9.572,5.513)--(9.583,5.541)--(9.593,5.553)--(9.604,5.566)%
  --(9.614,5.579)--(9.625,5.592)--(9.635,5.605)--(9.646,5.618)--(9.656,5.631)--(9.667,5.645)%
  --(9.677,5.659)--(9.687,5.672)--(9.698,5.686)--(9.708,5.700)--(9.719,5.714)--(9.729,5.729)%
  --(9.740,5.743)--(9.750,5.758)--(9.761,5.772)--(9.771,5.787)--(9.782,5.802)--(9.792,5.817)%
  --(9.803,5.832)--(9.813,5.848)--(9.824,5.863)--(9.834,5.879)--(9.845,5.895)--(9.855,5.911)%
  --(9.866,5.912)--(9.876,5.928)--(9.886,5.945)--(9.897,5.961)--(9.907,5.978)--(9.918,5.995)%
  --(9.928,6.012)--(9.939,6.014)--(9.949,6.032)--(9.960,6.049)--(9.970,6.067)--(9.981,6.070)%
  --(9.991,6.088)--(10.002,6.106)--(10.012,6.124)--(10.023,6.128)--(10.033,6.146)--(10.044,6.165)%
  --(10.054,6.184)--(10.065,6.188)--(10.075,6.208)--(10.086,6.227)--(10.096,6.232)--(10.106,6.252)%
  --(10.117,6.257)--(10.127,6.277)--(10.138,6.297)--(10.148,6.303)--(10.159,6.324)--(10.169,6.330)%
  --(10.180,6.351)--(10.190,6.358)--(10.201,6.379)--(10.211,6.386)--(10.222,6.407)--(10.232,6.415)%
  --(10.243,6.437)--(10.253,6.445)--(10.264,6.467)--(10.274,6.475)--(10.285,6.498)--(10.295,6.506)%
  --(10.305,6.515)--(10.316,6.538)--(10.326,6.548)--(10.337,6.571)--(10.347,6.581)--(10.358,6.591)%
  --(10.368,6.601)--(10.379,6.625)--(10.389,6.636)--(10.400,6.647)--(10.410,6.657)--(10.421,6.683)%
  --(10.431,6.694)--(10.442,6.706)--(10.452,6.718)--(10.463,6.730)--(10.473,6.756)--(10.484,6.768)%
  --(10.494,6.781)--(10.504,6.794)--(10.515,6.807)--(10.525,6.821)--(10.536,6.835)--(10.546,6.849)%
  --(10.557,6.863)--(10.567,6.877)--(10.578,6.892)--(10.588,6.907)--(10.599,6.922)--(10.609,6.934)%
  --(10.620,6.950)--(10.630,6.959)--(10.641,6.975)--(10.651,6.992)--(10.662,7.002)--(10.672,7.019)%
  --(10.683,7.029)--(10.693,7.047)--(10.704,7.058)--(10.714,7.070)--(10.724,7.088)--(10.735,7.100)%
  --(10.745,7.112)--(10.756,7.125)--(10.766,7.145)--(10.777,7.158)--(10.787,7.171)--(10.798,7.185)%
  --(10.808,7.199)--(10.819,7.207)--(10.829,7.222)--(10.840,7.237)--(10.850,7.253)--(10.861,7.262)%
  --(10.871,7.279)--(10.882,7.295)--(10.892,7.305)--(10.903,7.323)--(10.913,7.334)--(10.923,7.345)%
  --(10.934,7.364)--(10.944,7.376)--(10.955,7.389)--(10.965,7.402)--(10.976,7.415)--(10.986,7.429)%
  --(10.997,7.443)--(11.007,7.458)--(11.018,7.467)--(11.028,7.482)--(11.039,7.498)--(11.049,7.508)%
  --(11.060,7.519)--(11.070,7.536)--(11.081,7.547)--(11.091,7.560)--(11.102,7.572)--(11.112,7.591)%
  --(11.123,7.598)--(11.133,7.612)--(11.143,7.627)--(11.154,7.642)--(11.164,7.651)--(11.175,7.667)%
  --(11.185,7.678)--(11.196,7.695)--(11.206,7.706)--(11.217,7.719)--(11.227,7.732)--(11.238,7.745)%
  --(11.248,7.759)--(11.259,7.768)--(11.269,7.783)--(11.280,7.794)--(11.290,7.810)--(11.301,7.822)%
  --(11.311,7.834)--(11.322,7.847)--(11.332,7.860)--(11.342,7.875)--(11.353,7.885)--(11.363,7.895)%
  --(11.374,7.911)--(11.384,7.923)--(11.395,7.936)--(11.405,7.950)--(11.416,7.959)--(11.426,7.974)%
  --(11.437,7.985)--(11.447,7.997)--(11.458,8.010)--(11.468,8.019)--(11.479,8.034)--(11.489,8.045)%
  --(11.500,8.057)--(11.510,8.069)--(11.521,8.083)--(11.531,8.094)--(11.542,8.105)--(11.552,8.118)%
  --(11.562,8.132)--(11.573,8.143)--(11.583,8.155)--(11.594,8.168)--(11.604,8.178)--(11.615,8.190)%
  --(11.625,8.203)--(11.636,8.214)--(11.646,8.226)--(11.657,8.236)--(11.667,8.248)--(11.678,8.261)%
  --(11.688,8.273)--(11.699,8.283)--(11.709,8.295)--(11.720,8.305)--(11.730,8.318)--(11.741,8.330)%
  --(11.751,8.341)--(11.761,8.351)--(11.772,8.361)--(11.782,8.375)--(11.793,8.385)--(11.803,8.396)%
  --(11.814,8.407)--(11.824,8.418)--(11.835,8.429)--(11.845,8.440)--(11.856,8.450)--(11.866,8.462)%
  --(11.877,8.473)--(11.887,8.483)--(11.898,8.494)--(11.908,8.504)--(11.919,8.515);
\gpcolor{color=gp lt color border}
\draw[gp path] (1.136,8.631)--(1.136,0.985)--(13.447,0.985)--(13.447,8.631)--cycle;
%% coordinates of the plot area
\gpdefrectangularnode{gp plot 1}{\pgfpoint{1.136cm}{0.985cm}}{\pgfpoint{13.447cm}{8.631cm}}
\end{tikzpicture}
%% gnuplot variables

	\caption{Gráfico da pressão obtido através da Equação~\eqref{Eq:Pressao}. Devido ao fato de que estamos usando $\varepsilon_o = 0$ por enquanto, a escala vertical do gráfico está deslocada no sentido positivo. \protect[Parameters: eNJL1; Proton fraction: 1/2] }
	\label{Fig:pressure_graph}
\end{figure*}

\begin{figure*}
	\begin{tikzpicture}[gnuplot]
%% generated with GNUPLOT 5.0p2 (Lua 5.2; terminal rev. 99, script rev. 100)
%% Fri Mar  4 16:19:10 2016
\path (0.000,0.000) rectangle (14.000,9.000);
\gpcolor{color=gp lt color border}
\gpsetlinetype{gp lt border}
\gpsetdashtype{gp dt solid}
\gpsetlinewidth{1.00}
\draw[gp path] (1.320,0.985)--(1.500,0.985);
\draw[gp path] (13.447,0.985)--(13.267,0.985);
\node[gp node right] at (1.136,0.985) {$922$};
\draw[gp path] (1.320,1.750)--(1.500,1.750);
\draw[gp path] (13.447,1.750)--(13.267,1.750);
\node[gp node right] at (1.136,1.750) {$924$};
\draw[gp path] (1.320,2.514)--(1.500,2.514);
\draw[gp path] (13.447,2.514)--(13.267,2.514);
\node[gp node right] at (1.136,2.514) {$926$};
\draw[gp path] (1.320,3.279)--(1.500,3.279);
\draw[gp path] (13.447,3.279)--(13.267,3.279);
\node[gp node right] at (1.136,3.279) {$928$};
\draw[gp path] (1.320,4.043)--(1.500,4.043);
\draw[gp path] (13.447,4.043)--(13.267,4.043);
\node[gp node right] at (1.136,4.043) {$930$};
\draw[gp path] (1.320,4.808)--(1.500,4.808);
\draw[gp path] (13.447,4.808)--(13.267,4.808);
\node[gp node right] at (1.136,4.808) {$932$};
\draw[gp path] (1.320,5.573)--(1.500,5.573);
\draw[gp path] (13.447,5.573)--(13.267,5.573);
\node[gp node right] at (1.136,5.573) {$934$};
\draw[gp path] (1.320,6.337)--(1.500,6.337);
\draw[gp path] (13.447,6.337)--(13.267,6.337);
\node[gp node right] at (1.136,6.337) {$936$};
\draw[gp path] (1.320,7.102)--(1.500,7.102);
\draw[gp path] (13.447,7.102)--(13.267,7.102);
\node[gp node right] at (1.136,7.102) {$938$};
\draw[gp path] (1.320,7.866)--(1.500,7.866);
\draw[gp path] (13.447,7.866)--(13.267,7.866);
\node[gp node right] at (1.136,7.866) {$940$};
\draw[gp path] (1.320,8.631)--(1.500,8.631);
\draw[gp path] (13.447,8.631)--(13.267,8.631);
\node[gp node right] at (1.136,8.631) {$942$};
\draw[gp path] (1.320,0.985)--(1.320,1.165);
\draw[gp path] (1.320,8.631)--(1.320,8.451);
\node[gp node center] at (1.320,0.677) {$0$};
\draw[gp path] (2.836,0.985)--(2.836,1.165);
\draw[gp path] (2.836,8.631)--(2.836,8.451);
\node[gp node center] at (2.836,0.677) {$0.05$};
\draw[gp path] (4.352,0.985)--(4.352,1.165);
\draw[gp path] (4.352,8.631)--(4.352,8.451);
\node[gp node center] at (4.352,0.677) {$0.1$};
\draw[gp path] (5.868,0.985)--(5.868,1.165);
\draw[gp path] (5.868,8.631)--(5.868,8.451);
\node[gp node center] at (5.868,0.677) {$0.15$};
\draw[gp path] (7.384,0.985)--(7.384,1.165);
\draw[gp path] (7.384,8.631)--(7.384,8.451);
\node[gp node center] at (7.384,0.677) {$0.2$};
\draw[gp path] (8.899,0.985)--(8.899,1.165);
\draw[gp path] (8.899,8.631)--(8.899,8.451);
\node[gp node center] at (8.899,0.677) {$0.25$};
\draw[gp path] (10.415,0.985)--(10.415,1.165);
\draw[gp path] (10.415,8.631)--(10.415,8.451);
\node[gp node center] at (10.415,0.677) {$0.3$};
\draw[gp path] (11.931,0.985)--(11.931,1.165);
\draw[gp path] (11.931,8.631)--(11.931,8.451);
\node[gp node center] at (11.931,0.677) {$0.35$};
\draw[gp path] (13.447,0.985)--(13.447,1.165);
\draw[gp path] (13.447,8.631)--(13.447,8.451);
\node[gp node center] at (13.447,0.677) {$0.4$};
\draw[gp path] (1.320,8.631)--(1.320,0.985)--(13.447,0.985)--(13.447,8.631)--cycle;
\node[gp node center,rotate=-270] at (0.246,4.808) {$E/A = \varepsilon/\rho$ (MeV)};
\node[gp node center] at (7.383,0.215) {$\rho$ $\rm{fm}^{-3}$};
\gpcolor{rgb color={0.580,0.000,0.827}}
\draw[gp path] (1.633,7.149)--(1.644,7.125)--(1.654,7.101)--(1.664,7.077)--(1.675,7.053)%
  --(1.685,7.028)--(1.695,7.004)--(1.706,6.979)--(1.716,6.955)--(1.726,6.930)--(1.737,6.905)%
  --(1.747,6.880)--(1.757,6.855)--(1.768,6.829)--(1.778,6.804)--(1.788,6.779)--(1.799,6.753)%
  --(1.809,6.728)--(1.819,6.702)--(1.830,6.677)--(1.840,6.651)--(1.850,6.626)--(1.860,6.600)%
  --(1.871,6.574)--(1.881,6.549)--(1.891,6.523)--(1.902,6.497)--(1.912,6.472)--(1.922,6.446)%
  --(1.933,6.420)--(1.943,6.394)--(1.953,6.368)--(1.964,6.343)--(1.974,6.317)--(1.984,6.291)%
  --(1.995,6.265)--(2.005,6.240)--(2.015,6.214)--(2.026,6.188)--(2.036,6.163)--(2.046,6.137)%
  --(2.057,6.111)--(2.067,6.086)--(2.077,6.060)--(2.087,6.034)--(2.098,6.009)--(2.108,5.983)%
  --(2.118,5.958)--(2.129,5.932)--(2.139,5.907)--(2.149,5.881)--(2.160,5.856)--(2.170,5.831)%
  --(2.180,5.805)--(2.191,5.780)--(2.201,5.755)--(2.211,5.730)--(2.222,5.705)--(2.232,5.679)%
  --(2.242,5.654)--(2.253,5.629)--(2.263,5.604)--(2.273,5.579)--(2.284,5.555)--(2.294,5.530)%
  --(2.304,5.505)--(2.314,5.480)--(2.325,5.456)--(2.335,5.431)--(2.345,5.406)--(2.356,5.382)%
  --(2.366,5.357)--(2.376,5.333)--(2.387,5.309)--(2.397,5.284)--(2.407,5.260)--(2.418,5.236)%
  --(2.428,5.212)--(2.438,5.188)--(2.449,5.164)--(2.459,5.140)--(2.469,5.116)--(2.480,5.092)%
  --(2.490,5.068)--(2.500,5.044)--(2.511,5.021)--(2.521,4.997)--(2.531,4.974)--(2.542,4.950)%
  --(2.552,4.927)--(2.562,4.903)--(2.572,4.880)--(2.583,4.857)--(2.593,4.834)--(2.603,4.811)%
  --(2.614,4.788)--(2.624,4.765)--(2.634,4.742)--(2.645,4.719)--(2.655,4.696)--(2.665,4.674)%
  --(2.676,4.651)--(2.686,4.629)--(2.696,4.606)--(2.707,4.584)--(2.717,4.561)--(2.727,4.539)%
  --(2.738,4.517)--(2.748,4.495)--(2.758,4.473)--(2.769,4.451)--(2.779,4.429)--(2.789,4.407)%
  --(2.799,4.385)--(2.810,4.364)--(2.820,4.342)--(2.830,4.320)--(2.841,4.299)--(2.851,4.277)%
  --(2.861,4.256)--(2.872,4.235)--(2.882,4.214)--(2.892,4.193)--(2.903,4.171)--(2.913,4.150)%
  --(2.923,4.130)--(2.934,4.109)--(2.944,4.088)--(2.954,4.067)--(2.965,4.047)--(2.975,4.026)%
  --(2.985,4.006)--(2.996,3.985)--(3.006,3.965)--(3.016,3.945)--(3.026,3.925)--(3.037,3.905)%
  --(3.047,3.885)--(3.057,3.865)--(3.068,3.845)--(3.078,3.825)--(3.088,3.805)--(3.099,3.786)%
  --(3.109,3.766)--(3.119,3.747)--(3.130,3.727)--(3.140,3.708)--(3.150,3.689)--(3.161,3.669)%
  --(3.171,3.650)--(3.181,3.631)--(3.192,3.612)--(3.202,3.594)--(3.212,3.575)--(3.223,3.556)%
  --(3.233,3.537)--(3.243,3.519)--(3.253,3.500)--(3.264,3.482)--(3.274,3.464)--(3.284,3.445)%
  --(3.295,3.427)--(3.305,3.409)--(3.315,3.391)--(3.326,3.373)--(3.336,3.355)--(3.346,3.337)%
  --(3.357,3.320)--(3.367,3.302)--(3.377,3.284)--(3.388,3.267)--(3.398,3.250)--(3.408,3.232)%
  --(3.419,3.215)--(3.429,3.198)--(3.439,3.181)--(3.450,3.164)--(3.460,3.147)--(3.470,3.130)%
  --(3.480,3.113)--(3.491,3.096)--(3.501,3.080)--(3.511,3.063)--(3.522,3.047)--(3.532,3.030)%
  --(3.542,3.014)--(3.553,2.998)--(3.563,2.982)--(3.573,2.966)--(3.584,2.950)--(3.594,2.934)%
  --(3.604,2.918)--(3.615,2.902)--(3.625,2.886)--(3.635,2.871)--(3.646,2.855)--(3.656,2.840)%
  --(3.666,2.824)--(3.677,2.809)--(3.687,2.794)--(3.697,2.779)--(3.707,2.764)--(3.718,2.749)%
  --(3.728,2.734)--(3.738,2.719)--(3.749,2.704)--(3.759,2.689)--(3.769,2.675)--(3.780,2.660)%
  --(3.790,2.646)--(3.800,2.631)--(3.811,2.617)--(3.821,2.603)--(3.831,2.589)--(3.842,2.575)%
  --(3.852,2.561)--(3.862,2.547)--(3.873,2.533)--(3.883,2.519)--(3.893,2.505)--(3.904,2.492)%
  --(3.914,2.478)--(3.924,2.465)--(3.934,2.451)--(3.945,2.438)--(3.955,2.425)--(3.965,2.412)%
  --(3.976,2.399)--(3.986,2.386)--(3.996,2.373)--(4.007,2.360)--(4.017,2.347)--(4.027,2.335)%
  --(4.038,2.322)--(4.048,2.309)--(4.058,2.297)--(4.069,2.285)--(4.079,2.272)--(4.089,2.260)%
  --(4.100,2.248)--(4.110,2.236)--(4.120,2.224)--(4.131,2.212)--(4.141,2.200)--(4.151,2.188)%
  --(4.161,2.177)--(4.172,2.165)--(4.182,2.153)--(4.192,2.142)--(4.203,2.131)--(4.213,2.119)%
  --(4.223,2.108)--(4.234,2.097)--(4.244,2.086)--(4.254,2.075)--(4.265,2.064)--(4.275,2.053)%
  --(4.285,2.042)--(4.296,2.031)--(4.306,2.021)--(4.316,2.010)--(4.327,2.000)--(4.337,1.989)%
  --(4.347,1.979)--(4.358,1.969)--(4.368,1.959)--(4.378,1.948)--(4.388,1.938)--(4.399,1.928)%
  --(4.409,1.919)--(4.419,1.909)--(4.430,1.899)--(4.440,1.889)--(4.450,1.880)--(4.461,1.870)%
  --(4.471,1.861)--(4.481,1.851)--(4.492,1.842)--(4.502,1.833)--(4.512,1.824)--(4.523,1.815)%
  --(4.533,1.806)--(4.543,1.797)--(4.554,1.788)--(4.564,1.779)--(4.574,1.770)--(4.585,1.762)%
  --(4.595,1.753)--(4.605,1.745)--(4.615,1.736)--(4.626,1.728)--(4.636,1.720)--(4.646,1.712)%
  --(4.657,1.703)--(4.667,1.695)--(4.677,1.687)--(4.688,1.680)--(4.698,1.672)--(4.708,1.664)%
  --(4.719,1.656)--(4.729,1.649)--(4.739,1.641)--(4.750,1.634)--(4.760,1.626)--(4.770,1.619)%
  --(4.781,1.612)--(4.791,1.604)--(4.801,1.597)--(4.812,1.590)--(4.822,1.583)--(4.832,1.576)%
  --(4.842,1.570)--(4.853,1.563)--(4.863,1.556)--(4.873,1.550)--(4.884,1.543)--(4.894,1.537)%
  --(4.904,1.530)--(4.915,1.524)--(4.925,1.518)--(4.935,1.511)--(4.946,1.505)--(4.956,1.499)%
  --(4.966,1.493)--(4.977,1.487)--(4.987,1.482)--(4.997,1.476)--(5.008,1.470)--(5.018,1.464)%
  --(5.028,1.459)--(5.039,1.453)--(5.049,1.448)--(5.059,1.443)--(5.069,1.437)--(5.080,1.432)%
  --(5.090,1.427)--(5.100,1.422)--(5.111,1.417)--(5.121,1.412)--(5.131,1.407)--(5.142,1.402)%
  --(5.152,1.398)--(5.162,1.393)--(5.173,1.388)--(5.183,1.384)--(5.193,1.379)--(5.204,1.375)%
  --(5.214,1.371)--(5.224,1.366)--(5.235,1.362)--(5.245,1.358)--(5.255,1.354)--(5.266,1.350)%
  --(5.276,1.346)--(5.286,1.342)--(5.296,1.339)--(5.307,1.335)--(5.317,1.331)--(5.327,1.328)%
  --(5.338,1.324)--(5.348,1.321)--(5.358,1.317)--(5.369,1.314)--(5.379,1.311)--(5.389,1.308)%
  --(5.400,1.305)--(5.410,1.302)--(5.420,1.299)--(5.431,1.296)--(5.441,1.293)--(5.451,1.290)%
  --(5.462,1.287)--(5.472,1.285)--(5.482,1.282)--(5.493,1.280)--(5.503,1.277)--(5.513,1.275)%
  --(5.523,1.273)--(5.534,1.270)--(5.544,1.268)--(5.554,1.266)--(5.565,1.264)--(5.575,1.262)%
  --(5.585,1.260)--(5.596,1.258)--(5.606,1.256)--(5.616,1.255)--(5.627,1.253)--(5.637,1.251)%
  --(5.647,1.250)--(5.658,1.248)--(5.668,1.247)--(5.678,1.246)--(5.689,1.244)--(5.699,1.243)%
  --(5.709,1.242)--(5.720,1.241)--(5.730,1.240)--(5.740,1.239)--(5.750,1.238)--(5.761,1.237)%
  --(5.771,1.236)--(5.781,1.236)--(5.792,1.235)--(5.802,1.234)--(5.812,1.234)--(5.823,1.233)%
  --(5.833,1.233)--(5.843,1.233)--(5.854,1.232)--(5.864,1.232)--(5.874,1.232)--(5.885,1.232)%
  --(5.895,1.232)--(5.905,1.232)--(5.916,1.232)--(5.926,1.232)--(5.936,1.232)--(5.947,1.233)%
  --(5.957,1.233)--(5.967,1.233)--(5.977,1.234)--(5.988,1.234)--(5.998,1.235)--(6.008,1.236)%
  --(6.019,1.236)--(6.029,1.237)--(6.039,1.238)--(6.050,1.239)--(6.060,1.240)--(6.070,1.241)%
  --(6.081,1.242)--(6.091,1.243)--(6.101,1.244)--(6.112,1.245)--(6.122,1.247)--(6.132,1.248)%
  --(6.143,1.249)--(6.153,1.251)--(6.163,1.252)--(6.174,1.254)--(6.184,1.256)--(6.194,1.257)%
  --(6.204,1.259)--(6.215,1.261)--(6.225,1.263)--(6.235,1.265)--(6.246,1.267)--(6.256,1.269)%
  --(6.266,1.271)--(6.277,1.273)--(6.287,1.276)--(6.297,1.278)--(6.308,1.280)--(6.318,1.283)%
  --(6.328,1.285)--(6.339,1.288)--(6.349,1.290)--(6.359,1.293)--(6.370,1.295)--(6.380,1.298)%
  --(6.390,1.301)--(6.401,1.304)--(6.411,1.307)--(6.421,1.310)--(6.431,1.313)--(6.442,1.316)%
  --(6.452,1.319)--(6.462,1.322)--(6.473,1.326)--(6.483,1.329)--(6.493,1.332)--(6.504,1.336)%
  --(6.514,1.339)--(6.524,1.343)--(6.535,1.346)--(6.545,1.350)--(6.555,1.354)--(6.566,1.357)%
  --(6.576,1.361)--(6.586,1.365)--(6.597,1.369)--(6.607,1.373)--(6.617,1.377)--(6.628,1.381)%
  --(6.638,1.385)--(6.648,1.389)--(6.658,1.394)--(6.669,1.398)--(6.679,1.402)--(6.689,1.407)%
  --(6.700,1.411)--(6.710,1.416)--(6.720,1.420)--(6.731,1.425)--(6.741,1.429)--(6.751,1.434)%
  --(6.762,1.439)--(6.772,1.444)--(6.782,1.449)--(6.793,1.454)--(6.803,1.459)--(6.813,1.464)%
  --(6.824,1.469)--(6.834,1.474)--(6.844,1.479)--(6.855,1.484)--(6.865,1.490)--(6.875,1.495)%
  --(6.885,1.500)--(6.896,1.506)--(6.906,1.511)--(6.916,1.517)--(6.927,1.522)--(6.937,1.528)%
  --(6.947,1.534)--(6.958,1.540)--(6.968,1.545)--(6.978,1.551)--(6.989,1.557)--(6.999,1.563)%
  --(7.009,1.569)--(7.020,1.575)--(7.030,1.581)--(7.040,1.587)--(7.051,1.594)--(7.061,1.600)%
  --(7.071,1.606)--(7.082,1.613)--(7.092,1.619)--(7.102,1.625)--(7.112,1.632)--(7.123,1.638)%
  --(7.133,1.645)--(7.143,1.652)--(7.154,1.658)--(7.164,1.665)--(7.174,1.672)--(7.185,1.679)%
  --(7.195,1.686)--(7.205,1.693)--(7.216,1.700)--(7.226,1.707)--(7.236,1.714)--(7.247,1.721)%
  --(7.257,1.728)--(7.267,1.735)--(7.278,1.742)--(7.288,1.750)--(7.298,1.757)--(7.309,1.765)%
  --(7.319,1.772)--(7.329,1.780)--(7.339,1.787)--(7.350,1.795)--(7.360,1.802)--(7.370,1.810)%
  --(7.381,1.818)--(7.391,1.826)--(7.401,1.833)--(7.412,1.841)--(7.422,1.849)--(7.432,1.857)%
  --(7.443,1.865)--(7.453,1.873)--(7.463,1.881)--(7.474,1.889)--(7.484,1.898)--(7.494,1.906)%
  --(7.505,1.914)--(7.515,1.922)--(7.525,1.931)--(7.536,1.939)--(7.546,1.948)--(7.556,1.956)%
  --(7.566,1.965)--(7.577,1.973)--(7.587,1.982)--(7.597,1.991)--(7.608,1.999)--(7.618,2.008)%
  --(7.628,2.017)--(7.639,2.026)--(7.649,2.035)--(7.659,2.044)--(7.670,2.053)--(7.680,2.062)%
  --(7.690,2.071)--(7.701,2.080)--(7.711,2.089)--(7.721,2.098)--(7.732,2.107)--(7.742,2.117)%
  --(7.752,2.126)--(7.763,2.135)--(7.773,2.145)--(7.783,2.154)--(7.793,2.164)--(7.804,2.173)%
  --(7.814,2.183)--(7.824,2.192)--(7.835,2.202)--(7.845,2.212)--(7.855,2.221)--(7.866,2.231)%
  --(7.876,2.241)--(7.886,2.251)--(7.897,2.261)--(7.907,2.271)--(7.917,2.281)--(7.928,2.291)%
  --(7.938,2.301)--(7.948,2.311)--(7.959,2.321)--(7.969,2.331)--(7.979,2.341)--(7.990,2.351)%
  --(8.000,2.362)--(8.010,2.372)--(8.021,2.382)--(8.031,2.393)--(8.041,2.403)--(8.051,2.414)%
  --(8.062,2.424)--(8.072,2.435)--(8.082,2.445)--(8.093,2.456)--(8.103,2.467)--(8.113,2.477)%
  --(8.124,2.488)--(8.134,2.499)--(8.144,2.510)--(8.155,2.521)--(8.165,2.532)--(8.175,2.542)%
  --(8.186,2.553)--(8.196,2.564)--(8.206,2.575)--(8.217,2.587)--(8.227,2.598)--(8.237,2.609)%
  --(8.248,2.620)--(8.258,2.631)--(8.268,2.643)--(8.278,2.654)--(8.289,2.665)--(8.299,2.677)%
  --(8.309,2.688)--(8.320,2.699)--(8.330,2.711)--(8.340,2.722)--(8.351,2.734)--(8.361,2.745)%
  --(8.371,2.757)--(8.382,2.769)--(8.392,2.780)--(8.402,2.792)--(8.413,2.804)--(8.423,2.816)%
  --(8.433,2.828)--(8.444,2.839)--(8.454,2.851)--(8.464,2.863)--(8.475,2.875)--(8.485,2.887)%
  --(8.495,2.899)--(8.505,2.911)--(8.516,2.923)--(8.526,2.935)--(8.536,2.948)--(8.547,2.960)%
  --(8.557,2.972)--(8.567,2.984)--(8.578,2.997)--(8.588,3.009)--(8.598,3.021)--(8.609,3.034)%
  --(8.619,3.046)--(8.629,3.059)--(8.640,3.071)--(8.650,3.084)--(8.660,3.096)--(8.671,3.109)%
  --(8.681,3.121)--(8.691,3.134)--(8.702,3.147)--(8.712,3.159)--(8.722,3.172)--(8.732,3.185)%
  --(8.743,3.198)--(8.753,3.211)--(8.763,3.223)--(8.774,3.236)--(8.784,3.249)--(8.794,3.262)%
  --(8.805,3.275)--(8.815,3.288)--(8.825,3.301)--(8.836,3.314)--(8.846,3.327)--(8.856,3.340)%
  --(8.867,3.354)--(8.877,3.367)--(8.887,3.380)--(8.898,3.393)--(8.908,3.407)--(8.918,3.420)%
  --(8.929,3.433)--(8.939,3.447)--(8.949,3.460)--(8.959,3.473)--(8.970,3.487)--(8.980,3.500)%
  --(8.990,3.514)--(9.001,3.527)--(9.011,3.541)--(9.021,3.555)--(9.032,3.568)--(9.042,3.582)%
  --(9.052,3.595)--(9.063,3.609)--(9.073,3.623)--(9.083,3.637)--(9.094,3.650)--(9.104,3.664)%
  --(9.114,3.678)--(9.125,3.692)--(9.135,3.706)--(9.145,3.720)--(9.156,3.734)--(9.166,3.748)%
  --(9.176,3.762)--(9.186,3.776)--(9.197,3.790)--(9.207,3.804)--(9.217,3.818)--(9.228,3.832)%
  --(9.238,3.846)--(9.248,3.860)--(9.259,3.874)--(9.269,3.889)--(9.279,3.903)--(9.290,3.917)%
  --(9.300,3.931)--(9.310,3.946)--(9.321,3.960)--(9.331,3.974)--(9.341,3.989)--(9.352,4.003)%
  --(9.362,4.018)--(9.372,4.032)--(9.383,4.047)--(9.393,4.061)--(9.403,4.076)--(9.413,4.090)%
  --(9.424,4.105)--(9.434,4.119)--(9.444,4.134)--(9.455,4.149)--(9.465,4.163)--(9.475,4.178)%
  --(9.486,4.193)--(9.496,4.207)--(9.506,4.222)--(9.517,4.237)--(9.527,4.252)--(9.537,4.267)%
  --(9.548,4.281)--(9.558,4.296)--(9.568,4.311)--(9.579,4.326)--(9.589,4.341)--(9.599,4.356)%
  --(9.610,4.371)--(9.620,4.386)--(9.630,4.401)--(9.640,4.416)--(9.651,4.431)--(9.661,4.446)%
  --(9.671,4.461)--(9.682,4.476)--(9.692,4.491)--(9.702,4.507)--(9.713,4.522)--(9.723,4.537)%
  --(9.733,4.552)--(9.744,4.567)--(9.754,4.583)--(9.764,4.598)--(9.775,4.613)--(9.785,4.628)%
  --(9.795,4.644)--(9.806,4.659)--(9.816,4.675)--(9.826,4.690)--(9.837,4.705)--(9.847,4.721)%
  --(9.857,4.736)--(9.867,4.752)--(9.878,4.767)--(9.888,4.783)--(9.898,4.798)--(9.909,4.814)%
  --(9.919,4.829)--(9.929,4.845)--(9.940,4.860)--(9.950,4.876)--(9.960,4.891)--(9.971,4.907)%
  --(9.981,4.923)--(9.991,4.938)--(10.002,4.954)--(10.012,4.970)--(10.022,4.986)--(10.033,5.001)%
  --(10.043,5.017)--(10.053,5.033)--(10.064,5.049)--(10.074,5.064)--(10.084,5.080)--(10.094,5.096)%
  --(10.105,5.112)--(10.115,5.128)--(10.125,5.143)--(10.136,5.159)--(10.146,5.175)--(10.156,5.191)%
  --(10.167,5.207)--(10.177,5.223)--(10.187,5.239)--(10.198,5.255)--(10.208,5.271)--(10.218,5.287)%
  --(10.229,5.303)--(10.239,5.319)--(10.249,5.335)--(10.260,5.351)--(10.270,5.367)--(10.280,5.383)%
  --(10.291,5.399)--(10.301,5.415)--(10.311,5.432)--(10.321,5.448)--(10.332,5.464)--(10.342,5.480)%
  --(10.352,5.496)--(10.363,5.512)--(10.373,5.528)--(10.383,5.545)--(10.394,5.561)--(10.404,5.577)%
  --(10.414,5.593)--(10.425,5.610)--(10.435,5.626)--(10.445,5.642)--(10.456,5.658)--(10.466,5.675)%
  --(10.476,5.691)--(10.487,5.707)--(10.497,5.724)--(10.507,5.740)--(10.518,5.756)--(10.528,5.773)%
  --(10.538,5.789)--(10.548,5.806)--(10.559,5.822)--(10.569,5.838)--(10.579,5.855)--(10.590,5.871)%
  --(10.600,5.888)--(10.610,5.904)--(10.621,5.920)--(10.631,5.937)--(10.641,5.953)--(10.652,5.970)%
  --(10.662,5.986)--(10.672,6.003)--(10.683,6.019)--(10.693,6.036)--(10.703,6.052)--(10.714,6.069)%
  --(10.724,6.085)--(10.734,6.102)--(10.745,6.119)--(10.755,6.135)--(10.765,6.152)--(10.775,6.168)%
  --(10.786,6.185)--(10.796,6.201)--(10.806,6.218)--(10.817,6.235)--(10.827,6.251)--(10.837,6.268)%
  --(10.848,6.285)--(10.858,6.301)--(10.868,6.318)--(10.879,6.334)--(10.889,6.351)--(10.899,6.368)%
  --(10.910,6.384)--(10.920,6.401)--(10.930,6.418)--(10.941,6.434)--(10.951,6.451)--(10.961,6.468)%
  --(10.972,6.485)--(10.982,6.501)--(10.992,6.518)--(11.002,6.535)--(11.013,6.551)--(11.023,6.568)%
  --(11.033,6.585)--(11.044,6.602)--(11.054,6.618)--(11.064,6.635)--(11.075,6.652)--(11.085,6.669)%
  --(11.095,6.685)--(11.106,6.702)--(11.116,6.719)--(11.126,6.736)--(11.137,6.752)--(11.147,6.769)%
  --(11.157,6.786)--(11.168,6.803)--(11.178,6.819)--(11.188,6.836)--(11.199,6.853)--(11.209,6.870)%
  --(11.219,6.887)--(11.229,6.903)--(11.240,6.920)--(11.250,6.937)--(11.260,6.954)--(11.271,6.971)%
  --(11.281,6.987)--(11.291,7.004)--(11.302,7.021)--(11.312,7.038)--(11.322,7.055)--(11.333,7.071)%
  --(11.343,7.088)--(11.353,7.105)--(11.364,7.122)--(11.374,7.139)--(11.384,7.156)--(11.395,7.172)%
  --(11.405,7.189)--(11.415,7.206)--(11.426,7.223)--(11.436,7.240)--(11.446,7.256)--(11.456,7.273)%
  --(11.467,7.290)--(11.477,7.307)--(11.487,7.324)--(11.498,7.341)--(11.508,7.357)--(11.518,7.374)%
  --(11.529,7.391)--(11.539,7.408)--(11.549,7.425)--(11.560,7.441)--(11.570,7.458)--(11.580,7.475)%
  --(11.591,7.492)--(11.601,7.509)--(11.611,7.525)--(11.622,7.542)--(11.632,7.559)--(11.642,7.576)%
  --(11.653,7.593)--(11.663,7.609)--(11.673,7.626)--(11.683,7.643)--(11.694,7.660)--(11.704,7.677)%
  --(11.714,7.693)--(11.725,7.710)--(11.735,7.727)--(11.745,7.744)--(11.756,7.760)--(11.766,7.777)%
  --(11.776,7.794)--(11.787,7.811)--(11.797,7.827)--(11.807,7.844)--(11.818,7.861)--(11.828,7.878)%
  --(11.838,7.894)--(11.849,7.911)--(11.859,7.928)--(11.869,7.944)--(11.880,7.961)--(11.890,7.978)%
  --(11.900,7.995)--(11.910,8.011)--(11.921,8.028)--(11.931,8.045)--(11.941,8.061);
\gpcolor{color=gp lt color border}
\draw[gp path] (1.320,8.631)--(1.320,0.985)--(13.447,0.985)--(13.447,8.631)--cycle;
%% coordinates of the plot area
\gpdefrectangularnode{gp plot 1}{\pgfpoint{1.320cm}{0.985cm}}{\pgfpoint{13.447cm}{8.631cm}}
\end{tikzpicture}
%% gnuplot variables

	\caption{Gráfico da razão entre a densidade de energia $\varepsilon$ e a densidade $\rho$. Note que o mínimo da energia ocorre em aproximadamente \np[fm^{-3}]{0.15}, que é a densidade de saturação da matéria nuclear. Para obtermos o valor de \np[MeV]{16} por nucleon, basta subtrairmos o valor da massa do nucleon. \protect[Parameters: eNJL1; Proton fraction: 1/2] }
	\label{Fig:energy_by_nucleon_graph}
\end{figure*}

\FloatBarrier
%%%%%%%%%%%%%%%%%%%%%%%%%%%%%%%%%%%%%%%%%%%%%%%%%%%%%%%
\section{Adição da massa constituinte à equação do Gap}
%%%%%%%%%%%%%%%%%%%%%%%%%%%%%%%%%%%%%%%%%%%%%%%%%%%%%%%

Utilizando a Equação~\ref{Eq:Gap} com $m \neq 0$, os resultados para a massa não vão a zero, apresentando o comportamento mostrado na Fig.~\ref{Fig:mass_graph_eNJL1m}.\footnote[][3cm]{No artigo Pais, Menezes, Providência, não há nenhuma alteração na energia cinética (Eq.~(5)), porém deveria haver o segundo termo da Eq.~\eqref{Eq:Engergia_cin_separada}, não?}

\begin{figure*}
	\begin{tikzpicture}[gnuplot]
%% generated with GNUPLOT 5.0p2 (Lua 5.2; terminal rev. 99, script rev. 100)
%% Wed Mar  9 15:32:49 2016
\path (0.000,0.000) rectangle (14.000,9.000);
\gpcolor{color=gp lt color border}
\gpsetlinetype{gp lt border}
\gpsetdashtype{gp dt solid}
\gpsetlinewidth{1.00}
\draw[gp path] (1.504,0.985)--(1.684,0.985);
\draw[gp path] (13.447,0.985)--(13.267,0.985);
\node[gp node right] at (1.320,0.985) {$-600$};
\draw[gp path] (1.504,1.835)--(1.684,1.835);
\draw[gp path] (13.447,1.835)--(13.267,1.835);
\node[gp node right] at (1.320,1.835) {$-400$};
\draw[gp path] (1.504,2.684)--(1.684,2.684);
\draw[gp path] (13.447,2.684)--(13.267,2.684);
\node[gp node right] at (1.320,2.684) {$-200$};
\draw[gp path] (1.504,3.534)--(1.684,3.534);
\draw[gp path] (13.447,3.534)--(13.267,3.534);
\node[gp node right] at (1.320,3.534) {$0$};
\draw[gp path] (1.504,4.383)--(1.684,4.383);
\draw[gp path] (13.447,4.383)--(13.267,4.383);
\node[gp node right] at (1.320,4.383) {$200$};
\draw[gp path] (1.504,5.233)--(1.684,5.233);
\draw[gp path] (13.447,5.233)--(13.267,5.233);
\node[gp node right] at (1.320,5.233) {$400$};
\draw[gp path] (1.504,6.082)--(1.684,6.082);
\draw[gp path] (13.447,6.082)--(13.267,6.082);
\node[gp node right] at (1.320,6.082) {$600$};
\draw[gp path] (1.504,6.932)--(1.684,6.932);
\draw[gp path] (13.447,6.932)--(13.267,6.932);
\node[gp node right] at (1.320,6.932) {$800$};
\draw[gp path] (1.504,7.781)--(1.684,7.781);
\draw[gp path] (13.447,7.781)--(13.267,7.781);
\node[gp node right] at (1.320,7.781) {$1000$};
\draw[gp path] (1.504,8.631)--(1.684,8.631);
\draw[gp path] (13.447,8.631)--(13.267,8.631);
\node[gp node right] at (1.320,8.631) {$1200$};
\draw[gp path] (1.504,0.985)--(1.504,1.165);
\draw[gp path] (1.504,8.631)--(1.504,8.451);
\node[gp node center] at (1.504,0.677) {$0$};
\draw[gp path] (2.698,0.985)--(2.698,1.165);
\draw[gp path] (2.698,8.631)--(2.698,8.451);
\node[gp node center] at (2.698,0.677) {$200$};
\draw[gp path] (3.893,0.985)--(3.893,1.165);
\draw[gp path] (3.893,8.631)--(3.893,8.451);
\node[gp node center] at (3.893,0.677) {$400$};
\draw[gp path] (5.087,0.985)--(5.087,1.165);
\draw[gp path] (5.087,8.631)--(5.087,8.451);
\node[gp node center] at (5.087,0.677) {$600$};
\draw[gp path] (6.281,0.985)--(6.281,1.165);
\draw[gp path] (6.281,8.631)--(6.281,8.451);
\node[gp node center] at (6.281,0.677) {$800$};
\draw[gp path] (7.476,0.985)--(7.476,1.165);
\draw[gp path] (7.476,8.631)--(7.476,8.451);
\node[gp node center] at (7.476,0.677) {$1000$};
\draw[gp path] (8.670,0.985)--(8.670,1.165);
\draw[gp path] (8.670,8.631)--(8.670,8.451);
\node[gp node center] at (8.670,0.677) {$1200$};
\draw[gp path] (9.864,0.985)--(9.864,1.165);
\draw[gp path] (9.864,8.631)--(9.864,8.451);
\node[gp node center] at (9.864,0.677) {$1400$};
\draw[gp path] (11.058,0.985)--(11.058,1.165);
\draw[gp path] (11.058,8.631)--(11.058,8.451);
\node[gp node center] at (11.058,0.677) {$1600$};
\draw[gp path] (12.253,0.985)--(12.253,1.165);
\draw[gp path] (12.253,8.631)--(12.253,8.451);
\node[gp node center] at (12.253,0.677) {$1800$};
\draw[gp path] (13.447,0.985)--(13.447,1.165);
\draw[gp path] (13.447,8.631)--(13.447,8.451);
\node[gp node center] at (13.447,0.677) {$2000$};
\draw[gp path] (1.504,8.631)--(1.504,0.985)--(13.447,0.985)--(13.447,8.631)--cycle;
\node[gp node center,rotate=-270] at (0.246,4.808) {$f(M) = M + G_s\rho_s - 2G_{sv}\rho_s\rho^2$ (MeV)};
\node[gp node center] at (7.475,0.215) {$M$ (MeV)};
\node[gp node left] at (2.972,8.297) {$\rho_{\rm{min}}$};
\gpcolor{rgb color={0.580,0.000,0.827}}
\draw[gp path] (1.872,8.297)--(2.788,8.297);
\draw[gp path] (1.507,1.621)--(1.510,1.620)--(1.513,1.619)--(1.516,1.617)--(1.519,1.616)%
  --(1.522,1.615)--(1.525,1.614)--(1.528,1.613)--(1.531,1.611)--(1.534,1.610)--(1.537,1.609)%
  --(1.540,1.608)--(1.543,1.607)--(1.546,1.605)--(1.549,1.604)--(1.552,1.603)--(1.555,1.602)%
  --(1.558,1.601)--(1.561,1.599)--(1.564,1.598)--(1.567,1.597)--(1.570,1.596)--(1.573,1.595)%
  --(1.576,1.593)--(1.579,1.592)--(1.582,1.591)--(1.585,1.590)--(1.588,1.589)--(1.591,1.587)%
  --(1.594,1.586)--(1.597,1.585)--(1.600,1.584)--(1.603,1.583)--(1.606,1.581)--(1.609,1.580)%
  --(1.611,1.579)--(1.614,1.578)--(1.617,1.577)--(1.620,1.576)--(1.623,1.574)--(1.626,1.573)%
  --(1.629,1.572)--(1.632,1.571)--(1.635,1.570)--(1.638,1.569)--(1.641,1.567)--(1.644,1.566)%
  --(1.647,1.565)--(1.650,1.564)--(1.653,1.563)--(1.656,1.562)--(1.659,1.560)--(1.662,1.559)%
  --(1.665,1.558)--(1.668,1.557)--(1.671,1.556)--(1.674,1.555)--(1.677,1.553)--(1.680,1.552)%
  --(1.683,1.551)--(1.686,1.550)--(1.689,1.549)--(1.692,1.548)--(1.695,1.547)--(1.698,1.545)%
  --(1.701,1.544)--(1.704,1.543)--(1.707,1.542)--(1.710,1.541)--(1.713,1.540)--(1.716,1.539)%
  --(1.719,1.538)--(1.722,1.536)--(1.725,1.535)--(1.728,1.534)--(1.731,1.533)--(1.734,1.532)%
  --(1.737,1.531)--(1.740,1.530)--(1.743,1.529)--(1.746,1.528)--(1.749,1.527)--(1.752,1.525)%
  --(1.755,1.524)--(1.758,1.523)--(1.761,1.522)--(1.764,1.521)--(1.767,1.520)--(1.770,1.519)%
  --(1.773,1.518)--(1.776,1.517)--(1.779,1.516)--(1.782,1.515)--(1.785,1.514)--(1.788,1.513)%
  --(1.791,1.511)--(1.794,1.510)--(1.797,1.509)--(1.800,1.508)--(1.803,1.507)--(1.806,1.506)%
  --(1.809,1.505)--(1.812,1.504)--(1.815,1.503)--(1.818,1.502)--(1.820,1.501)--(1.823,1.500)%
  --(1.826,1.499)--(1.829,1.498)--(1.832,1.497)--(1.835,1.496)--(1.838,1.495)--(1.841,1.494)%
  --(1.844,1.493)--(1.847,1.492)--(1.850,1.491)--(1.853,1.490)--(1.856,1.489)--(1.859,1.488)%
  --(1.862,1.487)--(1.865,1.486)--(1.868,1.485)--(1.871,1.484)--(1.874,1.483)--(1.877,1.482)%
  --(1.880,1.481)--(1.883,1.480)--(1.886,1.479)--(1.889,1.478)--(1.892,1.477)--(1.895,1.476)%
  --(1.898,1.476)--(1.901,1.475)--(1.904,1.474)--(1.907,1.473)--(1.910,1.472)--(1.913,1.471)%
  --(1.916,1.470)--(1.919,1.469)--(1.922,1.468)--(1.925,1.467)--(1.928,1.466)--(1.931,1.465)%
  --(1.934,1.464)--(1.937,1.464)--(1.940,1.463)--(1.943,1.462)--(1.946,1.461)--(1.949,1.460)%
  --(1.952,1.459)--(1.955,1.458)--(1.958,1.457)--(1.961,1.457)--(1.964,1.456)--(1.967,1.455)%
  --(1.970,1.454)--(1.973,1.453)--(1.976,1.452)--(1.979,1.451)--(1.982,1.451)--(1.985,1.450)%
  --(1.988,1.449)--(1.991,1.448)--(1.994,1.447)--(1.997,1.447)--(2.000,1.446)--(2.003,1.445)%
  --(2.006,1.444)--(2.009,1.443)--(2.012,1.442)--(2.015,1.442)--(2.018,1.441)--(2.021,1.440)%
  --(2.024,1.439)--(2.027,1.439)--(2.029,1.438)--(2.032,1.437)--(2.035,1.436)--(2.038,1.435)%
  --(2.041,1.435)--(2.044,1.434)--(2.047,1.433)--(2.050,1.432)--(2.053,1.432)--(2.056,1.431)%
  --(2.059,1.430)--(2.062,1.430)--(2.065,1.429)--(2.068,1.428)--(2.071,1.427)--(2.074,1.427)%
  --(2.077,1.426)--(2.080,1.425)--(2.083,1.425)--(2.086,1.424)--(2.089,1.423)--(2.092,1.422)%
  --(2.095,1.422)--(2.098,1.421)--(2.101,1.420)--(2.104,1.420)--(2.107,1.419)--(2.110,1.418)%
  --(2.113,1.418)--(2.116,1.417)--(2.119,1.416)--(2.122,1.416)--(2.125,1.415)--(2.128,1.415)%
  --(2.131,1.414)--(2.134,1.413)--(2.137,1.413)--(2.140,1.412)--(2.143,1.411)--(2.146,1.411)%
  --(2.149,1.410)--(2.152,1.410)--(2.155,1.409)--(2.158,1.408)--(2.161,1.408)--(2.164,1.407)%
  --(2.167,1.407)--(2.170,1.406)--(2.173,1.405)--(2.176,1.405)--(2.179,1.404)--(2.182,1.404)%
  --(2.185,1.403)--(2.188,1.403)--(2.191,1.402)--(2.194,1.402)--(2.197,1.401)--(2.200,1.400)%
  --(2.203,1.400)--(2.206,1.399)--(2.209,1.399)--(2.212,1.398)--(2.215,1.398)--(2.218,1.397)%
  --(2.221,1.397)--(2.224,1.396)--(2.227,1.396)--(2.230,1.395)--(2.233,1.395)--(2.236,1.394)%
  --(2.238,1.394)--(2.241,1.393)--(2.244,1.393)--(2.247,1.393)--(2.250,1.392)--(2.253,1.392)%
  --(2.256,1.391)--(2.259,1.391)--(2.262,1.390)--(2.265,1.390)--(2.268,1.389)--(2.271,1.389)%
  --(2.274,1.388)--(2.277,1.388)--(2.280,1.388)--(2.283,1.387)--(2.286,1.387)--(2.289,1.386)%
  --(2.292,1.386)--(2.295,1.386)--(2.298,1.385)--(2.301,1.385)--(2.304,1.384)--(2.307,1.384)%
  --(2.310,1.384)--(2.313,1.383)--(2.316,1.383)--(2.319,1.383)--(2.322,1.382)--(2.325,1.382)%
  --(2.328,1.382)--(2.331,1.381)--(2.334,1.381)--(2.337,1.380)--(2.340,1.380)--(2.343,1.380)%
  --(2.346,1.380)--(2.349,1.379)--(2.352,1.379)--(2.355,1.379)--(2.358,1.378)--(2.361,1.378)%
  --(2.364,1.378)--(2.367,1.377)--(2.370,1.377)--(2.373,1.377)--(2.376,1.376)--(2.379,1.376)%
  --(2.382,1.376)--(2.385,1.376)--(2.388,1.375)--(2.391,1.375)--(2.394,1.375)--(2.397,1.375)%
  --(2.400,1.374)--(2.403,1.374)--(2.406,1.374)--(2.409,1.374)--(2.412,1.373)--(2.415,1.373)%
  --(2.418,1.373)--(2.421,1.373)--(2.424,1.373)--(2.427,1.372)--(2.430,1.372)--(2.433,1.372)%
  --(2.436,1.372)--(2.439,1.372)--(2.442,1.371)--(2.445,1.371)--(2.447,1.371)--(2.450,1.371)%
  --(2.453,1.371)--(2.456,1.370)--(2.459,1.370)--(2.462,1.370)--(2.465,1.370)--(2.468,1.370)%
  --(2.471,1.370)--(2.474,1.370)--(2.477,1.369)--(2.480,1.369)--(2.483,1.369)--(2.486,1.369)%
  --(2.489,1.369)--(2.492,1.369)--(2.495,1.369)--(2.498,1.369)--(2.501,1.368)--(2.504,1.368)%
  --(2.507,1.368)--(2.510,1.368)--(2.513,1.368)--(2.516,1.368)--(2.519,1.368)--(2.522,1.368)%
  --(2.525,1.368)--(2.528,1.368)--(2.531,1.368)--(2.534,1.368)--(2.537,1.367)--(2.540,1.367)%
  --(2.543,1.367)--(2.546,1.367)--(2.549,1.367)--(2.552,1.367)--(2.555,1.367)--(2.558,1.367)%
  --(2.561,1.367)--(2.564,1.367)--(2.567,1.367)--(2.570,1.367)--(2.573,1.367)--(2.576,1.367)%
  --(2.579,1.367)--(2.582,1.367)--(2.585,1.367)--(2.588,1.367)--(2.591,1.367)--(2.594,1.367)%
  --(2.597,1.367)--(2.600,1.367)--(2.603,1.367)--(2.606,1.367)--(2.609,1.367)--(2.612,1.367)%
  --(2.615,1.367)--(2.618,1.367)--(2.621,1.368)--(2.624,1.368)--(2.627,1.368)--(2.630,1.368)%
  --(2.633,1.368)--(2.636,1.368)--(2.639,1.368)--(2.642,1.368)--(2.645,1.368)--(2.648,1.368)%
  --(2.651,1.368)--(2.654,1.368)--(2.656,1.369)--(2.659,1.369)--(2.662,1.369)--(2.665,1.369)%
  --(2.668,1.369)--(2.671,1.369)--(2.674,1.369)--(2.677,1.369)--(2.680,1.369)--(2.683,1.370)%
  --(2.686,1.370)--(2.689,1.370)--(2.692,1.370)--(2.695,1.370)--(2.698,1.370)--(2.701,1.371)%
  --(2.704,1.371)--(2.707,1.371)--(2.710,1.371)--(2.713,1.371)--(2.716,1.371)--(2.719,1.372)%
  --(2.722,1.372)--(2.725,1.372)--(2.728,1.372)--(2.731,1.372)--(2.734,1.373)--(2.737,1.373)%
  --(2.740,1.373)--(2.743,1.373)--(2.746,1.373)--(2.749,1.374)--(2.752,1.374)--(2.755,1.374)%
  --(2.758,1.374)--(2.761,1.375)--(2.764,1.375)--(2.767,1.375)--(2.770,1.375)--(2.773,1.375)%
  --(2.776,1.376)--(2.779,1.376)--(2.782,1.376)--(2.785,1.377)--(2.788,1.377)--(2.791,1.377)%
  --(2.794,1.377)--(2.797,1.378)--(2.800,1.378)--(2.803,1.378)--(2.806,1.378)--(2.809,1.379)%
  --(2.812,1.379)--(2.815,1.379)--(2.818,1.380)--(2.821,1.380)--(2.824,1.380)--(2.827,1.381)%
  --(2.830,1.381)--(2.833,1.381)--(2.836,1.382)--(2.839,1.382)--(2.842,1.382)--(2.845,1.383)%
  --(2.848,1.383)--(2.851,1.383)--(2.854,1.384)--(2.857,1.384)--(2.860,1.384)--(2.863,1.385)%
  --(2.866,1.385)--(2.868,1.385)--(2.871,1.386)--(2.874,1.386)--(2.877,1.387)--(2.880,1.387)%
  --(2.883,1.387)--(2.886,1.388)--(2.889,1.388)--(2.892,1.388)--(2.895,1.389)--(2.898,1.389)%
  --(2.901,1.390)--(2.904,1.390)--(2.907,1.391)--(2.910,1.391)--(2.913,1.391)--(2.916,1.392)%
  --(2.919,1.392)--(2.922,1.393)--(2.925,1.393)--(2.928,1.393)--(2.931,1.394)--(2.934,1.394)%
  --(2.937,1.395)--(2.940,1.395)--(2.943,1.396)--(2.946,1.396)--(2.949,1.397)--(2.952,1.397)%
  --(2.955,1.398)--(2.958,1.398)--(2.961,1.398)--(2.964,1.399)--(2.967,1.399)--(2.970,1.400)%
  --(2.973,1.400)--(2.976,1.401)--(2.979,1.401)--(2.982,1.402)--(2.985,1.402)--(2.988,1.403)%
  --(2.991,1.403)--(2.994,1.404)--(2.997,1.404)--(3.000,1.405)--(3.003,1.405)--(3.006,1.406)%
  --(3.009,1.406)--(3.012,1.407)--(3.015,1.408)--(3.018,1.408)--(3.021,1.409)--(3.024,1.409)%
  --(3.027,1.410)--(3.030,1.410)--(3.033,1.411)--(3.036,1.411)--(3.039,1.412)--(3.042,1.412)%
  --(3.045,1.413)--(3.048,1.414)--(3.051,1.414)--(3.054,1.415)--(3.057,1.415)--(3.060,1.416)%
  --(3.063,1.416)--(3.066,1.417)--(3.069,1.418)--(3.072,1.418)--(3.075,1.419)--(3.077,1.419)%
  --(3.080,1.420)--(3.083,1.421)--(3.086,1.421)--(3.089,1.422)--(3.092,1.422)--(3.095,1.423)%
  --(3.098,1.424)--(3.101,1.424)--(3.104,1.425)--(3.107,1.426)--(3.110,1.426)--(3.113,1.427)%
  --(3.116,1.427)--(3.119,1.428)--(3.122,1.429)--(3.125,1.429)--(3.128,1.430)--(3.131,1.431)%
  --(3.134,1.431)--(3.137,1.432)--(3.140,1.433)--(3.143,1.433)--(3.146,1.434)--(3.149,1.435)%
  --(3.152,1.435)--(3.155,1.436)--(3.158,1.437)--(3.161,1.437)--(3.164,1.438)--(3.167,1.439)%
  --(3.170,1.439)--(3.173,1.440)--(3.176,1.441)--(3.179,1.442)--(3.182,1.442)--(3.185,1.443)%
  --(3.188,1.444)--(3.191,1.444)--(3.194,1.445)--(3.197,1.446)--(3.200,1.447)--(3.203,1.447)%
  --(3.206,1.448)--(3.209,1.449)--(3.212,1.449)--(3.215,1.450)--(3.218,1.451)--(3.221,1.452)%
  --(3.224,1.452)--(3.227,1.453)--(3.230,1.454)--(3.233,1.455)--(3.236,1.455)--(3.239,1.456)%
  --(3.242,1.457)--(3.245,1.458)--(3.248,1.458)--(3.251,1.459)--(3.254,1.460)--(3.257,1.461)%
  --(3.260,1.462)--(3.263,1.462)--(3.266,1.463)--(3.269,1.464)--(3.272,1.465)--(3.275,1.465)%
  --(3.278,1.466)--(3.281,1.467)--(3.284,1.468)--(3.286,1.469)--(3.289,1.469)--(3.292,1.470)%
  --(3.295,1.471)--(3.298,1.472)--(3.301,1.473)--(3.304,1.473)--(3.307,1.474)--(3.310,1.475)%
  --(3.313,1.476)--(3.316,1.477)--(3.319,1.478)--(3.322,1.478)--(3.325,1.479)--(3.328,1.480)%
  --(3.331,1.481)--(3.334,1.482)--(3.337,1.483)--(3.340,1.483)--(3.343,1.484)--(3.346,1.485)%
  --(3.349,1.486)--(3.352,1.487)--(3.355,1.488)--(3.358,1.489)--(3.361,1.489)--(3.364,1.490)%
  --(3.367,1.491)--(3.370,1.492)--(3.373,1.493)--(3.376,1.494)--(3.379,1.495)--(3.382,1.496)%
  --(3.385,1.497)--(3.388,1.497)--(3.391,1.498)--(3.394,1.499)--(3.397,1.500)--(3.400,1.501)%
  --(3.403,1.502)--(3.406,1.503)--(3.409,1.504)--(3.412,1.505)--(3.415,1.506)--(3.418,1.506)%
  --(3.421,1.507)--(3.424,1.508)--(3.427,1.509)--(3.430,1.510)--(3.433,1.511)--(3.436,1.512)%
  --(3.439,1.513)--(3.442,1.514)--(3.445,1.515)--(3.448,1.516)--(3.451,1.517)--(3.454,1.518)%
  --(3.457,1.519)--(3.460,1.519)--(3.463,1.520)--(3.466,1.521)--(3.469,1.522)--(3.472,1.523)%
  --(3.475,1.524)--(3.478,1.525)--(3.481,1.526)--(3.484,1.527)--(3.487,1.528)--(3.490,1.529)%
  --(3.493,1.530)--(3.495,1.531)--(3.498,1.532)--(3.501,1.533)--(3.504,1.534)--(3.507,1.535)%
  --(3.510,1.536)--(3.513,1.537)--(3.516,1.538)--(3.519,1.539)--(3.522,1.540)--(3.525,1.541)%
  --(3.528,1.542)--(3.531,1.543)--(3.534,1.544)--(3.537,1.545)--(3.540,1.546)--(3.543,1.547)%
  --(3.546,1.548)--(3.549,1.549)--(3.552,1.550)--(3.555,1.551)--(3.558,1.552)--(3.561,1.553)%
  --(3.564,1.554)--(3.567,1.555)--(3.570,1.556)--(3.573,1.557)--(3.576,1.558)--(3.579,1.559)%
  --(3.582,1.560)--(3.585,1.561)--(3.588,1.562)--(3.591,1.564)--(3.594,1.565)--(3.597,1.566)%
  --(3.600,1.567)--(3.603,1.568)--(3.606,1.569)--(3.609,1.570)--(3.612,1.571)--(3.615,1.572)%
  --(3.618,1.573)--(3.621,1.574)--(3.624,1.575)--(3.627,1.576)--(3.630,1.577)--(3.633,1.578)%
  --(3.636,1.580)--(3.639,1.581)--(3.642,1.582)--(3.645,1.583)--(3.648,1.584)--(3.651,1.585)%
  --(3.654,1.586)--(3.657,1.587)--(3.660,1.588)--(3.663,1.589)--(3.666,1.590)--(3.669,1.592)%
  --(3.672,1.593)--(3.675,1.594)--(3.678,1.595)--(3.681,1.596)--(3.684,1.597)--(3.687,1.598)%
  --(3.690,1.599)--(3.693,1.600)--(3.696,1.602)--(3.699,1.603)--(3.702,1.604)--(3.704,1.605)%
  --(3.707,1.606)--(3.710,1.607)--(3.713,1.608)--(3.716,1.610)--(3.719,1.611)--(3.722,1.612)%
  --(3.725,1.613)--(3.728,1.614)--(3.731,1.615)--(3.734,1.616)--(3.737,1.618)--(3.740,1.619)%
  --(3.743,1.620)--(3.746,1.621)--(3.749,1.622)--(3.752,1.623)--(3.755,1.624)--(3.758,1.626)%
  --(3.761,1.627)--(3.764,1.628)--(3.767,1.629)--(3.770,1.630)--(3.773,1.631)--(3.776,1.633)%
  --(3.779,1.634)--(3.782,1.635)--(3.785,1.636)--(3.788,1.637)--(3.791,1.639)--(3.794,1.640)%
  --(3.797,1.641)--(3.800,1.642)--(3.803,1.643)--(3.806,1.645)--(3.809,1.646)--(3.812,1.647)%
  --(3.815,1.648)--(3.818,1.649)--(3.821,1.651)--(3.824,1.652)--(3.827,1.653)--(3.830,1.654)%
  --(3.833,1.655)--(3.836,1.657)--(3.839,1.658)--(3.842,1.659)--(3.845,1.660)--(3.848,1.661)%
  --(3.851,1.663)--(3.854,1.664)--(3.857,1.665)--(3.860,1.666)--(3.863,1.668)--(3.866,1.669)%
  --(3.869,1.670)--(3.872,1.671)--(3.875,1.673)--(3.878,1.674)--(3.881,1.675)--(3.884,1.676)%
  --(3.887,1.678)--(3.890,1.679)--(3.893,1.680)--(3.896,1.681)--(3.899,1.682)--(3.902,1.684)%
  --(3.905,1.685)--(3.908,1.686)--(3.911,1.688)--(3.914,1.689)--(3.916,1.690)--(3.919,1.691)%
  --(3.922,1.693)--(3.925,1.694)--(3.928,1.695)--(3.931,1.696)--(3.934,1.698)--(3.937,1.699)%
  --(3.940,1.700)--(3.943,1.701)--(3.946,1.703)--(3.949,1.704)--(3.952,1.705)--(3.955,1.707)%
  --(3.958,1.708)--(3.961,1.709)--(3.964,1.710)--(3.967,1.712)--(3.970,1.713)--(3.973,1.714)%
  --(3.976,1.716)--(3.979,1.717)--(3.982,1.718)--(3.985,1.720)--(3.988,1.721)--(3.991,1.722)%
  --(3.994,1.723)--(3.997,1.725)--(4.000,1.726)--(4.003,1.727)--(4.006,1.729)--(4.009,1.730)%
  --(4.012,1.731)--(4.015,1.733)--(4.018,1.734)--(4.021,1.735)--(4.024,1.737)--(4.027,1.738)%
  --(4.030,1.739)--(4.033,1.741)--(4.036,1.742)--(4.039,1.743)--(4.042,1.745)--(4.045,1.746)%
  --(4.048,1.747)--(4.051,1.749)--(4.054,1.750)--(4.057,1.751)--(4.060,1.753)--(4.063,1.754)%
  --(4.066,1.755)--(4.069,1.757)--(4.072,1.758)--(4.075,1.759)--(4.078,1.761)--(4.081,1.762)%
  --(4.084,1.763)--(4.087,1.765)--(4.090,1.766)--(4.093,1.767)--(4.096,1.769)--(4.099,1.770)%
  --(4.102,1.771)--(4.105,1.773)--(4.108,1.774)--(4.111,1.776)--(4.114,1.777)--(4.117,1.778)%
  --(4.120,1.780)--(4.123,1.781)--(4.125,1.782)--(4.128,1.784)--(4.131,1.785)--(4.134,1.787)%
  --(4.137,1.788)--(4.140,1.789)--(4.143,1.791)--(4.146,1.792)--(4.149,1.793)--(4.152,1.795)%
  --(4.155,1.796)--(4.158,1.798)--(4.161,1.799)--(4.164,1.800)--(4.167,1.802)--(4.170,1.803)%
  --(4.173,1.805)--(4.176,1.806)--(4.179,1.807)--(4.182,1.809)--(4.185,1.810)--(4.188,1.812)%
  --(4.191,1.813)--(4.194,1.814)--(4.197,1.816)--(4.200,1.817)--(4.203,1.819)--(4.206,1.820)%
  --(4.209,1.821)--(4.212,1.823)--(4.215,1.824)--(4.218,1.826)--(4.221,1.827)--(4.224,1.829)%
  --(4.227,1.830)--(4.230,1.831)--(4.233,1.833)--(4.236,1.834)--(4.239,1.836)--(4.242,1.837)%
  --(4.245,1.839)--(4.248,1.840)--(4.251,1.841)--(4.254,1.843)--(4.257,1.844)--(4.260,1.846)%
  --(4.263,1.847)--(4.266,1.849)--(4.269,1.850)--(4.272,1.851)--(4.275,1.853)--(4.278,1.854)%
  --(4.281,1.856)--(4.284,1.857)--(4.287,1.859)--(4.290,1.860)--(4.293,1.862)--(4.296,1.863)%
  --(4.299,1.864)--(4.302,1.866)--(4.305,1.867)--(4.308,1.869)--(4.311,1.870)--(4.314,1.872)%
  --(4.317,1.873)--(4.320,1.875)--(4.323,1.876)--(4.326,1.878)--(4.329,1.879)--(4.332,1.881)%
  --(4.334,1.882)--(4.337,1.883)--(4.340,1.885)--(4.343,1.886)--(4.346,1.888)--(4.349,1.889)%
  --(4.352,1.891)--(4.355,1.892)--(4.358,1.894)--(4.361,1.895)--(4.364,1.897)--(4.367,1.898)%
  --(4.370,1.900)--(4.373,1.901)--(4.376,1.903)--(4.379,1.904)--(4.382,1.906)--(4.385,1.907)%
  --(4.388,1.909)--(4.391,1.910)--(4.394,1.912)--(4.397,1.913)--(4.400,1.915)--(4.403,1.916)%
  --(4.406,1.918)--(4.409,1.919)--(4.412,1.921)--(4.415,1.922)--(4.418,1.924)--(4.421,1.925)%
  --(4.424,1.927)--(4.427,1.928)--(4.430,1.930)--(4.433,1.931)--(4.436,1.933)--(4.439,1.934)%
  --(4.442,1.936)--(4.445,1.937)--(4.448,1.939)--(4.451,1.940)--(4.454,1.942)--(4.457,1.943)%
  --(4.460,1.945)--(4.463,1.946)--(4.466,1.948)--(4.469,1.949)--(4.472,1.951)--(4.475,1.952)%
  --(4.478,1.954)--(4.481,1.955)--(4.484,1.957)--(4.487,1.959)--(4.490,1.960)--(4.493,1.962)%
  --(4.496,1.963)--(4.499,1.965)--(4.502,1.966)--(4.505,1.968)--(4.508,1.969)--(4.511,1.971)%
  --(4.514,1.972)--(4.517,1.974)--(4.520,1.975)--(4.523,1.977)--(4.526,1.978)--(4.529,1.980)%
  --(4.532,1.982)--(4.535,1.983)--(4.538,1.985)--(4.541,1.986)--(4.543,1.988)--(4.546,1.989)%
  --(4.549,1.991)--(4.552,1.992)--(4.555,1.994)--(4.558,1.996)--(4.561,1.997)--(4.564,1.999)%
  --(4.567,2.000)--(4.570,2.002)--(4.573,2.003)--(4.576,2.005)--(4.579,2.006)--(4.582,2.008)%
  --(4.585,2.010)--(4.588,2.011)--(4.591,2.013)--(4.594,2.014)--(4.597,2.016)--(4.600,2.017)%
  --(4.603,2.019)--(4.606,2.021)--(4.609,2.022)--(4.612,2.024)--(4.615,2.025)--(4.618,2.027)%
  --(4.621,2.028)--(4.624,2.030)--(4.627,2.032)--(4.630,2.033)--(4.633,2.035)--(4.636,2.036)%
  --(4.639,2.038)--(4.642,2.040)--(4.645,2.041)--(4.648,2.043)--(4.651,2.044)--(4.654,2.046)%
  --(4.657,2.047)--(4.660,2.049)--(4.663,2.051)--(4.666,2.052)--(4.669,2.054)--(4.672,2.055)%
  --(4.675,2.057)--(4.678,2.059)--(4.681,2.060)--(4.684,2.062)--(4.687,2.063)--(4.690,2.065)%
  --(4.693,2.067)--(4.696,2.068)--(4.699,2.070)--(4.702,2.071)--(4.705,2.073)--(4.708,2.075)%
  --(4.711,2.076)--(4.714,2.078)--(4.717,2.079)--(4.720,2.081)--(4.723,2.083)--(4.726,2.084)%
  --(4.729,2.086)--(4.732,2.088)--(4.735,2.089)--(4.738,2.091)--(4.741,2.092)--(4.744,2.094)%
  --(4.747,2.096)--(4.750,2.097)--(4.752,2.099)--(4.755,2.100)--(4.758,2.102)--(4.761,2.104)%
  --(4.764,2.105)--(4.767,2.107)--(4.770,2.109)--(4.773,2.110)--(4.776,2.112)--(4.779,2.113)%
  --(4.782,2.115)--(4.785,2.117)--(4.788,2.118)--(4.791,2.120)--(4.794,2.122)--(4.797,2.123)%
  --(4.800,2.125)--(4.803,2.127)--(4.806,2.128)--(4.809,2.130)--(4.812,2.131)--(4.815,2.133)%
  --(4.818,2.135)--(4.821,2.136)--(4.824,2.138)--(4.827,2.140)--(4.830,2.141)--(4.833,2.143)%
  --(4.836,2.145)--(4.839,2.146)--(4.842,2.148)--(4.845,2.150)--(4.848,2.151)--(4.851,2.153)%
  --(4.854,2.154)--(4.857,2.156)--(4.860,2.158)--(4.863,2.159)--(4.866,2.161)--(4.869,2.163)%
  --(4.872,2.164)--(4.875,2.166)--(4.878,2.168)--(4.881,2.169)--(4.884,2.171)--(4.887,2.173)%
  --(4.890,2.174)--(4.893,2.176)--(4.896,2.178)--(4.899,2.179)--(4.902,2.181)--(4.905,2.183)%
  --(4.908,2.184)--(4.911,2.186)--(4.914,2.188)--(4.917,2.189)--(4.920,2.191)--(4.923,2.193)%
  --(4.926,2.194)--(4.929,2.196)--(4.932,2.198)--(4.935,2.199)--(4.938,2.201)--(4.941,2.203)%
  --(4.944,2.204)--(4.947,2.206)--(4.950,2.208)--(4.953,2.209)--(4.956,2.211)--(4.959,2.213)%
  --(4.961,2.214)--(4.964,2.216)--(4.967,2.218)--(4.970,2.220)--(4.973,2.221)--(4.976,2.223)%
  --(4.979,2.225)--(4.982,2.226)--(4.985,2.228)--(4.988,2.230)--(4.991,2.231)--(4.994,2.233)%
  --(4.997,2.235)--(5.000,2.236)--(5.003,2.238)--(5.006,2.240)--(5.009,2.241)--(5.012,2.243)%
  --(5.015,2.245)--(5.018,2.247)--(5.021,2.248)--(5.024,2.250)--(5.027,2.252)--(5.030,2.253)%
  --(5.033,2.255)--(5.036,2.257)--(5.039,2.258)--(5.042,2.260)--(5.045,2.262)--(5.048,2.264)%
  --(5.051,2.265)--(5.054,2.267)--(5.057,2.269)--(5.060,2.270)--(5.063,2.272)--(5.066,2.274)%
  --(5.069,2.275)--(5.072,2.277)--(5.075,2.279)--(5.078,2.281)--(5.081,2.282)--(5.084,2.284)%
  --(5.087,2.286)--(5.090,2.287)--(5.093,2.289)--(5.096,2.291)--(5.099,2.293)--(5.102,2.294)%
  --(5.105,2.296)--(5.108,2.298)--(5.111,2.299)--(5.114,2.301)--(5.117,2.303)--(5.120,2.305)%
  --(5.123,2.306)--(5.126,2.308)--(5.129,2.310)--(5.132,2.312)--(5.135,2.313)--(5.138,2.315)%
  --(5.141,2.317)--(5.144,2.318)--(5.147,2.320)--(5.150,2.322)--(5.153,2.324)--(5.156,2.325)%
  --(5.159,2.327)--(5.162,2.329)--(5.165,2.331)--(5.168,2.332)--(5.171,2.334)--(5.173,2.336)%
  --(5.176,2.337)--(5.179,2.339)--(5.182,2.341)--(5.185,2.343)--(5.188,2.344)--(5.191,2.346)%
  --(5.194,2.348)--(5.197,2.350)--(5.200,2.351)--(5.203,2.353)--(5.206,2.355)--(5.209,2.357)%
  --(5.212,2.358)--(5.215,2.360)--(5.218,2.362)--(5.221,2.364)--(5.224,2.365)--(5.227,2.367)%
  --(5.230,2.369)--(5.233,2.371)--(5.236,2.372)--(5.239,2.374)--(5.242,2.376)--(5.245,2.377)%
  --(5.248,2.379)--(5.251,2.381)--(5.254,2.383)--(5.257,2.385)--(5.260,2.386)--(5.263,2.388)%
  --(5.266,2.390)--(5.269,2.392)--(5.272,2.393)--(5.275,2.395)--(5.278,2.397)--(5.281,2.399)%
  --(5.284,2.400)--(5.287,2.402)--(5.290,2.404)--(5.293,2.406)--(5.296,2.407)--(5.299,2.409)%
  --(5.302,2.411)--(5.305,2.413)--(5.308,2.414)--(5.311,2.416)--(5.314,2.418)--(5.317,2.420)%
  --(5.320,2.421)--(5.323,2.423)--(5.326,2.425)--(5.329,2.427)--(5.332,2.428)--(5.335,2.430)%
  --(5.338,2.432)--(5.341,2.434)--(5.344,2.436)--(5.347,2.437)--(5.350,2.439)--(5.353,2.441)%
  --(5.356,2.443)--(5.359,2.444)--(5.362,2.446)--(5.365,2.448)--(5.368,2.450)--(5.371,2.452)%
  --(5.374,2.453)--(5.377,2.455)--(5.380,2.457)--(5.382,2.459)--(5.385,2.460)--(5.388,2.462)%
  --(5.391,2.464)--(5.394,2.466)--(5.397,2.468)--(5.400,2.469)--(5.403,2.471)--(5.406,2.473)%
  --(5.409,2.475)--(5.412,2.476)--(5.415,2.478)--(5.418,2.480)--(5.421,2.482)--(5.424,2.484)%
  --(5.427,2.485)--(5.430,2.487)--(5.433,2.489)--(5.436,2.491)--(5.439,2.492)--(5.442,2.494)%
  --(5.445,2.496)--(5.448,2.498)--(5.451,2.500)--(5.454,2.501)--(5.457,2.503)--(5.460,2.505)%
  --(5.463,2.507)--(5.466,2.509)--(5.469,2.510)--(5.472,2.512)--(5.475,2.514)--(5.478,2.516)%
  --(5.481,2.518)--(5.484,2.519)--(5.487,2.521)--(5.490,2.523)--(5.493,2.525)--(5.496,2.527)%
  --(5.499,2.528)--(5.502,2.530)--(5.505,2.532)--(5.508,2.534)--(5.511,2.536)--(5.514,2.537)%
  --(5.517,2.539)--(5.520,2.541)--(5.523,2.543)--(5.526,2.545)--(5.529,2.546)--(5.532,2.548)%
  --(5.535,2.550)--(5.538,2.552)--(5.541,2.554)--(5.544,2.555)--(5.547,2.557)--(5.550,2.559)%
  --(5.553,2.561)--(5.556,2.563)--(5.559,2.564)--(5.562,2.566)--(5.565,2.568)--(5.568,2.570)%
  --(5.571,2.572)--(5.574,2.574)--(5.577,2.575)--(5.580,2.577)--(5.583,2.579)--(5.586,2.581)%
  --(5.589,2.583)--(5.591,2.584)--(5.594,2.586)--(5.597,2.588)--(5.600,2.590)--(5.603,2.592)%
  --(5.606,2.593)--(5.609,2.595)--(5.612,2.597)--(5.615,2.599)--(5.618,2.601)--(5.621,2.603)%
  --(5.624,2.604)--(5.627,2.606)--(5.630,2.608)--(5.633,2.610)--(5.636,2.612)--(5.639,2.614)%
  --(5.642,2.615)--(5.645,2.617)--(5.648,2.619)--(5.651,2.621)--(5.654,2.623)--(5.657,2.624)%
  --(5.660,2.626)--(5.663,2.628)--(5.666,2.630)--(5.669,2.632)--(5.672,2.634)--(5.675,2.635)%
  --(5.678,2.637)--(5.681,2.639)--(5.684,2.641)--(5.687,2.643)--(5.690,2.645)--(5.693,2.646)%
  --(5.696,2.648)--(5.699,2.650)--(5.702,2.652)--(5.705,2.654)--(5.708,2.656)--(5.711,2.657)%
  --(5.714,2.659)--(5.717,2.661)--(5.720,2.663)--(5.723,2.665)--(5.726,2.667)--(5.729,2.668)%
  --(5.732,2.670)--(5.735,2.672)--(5.738,2.674)--(5.741,2.676)--(5.744,2.678)--(5.747,2.679)%
  --(5.750,2.681)--(5.753,2.683)--(5.756,2.685)--(5.759,2.687)--(5.762,2.689)--(5.765,2.691)%
  --(5.768,2.692)--(5.771,2.694)--(5.774,2.696)--(5.777,2.698)--(5.780,2.700)--(5.783,2.702)%
  --(5.786,2.703)--(5.789,2.705)--(5.792,2.707)--(5.795,2.709)--(5.798,2.711)--(5.800,2.713)%
  --(5.803,2.715)--(5.806,2.716)--(5.809,2.718)--(5.812,2.720)--(5.815,2.722)--(5.818,2.724)%
  --(5.821,2.726)--(5.824,2.728)--(5.827,2.729)--(5.830,2.731)--(5.833,2.733)--(5.836,2.735)%
  --(5.839,2.737)--(5.842,2.739)--(5.845,2.740)--(5.848,2.742)--(5.851,2.744)--(5.854,2.746)%
  --(5.857,2.748)--(5.860,2.750)--(5.863,2.752)--(5.866,2.753)--(5.869,2.755)--(5.872,2.757)%
  --(5.875,2.759)--(5.878,2.761)--(5.881,2.763)--(5.884,2.765)--(5.887,2.767)--(5.890,2.768)%
  --(5.893,2.770)--(5.896,2.772)--(5.899,2.774)--(5.902,2.776)--(5.905,2.778)--(5.908,2.780)%
  --(5.911,2.781)--(5.914,2.783)--(5.917,2.785)--(5.920,2.787)--(5.923,2.789)--(5.926,2.791)%
  --(5.929,2.793)--(5.932,2.794)--(5.935,2.796)--(5.938,2.798)--(5.941,2.800)--(5.944,2.802)%
  --(5.947,2.804)--(5.950,2.806)--(5.953,2.808)--(5.956,2.809)--(5.959,2.811)--(5.962,2.813)%
  --(5.965,2.815)--(5.968,2.817)--(5.971,2.819)--(5.974,2.821)--(5.977,2.823)--(5.980,2.824)%
  --(5.983,2.826)--(5.986,2.828)--(5.989,2.830)--(5.992,2.832)--(5.995,2.834)--(5.998,2.836)%
  --(6.001,2.838)--(6.004,2.839)--(6.007,2.841)--(6.009,2.843)--(6.012,2.845)--(6.015,2.847)%
  --(6.018,2.849)--(6.021,2.851)--(6.024,2.853)--(6.027,2.854)--(6.030,2.856)--(6.033,2.858)%
  --(6.036,2.860)--(6.039,2.862)--(6.042,2.864)--(6.045,2.866)--(6.048,2.868)--(6.051,2.869)%
  --(6.054,2.871)--(6.057,2.873)--(6.060,2.875)--(6.063,2.877)--(6.066,2.879)--(6.069,2.881)%
  --(6.072,2.883)--(6.075,2.885)--(6.078,2.886)--(6.081,2.888)--(6.084,2.890)--(6.087,2.892)%
  --(6.090,2.894)--(6.093,2.896)--(6.096,2.898)--(6.099,2.900)--(6.102,2.902)--(6.105,2.903)%
  --(6.108,2.905)--(6.111,2.907)--(6.114,2.909)--(6.117,2.911)--(6.120,2.913)--(6.123,2.915)%
  --(6.126,2.917)--(6.129,2.919)--(6.132,2.920)--(6.135,2.922)--(6.138,2.924)--(6.141,2.926)%
  --(6.144,2.928)--(6.147,2.930)--(6.150,2.932)--(6.153,2.934)--(6.156,2.936)--(6.159,2.938)%
  --(6.162,2.939)--(6.165,2.941)--(6.168,2.943)--(6.171,2.945)--(6.174,2.947)--(6.177,2.949)%
  --(6.180,2.951)--(6.183,2.953)--(6.186,2.955)--(6.189,2.957)--(6.192,2.958)--(6.195,2.960)%
  --(6.198,2.962)--(6.201,2.964)--(6.204,2.966)--(6.207,2.968)--(6.210,2.970)--(6.213,2.972)%
  --(6.216,2.974)--(6.218,2.976)--(6.221,2.977)--(6.224,2.979)--(6.227,2.981)--(6.230,2.983)%
  --(6.233,2.985)--(6.236,2.987)--(6.239,2.989)--(6.242,2.991)--(6.245,2.993)--(6.248,2.995)%
  --(6.251,2.996)--(6.254,2.998)--(6.257,3.000)--(6.260,3.002)--(6.263,3.004)--(6.266,3.006)%
  --(6.269,3.008)--(6.272,3.010)--(6.275,3.012)--(6.278,3.014)--(6.281,3.016)--(6.284,3.017)%
  --(6.287,3.019)--(6.290,3.021)--(6.293,3.023)--(6.296,3.025)--(6.299,3.027)--(6.302,3.029)%
  --(6.305,3.031)--(6.308,3.033)--(6.311,3.035)--(6.314,3.037)--(6.317,3.038)--(6.320,3.040)%
  --(6.323,3.042)--(6.326,3.044)--(6.329,3.046)--(6.332,3.048)--(6.335,3.050)--(6.338,3.052)%
  --(6.341,3.054)--(6.344,3.056)--(6.347,3.058)--(6.350,3.060)--(6.353,3.061)--(6.356,3.063)%
  --(6.359,3.065)--(6.362,3.067)--(6.365,3.069)--(6.368,3.071)--(6.371,3.073)--(6.374,3.075)%
  --(6.377,3.077)--(6.380,3.079)--(6.383,3.081)--(6.386,3.083)--(6.389,3.085)--(6.392,3.086)%
  --(6.395,3.088)--(6.398,3.090)--(6.401,3.092)--(6.404,3.094)--(6.407,3.096)--(6.410,3.098)%
  --(6.413,3.100)--(6.416,3.102)--(6.419,3.104)--(6.422,3.106)--(6.425,3.108)--(6.428,3.110)%
  --(6.430,3.111)--(6.433,3.113)--(6.436,3.115)--(6.439,3.117)--(6.442,3.119)--(6.445,3.121)%
  --(6.448,3.123)--(6.451,3.125)--(6.454,3.127)--(6.457,3.129)--(6.460,3.131)--(6.463,3.133)%
  --(6.466,3.135)--(6.469,3.136)--(6.472,3.138)--(6.475,3.140)--(6.478,3.142)--(6.481,3.144)%
  --(6.484,3.146)--(6.487,3.148)--(6.490,3.150)--(6.493,3.152)--(6.496,3.154)--(6.499,3.156)%
  --(6.502,3.158)--(6.505,3.160)--(6.508,3.162)--(6.511,3.164)--(6.514,3.165)--(6.517,3.167)%
  --(6.520,3.169)--(6.523,3.171)--(6.526,3.173)--(6.529,3.175)--(6.532,3.177)--(6.535,3.179)%
  --(6.538,3.181)--(6.541,3.183)--(6.544,3.185)--(6.547,3.187)--(6.550,3.189)--(6.553,3.191)%
  --(6.556,3.193)--(6.559,3.194)--(6.562,3.196)--(6.565,3.198)--(6.568,3.200)--(6.571,3.202)%
  --(6.574,3.204)--(6.577,3.206)--(6.580,3.208)--(6.583,3.210)--(6.586,3.212)--(6.589,3.214)%
  --(6.592,3.216)--(6.595,3.218)--(6.598,3.220)--(6.601,3.222)--(6.604,3.224)--(6.607,3.226)%
  --(6.610,3.227)--(6.613,3.229)--(6.616,3.231)--(6.619,3.233)--(6.622,3.235)--(6.625,3.237)%
  --(6.628,3.239)--(6.631,3.241)--(6.634,3.243)--(6.637,3.245)--(6.639,3.247)--(6.642,3.249)%
  --(6.645,3.251)--(6.648,3.253)--(6.651,3.255)--(6.654,3.257)--(6.657,3.259)--(6.660,3.261)%
  --(6.663,3.262)--(6.666,3.264)--(6.669,3.266)--(6.672,3.268)--(6.675,3.270)--(6.678,3.272)%
  --(6.681,3.274)--(6.684,3.276)--(6.687,3.278)--(6.690,3.280)--(6.693,3.282)--(6.696,3.284)%
  --(6.699,3.286)--(6.702,3.288)--(6.705,3.290)--(6.708,3.292)--(6.711,3.294)--(6.714,3.296)%
  --(6.717,3.298)--(6.720,3.299)--(6.723,3.301)--(6.726,3.303)--(6.729,3.305)--(6.732,3.307)%
  --(6.735,3.309)--(6.738,3.311)--(6.741,3.313)--(6.744,3.315)--(6.747,3.317)--(6.750,3.319)%
  --(6.753,3.321)--(6.756,3.323)--(6.759,3.325)--(6.762,3.327)--(6.765,3.329)--(6.768,3.331)%
  --(6.771,3.333)--(6.774,3.335)--(6.777,3.337)--(6.780,3.339)--(6.783,3.341)--(6.786,3.342)%
  --(6.789,3.344)--(6.792,3.346)--(6.795,3.348)--(6.798,3.350)--(6.801,3.352)--(6.804,3.354)%
  --(6.807,3.356)--(6.810,3.358)--(6.813,3.360)--(6.816,3.362)--(6.819,3.364)--(6.822,3.366)%
  --(6.825,3.368)--(6.828,3.370)--(6.831,3.372)--(6.834,3.374)--(6.837,3.376)--(6.840,3.378)%
  --(6.843,3.380)--(6.846,3.382)--(6.848,3.384)--(6.851,3.386)--(6.854,3.388)--(6.857,3.390)%
  --(6.860,3.391)--(6.863,3.393)--(6.866,3.395)--(6.869,3.397)--(6.872,3.399)--(6.875,3.401)%
  --(6.878,3.403)--(6.881,3.405)--(6.884,3.407)--(6.887,3.409)--(6.890,3.411)--(6.893,3.413)%
  --(6.896,3.415)--(6.899,3.417)--(6.902,3.419)--(6.905,3.421)--(6.908,3.423)--(6.911,3.425)%
  --(6.914,3.427)--(6.917,3.429)--(6.920,3.431)--(6.923,3.433)--(6.926,3.435)--(6.929,3.437)%
  --(6.932,3.439)--(6.935,3.441)--(6.938,3.443)--(6.941,3.445)--(6.944,3.447)--(6.947,3.448)%
  --(6.950,3.450)--(6.953,3.452)--(6.956,3.454)--(6.959,3.456)--(6.962,3.458)--(6.965,3.460)%
  --(6.968,3.462)--(6.971,3.464)--(6.974,3.466)--(6.977,3.468)--(6.980,3.470)--(6.983,3.472)%
  --(6.986,3.474)--(6.989,3.476)--(6.992,3.478)--(6.995,3.480)--(6.998,3.482)--(7.001,3.484)%
  --(7.004,3.486)--(7.007,3.488)--(7.010,3.490)--(7.013,3.492)--(7.016,3.494)--(7.019,3.496)%
  --(7.022,3.498)--(7.025,3.500)--(7.028,3.502)--(7.031,3.504)--(7.034,3.506)--(7.037,3.508)%
  --(7.040,3.510)--(7.043,3.512)--(7.046,3.514)--(7.049,3.516)--(7.052,3.518)--(7.055,3.519)%
  --(7.057,3.521)--(7.060,3.523)--(7.063,3.525)--(7.066,3.527)--(7.069,3.529)--(7.072,3.531)%
  --(7.075,3.533)--(7.078,3.535)--(7.081,3.537)--(7.084,3.539)--(7.087,3.541)--(7.090,3.543)%
  --(7.093,3.545)--(7.096,3.547)--(7.099,3.549)--(7.102,3.551)--(7.105,3.553)--(7.108,3.555)%
  --(7.111,3.557)--(7.114,3.559)--(7.117,3.561)--(7.120,3.563)--(7.123,3.565)--(7.126,3.567)%
  --(7.129,3.569)--(7.132,3.571)--(7.135,3.573)--(7.138,3.575)--(7.141,3.577)--(7.144,3.579)%
  --(7.147,3.581)--(7.150,3.583)--(7.153,3.585)--(7.156,3.587)--(7.159,3.589)--(7.162,3.591)%
  --(7.165,3.593)--(7.168,3.595)--(7.171,3.597)--(7.174,3.599)--(7.177,3.601)--(7.180,3.603)%
  --(7.183,3.605)--(7.186,3.607)--(7.189,3.609)--(7.192,3.611)--(7.195,3.613)--(7.198,3.615)%
  --(7.201,3.617)--(7.204,3.619)--(7.207,3.621)--(7.210,3.623)--(7.213,3.625)--(7.216,3.627)%
  --(7.219,3.628)--(7.222,3.630)--(7.225,3.632)--(7.228,3.634)--(7.231,3.636)--(7.234,3.638)%
  --(7.237,3.640)--(7.240,3.642)--(7.243,3.644)--(7.246,3.646)--(7.249,3.648)--(7.252,3.650)%
  --(7.255,3.652)--(7.258,3.654)--(7.261,3.656)--(7.264,3.658)--(7.266,3.660)--(7.269,3.662)%
  --(7.272,3.664)--(7.275,3.666)--(7.278,3.668)--(7.281,3.670)--(7.284,3.672)--(7.287,3.674)%
  --(7.290,3.676)--(7.293,3.678)--(7.296,3.680)--(7.299,3.682)--(7.302,3.684)--(7.305,3.686)%
  --(7.308,3.688)--(7.311,3.690)--(7.314,3.692)--(7.317,3.694)--(7.320,3.696)--(7.323,3.698)%
  --(7.326,3.700)--(7.329,3.702)--(7.332,3.704)--(7.335,3.706)--(7.338,3.708)--(7.341,3.710)%
  --(7.344,3.712)--(7.347,3.714)--(7.350,3.716)--(7.353,3.718)--(7.356,3.720)--(7.359,3.722)%
  --(7.362,3.724)--(7.365,3.726)--(7.368,3.728)--(7.371,3.730)--(7.374,3.732)--(7.377,3.734)%
  --(7.380,3.736)--(7.383,3.738)--(7.386,3.740)--(7.389,3.742)--(7.392,3.744)--(7.395,3.746)%
  --(7.398,3.748)--(7.401,3.750)--(7.404,3.752)--(7.407,3.754)--(7.410,3.756)--(7.413,3.758)%
  --(7.416,3.760)--(7.419,3.762)--(7.422,3.764)--(7.425,3.766)--(7.428,3.768)--(7.431,3.770)%
  --(7.434,3.772)--(7.437,3.774)--(7.440,3.776)--(7.443,3.778)--(7.446,3.780)--(7.449,3.782)%
  --(7.452,3.784)--(7.455,3.786)--(7.458,3.788)--(7.461,3.790)--(7.464,3.792)--(7.467,3.794)%
  --(7.470,3.796)--(7.473,3.798)--(7.476,3.800)--(7.478,3.802)--(7.481,3.804)--(7.484,3.806)%
  --(7.487,3.808)--(7.490,3.810)--(7.493,3.812)--(7.496,3.814)--(7.499,3.816)--(7.502,3.818)%
  --(7.505,3.820)--(7.508,3.822)--(7.511,3.824)--(7.514,3.826)--(7.517,3.828)--(7.520,3.830)%
  --(7.523,3.832)--(7.526,3.834)--(7.529,3.836)--(7.532,3.838)--(7.535,3.840)--(7.538,3.842)%
  --(7.541,3.844)--(7.544,3.846)--(7.547,3.848)--(7.550,3.850)--(7.553,3.852)--(7.556,3.854)%
  --(7.559,3.856)--(7.562,3.858)--(7.565,3.860)--(7.568,3.862)--(7.571,3.864)--(7.574,3.866)%
  --(7.577,3.868)--(7.580,3.870)--(7.583,3.872)--(7.586,3.874)--(7.589,3.876)--(7.592,3.878)%
  --(7.595,3.880)--(7.598,3.882)--(7.601,3.884)--(7.604,3.886)--(7.607,3.888)--(7.610,3.890)%
  --(7.613,3.892)--(7.616,3.894)--(7.619,3.896)--(7.622,3.898)--(7.625,3.900)--(7.628,3.902)%
  --(7.631,3.904)--(7.634,3.906)--(7.637,3.908)--(7.640,3.910)--(7.643,3.912)--(7.646,3.914)%
  --(7.649,3.916)--(7.652,3.918)--(7.655,3.920)--(7.658,3.922)--(7.661,3.924)--(7.664,3.926)%
  --(7.667,3.928)--(7.670,3.930)--(7.673,3.932)--(7.676,3.934)--(7.679,3.936)--(7.682,3.938)%
  --(7.685,3.940)--(7.687,3.943)--(7.690,3.945)--(7.693,3.947)--(7.696,3.949)--(7.699,3.951)%
  --(7.702,3.953)--(7.705,3.955)--(7.708,3.957)--(7.711,3.959)--(7.714,3.961)--(7.717,3.963)%
  --(7.720,3.965)--(7.723,3.967)--(7.726,3.969)--(7.729,3.971)--(7.732,3.973)--(7.735,3.975)%
  --(7.738,3.977)--(7.741,3.979)--(7.744,3.981)--(7.747,3.983)--(7.750,3.985)--(7.753,3.987)%
  --(7.756,3.989)--(7.759,3.991)--(7.762,3.993)--(7.765,3.995)--(7.768,3.997)--(7.771,3.999)%
  --(7.774,4.001)--(7.777,4.003)--(7.780,4.005)--(7.783,4.007)--(7.786,4.009)--(7.789,4.011)%
  --(7.792,4.013)--(7.795,4.015)--(7.798,4.017)--(7.801,4.019)--(7.804,4.021)--(7.807,4.023)%
  --(7.810,4.025)--(7.813,4.027)--(7.816,4.029)--(7.819,4.031)--(7.822,4.033)--(7.825,4.035)%
  --(7.828,4.037)--(7.831,4.039)--(7.834,4.041)--(7.837,4.043)--(7.840,4.045)--(7.843,4.047)%
  --(7.846,4.049)--(7.849,4.051)--(7.852,4.053)--(7.855,4.055)--(7.858,4.057)--(7.861,4.059)%
  --(7.864,4.061)--(7.867,4.063)--(7.870,4.066)--(7.873,4.068)--(7.876,4.070)--(7.879,4.072)%
  --(7.882,4.074)--(7.885,4.076)--(7.888,4.078)--(7.891,4.080)--(7.894,4.082)--(7.896,4.084)%
  --(7.899,4.086)--(7.902,4.088)--(7.905,4.090)--(7.908,4.092)--(7.911,4.094)--(7.914,4.096)%
  --(7.917,4.098)--(7.920,4.100)--(7.923,4.102)--(7.926,4.104)--(7.929,4.106)--(7.932,4.108)%
  --(7.935,4.110)--(7.938,4.112)--(7.941,4.114)--(7.944,4.116)--(7.947,4.118)--(7.950,4.120)%
  --(7.953,4.122)--(7.956,4.124)--(7.959,4.126)--(7.962,4.128)--(7.965,4.130)--(7.968,4.132)%
  --(7.971,4.134)--(7.974,4.136)--(7.977,4.138)--(7.980,4.140)--(7.983,4.142)--(7.986,4.144)%
  --(7.989,4.146)--(7.992,4.148)--(7.995,4.150)--(7.998,4.153)--(8.001,4.155)--(8.004,4.157)%
  --(8.007,4.159)--(8.010,4.161)--(8.013,4.163)--(8.016,4.165)--(8.019,4.167)--(8.022,4.169)%
  --(8.025,4.171)--(8.028,4.173)--(8.031,4.175)--(8.034,4.177)--(8.037,4.179)--(8.040,4.181)%
  --(8.043,4.183)--(8.046,4.185)--(8.049,4.187)--(8.052,4.189)--(8.055,4.191)--(8.058,4.193)%
  --(8.061,4.195)--(8.064,4.197)--(8.067,4.199)--(8.070,4.201)--(8.073,4.203)--(8.076,4.205)%
  --(8.079,4.207)--(8.082,4.209)--(8.085,4.211)--(8.088,4.213)--(8.091,4.215)--(8.094,4.217)%
  --(8.097,4.219)--(8.100,4.221)--(8.103,4.223)--(8.105,4.226)--(8.108,4.228)--(8.111,4.230)%
  --(8.114,4.232)--(8.117,4.234)--(8.120,4.236)--(8.123,4.238)--(8.126,4.240)--(8.129,4.242)%
  --(8.132,4.244)--(8.135,4.246)--(8.138,4.248)--(8.141,4.250)--(8.144,4.252)--(8.147,4.254)%
  --(8.150,4.256)--(8.153,4.258)--(8.156,4.260)--(8.159,4.262)--(8.162,4.264)--(8.165,4.266)%
  --(8.168,4.268)--(8.171,4.270)--(8.174,4.272)--(8.177,4.274)--(8.180,4.276)--(8.183,4.278)%
  --(8.186,4.280)--(8.189,4.282)--(8.192,4.284)--(8.195,4.286)--(8.198,4.288)--(8.201,4.291)%
  --(8.204,4.293)--(8.207,4.295)--(8.210,4.297)--(8.213,4.299)--(8.216,4.301)--(8.219,4.303)%
  --(8.222,4.305)--(8.225,4.307)--(8.228,4.309)--(8.231,4.311)--(8.234,4.313)--(8.237,4.315)%
  --(8.240,4.317)--(8.243,4.319)--(8.246,4.321)--(8.249,4.323)--(8.252,4.325)--(8.255,4.327)%
  --(8.258,4.329)--(8.261,4.331)--(8.264,4.333)--(8.267,4.335)--(8.270,4.337)--(8.273,4.339)%
  --(8.276,4.341)--(8.279,4.343)--(8.282,4.345)--(8.285,4.347)--(8.288,4.350)--(8.291,4.352)%
  --(8.294,4.354)--(8.297,4.356)--(8.300,4.358)--(8.303,4.360)--(8.306,4.362)--(8.309,4.364)%
  --(8.312,4.366)--(8.314,4.368)--(8.317,4.370)--(8.320,4.372)--(8.323,4.374)--(8.326,4.376)%
  --(8.329,4.378)--(8.332,4.380)--(8.335,4.382)--(8.338,4.384)--(8.341,4.386)--(8.344,4.388)%
  --(8.347,4.390)--(8.350,4.392)--(8.353,4.394)--(8.356,4.396)--(8.359,4.398)--(8.362,4.400)%
  --(8.365,4.403)--(8.368,4.405)--(8.371,4.407)--(8.374,4.409)--(8.377,4.411)--(8.380,4.413)%
  --(8.383,4.415)--(8.386,4.417)--(8.389,4.419)--(8.392,4.421)--(8.395,4.423)--(8.398,4.425)%
  --(8.401,4.427)--(8.404,4.429)--(8.407,4.431)--(8.410,4.433)--(8.413,4.435)--(8.416,4.437)%
  --(8.419,4.439)--(8.422,4.441)--(8.425,4.443)--(8.428,4.445)--(8.431,4.447)--(8.434,4.449)%
  --(8.437,4.451)--(8.440,4.454)--(8.443,4.456)--(8.446,4.458)--(8.449,4.460)--(8.452,4.462)%
  --(8.455,4.464)--(8.458,4.466)--(8.461,4.468)--(8.464,4.470)--(8.467,4.472)--(8.470,4.474)%
  --(8.473,4.476)--(8.476,4.478)--(8.479,4.480)--(8.482,4.482)--(8.485,4.484)--(8.488,4.486)%
  --(8.491,4.488)--(8.494,4.490)--(8.497,4.492)--(8.500,4.494)--(8.503,4.496)--(8.506,4.498)%
  --(8.509,4.500)--(8.512,4.503)--(8.515,4.505)--(8.518,4.507)--(8.521,4.509)--(8.523,4.511)%
  --(8.526,4.513)--(8.529,4.515)--(8.532,4.517)--(8.535,4.519)--(8.538,4.521)--(8.541,4.523)%
  --(8.544,4.525)--(8.547,4.527)--(8.550,4.529)--(8.553,4.531)--(8.556,4.533)--(8.559,4.535)%
  --(8.562,4.537)--(8.565,4.539)--(8.568,4.541)--(8.571,4.543)--(8.574,4.545)--(8.577,4.548)%
  --(8.580,4.550)--(8.583,4.552)--(8.586,4.554)--(8.589,4.556)--(8.592,4.558)--(8.595,4.560)%
  --(8.598,4.562)--(8.601,4.564)--(8.604,4.566)--(8.607,4.568)--(8.610,4.570)--(8.613,4.572)%
  --(8.616,4.574)--(8.619,4.576)--(8.622,4.578)--(8.625,4.580)--(8.628,4.582)--(8.631,4.584)%
  --(8.634,4.586)--(8.637,4.588)--(8.640,4.591)--(8.643,4.593)--(8.646,4.595)--(8.649,4.597)%
  --(8.652,4.599)--(8.655,4.601)--(8.658,4.603)--(8.661,4.605)--(8.664,4.607)--(8.667,4.609)%
  --(8.670,4.611)--(8.673,4.613)--(8.676,4.615)--(8.679,4.617)--(8.682,4.619)--(8.685,4.621)%
  --(8.688,4.623)--(8.691,4.625)--(8.694,4.627)--(8.697,4.629)--(8.700,4.631)--(8.703,4.634)%
  --(8.706,4.636)--(8.709,4.638)--(8.712,4.640)--(8.715,4.642)--(8.718,4.644)--(8.721,4.646)%
  --(8.724,4.648)--(8.727,4.650)--(8.730,4.652)--(8.733,4.654)--(8.735,4.656)--(8.738,4.658)%
  --(8.741,4.660)--(8.744,4.662)--(8.747,4.664)--(8.750,4.666)--(8.753,4.668)--(8.756,4.670)%
  --(8.759,4.672)--(8.762,4.675)--(8.765,4.677)--(8.768,4.679)--(8.771,4.681)--(8.774,4.683)%
  --(8.777,4.685)--(8.780,4.687)--(8.783,4.689)--(8.786,4.691)--(8.789,4.693)--(8.792,4.695)%
  --(8.795,4.697)--(8.798,4.699)--(8.801,4.701)--(8.804,4.703)--(8.807,4.705)--(8.810,4.707)%
  --(8.813,4.709)--(8.816,4.711)--(8.819,4.714)--(8.822,4.716)--(8.825,4.718)--(8.828,4.720)%
  --(8.831,4.722)--(8.834,4.724)--(8.837,4.726)--(8.840,4.728)--(8.843,4.730)--(8.846,4.732)%
  --(8.849,4.734)--(8.852,4.736)--(8.855,4.738)--(8.858,4.740)--(8.861,4.742)--(8.864,4.744)%
  --(8.867,4.746)--(8.870,4.748)--(8.873,4.750)--(8.876,4.753)--(8.879,4.755)--(8.882,4.757)%
  --(8.885,4.759)--(8.888,4.761)--(8.891,4.763)--(8.894,4.765)--(8.897,4.767)--(8.900,4.769)%
  --(8.903,4.771)--(8.906,4.773)--(8.909,4.775)--(8.912,4.777)--(8.915,4.779)--(8.918,4.781)%
  --(8.921,4.783)--(8.924,4.785)--(8.927,4.787)--(8.930,4.790)--(8.933,4.792)--(8.936,4.794)%
  --(8.939,4.796)--(8.942,4.798)--(8.944,4.800)--(8.947,4.802)--(8.950,4.804)--(8.953,4.806)%
  --(8.956,4.808)--(8.959,4.810)--(8.962,4.812)--(8.965,4.814)--(8.968,4.816)--(8.971,4.818)%
  --(8.974,4.820)--(8.977,4.822)--(8.980,4.824)--(8.983,4.827)--(8.986,4.829)--(8.989,4.831)%
  --(8.992,4.833)--(8.995,4.835)--(8.998,4.837)--(9.001,4.839)--(9.004,4.841)--(9.007,4.843)%
  --(9.010,4.845)--(9.013,4.847)--(9.016,4.849)--(9.019,4.851)--(9.022,4.853)--(9.025,4.855)%
  --(9.028,4.857)--(9.031,4.859)--(9.034,4.861)--(9.037,4.864)--(9.040,4.866)--(9.043,4.868)%
  --(9.046,4.870)--(9.049,4.872)--(9.052,4.874)--(9.055,4.876)--(9.058,4.878)--(9.061,4.880)%
  --(9.064,4.882)--(9.067,4.884)--(9.070,4.886)--(9.073,4.888)--(9.076,4.890)--(9.079,4.892)%
  --(9.082,4.894)--(9.085,4.896)--(9.088,4.899)--(9.091,4.901)--(9.094,4.903)--(9.097,4.905)%
  --(9.100,4.907)--(9.103,4.909)--(9.106,4.911)--(9.109,4.913)--(9.112,4.915)--(9.115,4.917)%
  --(9.118,4.919)--(9.121,4.921)--(9.124,4.923)--(9.127,4.925)--(9.130,4.927)--(9.133,4.929)%
  --(9.136,4.931)--(9.139,4.934)--(9.142,4.936)--(9.145,4.938)--(9.148,4.940)--(9.151,4.942)%
  --(9.153,4.944)--(9.156,4.946)--(9.159,4.948)--(9.162,4.950)--(9.165,4.952)--(9.168,4.954)%
  --(9.171,4.956)--(9.174,4.958)--(9.177,4.960)--(9.180,4.962)--(9.183,4.964)--(9.186,4.967)%
  --(9.189,4.969)--(9.192,4.971)--(9.195,4.973)--(9.198,4.975)--(9.201,4.977)--(9.204,4.979)%
  --(9.207,4.981)--(9.210,4.983)--(9.213,4.985)--(9.216,4.987)--(9.219,4.989)--(9.222,4.991)%
  --(9.225,4.993)--(9.228,4.995)--(9.231,4.997)--(9.234,5.000)--(9.237,5.002)--(9.240,5.004)%
  --(9.243,5.006)--(9.246,5.008)--(9.249,5.010)--(9.252,5.012)--(9.255,5.014)--(9.258,5.016)%
  --(9.261,5.018)--(9.264,5.020)--(9.267,5.022)--(9.270,5.024)--(9.273,5.026)--(9.276,5.028)%
  --(9.279,5.030)--(9.282,5.033)--(9.285,5.035)--(9.288,5.037)--(9.291,5.039)--(9.294,5.041)%
  --(9.297,5.043)--(9.300,5.045)--(9.303,5.047)--(9.306,5.049)--(9.309,5.051)--(9.312,5.053)%
  --(9.315,5.055)--(9.318,5.057)--(9.321,5.059)--(9.324,5.061)--(9.327,5.063)--(9.330,5.066)%
  --(9.333,5.068)--(9.336,5.070)--(9.339,5.072)--(9.342,5.074)--(9.345,5.076)--(9.348,5.078)%
  --(9.351,5.080)--(9.354,5.082)--(9.357,5.084)--(9.360,5.086)--(9.362,5.088)--(9.365,5.090)%
  --(9.368,5.092)--(9.371,5.094)--(9.374,5.097)--(9.377,5.099)--(9.380,5.101)--(9.383,5.103)%
  --(9.386,5.105)--(9.389,5.107)--(9.392,5.109)--(9.395,5.111)--(9.398,5.113)--(9.401,5.115)%
  --(9.404,5.117)--(9.407,5.119)--(9.410,5.121)--(9.413,5.123)--(9.416,5.125)--(9.419,5.128)%
  --(9.422,5.130)--(9.425,5.132)--(9.428,5.134)--(9.431,5.136)--(9.434,5.138)--(9.437,5.140)%
  --(9.440,5.142)--(9.443,5.144)--(9.446,5.146)--(9.449,5.148)--(9.452,5.150)--(9.455,5.152)%
  --(9.458,5.154)--(9.461,5.156)--(9.464,5.159)--(9.467,5.161)--(9.470,5.163)--(9.473,5.165)%
  --(9.476,5.167)--(9.479,5.169)--(9.482,5.171)--(9.485,5.173)--(9.488,5.175)--(9.491,5.177)%
  --(9.494,5.179)--(9.497,5.181)--(9.500,5.183)--(9.503,5.185)--(9.506,5.187)--(9.509,5.190)%
  --(9.512,5.192)--(9.515,5.194)--(9.518,5.196)--(9.521,5.198)--(9.524,5.200)--(9.527,5.202)%
  --(9.530,5.204)--(9.533,5.206)--(9.536,5.208)--(9.539,5.210)--(9.542,5.212)--(9.545,5.214)%
  --(9.548,5.216)--(9.551,5.218)--(9.554,5.221)--(9.557,5.223)--(9.560,5.225)--(9.563,5.227)%
  --(9.566,5.229)--(9.569,5.231)--(9.571,5.233)--(9.574,5.235)--(9.577,5.237)--(9.580,5.239)%
  --(9.583,5.241)--(9.586,5.243)--(9.589,5.245)--(9.592,5.247)--(9.595,5.250)--(9.598,5.252)%
  --(9.601,5.254)--(9.604,5.256)--(9.607,5.258)--(9.610,5.260)--(9.613,5.262)--(9.616,5.264)%
  --(9.619,5.266)--(9.622,5.268)--(9.625,5.270)--(9.628,5.272)--(9.631,5.274)--(9.634,5.276)%
  --(9.637,5.279)--(9.640,5.281)--(9.643,5.283)--(9.646,5.285)--(9.649,5.287)--(9.652,5.289)%
  --(9.655,5.291)--(9.658,5.293)--(9.661,5.295)--(9.664,5.297)--(9.667,5.299)--(9.670,5.301)%
  --(9.673,5.303)--(9.676,5.305)--(9.679,5.308)--(9.682,5.310)--(9.685,5.312)--(9.688,5.314)%
  --(9.691,5.316)--(9.694,5.318)--(9.697,5.320)--(9.700,5.322)--(9.703,5.324)--(9.706,5.326)%
  --(9.709,5.328)--(9.712,5.330)--(9.715,5.332)--(9.718,5.334)--(9.721,5.337)--(9.724,5.339)%
  --(9.727,5.341)--(9.730,5.343)--(9.733,5.345)--(9.736,5.347)--(9.739,5.349)--(9.742,5.351)%
  --(9.745,5.353)--(9.748,5.355)--(9.751,5.357)--(9.754,5.359)--(9.757,5.361)--(9.760,5.363)%
  --(9.763,5.366)--(9.766,5.368)--(9.769,5.370)--(9.772,5.372)--(9.775,5.374)--(9.778,5.376)%
  --(9.780,5.378)--(9.783,5.380)--(9.786,5.382)--(9.789,5.384)--(9.792,5.386)--(9.795,5.388)%
  --(9.798,5.390)--(9.801,5.392)--(9.804,5.395)--(9.807,5.397)--(9.810,5.399)--(9.813,5.401)%
  --(9.816,5.403)--(9.819,5.405)--(9.822,5.407)--(9.825,5.409)--(9.828,5.411)--(9.831,5.413)%
  --(9.834,5.415)--(9.837,5.417)--(9.840,5.419)--(9.843,5.422)--(9.846,5.424)--(9.849,5.426)%
  --(9.852,5.428)--(9.855,5.430)--(9.858,5.432)--(9.861,5.434)--(9.864,5.436)--(9.867,5.438)%
  --(9.870,5.440)--(9.873,5.442)--(9.876,5.444)--(9.879,5.446)--(9.882,5.448)--(9.885,5.451)%
  --(9.888,5.453)--(9.891,5.455)--(9.894,5.457)--(9.897,5.459)--(9.900,5.461)--(9.903,5.463)%
  --(9.906,5.465)--(9.909,5.467)--(9.912,5.469)--(9.915,5.471)--(9.918,5.473)--(9.921,5.475)%
  --(9.924,5.478)--(9.927,5.480)--(9.930,5.482)--(9.933,5.484)--(9.936,5.486)--(9.939,5.488)%
  --(9.942,5.490)--(9.945,5.492)--(9.948,5.494)--(9.951,5.496)--(9.954,5.498)--(9.957,5.500)%
  --(9.960,5.502)--(9.963,5.505)--(9.966,5.507)--(9.969,5.509)--(9.972,5.511)--(9.975,5.513)%
  --(9.978,5.515)--(9.981,5.517)--(9.984,5.519)--(9.987,5.521)--(9.990,5.523)--(9.992,5.525)%
  --(9.995,5.527)--(9.998,5.529)--(10.001,5.531)--(10.004,5.534)--(10.007,5.536)--(10.010,5.538)%
  --(10.013,5.540)--(10.016,5.542)--(10.019,5.544)--(10.022,5.546)--(10.025,5.548)--(10.028,5.550)%
  --(10.031,5.552)--(10.034,5.554)--(10.037,5.556)--(10.040,5.558)--(10.043,5.561)--(10.046,5.563)%
  --(10.049,5.565)--(10.052,5.567)--(10.055,5.569)--(10.058,5.571)--(10.061,5.573)--(10.064,5.575)%
  --(10.067,5.577)--(10.070,5.579)--(10.073,5.581)--(10.076,5.583)--(10.079,5.586)--(10.082,5.588)%
  --(10.085,5.590)--(10.088,5.592)--(10.091,5.594)--(10.094,5.596)--(10.097,5.598)--(10.100,5.600)%
  --(10.103,5.602)--(10.106,5.604)--(10.109,5.606)--(10.112,5.608)--(10.115,5.610)--(10.118,5.613)%
  --(10.121,5.615)--(10.124,5.617)--(10.127,5.619)--(10.130,5.621)--(10.133,5.623)--(10.136,5.625)%
  --(10.139,5.627)--(10.142,5.629)--(10.145,5.631)--(10.148,5.633)--(10.151,5.635)--(10.154,5.637)%
  --(10.157,5.640)--(10.160,5.642)--(10.163,5.644)--(10.166,5.646)--(10.169,5.648)--(10.172,5.650)%
  --(10.175,5.652)--(10.178,5.654)--(10.181,5.656)--(10.184,5.658)--(10.187,5.660)--(10.190,5.662)%
  --(10.193,5.664)--(10.196,5.667)--(10.199,5.669)--(10.201,5.671)--(10.204,5.673)--(10.207,5.675)%
  --(10.210,5.677)--(10.213,5.679)--(10.216,5.681)--(10.219,5.683)--(10.222,5.685)--(10.225,5.687)%
  --(10.228,5.689)--(10.231,5.692)--(10.234,5.694)--(10.237,5.696)--(10.240,5.698)--(10.243,5.700)%
  --(10.246,5.702)--(10.249,5.704)--(10.252,5.706)--(10.255,5.708)--(10.258,5.710)--(10.261,5.712)%
  --(10.264,5.714)--(10.267,5.716)--(10.270,5.719)--(10.273,5.721)--(10.276,5.723)--(10.279,5.725)%
  --(10.282,5.727)--(10.285,5.729)--(10.288,5.731)--(10.291,5.733)--(10.294,5.735)--(10.297,5.737)%
  --(10.300,5.739)--(10.303,5.741)--(10.306,5.744)--(10.309,5.746)--(10.312,5.748)--(10.315,5.750)%
  --(10.318,5.752)--(10.321,5.754)--(10.324,5.756)--(10.327,5.758)--(10.330,5.760)--(10.333,5.762)%
  --(10.336,5.764)--(10.339,5.766)--(10.342,5.769)--(10.345,5.771)--(10.348,5.773)--(10.351,5.775)%
  --(10.354,5.777)--(10.357,5.779)--(10.360,5.781)--(10.363,5.783)--(10.366,5.785)--(10.369,5.787)%
  --(10.372,5.789)--(10.375,5.791)--(10.378,5.793)--(10.381,5.796)--(10.384,5.798)--(10.387,5.800)%
  --(10.390,5.802)--(10.393,5.804)--(10.396,5.806)--(10.399,5.808)--(10.402,5.810)--(10.405,5.812)%
  --(10.408,5.814)--(10.410,5.816)--(10.413,5.818)--(10.416,5.821)--(10.419,5.823)--(10.422,5.825)%
  --(10.425,5.827)--(10.428,5.829)--(10.431,5.831)--(10.434,5.833)--(10.437,5.835)--(10.440,5.837)%
  --(10.443,5.839)--(10.446,5.841)--(10.449,5.843)--(10.452,5.846)--(10.455,5.848)--(10.458,5.850)%
  --(10.461,5.852)--(10.464,5.854)--(10.467,5.856)--(10.470,5.858)--(10.473,5.860)--(10.476,5.862)%
  --(10.479,5.864)--(10.482,5.866)--(10.485,5.868)--(10.488,5.871)--(10.491,5.873)--(10.494,5.875)%
  --(10.497,5.877)--(10.500,5.879)--(10.503,5.881)--(10.506,5.883)--(10.509,5.885)--(10.512,5.887)%
  --(10.515,5.889)--(10.518,5.891)--(10.521,5.893)--(10.524,5.896)--(10.527,5.898)--(10.530,5.900)%
  --(10.533,5.902)--(10.536,5.904)--(10.539,5.906)--(10.542,5.908)--(10.545,5.910)--(10.548,5.912)%
  --(10.551,5.914)--(10.554,5.916)--(10.557,5.918)--(10.560,5.921)--(10.563,5.923)--(10.566,5.925)%
  --(10.569,5.927)--(10.572,5.929)--(10.575,5.931)--(10.578,5.933)--(10.581,5.935)--(10.584,5.937)%
  --(10.587,5.939)--(10.590,5.941)--(10.593,5.943)--(10.596,5.946)--(10.599,5.948)--(10.602,5.950)%
  --(10.605,5.952)--(10.608,5.954)--(10.611,5.956)--(10.614,5.958)--(10.617,5.960)--(10.619,5.962)%
  --(10.622,5.964)--(10.625,5.966)--(10.628,5.969)--(10.631,5.971)--(10.634,5.973)--(10.637,5.975)%
  --(10.640,5.977)--(10.643,5.979)--(10.646,5.981)--(10.649,5.983)--(10.652,5.985)--(10.655,5.987)%
  --(10.658,5.989)--(10.661,5.991)--(10.664,5.994)--(10.667,5.996)--(10.670,5.998)--(10.673,6.000)%
  --(10.676,6.002)--(10.679,6.004)--(10.682,6.006)--(10.685,6.008)--(10.688,6.010)--(10.691,6.012)%
  --(10.694,6.014)--(10.697,6.016)--(10.700,6.019)--(10.703,6.021)--(10.706,6.023)--(10.709,6.025)%
  --(10.712,6.027)--(10.715,6.029)--(10.718,6.031)--(10.721,6.033)--(10.724,6.035)--(10.727,6.037)%
  --(10.730,6.039)--(10.733,6.042)--(10.736,6.044)--(10.739,6.046)--(10.742,6.048)--(10.745,6.050)%
  --(10.748,6.052)--(10.751,6.054)--(10.754,6.056)--(10.757,6.058)--(10.760,6.060)--(10.763,6.062)%
  --(10.766,6.064)--(10.769,6.067)--(10.772,6.069)--(10.775,6.071)--(10.778,6.073)--(10.781,6.075)%
  --(10.784,6.077)--(10.787,6.079)--(10.790,6.081)--(10.793,6.083)--(10.796,6.085)--(10.799,6.087)%
  --(10.802,6.090)--(10.805,6.092)--(10.808,6.094)--(10.811,6.096)--(10.814,6.098)--(10.817,6.100)%
  --(10.820,6.102)--(10.823,6.104)--(10.826,6.106)--(10.828,6.108)--(10.831,6.110)--(10.834,6.112)%
  --(10.837,6.115)--(10.840,6.117)--(10.843,6.119)--(10.846,6.121)--(10.849,6.123)--(10.852,6.125)%
  --(10.855,6.127)--(10.858,6.129)--(10.861,6.131)--(10.864,6.133)--(10.867,6.135)--(10.870,6.138)%
  --(10.873,6.140)--(10.876,6.142)--(10.879,6.144)--(10.882,6.146)--(10.885,6.148)--(10.888,6.150)%
  --(10.891,6.152)--(10.894,6.154)--(10.897,6.156)--(10.900,6.158)--(10.903,6.160)--(10.906,6.163)%
  --(10.909,6.165)--(10.912,6.167)--(10.915,6.169)--(10.918,6.171)--(10.921,6.173)--(10.924,6.175)%
  --(10.927,6.177)--(10.930,6.179)--(10.933,6.181)--(10.936,6.183)--(10.939,6.186)--(10.942,6.188)%
  --(10.945,6.190)--(10.948,6.192)--(10.951,6.194)--(10.954,6.196)--(10.957,6.198)--(10.960,6.200)%
  --(10.963,6.202)--(10.966,6.204)--(10.969,6.206)--(10.972,6.209)--(10.975,6.211)--(10.978,6.213)%
  --(10.981,6.215)--(10.984,6.217)--(10.987,6.219)--(10.990,6.221)--(10.993,6.223)--(10.996,6.225)%
  --(10.999,6.227)--(11.002,6.229)--(11.005,6.232)--(11.008,6.234)--(11.011,6.236)--(11.014,6.238)%
  --(11.017,6.240)--(11.020,6.242)--(11.023,6.244)--(11.026,6.246)--(11.029,6.248)--(11.032,6.250)%
  --(11.035,6.252)--(11.037,6.254)--(11.040,6.257)--(11.043,6.259)--(11.046,6.261)--(11.049,6.263)%
  --(11.052,6.265)--(11.055,6.267)--(11.058,6.269)--(11.061,6.271)--(11.064,6.273)--(11.067,6.275)%
  --(11.070,6.277)--(11.073,6.280)--(11.076,6.282)--(11.079,6.284)--(11.082,6.286)--(11.085,6.288)%
  --(11.088,6.290)--(11.091,6.292)--(11.094,6.294)--(11.097,6.296)--(11.100,6.298)--(11.103,6.300)%
  --(11.106,6.303)--(11.109,6.305)--(11.112,6.307)--(11.115,6.309)--(11.118,6.311)--(11.121,6.313)%
  --(11.124,6.315)--(11.127,6.317)--(11.130,6.319)--(11.133,6.321)--(11.136,6.323)--(11.139,6.326)%
  --(11.142,6.328)--(11.145,6.330)--(11.148,6.332)--(11.151,6.334)--(11.154,6.336)--(11.157,6.338)%
  --(11.160,6.340)--(11.163,6.342)--(11.166,6.344)--(11.169,6.346)--(11.172,6.349)--(11.175,6.351)%
  --(11.178,6.353)--(11.181,6.355)--(11.184,6.357)--(11.187,6.359)--(11.190,6.361)--(11.193,6.363)%
  --(11.196,6.365)--(11.199,6.367)--(11.202,6.369)--(11.205,6.372)--(11.208,6.374)--(11.211,6.376)%
  --(11.214,6.378)--(11.217,6.380)--(11.220,6.382)--(11.223,6.384)--(11.226,6.386)--(11.229,6.388)%
  --(11.232,6.390)--(11.235,6.392)--(11.238,6.395)--(11.241,6.397)--(11.244,6.399)--(11.247,6.401)%
  --(11.249,6.403)--(11.252,6.405)--(11.255,6.407)--(11.258,6.409)--(11.261,6.411)--(11.264,6.413)%
  --(11.267,6.415)--(11.270,6.418)--(11.273,6.420)--(11.276,6.422)--(11.279,6.424)--(11.282,6.426)%
  --(11.285,6.428)--(11.288,6.430)--(11.291,6.432)--(11.294,6.434)--(11.297,6.436)--(11.300,6.438)%
  --(11.303,6.441)--(11.306,6.443)--(11.309,6.445)--(11.312,6.447)--(11.315,6.449)--(11.318,6.451)%
  --(11.321,6.453)--(11.324,6.455)--(11.327,6.457)--(11.330,6.459)--(11.333,6.462)--(11.336,6.464)%
  --(11.339,6.466)--(11.342,6.468)--(11.345,6.470)--(11.348,6.472)--(11.351,6.474)--(11.354,6.476)%
  --(11.357,6.478)--(11.360,6.480)--(11.363,6.482)--(11.366,6.485)--(11.369,6.487)--(11.372,6.489)%
  --(11.375,6.491)--(11.378,6.493)--(11.381,6.495)--(11.384,6.497)--(11.387,6.499)--(11.390,6.501)%
  --(11.393,6.503)--(11.396,6.505)--(11.399,6.508)--(11.402,6.510)--(11.405,6.512)--(11.408,6.514)%
  --(11.411,6.516)--(11.414,6.518)--(11.417,6.520)--(11.420,6.522)--(11.423,6.524)--(11.426,6.526)%
  --(11.429,6.528)--(11.432,6.531)--(11.435,6.533)--(11.438,6.535)--(11.441,6.537)--(11.444,6.539)%
  --(11.447,6.541)--(11.450,6.543)--(11.453,6.545)--(11.456,6.547)--(11.458,6.549)--(11.461,6.552)%
  --(11.464,6.554)--(11.467,6.556)--(11.470,6.558)--(11.473,6.560)--(11.476,6.562)--(11.479,6.564)%
  --(11.482,6.566)--(11.485,6.568)--(11.488,6.570)--(11.491,6.572)--(11.494,6.575)--(11.497,6.577)%
  --(11.500,6.579)--(11.503,6.581)--(11.506,6.583)--(11.509,6.585)--(11.512,6.587)--(11.515,6.589)%
  --(11.518,6.591)--(11.521,6.593)--(11.524,6.595)--(11.527,6.598)--(11.530,6.600)--(11.533,6.602)%
  --(11.536,6.604)--(11.539,6.606)--(11.542,6.608)--(11.545,6.610)--(11.548,6.612)--(11.551,6.614)%
  --(11.554,6.616)--(11.557,6.619)--(11.560,6.621)--(11.563,6.623)--(11.566,6.625)--(11.569,6.627)%
  --(11.572,6.629)--(11.575,6.631)--(11.578,6.633)--(11.581,6.635)--(11.584,6.637)--(11.587,6.639)%
  --(11.590,6.642)--(11.593,6.644)--(11.596,6.646)--(11.599,6.648)--(11.602,6.650)--(11.605,6.652)%
  --(11.608,6.654)--(11.611,6.656)--(11.614,6.658)--(11.617,6.660)--(11.620,6.663)--(11.623,6.665)%
  --(11.626,6.667)--(11.629,6.669)--(11.632,6.671)--(11.635,6.673)--(11.638,6.675)--(11.641,6.677)%
  --(11.644,6.679)--(11.647,6.681)--(11.650,6.683)--(11.653,6.686)--(11.656,6.688)--(11.659,6.690)%
  --(11.662,6.692)--(11.665,6.694)--(11.667,6.696)--(11.670,6.698)--(11.673,6.700)--(11.676,6.702)%
  --(11.679,6.704)--(11.682,6.707)--(11.685,6.709)--(11.688,6.711)--(11.691,6.713)--(11.694,6.715)%
  --(11.697,6.717)--(11.700,6.719)--(11.703,6.721)--(11.706,6.723)--(11.709,6.725)--(11.712,6.727)%
  --(11.715,6.730)--(11.718,6.732)--(11.721,6.734)--(11.724,6.736)--(11.727,6.738)--(11.730,6.740)%
  --(11.733,6.742)--(11.736,6.744)--(11.739,6.746)--(11.742,6.748)--(11.745,6.751)--(11.748,6.753)%
  --(11.751,6.755)--(11.754,6.757)--(11.757,6.759)--(11.760,6.761)--(11.763,6.763)--(11.766,6.765)%
  --(11.769,6.767)--(11.772,6.769)--(11.775,6.771)--(11.778,6.774)--(11.781,6.776)--(11.784,6.778)%
  --(11.787,6.780)--(11.790,6.782)--(11.793,6.784)--(11.796,6.786)--(11.799,6.788)--(11.802,6.790)%
  --(11.805,6.792)--(11.808,6.795)--(11.811,6.797)--(11.814,6.799)--(11.817,6.801)--(11.820,6.803)%
  --(11.823,6.805)--(11.826,6.807)--(11.829,6.809)--(11.832,6.811)--(11.835,6.813)--(11.838,6.815)%
  --(11.841,6.818)--(11.844,6.820)--(11.847,6.822)--(11.850,6.824)--(11.853,6.826)--(11.856,6.828)%
  --(11.859,6.830)--(11.862,6.832)--(11.865,6.834)--(11.868,6.836)--(11.871,6.839)--(11.874,6.841)%
  --(11.876,6.843)--(11.879,6.845)--(11.882,6.847)--(11.885,6.849)--(11.888,6.851)--(11.891,6.853)%
  --(11.894,6.855)--(11.897,6.857)--(11.900,6.860)--(11.903,6.862)--(11.906,6.864)--(11.909,6.866)%
  --(11.912,6.868)--(11.915,6.870)--(11.918,6.872)--(11.921,6.874)--(11.924,6.876)--(11.927,6.878)%
  --(11.930,6.881)--(11.933,6.883)--(11.936,6.885)--(11.939,6.887)--(11.942,6.889)--(11.945,6.891)%
  --(11.948,6.893)--(11.951,6.895)--(11.954,6.897)--(11.957,6.899)--(11.960,6.901)--(11.963,6.904)%
  --(11.966,6.906)--(11.969,6.908)--(11.972,6.910)--(11.975,6.912)--(11.978,6.914)--(11.981,6.916)%
  --(11.984,6.918)--(11.987,6.920)--(11.990,6.922)--(11.993,6.925)--(11.996,6.927)--(11.999,6.929)%
  --(12.002,6.931)--(12.005,6.933)--(12.008,6.935)--(12.011,6.937)--(12.014,6.939)--(12.017,6.941)%
  --(12.020,6.943)--(12.023,6.946)--(12.026,6.948)--(12.029,6.950)--(12.032,6.952)--(12.035,6.954)%
  --(12.038,6.956)--(12.041,6.958)--(12.044,6.960)--(12.047,6.962)--(12.050,6.964)--(12.053,6.967)%
  --(12.056,6.969)--(12.059,6.971)--(12.062,6.973)--(12.065,6.975)--(12.068,6.977)--(12.071,6.979)%
  --(12.074,6.981)--(12.077,6.983)--(12.080,6.985)--(12.083,6.987)--(12.085,6.990)--(12.088,6.992)%
  --(12.091,6.994)--(12.094,6.996)--(12.097,6.998)--(12.100,7.000)--(12.103,7.002)--(12.106,7.004)%
  --(12.109,7.006)--(12.112,7.008)--(12.115,7.011)--(12.118,7.013)--(12.121,7.015)--(12.124,7.017)%
  --(12.127,7.019)--(12.130,7.021)--(12.133,7.023)--(12.136,7.025)--(12.139,7.027)--(12.142,7.029)%
  --(12.145,7.032)--(12.148,7.034)--(12.151,7.036)--(12.154,7.038)--(12.157,7.040)--(12.160,7.042)%
  --(12.163,7.044)--(12.166,7.046)--(12.169,7.048)--(12.172,7.050)--(12.175,7.053)--(12.178,7.055)%
  --(12.181,7.057)--(12.184,7.059)--(12.187,7.061)--(12.190,7.063)--(12.193,7.065)--(12.196,7.067)%
  --(12.199,7.069)--(12.202,7.071)--(12.205,7.074)--(12.208,7.076)--(12.211,7.078)--(12.214,7.080)%
  --(12.217,7.082)--(12.220,7.084)--(12.223,7.086)--(12.226,7.088)--(12.229,7.090)--(12.232,7.092)%
  --(12.235,7.095)--(12.238,7.097)--(12.241,7.099)--(12.244,7.101)--(12.247,7.103)--(12.250,7.105)%
  --(12.253,7.107)--(12.256,7.109)--(12.259,7.111)--(12.262,7.113)--(12.265,7.116)--(12.268,7.118)%
  --(12.271,7.120)--(12.274,7.122)--(12.277,7.124)--(12.280,7.126)--(12.283,7.128)--(12.286,7.130)%
  --(12.289,7.132)--(12.292,7.134)--(12.295,7.137)--(12.297,7.139)--(12.300,7.141)--(12.303,7.143)%
  --(12.306,7.145)--(12.309,7.147)--(12.312,7.149)--(12.315,7.151)--(12.318,7.153)--(12.321,7.155)%
  --(12.324,7.158)--(12.327,7.160)--(12.330,7.162)--(12.333,7.164)--(12.336,7.166)--(12.339,7.168)%
  --(12.342,7.170)--(12.345,7.172)--(12.348,7.174)--(12.351,7.176)--(12.354,7.179)--(12.357,7.181)%
  --(12.360,7.183)--(12.363,7.185)--(12.366,7.187)--(12.369,7.189)--(12.372,7.191)--(12.375,7.193)%
  --(12.378,7.195)--(12.381,7.197)--(12.384,7.200)--(12.387,7.202)--(12.390,7.204)--(12.393,7.206)%
  --(12.396,7.208)--(12.399,7.210)--(12.402,7.212)--(12.405,7.214)--(12.408,7.216)--(12.411,7.218)%
  --(12.414,7.221)--(12.417,7.223)--(12.420,7.225)--(12.423,7.227)--(12.426,7.229)--(12.429,7.231)%
  --(12.432,7.233)--(12.435,7.235)--(12.438,7.237)--(12.441,7.239)--(12.444,7.242)--(12.447,7.244)%
  --(12.450,7.246)--(12.453,7.248)--(12.456,7.250)--(12.459,7.252)--(12.462,7.254)--(12.465,7.256)%
  --(12.468,7.258)--(12.471,7.260)--(12.474,7.263)--(12.477,7.265)--(12.480,7.267)--(12.483,7.269)%
  --(12.486,7.271)--(12.489,7.273)--(12.492,7.275)--(12.495,7.277)--(12.498,7.279)--(12.501,7.281)%
  --(12.504,7.284)--(12.506,7.286)--(12.509,7.288)--(12.512,7.290)--(12.515,7.292)--(12.518,7.294)%
  --(12.521,7.296)--(12.524,7.298)--(12.527,7.300)--(12.530,7.302)--(12.533,7.305)--(12.536,7.307)%
  --(12.539,7.309)--(12.542,7.311)--(12.545,7.313)--(12.548,7.315)--(12.551,7.317)--(12.554,7.319)%
  --(12.557,7.321)--(12.560,7.323)--(12.563,7.326)--(12.566,7.328)--(12.569,7.330)--(12.572,7.332)%
  --(12.575,7.334)--(12.578,7.336)--(12.581,7.338)--(12.584,7.340)--(12.587,7.342)--(12.590,7.344)%
  --(12.593,7.347)--(12.596,7.349)--(12.599,7.351)--(12.602,7.353)--(12.605,7.355)--(12.608,7.357)%
  --(12.611,7.359)--(12.614,7.361)--(12.617,7.363)--(12.620,7.366)--(12.623,7.368)--(12.626,7.370)%
  --(12.629,7.372)--(12.632,7.374)--(12.635,7.376)--(12.638,7.378)--(12.641,7.380)--(12.644,7.382)%
  --(12.647,7.384)--(12.650,7.387)--(12.653,7.389)--(12.656,7.391)--(12.659,7.393)--(12.662,7.395)%
  --(12.665,7.397)--(12.668,7.399)--(12.671,7.401)--(12.674,7.403)--(12.677,7.405)--(12.680,7.408)%
  --(12.683,7.410)--(12.686,7.412)--(12.689,7.414)--(12.692,7.416)--(12.695,7.418)--(12.698,7.420)%
  --(12.701,7.422)--(12.704,7.424)--(12.707,7.426)--(12.710,7.429)--(12.713,7.431)--(12.715,7.433)%
  --(12.718,7.435)--(12.721,7.437)--(12.724,7.439)--(12.727,7.441)--(12.730,7.443)--(12.733,7.445)%
  --(12.736,7.447)--(12.739,7.450)--(12.742,7.452)--(12.745,7.454)--(12.748,7.456)--(12.751,7.458)%
  --(12.754,7.460)--(12.757,7.462)--(12.760,7.464)--(12.763,7.466)--(12.766,7.469)--(12.769,7.471)%
  --(12.772,7.473)--(12.775,7.475)--(12.778,7.477)--(12.781,7.479)--(12.784,7.481)--(12.787,7.483)%
  --(12.790,7.485)--(12.793,7.487)--(12.796,7.490)--(12.799,7.492)--(12.802,7.494)--(12.805,7.496)%
  --(12.808,7.498)--(12.811,7.500)--(12.814,7.502)--(12.817,7.504)--(12.820,7.506)--(12.823,7.508)%
  --(12.826,7.511)--(12.829,7.513)--(12.832,7.515)--(12.835,7.517)--(12.838,7.519)--(12.841,7.521)%
  --(12.844,7.523)--(12.847,7.525)--(12.850,7.527)--(12.853,7.529)--(12.856,7.532)--(12.859,7.534)%
  --(12.862,7.536)--(12.865,7.538)--(12.868,7.540)--(12.871,7.542)--(12.874,7.544)--(12.877,7.546)%
  --(12.880,7.548)--(12.883,7.551)--(12.886,7.553)--(12.889,7.555)--(12.892,7.557)--(12.895,7.559)%
  --(12.898,7.561)--(12.901,7.563)--(12.904,7.565)--(12.907,7.567)--(12.910,7.569)--(12.913,7.572)%
  --(12.916,7.574)--(12.919,7.576)--(12.922,7.578)--(12.924,7.580)--(12.927,7.582)--(12.930,7.584)%
  --(12.933,7.586)--(12.936,7.588)--(12.939,7.590)--(12.942,7.593)--(12.945,7.595)--(12.948,7.597)%
  --(12.951,7.599)--(12.954,7.601)--(12.957,7.603)--(12.960,7.605)--(12.963,7.607)--(12.966,7.609)%
  --(12.969,7.612)--(12.972,7.614)--(12.975,7.616)--(12.978,7.618)--(12.981,7.620)--(12.984,7.622)%
  --(12.987,7.624)--(12.990,7.626)--(12.993,7.628)--(12.996,7.630)--(12.999,7.633)--(13.002,7.635)%
  --(13.005,7.637)--(13.008,7.639)--(13.011,7.641)--(13.014,7.643)--(13.017,7.645)--(13.020,7.647)%
  --(13.023,7.649)--(13.026,7.652)--(13.029,7.654)--(13.032,7.656)--(13.035,7.658)--(13.038,7.660)%
  --(13.041,7.662)--(13.044,7.664)--(13.047,7.666)--(13.050,7.668)--(13.053,7.670)--(13.056,7.673)%
  --(13.059,7.675)--(13.062,7.677)--(13.065,7.679)--(13.068,7.681)--(13.071,7.683)--(13.074,7.685)%
  --(13.077,7.687)--(13.080,7.689)--(13.083,7.691)--(13.086,7.694)--(13.089,7.696)--(13.092,7.698)%
  --(13.095,7.700)--(13.098,7.702)--(13.101,7.704)--(13.104,7.706)--(13.107,7.708)--(13.110,7.710)%
  --(13.113,7.713)--(13.116,7.715)--(13.119,7.717)--(13.122,7.719)--(13.125,7.721)--(13.128,7.723)%
  --(13.131,7.725)--(13.133,7.727)--(13.136,7.729)--(13.139,7.731)--(13.142,7.734)--(13.145,7.736)%
  --(13.148,7.738)--(13.151,7.740)--(13.154,7.742)--(13.157,7.744)--(13.160,7.746)--(13.163,7.748)%
  --(13.166,7.750)--(13.169,7.753)--(13.172,7.755)--(13.175,7.757)--(13.178,7.759)--(13.181,7.761)%
  --(13.184,7.763)--(13.187,7.765)--(13.190,7.767)--(13.193,7.769)--(13.196,7.771)--(13.199,7.774)%
  --(13.202,7.776)--(13.205,7.778)--(13.208,7.780)--(13.211,7.782)--(13.214,7.784)--(13.217,7.786)%
  --(13.220,7.788)--(13.223,7.790)--(13.226,7.792)--(13.229,7.795)--(13.232,7.797)--(13.235,7.799)%
  --(13.238,7.801)--(13.241,7.803)--(13.244,7.805)--(13.247,7.807)--(13.250,7.809)--(13.253,7.811)%
  --(13.256,7.814)--(13.259,7.816)--(13.262,7.818)--(13.265,7.820)--(13.268,7.822)--(13.271,7.824)%
  --(13.274,7.826)--(13.277,7.828)--(13.280,7.830)--(13.283,7.832)--(13.286,7.835)--(13.289,7.837)%
  --(13.292,7.839)--(13.295,7.841)--(13.298,7.843)--(13.301,7.845)--(13.304,7.847)--(13.307,7.849)%
  --(13.310,7.851)--(13.313,7.854)--(13.316,7.856)--(13.319,7.858)--(13.322,7.860)--(13.325,7.862)%
  --(13.328,7.864)--(13.331,7.866)--(13.334,7.868)--(13.337,7.870)--(13.340,7.872)--(13.342,7.875)%
  --(13.345,7.877)--(13.348,7.879)--(13.351,7.881)--(13.354,7.883)--(13.357,7.885)--(13.360,7.887)%
  --(13.363,7.889)--(13.366,7.891)--(13.369,7.894)--(13.372,7.896)--(13.375,7.898)--(13.378,7.900)%
  --(13.381,7.902)--(13.384,7.904)--(13.387,7.906)--(13.390,7.908)--(13.393,7.910)--(13.396,7.912)%
  --(13.399,7.915)--(13.402,7.917)--(13.405,7.919)--(13.408,7.921)--(13.411,7.923)--(13.414,7.925)%
  --(13.417,7.927)--(13.420,7.929)--(13.423,7.931)--(13.426,7.934)--(13.429,7.936)--(13.432,7.938)%
  --(13.435,7.940)--(13.438,7.942)--(13.441,7.944)--(13.444,7.946);
\gpcolor{color=gp lt color border}
\node[gp node left] at (2.972,7.989) {$\rho \approx \nicefrac{1}{3} \cdot \rho_{\rm{max}}$};
\gpcolor{rgb color={0.000,0.620,0.451}}
\draw[gp path] (1.872,7.989)--(2.788,7.989);
\draw[gp path] (1.507,1.622)--(1.510,1.621)--(1.513,1.620)--(1.516,1.620)--(1.519,1.619)%
  --(1.522,1.619)--(1.525,1.618)--(1.528,1.618)--(1.531,1.617)--(1.534,1.616)--(1.537,1.616)%
  --(1.540,1.615)--(1.543,1.615)--(1.546,1.614)--(1.549,1.614)--(1.552,1.613)--(1.555,1.613)%
  --(1.558,1.612)--(1.561,1.611)--(1.564,1.611)--(1.567,1.610)--(1.570,1.610)--(1.573,1.609)%
  --(1.576,1.609)--(1.579,1.608)--(1.582,1.607)--(1.585,1.607)--(1.588,1.606)--(1.591,1.606)%
  --(1.594,1.605)--(1.597,1.605)--(1.600,1.604)--(1.603,1.604)--(1.606,1.603)--(1.609,1.602)%
  --(1.611,1.602)--(1.614,1.601)--(1.617,1.601)--(1.620,1.600)--(1.623,1.600)--(1.626,1.599)%
  --(1.629,1.599)--(1.632,1.598)--(1.635,1.597)--(1.638,1.597)--(1.641,1.596)--(1.644,1.596)%
  --(1.647,1.595)--(1.650,1.595)--(1.653,1.594)--(1.656,1.594)--(1.659,1.593)--(1.662,1.593)%
  --(1.665,1.592)--(1.668,1.591)--(1.671,1.591)--(1.674,1.590)--(1.677,1.590)--(1.680,1.589)%
  --(1.683,1.589)--(1.686,1.588)--(1.689,1.588)--(1.692,1.587)--(1.695,1.587)--(1.698,1.586)%
  --(1.701,1.586)--(1.704,1.585)--(1.707,1.584)--(1.710,1.584)--(1.713,1.583)--(1.716,1.583)%
  --(1.719,1.582)--(1.722,1.582)--(1.725,1.581)--(1.728,1.581)--(1.731,1.580)--(1.734,1.580)%
  --(1.737,1.579)--(1.740,1.579)--(1.743,1.578)--(1.746,1.578)--(1.749,1.577)--(1.752,1.577)%
  --(1.755,1.576)--(1.758,1.576)--(1.761,1.575)--(1.764,1.575)--(1.767,1.574)--(1.770,1.574)%
  --(1.773,1.573)--(1.776,1.573)--(1.779,1.572)--(1.782,1.572)--(1.785,1.571)--(1.788,1.571)%
  --(1.791,1.570)--(1.794,1.570)--(1.797,1.569)--(1.800,1.569)--(1.803,1.568)--(1.806,1.568)%
  --(1.809,1.567)--(1.812,1.567)--(1.815,1.566)--(1.818,1.566)--(1.820,1.565)--(1.823,1.565)%
  --(1.826,1.564)--(1.829,1.564)--(1.832,1.563)--(1.835,1.563)--(1.838,1.562)--(1.841,1.562)%
  --(1.844,1.561)--(1.847,1.561)--(1.850,1.561)--(1.853,1.560)--(1.856,1.560)--(1.859,1.559)%
  --(1.862,1.559)--(1.865,1.558)--(1.868,1.558)--(1.871,1.557)--(1.874,1.557)--(1.877,1.556)%
  --(1.880,1.556)--(1.883,1.556)--(1.886,1.555)--(1.889,1.555)--(1.892,1.554)--(1.895,1.554)%
  --(1.898,1.553)--(1.901,1.553)--(1.904,1.552)--(1.907,1.552)--(1.910,1.552)--(1.913,1.551)%
  --(1.916,1.551)--(1.919,1.550)--(1.922,1.550)--(1.925,1.550)--(1.928,1.549)--(1.931,1.549)%
  --(1.934,1.548)--(1.937,1.548)--(1.940,1.547)--(1.943,1.547)--(1.946,1.547)--(1.949,1.546)%
  --(1.952,1.546)--(1.955,1.545)--(1.958,1.545)--(1.961,1.545)--(1.964,1.544)--(1.967,1.544)%
  --(1.970,1.544)--(1.973,1.543)--(1.976,1.543)--(1.979,1.542)--(1.982,1.542)--(1.985,1.542)%
  --(1.988,1.541)--(1.991,1.541)--(1.994,1.541)--(1.997,1.540)--(2.000,1.540)--(2.003,1.539)%
  --(2.006,1.539)--(2.009,1.539)--(2.012,1.538)--(2.015,1.538)--(2.018,1.538)--(2.021,1.537)%
  --(2.024,1.537)--(2.027,1.537)--(2.029,1.536)--(2.032,1.536)--(2.035,1.536)--(2.038,1.535)%
  --(2.041,1.535)--(2.044,1.535)--(2.047,1.534)--(2.050,1.534)--(2.053,1.534)--(2.056,1.533)%
  --(2.059,1.533)--(2.062,1.533)--(2.065,1.532)--(2.068,1.532)--(2.071,1.532)--(2.074,1.532)%
  --(2.077,1.531)--(2.080,1.531)--(2.083,1.531)--(2.086,1.530)--(2.089,1.530)--(2.092,1.530)%
  --(2.095,1.530)--(2.098,1.529)--(2.101,1.529)--(2.104,1.529)--(2.107,1.528)--(2.110,1.528)%
  --(2.113,1.528)--(2.116,1.528)--(2.119,1.527)--(2.122,1.527)--(2.125,1.527)--(2.128,1.527)%
  --(2.131,1.526)--(2.134,1.526)--(2.137,1.526)--(2.140,1.526)--(2.143,1.525)--(2.146,1.525)%
  --(2.149,1.525)--(2.152,1.525)--(2.155,1.524)--(2.158,1.524)--(2.161,1.524)--(2.164,1.524)%
  --(2.167,1.524)--(2.170,1.523)--(2.173,1.523)--(2.176,1.523)--(2.179,1.523)--(2.182,1.522)%
  --(2.185,1.522)--(2.188,1.522)--(2.191,1.522)--(2.194,1.522)--(2.197,1.521)--(2.200,1.521)%
  --(2.203,1.521)--(2.206,1.521)--(2.209,1.521)--(2.212,1.521)--(2.215,1.520)--(2.218,1.520)%
  --(2.221,1.520)--(2.224,1.520)--(2.227,1.520)--(2.230,1.520)--(2.233,1.519)--(2.236,1.519)%
  --(2.238,1.519)--(2.241,1.519)--(2.244,1.519)--(2.247,1.519)--(2.250,1.519)--(2.253,1.518)%
  --(2.256,1.518)--(2.259,1.518)--(2.262,1.518)--(2.265,1.518)--(2.268,1.518)--(2.271,1.518)%
  --(2.274,1.517)--(2.277,1.517)--(2.280,1.517)--(2.283,1.517)--(2.286,1.517)--(2.289,1.517)%
  --(2.292,1.517)--(2.295,1.517)--(2.298,1.517)--(2.301,1.517)--(2.304,1.516)--(2.307,1.516)%
  --(2.310,1.516)--(2.313,1.516)--(2.316,1.516)--(2.319,1.516)--(2.322,1.516)--(2.325,1.516)%
  --(2.328,1.516)--(2.331,1.516)--(2.334,1.516)--(2.337,1.516)--(2.340,1.516)--(2.343,1.516)%
  --(2.346,1.516)--(2.349,1.516)--(2.352,1.515)--(2.355,1.515)--(2.358,1.515)--(2.361,1.515)%
  --(2.364,1.515)--(2.367,1.515)--(2.370,1.515)--(2.373,1.515)--(2.376,1.515)--(2.379,1.515)%
  --(2.382,1.515)--(2.385,1.515)--(2.388,1.515)--(2.391,1.515)--(2.394,1.515)--(2.397,1.515)%
  --(2.400,1.515)--(2.403,1.515)--(2.406,1.515)--(2.409,1.515)--(2.412,1.515)--(2.415,1.515)%
  --(2.418,1.515)--(2.421,1.515)--(2.424,1.515)--(2.427,1.516)--(2.430,1.516)--(2.433,1.516)%
  --(2.436,1.516)--(2.439,1.516)--(2.442,1.516)--(2.445,1.516)--(2.447,1.516)--(2.450,1.516)%
  --(2.453,1.516)--(2.456,1.516)--(2.459,1.516)--(2.462,1.516)--(2.465,1.516)--(2.468,1.516)%
  --(2.471,1.516)--(2.474,1.517)--(2.477,1.517)--(2.480,1.517)--(2.483,1.517)--(2.486,1.517)%
  --(2.489,1.517)--(2.492,1.517)--(2.495,1.517)--(2.498,1.517)--(2.501,1.517)--(2.504,1.518)%
  --(2.507,1.518)--(2.510,1.518)--(2.513,1.518)--(2.516,1.518)--(2.519,1.518)--(2.522,1.518)%
  --(2.525,1.519)--(2.528,1.519)--(2.531,1.519)--(2.534,1.519)--(2.537,1.519)--(2.540,1.519)%
  --(2.543,1.519)--(2.546,1.520)--(2.549,1.520)--(2.552,1.520)--(2.555,1.520)--(2.558,1.520)%
  --(2.561,1.520)--(2.564,1.521)--(2.567,1.521)--(2.570,1.521)--(2.573,1.521)--(2.576,1.521)%
  --(2.579,1.522)--(2.582,1.522)--(2.585,1.522)--(2.588,1.522)--(2.591,1.522)--(2.594,1.523)%
  --(2.597,1.523)--(2.600,1.523)--(2.603,1.523)--(2.606,1.523)--(2.609,1.524)--(2.612,1.524)%
  --(2.615,1.524)--(2.618,1.524)--(2.621,1.525)--(2.624,1.525)--(2.627,1.525)--(2.630,1.525)%
  --(2.633,1.526)--(2.636,1.526)--(2.639,1.526)--(2.642,1.526)--(2.645,1.527)--(2.648,1.527)%
  --(2.651,1.527)--(2.654,1.527)--(2.656,1.528)--(2.659,1.528)--(2.662,1.528)--(2.665,1.529)%
  --(2.668,1.529)--(2.671,1.529)--(2.674,1.529)--(2.677,1.530)--(2.680,1.530)--(2.683,1.530)%
  --(2.686,1.531)--(2.689,1.531)--(2.692,1.531)--(2.695,1.532)--(2.698,1.532)--(2.701,1.532)%
  --(2.704,1.533)--(2.707,1.533)--(2.710,1.533)--(2.713,1.534)--(2.716,1.534)--(2.719,1.534)%
  --(2.722,1.535)--(2.725,1.535)--(2.728,1.535)--(2.731,1.536)--(2.734,1.536)--(2.737,1.536)%
  --(2.740,1.537)--(2.743,1.537)--(2.746,1.537)--(2.749,1.538)--(2.752,1.538)--(2.755,1.539)%
  --(2.758,1.539)--(2.761,1.539)--(2.764,1.540)--(2.767,1.540)--(2.770,1.541)--(2.773,1.541)%
  --(2.776,1.541)--(2.779,1.542)--(2.782,1.542)--(2.785,1.543)--(2.788,1.543)--(2.791,1.543)%
  --(2.794,1.544)--(2.797,1.544)--(2.800,1.545)--(2.803,1.545)--(2.806,1.545)--(2.809,1.546)%
  --(2.812,1.546)--(2.815,1.547)--(2.818,1.547)--(2.821,1.548)--(2.824,1.548)--(2.827,1.549)%
  --(2.830,1.549)--(2.833,1.549)--(2.836,1.550)--(2.839,1.550)--(2.842,1.551)--(2.845,1.551)%
  --(2.848,1.552)--(2.851,1.552)--(2.854,1.553)--(2.857,1.553)--(2.860,1.554)--(2.863,1.554)%
  --(2.866,1.555)--(2.868,1.555)--(2.871,1.556)--(2.874,1.556)--(2.877,1.557)--(2.880,1.557)%
  --(2.883,1.558)--(2.886,1.558)--(2.889,1.559)--(2.892,1.559)--(2.895,1.560)--(2.898,1.560)%
  --(2.901,1.561)--(2.904,1.561)--(2.907,1.562)--(2.910,1.562)--(2.913,1.563)--(2.916,1.563)%
  --(2.919,1.564)--(2.922,1.565)--(2.925,1.565)--(2.928,1.566)--(2.931,1.566)--(2.934,1.567)%
  --(2.937,1.567)--(2.940,1.568)--(2.943,1.568)--(2.946,1.569)--(2.949,1.570)--(2.952,1.570)%
  --(2.955,1.571)--(2.958,1.571)--(2.961,1.572)--(2.964,1.572)--(2.967,1.573)--(2.970,1.574)%
  --(2.973,1.574)--(2.976,1.575)--(2.979,1.575)--(2.982,1.576)--(2.985,1.577)--(2.988,1.577)%
  --(2.991,1.578)--(2.994,1.578)--(2.997,1.579)--(3.000,1.580)--(3.003,1.580)--(3.006,1.581)%
  --(3.009,1.582)--(3.012,1.582)--(3.015,1.583)--(3.018,1.583)--(3.021,1.584)--(3.024,1.585)%
  --(3.027,1.585)--(3.030,1.586)--(3.033,1.587)--(3.036,1.587)--(3.039,1.588)--(3.042,1.589)%
  --(3.045,1.589)--(3.048,1.590)--(3.051,1.591)--(3.054,1.591)--(3.057,1.592)--(3.060,1.593)%
  --(3.063,1.593)--(3.066,1.594)--(3.069,1.595)--(3.072,1.595)--(3.075,1.596)--(3.077,1.597)%
  --(3.080,1.597)--(3.083,1.598)--(3.086,1.599)--(3.089,1.599)--(3.092,1.600)--(3.095,1.601)%
  --(3.098,1.602)--(3.101,1.602)--(3.104,1.603)--(3.107,1.604)--(3.110,1.604)--(3.113,1.605)%
  --(3.116,1.606)--(3.119,1.607)--(3.122,1.607)--(3.125,1.608)--(3.128,1.609)--(3.131,1.609)%
  --(3.134,1.610)--(3.137,1.611)--(3.140,1.612)--(3.143,1.612)--(3.146,1.613)--(3.149,1.614)%
  --(3.152,1.615)--(3.155,1.615)--(3.158,1.616)--(3.161,1.617)--(3.164,1.618)--(3.167,1.619)%
  --(3.170,1.619)--(3.173,1.620)--(3.176,1.621)--(3.179,1.622)--(3.182,1.622)--(3.185,1.623)%
  --(3.188,1.624)--(3.191,1.625)--(3.194,1.626)--(3.197,1.626)--(3.200,1.627)--(3.203,1.628)%
  --(3.206,1.629)--(3.209,1.630)--(3.212,1.630)--(3.215,1.631)--(3.218,1.632)--(3.221,1.633)%
  --(3.224,1.634)--(3.227,1.634)--(3.230,1.635)--(3.233,1.636)--(3.236,1.637)--(3.239,1.638)%
  --(3.242,1.639)--(3.245,1.639)--(3.248,1.640)--(3.251,1.641)--(3.254,1.642)--(3.257,1.643)%
  --(3.260,1.644)--(3.263,1.644)--(3.266,1.645)--(3.269,1.646)--(3.272,1.647)--(3.275,1.648)%
  --(3.278,1.649)--(3.281,1.650)--(3.284,1.650)--(3.286,1.651)--(3.289,1.652)--(3.292,1.653)%
  --(3.295,1.654)--(3.298,1.655)--(3.301,1.656)--(3.304,1.657)--(3.307,1.657)--(3.310,1.658)%
  --(3.313,1.659)--(3.316,1.660)--(3.319,1.661)--(3.322,1.662)--(3.325,1.663)--(3.328,1.664)%
  --(3.331,1.665)--(3.334,1.665)--(3.337,1.666)--(3.340,1.667)--(3.343,1.668)--(3.346,1.669)%
  --(3.349,1.670)--(3.352,1.671)--(3.355,1.672)--(3.358,1.673)--(3.361,1.674)--(3.364,1.675)%
  --(3.367,1.676)--(3.370,1.677)--(3.373,1.677)--(3.376,1.678)--(3.379,1.679)--(3.382,1.680)%
  --(3.385,1.681)--(3.388,1.682)--(3.391,1.683)--(3.394,1.684)--(3.397,1.685)--(3.400,1.686)%
  --(3.403,1.687)--(3.406,1.688)--(3.409,1.689)--(3.412,1.690)--(3.415,1.691)--(3.418,1.692)%
  --(3.421,1.693)--(3.424,1.694)--(3.427,1.695)--(3.430,1.696)--(3.433,1.697)--(3.436,1.698)%
  --(3.439,1.699)--(3.442,1.700)--(3.445,1.701)--(3.448,1.702)--(3.451,1.703)--(3.454,1.704)%
  --(3.457,1.705)--(3.460,1.706)--(3.463,1.707)--(3.466,1.708)--(3.469,1.709)--(3.472,1.710)%
  --(3.475,1.711)--(3.478,1.712)--(3.481,1.713)--(3.484,1.714)--(3.487,1.715)--(3.490,1.716)%
  --(3.493,1.717)--(3.495,1.718)--(3.498,1.719)--(3.501,1.720)--(3.504,1.721)--(3.507,1.722)%
  --(3.510,1.723)--(3.513,1.724)--(3.516,1.725)--(3.519,1.726)--(3.522,1.727)--(3.525,1.728)%
  --(3.528,1.729)--(3.531,1.730)--(3.534,1.731)--(3.537,1.732)--(3.540,1.733)--(3.543,1.734)%
  --(3.546,1.735)--(3.549,1.737)--(3.552,1.738)--(3.555,1.739)--(3.558,1.740)--(3.561,1.741)%
  --(3.564,1.742)--(3.567,1.743)--(3.570,1.744)--(3.573,1.745)--(3.576,1.746)--(3.579,1.747)%
  --(3.582,1.748)--(3.585,1.749)--(3.588,1.751)--(3.591,1.752)--(3.594,1.753)--(3.597,1.754)%
  --(3.600,1.755)--(3.603,1.756)--(3.606,1.757)--(3.609,1.758)--(3.612,1.759)--(3.615,1.760)%
  --(3.618,1.762)--(3.621,1.763)--(3.624,1.764)--(3.627,1.765)--(3.630,1.766)--(3.633,1.767)%
  --(3.636,1.768)--(3.639,1.769)--(3.642,1.770)--(3.645,1.772)--(3.648,1.773)--(3.651,1.774)%
  --(3.654,1.775)--(3.657,1.776)--(3.660,1.777)--(3.663,1.778)--(3.666,1.780)--(3.669,1.781)%
  --(3.672,1.782)--(3.675,1.783)--(3.678,1.784)--(3.681,1.785)--(3.684,1.786)--(3.687,1.788)%
  --(3.690,1.789)--(3.693,1.790)--(3.696,1.791)--(3.699,1.792)--(3.702,1.793)--(3.704,1.795)%
  --(3.707,1.796)--(3.710,1.797)--(3.713,1.798)--(3.716,1.799)--(3.719,1.800)--(3.722,1.802)%
  --(3.725,1.803)--(3.728,1.804)--(3.731,1.805)--(3.734,1.806)--(3.737,1.808)--(3.740,1.809)%
  --(3.743,1.810)--(3.746,1.811)--(3.749,1.812)--(3.752,1.813)--(3.755,1.815)--(3.758,1.816)%
  --(3.761,1.817)--(3.764,1.818)--(3.767,1.819)--(3.770,1.821)--(3.773,1.822)--(3.776,1.823)%
  --(3.779,1.824)--(3.782,1.826)--(3.785,1.827)--(3.788,1.828)--(3.791,1.829)--(3.794,1.830)%
  --(3.797,1.832)--(3.800,1.833)--(3.803,1.834)--(3.806,1.835)--(3.809,1.837)--(3.812,1.838)%
  --(3.815,1.839)--(3.818,1.840)--(3.821,1.841)--(3.824,1.843)--(3.827,1.844)--(3.830,1.845)%
  --(3.833,1.846)--(3.836,1.848)--(3.839,1.849)--(3.842,1.850)--(3.845,1.851)--(3.848,1.853)%
  --(3.851,1.854)--(3.854,1.855)--(3.857,1.856)--(3.860,1.858)--(3.863,1.859)--(3.866,1.860)%
  --(3.869,1.861)--(3.872,1.863)--(3.875,1.864)--(3.878,1.865)--(3.881,1.867)--(3.884,1.868)%
  --(3.887,1.869)--(3.890,1.870)--(3.893,1.872)--(3.896,1.873)--(3.899,1.874)--(3.902,1.875)%
  --(3.905,1.877)--(3.908,1.878)--(3.911,1.879)--(3.914,1.881)--(3.916,1.882)--(3.919,1.883)%
  --(3.922,1.884)--(3.925,1.886)--(3.928,1.887)--(3.931,1.888)--(3.934,1.890)--(3.937,1.891)%
  --(3.940,1.892)--(3.943,1.894)--(3.946,1.895)--(3.949,1.896)--(3.952,1.897)--(3.955,1.899)%
  --(3.958,1.900)--(3.961,1.901)--(3.964,1.903)--(3.967,1.904)--(3.970,1.905)--(3.973,1.907)%
  --(3.976,1.908)--(3.979,1.909)--(3.982,1.911)--(3.985,1.912)--(3.988,1.913)--(3.991,1.915)%
  --(3.994,1.916)--(3.997,1.917)--(4.000,1.919)--(4.003,1.920)--(4.006,1.921)--(4.009,1.923)%
  --(4.012,1.924)--(4.015,1.925)--(4.018,1.927)--(4.021,1.928)--(4.024,1.929)--(4.027,1.931)%
  --(4.030,1.932)--(4.033,1.933)--(4.036,1.935)--(4.039,1.936)--(4.042,1.937)--(4.045,1.939)%
  --(4.048,1.940)--(4.051,1.941)--(4.054,1.943)--(4.057,1.944)--(4.060,1.946)--(4.063,1.947)%
  --(4.066,1.948)--(4.069,1.950)--(4.072,1.951)--(4.075,1.952)--(4.078,1.954)--(4.081,1.955)%
  --(4.084,1.957)--(4.087,1.958)--(4.090,1.959)--(4.093,1.961)--(4.096,1.962)--(4.099,1.963)%
  --(4.102,1.965)--(4.105,1.966)--(4.108,1.968)--(4.111,1.969)--(4.114,1.970)--(4.117,1.972)%
  --(4.120,1.973)--(4.123,1.975)--(4.125,1.976)--(4.128,1.977)--(4.131,1.979)--(4.134,1.980)%
  --(4.137,1.981)--(4.140,1.983)--(4.143,1.984)--(4.146,1.986)--(4.149,1.987)--(4.152,1.989)%
  --(4.155,1.990)--(4.158,1.991)--(4.161,1.993)--(4.164,1.994)--(4.167,1.996)--(4.170,1.997)%
  --(4.173,1.998)--(4.176,2.000)--(4.179,2.001)--(4.182,2.003)--(4.185,2.004)--(4.188,2.005)%
  --(4.191,2.007)--(4.194,2.008)--(4.197,2.010)--(4.200,2.011)--(4.203,2.013)--(4.206,2.014)%
  --(4.209,2.015)--(4.212,2.017)--(4.215,2.018)--(4.218,2.020)--(4.221,2.021)--(4.224,2.023)%
  --(4.227,2.024)--(4.230,2.026)--(4.233,2.027)--(4.236,2.028)--(4.239,2.030)--(4.242,2.031)%
  --(4.245,2.033)--(4.248,2.034)--(4.251,2.036)--(4.254,2.037)--(4.257,2.039)--(4.260,2.040)%
  --(4.263,2.041)--(4.266,2.043)--(4.269,2.044)--(4.272,2.046)--(4.275,2.047)--(4.278,2.049)%
  --(4.281,2.050)--(4.284,2.052)--(4.287,2.053)--(4.290,2.055)--(4.293,2.056)--(4.296,2.058)%
  --(4.299,2.059)--(4.302,2.061)--(4.305,2.062)--(4.308,2.063)--(4.311,2.065)--(4.314,2.066)%
  --(4.317,2.068)--(4.320,2.069)--(4.323,2.071)--(4.326,2.072)--(4.329,2.074)--(4.332,2.075)%
  --(4.334,2.077)--(4.337,2.078)--(4.340,2.080)--(4.343,2.081)--(4.346,2.083)--(4.349,2.084)%
  --(4.352,2.086)--(4.355,2.087)--(4.358,2.089)--(4.361,2.090)--(4.364,2.092)--(4.367,2.093)%
  --(4.370,2.095)--(4.373,2.096)--(4.376,2.098)--(4.379,2.099)--(4.382,2.101)--(4.385,2.102)%
  --(4.388,2.104)--(4.391,2.105)--(4.394,2.107)--(4.397,2.108)--(4.400,2.110)--(4.403,2.111)%
  --(4.406,2.113)--(4.409,2.114)--(4.412,2.116)--(4.415,2.117)--(4.418,2.119)--(4.421,2.120)%
  --(4.424,2.122)--(4.427,2.123)--(4.430,2.125)--(4.433,2.126)--(4.436,2.128)--(4.439,2.129)%
  --(4.442,2.131)--(4.445,2.133)--(4.448,2.134)--(4.451,2.136)--(4.454,2.137)--(4.457,2.139)%
  --(4.460,2.140)--(4.463,2.142)--(4.466,2.143)--(4.469,2.145)--(4.472,2.146)--(4.475,2.148)%
  --(4.478,2.149)--(4.481,2.151)--(4.484,2.152)--(4.487,2.154)--(4.490,2.156)--(4.493,2.157)%
  --(4.496,2.159)--(4.499,2.160)--(4.502,2.162)--(4.505,2.163)--(4.508,2.165)--(4.511,2.166)%
  --(4.514,2.168)--(4.517,2.170)--(4.520,2.171)--(4.523,2.173)--(4.526,2.174)--(4.529,2.176)%
  --(4.532,2.177)--(4.535,2.179)--(4.538,2.180)--(4.541,2.182)--(4.543,2.184)--(4.546,2.185)%
  --(4.549,2.187)--(4.552,2.188)--(4.555,2.190)--(4.558,2.191)--(4.561,2.193)--(4.564,2.195)%
  --(4.567,2.196)--(4.570,2.198)--(4.573,2.199)--(4.576,2.201)--(4.579,2.202)--(4.582,2.204)%
  --(4.585,2.206)--(4.588,2.207)--(4.591,2.209)--(4.594,2.210)--(4.597,2.212)--(4.600,2.213)%
  --(4.603,2.215)--(4.606,2.217)--(4.609,2.218)--(4.612,2.220)--(4.615,2.221)--(4.618,2.223)%
  --(4.621,2.225)--(4.624,2.226)--(4.627,2.228)--(4.630,2.229)--(4.633,2.231)--(4.636,2.232)%
  --(4.639,2.234)--(4.642,2.236)--(4.645,2.237)--(4.648,2.239)--(4.651,2.240)--(4.654,2.242)%
  --(4.657,2.244)--(4.660,2.245)--(4.663,2.247)--(4.666,2.248)--(4.669,2.250)--(4.672,2.252)%
  --(4.675,2.253)--(4.678,2.255)--(4.681,2.257)--(4.684,2.258)--(4.687,2.260)--(4.690,2.261)%
  --(4.693,2.263)--(4.696,2.265)--(4.699,2.266)--(4.702,2.268)--(4.705,2.269)--(4.708,2.271)%
  --(4.711,2.273)--(4.714,2.274)--(4.717,2.276)--(4.720,2.278)--(4.723,2.279)--(4.726,2.281)%
  --(4.729,2.282)--(4.732,2.284)--(4.735,2.286)--(4.738,2.287)--(4.741,2.289)--(4.744,2.291)%
  --(4.747,2.292)--(4.750,2.294)--(4.752,2.295)--(4.755,2.297)--(4.758,2.299)--(4.761,2.300)%
  --(4.764,2.302)--(4.767,2.304)--(4.770,2.305)--(4.773,2.307)--(4.776,2.308)--(4.779,2.310)%
  --(4.782,2.312)--(4.785,2.313)--(4.788,2.315)--(4.791,2.317)--(4.794,2.318)--(4.797,2.320)%
  --(4.800,2.322)--(4.803,2.323)--(4.806,2.325)--(4.809,2.327)--(4.812,2.328)--(4.815,2.330)%
  --(4.818,2.331)--(4.821,2.333)--(4.824,2.335)--(4.827,2.336)--(4.830,2.338)--(4.833,2.340)%
  --(4.836,2.341)--(4.839,2.343)--(4.842,2.345)--(4.845,2.346)--(4.848,2.348)--(4.851,2.350)%
  --(4.854,2.351)--(4.857,2.353)--(4.860,2.355)--(4.863,2.356)--(4.866,2.358)--(4.869,2.360)%
  --(4.872,2.361)--(4.875,2.363)--(4.878,2.365)--(4.881,2.366)--(4.884,2.368)--(4.887,2.370)%
  --(4.890,2.371)--(4.893,2.373)--(4.896,2.375)--(4.899,2.376)--(4.902,2.378)--(4.905,2.380)%
  --(4.908,2.381)--(4.911,2.383)--(4.914,2.385)--(4.917,2.386)--(4.920,2.388)--(4.923,2.390)%
  --(4.926,2.391)--(4.929,2.393)--(4.932,2.395)--(4.935,2.396)--(4.938,2.398)--(4.941,2.400)%
  --(4.944,2.401)--(4.947,2.403)--(4.950,2.405)--(4.953,2.407)--(4.956,2.408)--(4.959,2.410)%
  --(4.961,2.412)--(4.964,2.413)--(4.967,2.415)--(4.970,2.417)--(4.973,2.418)--(4.976,2.420)%
  --(4.979,2.422)--(4.982,2.423)--(4.985,2.425)--(4.988,2.427)--(4.991,2.429)--(4.994,2.430)%
  --(4.997,2.432)--(5.000,2.434)--(5.003,2.435)--(5.006,2.437)--(5.009,2.439)--(5.012,2.440)%
  --(5.015,2.442)--(5.018,2.444)--(5.021,2.446)--(5.024,2.447)--(5.027,2.449)--(5.030,2.451)%
  --(5.033,2.452)--(5.036,2.454)--(5.039,2.456)--(5.042,2.457)--(5.045,2.459)--(5.048,2.461)%
  --(5.051,2.463)--(5.054,2.464)--(5.057,2.466)--(5.060,2.468)--(5.063,2.469)--(5.066,2.471)%
  --(5.069,2.473)--(5.072,2.475)--(5.075,2.476)--(5.078,2.478)--(5.081,2.480)--(5.084,2.481)%
  --(5.087,2.483)--(5.090,2.485)--(5.093,2.487)--(5.096,2.488)--(5.099,2.490)--(5.102,2.492)%
  --(5.105,2.494)--(5.108,2.495)--(5.111,2.497)--(5.114,2.499)--(5.117,2.500)--(5.120,2.502)%
  --(5.123,2.504)--(5.126,2.506)--(5.129,2.507)--(5.132,2.509)--(5.135,2.511)--(5.138,2.513)%
  --(5.141,2.514)--(5.144,2.516)--(5.147,2.518)--(5.150,2.519)--(5.153,2.521)--(5.156,2.523)%
  --(5.159,2.525)--(5.162,2.526)--(5.165,2.528)--(5.168,2.530)--(5.171,2.532)--(5.173,2.533)%
  --(5.176,2.535)--(5.179,2.537)--(5.182,2.539)--(5.185,2.540)--(5.188,2.542)--(5.191,2.544)%
  --(5.194,2.546)--(5.197,2.547)--(5.200,2.549)--(5.203,2.551)--(5.206,2.553)--(5.209,2.554)%
  --(5.212,2.556)--(5.215,2.558)--(5.218,2.560)--(5.221,2.561)--(5.224,2.563)--(5.227,2.565)%
  --(5.230,2.567)--(5.233,2.568)--(5.236,2.570)--(5.239,2.572)--(5.242,2.574)--(5.245,2.575)%
  --(5.248,2.577)--(5.251,2.579)--(5.254,2.581)--(5.257,2.582)--(5.260,2.584)--(5.263,2.586)%
  --(5.266,2.588)--(5.269,2.589)--(5.272,2.591)--(5.275,2.593)--(5.278,2.595)--(5.281,2.596)%
  --(5.284,2.598)--(5.287,2.600)--(5.290,2.602)--(5.293,2.603)--(5.296,2.605)--(5.299,2.607)%
  --(5.302,2.609)--(5.305,2.610)--(5.308,2.612)--(5.311,2.614)--(5.314,2.616)--(5.317,2.618)%
  --(5.320,2.619)--(5.323,2.621)--(5.326,2.623)--(5.329,2.625)--(5.332,2.626)--(5.335,2.628)%
  --(5.338,2.630)--(5.341,2.632)--(5.344,2.634)--(5.347,2.635)--(5.350,2.637)--(5.353,2.639)%
  --(5.356,2.641)--(5.359,2.642)--(5.362,2.644)--(5.365,2.646)--(5.368,2.648)--(5.371,2.650)%
  --(5.374,2.651)--(5.377,2.653)--(5.380,2.655)--(5.382,2.657)--(5.385,2.658)--(5.388,2.660)%
  --(5.391,2.662)--(5.394,2.664)--(5.397,2.666)--(5.400,2.667)--(5.403,2.669)--(5.406,2.671)%
  --(5.409,2.673)--(5.412,2.674)--(5.415,2.676)--(5.418,2.678)--(5.421,2.680)--(5.424,2.682)%
  --(5.427,2.683)--(5.430,2.685)--(5.433,2.687)--(5.436,2.689)--(5.439,2.691)--(5.442,2.692)%
  --(5.445,2.694)--(5.448,2.696)--(5.451,2.698)--(5.454,2.700)--(5.457,2.701)--(5.460,2.703)%
  --(5.463,2.705)--(5.466,2.707)--(5.469,2.709)--(5.472,2.710)--(5.475,2.712)--(5.478,2.714)%
  --(5.481,2.716)--(5.484,2.718)--(5.487,2.719)--(5.490,2.721)--(5.493,2.723)--(5.496,2.725)%
  --(5.499,2.727)--(5.502,2.728)--(5.505,2.730)--(5.508,2.732)--(5.511,2.734)--(5.514,2.736)%
  --(5.517,2.737)--(5.520,2.739)--(5.523,2.741)--(5.526,2.743)--(5.529,2.745)--(5.532,2.746)%
  --(5.535,2.748)--(5.538,2.750)--(5.541,2.752)--(5.544,2.754)--(5.547,2.755)--(5.550,2.757)%
  --(5.553,2.759)--(5.556,2.761)--(5.559,2.763)--(5.562,2.765)--(5.565,2.766)--(5.568,2.768)%
  --(5.571,2.770)--(5.574,2.772)--(5.577,2.774)--(5.580,2.775)--(5.583,2.777)--(5.586,2.779)%
  --(5.589,2.781)--(5.591,2.783)--(5.594,2.785)--(5.597,2.786)--(5.600,2.788)--(5.603,2.790)%
  --(5.606,2.792)--(5.609,2.794)--(5.612,2.795)--(5.615,2.797)--(5.618,2.799)--(5.621,2.801)%
  --(5.624,2.803)--(5.627,2.805)--(5.630,2.806)--(5.633,2.808)--(5.636,2.810)--(5.639,2.812)%
  --(5.642,2.814)--(5.645,2.816)--(5.648,2.817)--(5.651,2.819)--(5.654,2.821)--(5.657,2.823)%
  --(5.660,2.825)--(5.663,2.827)--(5.666,2.828)--(5.669,2.830)--(5.672,2.832)--(5.675,2.834)%
  --(5.678,2.836)--(5.681,2.838)--(5.684,2.839)--(5.687,2.841)--(5.690,2.843)--(5.693,2.845)%
  --(5.696,2.847)--(5.699,2.849)--(5.702,2.850)--(5.705,2.852)--(5.708,2.854)--(5.711,2.856)%
  --(5.714,2.858)--(5.717,2.860)--(5.720,2.861)--(5.723,2.863)--(5.726,2.865)--(5.729,2.867)%
  --(5.732,2.869)--(5.735,2.871)--(5.738,2.872)--(5.741,2.874)--(5.744,2.876)--(5.747,2.878)%
  --(5.750,2.880)--(5.753,2.882)--(5.756,2.884)--(5.759,2.885)--(5.762,2.887)--(5.765,2.889)%
  --(5.768,2.891)--(5.771,2.893)--(5.774,2.895)--(5.777,2.896)--(5.780,2.898)--(5.783,2.900)%
  --(5.786,2.902)--(5.789,2.904)--(5.792,2.906)--(5.795,2.908)--(5.798,2.909)--(5.800,2.911)%
  --(5.803,2.913)--(5.806,2.915)--(5.809,2.917)--(5.812,2.919)--(5.815,2.921)--(5.818,2.922)%
  --(5.821,2.924)--(5.824,2.926)--(5.827,2.928)--(5.830,2.930)--(5.833,2.932)--(5.836,2.934)%
  --(5.839,2.935)--(5.842,2.937)--(5.845,2.939)--(5.848,2.941)--(5.851,2.943)--(5.854,2.945)%
  --(5.857,2.947)--(5.860,2.948)--(5.863,2.950)--(5.866,2.952)--(5.869,2.954)--(5.872,2.956)%
  --(5.875,2.958)--(5.878,2.960)--(5.881,2.961)--(5.884,2.963)--(5.887,2.965)--(5.890,2.967)%
  --(5.893,2.969)--(5.896,2.971)--(5.899,2.973)--(5.902,2.975)--(5.905,2.976)--(5.908,2.978)%
  --(5.911,2.980)--(5.914,2.982)--(5.917,2.984)--(5.920,2.986)--(5.923,2.988)--(5.926,2.989)%
  --(5.929,2.991)--(5.932,2.993)--(5.935,2.995)--(5.938,2.997)--(5.941,2.999)--(5.944,3.001)%
  --(5.947,3.003)--(5.950,3.004)--(5.953,3.006)--(5.956,3.008)--(5.959,3.010)--(5.962,3.012)%
  --(5.965,3.014)--(5.968,3.016)--(5.971,3.018)--(5.974,3.019)--(5.977,3.021)--(5.980,3.023)%
  --(5.983,3.025)--(5.986,3.027)--(5.989,3.029)--(5.992,3.031)--(5.995,3.033)--(5.998,3.034)%
  --(6.001,3.036)--(6.004,3.038)--(6.007,3.040)--(6.009,3.042)--(6.012,3.044)--(6.015,3.046)%
  --(6.018,3.048)--(6.021,3.049)--(6.024,3.051)--(6.027,3.053)--(6.030,3.055)--(6.033,3.057)%
  --(6.036,3.059)--(6.039,3.061)--(6.042,3.063)--(6.045,3.065)--(6.048,3.066)--(6.051,3.068)%
  --(6.054,3.070)--(6.057,3.072)--(6.060,3.074)--(6.063,3.076)--(6.066,3.078)--(6.069,3.080)%
  --(6.072,3.082)--(6.075,3.083)--(6.078,3.085)--(6.081,3.087)--(6.084,3.089)--(6.087,3.091)%
  --(6.090,3.093)--(6.093,3.095)--(6.096,3.097)--(6.099,3.099)--(6.102,3.100)--(6.105,3.102)%
  --(6.108,3.104)--(6.111,3.106)--(6.114,3.108)--(6.117,3.110)--(6.120,3.112)--(6.123,3.114)%
  --(6.126,3.116)--(6.129,3.117)--(6.132,3.119)--(6.135,3.121)--(6.138,3.123)--(6.141,3.125)%
  --(6.144,3.127)--(6.147,3.129)--(6.150,3.131)--(6.153,3.133)--(6.156,3.135)--(6.159,3.136)%
  --(6.162,3.138)--(6.165,3.140)--(6.168,3.142)--(6.171,3.144)--(6.174,3.146)--(6.177,3.148)%
  --(6.180,3.150)--(6.183,3.152)--(6.186,3.154)--(6.189,3.155)--(6.192,3.157)--(6.195,3.159)%
  --(6.198,3.161)--(6.201,3.163)--(6.204,3.165)--(6.207,3.167)--(6.210,3.169)--(6.213,3.171)%
  --(6.216,3.173)--(6.218,3.174)--(6.221,3.176)--(6.224,3.178)--(6.227,3.180)--(6.230,3.182)%
  --(6.233,3.184)--(6.236,3.186)--(6.239,3.188)--(6.242,3.190)--(6.245,3.192)--(6.248,3.194)%
  --(6.251,3.195)--(6.254,3.197)--(6.257,3.199)--(6.260,3.201)--(6.263,3.203)--(6.266,3.205)%
  --(6.269,3.207)--(6.272,3.209)--(6.275,3.211)--(6.278,3.213)--(6.281,3.215)--(6.284,3.216)%
  --(6.287,3.218)--(6.290,3.220)--(6.293,3.222)--(6.296,3.224)--(6.299,3.226)--(6.302,3.228)%
  --(6.305,3.230)--(6.308,3.232)--(6.311,3.234)--(6.314,3.236)--(6.317,3.238)--(6.320,3.239)%
  --(6.323,3.241)--(6.326,3.243)--(6.329,3.245)--(6.332,3.247)--(6.335,3.249)--(6.338,3.251)%
  --(6.341,3.253)--(6.344,3.255)--(6.347,3.257)--(6.350,3.259)--(6.353,3.261)--(6.356,3.262)%
  --(6.359,3.264)--(6.362,3.266)--(6.365,3.268)--(6.368,3.270)--(6.371,3.272)--(6.374,3.274)%
  --(6.377,3.276)--(6.380,3.278)--(6.383,3.280)--(6.386,3.282)--(6.389,3.284)--(6.392,3.286)%
  --(6.395,3.287)--(6.398,3.289)--(6.401,3.291)--(6.404,3.293)--(6.407,3.295)--(6.410,3.297)%
  --(6.413,3.299)--(6.416,3.301)--(6.419,3.303)--(6.422,3.305)--(6.425,3.307)--(6.428,3.309)%
  --(6.430,3.311)--(6.433,3.312)--(6.436,3.314)--(6.439,3.316)--(6.442,3.318)--(6.445,3.320)%
  --(6.448,3.322)--(6.451,3.324)--(6.454,3.326)--(6.457,3.328)--(6.460,3.330)--(6.463,3.332)%
  --(6.466,3.334)--(6.469,3.336)--(6.472,3.338)--(6.475,3.339)--(6.478,3.341)--(6.481,3.343)%
  --(6.484,3.345)--(6.487,3.347)--(6.490,3.349)--(6.493,3.351)--(6.496,3.353)--(6.499,3.355)%
  --(6.502,3.357)--(6.505,3.359)--(6.508,3.361)--(6.511,3.363)--(6.514,3.365)--(6.517,3.367)%
  --(6.520,3.368)--(6.523,3.370)--(6.526,3.372)--(6.529,3.374)--(6.532,3.376)--(6.535,3.378)%
  --(6.538,3.380)--(6.541,3.382)--(6.544,3.384)--(6.547,3.386)--(6.550,3.388)--(6.553,3.390)%
  --(6.556,3.392)--(6.559,3.394)--(6.562,3.396)--(6.565,3.398)--(6.568,3.399)--(6.571,3.401)%
  --(6.574,3.403)--(6.577,3.405)--(6.580,3.407)--(6.583,3.409)--(6.586,3.411)--(6.589,3.413)%
  --(6.592,3.415)--(6.595,3.417)--(6.598,3.419)--(6.601,3.421)--(6.604,3.423)--(6.607,3.425)%
  --(6.610,3.427)--(6.613,3.429)--(6.616,3.431)--(6.619,3.433)--(6.622,3.434)--(6.625,3.436)%
  --(6.628,3.438)--(6.631,3.440)--(6.634,3.442)--(6.637,3.444)--(6.639,3.446)--(6.642,3.448)%
  --(6.645,3.450)--(6.648,3.452)--(6.651,3.454)--(6.654,3.456)--(6.657,3.458)--(6.660,3.460)%
  --(6.663,3.462)--(6.666,3.464)--(6.669,3.466)--(6.672,3.468)--(6.675,3.470)--(6.678,3.471)%
  --(6.681,3.473)--(6.684,3.475)--(6.687,3.477)--(6.690,3.479)--(6.693,3.481)--(6.696,3.483)%
  --(6.699,3.485)--(6.702,3.487)--(6.705,3.489)--(6.708,3.491)--(6.711,3.493)--(6.714,3.495)%
  --(6.717,3.497)--(6.720,3.499)--(6.723,3.501)--(6.726,3.503)--(6.729,3.505)--(6.732,3.507)%
  --(6.735,3.509)--(6.738,3.510)--(6.741,3.512)--(6.744,3.514)--(6.747,3.516)--(6.750,3.518)%
  --(6.753,3.520)--(6.756,3.522)--(6.759,3.524)--(6.762,3.526)--(6.765,3.528)--(6.768,3.530)%
  --(6.771,3.532)--(6.774,3.534)--(6.777,3.536)--(6.780,3.538)--(6.783,3.540)--(6.786,3.542)%
  --(6.789,3.544)--(6.792,3.546)--(6.795,3.548)--(6.798,3.550)--(6.801,3.552)--(6.804,3.554)%
  --(6.807,3.556)--(6.810,3.557)--(6.813,3.559)--(6.816,3.561)--(6.819,3.563)--(6.822,3.565)%
  --(6.825,3.567)--(6.828,3.569)--(6.831,3.571)--(6.834,3.573)--(6.837,3.575)--(6.840,3.577)%
  --(6.843,3.579)--(6.846,3.581)--(6.848,3.583)--(6.851,3.585)--(6.854,3.587)--(6.857,3.589)%
  --(6.860,3.591)--(6.863,3.593)--(6.866,3.595)--(6.869,3.597)--(6.872,3.599)--(6.875,3.601)%
  --(6.878,3.603)--(6.881,3.605)--(6.884,3.607)--(6.887,3.609)--(6.890,3.610)--(6.893,3.612)%
  --(6.896,3.614)--(6.899,3.616)--(6.902,3.618)--(6.905,3.620)--(6.908,3.622)--(6.911,3.624)%
  --(6.914,3.626)--(6.917,3.628)--(6.920,3.630)--(6.923,3.632)--(6.926,3.634)--(6.929,3.636)%
  --(6.932,3.638)--(6.935,3.640)--(6.938,3.642)--(6.941,3.644)--(6.944,3.646)--(6.947,3.648)%
  --(6.950,3.650)--(6.953,3.652)--(6.956,3.654)--(6.959,3.656)--(6.962,3.658)--(6.965,3.660)%
  --(6.968,3.662)--(6.971,3.664)--(6.974,3.666)--(6.977,3.668)--(6.980,3.670)--(6.983,3.672)%
  --(6.986,3.673)--(6.989,3.675)--(6.992,3.677)--(6.995,3.679)--(6.998,3.681)--(7.001,3.683)%
  --(7.004,3.685)--(7.007,3.687)--(7.010,3.689)--(7.013,3.691)--(7.016,3.693)--(7.019,3.695)%
  --(7.022,3.697)--(7.025,3.699)--(7.028,3.701)--(7.031,3.703)--(7.034,3.705)--(7.037,3.707)%
  --(7.040,3.709)--(7.043,3.711)--(7.046,3.713)--(7.049,3.715)--(7.052,3.717)--(7.055,3.719)%
  --(7.057,3.721)--(7.060,3.723)--(7.063,3.725)--(7.066,3.727)--(7.069,3.729)--(7.072,3.731)%
  --(7.075,3.733)--(7.078,3.735)--(7.081,3.737)--(7.084,3.739)--(7.087,3.741)--(7.090,3.743)%
  --(7.093,3.745)--(7.096,3.747)--(7.099,3.749)--(7.102,3.751)--(7.105,3.753)--(7.108,3.755)%
  --(7.111,3.757)--(7.114,3.759)--(7.117,3.760)--(7.120,3.762)--(7.123,3.764)--(7.126,3.766)%
  --(7.129,3.768)--(7.132,3.770)--(7.135,3.772)--(7.138,3.774)--(7.141,3.776)--(7.144,3.778)%
  --(7.147,3.780)--(7.150,3.782)--(7.153,3.784)--(7.156,3.786)--(7.159,3.788)--(7.162,3.790)%
  --(7.165,3.792)--(7.168,3.794)--(7.171,3.796)--(7.174,3.798)--(7.177,3.800)--(7.180,3.802)%
  --(7.183,3.804)--(7.186,3.806)--(7.189,3.808)--(7.192,3.810)--(7.195,3.812)--(7.198,3.814)%
  --(7.201,3.816)--(7.204,3.818)--(7.207,3.820)--(7.210,3.822)--(7.213,3.824)--(7.216,3.826)%
  --(7.219,3.828)--(7.222,3.830)--(7.225,3.832)--(7.228,3.834)--(7.231,3.836)--(7.234,3.838)%
  --(7.237,3.840)--(7.240,3.842)--(7.243,3.844)--(7.246,3.846)--(7.249,3.848)--(7.252,3.850)%
  --(7.255,3.852)--(7.258,3.854)--(7.261,3.856)--(7.264,3.858)--(7.266,3.860)--(7.269,3.862)%
  --(7.272,3.864)--(7.275,3.866)--(7.278,3.868)--(7.281,3.870)--(7.284,3.872)--(7.287,3.874)%
  --(7.290,3.876)--(7.293,3.878)--(7.296,3.880)--(7.299,3.882)--(7.302,3.884)--(7.305,3.886)%
  --(7.308,3.888)--(7.311,3.890)--(7.314,3.892)--(7.317,3.894)--(7.320,3.896)--(7.323,3.898)%
  --(7.326,3.900)--(7.329,3.902)--(7.332,3.904)--(7.335,3.906)--(7.338,3.908)--(7.341,3.910)%
  --(7.344,3.912)--(7.347,3.914)--(7.350,3.916)--(7.353,3.918)--(7.356,3.920)--(7.359,3.922)%
  --(7.362,3.924)--(7.365,3.926)--(7.368,3.928)--(7.371,3.930)--(7.374,3.932)--(7.377,3.934)%
  --(7.380,3.936)--(7.383,3.938)--(7.386,3.940)--(7.389,3.942)--(7.392,3.944)--(7.395,3.946)%
  --(7.398,3.948)--(7.401,3.950)--(7.404,3.952)--(7.407,3.954)--(7.410,3.956)--(7.413,3.958)%
  --(7.416,3.960)--(7.419,3.962)--(7.422,3.964)--(7.425,3.966)--(7.428,3.968)--(7.431,3.970)%
  --(7.434,3.972)--(7.437,3.974)--(7.440,3.976)--(7.443,3.978)--(7.446,3.980)--(7.449,3.982)%
  --(7.452,3.984)--(7.455,3.986)--(7.458,3.988)--(7.461,3.990)--(7.464,3.992)--(7.467,3.994)%
  --(7.470,3.996)--(7.473,3.998)--(7.476,4.000)--(7.478,4.002)--(7.481,4.004)--(7.484,4.006)%
  --(7.487,4.008)--(7.490,4.010)--(7.493,4.012)--(7.496,4.014)--(7.499,4.016)--(7.502,4.018)%
  --(7.505,4.020)--(7.508,4.022)--(7.511,4.024)--(7.514,4.026)--(7.517,4.028)--(7.520,4.030)%
  --(7.523,4.032)--(7.526,4.034)--(7.529,4.036)--(7.532,4.038)--(7.535,4.040)--(7.538,4.042)%
  --(7.541,4.044)--(7.544,4.046)--(7.547,4.048)--(7.550,4.050)--(7.553,4.052)--(7.556,4.054)%
  --(7.559,4.056)--(7.562,4.058)--(7.565,4.060)--(7.568,4.062)--(7.571,4.064)--(7.574,4.066)%
  --(7.577,4.068)--(7.580,4.070)--(7.583,4.072)--(7.586,4.074)--(7.589,4.076)--(7.592,4.078)%
  --(7.595,4.080)--(7.598,4.082)--(7.601,4.084)--(7.604,4.086)--(7.607,4.088)--(7.610,4.090)%
  --(7.613,4.092)--(7.616,4.094)--(7.619,4.096)--(7.622,4.098)--(7.625,4.100)--(7.628,4.102)%
  --(7.631,4.104)--(7.634,4.106)--(7.637,4.108)--(7.640,4.110)--(7.643,4.112)--(7.646,4.114)%
  --(7.649,4.116)--(7.652,4.118)--(7.655,4.120)--(7.658,4.122)--(7.661,4.124)--(7.664,4.126)%
  --(7.667,4.128)--(7.670,4.130)--(7.673,4.132)--(7.676,4.134)--(7.679,4.136)--(7.682,4.138)%
  --(7.685,4.140)--(7.687,4.142)--(7.690,4.144)--(7.693,4.146)--(7.696,4.148)--(7.699,4.150)%
  --(7.702,4.152)--(7.705,4.154)--(7.708,4.156)--(7.711,4.158)--(7.714,4.160)--(7.717,4.162)%
  --(7.720,4.164)--(7.723,4.166)--(7.726,4.168)--(7.729,4.170)--(7.732,4.172)--(7.735,4.174)%
  --(7.738,4.176)--(7.741,4.178)--(7.744,4.180)--(7.747,4.182)--(7.750,4.184)--(7.753,4.186)%
  --(7.756,4.188)--(7.759,4.190)--(7.762,4.192)--(7.765,4.195)--(7.768,4.197)--(7.771,4.199)%
  --(7.774,4.201)--(7.777,4.203)--(7.780,4.205)--(7.783,4.207)--(7.786,4.209)--(7.789,4.211)%
  --(7.792,4.213)--(7.795,4.215)--(7.798,4.217)--(7.801,4.219)--(7.804,4.221)--(7.807,4.223)%
  --(7.810,4.225)--(7.813,4.227)--(7.816,4.229)--(7.819,4.231)--(7.822,4.233)--(7.825,4.235)%
  --(7.828,4.237)--(7.831,4.239)--(7.834,4.241)--(7.837,4.243)--(7.840,4.245)--(7.843,4.247)%
  --(7.846,4.249)--(7.849,4.251)--(7.852,4.253)--(7.855,4.255)--(7.858,4.257)--(7.861,4.259)%
  --(7.864,4.261)--(7.867,4.263)--(7.870,4.265)--(7.873,4.267)--(7.876,4.269)--(7.879,4.271)%
  --(7.882,4.273)--(7.885,4.275)--(7.888,4.277)--(7.891,4.279)--(7.894,4.281)--(7.896,4.283)%
  --(7.899,4.285)--(7.902,4.287)--(7.905,4.289)--(7.908,4.291)--(7.911,4.293)--(7.914,4.295)%
  --(7.917,4.298)--(7.920,4.300)--(7.923,4.302)--(7.926,4.304)--(7.929,4.306)--(7.932,4.308)%
  --(7.935,4.310)--(7.938,4.312)--(7.941,4.314)--(7.944,4.316)--(7.947,4.318)--(7.950,4.320)%
  --(7.953,4.322)--(7.956,4.324)--(7.959,4.326)--(7.962,4.328)--(7.965,4.330)--(7.968,4.332)%
  --(7.971,4.334)--(7.974,4.336)--(7.977,4.338)--(7.980,4.340)--(7.983,4.342)--(7.986,4.344)%
  --(7.989,4.346)--(7.992,4.348)--(7.995,4.350)--(7.998,4.352)--(8.001,4.354)--(8.004,4.356)%
  --(8.007,4.358)--(8.010,4.360)--(8.013,4.362)--(8.016,4.364)--(8.019,4.366)--(8.022,4.368)%
  --(8.025,4.370)--(8.028,4.372)--(8.031,4.374)--(8.034,4.377)--(8.037,4.379)--(8.040,4.381)%
  --(8.043,4.383)--(8.046,4.385)--(8.049,4.387)--(8.052,4.389)--(8.055,4.391)--(8.058,4.393)%
  --(8.061,4.395)--(8.064,4.397)--(8.067,4.399)--(8.070,4.401)--(8.073,4.403)--(8.076,4.405)%
  --(8.079,4.407)--(8.082,4.409)--(8.085,4.411)--(8.088,4.413)--(8.091,4.415)--(8.094,4.417)%
  --(8.097,4.419)--(8.100,4.421)--(8.103,4.423)--(8.105,4.425)--(8.108,4.427)--(8.111,4.429)%
  --(8.114,4.431)--(8.117,4.433)--(8.120,4.435)--(8.123,4.437)--(8.126,4.439)--(8.129,4.441)%
  --(8.132,4.443)--(8.135,4.446)--(8.138,4.448)--(8.141,4.450)--(8.144,4.452)--(8.147,4.454)%
  --(8.150,4.456)--(8.153,4.458)--(8.156,4.460)--(8.159,4.462)--(8.162,4.464)--(8.165,4.466)%
  --(8.168,4.468)--(8.171,4.470)--(8.174,4.472)--(8.177,4.474)--(8.180,4.476)--(8.183,4.478)%
  --(8.186,4.480)--(8.189,4.482)--(8.192,4.484)--(8.195,4.486)--(8.198,4.488)--(8.201,4.490)%
  --(8.204,4.492)--(8.207,4.494)--(8.210,4.496)--(8.213,4.498)--(8.216,4.500)--(8.219,4.502)%
  --(8.222,4.504)--(8.225,4.507)--(8.228,4.509)--(8.231,4.511)--(8.234,4.513)--(8.237,4.515)%
  --(8.240,4.517)--(8.243,4.519)--(8.246,4.521)--(8.249,4.523)--(8.252,4.525)--(8.255,4.527)%
  --(8.258,4.529)--(8.261,4.531)--(8.264,4.533)--(8.267,4.535)--(8.270,4.537)--(8.273,4.539)%
  --(8.276,4.541)--(8.279,4.543)--(8.282,4.545)--(8.285,4.547)--(8.288,4.549)--(8.291,4.551)%
  --(8.294,4.553)--(8.297,4.555)--(8.300,4.557)--(8.303,4.559)--(8.306,4.561)--(8.309,4.564)%
  --(8.312,4.566)--(8.314,4.568)--(8.317,4.570)--(8.320,4.572)--(8.323,4.574)--(8.326,4.576)%
  --(8.329,4.578)--(8.332,4.580)--(8.335,4.582)--(8.338,4.584)--(8.341,4.586)--(8.344,4.588)%
  --(8.347,4.590)--(8.350,4.592)--(8.353,4.594)--(8.356,4.596)--(8.359,4.598)--(8.362,4.600)%
  --(8.365,4.602)--(8.368,4.604)--(8.371,4.606)--(8.374,4.608)--(8.377,4.610)--(8.380,4.612)%
  --(8.383,4.614)--(8.386,4.617)--(8.389,4.619)--(8.392,4.621)--(8.395,4.623)--(8.398,4.625)%
  --(8.401,4.627)--(8.404,4.629)--(8.407,4.631)--(8.410,4.633)--(8.413,4.635)--(8.416,4.637)%
  --(8.419,4.639)--(8.422,4.641)--(8.425,4.643)--(8.428,4.645)--(8.431,4.647)--(8.434,4.649)%
  --(8.437,4.651)--(8.440,4.653)--(8.443,4.655)--(8.446,4.657)--(8.449,4.659)--(8.452,4.661)%
  --(8.455,4.663)--(8.458,4.666)--(8.461,4.668)--(8.464,4.670)--(8.467,4.672)--(8.470,4.674)%
  --(8.473,4.676)--(8.476,4.678)--(8.479,4.680)--(8.482,4.682)--(8.485,4.684)--(8.488,4.686)%
  --(8.491,4.688)--(8.494,4.690)--(8.497,4.692)--(8.500,4.694)--(8.503,4.696)--(8.506,4.698)%
  --(8.509,4.700)--(8.512,4.702)--(8.515,4.704)--(8.518,4.706)--(8.521,4.708)--(8.523,4.710)%
  --(8.526,4.713)--(8.529,4.715)--(8.532,4.717)--(8.535,4.719)--(8.538,4.721)--(8.541,4.723)%
  --(8.544,4.725)--(8.547,4.727)--(8.550,4.729)--(8.553,4.731)--(8.556,4.733)--(8.559,4.735)%
  --(8.562,4.737)--(8.565,4.739)--(8.568,4.741)--(8.571,4.743)--(8.574,4.745)--(8.577,4.747)%
  --(8.580,4.749)--(8.583,4.751)--(8.586,4.753)--(8.589,4.755)--(8.592,4.758)--(8.595,4.760)%
  --(8.598,4.762)--(8.601,4.764)--(8.604,4.766)--(8.607,4.768)--(8.610,4.770)--(8.613,4.772)%
  --(8.616,4.774)--(8.619,4.776)--(8.622,4.778)--(8.625,4.780)--(8.628,4.782)--(8.631,4.784)%
  --(8.634,4.786)--(8.637,4.788)--(8.640,4.790)--(8.643,4.792)--(8.646,4.794)--(8.649,4.796)%
  --(8.652,4.798)--(8.655,4.801)--(8.658,4.803)--(8.661,4.805)--(8.664,4.807)--(8.667,4.809)%
  --(8.670,4.811)--(8.673,4.813)--(8.676,4.815)--(8.679,4.817)--(8.682,4.819)--(8.685,4.821)%
  --(8.688,4.823)--(8.691,4.825)--(8.694,4.827)--(8.697,4.829)--(8.700,4.831)--(8.703,4.833)%
  --(8.706,4.835)--(8.709,4.837)--(8.712,4.839)--(8.715,4.842)--(8.718,4.844)--(8.721,4.846)%
  --(8.724,4.848)--(8.727,4.850)--(8.730,4.852)--(8.733,4.854)--(8.735,4.856)--(8.738,4.858)%
  --(8.741,4.860)--(8.744,4.862)--(8.747,4.864)--(8.750,4.866)--(8.753,4.868)--(8.756,4.870)%
  --(8.759,4.872)--(8.762,4.874)--(8.765,4.876)--(8.768,4.878)--(8.771,4.880)--(8.774,4.883)%
  --(8.777,4.885)--(8.780,4.887)--(8.783,4.889)--(8.786,4.891)--(8.789,4.893)--(8.792,4.895)%
  --(8.795,4.897)--(8.798,4.899)--(8.801,4.901)--(8.804,4.903)--(8.807,4.905)--(8.810,4.907)%
  --(8.813,4.909)--(8.816,4.911)--(8.819,4.913)--(8.822,4.915)--(8.825,4.917)--(8.828,4.919)%
  --(8.831,4.922)--(8.834,4.924)--(8.837,4.926)--(8.840,4.928)--(8.843,4.930)--(8.846,4.932)%
  --(8.849,4.934)--(8.852,4.936)--(8.855,4.938)--(8.858,4.940)--(8.861,4.942)--(8.864,4.944)%
  --(8.867,4.946)--(8.870,4.948)--(8.873,4.950)--(8.876,4.952)--(8.879,4.954)--(8.882,4.956)%
  --(8.885,4.958)--(8.888,4.961)--(8.891,4.963)--(8.894,4.965)--(8.897,4.967)--(8.900,4.969)%
  --(8.903,4.971)--(8.906,4.973)--(8.909,4.975)--(8.912,4.977)--(8.915,4.979)--(8.918,4.981)%
  --(8.921,4.983)--(8.924,4.985)--(8.927,4.987)--(8.930,4.989)--(8.933,4.991)--(8.936,4.993)%
  --(8.939,4.995)--(8.942,4.998)--(8.944,5.000)--(8.947,5.002)--(8.950,5.004)--(8.953,5.006)%
  --(8.956,5.008)--(8.959,5.010)--(8.962,5.012)--(8.965,5.014)--(8.968,5.016)--(8.971,5.018)%
  --(8.974,5.020)--(8.977,5.022)--(8.980,5.024)--(8.983,5.026)--(8.986,5.028)--(8.989,5.030)%
  --(8.992,5.033)--(8.995,5.035)--(8.998,5.037)--(9.001,5.039)--(9.004,5.041)--(9.007,5.043)%
  --(9.010,5.045)--(9.013,5.047)--(9.016,5.049)--(9.019,5.051)--(9.022,5.053)--(9.025,5.055)%
  --(9.028,5.057)--(9.031,5.059)--(9.034,5.061)--(9.037,5.063)--(9.040,5.065)--(9.043,5.067)%
  --(9.046,5.070)--(9.049,5.072)--(9.052,5.074)--(9.055,5.076)--(9.058,5.078)--(9.061,5.080)%
  --(9.064,5.082)--(9.067,5.084)--(9.070,5.086)--(9.073,5.088)--(9.076,5.090)--(9.079,5.092)%
  --(9.082,5.094)--(9.085,5.096)--(9.088,5.098)--(9.091,5.100)--(9.094,5.102)--(9.097,5.105)%
  --(9.100,5.107)--(9.103,5.109)--(9.106,5.111)--(9.109,5.113)--(9.112,5.115)--(9.115,5.117)%
  --(9.118,5.119)--(9.121,5.121)--(9.124,5.123)--(9.127,5.125)--(9.130,5.127)--(9.133,5.129)%
  --(9.136,5.131)--(9.139,5.133)--(9.142,5.135)--(9.145,5.138)--(9.148,5.140)--(9.151,5.142)%
  --(9.153,5.144)--(9.156,5.146)--(9.159,5.148)--(9.162,5.150)--(9.165,5.152)--(9.168,5.154)%
  --(9.171,5.156)--(9.174,5.158)--(9.177,5.160)--(9.180,5.162)--(9.183,5.164)--(9.186,5.166)%
  --(9.189,5.168)--(9.192,5.170)--(9.195,5.173)--(9.198,5.175)--(9.201,5.177)--(9.204,5.179)%
  --(9.207,5.181)--(9.210,5.183)--(9.213,5.185)--(9.216,5.187)--(9.219,5.189)--(9.222,5.191)%
  --(9.225,5.193)--(9.228,5.195)--(9.231,5.197)--(9.234,5.199)--(9.237,5.201)--(9.240,5.203)%
  --(9.243,5.206)--(9.246,5.208)--(9.249,5.210)--(9.252,5.212)--(9.255,5.214)--(9.258,5.216)%
  --(9.261,5.218)--(9.264,5.220)--(9.267,5.222)--(9.270,5.224)--(9.273,5.226)--(9.276,5.228)%
  --(9.279,5.230)--(9.282,5.232)--(9.285,5.234)--(9.288,5.237)--(9.291,5.239)--(9.294,5.241)%
  --(9.297,5.243)--(9.300,5.245)--(9.303,5.247)--(9.306,5.249)--(9.309,5.251)--(9.312,5.253)%
  --(9.315,5.255)--(9.318,5.257)--(9.321,5.259)--(9.324,5.261)--(9.327,5.263)--(9.330,5.265)%
  --(9.333,5.267)--(9.336,5.270)--(9.339,5.272)--(9.342,5.274)--(9.345,5.276)--(9.348,5.278)%
  --(9.351,5.280)--(9.354,5.282)--(9.357,5.284)--(9.360,5.286)--(9.362,5.288)--(9.365,5.290)%
  --(9.368,5.292)--(9.371,5.294)--(9.374,5.296)--(9.377,5.298)--(9.380,5.301)--(9.383,5.303)%
  --(9.386,5.305)--(9.389,5.307)--(9.392,5.309)--(9.395,5.311)--(9.398,5.313)--(9.401,5.315)%
  --(9.404,5.317)--(9.407,5.319)--(9.410,5.321)--(9.413,5.323)--(9.416,5.325)--(9.419,5.327)%
  --(9.422,5.329)--(9.425,5.332)--(9.428,5.334)--(9.431,5.336)--(9.434,5.338)--(9.437,5.340)%
  --(9.440,5.342)--(9.443,5.344)--(9.446,5.346)--(9.449,5.348)--(9.452,5.350)--(9.455,5.352)%
  --(9.458,5.354)--(9.461,5.356)--(9.464,5.358)--(9.467,5.360)--(9.470,5.363)--(9.473,5.365)%
  --(9.476,5.367)--(9.479,5.369)--(9.482,5.371)--(9.485,5.373)--(9.488,5.375)--(9.491,5.377)%
  --(9.494,5.379)--(9.497,5.381)--(9.500,5.383)--(9.503,5.385)--(9.506,5.387)--(9.509,5.389)%
  --(9.512,5.391)--(9.515,5.394)--(9.518,5.396)--(9.521,5.398)--(9.524,5.400)--(9.527,5.402)%
  --(9.530,5.404)--(9.533,5.406)--(9.536,5.408)--(9.539,5.410)--(9.542,5.412)--(9.545,5.414)%
  --(9.548,5.416)--(9.551,5.418)--(9.554,5.420)--(9.557,5.423)--(9.560,5.425)--(9.563,5.427)%
  --(9.566,5.429)--(9.569,5.431)--(9.571,5.433)--(9.574,5.435)--(9.577,5.437)--(9.580,5.439)%
  --(9.583,5.441)--(9.586,5.443)--(9.589,5.445)--(9.592,5.447)--(9.595,5.449)--(9.598,5.451)%
  --(9.601,5.454)--(9.604,5.456)--(9.607,5.458)--(9.610,5.460)--(9.613,5.462)--(9.616,5.464)%
  --(9.619,5.466)--(9.622,5.468)--(9.625,5.470)--(9.628,5.472)--(9.631,5.474)--(9.634,5.476)%
  --(9.637,5.478)--(9.640,5.480)--(9.643,5.483)--(9.646,5.485)--(9.649,5.487)--(9.652,5.489)%
  --(9.655,5.491)--(9.658,5.493)--(9.661,5.495)--(9.664,5.497)--(9.667,5.499)--(9.670,5.501)%
  --(9.673,5.503)--(9.676,5.505)--(9.679,5.507)--(9.682,5.509)--(9.685,5.512)--(9.688,5.514)%
  --(9.691,5.516)--(9.694,5.518)--(9.697,5.520)--(9.700,5.522)--(9.703,5.524)--(9.706,5.526)%
  --(9.709,5.528)--(9.712,5.530)--(9.715,5.532)--(9.718,5.534)--(9.721,5.536)--(9.724,5.538)%
  --(9.727,5.541)--(9.730,5.543)--(9.733,5.545)--(9.736,5.547)--(9.739,5.549)--(9.742,5.551)%
  --(9.745,5.553)--(9.748,5.555)--(9.751,5.557)--(9.754,5.559)--(9.757,5.561)--(9.760,5.563)%
  --(9.763,5.565)--(9.766,5.567)--(9.769,5.570)--(9.772,5.572)--(9.775,5.574)--(9.778,5.576)%
  --(9.780,5.578)--(9.783,5.580)--(9.786,5.582)--(9.789,5.584)--(9.792,5.586)--(9.795,5.588)%
  --(9.798,5.590)--(9.801,5.592)--(9.804,5.594)--(9.807,5.597)--(9.810,5.599)--(9.813,5.601)%
  --(9.816,5.603)--(9.819,5.605)--(9.822,5.607)--(9.825,5.609)--(9.828,5.611)--(9.831,5.613)%
  --(9.834,5.615)--(9.837,5.617)--(9.840,5.619)--(9.843,5.621)--(9.846,5.623)--(9.849,5.626)%
  --(9.852,5.628)--(9.855,5.630)--(9.858,5.632)--(9.861,5.634)--(9.864,5.636)--(9.867,5.638)%
  --(9.870,5.640)--(9.873,5.642)--(9.876,5.644)--(9.879,5.646)--(9.882,5.648)--(9.885,5.650)%
  --(9.888,5.653)--(9.891,5.655)--(9.894,5.657)--(9.897,5.659)--(9.900,5.661)--(9.903,5.663)%
  --(9.906,5.665)--(9.909,5.667)--(9.912,5.669)--(9.915,5.671)--(9.918,5.673)--(9.921,5.675)%
  --(9.924,5.677)--(9.927,5.679)--(9.930,5.682)--(9.933,5.684)--(9.936,5.686)--(9.939,5.688)%
  --(9.942,5.690)--(9.945,5.692)--(9.948,5.694)--(9.951,5.696)--(9.954,5.698)--(9.957,5.700)%
  --(9.960,5.702)--(9.963,5.704)--(9.966,5.706)--(9.969,5.709)--(9.972,5.711)--(9.975,5.713)%
  --(9.978,5.715)--(9.981,5.717)--(9.984,5.719)--(9.987,5.721)--(9.990,5.723)--(9.992,5.725)%
  --(9.995,5.727)--(9.998,5.729)--(10.001,5.731)--(10.004,5.733)--(10.007,5.736)--(10.010,5.738)%
  --(10.013,5.740)--(10.016,5.742)--(10.019,5.744)--(10.022,5.746)--(10.025,5.748)--(10.028,5.750)%
  --(10.031,5.752)--(10.034,5.754)--(10.037,5.756)--(10.040,5.758)--(10.043,5.760)--(10.046,5.763)%
  --(10.049,5.765)--(10.052,5.767)--(10.055,5.769)--(10.058,5.771)--(10.061,5.773)--(10.064,5.775)%
  --(10.067,5.777)--(10.070,5.779)--(10.073,5.781)--(10.076,5.783)--(10.079,5.785)--(10.082,5.787)%
  --(10.085,5.790)--(10.088,5.792)--(10.091,5.794)--(10.094,5.796)--(10.097,5.798)--(10.100,5.800)%
  --(10.103,5.802)--(10.106,5.804)--(10.109,5.806)--(10.112,5.808)--(10.115,5.810)--(10.118,5.812)%
  --(10.121,5.815)--(10.124,5.817)--(10.127,5.819)--(10.130,5.821)--(10.133,5.823)--(10.136,5.825)%
  --(10.139,5.827)--(10.142,5.829)--(10.145,5.831)--(10.148,5.833)--(10.151,5.835)--(10.154,5.837)%
  --(10.157,5.839)--(10.160,5.842)--(10.163,5.844)--(10.166,5.846)--(10.169,5.848)--(10.172,5.850)%
  --(10.175,5.852)--(10.178,5.854)--(10.181,5.856)--(10.184,5.858)--(10.187,5.860)--(10.190,5.862)%
  --(10.193,5.864)--(10.196,5.866)--(10.199,5.869)--(10.201,5.871)--(10.204,5.873)--(10.207,5.875)%
  --(10.210,5.877)--(10.213,5.879)--(10.216,5.881)--(10.219,5.883)--(10.222,5.885)--(10.225,5.887)%
  --(10.228,5.889)--(10.231,5.891)--(10.234,5.894)--(10.237,5.896)--(10.240,5.898)--(10.243,5.900)%
  --(10.246,5.902)--(10.249,5.904)--(10.252,5.906)--(10.255,5.908)--(10.258,5.910)--(10.261,5.912)%
  --(10.264,5.914)--(10.267,5.916)--(10.270,5.918)--(10.273,5.921)--(10.276,5.923)--(10.279,5.925)%
  --(10.282,5.927)--(10.285,5.929)--(10.288,5.931)--(10.291,5.933)--(10.294,5.935)--(10.297,5.937)%
  --(10.300,5.939)--(10.303,5.941)--(10.306,5.943)--(10.309,5.946)--(10.312,5.948)--(10.315,5.950)%
  --(10.318,5.952)--(10.321,5.954)--(10.324,5.956)--(10.327,5.958)--(10.330,5.960)--(10.333,5.962)%
  --(10.336,5.964)--(10.339,5.966)--(10.342,5.968)--(10.345,5.971)--(10.348,5.973)--(10.351,5.975)%
  --(10.354,5.977)--(10.357,5.979)--(10.360,5.981)--(10.363,5.983)--(10.366,5.985)--(10.369,5.987)%
  --(10.372,5.989)--(10.375,5.991)--(10.378,5.993)--(10.381,5.996)--(10.384,5.998)--(10.387,6.000)%
  --(10.390,6.002)--(10.393,6.004)--(10.396,6.006)--(10.399,6.008)--(10.402,6.010)--(10.405,6.012)%
  --(10.408,6.014)--(10.410,6.016)--(10.413,6.018)--(10.416,6.020)--(10.419,6.023)--(10.422,6.025)%
  --(10.425,6.027)--(10.428,6.029)--(10.431,6.031)--(10.434,6.033)--(10.437,6.035)--(10.440,6.037)%
  --(10.443,6.039)--(10.446,6.041)--(10.449,6.043)--(10.452,6.045)--(10.455,6.048)--(10.458,6.050)%
  --(10.461,6.052)--(10.464,6.054)--(10.467,6.056)--(10.470,6.058)--(10.473,6.060)--(10.476,6.062)%
  --(10.479,6.064)--(10.482,6.066)--(10.485,6.068)--(10.488,6.070)--(10.491,6.073)--(10.494,6.075)%
  --(10.497,6.077)--(10.500,6.079)--(10.503,6.081)--(10.506,6.083)--(10.509,6.085)--(10.512,6.087)%
  --(10.515,6.089)--(10.518,6.091)--(10.521,6.093)--(10.524,6.095)--(10.527,6.098)--(10.530,6.100)%
  --(10.533,6.102)--(10.536,6.104)--(10.539,6.106)--(10.542,6.108)--(10.545,6.110)--(10.548,6.112)%
  --(10.551,6.114)--(10.554,6.116)--(10.557,6.118)--(10.560,6.121)--(10.563,6.123)--(10.566,6.125)%
  --(10.569,6.127)--(10.572,6.129)--(10.575,6.131)--(10.578,6.133)--(10.581,6.135)--(10.584,6.137)%
  --(10.587,6.139)--(10.590,6.141)--(10.593,6.143)--(10.596,6.146)--(10.599,6.148)--(10.602,6.150)%
  --(10.605,6.152)--(10.608,6.154)--(10.611,6.156)--(10.614,6.158)--(10.617,6.160)--(10.619,6.162)%
  --(10.622,6.164)--(10.625,6.166)--(10.628,6.168)--(10.631,6.171)--(10.634,6.173)--(10.637,6.175)%
  --(10.640,6.177)--(10.643,6.179)--(10.646,6.181)--(10.649,6.183)--(10.652,6.185)--(10.655,6.187)%
  --(10.658,6.189)--(10.661,6.191)--(10.664,6.193)--(10.667,6.196)--(10.670,6.198)--(10.673,6.200)%
  --(10.676,6.202)--(10.679,6.204)--(10.682,6.206)--(10.685,6.208)--(10.688,6.210)--(10.691,6.212)%
  --(10.694,6.214)--(10.697,6.216)--(10.700,6.219)--(10.703,6.221)--(10.706,6.223)--(10.709,6.225)%
  --(10.712,6.227)--(10.715,6.229)--(10.718,6.231)--(10.721,6.233)--(10.724,6.235)--(10.727,6.237)%
  --(10.730,6.239)--(10.733,6.241)--(10.736,6.244)--(10.739,6.246)--(10.742,6.248)--(10.745,6.250)%
  --(10.748,6.252)--(10.751,6.254)--(10.754,6.256)--(10.757,6.258)--(10.760,6.260)--(10.763,6.262)%
  --(10.766,6.264)--(10.769,6.267)--(10.772,6.269)--(10.775,6.271)--(10.778,6.273)--(10.781,6.275)%
  --(10.784,6.277)--(10.787,6.279)--(10.790,6.281)--(10.793,6.283)--(10.796,6.285)--(10.799,6.287)%
  --(10.802,6.289)--(10.805,6.292)--(10.808,6.294)--(10.811,6.296)--(10.814,6.298)--(10.817,6.300)%
  --(10.820,6.302)--(10.823,6.304)--(10.826,6.306)--(10.828,6.308)--(10.831,6.310)--(10.834,6.312)%
  --(10.837,6.315)--(10.840,6.317)--(10.843,6.319)--(10.846,6.321)--(10.849,6.323)--(10.852,6.325)%
  --(10.855,6.327)--(10.858,6.329)--(10.861,6.331)--(10.864,6.333)--(10.867,6.335)--(10.870,6.337)%
  --(10.873,6.340)--(10.876,6.342)--(10.879,6.344)--(10.882,6.346)--(10.885,6.348)--(10.888,6.350)%
  --(10.891,6.352)--(10.894,6.354)--(10.897,6.356)--(10.900,6.358)--(10.903,6.360)--(10.906,6.363)%
  --(10.909,6.365)--(10.912,6.367)--(10.915,6.369)--(10.918,6.371)--(10.921,6.373)--(10.924,6.375)%
  --(10.927,6.377)--(10.930,6.379)--(10.933,6.381)--(10.936,6.383)--(10.939,6.386)--(10.942,6.388)%
  --(10.945,6.390)--(10.948,6.392)--(10.951,6.394)--(10.954,6.396)--(10.957,6.398)--(10.960,6.400)%
  --(10.963,6.402)--(10.966,6.404)--(10.969,6.406)--(10.972,6.408)--(10.975,6.411)--(10.978,6.413)%
  --(10.981,6.415)--(10.984,6.417)--(10.987,6.419)--(10.990,6.421)--(10.993,6.423)--(10.996,6.425)%
  --(10.999,6.427)--(11.002,6.429)--(11.005,6.431)--(11.008,6.434)--(11.011,6.436)--(11.014,6.438)%
  --(11.017,6.440)--(11.020,6.442)--(11.023,6.444)--(11.026,6.446)--(11.029,6.448)--(11.032,6.450)%
  --(11.035,6.452)--(11.037,6.454)--(11.040,6.457)--(11.043,6.459)--(11.046,6.461)--(11.049,6.463)%
  --(11.052,6.465)--(11.055,6.467)--(11.058,6.469)--(11.061,6.471)--(11.064,6.473)--(11.067,6.475)%
  --(11.070,6.477)--(11.073,6.480)--(11.076,6.482)--(11.079,6.484)--(11.082,6.486)--(11.085,6.488)%
  --(11.088,6.490)--(11.091,6.492)--(11.094,6.494)--(11.097,6.496)--(11.100,6.498)--(11.103,6.500)%
  --(11.106,6.503)--(11.109,6.505)--(11.112,6.507)--(11.115,6.509)--(11.118,6.511)--(11.121,6.513)%
  --(11.124,6.515)--(11.127,6.517)--(11.130,6.519)--(11.133,6.521)--(11.136,6.523)--(11.139,6.526)%
  --(11.142,6.528)--(11.145,6.530)--(11.148,6.532)--(11.151,6.534)--(11.154,6.536)--(11.157,6.538)%
  --(11.160,6.540)--(11.163,6.542)--(11.166,6.544)--(11.169,6.546)--(11.172,6.549)--(11.175,6.551)%
  --(11.178,6.553)--(11.181,6.555)--(11.184,6.557)--(11.187,6.559)--(11.190,6.561)--(11.193,6.563)%
  --(11.196,6.565)--(11.199,6.567)--(11.202,6.569)--(11.205,6.572)--(11.208,6.574)--(11.211,6.576)%
  --(11.214,6.578)--(11.217,6.580)--(11.220,6.582)--(11.223,6.584)--(11.226,6.586)--(11.229,6.588)%
  --(11.232,6.590)--(11.235,6.592)--(11.238,6.595)--(11.241,6.597)--(11.244,6.599)--(11.247,6.601)%
  --(11.249,6.603)--(11.252,6.605)--(11.255,6.607)--(11.258,6.609)--(11.261,6.611)--(11.264,6.613)%
  --(11.267,6.615)--(11.270,6.618)--(11.273,6.620)--(11.276,6.622)--(11.279,6.624)--(11.282,6.626)%
  --(11.285,6.628)--(11.288,6.630)--(11.291,6.632)--(11.294,6.634)--(11.297,6.636)--(11.300,6.638)%
  --(11.303,6.641)--(11.306,6.643)--(11.309,6.645)--(11.312,6.647)--(11.315,6.649)--(11.318,6.651)%
  --(11.321,6.653)--(11.324,6.655)--(11.327,6.657)--(11.330,6.659)--(11.333,6.661)--(11.336,6.664)%
  --(11.339,6.666)--(11.342,6.668)--(11.345,6.670)--(11.348,6.672)--(11.351,6.674)--(11.354,6.676)%
  --(11.357,6.678)--(11.360,6.680)--(11.363,6.682)--(11.366,6.684)--(11.369,6.687)--(11.372,6.689)%
  --(11.375,6.691)--(11.378,6.693)--(11.381,6.695)--(11.384,6.697)--(11.387,6.699)--(11.390,6.701)%
  --(11.393,6.703)--(11.396,6.705)--(11.399,6.708)--(11.402,6.710)--(11.405,6.712)--(11.408,6.714)%
  --(11.411,6.716)--(11.414,6.718)--(11.417,6.720)--(11.420,6.722)--(11.423,6.724)--(11.426,6.726)%
  --(11.429,6.728)--(11.432,6.731)--(11.435,6.733)--(11.438,6.735)--(11.441,6.737)--(11.444,6.739)%
  --(11.447,6.741)--(11.450,6.743)--(11.453,6.745)--(11.456,6.747)--(11.458,6.749)--(11.461,6.751)%
  --(11.464,6.754)--(11.467,6.756)--(11.470,6.758)--(11.473,6.760)--(11.476,6.762)--(11.479,6.764)%
  --(11.482,6.766)--(11.485,6.768)--(11.488,6.770)--(11.491,6.772)--(11.494,6.775)--(11.497,6.777)%
  --(11.500,6.779)--(11.503,6.781)--(11.506,6.783)--(11.509,6.785)--(11.512,6.787)--(11.515,6.789)%
  --(11.518,6.791)--(11.521,6.793)--(11.524,6.795)--(11.527,6.798)--(11.530,6.800)--(11.533,6.802)%
  --(11.536,6.804)--(11.539,6.806)--(11.542,6.808)--(11.545,6.810)--(11.548,6.812)--(11.551,6.814)%
  --(11.554,6.816)--(11.557,6.818)--(11.560,6.821)--(11.563,6.823)--(11.566,6.825)--(11.569,6.827)%
  --(11.572,6.829)--(11.575,6.831)--(11.578,6.833)--(11.581,6.835)--(11.584,6.837)--(11.587,6.839)%
  --(11.590,6.842)--(11.593,6.844)--(11.596,6.846)--(11.599,6.848)--(11.602,6.850)--(11.605,6.852)%
  --(11.608,6.854)--(11.611,6.856)--(11.614,6.858)--(11.617,6.860)--(11.620,6.862)--(11.623,6.865)%
  --(11.626,6.867)--(11.629,6.869)--(11.632,6.871)--(11.635,6.873)--(11.638,6.875)--(11.641,6.877)%
  --(11.644,6.879)--(11.647,6.881)--(11.650,6.883)--(11.653,6.886)--(11.656,6.888)--(11.659,6.890)%
  --(11.662,6.892)--(11.665,6.894)--(11.667,6.896)--(11.670,6.898)--(11.673,6.900)--(11.676,6.902)%
  --(11.679,6.904)--(11.682,6.906)--(11.685,6.909)--(11.688,6.911)--(11.691,6.913)--(11.694,6.915)%
  --(11.697,6.917)--(11.700,6.919)--(11.703,6.921)--(11.706,6.923)--(11.709,6.925)--(11.712,6.927)%
  --(11.715,6.930)--(11.718,6.932)--(11.721,6.934)--(11.724,6.936)--(11.727,6.938)--(11.730,6.940)%
  --(11.733,6.942)--(11.736,6.944)--(11.739,6.946)--(11.742,6.948)--(11.745,6.950)--(11.748,6.953)%
  --(11.751,6.955)--(11.754,6.957)--(11.757,6.959)--(11.760,6.961)--(11.763,6.963)--(11.766,6.965)%
  --(11.769,6.967)--(11.772,6.969)--(11.775,6.971)--(11.778,6.974)--(11.781,6.976)--(11.784,6.978)%
  --(11.787,6.980)--(11.790,6.982)--(11.793,6.984)--(11.796,6.986)--(11.799,6.988)--(11.802,6.990)%
  --(11.805,6.992)--(11.808,6.995)--(11.811,6.997)--(11.814,6.999)--(11.817,7.001)--(11.820,7.003)%
  --(11.823,7.005)--(11.826,7.007)--(11.829,7.009)--(11.832,7.011)--(11.835,7.013)--(11.838,7.015)%
  --(11.841,7.018)--(11.844,7.020)--(11.847,7.022)--(11.850,7.024)--(11.853,7.026)--(11.856,7.028)%
  --(11.859,7.030)--(11.862,7.032)--(11.865,7.034)--(11.868,7.036)--(11.871,7.039)--(11.874,7.041)%
  --(11.876,7.043)--(11.879,7.045)--(11.882,7.047)--(11.885,7.049)--(11.888,7.051)--(11.891,7.053)%
  --(11.894,7.055)--(11.897,7.057)--(11.900,7.060)--(11.903,7.062)--(11.906,7.064)--(11.909,7.066)%
  --(11.912,7.068)--(11.915,7.070)--(11.918,7.072)--(11.921,7.074)--(11.924,7.076)--(11.927,7.078)%
  --(11.930,7.080)--(11.933,7.083)--(11.936,7.085)--(11.939,7.087)--(11.942,7.089)--(11.945,7.091)%
  --(11.948,7.093)--(11.951,7.095)--(11.954,7.097)--(11.957,7.099)--(11.960,7.101)--(11.963,7.104)%
  --(11.966,7.106)--(11.969,7.108)--(11.972,7.110)--(11.975,7.112)--(11.978,7.114)--(11.981,7.116)%
  --(11.984,7.118)--(11.987,7.120)--(11.990,7.122)--(11.993,7.125)--(11.996,7.127)--(11.999,7.129)%
  --(12.002,7.131)--(12.005,7.133)--(12.008,7.135)--(12.011,7.137)--(12.014,7.139)--(12.017,7.141)%
  --(12.020,7.143)--(12.023,7.146)--(12.026,7.148)--(12.029,7.150)--(12.032,7.152)--(12.035,7.154)%
  --(12.038,7.156)--(12.041,7.158)--(12.044,7.160)--(12.047,7.162)--(12.050,7.164)--(12.053,7.166)%
  --(12.056,7.169)--(12.059,7.171)--(12.062,7.173)--(12.065,7.175)--(12.068,7.177)--(12.071,7.179)%
  --(12.074,7.181)--(12.077,7.183)--(12.080,7.185)--(12.083,7.187)--(12.085,7.190)--(12.088,7.192)%
  --(12.091,7.194)--(12.094,7.196)--(12.097,7.198)--(12.100,7.200)--(12.103,7.202)--(12.106,7.204)%
  --(12.109,7.206)--(12.112,7.208)--(12.115,7.211)--(12.118,7.213)--(12.121,7.215)--(12.124,7.217)%
  --(12.127,7.219)--(12.130,7.221)--(12.133,7.223)--(12.136,7.225)--(12.139,7.227)--(12.142,7.229)%
  --(12.145,7.232)--(12.148,7.234)--(12.151,7.236)--(12.154,7.238)--(12.157,7.240)--(12.160,7.242)%
  --(12.163,7.244)--(12.166,7.246)--(12.169,7.248)--(12.172,7.250)--(12.175,7.253)--(12.178,7.255)%
  --(12.181,7.257)--(12.184,7.259)--(12.187,7.261)--(12.190,7.263)--(12.193,7.265)--(12.196,7.267)%
  --(12.199,7.269)--(12.202,7.271)--(12.205,7.274)--(12.208,7.276)--(12.211,7.278)--(12.214,7.280)%
  --(12.217,7.282)--(12.220,7.284)--(12.223,7.286)--(12.226,7.288)--(12.229,7.290)--(12.232,7.292)%
  --(12.235,7.295)--(12.238,7.297)--(12.241,7.299)--(12.244,7.301)--(12.247,7.303)--(12.250,7.305)%
  --(12.253,7.307)--(12.256,7.309)--(12.259,7.311)--(12.262,7.313)--(12.265,7.316)--(12.268,7.318)%
  --(12.271,7.320)--(12.274,7.322)--(12.277,7.324)--(12.280,7.326)--(12.283,7.328)--(12.286,7.330)%
  --(12.289,7.332)--(12.292,7.334)--(12.295,7.337)--(12.297,7.339)--(12.300,7.341)--(12.303,7.343)%
  --(12.306,7.345)--(12.309,7.347)--(12.312,7.349)--(12.315,7.351)--(12.318,7.353)--(12.321,7.355)%
  --(12.324,7.358)--(12.327,7.360)--(12.330,7.362)--(12.333,7.364)--(12.336,7.366)--(12.339,7.368)%
  --(12.342,7.370)--(12.345,7.372)--(12.348,7.374)--(12.351,7.376)--(12.354,7.379)--(12.357,7.381)%
  --(12.360,7.383)--(12.363,7.385)--(12.366,7.387)--(12.369,7.389)--(12.372,7.391)--(12.375,7.393)%
  --(12.378,7.395)--(12.381,7.397)--(12.384,7.400)--(12.387,7.402)--(12.390,7.404)--(12.393,7.406)%
  --(12.396,7.408)--(12.399,7.410)--(12.402,7.412)--(12.405,7.414)--(12.408,7.416)--(12.411,7.418)%
  --(12.414,7.421)--(12.417,7.423)--(12.420,7.425)--(12.423,7.427)--(12.426,7.429)--(12.429,7.431)%
  --(12.432,7.433)--(12.435,7.435)--(12.438,7.437)--(12.441,7.439)--(12.444,7.442)--(12.447,7.444)%
  --(12.450,7.446)--(12.453,7.448)--(12.456,7.450)--(12.459,7.452)--(12.462,7.454)--(12.465,7.456)%
  --(12.468,7.458)--(12.471,7.460)--(12.474,7.463)--(12.477,7.465)--(12.480,7.467)--(12.483,7.469)%
  --(12.486,7.471)--(12.489,7.473)--(12.492,7.475)--(12.495,7.477)--(12.498,7.479)--(12.501,7.481)%
  --(12.504,7.484)--(12.506,7.486)--(12.509,7.488)--(12.512,7.490)--(12.515,7.492)--(12.518,7.494)%
  --(12.521,7.496)--(12.524,7.498)--(12.527,7.500)--(12.530,7.502)--(12.533,7.505)--(12.536,7.507)%
  --(12.539,7.509)--(12.542,7.511)--(12.545,7.513)--(12.548,7.515)--(12.551,7.517)--(12.554,7.519)%
  --(12.557,7.521)--(12.560,7.523)--(12.563,7.526)--(12.566,7.528)--(12.569,7.530)--(12.572,7.532)%
  --(12.575,7.534)--(12.578,7.536)--(12.581,7.538)--(12.584,7.540)--(12.587,7.542)--(12.590,7.544)%
  --(12.593,7.547)--(12.596,7.549)--(12.599,7.551)--(12.602,7.553)--(12.605,7.555)--(12.608,7.557)%
  --(12.611,7.559)--(12.614,7.561)--(12.617,7.563)--(12.620,7.566)--(12.623,7.568)--(12.626,7.570)%
  --(12.629,7.572)--(12.632,7.574)--(12.635,7.576)--(12.638,7.578)--(12.641,7.580)--(12.644,7.582)%
  --(12.647,7.584)--(12.650,7.587)--(12.653,7.589)--(12.656,7.591)--(12.659,7.593)--(12.662,7.595)%
  --(12.665,7.597)--(12.668,7.599)--(12.671,7.601)--(12.674,7.603)--(12.677,7.605)--(12.680,7.608)%
  --(12.683,7.610)--(12.686,7.612)--(12.689,7.614)--(12.692,7.616)--(12.695,7.618)--(12.698,7.620)%
  --(12.701,7.622)--(12.704,7.624)--(12.707,7.626)--(12.710,7.629)--(12.713,7.631)--(12.715,7.633)%
  --(12.718,7.635)--(12.721,7.637)--(12.724,7.639)--(12.727,7.641)--(12.730,7.643)--(12.733,7.645)%
  --(12.736,7.647)--(12.739,7.650)--(12.742,7.652)--(12.745,7.654)--(12.748,7.656)--(12.751,7.658)%
  --(12.754,7.660)--(12.757,7.662)--(12.760,7.664)--(12.763,7.666)--(12.766,7.669)--(12.769,7.671)%
  --(12.772,7.673)--(12.775,7.675)--(12.778,7.677)--(12.781,7.679)--(12.784,7.681)--(12.787,7.683)%
  --(12.790,7.685)--(12.793,7.687)--(12.796,7.690)--(12.799,7.692)--(12.802,7.694)--(12.805,7.696)%
  --(12.808,7.698)--(12.811,7.700)--(12.814,7.702)--(12.817,7.704)--(12.820,7.706)--(12.823,7.708)%
  --(12.826,7.711)--(12.829,7.713)--(12.832,7.715)--(12.835,7.717)--(12.838,7.719)--(12.841,7.721)%
  --(12.844,7.723)--(12.847,7.725)--(12.850,7.727)--(12.853,7.730)--(12.856,7.732)--(12.859,7.734)%
  --(12.862,7.736)--(12.865,7.738)--(12.868,7.740)--(12.871,7.742)--(12.874,7.744)--(12.877,7.746)%
  --(12.880,7.748)--(12.883,7.751)--(12.886,7.753)--(12.889,7.755)--(12.892,7.757)--(12.895,7.759)%
  --(12.898,7.761)--(12.901,7.763)--(12.904,7.765)--(12.907,7.767)--(12.910,7.769)--(12.913,7.772)%
  --(12.916,7.774)--(12.919,7.776)--(12.922,7.778)--(12.924,7.780)--(12.927,7.782)--(12.930,7.784)%
  --(12.933,7.786)--(12.936,7.788)--(12.939,7.791)--(12.942,7.793)--(12.945,7.795)--(12.948,7.797)%
  --(12.951,7.799)--(12.954,7.801)--(12.957,7.803)--(12.960,7.805)--(12.963,7.807)--(12.966,7.809)%
  --(12.969,7.812)--(12.972,7.814)--(12.975,7.816)--(12.978,7.818)--(12.981,7.820)--(12.984,7.822)%
  --(12.987,7.824)--(12.990,7.826)--(12.993,7.828)--(12.996,7.830)--(12.999,7.833)--(13.002,7.835)%
  --(13.005,7.837)--(13.008,7.839)--(13.011,7.841)--(13.014,7.843)--(13.017,7.845)--(13.020,7.847)%
  --(13.023,7.849)--(13.026,7.852)--(13.029,7.854)--(13.032,7.856)--(13.035,7.858)--(13.038,7.860)%
  --(13.041,7.862)--(13.044,7.864)--(13.047,7.866)--(13.050,7.868)--(13.053,7.870)--(13.056,7.873)%
  --(13.059,7.875)--(13.062,7.877)--(13.065,7.879)--(13.068,7.881)--(13.071,7.883)--(13.074,7.885)%
  --(13.077,7.887)--(13.080,7.889)--(13.083,7.891)--(13.086,7.894)--(13.089,7.896)--(13.092,7.898)%
  --(13.095,7.900)--(13.098,7.902)--(13.101,7.904)--(13.104,7.906)--(13.107,7.908)--(13.110,7.910)%
  --(13.113,7.913)--(13.116,7.915)--(13.119,7.917)--(13.122,7.919)--(13.125,7.921)--(13.128,7.923)%
  --(13.131,7.925)--(13.133,7.927)--(13.136,7.929)--(13.139,7.931)--(13.142,7.934)--(13.145,7.936)%
  --(13.148,7.938)--(13.151,7.940)--(13.154,7.942)--(13.157,7.944)--(13.160,7.946)--(13.163,7.948)%
  --(13.166,7.950)--(13.169,7.953)--(13.172,7.955)--(13.175,7.957)--(13.178,7.959)--(13.181,7.961)%
  --(13.184,7.963)--(13.187,7.965)--(13.190,7.967)--(13.193,7.969)--(13.196,7.971)--(13.199,7.974)%
  --(13.202,7.976)--(13.205,7.978)--(13.208,7.980)--(13.211,7.982)--(13.214,7.984)--(13.217,7.986)%
  --(13.220,7.988)--(13.223,7.990)--(13.226,7.993)--(13.229,7.995)--(13.232,7.997)--(13.235,7.999)%
  --(13.238,8.001)--(13.241,8.003)--(13.244,8.005)--(13.247,8.007)--(13.250,8.009)--(13.253,8.011)%
  --(13.256,8.014)--(13.259,8.016)--(13.262,8.018)--(13.265,8.020)--(13.268,8.022)--(13.271,8.024)%
  --(13.274,8.026)--(13.277,8.028)--(13.280,8.030)--(13.283,8.033)--(13.286,8.035)--(13.289,8.037)%
  --(13.292,8.039)--(13.295,8.041)--(13.298,8.043)--(13.301,8.045)--(13.304,8.047)--(13.307,8.049)%
  --(13.310,8.051)--(13.313,8.054)--(13.316,8.056)--(13.319,8.058)--(13.322,8.060)--(13.325,8.062)%
  --(13.328,8.064)--(13.331,8.066)--(13.334,8.068)--(13.337,8.070)--(13.340,8.073)--(13.342,8.075)%
  --(13.345,8.077)--(13.348,8.079)--(13.351,8.081)--(13.354,8.083)--(13.357,8.085)--(13.360,8.087)%
  --(13.363,8.089)--(13.366,8.091)--(13.369,8.094)--(13.372,8.096)--(13.375,8.098)--(13.378,8.100)%
  --(13.381,8.102)--(13.384,8.104)--(13.387,8.106)--(13.390,8.108)--(13.393,8.110)--(13.396,8.113)%
  --(13.399,8.115)--(13.402,8.117)--(13.405,8.119)--(13.408,8.121)--(13.411,8.123)--(13.414,8.125)%
  --(13.417,8.127)--(13.420,8.129)--(13.423,8.131)--(13.426,8.134)--(13.429,8.136)--(13.432,8.138)%
  --(13.435,8.140)--(13.438,8.142)--(13.441,8.144)--(13.444,8.146);
\gpcolor{color=gp lt color border}
\node[gp node left] at (2.972,7.681) {$\rho \approx \nicefrac{2}{3} \cdot \rho_{\rm{max}}$};
\gpcolor{rgb color={0.337,0.706,0.914}}
\draw[gp path] (1.872,7.681)--(2.788,7.681);
\draw[gp path] (1.507,1.622)--(1.510,1.622)--(1.513,1.621)--(1.516,1.621)--(1.519,1.621)%
  --(1.522,1.621)--(1.525,1.620)--(1.528,1.620)--(1.531,1.620)--(1.534,1.620)--(1.537,1.619)%
  --(1.540,1.619)--(1.543,1.619)--(1.546,1.619)--(1.549,1.618)--(1.552,1.618)--(1.555,1.618)%
  --(1.558,1.618)--(1.561,1.617)--(1.564,1.617)--(1.567,1.617)--(1.570,1.617)--(1.573,1.616)%
  --(1.576,1.616)--(1.579,1.616)--(1.582,1.616)--(1.585,1.615)--(1.588,1.615)--(1.591,1.615)%
  --(1.594,1.615)--(1.597,1.614)--(1.600,1.614)--(1.603,1.614)--(1.606,1.614)--(1.609,1.613)%
  --(1.611,1.613)--(1.614,1.613)--(1.617,1.613)--(1.620,1.612)--(1.623,1.612)--(1.626,1.612)%
  --(1.629,1.612)--(1.632,1.611)--(1.635,1.611)--(1.638,1.611)--(1.641,1.611)--(1.644,1.611)%
  --(1.647,1.610)--(1.650,1.610)--(1.653,1.610)--(1.656,1.610)--(1.659,1.609)--(1.662,1.609)%
  --(1.665,1.609)--(1.668,1.609)--(1.671,1.608)--(1.674,1.608)--(1.677,1.608)--(1.680,1.608)%
  --(1.683,1.607)--(1.686,1.607)--(1.689,1.607)--(1.692,1.607)--(1.695,1.607)--(1.698,1.606)%
  --(1.701,1.606)--(1.704,1.606)--(1.707,1.606)--(1.710,1.605)--(1.713,1.605)--(1.716,1.605)%
  --(1.719,1.605)--(1.722,1.605)--(1.725,1.604)--(1.728,1.604)--(1.731,1.604)--(1.734,1.604)%
  --(1.737,1.603)--(1.740,1.603)--(1.743,1.603)--(1.746,1.603)--(1.749,1.603)--(1.752,1.602)%
  --(1.755,1.602)--(1.758,1.602)--(1.761,1.602)--(1.764,1.602)--(1.767,1.601)--(1.770,1.601)%
  --(1.773,1.601)--(1.776,1.601)--(1.779,1.601)--(1.782,1.600)--(1.785,1.600)--(1.788,1.600)%
  --(1.791,1.600)--(1.794,1.600)--(1.797,1.599)--(1.800,1.599)--(1.803,1.599)--(1.806,1.599)%
  --(1.809,1.599)--(1.812,1.598)--(1.815,1.598)--(1.818,1.598)--(1.820,1.598)--(1.823,1.598)%
  --(1.826,1.597)--(1.829,1.597)--(1.832,1.597)--(1.835,1.597)--(1.838,1.597)--(1.841,1.597)%
  --(1.844,1.596)--(1.847,1.596)--(1.850,1.596)--(1.853,1.596)--(1.856,1.596)--(1.859,1.595)%
  --(1.862,1.595)--(1.865,1.595)--(1.868,1.595)--(1.871,1.595)--(1.874,1.595)--(1.877,1.594)%
  --(1.880,1.594)--(1.883,1.594)--(1.886,1.594)--(1.889,1.594)--(1.892,1.594)--(1.895,1.594)%
  --(1.898,1.593)--(1.901,1.593)--(1.904,1.593)--(1.907,1.593)--(1.910,1.593)--(1.913,1.593)%
  --(1.916,1.592)--(1.919,1.592)--(1.922,1.592)--(1.925,1.592)--(1.928,1.592)--(1.931,1.592)%
  --(1.934,1.592)--(1.937,1.592)--(1.940,1.591)--(1.943,1.591)--(1.946,1.591)--(1.949,1.591)%
  --(1.952,1.591)--(1.955,1.591)--(1.958,1.591)--(1.961,1.591)--(1.964,1.590)--(1.967,1.590)%
  --(1.970,1.590)--(1.973,1.590)--(1.976,1.590)--(1.979,1.590)--(1.982,1.590)--(1.985,1.590)%
  --(1.988,1.590)--(1.991,1.589)--(1.994,1.589)--(1.997,1.589)--(2.000,1.589)--(2.003,1.589)%
  --(2.006,1.589)--(2.009,1.589)--(2.012,1.589)--(2.015,1.589)--(2.018,1.589)--(2.021,1.588)%
  --(2.024,1.588)--(2.027,1.588)--(2.029,1.588)--(2.032,1.588)--(2.035,1.588)--(2.038,1.588)%
  --(2.041,1.588)--(2.044,1.588)--(2.047,1.588)--(2.050,1.588)--(2.053,1.588)--(2.056,1.588)%
  --(2.059,1.588)--(2.062,1.587)--(2.065,1.587)--(2.068,1.587)--(2.071,1.587)--(2.074,1.587)%
  --(2.077,1.587)--(2.080,1.587)--(2.083,1.587)--(2.086,1.587)--(2.089,1.587)--(2.092,1.587)%
  --(2.095,1.587)--(2.098,1.587)--(2.101,1.587)--(2.104,1.587)--(2.107,1.587)--(2.110,1.587)%
  --(2.113,1.587)--(2.116,1.587)--(2.119,1.587)--(2.122,1.587)--(2.125,1.587)--(2.128,1.587)%
  --(2.131,1.587)--(2.134,1.587)--(2.137,1.587)--(2.140,1.587)--(2.143,1.587)--(2.146,1.587)%
  --(2.149,1.587)--(2.152,1.587)--(2.155,1.587)--(2.158,1.587)--(2.161,1.587)--(2.164,1.587)%
  --(2.167,1.587)--(2.170,1.587)--(2.173,1.587)--(2.176,1.587)--(2.179,1.587)--(2.182,1.587)%
  --(2.185,1.587)--(2.188,1.587)--(2.191,1.587)--(2.194,1.587)--(2.197,1.587)--(2.200,1.587)%
  --(2.203,1.587)--(2.206,1.587)--(2.209,1.587)--(2.212,1.587)--(2.215,1.587)--(2.218,1.587)%
  --(2.221,1.587)--(2.224,1.587)--(2.227,1.587)--(2.230,1.587)--(2.233,1.588)--(2.236,1.588)%
  --(2.238,1.588)--(2.241,1.588)--(2.244,1.588)--(2.247,1.588)--(2.250,1.588)--(2.253,1.588)%
  --(2.256,1.588)--(2.259,1.588)--(2.262,1.588)--(2.265,1.588)--(2.268,1.588)--(2.271,1.589)%
  --(2.274,1.589)--(2.277,1.589)--(2.280,1.589)--(2.283,1.589)--(2.286,1.589)--(2.289,1.589)%
  --(2.292,1.589)--(2.295,1.589)--(2.298,1.589)--(2.301,1.590)--(2.304,1.590)--(2.307,1.590)%
  --(2.310,1.590)--(2.313,1.590)--(2.316,1.590)--(2.319,1.590)--(2.322,1.590)--(2.325,1.591)%
  --(2.328,1.591)--(2.331,1.591)--(2.334,1.591)--(2.337,1.591)--(2.340,1.591)--(2.343,1.591)%
  --(2.346,1.592)--(2.349,1.592)--(2.352,1.592)--(2.355,1.592)--(2.358,1.592)--(2.361,1.592)%
  --(2.364,1.593)--(2.367,1.593)--(2.370,1.593)--(2.373,1.593)--(2.376,1.593)--(2.379,1.593)%
  --(2.382,1.594)--(2.385,1.594)--(2.388,1.594)--(2.391,1.594)--(2.394,1.594)--(2.397,1.594)%
  --(2.400,1.595)--(2.403,1.595)--(2.406,1.595)--(2.409,1.595)--(2.412,1.595)--(2.415,1.596)%
  --(2.418,1.596)--(2.421,1.596)--(2.424,1.596)--(2.427,1.597)--(2.430,1.597)--(2.433,1.597)%
  --(2.436,1.597)--(2.439,1.597)--(2.442,1.598)--(2.445,1.598)--(2.447,1.598)--(2.450,1.598)%
  --(2.453,1.599)--(2.456,1.599)--(2.459,1.599)--(2.462,1.599)--(2.465,1.600)--(2.468,1.600)%
  --(2.471,1.600)--(2.474,1.600)--(2.477,1.601)--(2.480,1.601)--(2.483,1.601)--(2.486,1.601)%
  --(2.489,1.602)--(2.492,1.602)--(2.495,1.602)--(2.498,1.603)--(2.501,1.603)--(2.504,1.603)%
  --(2.507,1.603)--(2.510,1.604)--(2.513,1.604)--(2.516,1.604)--(2.519,1.605)--(2.522,1.605)%
  --(2.525,1.605)--(2.528,1.605)--(2.531,1.606)--(2.534,1.606)--(2.537,1.606)--(2.540,1.607)%
  --(2.543,1.607)--(2.546,1.607)--(2.549,1.608)--(2.552,1.608)--(2.555,1.608)--(2.558,1.609)%
  --(2.561,1.609)--(2.564,1.609)--(2.567,1.610)--(2.570,1.610)--(2.573,1.610)--(2.576,1.611)%
  --(2.579,1.611)--(2.582,1.611)--(2.585,1.612)--(2.588,1.612)--(2.591,1.612)--(2.594,1.613)%
  --(2.597,1.613)--(2.600,1.613)--(2.603,1.614)--(2.606,1.614)--(2.609,1.615)--(2.612,1.615)%
  --(2.615,1.615)--(2.618,1.616)--(2.621,1.616)--(2.624,1.617)--(2.627,1.617)--(2.630,1.617)%
  --(2.633,1.618)--(2.636,1.618)--(2.639,1.618)--(2.642,1.619)--(2.645,1.619)--(2.648,1.620)%
  --(2.651,1.620)--(2.654,1.621)--(2.656,1.621)--(2.659,1.621)--(2.662,1.622)--(2.665,1.622)%
  --(2.668,1.623)--(2.671,1.623)--(2.674,1.623)--(2.677,1.624)--(2.680,1.624)--(2.683,1.625)%
  --(2.686,1.625)--(2.689,1.626)--(2.692,1.626)--(2.695,1.627)--(2.698,1.627)--(2.701,1.627)%
  --(2.704,1.628)--(2.707,1.628)--(2.710,1.629)--(2.713,1.629)--(2.716,1.630)--(2.719,1.630)%
  --(2.722,1.631)--(2.725,1.631)--(2.728,1.632)--(2.731,1.632)--(2.734,1.633)--(2.737,1.633)%
  --(2.740,1.634)--(2.743,1.634)--(2.746,1.635)--(2.749,1.635)--(2.752,1.636)--(2.755,1.636)%
  --(2.758,1.637)--(2.761,1.637)--(2.764,1.638)--(2.767,1.638)--(2.770,1.639)--(2.773,1.639)%
  --(2.776,1.640)--(2.779,1.640)--(2.782,1.641)--(2.785,1.641)--(2.788,1.642)--(2.791,1.642)%
  --(2.794,1.643)--(2.797,1.643)--(2.800,1.644)--(2.803,1.644)--(2.806,1.645)--(2.809,1.645)%
  --(2.812,1.646)--(2.815,1.647)--(2.818,1.647)--(2.821,1.648)--(2.824,1.648)--(2.827,1.649)%
  --(2.830,1.649)--(2.833,1.650)--(2.836,1.650)--(2.839,1.651)--(2.842,1.652)--(2.845,1.652)%
  --(2.848,1.653)--(2.851,1.653)--(2.854,1.654)--(2.857,1.655)--(2.860,1.655)--(2.863,1.656)%
  --(2.866,1.656)--(2.868,1.657)--(2.871,1.658)--(2.874,1.658)--(2.877,1.659)--(2.880,1.659)%
  --(2.883,1.660)--(2.886,1.661)--(2.889,1.661)--(2.892,1.662)--(2.895,1.662)--(2.898,1.663)%
  --(2.901,1.664)--(2.904,1.664)--(2.907,1.665)--(2.910,1.666)--(2.913,1.666)--(2.916,1.667)%
  --(2.919,1.667)--(2.922,1.668)--(2.925,1.669)--(2.928,1.669)--(2.931,1.670)--(2.934,1.671)%
  --(2.937,1.671)--(2.940,1.672)--(2.943,1.673)--(2.946,1.673)--(2.949,1.674)--(2.952,1.675)%
  --(2.955,1.675)--(2.958,1.676)--(2.961,1.677)--(2.964,1.677)--(2.967,1.678)--(2.970,1.679)%
  --(2.973,1.679)--(2.976,1.680)--(2.979,1.681)--(2.982,1.681)--(2.985,1.682)--(2.988,1.683)%
  --(2.991,1.683)--(2.994,1.684)--(2.997,1.685)--(3.000,1.686)--(3.003,1.686)--(3.006,1.687)%
  --(3.009,1.688)--(3.012,1.688)--(3.015,1.689)--(3.018,1.690)--(3.021,1.691)--(3.024,1.691)%
  --(3.027,1.692)--(3.030,1.693)--(3.033,1.694)--(3.036,1.694)--(3.039,1.695)--(3.042,1.696)%
  --(3.045,1.696)--(3.048,1.697)--(3.051,1.698)--(3.054,1.699)--(3.057,1.699)--(3.060,1.700)%
  --(3.063,1.701)--(3.066,1.702)--(3.069,1.702)--(3.072,1.703)--(3.075,1.704)--(3.077,1.705)%
  --(3.080,1.706)--(3.083,1.706)--(3.086,1.707)--(3.089,1.708)--(3.092,1.709)--(3.095,1.709)%
  --(3.098,1.710)--(3.101,1.711)--(3.104,1.712)--(3.107,1.713)--(3.110,1.713)--(3.113,1.714)%
  --(3.116,1.715)--(3.119,1.716)--(3.122,1.717)--(3.125,1.717)--(3.128,1.718)--(3.131,1.719)%
  --(3.134,1.720)--(3.137,1.721)--(3.140,1.721)--(3.143,1.722)--(3.146,1.723)--(3.149,1.724)%
  --(3.152,1.725)--(3.155,1.726)--(3.158,1.726)--(3.161,1.727)--(3.164,1.728)--(3.167,1.729)%
  --(3.170,1.730)--(3.173,1.731)--(3.176,1.731)--(3.179,1.732)--(3.182,1.733)--(3.185,1.734)%
  --(3.188,1.735)--(3.191,1.736)--(3.194,1.737)--(3.197,1.737)--(3.200,1.738)--(3.203,1.739)%
  --(3.206,1.740)--(3.209,1.741)--(3.212,1.742)--(3.215,1.743)--(3.218,1.743)--(3.221,1.744)%
  --(3.224,1.745)--(3.227,1.746)--(3.230,1.747)--(3.233,1.748)--(3.236,1.749)--(3.239,1.750)%
  --(3.242,1.751)--(3.245,1.751)--(3.248,1.752)--(3.251,1.753)--(3.254,1.754)--(3.257,1.755)%
  --(3.260,1.756)--(3.263,1.757)--(3.266,1.758)--(3.269,1.759)--(3.272,1.760)--(3.275,1.761)%
  --(3.278,1.761)--(3.281,1.762)--(3.284,1.763)--(3.286,1.764)--(3.289,1.765)--(3.292,1.766)%
  --(3.295,1.767)--(3.298,1.768)--(3.301,1.769)--(3.304,1.770)--(3.307,1.771)--(3.310,1.772)%
  --(3.313,1.773)--(3.316,1.774)--(3.319,1.775)--(3.322,1.775)--(3.325,1.776)--(3.328,1.777)%
  --(3.331,1.778)--(3.334,1.779)--(3.337,1.780)--(3.340,1.781)--(3.343,1.782)--(3.346,1.783)%
  --(3.349,1.784)--(3.352,1.785)--(3.355,1.786)--(3.358,1.787)--(3.361,1.788)--(3.364,1.789)%
  --(3.367,1.790)--(3.370,1.791)--(3.373,1.792)--(3.376,1.793)--(3.379,1.794)--(3.382,1.795)%
  --(3.385,1.796)--(3.388,1.797)--(3.391,1.798)--(3.394,1.799)--(3.397,1.800)--(3.400,1.801)%
  --(3.403,1.802)--(3.406,1.803)--(3.409,1.804)--(3.412,1.805)--(3.415,1.806)--(3.418,1.807)%
  --(3.421,1.808)--(3.424,1.809)--(3.427,1.810)--(3.430,1.811)--(3.433,1.812)--(3.436,1.813)%
  --(3.439,1.814)--(3.442,1.815)--(3.445,1.816)--(3.448,1.817)--(3.451,1.818)--(3.454,1.819)%
  --(3.457,1.820)--(3.460,1.822)--(3.463,1.823)--(3.466,1.824)--(3.469,1.825)--(3.472,1.826)%
  --(3.475,1.827)--(3.478,1.828)--(3.481,1.829)--(3.484,1.830)--(3.487,1.831)--(3.490,1.832)%
  --(3.493,1.833)--(3.495,1.834)--(3.498,1.835)--(3.501,1.836)--(3.504,1.837)--(3.507,1.839)%
  --(3.510,1.840)--(3.513,1.841)--(3.516,1.842)--(3.519,1.843)--(3.522,1.844)--(3.525,1.845)%
  --(3.528,1.846)--(3.531,1.847)--(3.534,1.848)--(3.537,1.849)--(3.540,1.851)--(3.543,1.852)%
  --(3.546,1.853)--(3.549,1.854)--(3.552,1.855)--(3.555,1.856)--(3.558,1.857)--(3.561,1.858)%
  --(3.564,1.859)--(3.567,1.860)--(3.570,1.862)--(3.573,1.863)--(3.576,1.864)--(3.579,1.865)%
  --(3.582,1.866)--(3.585,1.867)--(3.588,1.868)--(3.591,1.869)--(3.594,1.871)--(3.597,1.872)%
  --(3.600,1.873)--(3.603,1.874)--(3.606,1.875)--(3.609,1.876)--(3.612,1.877)--(3.615,1.879)%
  --(3.618,1.880)--(3.621,1.881)--(3.624,1.882)--(3.627,1.883)--(3.630,1.884)--(3.633,1.885)%
  --(3.636,1.887)--(3.639,1.888)--(3.642,1.889)--(3.645,1.890)--(3.648,1.891)--(3.651,1.892)%
  --(3.654,1.894)--(3.657,1.895)--(3.660,1.896)--(3.663,1.897)--(3.666,1.898)--(3.669,1.900)%
  --(3.672,1.901)--(3.675,1.902)--(3.678,1.903)--(3.681,1.904)--(3.684,1.905)--(3.687,1.907)%
  --(3.690,1.908)--(3.693,1.909)--(3.696,1.910)--(3.699,1.911)--(3.702,1.913)--(3.704,1.914)%
  --(3.707,1.915)--(3.710,1.916)--(3.713,1.917)--(3.716,1.919)--(3.719,1.920)--(3.722,1.921)%
  --(3.725,1.922)--(3.728,1.923)--(3.731,1.925)--(3.734,1.926)--(3.737,1.927)--(3.740,1.928)%
  --(3.743,1.930)--(3.746,1.931)--(3.749,1.932)--(3.752,1.933)--(3.755,1.934)--(3.758,1.936)%
  --(3.761,1.937)--(3.764,1.938)--(3.767,1.939)--(3.770,1.941)--(3.773,1.942)--(3.776,1.943)%
  --(3.779,1.944)--(3.782,1.946)--(3.785,1.947)--(3.788,1.948)--(3.791,1.949)--(3.794,1.951)%
  --(3.797,1.952)--(3.800,1.953)--(3.803,1.954)--(3.806,1.956)--(3.809,1.957)--(3.812,1.958)%
  --(3.815,1.959)--(3.818,1.961)--(3.821,1.962)--(3.824,1.963)--(3.827,1.964)--(3.830,1.966)%
  --(3.833,1.967)--(3.836,1.968)--(3.839,1.969)--(3.842,1.971)--(3.845,1.972)--(3.848,1.973)%
  --(3.851,1.975)--(3.854,1.976)--(3.857,1.977)--(3.860,1.978)--(3.863,1.980)--(3.866,1.981)%
  --(3.869,1.982)--(3.872,1.984)--(3.875,1.985)--(3.878,1.986)--(3.881,1.987)--(3.884,1.989)%
  --(3.887,1.990)--(3.890,1.991)--(3.893,1.993)--(3.896,1.994)--(3.899,1.995)--(3.902,1.997)%
  --(3.905,1.998)--(3.908,1.999)--(3.911,2.000)--(3.914,2.002)--(3.916,2.003)--(3.919,2.004)%
  --(3.922,2.006)--(3.925,2.007)--(3.928,2.008)--(3.931,2.010)--(3.934,2.011)--(3.937,2.012)%
  --(3.940,2.014)--(3.943,2.015)--(3.946,2.016)--(3.949,2.018)--(3.952,2.019)--(3.955,2.020)%
  --(3.958,2.022)--(3.961,2.023)--(3.964,2.024)--(3.967,2.026)--(3.970,2.027)--(3.973,2.028)%
  --(3.976,2.030)--(3.979,2.031)--(3.982,2.032)--(3.985,2.034)--(3.988,2.035)--(3.991,2.036)%
  --(3.994,2.038)--(3.997,2.039)--(4.000,2.040)--(4.003,2.042)--(4.006,2.043)--(4.009,2.044)%
  --(4.012,2.046)--(4.015,2.047)--(4.018,2.049)--(4.021,2.050)--(4.024,2.051)--(4.027,2.053)%
  --(4.030,2.054)--(4.033,2.055)--(4.036,2.057)--(4.039,2.058)--(4.042,2.060)--(4.045,2.061)%
  --(4.048,2.062)--(4.051,2.064)--(4.054,2.065)--(4.057,2.066)--(4.060,2.068)--(4.063,2.069)%
  --(4.066,2.071)--(4.069,2.072)--(4.072,2.073)--(4.075,2.075)--(4.078,2.076)--(4.081,2.078)%
  --(4.084,2.079)--(4.087,2.080)--(4.090,2.082)--(4.093,2.083)--(4.096,2.084)--(4.099,2.086)%
  --(4.102,2.087)--(4.105,2.089)--(4.108,2.090)--(4.111,2.091)--(4.114,2.093)--(4.117,2.094)%
  --(4.120,2.096)--(4.123,2.097)--(4.125,2.099)--(4.128,2.100)--(4.131,2.101)--(4.134,2.103)%
  --(4.137,2.104)--(4.140,2.106)--(4.143,2.107)--(4.146,2.108)--(4.149,2.110)--(4.152,2.111)%
  --(4.155,2.113)--(4.158,2.114)--(4.161,2.116)--(4.164,2.117)--(4.167,2.118)--(4.170,2.120)%
  --(4.173,2.121)--(4.176,2.123)--(4.179,2.124)--(4.182,2.126)--(4.185,2.127)--(4.188,2.128)%
  --(4.191,2.130)--(4.194,2.131)--(4.197,2.133)--(4.200,2.134)--(4.203,2.136)--(4.206,2.137)%
  --(4.209,2.139)--(4.212,2.140)--(4.215,2.141)--(4.218,2.143)--(4.221,2.144)--(4.224,2.146)%
  --(4.227,2.147)--(4.230,2.149)--(4.233,2.150)--(4.236,2.152)--(4.239,2.153)--(4.242,2.155)%
  --(4.245,2.156)--(4.248,2.158)--(4.251,2.159)--(4.254,2.160)--(4.257,2.162)--(4.260,2.163)%
  --(4.263,2.165)--(4.266,2.166)--(4.269,2.168)--(4.272,2.169)--(4.275,2.171)--(4.278,2.172)%
  --(4.281,2.174)--(4.284,2.175)--(4.287,2.177)--(4.290,2.178)--(4.293,2.180)--(4.296,2.181)%
  --(4.299,2.183)--(4.302,2.184)--(4.305,2.186)--(4.308,2.187)--(4.311,2.189)--(4.314,2.190)%
  --(4.317,2.192)--(4.320,2.193)--(4.323,2.194)--(4.326,2.196)--(4.329,2.197)--(4.332,2.199)%
  --(4.334,2.200)--(4.337,2.202)--(4.340,2.203)--(4.343,2.205)--(4.346,2.206)--(4.349,2.208)%
  --(4.352,2.209)--(4.355,2.211)--(4.358,2.213)--(4.361,2.214)--(4.364,2.216)--(4.367,2.217)%
  --(4.370,2.219)--(4.373,2.220)--(4.376,2.222)--(4.379,2.223)--(4.382,2.225)--(4.385,2.226)%
  --(4.388,2.228)--(4.391,2.229)--(4.394,2.231)--(4.397,2.232)--(4.400,2.234)--(4.403,2.235)%
  --(4.406,2.237)--(4.409,2.238)--(4.412,2.240)--(4.415,2.241)--(4.418,2.243)--(4.421,2.244)%
  --(4.424,2.246)--(4.427,2.247)--(4.430,2.249)--(4.433,2.251)--(4.436,2.252)--(4.439,2.254)%
  --(4.442,2.255)--(4.445,2.257)--(4.448,2.258)--(4.451,2.260)--(4.454,2.261)--(4.457,2.263)%
  --(4.460,2.264)--(4.463,2.266)--(4.466,2.267)--(4.469,2.269)--(4.472,2.271)--(4.475,2.272)%
  --(4.478,2.274)--(4.481,2.275)--(4.484,2.277)--(4.487,2.278)--(4.490,2.280)--(4.493,2.281)%
  --(4.496,2.283)--(4.499,2.285)--(4.502,2.286)--(4.505,2.288)--(4.508,2.289)--(4.511,2.291)%
  --(4.514,2.292)--(4.517,2.294)--(4.520,2.295)--(4.523,2.297)--(4.526,2.299)--(4.529,2.300)%
  --(4.532,2.302)--(4.535,2.303)--(4.538,2.305)--(4.541,2.306)--(4.543,2.308)--(4.546,2.310)%
  --(4.549,2.311)--(4.552,2.313)--(4.555,2.314)--(4.558,2.316)--(4.561,2.317)--(4.564,2.319)%
  --(4.567,2.321)--(4.570,2.322)--(4.573,2.324)--(4.576,2.325)--(4.579,2.327)--(4.582,2.329)%
  --(4.585,2.330)--(4.588,2.332)--(4.591,2.333)--(4.594,2.335)--(4.597,2.337)--(4.600,2.338)%
  --(4.603,2.340)--(4.606,2.341)--(4.609,2.343)--(4.612,2.344)--(4.615,2.346)--(4.618,2.348)%
  --(4.621,2.349)--(4.624,2.351)--(4.627,2.352)--(4.630,2.354)--(4.633,2.356)--(4.636,2.357)%
  --(4.639,2.359)--(4.642,2.360)--(4.645,2.362)--(4.648,2.364)--(4.651,2.365)--(4.654,2.367)%
  --(4.657,2.368)--(4.660,2.370)--(4.663,2.372)--(4.666,2.373)--(4.669,2.375)--(4.672,2.377)%
  --(4.675,2.378)--(4.678,2.380)--(4.681,2.381)--(4.684,2.383)--(4.687,2.385)--(4.690,2.386)%
  --(4.693,2.388)--(4.696,2.389)--(4.699,2.391)--(4.702,2.393)--(4.705,2.394)--(4.708,2.396)%
  --(4.711,2.398)--(4.714,2.399)--(4.717,2.401)--(4.720,2.402)--(4.723,2.404)--(4.726,2.406)%
  --(4.729,2.407)--(4.732,2.409)--(4.735,2.411)--(4.738,2.412)--(4.741,2.414)--(4.744,2.416)%
  --(4.747,2.417)--(4.750,2.419)--(4.752,2.420)--(4.755,2.422)--(4.758,2.424)--(4.761,2.425)%
  --(4.764,2.427)--(4.767,2.429)--(4.770,2.430)--(4.773,2.432)--(4.776,2.434)--(4.779,2.435)%
  --(4.782,2.437)--(4.785,2.439)--(4.788,2.440)--(4.791,2.442)--(4.794,2.443)--(4.797,2.445)%
  --(4.800,2.447)--(4.803,2.448)--(4.806,2.450)--(4.809,2.452)--(4.812,2.453)--(4.815,2.455)%
  --(4.818,2.457)--(4.821,2.458)--(4.824,2.460)--(4.827,2.462)--(4.830,2.463)--(4.833,2.465)%
  --(4.836,2.467)--(4.839,2.468)--(4.842,2.470)--(4.845,2.472)--(4.848,2.473)--(4.851,2.475)%
  --(4.854,2.477)--(4.857,2.478)--(4.860,2.480)--(4.863,2.482)--(4.866,2.483)--(4.869,2.485)%
  --(4.872,2.487)--(4.875,2.488)--(4.878,2.490)--(4.881,2.492)--(4.884,2.493)--(4.887,2.495)%
  --(4.890,2.497)--(4.893,2.498)--(4.896,2.500)--(4.899,2.502)--(4.902,2.503)--(4.905,2.505)%
  --(4.908,2.507)--(4.911,2.508)--(4.914,2.510)--(4.917,2.512)--(4.920,2.513)--(4.923,2.515)%
  --(4.926,2.517)--(4.929,2.519)--(4.932,2.520)--(4.935,2.522)--(4.938,2.524)--(4.941,2.525)%
  --(4.944,2.527)--(4.947,2.529)--(4.950,2.530)--(4.953,2.532)--(4.956,2.534)--(4.959,2.535)%
  --(4.961,2.537)--(4.964,2.539)--(4.967,2.540)--(4.970,2.542)--(4.973,2.544)--(4.976,2.546)%
  --(4.979,2.547)--(4.982,2.549)--(4.985,2.551)--(4.988,2.552)--(4.991,2.554)--(4.994,2.556)%
  --(4.997,2.557)--(5.000,2.559)--(5.003,2.561)--(5.006,2.563)--(5.009,2.564)--(5.012,2.566)%
  --(5.015,2.568)--(5.018,2.569)--(5.021,2.571)--(5.024,2.573)--(5.027,2.575)--(5.030,2.576)%
  --(5.033,2.578)--(5.036,2.580)--(5.039,2.581)--(5.042,2.583)--(5.045,2.585)--(5.048,2.587)%
  --(5.051,2.588)--(5.054,2.590)--(5.057,2.592)--(5.060,2.593)--(5.063,2.595)--(5.066,2.597)%
  --(5.069,2.599)--(5.072,2.600)--(5.075,2.602)--(5.078,2.604)--(5.081,2.605)--(5.084,2.607)%
  --(5.087,2.609)--(5.090,2.611)--(5.093,2.612)--(5.096,2.614)--(5.099,2.616)--(5.102,2.617)%
  --(5.105,2.619)--(5.108,2.621)--(5.111,2.623)--(5.114,2.624)--(5.117,2.626)--(5.120,2.628)%
  --(5.123,2.630)--(5.126,2.631)--(5.129,2.633)--(5.132,2.635)--(5.135,2.637)--(5.138,2.638)%
  --(5.141,2.640)--(5.144,2.642)--(5.147,2.643)--(5.150,2.645)--(5.153,2.647)--(5.156,2.649)%
  --(5.159,2.650)--(5.162,2.652)--(5.165,2.654)--(5.168,2.656)--(5.171,2.657)--(5.173,2.659)%
  --(5.176,2.661)--(5.179,2.663)--(5.182,2.664)--(5.185,2.666)--(5.188,2.668)--(5.191,2.670)%
  --(5.194,2.671)--(5.197,2.673)--(5.200,2.675)--(5.203,2.677)--(5.206,2.678)--(5.209,2.680)%
  --(5.212,2.682)--(5.215,2.684)--(5.218,2.685)--(5.221,2.687)--(5.224,2.689)--(5.227,2.691)%
  --(5.230,2.692)--(5.233,2.694)--(5.236,2.696)--(5.239,2.698)--(5.242,2.699)--(5.245,2.701)%
  --(5.248,2.703)--(5.251,2.705)--(5.254,2.706)--(5.257,2.708)--(5.260,2.710)--(5.263,2.712)%
  --(5.266,2.713)--(5.269,2.715)--(5.272,2.717)--(5.275,2.719)--(5.278,2.721)--(5.281,2.722)%
  --(5.284,2.724)--(5.287,2.726)--(5.290,2.728)--(5.293,2.729)--(5.296,2.731)--(5.299,2.733)%
  --(5.302,2.735)--(5.305,2.736)--(5.308,2.738)--(5.311,2.740)--(5.314,2.742)--(5.317,2.743)%
  --(5.320,2.745)--(5.323,2.747)--(5.326,2.749)--(5.329,2.751)--(5.332,2.752)--(5.335,2.754)%
  --(5.338,2.756)--(5.341,2.758)--(5.344,2.759)--(5.347,2.761)--(5.350,2.763)--(5.353,2.765)%
  --(5.356,2.767)--(5.359,2.768)--(5.362,2.770)--(5.365,2.772)--(5.368,2.774)--(5.371,2.775)%
  --(5.374,2.777)--(5.377,2.779)--(5.380,2.781)--(5.382,2.783)--(5.385,2.784)--(5.388,2.786)%
  --(5.391,2.788)--(5.394,2.790)--(5.397,2.792)--(5.400,2.793)--(5.403,2.795)--(5.406,2.797)%
  --(5.409,2.799)--(5.412,2.801)--(5.415,2.802)--(5.418,2.804)--(5.421,2.806)--(5.424,2.808)%
  --(5.427,2.809)--(5.430,2.811)--(5.433,2.813)--(5.436,2.815)--(5.439,2.817)--(5.442,2.818)%
  --(5.445,2.820)--(5.448,2.822)--(5.451,2.824)--(5.454,2.826)--(5.457,2.827)--(5.460,2.829)%
  --(5.463,2.831)--(5.466,2.833)--(5.469,2.835)--(5.472,2.836)--(5.475,2.838)--(5.478,2.840)%
  --(5.481,2.842)--(5.484,2.844)--(5.487,2.845)--(5.490,2.847)--(5.493,2.849)--(5.496,2.851)%
  --(5.499,2.853)--(5.502,2.854)--(5.505,2.856)--(5.508,2.858)--(5.511,2.860)--(5.514,2.862)%
  --(5.517,2.863)--(5.520,2.865)--(5.523,2.867)--(5.526,2.869)--(5.529,2.871)--(5.532,2.873)%
  --(5.535,2.874)--(5.538,2.876)--(5.541,2.878)--(5.544,2.880)--(5.547,2.882)--(5.550,2.883)%
  --(5.553,2.885)--(5.556,2.887)--(5.559,2.889)--(5.562,2.891)--(5.565,2.893)--(5.568,2.894)%
  --(5.571,2.896)--(5.574,2.898)--(5.577,2.900)--(5.580,2.902)--(5.583,2.903)--(5.586,2.905)%
  --(5.589,2.907)--(5.591,2.909)--(5.594,2.911)--(5.597,2.913)--(5.600,2.914)--(5.603,2.916)%
  --(5.606,2.918)--(5.609,2.920)--(5.612,2.922)--(5.615,2.923)--(5.618,2.925)--(5.621,2.927)%
  --(5.624,2.929)--(5.627,2.931)--(5.630,2.933)--(5.633,2.934)--(5.636,2.936)--(5.639,2.938)%
  --(5.642,2.940)--(5.645,2.942)--(5.648,2.944)--(5.651,2.945)--(5.654,2.947)--(5.657,2.949)%
  --(5.660,2.951)--(5.663,2.953)--(5.666,2.955)--(5.669,2.956)--(5.672,2.958)--(5.675,2.960)%
  --(5.678,2.962)--(5.681,2.964)--(5.684,2.966)--(5.687,2.967)--(5.690,2.969)--(5.693,2.971)%
  --(5.696,2.973)--(5.699,2.975)--(5.702,2.977)--(5.705,2.978)--(5.708,2.980)--(5.711,2.982)%
  --(5.714,2.984)--(5.717,2.986)--(5.720,2.988)--(5.723,2.989)--(5.726,2.991)--(5.729,2.993)%
  --(5.732,2.995)--(5.735,2.997)--(5.738,2.999)--(5.741,3.001)--(5.744,3.002)--(5.747,3.004)%
  --(5.750,3.006)--(5.753,3.008)--(5.756,3.010)--(5.759,3.012)--(5.762,3.013)--(5.765,3.015)%
  --(5.768,3.017)--(5.771,3.019)--(5.774,3.021)--(5.777,3.023)--(5.780,3.025)--(5.783,3.026)%
  --(5.786,3.028)--(5.789,3.030)--(5.792,3.032)--(5.795,3.034)--(5.798,3.036)--(5.800,3.038)%
  --(5.803,3.039)--(5.806,3.041)--(5.809,3.043)--(5.812,3.045)--(5.815,3.047)--(5.818,3.049)%
  --(5.821,3.051)--(5.824,3.052)--(5.827,3.054)--(5.830,3.056)--(5.833,3.058)--(5.836,3.060)%
  --(5.839,3.062)--(5.842,3.064)--(5.845,3.065)--(5.848,3.067)--(5.851,3.069)--(5.854,3.071)%
  --(5.857,3.073)--(5.860,3.075)--(5.863,3.077)--(5.866,3.078)--(5.869,3.080)--(5.872,3.082)%
  --(5.875,3.084)--(5.878,3.086)--(5.881,3.088)--(5.884,3.090)--(5.887,3.091)--(5.890,3.093)%
  --(5.893,3.095)--(5.896,3.097)--(5.899,3.099)--(5.902,3.101)--(5.905,3.103)--(5.908,3.105)%
  --(5.911,3.106)--(5.914,3.108)--(5.917,3.110)--(5.920,3.112)--(5.923,3.114)--(5.926,3.116)%
  --(5.929,3.118)--(5.932,3.119)--(5.935,3.121)--(5.938,3.123)--(5.941,3.125)--(5.944,3.127)%
  --(5.947,3.129)--(5.950,3.131)--(5.953,3.133)--(5.956,3.134)--(5.959,3.136)--(5.962,3.138)%
  --(5.965,3.140)--(5.968,3.142)--(5.971,3.144)--(5.974,3.146)--(5.977,3.148)--(5.980,3.149)%
  --(5.983,3.151)--(5.986,3.153)--(5.989,3.155)--(5.992,3.157)--(5.995,3.159)--(5.998,3.161)%
  --(6.001,3.163)--(6.004,3.164)--(6.007,3.166)--(6.009,3.168)--(6.012,3.170)--(6.015,3.172)%
  --(6.018,3.174)--(6.021,3.176)--(6.024,3.178)--(6.027,3.180)--(6.030,3.181)--(6.033,3.183)%
  --(6.036,3.185)--(6.039,3.187)--(6.042,3.189)--(6.045,3.191)--(6.048,3.193)--(6.051,3.195)%
  --(6.054,3.197)--(6.057,3.198)--(6.060,3.200)--(6.063,3.202)--(6.066,3.204)--(6.069,3.206)%
  --(6.072,3.208)--(6.075,3.210)--(6.078,3.212)--(6.081,3.214)--(6.084,3.215)--(6.087,3.217)%
  --(6.090,3.219)--(6.093,3.221)--(6.096,3.223)--(6.099,3.225)--(6.102,3.227)--(6.105,3.229)%
  --(6.108,3.231)--(6.111,3.232)--(6.114,3.234)--(6.117,3.236)--(6.120,3.238)--(6.123,3.240)%
  --(6.126,3.242)--(6.129,3.244)--(6.132,3.246)--(6.135,3.248)--(6.138,3.249)--(6.141,3.251)%
  --(6.144,3.253)--(6.147,3.255)--(6.150,3.257)--(6.153,3.259)--(6.156,3.261)--(6.159,3.263)%
  --(6.162,3.265)--(6.165,3.267)--(6.168,3.268)--(6.171,3.270)--(6.174,3.272)--(6.177,3.274)%
  --(6.180,3.276)--(6.183,3.278)--(6.186,3.280)--(6.189,3.282)--(6.192,3.284)--(6.195,3.286)%
  --(6.198,3.287)--(6.201,3.289)--(6.204,3.291)--(6.207,3.293)--(6.210,3.295)--(6.213,3.297)%
  --(6.216,3.299)--(6.218,3.301)--(6.221,3.303)--(6.224,3.305)--(6.227,3.307)--(6.230,3.308)%
  --(6.233,3.310)--(6.236,3.312)--(6.239,3.314)--(6.242,3.316)--(6.245,3.318)--(6.248,3.320)%
  --(6.251,3.322)--(6.254,3.324)--(6.257,3.326)--(6.260,3.328)--(6.263,3.329)--(6.266,3.331)%
  --(6.269,3.333)--(6.272,3.335)--(6.275,3.337)--(6.278,3.339)--(6.281,3.341)--(6.284,3.343)%
  --(6.287,3.345)--(6.290,3.347)--(6.293,3.349)--(6.296,3.350)--(6.299,3.352)--(6.302,3.354)%
  --(6.305,3.356)--(6.308,3.358)--(6.311,3.360)--(6.314,3.362)--(6.317,3.364)--(6.320,3.366)%
  --(6.323,3.368)--(6.326,3.370)--(6.329,3.372)--(6.332,3.373)--(6.335,3.375)--(6.338,3.377)%
  --(6.341,3.379)--(6.344,3.381)--(6.347,3.383)--(6.350,3.385)--(6.353,3.387)--(6.356,3.389)%
  --(6.359,3.391)--(6.362,3.393)--(6.365,3.395)--(6.368,3.396)--(6.371,3.398)--(6.374,3.400)%
  --(6.377,3.402)--(6.380,3.404)--(6.383,3.406)--(6.386,3.408)--(6.389,3.410)--(6.392,3.412)%
  --(6.395,3.414)--(6.398,3.416)--(6.401,3.418)--(6.404,3.420)--(6.407,3.422)--(6.410,3.423)%
  --(6.413,3.425)--(6.416,3.427)--(6.419,3.429)--(6.422,3.431)--(6.425,3.433)--(6.428,3.435)%
  --(6.430,3.437)--(6.433,3.439)--(6.436,3.441)--(6.439,3.443)--(6.442,3.445)--(6.445,3.447)%
  --(6.448,3.448)--(6.451,3.450)--(6.454,3.452)--(6.457,3.454)--(6.460,3.456)--(6.463,3.458)%
  --(6.466,3.460)--(6.469,3.462)--(6.472,3.464)--(6.475,3.466)--(6.478,3.468)--(6.481,3.470)%
  --(6.484,3.472)--(6.487,3.474)--(6.490,3.476)--(6.493,3.477)--(6.496,3.479)--(6.499,3.481)%
  --(6.502,3.483)--(6.505,3.485)--(6.508,3.487)--(6.511,3.489)--(6.514,3.491)--(6.517,3.493)%
  --(6.520,3.495)--(6.523,3.497)--(6.526,3.499)--(6.529,3.501)--(6.532,3.503)--(6.535,3.505)%
  --(6.538,3.506)--(6.541,3.508)--(6.544,3.510)--(6.547,3.512)--(6.550,3.514)--(6.553,3.516)%
  --(6.556,3.518)--(6.559,3.520)--(6.562,3.522)--(6.565,3.524)--(6.568,3.526)--(6.571,3.528)%
  --(6.574,3.530)--(6.577,3.532)--(6.580,3.534)--(6.583,3.536)--(6.586,3.538)--(6.589,3.539)%
  --(6.592,3.541)--(6.595,3.543)--(6.598,3.545)--(6.601,3.547)--(6.604,3.549)--(6.607,3.551)%
  --(6.610,3.553)--(6.613,3.555)--(6.616,3.557)--(6.619,3.559)--(6.622,3.561)--(6.625,3.563)%
  --(6.628,3.565)--(6.631,3.567)--(6.634,3.569)--(6.637,3.571)--(6.639,3.572)--(6.642,3.574)%
  --(6.645,3.576)--(6.648,3.578)--(6.651,3.580)--(6.654,3.582)--(6.657,3.584)--(6.660,3.586)%
  --(6.663,3.588)--(6.666,3.590)--(6.669,3.592)--(6.672,3.594)--(6.675,3.596)--(6.678,3.598)%
  --(6.681,3.600)--(6.684,3.602)--(6.687,3.604)--(6.690,3.606)--(6.693,3.608)--(6.696,3.610)%
  --(6.699,3.611)--(6.702,3.613)--(6.705,3.615)--(6.708,3.617)--(6.711,3.619)--(6.714,3.621)%
  --(6.717,3.623)--(6.720,3.625)--(6.723,3.627)--(6.726,3.629)--(6.729,3.631)--(6.732,3.633)%
  --(6.735,3.635)--(6.738,3.637)--(6.741,3.639)--(6.744,3.641)--(6.747,3.643)--(6.750,3.645)%
  --(6.753,3.647)--(6.756,3.649)--(6.759,3.651)--(6.762,3.652)--(6.765,3.654)--(6.768,3.656)%
  --(6.771,3.658)--(6.774,3.660)--(6.777,3.662)--(6.780,3.664)--(6.783,3.666)--(6.786,3.668)%
  --(6.789,3.670)--(6.792,3.672)--(6.795,3.674)--(6.798,3.676)--(6.801,3.678)--(6.804,3.680)%
  --(6.807,3.682)--(6.810,3.684)--(6.813,3.686)--(6.816,3.688)--(6.819,3.690)--(6.822,3.692)%
  --(6.825,3.694)--(6.828,3.696)--(6.831,3.698)--(6.834,3.700)--(6.837,3.701)--(6.840,3.703)%
  --(6.843,3.705)--(6.846,3.707)--(6.848,3.709)--(6.851,3.711)--(6.854,3.713)--(6.857,3.715)%
  --(6.860,3.717)--(6.863,3.719)--(6.866,3.721)--(6.869,3.723)--(6.872,3.725)--(6.875,3.727)%
  --(6.878,3.729)--(6.881,3.731)--(6.884,3.733)--(6.887,3.735)--(6.890,3.737)--(6.893,3.739)%
  --(6.896,3.741)--(6.899,3.743)--(6.902,3.745)--(6.905,3.747)--(6.908,3.749)--(6.911,3.751)%
  --(6.914,3.753)--(6.917,3.755)--(6.920,3.756)--(6.923,3.758)--(6.926,3.760)--(6.929,3.762)%
  --(6.932,3.764)--(6.935,3.766)--(6.938,3.768)--(6.941,3.770)--(6.944,3.772)--(6.947,3.774)%
  --(6.950,3.776)--(6.953,3.778)--(6.956,3.780)--(6.959,3.782)--(6.962,3.784)--(6.965,3.786)%
  --(6.968,3.788)--(6.971,3.790)--(6.974,3.792)--(6.977,3.794)--(6.980,3.796)--(6.983,3.798)%
  --(6.986,3.800)--(6.989,3.802)--(6.992,3.804)--(6.995,3.806)--(6.998,3.808)--(7.001,3.810)%
  --(7.004,3.812)--(7.007,3.814)--(7.010,3.816)--(7.013,3.818)--(7.016,3.820)--(7.019,3.822)%
  --(7.022,3.824)--(7.025,3.825)--(7.028,3.827)--(7.031,3.829)--(7.034,3.831)--(7.037,3.833)%
  --(7.040,3.835)--(7.043,3.837)--(7.046,3.839)--(7.049,3.841)--(7.052,3.843)--(7.055,3.845)%
  --(7.057,3.847)--(7.060,3.849)--(7.063,3.851)--(7.066,3.853)--(7.069,3.855)--(7.072,3.857)%
  --(7.075,3.859)--(7.078,3.861)--(7.081,3.863)--(7.084,3.865)--(7.087,3.867)--(7.090,3.869)%
  --(7.093,3.871)--(7.096,3.873)--(7.099,3.875)--(7.102,3.877)--(7.105,3.879)--(7.108,3.881)%
  --(7.111,3.883)--(7.114,3.885)--(7.117,3.887)--(7.120,3.889)--(7.123,3.891)--(7.126,3.893)%
  --(7.129,3.895)--(7.132,3.897)--(7.135,3.899)--(7.138,3.901)--(7.141,3.903)--(7.144,3.905)%
  --(7.147,3.907)--(7.150,3.909)--(7.153,3.911)--(7.156,3.913)--(7.159,3.915)--(7.162,3.917)%
  --(7.165,3.919)--(7.168,3.921)--(7.171,3.923)--(7.174,3.925)--(7.177,3.926)--(7.180,3.928)%
  --(7.183,3.930)--(7.186,3.932)--(7.189,3.934)--(7.192,3.936)--(7.195,3.938)--(7.198,3.940)%
  --(7.201,3.942)--(7.204,3.944)--(7.207,3.946)--(7.210,3.948)--(7.213,3.950)--(7.216,3.952)%
  --(7.219,3.954)--(7.222,3.956)--(7.225,3.958)--(7.228,3.960)--(7.231,3.962)--(7.234,3.964)%
  --(7.237,3.966)--(7.240,3.968)--(7.243,3.970)--(7.246,3.972)--(7.249,3.974)--(7.252,3.976)%
  --(7.255,3.978)--(7.258,3.980)--(7.261,3.982)--(7.264,3.984)--(7.266,3.986)--(7.269,3.988)%
  --(7.272,3.990)--(7.275,3.992)--(7.278,3.994)--(7.281,3.996)--(7.284,3.998)--(7.287,4.000)%
  --(7.290,4.002)--(7.293,4.004)--(7.296,4.006)--(7.299,4.008)--(7.302,4.010)--(7.305,4.012)%
  --(7.308,4.014)--(7.311,4.016)--(7.314,4.018)--(7.317,4.020)--(7.320,4.022)--(7.323,4.024)%
  --(7.326,4.026)--(7.329,4.028)--(7.332,4.030)--(7.335,4.032)--(7.338,4.034)--(7.341,4.036)%
  --(7.344,4.038)--(7.347,4.040)--(7.350,4.042)--(7.353,4.044)--(7.356,4.046)--(7.359,4.048)%
  --(7.362,4.050)--(7.365,4.052)--(7.368,4.054)--(7.371,4.056)--(7.374,4.058)--(7.377,4.060)%
  --(7.380,4.062)--(7.383,4.064)--(7.386,4.066)--(7.389,4.068)--(7.392,4.070)--(7.395,4.072)%
  --(7.398,4.074)--(7.401,4.076)--(7.404,4.078)--(7.407,4.080)--(7.410,4.082)--(7.413,4.084)%
  --(7.416,4.086)--(7.419,4.088)--(7.422,4.090)--(7.425,4.092)--(7.428,4.094)--(7.431,4.096)%
  --(7.434,4.098)--(7.437,4.100)--(7.440,4.102)--(7.443,4.104)--(7.446,4.106)--(7.449,4.108)%
  --(7.452,4.110)--(7.455,4.112)--(7.458,4.114)--(7.461,4.116)--(7.464,4.118)--(7.467,4.120)%
  --(7.470,4.122)--(7.473,4.124)--(7.476,4.126)--(7.478,4.128)--(7.481,4.130)--(7.484,4.132)%
  --(7.487,4.134)--(7.490,4.136)--(7.493,4.138)--(7.496,4.140)--(7.499,4.142)--(7.502,4.144)%
  --(7.505,4.146)--(7.508,4.148)--(7.511,4.150)--(7.514,4.152)--(7.517,4.154)--(7.520,4.156)%
  --(7.523,4.158)--(7.526,4.160)--(7.529,4.162)--(7.532,4.164)--(7.535,4.166)--(7.538,4.168)%
  --(7.541,4.170)--(7.544,4.172)--(7.547,4.174)--(7.550,4.176)--(7.553,4.178)--(7.556,4.180)%
  --(7.559,4.182)--(7.562,4.184)--(7.565,4.186)--(7.568,4.188)--(7.571,4.190)--(7.574,4.192)%
  --(7.577,4.194)--(7.580,4.196)--(7.583,4.198)--(7.586,4.200)--(7.589,4.202)--(7.592,4.204)%
  --(7.595,4.206)--(7.598,4.208)--(7.601,4.210)--(7.604,4.212)--(7.607,4.214)--(7.610,4.216)%
  --(7.613,4.218)--(7.616,4.220)--(7.619,4.222)--(7.622,4.224)--(7.625,4.226)--(7.628,4.228)%
  --(7.631,4.230)--(7.634,4.232)--(7.637,4.234)--(7.640,4.236)--(7.643,4.238)--(7.646,4.240)%
  --(7.649,4.242)--(7.652,4.244)--(7.655,4.246)--(7.658,4.248)--(7.661,4.250)--(7.664,4.252)%
  --(7.667,4.254)--(7.670,4.256)--(7.673,4.258)--(7.676,4.260)--(7.679,4.262)--(7.682,4.264)%
  --(7.685,4.266)--(7.687,4.268)--(7.690,4.270)--(7.693,4.272)--(7.696,4.274)--(7.699,4.276)%
  --(7.702,4.278)--(7.705,4.280)--(7.708,4.282)--(7.711,4.284)--(7.714,4.287)--(7.717,4.289)%
  --(7.720,4.291)--(7.723,4.293)--(7.726,4.295)--(7.729,4.297)--(7.732,4.299)--(7.735,4.301)%
  --(7.738,4.303)--(7.741,4.305)--(7.744,4.307)--(7.747,4.309)--(7.750,4.311)--(7.753,4.313)%
  --(7.756,4.315)--(7.759,4.317)--(7.762,4.319)--(7.765,4.321)--(7.768,4.323)--(7.771,4.325)%
  --(7.774,4.327)--(7.777,4.329)--(7.780,4.331)--(7.783,4.333)--(7.786,4.335)--(7.789,4.337)%
  --(7.792,4.339)--(7.795,4.341)--(7.798,4.343)--(7.801,4.345)--(7.804,4.347)--(7.807,4.349)%
  --(7.810,4.351)--(7.813,4.353)--(7.816,4.355)--(7.819,4.357)--(7.822,4.359)--(7.825,4.361)%
  --(7.828,4.363)--(7.831,4.365)--(7.834,4.367)--(7.837,4.369)--(7.840,4.371)--(7.843,4.373)%
  --(7.846,4.375)--(7.849,4.377)--(7.852,4.379)--(7.855,4.381)--(7.858,4.383)--(7.861,4.385)%
  --(7.864,4.387)--(7.867,4.389)--(7.870,4.391)--(7.873,4.393)--(7.876,4.395)--(7.879,4.397)%
  --(7.882,4.399)--(7.885,4.402)--(7.888,4.404)--(7.891,4.406)--(7.894,4.408)--(7.896,4.410)%
  --(7.899,4.412)--(7.902,4.414)--(7.905,4.416)--(7.908,4.418)--(7.911,4.420)--(7.914,4.422)%
  --(7.917,4.424)--(7.920,4.426)--(7.923,4.428)--(7.926,4.430)--(7.929,4.432)--(7.932,4.434)%
  --(7.935,4.436)--(7.938,4.438)--(7.941,4.440)--(7.944,4.442)--(7.947,4.444)--(7.950,4.446)%
  --(7.953,4.448)--(7.956,4.450)--(7.959,4.452)--(7.962,4.454)--(7.965,4.456)--(7.968,4.458)%
  --(7.971,4.460)--(7.974,4.462)--(7.977,4.464)--(7.980,4.466)--(7.983,4.468)--(7.986,4.470)%
  --(7.989,4.472)--(7.992,4.474)--(7.995,4.476)--(7.998,4.478)--(8.001,4.480)--(8.004,4.482)%
  --(8.007,4.484)--(8.010,4.487)--(8.013,4.489)--(8.016,4.491)--(8.019,4.493)--(8.022,4.495)%
  --(8.025,4.497)--(8.028,4.499)--(8.031,4.501)--(8.034,4.503)--(8.037,4.505)--(8.040,4.507)%
  --(8.043,4.509)--(8.046,4.511)--(8.049,4.513)--(8.052,4.515)--(8.055,4.517)--(8.058,4.519)%
  --(8.061,4.521)--(8.064,4.523)--(8.067,4.525)--(8.070,4.527)--(8.073,4.529)--(8.076,4.531)%
  --(8.079,4.533)--(8.082,4.535)--(8.085,4.537)--(8.088,4.539)--(8.091,4.541)--(8.094,4.543)%
  --(8.097,4.545)--(8.100,4.547)--(8.103,4.549)--(8.105,4.551)--(8.108,4.553)--(8.111,4.555)%
  --(8.114,4.557)--(8.117,4.560)--(8.120,4.562)--(8.123,4.564)--(8.126,4.566)--(8.129,4.568)%
  --(8.132,4.570)--(8.135,4.572)--(8.138,4.574)--(8.141,4.576)--(8.144,4.578)--(8.147,4.580)%
  --(8.150,4.582)--(8.153,4.584)--(8.156,4.586)--(8.159,4.588)--(8.162,4.590)--(8.165,4.592)%
  --(8.168,4.594)--(8.171,4.596)--(8.174,4.598)--(8.177,4.600)--(8.180,4.602)--(8.183,4.604)%
  --(8.186,4.606)--(8.189,4.608)--(8.192,4.610)--(8.195,4.612)--(8.198,4.614)--(8.201,4.616)%
  --(8.204,4.618)--(8.207,4.620)--(8.210,4.623)--(8.213,4.625)--(8.216,4.627)--(8.219,4.629)%
  --(8.222,4.631)--(8.225,4.633)--(8.228,4.635)--(8.231,4.637)--(8.234,4.639)--(8.237,4.641)%
  --(8.240,4.643)--(8.243,4.645)--(8.246,4.647)--(8.249,4.649)--(8.252,4.651)--(8.255,4.653)%
  --(8.258,4.655)--(8.261,4.657)--(8.264,4.659)--(8.267,4.661)--(8.270,4.663)--(8.273,4.665)%
  --(8.276,4.667)--(8.279,4.669)--(8.282,4.671)--(8.285,4.673)--(8.288,4.675)--(8.291,4.677)%
  --(8.294,4.680)--(8.297,4.682)--(8.300,4.684)--(8.303,4.686)--(8.306,4.688)--(8.309,4.690)%
  --(8.312,4.692)--(8.314,4.694)--(8.317,4.696)--(8.320,4.698)--(8.323,4.700)--(8.326,4.702)%
  --(8.329,4.704)--(8.332,4.706)--(8.335,4.708)--(8.338,4.710)--(8.341,4.712)--(8.344,4.714)%
  --(8.347,4.716)--(8.350,4.718)--(8.353,4.720)--(8.356,4.722)--(8.359,4.724)--(8.362,4.726)%
  --(8.365,4.728)--(8.368,4.730)--(8.371,4.732)--(8.374,4.735)--(8.377,4.737)--(8.380,4.739)%
  --(8.383,4.741)--(8.386,4.743)--(8.389,4.745)--(8.392,4.747)--(8.395,4.749)--(8.398,4.751)%
  --(8.401,4.753)--(8.404,4.755)--(8.407,4.757)--(8.410,4.759)--(8.413,4.761)--(8.416,4.763)%
  --(8.419,4.765)--(8.422,4.767)--(8.425,4.769)--(8.428,4.771)--(8.431,4.773)--(8.434,4.775)%
  --(8.437,4.777)--(8.440,4.779)--(8.443,4.781)--(8.446,4.784)--(8.449,4.786)--(8.452,4.788)%
  --(8.455,4.790)--(8.458,4.792)--(8.461,4.794)--(8.464,4.796)--(8.467,4.798)--(8.470,4.800)%
  --(8.473,4.802)--(8.476,4.804)--(8.479,4.806)--(8.482,4.808)--(8.485,4.810)--(8.488,4.812)%
  --(8.491,4.814)--(8.494,4.816)--(8.497,4.818)--(8.500,4.820)--(8.503,4.822)--(8.506,4.824)%
  --(8.509,4.826)--(8.512,4.828)--(8.515,4.830)--(8.518,4.833)--(8.521,4.835)--(8.523,4.837)%
  --(8.526,4.839)--(8.529,4.841)--(8.532,4.843)--(8.535,4.845)--(8.538,4.847)--(8.541,4.849)%
  --(8.544,4.851)--(8.547,4.853)--(8.550,4.855)--(8.553,4.857)--(8.556,4.859)--(8.559,4.861)%
  --(8.562,4.863)--(8.565,4.865)--(8.568,4.867)--(8.571,4.869)--(8.574,4.871)--(8.577,4.873)%
  --(8.580,4.875)--(8.583,4.878)--(8.586,4.880)--(8.589,4.882)--(8.592,4.884)--(8.595,4.886)%
  --(8.598,4.888)--(8.601,4.890)--(8.604,4.892)--(8.607,4.894)--(8.610,4.896)--(8.613,4.898)%
  --(8.616,4.900)--(8.619,4.902)--(8.622,4.904)--(8.625,4.906)--(8.628,4.908)--(8.631,4.910)%
  --(8.634,4.912)--(8.637,4.914)--(8.640,4.916)--(8.643,4.918)--(8.646,4.921)--(8.649,4.923)%
  --(8.652,4.925)--(8.655,4.927)--(8.658,4.929)--(8.661,4.931)--(8.664,4.933)--(8.667,4.935)%
  --(8.670,4.937)--(8.673,4.939)--(8.676,4.941)--(8.679,4.943)--(8.682,4.945)--(8.685,4.947)%
  --(8.688,4.949)--(8.691,4.951)--(8.694,4.953)--(8.697,4.955)--(8.700,4.957)--(8.703,4.959)%
  --(8.706,4.961)--(8.709,4.964)--(8.712,4.966)--(8.715,4.968)--(8.718,4.970)--(8.721,4.972)%
  --(8.724,4.974)--(8.727,4.976)--(8.730,4.978)--(8.733,4.980)--(8.735,4.982)--(8.738,4.984)%
  --(8.741,4.986)--(8.744,4.988)--(8.747,4.990)--(8.750,4.992)--(8.753,4.994)--(8.756,4.996)%
  --(8.759,4.998)--(8.762,5.000)--(8.765,5.002)--(8.768,5.005)--(8.771,5.007)--(8.774,5.009)%
  --(8.777,5.011)--(8.780,5.013)--(8.783,5.015)--(8.786,5.017)--(8.789,5.019)--(8.792,5.021)%
  --(8.795,5.023)--(8.798,5.025)--(8.801,5.027)--(8.804,5.029)--(8.807,5.031)--(8.810,5.033)%
  --(8.813,5.035)--(8.816,5.037)--(8.819,5.039)--(8.822,5.041)--(8.825,5.044)--(8.828,5.046)%
  --(8.831,5.048)--(8.834,5.050)--(8.837,5.052)--(8.840,5.054)--(8.843,5.056)--(8.846,5.058)%
  --(8.849,5.060)--(8.852,5.062)--(8.855,5.064)--(8.858,5.066)--(8.861,5.068)--(8.864,5.070)%
  --(8.867,5.072)--(8.870,5.074)--(8.873,5.076)--(8.876,5.078)--(8.879,5.080)--(8.882,5.083)%
  --(8.885,5.085)--(8.888,5.087)--(8.891,5.089)--(8.894,5.091)--(8.897,5.093)--(8.900,5.095)%
  --(8.903,5.097)--(8.906,5.099)--(8.909,5.101)--(8.912,5.103)--(8.915,5.105)--(8.918,5.107)%
  --(8.921,5.109)--(8.924,5.111)--(8.927,5.113)--(8.930,5.115)--(8.933,5.117)--(8.936,5.120)%
  --(8.939,5.122)--(8.942,5.124)--(8.944,5.126)--(8.947,5.128)--(8.950,5.130)--(8.953,5.132)%
  --(8.956,5.134)--(8.959,5.136)--(8.962,5.138)--(8.965,5.140)--(8.968,5.142)--(8.971,5.144)%
  --(8.974,5.146)--(8.977,5.148)--(8.980,5.150)--(8.983,5.152)--(8.986,5.154)--(8.989,5.157)%
  --(8.992,5.159)--(8.995,5.161)--(8.998,5.163)--(9.001,5.165)--(9.004,5.167)--(9.007,5.169)%
  --(9.010,5.171)--(9.013,5.173)--(9.016,5.175)--(9.019,5.177)--(9.022,5.179)--(9.025,5.181)%
  --(9.028,5.183)--(9.031,5.185)--(9.034,5.187)--(9.037,5.189)--(9.040,5.191)--(9.043,5.194)%
  --(9.046,5.196)--(9.049,5.198)--(9.052,5.200)--(9.055,5.202)--(9.058,5.204)--(9.061,5.206)%
  --(9.064,5.208)--(9.067,5.210)--(9.070,5.212)--(9.073,5.214)--(9.076,5.216)--(9.079,5.218)%
  --(9.082,5.220)--(9.085,5.222)--(9.088,5.224)--(9.091,5.226)--(9.094,5.229)--(9.097,5.231)%
  --(9.100,5.233)--(9.103,5.235)--(9.106,5.237)--(9.109,5.239)--(9.112,5.241)--(9.115,5.243)%
  --(9.118,5.245)--(9.121,5.247)--(9.124,5.249)--(9.127,5.251)--(9.130,5.253)--(9.133,5.255)%
  --(9.136,5.257)--(9.139,5.259)--(9.142,5.262)--(9.145,5.264)--(9.148,5.266)--(9.151,5.268)%
  --(9.153,5.270)--(9.156,5.272)--(9.159,5.274)--(9.162,5.276)--(9.165,5.278)--(9.168,5.280)%
  --(9.171,5.282)--(9.174,5.284)--(9.177,5.286)--(9.180,5.288)--(9.183,5.290)--(9.186,5.292)%
  --(9.189,5.294)--(9.192,5.297)--(9.195,5.299)--(9.198,5.301)--(9.201,5.303)--(9.204,5.305)%
  --(9.207,5.307)--(9.210,5.309)--(9.213,5.311)--(9.216,5.313)--(9.219,5.315)--(9.222,5.317)%
  --(9.225,5.319)--(9.228,5.321)--(9.231,5.323)--(9.234,5.325)--(9.237,5.327)--(9.240,5.330)%
  --(9.243,5.332)--(9.246,5.334)--(9.249,5.336)--(9.252,5.338)--(9.255,5.340)--(9.258,5.342)%
  --(9.261,5.344)--(9.264,5.346)--(9.267,5.348)--(9.270,5.350)--(9.273,5.352)--(9.276,5.354)%
  --(9.279,5.356)--(9.282,5.358)--(9.285,5.360)--(9.288,5.363)--(9.291,5.365)--(9.294,5.367)%
  --(9.297,5.369)--(9.300,5.371)--(9.303,5.373)--(9.306,5.375)--(9.309,5.377)--(9.312,5.379)%
  --(9.315,5.381)--(9.318,5.383)--(9.321,5.385)--(9.324,5.387)--(9.327,5.389)--(9.330,5.391)%
  --(9.333,5.394)--(9.336,5.396)--(9.339,5.398)--(9.342,5.400)--(9.345,5.402)--(9.348,5.404)%
  --(9.351,5.406)--(9.354,5.408)--(9.357,5.410)--(9.360,5.412)--(9.362,5.414)--(9.365,5.416)%
  --(9.368,5.418)--(9.371,5.420)--(9.374,5.422)--(9.377,5.424)--(9.380,5.427)--(9.383,5.429)%
  --(9.386,5.431)--(9.389,5.433)--(9.392,5.435)--(9.395,5.437)--(9.398,5.439)--(9.401,5.441)%
  --(9.404,5.443)--(9.407,5.445)--(9.410,5.447)--(9.413,5.449)--(9.416,5.451)--(9.419,5.453)%
  --(9.422,5.455)--(9.425,5.458)--(9.428,5.460)--(9.431,5.462)--(9.434,5.464)--(9.437,5.466)%
  --(9.440,5.468)--(9.443,5.470)--(9.446,5.472)--(9.449,5.474)--(9.452,5.476)--(9.455,5.478)%
  --(9.458,5.480)--(9.461,5.482)--(9.464,5.484)--(9.467,5.486)--(9.470,5.489)--(9.473,5.491)%
  --(9.476,5.493)--(9.479,5.495)--(9.482,5.497)--(9.485,5.499)--(9.488,5.501)--(9.491,5.503)%
  --(9.494,5.505)--(9.497,5.507)--(9.500,5.509)--(9.503,5.511)--(9.506,5.513)--(9.509,5.515)%
  --(9.512,5.517)--(9.515,5.520)--(9.518,5.522)--(9.521,5.524)--(9.524,5.526)--(9.527,5.528)%
  --(9.530,5.530)--(9.533,5.532)--(9.536,5.534)--(9.539,5.536)--(9.542,5.538)--(9.545,5.540)%
  --(9.548,5.542)--(9.551,5.544)--(9.554,5.546)--(9.557,5.549)--(9.560,5.551)--(9.563,5.553)%
  --(9.566,5.555)--(9.569,5.557)--(9.571,5.559)--(9.574,5.561)--(9.577,5.563)--(9.580,5.565)%
  --(9.583,5.567)--(9.586,5.569)--(9.589,5.571)--(9.592,5.573)--(9.595,5.575)--(9.598,5.577)%
  --(9.601,5.580)--(9.604,5.582)--(9.607,5.584)--(9.610,5.586)--(9.613,5.588)--(9.616,5.590)%
  --(9.619,5.592)--(9.622,5.594)--(9.625,5.596)--(9.628,5.598)--(9.631,5.600)--(9.634,5.602)%
  --(9.637,5.604)--(9.640,5.606)--(9.643,5.609)--(9.646,5.611)--(9.649,5.613)--(9.652,5.615)%
  --(9.655,5.617)--(9.658,5.619)--(9.661,5.621)--(9.664,5.623)--(9.667,5.625)--(9.670,5.627)%
  --(9.673,5.629)--(9.676,5.631)--(9.679,5.633)--(9.682,5.635)--(9.685,5.638)--(9.688,5.640)%
  --(9.691,5.642)--(9.694,5.644)--(9.697,5.646)--(9.700,5.648)--(9.703,5.650)--(9.706,5.652)%
  --(9.709,5.654)--(9.712,5.656)--(9.715,5.658)--(9.718,5.660)--(9.721,5.662)--(9.724,5.664)%
  --(9.727,5.667)--(9.730,5.669)--(9.733,5.671)--(9.736,5.673)--(9.739,5.675)--(9.742,5.677)%
  --(9.745,5.679)--(9.748,5.681)--(9.751,5.683)--(9.754,5.685)--(9.757,5.687)--(9.760,5.689)%
  --(9.763,5.691)--(9.766,5.693)--(9.769,5.696)--(9.772,5.698)--(9.775,5.700)--(9.778,5.702)%
  --(9.780,5.704)--(9.783,5.706)--(9.786,5.708)--(9.789,5.710)--(9.792,5.712)--(9.795,5.714)%
  --(9.798,5.716)--(9.801,5.718)--(9.804,5.720)--(9.807,5.722)--(9.810,5.725)--(9.813,5.727)%
  --(9.816,5.729)--(9.819,5.731)--(9.822,5.733)--(9.825,5.735)--(9.828,5.737)--(9.831,5.739)%
  --(9.834,5.741)--(9.837,5.743)--(9.840,5.745)--(9.843,5.747)--(9.846,5.749)--(9.849,5.752)%
  --(9.852,5.754)--(9.855,5.756)--(9.858,5.758)--(9.861,5.760)--(9.864,5.762)--(9.867,5.764)%
  --(9.870,5.766)--(9.873,5.768)--(9.876,5.770)--(9.879,5.772)--(9.882,5.774)--(9.885,5.776)%
  --(9.888,5.778)--(9.891,5.781)--(9.894,5.783)--(9.897,5.785)--(9.900,5.787)--(9.903,5.789)%
  --(9.906,5.791)--(9.909,5.793)--(9.912,5.795)--(9.915,5.797)--(9.918,5.799)--(9.921,5.801)%
  --(9.924,5.803)--(9.927,5.805)--(9.930,5.808)--(9.933,5.810)--(9.936,5.812)--(9.939,5.814)%
  --(9.942,5.816)--(9.945,5.818)--(9.948,5.820)--(9.951,5.822)--(9.954,5.824)--(9.957,5.826)%
  --(9.960,5.828)--(9.963,5.830)--(9.966,5.832)--(9.969,5.835)--(9.972,5.837)--(9.975,5.839)%
  --(9.978,5.841)--(9.981,5.843)--(9.984,5.845)--(9.987,5.847)--(9.990,5.849)--(9.992,5.851)%
  --(9.995,5.853)--(9.998,5.855)--(10.001,5.857)--(10.004,5.859)--(10.007,5.861)--(10.010,5.864)%
  --(10.013,5.866)--(10.016,5.868)--(10.019,5.870)--(10.022,5.872)--(10.025,5.874)--(10.028,5.876)%
  --(10.031,5.878)--(10.034,5.880)--(10.037,5.882)--(10.040,5.884)--(10.043,5.886)--(10.046,5.888)%
  --(10.049,5.891)--(10.052,5.893)--(10.055,5.895)--(10.058,5.897)--(10.061,5.899)--(10.064,5.901)%
  --(10.067,5.903)--(10.070,5.905)--(10.073,5.907)--(10.076,5.909)--(10.079,5.911)--(10.082,5.913)%
  --(10.085,5.916)--(10.088,5.918)--(10.091,5.920)--(10.094,5.922)--(10.097,5.924)--(10.100,5.926)%
  --(10.103,5.928)--(10.106,5.930)--(10.109,5.932)--(10.112,5.934)--(10.115,5.936)--(10.118,5.938)%
  --(10.121,5.940)--(10.124,5.943)--(10.127,5.945)--(10.130,5.947)--(10.133,5.949)--(10.136,5.951)%
  --(10.139,5.953)--(10.142,5.955)--(10.145,5.957)--(10.148,5.959)--(10.151,5.961)--(10.154,5.963)%
  --(10.157,5.965)--(10.160,5.967)--(10.163,5.970)--(10.166,5.972)--(10.169,5.974)--(10.172,5.976)%
  --(10.175,5.978)--(10.178,5.980)--(10.181,5.982)--(10.184,5.984)--(10.187,5.986)--(10.190,5.988)%
  --(10.193,5.990)--(10.196,5.992)--(10.199,5.994)--(10.201,5.997)--(10.204,5.999)--(10.207,6.001)%
  --(10.210,6.003)--(10.213,6.005)--(10.216,6.007)--(10.219,6.009)--(10.222,6.011)--(10.225,6.013)%
  --(10.228,6.015)--(10.231,6.017)--(10.234,6.019)--(10.237,6.022)--(10.240,6.024)--(10.243,6.026)%
  --(10.246,6.028)--(10.249,6.030)--(10.252,6.032)--(10.255,6.034)--(10.258,6.036)--(10.261,6.038)%
  --(10.264,6.040)--(10.267,6.042)--(10.270,6.044)--(10.273,6.046)--(10.276,6.049)--(10.279,6.051)%
  --(10.282,6.053)--(10.285,6.055)--(10.288,6.057)--(10.291,6.059)--(10.294,6.061)--(10.297,6.063)%
  --(10.300,6.065)--(10.303,6.067)--(10.306,6.069)--(10.309,6.071)--(10.312,6.074)--(10.315,6.076)%
  --(10.318,6.078)--(10.321,6.080)--(10.324,6.082)--(10.327,6.084)--(10.330,6.086)--(10.333,6.088)%
  --(10.336,6.090)--(10.339,6.092)--(10.342,6.094)--(10.345,6.096)--(10.348,6.099)--(10.351,6.101)%
  --(10.354,6.103)--(10.357,6.105)--(10.360,6.107)--(10.363,6.109)--(10.366,6.111)--(10.369,6.113)%
  --(10.372,6.115)--(10.375,6.117)--(10.378,6.119)--(10.381,6.121)--(10.384,6.123)--(10.387,6.126)%
  --(10.390,6.128)--(10.393,6.130)--(10.396,6.132)--(10.399,6.134)--(10.402,6.136)--(10.405,6.138)%
  --(10.408,6.140)--(10.410,6.142)--(10.413,6.144)--(10.416,6.146)--(10.419,6.148)--(10.422,6.151)%
  --(10.425,6.153)--(10.428,6.155)--(10.431,6.157)--(10.434,6.159)--(10.437,6.161)--(10.440,6.163)%
  --(10.443,6.165)--(10.446,6.167)--(10.449,6.169)--(10.452,6.171)--(10.455,6.173)--(10.458,6.176)%
  --(10.461,6.178)--(10.464,6.180)--(10.467,6.182)--(10.470,6.184)--(10.473,6.186)--(10.476,6.188)%
  --(10.479,6.190)--(10.482,6.192)--(10.485,6.194)--(10.488,6.196)--(10.491,6.198)--(10.494,6.201)%
  --(10.497,6.203)--(10.500,6.205)--(10.503,6.207)--(10.506,6.209)--(10.509,6.211)--(10.512,6.213)%
  --(10.515,6.215)--(10.518,6.217)--(10.521,6.219)--(10.524,6.221)--(10.527,6.223)--(10.530,6.226)%
  --(10.533,6.228)--(10.536,6.230)--(10.539,6.232)--(10.542,6.234)--(10.545,6.236)--(10.548,6.238)%
  --(10.551,6.240)--(10.554,6.242)--(10.557,6.244)--(10.560,6.246)--(10.563,6.248)--(10.566,6.251)%
  --(10.569,6.253)--(10.572,6.255)--(10.575,6.257)--(10.578,6.259)--(10.581,6.261)--(10.584,6.263)%
  --(10.587,6.265)--(10.590,6.267)--(10.593,6.269)--(10.596,6.271)--(10.599,6.273)--(10.602,6.276)%
  --(10.605,6.278)--(10.608,6.280)--(10.611,6.282)--(10.614,6.284)--(10.617,6.286)--(10.619,6.288)%
  --(10.622,6.290)--(10.625,6.292)--(10.628,6.294)--(10.631,6.296)--(10.634,6.298)--(10.637,6.301)%
  --(10.640,6.303)--(10.643,6.305)--(10.646,6.307)--(10.649,6.309)--(10.652,6.311)--(10.655,6.313)%
  --(10.658,6.315)--(10.661,6.317)--(10.664,6.319)--(10.667,6.321)--(10.670,6.324)--(10.673,6.326)%
  --(10.676,6.328)--(10.679,6.330)--(10.682,6.332)--(10.685,6.334)--(10.688,6.336)--(10.691,6.338)%
  --(10.694,6.340)--(10.697,6.342)--(10.700,6.344)--(10.703,6.346)--(10.706,6.349)--(10.709,6.351)%
  --(10.712,6.353)--(10.715,6.355)--(10.718,6.357)--(10.721,6.359)--(10.724,6.361)--(10.727,6.363)%
  --(10.730,6.365)--(10.733,6.367)--(10.736,6.369)--(10.739,6.371)--(10.742,6.374)--(10.745,6.376)%
  --(10.748,6.378)--(10.751,6.380)--(10.754,6.382)--(10.757,6.384)--(10.760,6.386)--(10.763,6.388)%
  --(10.766,6.390)--(10.769,6.392)--(10.772,6.394)--(10.775,6.397)--(10.778,6.399)--(10.781,6.401)%
  --(10.784,6.403)--(10.787,6.405)--(10.790,6.407)--(10.793,6.409)--(10.796,6.411)--(10.799,6.413)%
  --(10.802,6.415)--(10.805,6.417)--(10.808,6.419)--(10.811,6.422)--(10.814,6.424)--(10.817,6.426)%
  --(10.820,6.428)--(10.823,6.430)--(10.826,6.432)--(10.828,6.434)--(10.831,6.436)--(10.834,6.438)%
  --(10.837,6.440)--(10.840,6.442)--(10.843,6.445)--(10.846,6.447)--(10.849,6.449)--(10.852,6.451)%
  --(10.855,6.453)--(10.858,6.455)--(10.861,6.457)--(10.864,6.459)--(10.867,6.461)--(10.870,6.463)%
  --(10.873,6.465)--(10.876,6.467)--(10.879,6.470)--(10.882,6.472)--(10.885,6.474)--(10.888,6.476)%
  --(10.891,6.478)--(10.894,6.480)--(10.897,6.482)--(10.900,6.484)--(10.903,6.486)--(10.906,6.488)%
  --(10.909,6.490)--(10.912,6.493)--(10.915,6.495)--(10.918,6.497)--(10.921,6.499)--(10.924,6.501)%
  --(10.927,6.503)--(10.930,6.505)--(10.933,6.507)--(10.936,6.509)--(10.939,6.511)--(10.942,6.513)%
  --(10.945,6.516)--(10.948,6.518)--(10.951,6.520)--(10.954,6.522)--(10.957,6.524)--(10.960,6.526)%
  --(10.963,6.528)--(10.966,6.530)--(10.969,6.532)--(10.972,6.534)--(10.975,6.536)--(10.978,6.538)%
  --(10.981,6.541)--(10.984,6.543)--(10.987,6.545)--(10.990,6.547)--(10.993,6.549)--(10.996,6.551)%
  --(10.999,6.553)--(11.002,6.555)--(11.005,6.557)--(11.008,6.559)--(11.011,6.561)--(11.014,6.564)%
  --(11.017,6.566)--(11.020,6.568)--(11.023,6.570)--(11.026,6.572)--(11.029,6.574)--(11.032,6.576)%
  --(11.035,6.578)--(11.037,6.580)--(11.040,6.582)--(11.043,6.584)--(11.046,6.587)--(11.049,6.589)%
  --(11.052,6.591)--(11.055,6.593)--(11.058,6.595)--(11.061,6.597)--(11.064,6.599)--(11.067,6.601)%
  --(11.070,6.603)--(11.073,6.605)--(11.076,6.607)--(11.079,6.610)--(11.082,6.612)--(11.085,6.614)%
  --(11.088,6.616)--(11.091,6.618)--(11.094,6.620)--(11.097,6.622)--(11.100,6.624)--(11.103,6.626)%
  --(11.106,6.628)--(11.109,6.630)--(11.112,6.633)--(11.115,6.635)--(11.118,6.637)--(11.121,6.639)%
  --(11.124,6.641)--(11.127,6.643)--(11.130,6.645)--(11.133,6.647)--(11.136,6.649)--(11.139,6.651)%
  --(11.142,6.653)--(11.145,6.656)--(11.148,6.658)--(11.151,6.660)--(11.154,6.662)--(11.157,6.664)%
  --(11.160,6.666)--(11.163,6.668)--(11.166,6.670)--(11.169,6.672)--(11.172,6.674)--(11.175,6.676)%
  --(11.178,6.679)--(11.181,6.681)--(11.184,6.683)--(11.187,6.685)--(11.190,6.687)--(11.193,6.689)%
  --(11.196,6.691)--(11.199,6.693)--(11.202,6.695)--(11.205,6.697)--(11.208,6.699)--(11.211,6.702)%
  --(11.214,6.704)--(11.217,6.706)--(11.220,6.708)--(11.223,6.710)--(11.226,6.712)--(11.229,6.714)%
  --(11.232,6.716)--(11.235,6.718)--(11.238,6.720)--(11.241,6.722)--(11.244,6.725)--(11.247,6.727)%
  --(11.249,6.729)--(11.252,6.731)--(11.255,6.733)--(11.258,6.735)--(11.261,6.737)--(11.264,6.739)%
  --(11.267,6.741)--(11.270,6.743)--(11.273,6.745)--(11.276,6.748)--(11.279,6.750)--(11.282,6.752)%
  --(11.285,6.754)--(11.288,6.756)--(11.291,6.758)--(11.294,6.760)--(11.297,6.762)--(11.300,6.764)%
  --(11.303,6.766)--(11.306,6.768)--(11.309,6.771)--(11.312,6.773)--(11.315,6.775)--(11.318,6.777)%
  --(11.321,6.779)--(11.324,6.781)--(11.327,6.783)--(11.330,6.785)--(11.333,6.787)--(11.336,6.789)%
  --(11.339,6.791)--(11.342,6.794)--(11.345,6.796)--(11.348,6.798)--(11.351,6.800)--(11.354,6.802)%
  --(11.357,6.804)--(11.360,6.806)--(11.363,6.808)--(11.366,6.810)--(11.369,6.812)--(11.372,6.814)%
  --(11.375,6.817)--(11.378,6.819)--(11.381,6.821)--(11.384,6.823)--(11.387,6.825)--(11.390,6.827)%
  --(11.393,6.829)--(11.396,6.831)--(11.399,6.833)--(11.402,6.835)--(11.405,6.837)--(11.408,6.840)%
  --(11.411,6.842)--(11.414,6.844)--(11.417,6.846)--(11.420,6.848)--(11.423,6.850)--(11.426,6.852)%
  --(11.429,6.854)--(11.432,6.856)--(11.435,6.858)--(11.438,6.861)--(11.441,6.863)--(11.444,6.865)%
  --(11.447,6.867)--(11.450,6.869)--(11.453,6.871)--(11.456,6.873)--(11.458,6.875)--(11.461,6.877)%
  --(11.464,6.879)--(11.467,6.881)--(11.470,6.884)--(11.473,6.886)--(11.476,6.888)--(11.479,6.890)%
  --(11.482,6.892)--(11.485,6.894)--(11.488,6.896)--(11.491,6.898)--(11.494,6.900)--(11.497,6.902)%
  --(11.500,6.904)--(11.503,6.907)--(11.506,6.909)--(11.509,6.911)--(11.512,6.913)--(11.515,6.915)%
  --(11.518,6.917)--(11.521,6.919)--(11.524,6.921)--(11.527,6.923)--(11.530,6.925)--(11.533,6.928)%
  --(11.536,6.930)--(11.539,6.932)--(11.542,6.934)--(11.545,6.936)--(11.548,6.938)--(11.551,6.940)%
  --(11.554,6.942)--(11.557,6.944)--(11.560,6.946)--(11.563,6.948)--(11.566,6.951)--(11.569,6.953)%
  --(11.572,6.955)--(11.575,6.957)--(11.578,6.959)--(11.581,6.961)--(11.584,6.963)--(11.587,6.965)%
  --(11.590,6.967)--(11.593,6.969)--(11.596,6.971)--(11.599,6.974)--(11.602,6.976)--(11.605,6.978)%
  --(11.608,6.980)--(11.611,6.982)--(11.614,6.984)--(11.617,6.986)--(11.620,6.988)--(11.623,6.990)%
  --(11.626,6.992)--(11.629,6.995)--(11.632,6.997)--(11.635,6.999)--(11.638,7.001)--(11.641,7.003)%
  --(11.644,7.005)--(11.647,7.007)--(11.650,7.009)--(11.653,7.011)--(11.656,7.013)--(11.659,7.015)%
  --(11.662,7.018)--(11.665,7.020)--(11.667,7.022)--(11.670,7.024)--(11.673,7.026)--(11.676,7.028)%
  --(11.679,7.030)--(11.682,7.032)--(11.685,7.034)--(11.688,7.036)--(11.691,7.039)--(11.694,7.041)%
  --(11.697,7.043)--(11.700,7.045)--(11.703,7.047)--(11.706,7.049)--(11.709,7.051)--(11.712,7.053)%
  --(11.715,7.055)--(11.718,7.057)--(11.721,7.059)--(11.724,7.062)--(11.727,7.064)--(11.730,7.066)%
  --(11.733,7.068)--(11.736,7.070)--(11.739,7.072)--(11.742,7.074)--(11.745,7.076)--(11.748,7.078)%
  --(11.751,7.080)--(11.754,7.083)--(11.757,7.085)--(11.760,7.087)--(11.763,7.089)--(11.766,7.091)%
  --(11.769,7.093)--(11.772,7.095)--(11.775,7.097)--(11.778,7.099)--(11.781,7.101)--(11.784,7.104)%
  --(11.787,7.106)--(11.790,7.108)--(11.793,7.110)--(11.796,7.112)--(11.799,7.114)--(11.802,7.116)%
  --(11.805,7.118)--(11.808,7.120)--(11.811,7.122)--(11.814,7.124)--(11.817,7.127)--(11.820,7.129)%
  --(11.823,7.131)--(11.826,7.133)--(11.829,7.135)--(11.832,7.137)--(11.835,7.139)--(11.838,7.141)%
  --(11.841,7.143)--(11.844,7.145)--(11.847,7.148)--(11.850,7.150)--(11.853,7.152)--(11.856,7.154)%
  --(11.859,7.156)--(11.862,7.158)--(11.865,7.160)--(11.868,7.162)--(11.871,7.164)--(11.874,7.166)%
  --(11.876,7.168)--(11.879,7.171)--(11.882,7.173)--(11.885,7.175)--(11.888,7.177)--(11.891,7.179)%
  --(11.894,7.181)--(11.897,7.183)--(11.900,7.185)--(11.903,7.187)--(11.906,7.189)--(11.909,7.192)%
  --(11.912,7.194)--(11.915,7.196)--(11.918,7.198)--(11.921,7.200)--(11.924,7.202)--(11.927,7.204)%
  --(11.930,7.206)--(11.933,7.208)--(11.936,7.210)--(11.939,7.213)--(11.942,7.215)--(11.945,7.217)%
  --(11.948,7.219)--(11.951,7.221)--(11.954,7.223)--(11.957,7.225)--(11.960,7.227)--(11.963,7.229)%
  --(11.966,7.231)--(11.969,7.234)--(11.972,7.236)--(11.975,7.238)--(11.978,7.240)--(11.981,7.242)%
  --(11.984,7.244)--(11.987,7.246)--(11.990,7.248)--(11.993,7.250)--(11.996,7.252)--(11.999,7.254)%
  --(12.002,7.257)--(12.005,7.259)--(12.008,7.261)--(12.011,7.263)--(12.014,7.265)--(12.017,7.267)%
  --(12.020,7.269)--(12.023,7.271)--(12.026,7.273)--(12.029,7.275)--(12.032,7.278)--(12.035,7.280)%
  --(12.038,7.282)--(12.041,7.284)--(12.044,7.286)--(12.047,7.288)--(12.050,7.290)--(12.053,7.292)%
  --(12.056,7.294)--(12.059,7.296)--(12.062,7.299)--(12.065,7.301)--(12.068,7.303)--(12.071,7.305)%
  --(12.074,7.307)--(12.077,7.309)--(12.080,7.311)--(12.083,7.313)--(12.085,7.315)--(12.088,7.317)%
  --(12.091,7.320)--(12.094,7.322)--(12.097,7.324)--(12.100,7.326)--(12.103,7.328)--(12.106,7.330)%
  --(12.109,7.332)--(12.112,7.334)--(12.115,7.336)--(12.118,7.338)--(12.121,7.340)--(12.124,7.343)%
  --(12.127,7.345)--(12.130,7.347)--(12.133,7.349)--(12.136,7.351)--(12.139,7.353)--(12.142,7.355)%
  --(12.145,7.357)--(12.148,7.359)--(12.151,7.361)--(12.154,7.364)--(12.157,7.366)--(12.160,7.368)%
  --(12.163,7.370)--(12.166,7.372)--(12.169,7.374)--(12.172,7.376)--(12.175,7.378)--(12.178,7.380)%
  --(12.181,7.382)--(12.184,7.385)--(12.187,7.387)--(12.190,7.389)--(12.193,7.391)--(12.196,7.393)%
  --(12.199,7.395)--(12.202,7.397)--(12.205,7.399)--(12.208,7.401)--(12.211,7.403)--(12.214,7.406)%
  --(12.217,7.408)--(12.220,7.410)--(12.223,7.412)--(12.226,7.414)--(12.229,7.416)--(12.232,7.418)%
  --(12.235,7.420)--(12.238,7.422)--(12.241,7.424)--(12.244,7.427)--(12.247,7.429)--(12.250,7.431)%
  --(12.253,7.433)--(12.256,7.435)--(12.259,7.437)--(12.262,7.439)--(12.265,7.441)--(12.268,7.443)%
  --(12.271,7.445)--(12.274,7.448)--(12.277,7.450)--(12.280,7.452)--(12.283,7.454)--(12.286,7.456)%
  --(12.289,7.458)--(12.292,7.460)--(12.295,7.462)--(12.297,7.464)--(12.300,7.466)--(12.303,7.469)%
  --(12.306,7.471)--(12.309,7.473)--(12.312,7.475)--(12.315,7.477)--(12.318,7.479)--(12.321,7.481)%
  --(12.324,7.483)--(12.327,7.485)--(12.330,7.487)--(12.333,7.490)--(12.336,7.492)--(12.339,7.494)%
  --(12.342,7.496)--(12.345,7.498)--(12.348,7.500)--(12.351,7.502)--(12.354,7.504)--(12.357,7.506)%
  --(12.360,7.508)--(12.363,7.511)--(12.366,7.513)--(12.369,7.515)--(12.372,7.517)--(12.375,7.519)%
  --(12.378,7.521)--(12.381,7.523)--(12.384,7.525)--(12.387,7.527)--(12.390,7.529)--(12.393,7.532)%
  --(12.396,7.534)--(12.399,7.536)--(12.402,7.538)--(12.405,7.540)--(12.408,7.542)--(12.411,7.544)%
  --(12.414,7.546)--(12.417,7.548)--(12.420,7.550)--(12.423,7.553)--(12.426,7.555)--(12.429,7.557)%
  --(12.432,7.559)--(12.435,7.561)--(12.438,7.563)--(12.441,7.565)--(12.444,7.567)--(12.447,7.569)%
  --(12.450,7.571)--(12.453,7.574)--(12.456,7.576)--(12.459,7.578)--(12.462,7.580)--(12.465,7.582)%
  --(12.468,7.584)--(12.471,7.586)--(12.474,7.588)--(12.477,7.590)--(12.480,7.592)--(12.483,7.595)%
  --(12.486,7.597)--(12.489,7.599)--(12.492,7.601)--(12.495,7.603)--(12.498,7.605)--(12.501,7.607)%
  --(12.504,7.609)--(12.506,7.611)--(12.509,7.613)--(12.512,7.616)--(12.515,7.618)--(12.518,7.620)%
  --(12.521,7.622)--(12.524,7.624)--(12.527,7.626)--(12.530,7.628)--(12.533,7.630)--(12.536,7.632)%
  --(12.539,7.634)--(12.542,7.637)--(12.545,7.639)--(12.548,7.641)--(12.551,7.643)--(12.554,7.645)%
  --(12.557,7.647)--(12.560,7.649)--(12.563,7.651)--(12.566,7.653)--(12.569,7.655)--(12.572,7.658)%
  --(12.575,7.660)--(12.578,7.662)--(12.581,7.664)--(12.584,7.666)--(12.587,7.668)--(12.590,7.670)%
  --(12.593,7.672)--(12.596,7.674)--(12.599,7.676)--(12.602,7.679)--(12.605,7.681)--(12.608,7.683)%
  --(12.611,7.685)--(12.614,7.687)--(12.617,7.689)--(12.620,7.691)--(12.623,7.693)--(12.626,7.695)%
  --(12.629,7.698)--(12.632,7.700)--(12.635,7.702)--(12.638,7.704)--(12.641,7.706)--(12.644,7.708)%
  --(12.647,7.710)--(12.650,7.712)--(12.653,7.714)--(12.656,7.716)--(12.659,7.719)--(12.662,7.721)%
  --(12.665,7.723)--(12.668,7.725)--(12.671,7.727)--(12.674,7.729)--(12.677,7.731)--(12.680,7.733)%
  --(12.683,7.735)--(12.686,7.737)--(12.689,7.740)--(12.692,7.742)--(12.695,7.744)--(12.698,7.746)%
  --(12.701,7.748)--(12.704,7.750)--(12.707,7.752)--(12.710,7.754)--(12.713,7.756)--(12.715,7.758)%
  --(12.718,7.761)--(12.721,7.763)--(12.724,7.765)--(12.727,7.767)--(12.730,7.769)--(12.733,7.771)%
  --(12.736,7.773)--(12.739,7.775)--(12.742,7.777)--(12.745,7.779)--(12.748,7.782)--(12.751,7.784)%
  --(12.754,7.786)--(12.757,7.788)--(12.760,7.790)--(12.763,7.792)--(12.766,7.794)--(12.769,7.796)%
  --(12.772,7.798)--(12.775,7.801)--(12.778,7.803)--(12.781,7.805)--(12.784,7.807)--(12.787,7.809)%
  --(12.790,7.811)--(12.793,7.813)--(12.796,7.815)--(12.799,7.817)--(12.802,7.819)--(12.805,7.822)%
  --(12.808,7.824)--(12.811,7.826)--(12.814,7.828)--(12.817,7.830)--(12.820,7.832)--(12.823,7.834)%
  --(12.826,7.836)--(12.829,7.838)--(12.832,7.840)--(12.835,7.843)--(12.838,7.845)--(12.841,7.847)%
  --(12.844,7.849)--(12.847,7.851)--(12.850,7.853)--(12.853,7.855)--(12.856,7.857)--(12.859,7.859)%
  --(12.862,7.861)--(12.865,7.864)--(12.868,7.866)--(12.871,7.868)--(12.874,7.870)--(12.877,7.872)%
  --(12.880,7.874)--(12.883,7.876)--(12.886,7.878)--(12.889,7.880)--(12.892,7.883)--(12.895,7.885)%
  --(12.898,7.887)--(12.901,7.889)--(12.904,7.891)--(12.907,7.893)--(12.910,7.895)--(12.913,7.897)%
  --(12.916,7.899)--(12.919,7.901)--(12.922,7.904)--(12.924,7.906)--(12.927,7.908)--(12.930,7.910)%
  --(12.933,7.912)--(12.936,7.914)--(12.939,7.916)--(12.942,7.918)--(12.945,7.920)--(12.948,7.922)%
  --(12.951,7.925)--(12.954,7.927)--(12.957,7.929)--(12.960,7.931)--(12.963,7.933)--(12.966,7.935)%
  --(12.969,7.937)--(12.972,7.939)--(12.975,7.941)--(12.978,7.944)--(12.981,7.946)--(12.984,7.948)%
  --(12.987,7.950)--(12.990,7.952)--(12.993,7.954)--(12.996,7.956)--(12.999,7.958)--(13.002,7.960)%
  --(13.005,7.962)--(13.008,7.965)--(13.011,7.967)--(13.014,7.969)--(13.017,7.971)--(13.020,7.973)%
  --(13.023,7.975)--(13.026,7.977)--(13.029,7.979)--(13.032,7.981)--(13.035,7.983)--(13.038,7.986)%
  --(13.041,7.988)--(13.044,7.990)--(13.047,7.992)--(13.050,7.994)--(13.053,7.996)--(13.056,7.998)%
  --(13.059,8.000)--(13.062,8.002)--(13.065,8.005)--(13.068,8.007)--(13.071,8.009)--(13.074,8.011)%
  --(13.077,8.013)--(13.080,8.015)--(13.083,8.017)--(13.086,8.019)--(13.089,8.021)--(13.092,8.023)%
  --(13.095,8.026)--(13.098,8.028)--(13.101,8.030)--(13.104,8.032)--(13.107,8.034)--(13.110,8.036)%
  --(13.113,8.038)--(13.116,8.040)--(13.119,8.042)--(13.122,8.045)--(13.125,8.047)--(13.128,8.049)%
  --(13.131,8.051)--(13.133,8.053)--(13.136,8.055)--(13.139,8.057)--(13.142,8.059)--(13.145,8.061)%
  --(13.148,8.063)--(13.151,8.066)--(13.154,8.068)--(13.157,8.070)--(13.160,8.072)--(13.163,8.074)%
  --(13.166,8.076)--(13.169,8.078)--(13.172,8.080)--(13.175,8.082)--(13.178,8.084)--(13.181,8.087)%
  --(13.184,8.089)--(13.187,8.091)--(13.190,8.093)--(13.193,8.095)--(13.196,8.097)--(13.199,8.099)%
  --(13.202,8.101)--(13.205,8.103)--(13.208,8.106)--(13.211,8.108)--(13.214,8.110)--(13.217,8.112)%
  --(13.220,8.114)--(13.223,8.116)--(13.226,8.118)--(13.229,8.120)--(13.232,8.122)--(13.235,8.124)%
  --(13.238,8.127)--(13.241,8.129)--(13.244,8.131)--(13.247,8.133)--(13.250,8.135)--(13.253,8.137)%
  --(13.256,8.139)--(13.259,8.141)--(13.262,8.143)--(13.265,8.146)--(13.268,8.148)--(13.271,8.150)%
  --(13.274,8.152)--(13.277,8.154)--(13.280,8.156)--(13.283,8.158)--(13.286,8.160)--(13.289,8.162)%
  --(13.292,8.164)--(13.295,8.167)--(13.298,8.169)--(13.301,8.171)--(13.304,8.173)--(13.307,8.175)%
  --(13.310,8.177)--(13.313,8.179)--(13.316,8.181)--(13.319,8.183)--(13.322,8.186)--(13.325,8.188)%
  --(13.328,8.190)--(13.331,8.192)--(13.334,8.194)--(13.337,8.196)--(13.340,8.198)--(13.342,8.200)%
  --(13.345,8.202)--(13.348,8.204)--(13.351,8.207)--(13.354,8.209)--(13.357,8.211)--(13.360,8.213)%
  --(13.363,8.215)--(13.366,8.217)--(13.369,8.219)--(13.372,8.221)--(13.375,8.223)--(13.378,8.226)%
  --(13.381,8.228)--(13.384,8.230)--(13.387,8.232)--(13.390,8.234)--(13.393,8.236)--(13.396,8.238)%
  --(13.399,8.240)--(13.402,8.242)--(13.405,8.244)--(13.408,8.247)--(13.411,8.249)--(13.414,8.251)%
  --(13.417,8.253)--(13.420,8.255)--(13.423,8.257)--(13.426,8.259)--(13.429,8.261)--(13.432,8.263)%
  --(13.435,8.266)--(13.438,8.268)--(13.441,8.270)--(13.444,8.272);
\gpcolor{color=gp lt color border}
\node[gp node left] at (2.972,7.373) {$\rho \approx \rho_{\rm{max}}$};
\gpcolor{rgb color={0.902,0.624,0.000}}
\draw[gp path] (1.872,7.373)--(2.788,7.373);
\draw[gp path] (1.507,1.622)--(1.510,1.622)--(1.513,1.622)--(1.516,1.622)--(1.519,1.622)%
  --(1.522,1.622)--(1.525,1.622)--(1.528,1.622)--(1.531,1.622)--(1.534,1.622)--(1.537,1.622)%
  --(1.540,1.622)--(1.543,1.622)--(1.546,1.622)--(1.549,1.622)--(1.552,1.622)--(1.555,1.622)%
  --(1.558,1.622)--(1.561,1.621)--(1.564,1.621)--(1.567,1.621)--(1.570,1.621)--(1.573,1.621)%
  --(1.576,1.621)--(1.579,1.621)--(1.582,1.621)--(1.585,1.621)--(1.588,1.621)--(1.591,1.621)%
  --(1.594,1.621)--(1.597,1.621)--(1.600,1.621)--(1.603,1.621)--(1.606,1.621)--(1.609,1.621)%
  --(1.611,1.621)--(1.614,1.621)--(1.617,1.621)--(1.620,1.621)--(1.623,1.621)--(1.626,1.621)%
  --(1.629,1.621)--(1.632,1.621)--(1.635,1.621)--(1.638,1.621)--(1.641,1.621)--(1.644,1.621)%
  --(1.647,1.621)--(1.650,1.621)--(1.653,1.621)--(1.656,1.621)--(1.659,1.620)--(1.662,1.620)%
  --(1.665,1.620)--(1.668,1.620)--(1.671,1.620)--(1.674,1.620)--(1.677,1.620)--(1.680,1.620)%
  --(1.683,1.620)--(1.686,1.620)--(1.689,1.620)--(1.692,1.620)--(1.695,1.620)--(1.698,1.620)%
  --(1.701,1.620)--(1.704,1.620)--(1.707,1.620)--(1.710,1.620)--(1.713,1.620)--(1.716,1.620)%
  --(1.719,1.620)--(1.722,1.620)--(1.725,1.620)--(1.728,1.620)--(1.731,1.620)--(1.734,1.620)%
  --(1.737,1.620)--(1.740,1.620)--(1.743,1.620)--(1.746,1.620)--(1.749,1.620)--(1.752,1.620)%
  --(1.755,1.620)--(1.758,1.620)--(1.761,1.620)--(1.764,1.620)--(1.767,1.620)--(1.770,1.620)%
  --(1.773,1.620)--(1.776,1.620)--(1.779,1.620)--(1.782,1.620)--(1.785,1.620)--(1.788,1.620)%
  --(1.791,1.620)--(1.794,1.620)--(1.797,1.620)--(1.800,1.620)--(1.803,1.620)--(1.806,1.620)%
  --(1.809,1.620)--(1.812,1.620)--(1.815,1.620)--(1.818,1.620)--(1.820,1.620)--(1.823,1.620)%
  --(1.826,1.620)--(1.829,1.620)--(1.832,1.620)--(1.835,1.620)--(1.838,1.620)--(1.841,1.620)%
  --(1.844,1.620)--(1.847,1.620)--(1.850,1.620)--(1.853,1.620)--(1.856,1.620)--(1.859,1.620)%
  --(1.862,1.621)--(1.865,1.621)--(1.868,1.621)--(1.871,1.621)--(1.874,1.621)--(1.877,1.621)%
  --(1.880,1.621)--(1.883,1.621)--(1.886,1.621)--(1.889,1.621)--(1.892,1.621)--(1.895,1.621)%
  --(1.898,1.621)--(1.901,1.621)--(1.904,1.621)--(1.907,1.621)--(1.910,1.621)--(1.913,1.621)%
  --(1.916,1.621)--(1.919,1.621)--(1.922,1.621)--(1.925,1.621)--(1.928,1.621)--(1.931,1.622)%
  --(1.934,1.622)--(1.937,1.622)--(1.940,1.622)--(1.943,1.622)--(1.946,1.622)--(1.949,1.622)%
  --(1.952,1.622)--(1.955,1.622)--(1.958,1.622)--(1.961,1.622)--(1.964,1.622)--(1.967,1.622)%
  --(1.970,1.622)--(1.973,1.622)--(1.976,1.623)--(1.979,1.623)--(1.982,1.623)--(1.985,1.623)%
  --(1.988,1.623)--(1.991,1.623)--(1.994,1.623)--(1.997,1.623)--(2.000,1.623)--(2.003,1.623)%
  --(2.006,1.623)--(2.009,1.624)--(2.012,1.624)--(2.015,1.624)--(2.018,1.624)--(2.021,1.624)%
  --(2.024,1.624)--(2.027,1.624)--(2.029,1.624)--(2.032,1.624)--(2.035,1.624)--(2.038,1.625)%
  --(2.041,1.625)--(2.044,1.625)--(2.047,1.625)--(2.050,1.625)--(2.053,1.625)--(2.056,1.625)%
  --(2.059,1.625)--(2.062,1.625)--(2.065,1.626)--(2.068,1.626)--(2.071,1.626)--(2.074,1.626)%
  --(2.077,1.626)--(2.080,1.626)--(2.083,1.626)--(2.086,1.627)--(2.089,1.627)--(2.092,1.627)%
  --(2.095,1.627)--(2.098,1.627)--(2.101,1.627)--(2.104,1.627)--(2.107,1.628)--(2.110,1.628)%
  --(2.113,1.628)--(2.116,1.628)--(2.119,1.628)--(2.122,1.628)--(2.125,1.628)--(2.128,1.629)%
  --(2.131,1.629)--(2.134,1.629)--(2.137,1.629)--(2.140,1.629)--(2.143,1.629)--(2.146,1.630)%
  --(2.149,1.630)--(2.152,1.630)--(2.155,1.630)--(2.158,1.630)--(2.161,1.630)--(2.164,1.631)%
  --(2.167,1.631)--(2.170,1.631)--(2.173,1.631)--(2.176,1.631)--(2.179,1.632)--(2.182,1.632)%
  --(2.185,1.632)--(2.188,1.632)--(2.191,1.632)--(2.194,1.633)--(2.197,1.633)--(2.200,1.633)%
  --(2.203,1.633)--(2.206,1.633)--(2.209,1.634)--(2.212,1.634)--(2.215,1.634)--(2.218,1.634)%
  --(2.221,1.634)--(2.224,1.635)--(2.227,1.635)--(2.230,1.635)--(2.233,1.635)--(2.236,1.635)%
  --(2.238,1.636)--(2.241,1.636)--(2.244,1.636)--(2.247,1.636)--(2.250,1.637)--(2.253,1.637)%
  --(2.256,1.637)--(2.259,1.637)--(2.262,1.638)--(2.265,1.638)--(2.268,1.638)--(2.271,1.638)%
  --(2.274,1.639)--(2.277,1.639)--(2.280,1.639)--(2.283,1.639)--(2.286,1.640)--(2.289,1.640)%
  --(2.292,1.640)--(2.295,1.640)--(2.298,1.641)--(2.301,1.641)--(2.304,1.641)--(2.307,1.641)%
  --(2.310,1.642)--(2.313,1.642)--(2.316,1.642)--(2.319,1.643)--(2.322,1.643)--(2.325,1.643)%
  --(2.328,1.643)--(2.331,1.644)--(2.334,1.644)--(2.337,1.644)--(2.340,1.645)--(2.343,1.645)%
  --(2.346,1.645)--(2.349,1.645)--(2.352,1.646)--(2.355,1.646)--(2.358,1.646)--(2.361,1.647)%
  --(2.364,1.647)--(2.367,1.647)--(2.370,1.648)--(2.373,1.648)--(2.376,1.648)--(2.379,1.649)%
  --(2.382,1.649)--(2.385,1.649)--(2.388,1.649)--(2.391,1.650)--(2.394,1.650)--(2.397,1.650)%
  --(2.400,1.651)--(2.403,1.651)--(2.406,1.651)--(2.409,1.652)--(2.412,1.652)--(2.415,1.653)%
  --(2.418,1.653)--(2.421,1.653)--(2.424,1.654)--(2.427,1.654)--(2.430,1.654)--(2.433,1.655)%
  --(2.436,1.655)--(2.439,1.655)--(2.442,1.656)--(2.445,1.656)--(2.447,1.656)--(2.450,1.657)%
  --(2.453,1.657)--(2.456,1.658)--(2.459,1.658)--(2.462,1.658)--(2.465,1.659)--(2.468,1.659)%
  --(2.471,1.659)--(2.474,1.660)--(2.477,1.660)--(2.480,1.661)--(2.483,1.661)--(2.486,1.661)%
  --(2.489,1.662)--(2.492,1.662)--(2.495,1.663)--(2.498,1.663)--(2.501,1.663)--(2.504,1.664)%
  --(2.507,1.664)--(2.510,1.665)--(2.513,1.665)--(2.516,1.666)--(2.519,1.666)--(2.522,1.666)%
  --(2.525,1.667)--(2.528,1.667)--(2.531,1.668)--(2.534,1.668)--(2.537,1.669)--(2.540,1.669)%
  --(2.543,1.669)--(2.546,1.670)--(2.549,1.670)--(2.552,1.671)--(2.555,1.671)--(2.558,1.672)%
  --(2.561,1.672)--(2.564,1.673)--(2.567,1.673)--(2.570,1.673)--(2.573,1.674)--(2.576,1.674)%
  --(2.579,1.675)--(2.582,1.675)--(2.585,1.676)--(2.588,1.676)--(2.591,1.677)--(2.594,1.677)%
  --(2.597,1.678)--(2.600,1.678)--(2.603,1.679)--(2.606,1.679)--(2.609,1.680)--(2.612,1.680)%
  --(2.615,1.681)--(2.618,1.681)--(2.621,1.682)--(2.624,1.682)--(2.627,1.683)--(2.630,1.683)%
  --(2.633,1.684)--(2.636,1.684)--(2.639,1.685)--(2.642,1.685)--(2.645,1.686)--(2.648,1.686)%
  --(2.651,1.687)--(2.654,1.687)--(2.656,1.688)--(2.659,1.688)--(2.662,1.689)--(2.665,1.689)%
  --(2.668,1.690)--(2.671,1.690)--(2.674,1.691)--(2.677,1.691)--(2.680,1.692)--(2.683,1.693)%
  --(2.686,1.693)--(2.689,1.694)--(2.692,1.694)--(2.695,1.695)--(2.698,1.695)--(2.701,1.696)%
  --(2.704,1.696)--(2.707,1.697)--(2.710,1.698)--(2.713,1.698)--(2.716,1.699)--(2.719,1.699)%
  --(2.722,1.700)--(2.725,1.700)--(2.728,1.701)--(2.731,1.702)--(2.734,1.702)--(2.737,1.703)%
  --(2.740,1.703)--(2.743,1.704)--(2.746,1.704)--(2.749,1.705)--(2.752,1.706)--(2.755,1.706)%
  --(2.758,1.707)--(2.761,1.707)--(2.764,1.708)--(2.767,1.709)--(2.770,1.709)--(2.773,1.710)%
  --(2.776,1.710)--(2.779,1.711)--(2.782,1.712)--(2.785,1.712)--(2.788,1.713)--(2.791,1.714)%
  --(2.794,1.714)--(2.797,1.715)--(2.800,1.715)--(2.803,1.716)--(2.806,1.717)--(2.809,1.717)%
  --(2.812,1.718)--(2.815,1.719)--(2.818,1.719)--(2.821,1.720)--(2.824,1.721)--(2.827,1.721)%
  --(2.830,1.722)--(2.833,1.723)--(2.836,1.723)--(2.839,1.724)--(2.842,1.725)--(2.845,1.725)%
  --(2.848,1.726)--(2.851,1.727)--(2.854,1.727)--(2.857,1.728)--(2.860,1.729)--(2.863,1.729)%
  --(2.866,1.730)--(2.868,1.731)--(2.871,1.731)--(2.874,1.732)--(2.877,1.733)--(2.880,1.733)%
  --(2.883,1.734)--(2.886,1.735)--(2.889,1.735)--(2.892,1.736)--(2.895,1.737)--(2.898,1.738)%
  --(2.901,1.738)--(2.904,1.739)--(2.907,1.740)--(2.910,1.740)--(2.913,1.741)--(2.916,1.742)%
  --(2.919,1.743)--(2.922,1.743)--(2.925,1.744)--(2.928,1.745)--(2.931,1.745)--(2.934,1.746)%
  --(2.937,1.747)--(2.940,1.748)--(2.943,1.748)--(2.946,1.749)--(2.949,1.750)--(2.952,1.751)%
  --(2.955,1.751)--(2.958,1.752)--(2.961,1.753)--(2.964,1.754)--(2.967,1.754)--(2.970,1.755)%
  --(2.973,1.756)--(2.976,1.757)--(2.979,1.757)--(2.982,1.758)--(2.985,1.759)--(2.988,1.760)%
  --(2.991,1.760)--(2.994,1.761)--(2.997,1.762)--(3.000,1.763)--(3.003,1.764)--(3.006,1.764)%
  --(3.009,1.765)--(3.012,1.766)--(3.015,1.767)--(3.018,1.767)--(3.021,1.768)--(3.024,1.769)%
  --(3.027,1.770)--(3.030,1.771)--(3.033,1.771)--(3.036,1.772)--(3.039,1.773)--(3.042,1.774)%
  --(3.045,1.775)--(3.048,1.776)--(3.051,1.776)--(3.054,1.777)--(3.057,1.778)--(3.060,1.779)%
  --(3.063,1.780)--(3.066,1.780)--(3.069,1.781)--(3.072,1.782)--(3.075,1.783)--(3.077,1.784)%
  --(3.080,1.785)--(3.083,1.785)--(3.086,1.786)--(3.089,1.787)--(3.092,1.788)--(3.095,1.789)%
  --(3.098,1.790)--(3.101,1.791)--(3.104,1.791)--(3.107,1.792)--(3.110,1.793)--(3.113,1.794)%
  --(3.116,1.795)--(3.119,1.796)--(3.122,1.797)--(3.125,1.797)--(3.128,1.798)--(3.131,1.799)%
  --(3.134,1.800)--(3.137,1.801)--(3.140,1.802)--(3.143,1.803)--(3.146,1.804)--(3.149,1.804)%
  --(3.152,1.805)--(3.155,1.806)--(3.158,1.807)--(3.161,1.808)--(3.164,1.809)--(3.167,1.810)%
  --(3.170,1.811)--(3.173,1.812)--(3.176,1.812)--(3.179,1.813)--(3.182,1.814)--(3.185,1.815)%
  --(3.188,1.816)--(3.191,1.817)--(3.194,1.818)--(3.197,1.819)--(3.200,1.820)--(3.203,1.821)%
  --(3.206,1.822)--(3.209,1.823)--(3.212,1.823)--(3.215,1.824)--(3.218,1.825)--(3.221,1.826)%
  --(3.224,1.827)--(3.227,1.828)--(3.230,1.829)--(3.233,1.830)--(3.236,1.831)--(3.239,1.832)%
  --(3.242,1.833)--(3.245,1.834)--(3.248,1.835)--(3.251,1.836)--(3.254,1.837)--(3.257,1.838)%
  --(3.260,1.839)--(3.263,1.840)--(3.266,1.841)--(3.269,1.841)--(3.272,1.842)--(3.275,1.843)%
  --(3.278,1.844)--(3.281,1.845)--(3.284,1.846)--(3.286,1.847)--(3.289,1.848)--(3.292,1.849)%
  --(3.295,1.850)--(3.298,1.851)--(3.301,1.852)--(3.304,1.853)--(3.307,1.854)--(3.310,1.855)%
  --(3.313,1.856)--(3.316,1.857)--(3.319,1.858)--(3.322,1.859)--(3.325,1.860)--(3.328,1.861)%
  --(3.331,1.862)--(3.334,1.863)--(3.337,1.864)--(3.340,1.865)--(3.343,1.866)--(3.346,1.867)%
  --(3.349,1.868)--(3.352,1.869)--(3.355,1.870)--(3.358,1.871)--(3.361,1.872)--(3.364,1.873)%
  --(3.367,1.874)--(3.370,1.875)--(3.373,1.876)--(3.376,1.878)--(3.379,1.879)--(3.382,1.880)%
  --(3.385,1.881)--(3.388,1.882)--(3.391,1.883)--(3.394,1.884)--(3.397,1.885)--(3.400,1.886)%
  --(3.403,1.887)--(3.406,1.888)--(3.409,1.889)--(3.412,1.890)--(3.415,1.891)--(3.418,1.892)%
  --(3.421,1.893)--(3.424,1.894)--(3.427,1.895)--(3.430,1.896)--(3.433,1.898)--(3.436,1.899)%
  --(3.439,1.900)--(3.442,1.901)--(3.445,1.902)--(3.448,1.903)--(3.451,1.904)--(3.454,1.905)%
  --(3.457,1.906)--(3.460,1.907)--(3.463,1.908)--(3.466,1.909)--(3.469,1.911)--(3.472,1.912)%
  --(3.475,1.913)--(3.478,1.914)--(3.481,1.915)--(3.484,1.916)--(3.487,1.917)--(3.490,1.918)%
  --(3.493,1.919)--(3.495,1.920)--(3.498,1.922)--(3.501,1.923)--(3.504,1.924)--(3.507,1.925)%
  --(3.510,1.926)--(3.513,1.927)--(3.516,1.928)--(3.519,1.929)--(3.522,1.931)--(3.525,1.932)%
  --(3.528,1.933)--(3.531,1.934)--(3.534,1.935)--(3.537,1.936)--(3.540,1.937)--(3.543,1.938)%
  --(3.546,1.940)--(3.549,1.941)--(3.552,1.942)--(3.555,1.943)--(3.558,1.944)--(3.561,1.945)%
  --(3.564,1.946)--(3.567,1.948)--(3.570,1.949)--(3.573,1.950)--(3.576,1.951)--(3.579,1.952)%
  --(3.582,1.953)--(3.585,1.955)--(3.588,1.956)--(3.591,1.957)--(3.594,1.958)--(3.597,1.959)%
  --(3.600,1.960)--(3.603,1.962)--(3.606,1.963)--(3.609,1.964)--(3.612,1.965)--(3.615,1.966)%
  --(3.618,1.967)--(3.621,1.969)--(3.624,1.970)--(3.627,1.971)--(3.630,1.972)--(3.633,1.973)%
  --(3.636,1.975)--(3.639,1.976)--(3.642,1.977)--(3.645,1.978)--(3.648,1.979)--(3.651,1.981)%
  --(3.654,1.982)--(3.657,1.983)--(3.660,1.984)--(3.663,1.985)--(3.666,1.987)--(3.669,1.988)%
  --(3.672,1.989)--(3.675,1.990)--(3.678,1.991)--(3.681,1.993)--(3.684,1.994)--(3.687,1.995)%
  --(3.690,1.996)--(3.693,1.997)--(3.696,1.999)--(3.699,2.000)--(3.702,2.001)--(3.704,2.002)%
  --(3.707,2.004)--(3.710,2.005)--(3.713,2.006)--(3.716,2.007)--(3.719,2.009)--(3.722,2.010)%
  --(3.725,2.011)--(3.728,2.012)--(3.731,2.013)--(3.734,2.015)--(3.737,2.016)--(3.740,2.017)%
  --(3.743,2.018)--(3.746,2.020)--(3.749,2.021)--(3.752,2.022)--(3.755,2.023)--(3.758,2.025)%
  --(3.761,2.026)--(3.764,2.027)--(3.767,2.029)--(3.770,2.030)--(3.773,2.031)--(3.776,2.032)%
  --(3.779,2.034)--(3.782,2.035)--(3.785,2.036)--(3.788,2.037)--(3.791,2.039)--(3.794,2.040)%
  --(3.797,2.041)--(3.800,2.042)--(3.803,2.044)--(3.806,2.045)--(3.809,2.046)--(3.812,2.048)%
  --(3.815,2.049)--(3.818,2.050)--(3.821,2.051)--(3.824,2.053)--(3.827,2.054)--(3.830,2.055)%
  --(3.833,2.057)--(3.836,2.058)--(3.839,2.059)--(3.842,2.061)--(3.845,2.062)--(3.848,2.063)%
  --(3.851,2.064)--(3.854,2.066)--(3.857,2.067)--(3.860,2.068)--(3.863,2.070)--(3.866,2.071)%
  --(3.869,2.072)--(3.872,2.074)--(3.875,2.075)--(3.878,2.076)--(3.881,2.078)--(3.884,2.079)%
  --(3.887,2.080)--(3.890,2.082)--(3.893,2.083)--(3.896,2.084)--(3.899,2.085)--(3.902,2.087)%
  --(3.905,2.088)--(3.908,2.089)--(3.911,2.091)--(3.914,2.092)--(3.916,2.093)--(3.919,2.095)%
  --(3.922,2.096)--(3.925,2.097)--(3.928,2.099)--(3.931,2.100)--(3.934,2.102)--(3.937,2.103)%
  --(3.940,2.104)--(3.943,2.106)--(3.946,2.107)--(3.949,2.108)--(3.952,2.110)--(3.955,2.111)%
  --(3.958,2.112)--(3.961,2.114)--(3.964,2.115)--(3.967,2.116)--(3.970,2.118)--(3.973,2.119)%
  --(3.976,2.120)--(3.979,2.122)--(3.982,2.123)--(3.985,2.125)--(3.988,2.126)--(3.991,2.127)%
  --(3.994,2.129)--(3.997,2.130)--(4.000,2.131)--(4.003,2.133)--(4.006,2.134)--(4.009,2.136)%
  --(4.012,2.137)--(4.015,2.138)--(4.018,2.140)--(4.021,2.141)--(4.024,2.142)--(4.027,2.144)%
  --(4.030,2.145)--(4.033,2.147)--(4.036,2.148)--(4.039,2.149)--(4.042,2.151)--(4.045,2.152)%
  --(4.048,2.154)--(4.051,2.155)--(4.054,2.156)--(4.057,2.158)--(4.060,2.159)--(4.063,2.161)%
  --(4.066,2.162)--(4.069,2.163)--(4.072,2.165)--(4.075,2.166)--(4.078,2.168)--(4.081,2.169)%
  --(4.084,2.170)--(4.087,2.172)--(4.090,2.173)--(4.093,2.175)--(4.096,2.176)--(4.099,2.177)%
  --(4.102,2.179)--(4.105,2.180)--(4.108,2.182)--(4.111,2.183)--(4.114,2.185)--(4.117,2.186)%
  --(4.120,2.187)--(4.123,2.189)--(4.125,2.190)--(4.128,2.192)--(4.131,2.193)--(4.134,2.195)%
  --(4.137,2.196)--(4.140,2.197)--(4.143,2.199)--(4.146,2.200)--(4.149,2.202)--(4.152,2.203)%
  --(4.155,2.205)--(4.158,2.206)--(4.161,2.208)--(4.164,2.209)--(4.167,2.210)--(4.170,2.212)%
  --(4.173,2.213)--(4.176,2.215)--(4.179,2.216)--(4.182,2.218)--(4.185,2.219)--(4.188,2.221)%
  --(4.191,2.222)--(4.194,2.223)--(4.197,2.225)--(4.200,2.226)--(4.203,2.228)--(4.206,2.229)%
  --(4.209,2.231)--(4.212,2.232)--(4.215,2.234)--(4.218,2.235)--(4.221,2.237)--(4.224,2.238)%
  --(4.227,2.240)--(4.230,2.241)--(4.233,2.242)--(4.236,2.244)--(4.239,2.245)--(4.242,2.247)%
  --(4.245,2.248)--(4.248,2.250)--(4.251,2.251)--(4.254,2.253)--(4.257,2.254)--(4.260,2.256)%
  --(4.263,2.257)--(4.266,2.259)--(4.269,2.260)--(4.272,2.262)--(4.275,2.263)--(4.278,2.265)%
  --(4.281,2.266)--(4.284,2.268)--(4.287,2.269)--(4.290,2.271)--(4.293,2.272)--(4.296,2.274)%
  --(4.299,2.275)--(4.302,2.277)--(4.305,2.278)--(4.308,2.280)--(4.311,2.281)--(4.314,2.283)%
  --(4.317,2.284)--(4.320,2.286)--(4.323,2.287)--(4.326,2.289)--(4.329,2.290)--(4.332,2.292)%
  --(4.334,2.293)--(4.337,2.295)--(4.340,2.296)--(4.343,2.298)--(4.346,2.299)--(4.349,2.301)%
  --(4.352,2.302)--(4.355,2.304)--(4.358,2.305)--(4.361,2.307)--(4.364,2.308)--(4.367,2.310)%
  --(4.370,2.311)--(4.373,2.313)--(4.376,2.314)--(4.379,2.316)--(4.382,2.318)--(4.385,2.319)%
  --(4.388,2.321)--(4.391,2.322)--(4.394,2.324)--(4.397,2.325)--(4.400,2.327)--(4.403,2.328)%
  --(4.406,2.330)--(4.409,2.331)--(4.412,2.333)--(4.415,2.334)--(4.418,2.336)--(4.421,2.337)%
  --(4.424,2.339)--(4.427,2.341)--(4.430,2.342)--(4.433,2.344)--(4.436,2.345)--(4.439,2.347)%
  --(4.442,2.348)--(4.445,2.350)--(4.448,2.351)--(4.451,2.353)--(4.454,2.354)--(4.457,2.356)%
  --(4.460,2.358)--(4.463,2.359)--(4.466,2.361)--(4.469,2.362)--(4.472,2.364)--(4.475,2.365)%
  --(4.478,2.367)--(4.481,2.368)--(4.484,2.370)--(4.487,2.372)--(4.490,2.373)--(4.493,2.375)%
  --(4.496,2.376)--(4.499,2.378)--(4.502,2.379)--(4.505,2.381)--(4.508,2.383)--(4.511,2.384)%
  --(4.514,2.386)--(4.517,2.387)--(4.520,2.389)--(4.523,2.390)--(4.526,2.392)--(4.529,2.394)%
  --(4.532,2.395)--(4.535,2.397)--(4.538,2.398)--(4.541,2.400)--(4.543,2.401)--(4.546,2.403)%
  --(4.549,2.405)--(4.552,2.406)--(4.555,2.408)--(4.558,2.409)--(4.561,2.411)--(4.564,2.413)%
  --(4.567,2.414)--(4.570,2.416)--(4.573,2.417)--(4.576,2.419)--(4.579,2.421)--(4.582,2.422)%
  --(4.585,2.424)--(4.588,2.425)--(4.591,2.427)--(4.594,2.428)--(4.597,2.430)--(4.600,2.432)%
  --(4.603,2.433)--(4.606,2.435)--(4.609,2.436)--(4.612,2.438)--(4.615,2.440)--(4.618,2.441)%
  --(4.621,2.443)--(4.624,2.445)--(4.627,2.446)--(4.630,2.448)--(4.633,2.449)--(4.636,2.451)%
  --(4.639,2.453)--(4.642,2.454)--(4.645,2.456)--(4.648,2.457)--(4.651,2.459)--(4.654,2.461)%
  --(4.657,2.462)--(4.660,2.464)--(4.663,2.465)--(4.666,2.467)--(4.669,2.469)--(4.672,2.470)%
  --(4.675,2.472)--(4.678,2.474)--(4.681,2.475)--(4.684,2.477)--(4.687,2.478)--(4.690,2.480)%
  --(4.693,2.482)--(4.696,2.483)--(4.699,2.485)--(4.702,2.487)--(4.705,2.488)--(4.708,2.490)%
  --(4.711,2.491)--(4.714,2.493)--(4.717,2.495)--(4.720,2.496)--(4.723,2.498)--(4.726,2.500)%
  --(4.729,2.501)--(4.732,2.503)--(4.735,2.505)--(4.738,2.506)--(4.741,2.508)--(4.744,2.509)%
  --(4.747,2.511)--(4.750,2.513)--(4.752,2.514)--(4.755,2.516)--(4.758,2.518)--(4.761,2.519)%
  --(4.764,2.521)--(4.767,2.523)--(4.770,2.524)--(4.773,2.526)--(4.776,2.528)--(4.779,2.529)%
  --(4.782,2.531)--(4.785,2.533)--(4.788,2.534)--(4.791,2.536)--(4.794,2.538)--(4.797,2.539)%
  --(4.800,2.541)--(4.803,2.542)--(4.806,2.544)--(4.809,2.546)--(4.812,2.547)--(4.815,2.549)%
  --(4.818,2.551)--(4.821,2.552)--(4.824,2.554)--(4.827,2.556)--(4.830,2.557)--(4.833,2.559)%
  --(4.836,2.561)--(4.839,2.562)--(4.842,2.564)--(4.845,2.566)--(4.848,2.567)--(4.851,2.569)%
  --(4.854,2.571)--(4.857,2.572)--(4.860,2.574)--(4.863,2.576)--(4.866,2.577)--(4.869,2.579)%
  --(4.872,2.581)--(4.875,2.582)--(4.878,2.584)--(4.881,2.586)--(4.884,2.588)--(4.887,2.589)%
  --(4.890,2.591)--(4.893,2.593)--(4.896,2.594)--(4.899,2.596)--(4.902,2.598)--(4.905,2.599)%
  --(4.908,2.601)--(4.911,2.603)--(4.914,2.604)--(4.917,2.606)--(4.920,2.608)--(4.923,2.609)%
  --(4.926,2.611)--(4.929,2.613)--(4.932,2.614)--(4.935,2.616)--(4.938,2.618)--(4.941,2.620)%
  --(4.944,2.621)--(4.947,2.623)--(4.950,2.625)--(4.953,2.626)--(4.956,2.628)--(4.959,2.630)%
  --(4.961,2.631)--(4.964,2.633)--(4.967,2.635)--(4.970,2.637)--(4.973,2.638)--(4.976,2.640)%
  --(4.979,2.642)--(4.982,2.643)--(4.985,2.645)--(4.988,2.647)--(4.991,2.648)--(4.994,2.650)%
  --(4.997,2.652)--(5.000,2.654)--(5.003,2.655)--(5.006,2.657)--(5.009,2.659)--(5.012,2.660)%
  --(5.015,2.662)--(5.018,2.664)--(5.021,2.666)--(5.024,2.667)--(5.027,2.669)--(5.030,2.671)%
  --(5.033,2.672)--(5.036,2.674)--(5.039,2.676)--(5.042,2.678)--(5.045,2.679)--(5.048,2.681)%
  --(5.051,2.683)--(5.054,2.684)--(5.057,2.686)--(5.060,2.688)--(5.063,2.690)--(5.066,2.691)%
  --(5.069,2.693)--(5.072,2.695)--(5.075,2.696)--(5.078,2.698)--(5.081,2.700)--(5.084,2.702)%
  --(5.087,2.703)--(5.090,2.705)--(5.093,2.707)--(5.096,2.709)--(5.099,2.710)--(5.102,2.712)%
  --(5.105,2.714)--(5.108,2.715)--(5.111,2.717)--(5.114,2.719)--(5.117,2.721)--(5.120,2.722)%
  --(5.123,2.724)--(5.126,2.726)--(5.129,2.728)--(5.132,2.729)--(5.135,2.731)--(5.138,2.733)%
  --(5.141,2.735)--(5.144,2.736)--(5.147,2.738)--(5.150,2.740)--(5.153,2.742)--(5.156,2.743)%
  --(5.159,2.745)--(5.162,2.747)--(5.165,2.748)--(5.168,2.750)--(5.171,2.752)--(5.173,2.754)%
  --(5.176,2.755)--(5.179,2.757)--(5.182,2.759)--(5.185,2.761)--(5.188,2.762)--(5.191,2.764)%
  --(5.194,2.766)--(5.197,2.768)--(5.200,2.769)--(5.203,2.771)--(5.206,2.773)--(5.209,2.775)%
  --(5.212,2.776)--(5.215,2.778)--(5.218,2.780)--(5.221,2.782)--(5.224,2.784)--(5.227,2.785)%
  --(5.230,2.787)--(5.233,2.789)--(5.236,2.791)--(5.239,2.792)--(5.242,2.794)--(5.245,2.796)%
  --(5.248,2.798)--(5.251,2.799)--(5.254,2.801)--(5.257,2.803)--(5.260,2.805)--(5.263,2.806)%
  --(5.266,2.808)--(5.269,2.810)--(5.272,2.812)--(5.275,2.813)--(5.278,2.815)--(5.281,2.817)%
  --(5.284,2.819)--(5.287,2.821)--(5.290,2.822)--(5.293,2.824)--(5.296,2.826)--(5.299,2.828)%
  --(5.302,2.829)--(5.305,2.831)--(5.308,2.833)--(5.311,2.835)--(5.314,2.836)--(5.317,2.838)%
  --(5.320,2.840)--(5.323,2.842)--(5.326,2.844)--(5.329,2.845)--(5.332,2.847)--(5.335,2.849)%
  --(5.338,2.851)--(5.341,2.852)--(5.344,2.854)--(5.347,2.856)--(5.350,2.858)--(5.353,2.860)%
  --(5.356,2.861)--(5.359,2.863)--(5.362,2.865)--(5.365,2.867)--(5.368,2.868)--(5.371,2.870)%
  --(5.374,2.872)--(5.377,2.874)--(5.380,2.876)--(5.382,2.877)--(5.385,2.879)--(5.388,2.881)%
  --(5.391,2.883)--(5.394,2.885)--(5.397,2.886)--(5.400,2.888)--(5.403,2.890)--(5.406,2.892)%
  --(5.409,2.894)--(5.412,2.895)--(5.415,2.897)--(5.418,2.899)--(5.421,2.901)--(5.424,2.902)%
  --(5.427,2.904)--(5.430,2.906)--(5.433,2.908)--(5.436,2.910)--(5.439,2.911)--(5.442,2.913)%
  --(5.445,2.915)--(5.448,2.917)--(5.451,2.919)--(5.454,2.920)--(5.457,2.922)--(5.460,2.924)%
  --(5.463,2.926)--(5.466,2.928)--(5.469,2.929)--(5.472,2.931)--(5.475,2.933)--(5.478,2.935)%
  --(5.481,2.937)--(5.484,2.938)--(5.487,2.940)--(5.490,2.942)--(5.493,2.944)--(5.496,2.946)%
  --(5.499,2.948)--(5.502,2.949)--(5.505,2.951)--(5.508,2.953)--(5.511,2.955)--(5.514,2.957)%
  --(5.517,2.958)--(5.520,2.960)--(5.523,2.962)--(5.526,2.964)--(5.529,2.966)--(5.532,2.967)%
  --(5.535,2.969)--(5.538,2.971)--(5.541,2.973)--(5.544,2.975)--(5.547,2.977)--(5.550,2.978)%
  --(5.553,2.980)--(5.556,2.982)--(5.559,2.984)--(5.562,2.986)--(5.565,2.987)--(5.568,2.989)%
  --(5.571,2.991)--(5.574,2.993)--(5.577,2.995)--(5.580,2.997)--(5.583,2.998)--(5.586,3.000)%
  --(5.589,3.002)--(5.591,3.004)--(5.594,3.006)--(5.597,3.007)--(5.600,3.009)--(5.603,3.011)%
  --(5.606,3.013)--(5.609,3.015)--(5.612,3.017)--(5.615,3.018)--(5.618,3.020)--(5.621,3.022)%
  --(5.624,3.024)--(5.627,3.026)--(5.630,3.028)--(5.633,3.029)--(5.636,3.031)--(5.639,3.033)%
  --(5.642,3.035)--(5.645,3.037)--(5.648,3.039)--(5.651,3.040)--(5.654,3.042)--(5.657,3.044)%
  --(5.660,3.046)--(5.663,3.048)--(5.666,3.050)--(5.669,3.051)--(5.672,3.053)--(5.675,3.055)%
  --(5.678,3.057)--(5.681,3.059)--(5.684,3.061)--(5.687,3.062)--(5.690,3.064)--(5.693,3.066)%
  --(5.696,3.068)--(5.699,3.070)--(5.702,3.072)--(5.705,3.073)--(5.708,3.075)--(5.711,3.077)%
  --(5.714,3.079)--(5.717,3.081)--(5.720,3.083)--(5.723,3.084)--(5.726,3.086)--(5.729,3.088)%
  --(5.732,3.090)--(5.735,3.092)--(5.738,3.094)--(5.741,3.096)--(5.744,3.097)--(5.747,3.099)%
  --(5.750,3.101)--(5.753,3.103)--(5.756,3.105)--(5.759,3.107)--(5.762,3.108)--(5.765,3.110)%
  --(5.768,3.112)--(5.771,3.114)--(5.774,3.116)--(5.777,3.118)--(5.780,3.120)--(5.783,3.121)%
  --(5.786,3.123)--(5.789,3.125)--(5.792,3.127)--(5.795,3.129)--(5.798,3.131)--(5.800,3.133)%
  --(5.803,3.134)--(5.806,3.136)--(5.809,3.138)--(5.812,3.140)--(5.815,3.142)--(5.818,3.144)%
  --(5.821,3.146)--(5.824,3.147)--(5.827,3.149)--(5.830,3.151)--(5.833,3.153)--(5.836,3.155)%
  --(5.839,3.157)--(5.842,3.159)--(5.845,3.160)--(5.848,3.162)--(5.851,3.164)--(5.854,3.166)%
  --(5.857,3.168)--(5.860,3.170)--(5.863,3.172)--(5.866,3.173)--(5.869,3.175)--(5.872,3.177)%
  --(5.875,3.179)--(5.878,3.181)--(5.881,3.183)--(5.884,3.185)--(5.887,3.186)--(5.890,3.188)%
  --(5.893,3.190)--(5.896,3.192)--(5.899,3.194)--(5.902,3.196)--(5.905,3.198)--(5.908,3.200)%
  --(5.911,3.201)--(5.914,3.203)--(5.917,3.205)--(5.920,3.207)--(5.923,3.209)--(5.926,3.211)%
  --(5.929,3.213)--(5.932,3.214)--(5.935,3.216)--(5.938,3.218)--(5.941,3.220)--(5.944,3.222)%
  --(5.947,3.224)--(5.950,3.226)--(5.953,3.228)--(5.956,3.229)--(5.959,3.231)--(5.962,3.233)%
  --(5.965,3.235)--(5.968,3.237)--(5.971,3.239)--(5.974,3.241)--(5.977,3.243)--(5.980,3.244)%
  --(5.983,3.246)--(5.986,3.248)--(5.989,3.250)--(5.992,3.252)--(5.995,3.254)--(5.998,3.256)%
  --(6.001,3.258)--(6.004,3.260)--(6.007,3.261)--(6.009,3.263)--(6.012,3.265)--(6.015,3.267)%
  --(6.018,3.269)--(6.021,3.271)--(6.024,3.273)--(6.027,3.275)--(6.030,3.276)--(6.033,3.278)%
  --(6.036,3.280)--(6.039,3.282)--(6.042,3.284)--(6.045,3.286)--(6.048,3.288)--(6.051,3.290)%
  --(6.054,3.292)--(6.057,3.293)--(6.060,3.295)--(6.063,3.297)--(6.066,3.299)--(6.069,3.301)%
  --(6.072,3.303)--(6.075,3.305)--(6.078,3.307)--(6.081,3.309)--(6.084,3.310)--(6.087,3.312)%
  --(6.090,3.314)--(6.093,3.316)--(6.096,3.318)--(6.099,3.320)--(6.102,3.322)--(6.105,3.324)%
  --(6.108,3.326)--(6.111,3.327)--(6.114,3.329)--(6.117,3.331)--(6.120,3.333)--(6.123,3.335)%
  --(6.126,3.337)--(6.129,3.339)--(6.132,3.341)--(6.135,3.343)--(6.138,3.345)--(6.141,3.346)%
  --(6.144,3.348)--(6.147,3.350)--(6.150,3.352)--(6.153,3.354)--(6.156,3.356)--(6.159,3.358)%
  --(6.162,3.360)--(6.165,3.362)--(6.168,3.364)--(6.171,3.365)--(6.174,3.367)--(6.177,3.369)%
  --(6.180,3.371)--(6.183,3.373)--(6.186,3.375)--(6.189,3.377)--(6.192,3.379)--(6.195,3.381)%
  --(6.198,3.383)--(6.201,3.384)--(6.204,3.386)--(6.207,3.388)--(6.210,3.390)--(6.213,3.392)%
  --(6.216,3.394)--(6.218,3.396)--(6.221,3.398)--(6.224,3.400)--(6.227,3.402)--(6.230,3.403)%
  --(6.233,3.405)--(6.236,3.407)--(6.239,3.409)--(6.242,3.411)--(6.245,3.413)--(6.248,3.415)%
  --(6.251,3.417)--(6.254,3.419)--(6.257,3.421)--(6.260,3.423)--(6.263,3.424)--(6.266,3.426)%
  --(6.269,3.428)--(6.272,3.430)--(6.275,3.432)--(6.278,3.434)--(6.281,3.436)--(6.284,3.438)%
  --(6.287,3.440)--(6.290,3.442)--(6.293,3.444)--(6.296,3.446)--(6.299,3.447)--(6.302,3.449)%
  --(6.305,3.451)--(6.308,3.453)--(6.311,3.455)--(6.314,3.457)--(6.317,3.459)--(6.320,3.461)%
  --(6.323,3.463)--(6.326,3.465)--(6.329,3.467)--(6.332,3.469)--(6.335,3.470)--(6.338,3.472)%
  --(6.341,3.474)--(6.344,3.476)--(6.347,3.478)--(6.350,3.480)--(6.353,3.482)--(6.356,3.484)%
  --(6.359,3.486)--(6.362,3.488)--(6.365,3.490)--(6.368,3.492)--(6.371,3.493)--(6.374,3.495)%
  --(6.377,3.497)--(6.380,3.499)--(6.383,3.501)--(6.386,3.503)--(6.389,3.505)--(6.392,3.507)%
  --(6.395,3.509)--(6.398,3.511)--(6.401,3.513)--(6.404,3.515)--(6.407,3.517)--(6.410,3.518)%
  --(6.413,3.520)--(6.416,3.522)--(6.419,3.524)--(6.422,3.526)--(6.425,3.528)--(6.428,3.530)%
  --(6.430,3.532)--(6.433,3.534)--(6.436,3.536)--(6.439,3.538)--(6.442,3.540)--(6.445,3.542)%
  --(6.448,3.544)--(6.451,3.545)--(6.454,3.547)--(6.457,3.549)--(6.460,3.551)--(6.463,3.553)%
  --(6.466,3.555)--(6.469,3.557)--(6.472,3.559)--(6.475,3.561)--(6.478,3.563)--(6.481,3.565)%
  --(6.484,3.567)--(6.487,3.569)--(6.490,3.571)--(6.493,3.572)--(6.496,3.574)--(6.499,3.576)%
  --(6.502,3.578)--(6.505,3.580)--(6.508,3.582)--(6.511,3.584)--(6.514,3.586)--(6.517,3.588)%
  --(6.520,3.590)--(6.523,3.592)--(6.526,3.594)--(6.529,3.596)--(6.532,3.598)--(6.535,3.600)%
  --(6.538,3.602)--(6.541,3.603)--(6.544,3.605)--(6.547,3.607)--(6.550,3.609)--(6.553,3.611)%
  --(6.556,3.613)--(6.559,3.615)--(6.562,3.617)--(6.565,3.619)--(6.568,3.621)--(6.571,3.623)%
  --(6.574,3.625)--(6.577,3.627)--(6.580,3.629)--(6.583,3.631)--(6.586,3.633)--(6.589,3.634)%
  --(6.592,3.636)--(6.595,3.638)--(6.598,3.640)--(6.601,3.642)--(6.604,3.644)--(6.607,3.646)%
  --(6.610,3.648)--(6.613,3.650)--(6.616,3.652)--(6.619,3.654)--(6.622,3.656)--(6.625,3.658)%
  --(6.628,3.660)--(6.631,3.662)--(6.634,3.664)--(6.637,3.666)--(6.639,3.668)--(6.642,3.669)%
  --(6.645,3.671)--(6.648,3.673)--(6.651,3.675)--(6.654,3.677)--(6.657,3.679)--(6.660,3.681)%
  --(6.663,3.683)--(6.666,3.685)--(6.669,3.687)--(6.672,3.689)--(6.675,3.691)--(6.678,3.693)%
  --(6.681,3.695)--(6.684,3.697)--(6.687,3.699)--(6.690,3.701)--(6.693,3.703)--(6.696,3.705)%
  --(6.699,3.706)--(6.702,3.708)--(6.705,3.710)--(6.708,3.712)--(6.711,3.714)--(6.714,3.716)%
  --(6.717,3.718)--(6.720,3.720)--(6.723,3.722)--(6.726,3.724)--(6.729,3.726)--(6.732,3.728)%
  --(6.735,3.730)--(6.738,3.732)--(6.741,3.734)--(6.744,3.736)--(6.747,3.738)--(6.750,3.740)%
  --(6.753,3.742)--(6.756,3.744)--(6.759,3.746)--(6.762,3.748)--(6.765,3.749)--(6.768,3.751)%
  --(6.771,3.753)--(6.774,3.755)--(6.777,3.757)--(6.780,3.759)--(6.783,3.761)--(6.786,3.763)%
  --(6.789,3.765)--(6.792,3.767)--(6.795,3.769)--(6.798,3.771)--(6.801,3.773)--(6.804,3.775)%
  --(6.807,3.777)--(6.810,3.779)--(6.813,3.781)--(6.816,3.783)--(6.819,3.785)--(6.822,3.787)%
  --(6.825,3.789)--(6.828,3.791)--(6.831,3.793)--(6.834,3.795)--(6.837,3.796)--(6.840,3.798)%
  --(6.843,3.800)--(6.846,3.802)--(6.848,3.804)--(6.851,3.806)--(6.854,3.808)--(6.857,3.810)%
  --(6.860,3.812)--(6.863,3.814)--(6.866,3.816)--(6.869,3.818)--(6.872,3.820)--(6.875,3.822)%
  --(6.878,3.824)--(6.881,3.826)--(6.884,3.828)--(6.887,3.830)--(6.890,3.832)--(6.893,3.834)%
  --(6.896,3.836)--(6.899,3.838)--(6.902,3.840)--(6.905,3.842)--(6.908,3.844)--(6.911,3.846)%
  --(6.914,3.848)--(6.917,3.850)--(6.920,3.851)--(6.923,3.853)--(6.926,3.855)--(6.929,3.857)%
  --(6.932,3.859)--(6.935,3.861)--(6.938,3.863)--(6.941,3.865)--(6.944,3.867)--(6.947,3.869)%
  --(6.950,3.871)--(6.953,3.873)--(6.956,3.875)--(6.959,3.877)--(6.962,3.879)--(6.965,3.881)%
  --(6.968,3.883)--(6.971,3.885)--(6.974,3.887)--(6.977,3.889)--(6.980,3.891)--(6.983,3.893)%
  --(6.986,3.895)--(6.989,3.897)--(6.992,3.899)--(6.995,3.901)--(6.998,3.903)--(7.001,3.905)%
  --(7.004,3.907)--(7.007,3.909)--(7.010,3.911)--(7.013,3.913)--(7.016,3.915)--(7.019,3.917)%
  --(7.022,3.918)--(7.025,3.920)--(7.028,3.922)--(7.031,3.924)--(7.034,3.926)--(7.037,3.928)%
  --(7.040,3.930)--(7.043,3.932)--(7.046,3.934)--(7.049,3.936)--(7.052,3.938)--(7.055,3.940)%
  --(7.057,3.942)--(7.060,3.944)--(7.063,3.946)--(7.066,3.948)--(7.069,3.950)--(7.072,3.952)%
  --(7.075,3.954)--(7.078,3.956)--(7.081,3.958)--(7.084,3.960)--(7.087,3.962)--(7.090,3.964)%
  --(7.093,3.966)--(7.096,3.968)--(7.099,3.970)--(7.102,3.972)--(7.105,3.974)--(7.108,3.976)%
  --(7.111,3.978)--(7.114,3.980)--(7.117,3.982)--(7.120,3.984)--(7.123,3.986)--(7.126,3.988)%
  --(7.129,3.990)--(7.132,3.992)--(7.135,3.994)--(7.138,3.996)--(7.141,3.998)--(7.144,4.000)%
  --(7.147,4.002)--(7.150,4.004)--(7.153,4.006)--(7.156,4.008)--(7.159,4.010)--(7.162,4.012)%
  --(7.165,4.013)--(7.168,4.015)--(7.171,4.017)--(7.174,4.019)--(7.177,4.021)--(7.180,4.023)%
  --(7.183,4.025)--(7.186,4.027)--(7.189,4.029)--(7.192,4.031)--(7.195,4.033)--(7.198,4.035)%
  --(7.201,4.037)--(7.204,4.039)--(7.207,4.041)--(7.210,4.043)--(7.213,4.045)--(7.216,4.047)%
  --(7.219,4.049)--(7.222,4.051)--(7.225,4.053)--(7.228,4.055)--(7.231,4.057)--(7.234,4.059)%
  --(7.237,4.061)--(7.240,4.063)--(7.243,4.065)--(7.246,4.067)--(7.249,4.069)--(7.252,4.071)%
  --(7.255,4.073)--(7.258,4.075)--(7.261,4.077)--(7.264,4.079)--(7.266,4.081)--(7.269,4.083)%
  --(7.272,4.085)--(7.275,4.087)--(7.278,4.089)--(7.281,4.091)--(7.284,4.093)--(7.287,4.095)%
  --(7.290,4.097)--(7.293,4.099)--(7.296,4.101)--(7.299,4.103)--(7.302,4.105)--(7.305,4.107)%
  --(7.308,4.109)--(7.311,4.111)--(7.314,4.113)--(7.317,4.115)--(7.320,4.117)--(7.323,4.119)%
  --(7.326,4.121)--(7.329,4.123)--(7.332,4.125)--(7.335,4.127)--(7.338,4.129)--(7.341,4.131)%
  --(7.344,4.133)--(7.347,4.135)--(7.350,4.137)--(7.353,4.139)--(7.356,4.141)--(7.359,4.143)%
  --(7.362,4.145)--(7.365,4.147)--(7.368,4.149)--(7.371,4.151)--(7.374,4.153)--(7.377,4.155)%
  --(7.380,4.157)--(7.383,4.159)--(7.386,4.161)--(7.389,4.163)--(7.392,4.165)--(7.395,4.167)%
  --(7.398,4.169)--(7.401,4.171)--(7.404,4.173)--(7.407,4.175)--(7.410,4.177)--(7.413,4.179)%
  --(7.416,4.181)--(7.419,4.183)--(7.422,4.185)--(7.425,4.187)--(7.428,4.189)--(7.431,4.191)%
  --(7.434,4.193)--(7.437,4.195)--(7.440,4.197)--(7.443,4.199)--(7.446,4.201)--(7.449,4.203)%
  --(7.452,4.205)--(7.455,4.207)--(7.458,4.209)--(7.461,4.211)--(7.464,4.213)--(7.467,4.215)%
  --(7.470,4.217)--(7.473,4.219)--(7.476,4.221)--(7.478,4.223)--(7.481,4.225)--(7.484,4.227)%
  --(7.487,4.229)--(7.490,4.231)--(7.493,4.233)--(7.496,4.235)--(7.499,4.237)--(7.502,4.239)%
  --(7.505,4.241)--(7.508,4.243)--(7.511,4.245)--(7.514,4.247)--(7.517,4.249)--(7.520,4.251)%
  --(7.523,4.253)--(7.526,4.255)--(7.529,4.257)--(7.532,4.259)--(7.535,4.261)--(7.538,4.263)%
  --(7.541,4.265)--(7.544,4.267)--(7.547,4.269)--(7.550,4.271)--(7.553,4.273)--(7.556,4.275)%
  --(7.559,4.277)--(7.562,4.279)--(7.565,4.281)--(7.568,4.283)--(7.571,4.285)--(7.574,4.287)%
  --(7.577,4.289)--(7.580,4.291)--(7.583,4.293)--(7.586,4.295)--(7.589,4.297)--(7.592,4.299)%
  --(7.595,4.301)--(7.598,4.303)--(7.601,4.305)--(7.604,4.307)--(7.607,4.309)--(7.610,4.311)%
  --(7.613,4.313)--(7.616,4.315)--(7.619,4.317)--(7.622,4.319)--(7.625,4.321)--(7.628,4.323)%
  --(7.631,4.325)--(7.634,4.327)--(7.637,4.329)--(7.640,4.331)--(7.643,4.333)--(7.646,4.335)%
  --(7.649,4.337)--(7.652,4.339)--(7.655,4.341)--(7.658,4.343)--(7.661,4.345)--(7.664,4.347)%
  --(7.667,4.349)--(7.670,4.351)--(7.673,4.353)--(7.676,4.355)--(7.679,4.357)--(7.682,4.359)%
  --(7.685,4.361)--(7.687,4.363)--(7.690,4.365)--(7.693,4.367)--(7.696,4.369)--(7.699,4.371)%
  --(7.702,4.373)--(7.705,4.375)--(7.708,4.377)--(7.711,4.379)--(7.714,4.381)--(7.717,4.383)%
  --(7.720,4.385)--(7.723,4.387)--(7.726,4.389)--(7.729,4.391)--(7.732,4.393)--(7.735,4.395)%
  --(7.738,4.397)--(7.741,4.399)--(7.744,4.401)--(7.747,4.403)--(7.750,4.406)--(7.753,4.408)%
  --(7.756,4.410)--(7.759,4.412)--(7.762,4.414)--(7.765,4.416)--(7.768,4.418)--(7.771,4.420)%
  --(7.774,4.422)--(7.777,4.424)--(7.780,4.426)--(7.783,4.428)--(7.786,4.430)--(7.789,4.432)%
  --(7.792,4.434)--(7.795,4.436)--(7.798,4.438)--(7.801,4.440)--(7.804,4.442)--(7.807,4.444)%
  --(7.810,4.446)--(7.813,4.448)--(7.816,4.450)--(7.819,4.452)--(7.822,4.454)--(7.825,4.456)%
  --(7.828,4.458)--(7.831,4.460)--(7.834,4.462)--(7.837,4.464)--(7.840,4.466)--(7.843,4.468)%
  --(7.846,4.470)--(7.849,4.472)--(7.852,4.474)--(7.855,4.476)--(7.858,4.478)--(7.861,4.480)%
  --(7.864,4.482)--(7.867,4.484)--(7.870,4.486)--(7.873,4.488)--(7.876,4.490)--(7.879,4.492)%
  --(7.882,4.494)--(7.885,4.496)--(7.888,4.498)--(7.891,4.500)--(7.894,4.502)--(7.896,4.504)%
  --(7.899,4.506)--(7.902,4.508)--(7.905,4.510)--(7.908,4.512)--(7.911,4.514)--(7.914,4.517)%
  --(7.917,4.519)--(7.920,4.521)--(7.923,4.523)--(7.926,4.525)--(7.929,4.527)--(7.932,4.529)%
  --(7.935,4.531)--(7.938,4.533)--(7.941,4.535)--(7.944,4.537)--(7.947,4.539)--(7.950,4.541)%
  --(7.953,4.543)--(7.956,4.545)--(7.959,4.547)--(7.962,4.549)--(7.965,4.551)--(7.968,4.553)%
  --(7.971,4.555)--(7.974,4.557)--(7.977,4.559)--(7.980,4.561)--(7.983,4.563)--(7.986,4.565)%
  --(7.989,4.567)--(7.992,4.569)--(7.995,4.571)--(7.998,4.573)--(8.001,4.575)--(8.004,4.577)%
  --(8.007,4.579)--(8.010,4.581)--(8.013,4.583)--(8.016,4.585)--(8.019,4.587)--(8.022,4.589)%
  --(8.025,4.591)--(8.028,4.593)--(8.031,4.595)--(8.034,4.597)--(8.037,4.600)--(8.040,4.602)%
  --(8.043,4.604)--(8.046,4.606)--(8.049,4.608)--(8.052,4.610)--(8.055,4.612)--(8.058,4.614)%
  --(8.061,4.616)--(8.064,4.618)--(8.067,4.620)--(8.070,4.622)--(8.073,4.624)--(8.076,4.626)%
  --(8.079,4.628)--(8.082,4.630)--(8.085,4.632)--(8.088,4.634)--(8.091,4.636)--(8.094,4.638)%
  --(8.097,4.640)--(8.100,4.642)--(8.103,4.644)--(8.105,4.646)--(8.108,4.648)--(8.111,4.650)%
  --(8.114,4.652)--(8.117,4.654)--(8.120,4.656)--(8.123,4.658)--(8.126,4.660)--(8.129,4.662)%
  --(8.132,4.664)--(8.135,4.666)--(8.138,4.668)--(8.141,4.671)--(8.144,4.673)--(8.147,4.675)%
  --(8.150,4.677)--(8.153,4.679)--(8.156,4.681)--(8.159,4.683)--(8.162,4.685)--(8.165,4.687)%
  --(8.168,4.689)--(8.171,4.691)--(8.174,4.693)--(8.177,4.695)--(8.180,4.697)--(8.183,4.699)%
  --(8.186,4.701)--(8.189,4.703)--(8.192,4.705)--(8.195,4.707)--(8.198,4.709)--(8.201,4.711)%
  --(8.204,4.713)--(8.207,4.715)--(8.210,4.717)--(8.213,4.719)--(8.216,4.721)--(8.219,4.723)%
  --(8.222,4.725)--(8.225,4.727)--(8.228,4.729)--(8.231,4.731)--(8.234,4.734)--(8.237,4.736)%
  --(8.240,4.738)--(8.243,4.740)--(8.246,4.742)--(8.249,4.744)--(8.252,4.746)--(8.255,4.748)%
  --(8.258,4.750)--(8.261,4.752)--(8.264,4.754)--(8.267,4.756)--(8.270,4.758)--(8.273,4.760)%
  --(8.276,4.762)--(8.279,4.764)--(8.282,4.766)--(8.285,4.768)--(8.288,4.770)--(8.291,4.772)%
  --(8.294,4.774)--(8.297,4.776)--(8.300,4.778)--(8.303,4.780)--(8.306,4.782)--(8.309,4.784)%
  --(8.312,4.786)--(8.314,4.788)--(8.317,4.791)--(8.320,4.793)--(8.323,4.795)--(8.326,4.797)%
  --(8.329,4.799)--(8.332,4.801)--(8.335,4.803)--(8.338,4.805)--(8.341,4.807)--(8.344,4.809)%
  --(8.347,4.811)--(8.350,4.813)--(8.353,4.815)--(8.356,4.817)--(8.359,4.819)--(8.362,4.821)%
  --(8.365,4.823)--(8.368,4.825)--(8.371,4.827)--(8.374,4.829)--(8.377,4.831)--(8.380,4.833)%
  --(8.383,4.835)--(8.386,4.837)--(8.389,4.839)--(8.392,4.841)--(8.395,4.843)--(8.398,4.846)%
  --(8.401,4.848)--(8.404,4.850)--(8.407,4.852)--(8.410,4.854)--(8.413,4.856)--(8.416,4.858)%
  --(8.419,4.860)--(8.422,4.862)--(8.425,4.864)--(8.428,4.866)--(8.431,4.868)--(8.434,4.870)%
  --(8.437,4.872)--(8.440,4.874)--(8.443,4.876)--(8.446,4.878)--(8.449,4.880)--(8.452,4.882)%
  --(8.455,4.884)--(8.458,4.886)--(8.461,4.888)--(8.464,4.890)--(8.467,4.892)--(8.470,4.895)%
  --(8.473,4.897)--(8.476,4.899)--(8.479,4.901)--(8.482,4.903)--(8.485,4.905)--(8.488,4.907)%
  --(8.491,4.909)--(8.494,4.911)--(8.497,4.913)--(8.500,4.915)--(8.503,4.917)--(8.506,4.919)%
  --(8.509,4.921)--(8.512,4.923)--(8.515,4.925)--(8.518,4.927)--(8.521,4.929)--(8.523,4.931)%
  --(8.526,4.933)--(8.529,4.935)--(8.532,4.937)--(8.535,4.939)--(8.538,4.942)--(8.541,4.944)%
  --(8.544,4.946)--(8.547,4.948)--(8.550,4.950)--(8.553,4.952)--(8.556,4.954)--(8.559,4.956)%
  --(8.562,4.958)--(8.565,4.960)--(8.568,4.962)--(8.571,4.964)--(8.574,4.966)--(8.577,4.968)%
  --(8.580,4.970)--(8.583,4.972)--(8.586,4.974)--(8.589,4.976)--(8.592,4.978)--(8.595,4.980)%
  --(8.598,4.982)--(8.601,4.984)--(8.604,4.986)--(8.607,4.989)--(8.610,4.991)--(8.613,4.993)%
  --(8.616,4.995)--(8.619,4.997)--(8.622,4.999)--(8.625,5.001)--(8.628,5.003)--(8.631,5.005)%
  --(8.634,5.007)--(8.637,5.009)--(8.640,5.011)--(8.643,5.013)--(8.646,5.015)--(8.649,5.017)%
  --(8.652,5.019)--(8.655,5.021)--(8.658,5.023)--(8.661,5.025)--(8.664,5.027)--(8.667,5.029)%
  --(8.670,5.032)--(8.673,5.034)--(8.676,5.036)--(8.679,5.038)--(8.682,5.040)--(8.685,5.042)%
  --(8.688,5.044)--(8.691,5.046)--(8.694,5.048)--(8.697,5.050)--(8.700,5.052)--(8.703,5.054)%
  --(8.706,5.056)--(8.709,5.058)--(8.712,5.060)--(8.715,5.062)--(8.718,5.064)--(8.721,5.066)%
  --(8.724,5.068)--(8.727,5.070)--(8.730,5.072)--(8.733,5.075)--(8.735,5.077)--(8.738,5.079)%
  --(8.741,5.081)--(8.744,5.083)--(8.747,5.085)--(8.750,5.087)--(8.753,5.089)--(8.756,5.091)%
  --(8.759,5.093)--(8.762,5.095)--(8.765,5.097)--(8.768,5.099)--(8.771,5.101)--(8.774,5.103)%
  --(8.777,5.105)--(8.780,5.107)--(8.783,5.109)--(8.786,5.111)--(8.789,5.113)--(8.792,5.116)%
  --(8.795,5.118)--(8.798,5.120)--(8.801,5.122)--(8.804,5.124)--(8.807,5.126)--(8.810,5.128)%
  --(8.813,5.130)--(8.816,5.132)--(8.819,5.134)--(8.822,5.136)--(8.825,5.138)--(8.828,5.140)%
  --(8.831,5.142)--(8.834,5.144)--(8.837,5.146)--(8.840,5.148)--(8.843,5.150)--(8.846,5.152)%
  --(8.849,5.155)--(8.852,5.157)--(8.855,5.159)--(8.858,5.161)--(8.861,5.163)--(8.864,5.165)%
  --(8.867,5.167)--(8.870,5.169)--(8.873,5.171)--(8.876,5.173)--(8.879,5.175)--(8.882,5.177)%
  --(8.885,5.179)--(8.888,5.181)--(8.891,5.183)--(8.894,5.185)--(8.897,5.187)--(8.900,5.189)%
  --(8.903,5.191)--(8.906,5.194)--(8.909,5.196)--(8.912,5.198)--(8.915,5.200)--(8.918,5.202)%
  --(8.921,5.204)--(8.924,5.206)--(8.927,5.208)--(8.930,5.210)--(8.933,5.212)--(8.936,5.214)%
  --(8.939,5.216)--(8.942,5.218)--(8.944,5.220)--(8.947,5.222)--(8.950,5.224)--(8.953,5.226)%
  --(8.956,5.228)--(8.959,5.231)--(8.962,5.233)--(8.965,5.235)--(8.968,5.237)--(8.971,5.239)%
  --(8.974,5.241)--(8.977,5.243)--(8.980,5.245)--(8.983,5.247)--(8.986,5.249)--(8.989,5.251)%
  --(8.992,5.253)--(8.995,5.255)--(8.998,5.257)--(9.001,5.259)--(9.004,5.261)--(9.007,5.263)%
  --(9.010,5.265)--(9.013,5.268)--(9.016,5.270)--(9.019,5.272)--(9.022,5.274)--(9.025,5.276)%
  --(9.028,5.278)--(9.031,5.280)--(9.034,5.282)--(9.037,5.284)--(9.040,5.286)--(9.043,5.288)%
  --(9.046,5.290)--(9.049,5.292)--(9.052,5.294)--(9.055,5.296)--(9.058,5.298)--(9.061,5.300)%
  --(9.064,5.303)--(9.067,5.305)--(9.070,5.307)--(9.073,5.309)--(9.076,5.311)--(9.079,5.313)%
  --(9.082,5.315)--(9.085,5.317)--(9.088,5.319)--(9.091,5.321)--(9.094,5.323)--(9.097,5.325)%
  --(9.100,5.327)--(9.103,5.329)--(9.106,5.331)--(9.109,5.333)--(9.112,5.335)--(9.115,5.338)%
  --(9.118,5.340)--(9.121,5.342)--(9.124,5.344)--(9.127,5.346)--(9.130,5.348)--(9.133,5.350)%
  --(9.136,5.352)--(9.139,5.354)--(9.142,5.356)--(9.145,5.358)--(9.148,5.360)--(9.151,5.362)%
  --(9.153,5.364)--(9.156,5.366)--(9.159,5.368)--(9.162,5.370)--(9.165,5.373)--(9.168,5.375)%
  --(9.171,5.377)--(9.174,5.379)--(9.177,5.381)--(9.180,5.383)--(9.183,5.385)--(9.186,5.387)%
  --(9.189,5.389)--(9.192,5.391)--(9.195,5.393)--(9.198,5.395)--(9.201,5.397)--(9.204,5.399)%
  --(9.207,5.401)--(9.210,5.403)--(9.213,5.405)--(9.216,5.408)--(9.219,5.410)--(9.222,5.412)%
  --(9.225,5.414)--(9.228,5.416)--(9.231,5.418)--(9.234,5.420)--(9.237,5.422)--(9.240,5.424)%
  --(9.243,5.426)--(9.246,5.428)--(9.249,5.430)--(9.252,5.432)--(9.255,5.434)--(9.258,5.436)%
  --(9.261,5.438)--(9.264,5.441)--(9.267,5.443)--(9.270,5.445)--(9.273,5.447)--(9.276,5.449)%
  --(9.279,5.451)--(9.282,5.453)--(9.285,5.455)--(9.288,5.457)--(9.291,5.459)--(9.294,5.461)%
  --(9.297,5.463)--(9.300,5.465)--(9.303,5.467)--(9.306,5.469)--(9.309,5.471)--(9.312,5.474)%
  --(9.315,5.476)--(9.318,5.478)--(9.321,5.480)--(9.324,5.482)--(9.327,5.484)--(9.330,5.486)%
  --(9.333,5.488)--(9.336,5.490)--(9.339,5.492)--(9.342,5.494)--(9.345,5.496)--(9.348,5.498)%
  --(9.351,5.500)--(9.354,5.502)--(9.357,5.505)--(9.360,5.507)--(9.362,5.509)--(9.365,5.511)%
  --(9.368,5.513)--(9.371,5.515)--(9.374,5.517)--(9.377,5.519)--(9.380,5.521)--(9.383,5.523)%
  --(9.386,5.525)--(9.389,5.527)--(9.392,5.529)--(9.395,5.531)--(9.398,5.533)--(9.401,5.535)%
  --(9.404,5.538)--(9.407,5.540)--(9.410,5.542)--(9.413,5.544)--(9.416,5.546)--(9.419,5.548)%
  --(9.422,5.550)--(9.425,5.552)--(9.428,5.554)--(9.431,5.556)--(9.434,5.558)--(9.437,5.560)%
  --(9.440,5.562)--(9.443,5.564)--(9.446,5.566)--(9.449,5.569)--(9.452,5.571)--(9.455,5.573)%
  --(9.458,5.575)--(9.461,5.577)--(9.464,5.579)--(9.467,5.581)--(9.470,5.583)--(9.473,5.585)%
  --(9.476,5.587)--(9.479,5.589)--(9.482,5.591)--(9.485,5.593)--(9.488,5.595)--(9.491,5.597)%
  --(9.494,5.600)--(9.497,5.602)--(9.500,5.604)--(9.503,5.606)--(9.506,5.608)--(9.509,5.610)%
  --(9.512,5.612)--(9.515,5.614)--(9.518,5.616)--(9.521,5.618)--(9.524,5.620)--(9.527,5.622)%
  --(9.530,5.624)--(9.533,5.626)--(9.536,5.628)--(9.539,5.631)--(9.542,5.633)--(9.545,5.635)%
  --(9.548,5.637)--(9.551,5.639)--(9.554,5.641)--(9.557,5.643)--(9.560,5.645)--(9.563,5.647)%
  --(9.566,5.649)--(9.569,5.651)--(9.571,5.653)--(9.574,5.655)--(9.577,5.657)--(9.580,5.660)%
  --(9.583,5.662)--(9.586,5.664)--(9.589,5.666)--(9.592,5.668)--(9.595,5.670)--(9.598,5.672)%
  --(9.601,5.674)--(9.604,5.676)--(9.607,5.678)--(9.610,5.680)--(9.613,5.682)--(9.616,5.684)%
  --(9.619,5.686)--(9.622,5.688)--(9.625,5.691)--(9.628,5.693)--(9.631,5.695)--(9.634,5.697)%
  --(9.637,5.699)--(9.640,5.701)--(9.643,5.703)--(9.646,5.705)--(9.649,5.707)--(9.652,5.709)%
  --(9.655,5.711)--(9.658,5.713)--(9.661,5.715)--(9.664,5.717)--(9.667,5.720)--(9.670,5.722)%
  --(9.673,5.724)--(9.676,5.726)--(9.679,5.728)--(9.682,5.730)--(9.685,5.732)--(9.688,5.734)%
  --(9.691,5.736)--(9.694,5.738)--(9.697,5.740)--(9.700,5.742)--(9.703,5.744)--(9.706,5.746)%
  --(9.709,5.749)--(9.712,5.751)--(9.715,5.753)--(9.718,5.755)--(9.721,5.757)--(9.724,5.759)%
  --(9.727,5.761)--(9.730,5.763)--(9.733,5.765)--(9.736,5.767)--(9.739,5.769)--(9.742,5.771)%
  --(9.745,5.773)--(9.748,5.775)--(9.751,5.778)--(9.754,5.780)--(9.757,5.782)--(9.760,5.784)%
  --(9.763,5.786)--(9.766,5.788)--(9.769,5.790)--(9.772,5.792)--(9.775,5.794)--(9.778,5.796)%
  --(9.780,5.798)--(9.783,5.800)--(9.786,5.802)--(9.789,5.804)--(9.792,5.807)--(9.795,5.809)%
  --(9.798,5.811)--(9.801,5.813)--(9.804,5.815)--(9.807,5.817)--(9.810,5.819)--(9.813,5.821)%
  --(9.816,5.823)--(9.819,5.825)--(9.822,5.827)--(9.825,5.829)--(9.828,5.831)--(9.831,5.833)%
  --(9.834,5.836)--(9.837,5.838)--(9.840,5.840)--(9.843,5.842)--(9.846,5.844)--(9.849,5.846)%
  --(9.852,5.848)--(9.855,5.850)--(9.858,5.852)--(9.861,5.854)--(9.864,5.856)--(9.867,5.858)%
  --(9.870,5.860)--(9.873,5.863)--(9.876,5.865)--(9.879,5.867)--(9.882,5.869)--(9.885,5.871)%
  --(9.888,5.873)--(9.891,5.875)--(9.894,5.877)--(9.897,5.879)--(9.900,5.881)--(9.903,5.883)%
  --(9.906,5.885)--(9.909,5.887)--(9.912,5.889)--(9.915,5.892)--(9.918,5.894)--(9.921,5.896)%
  --(9.924,5.898)--(9.927,5.900)--(9.930,5.902)--(9.933,5.904)--(9.936,5.906)--(9.939,5.908)%
  --(9.942,5.910)--(9.945,5.912)--(9.948,5.914)--(9.951,5.916)--(9.954,5.919)--(9.957,5.921)%
  --(9.960,5.923)--(9.963,5.925)--(9.966,5.927)--(9.969,5.929)--(9.972,5.931)--(9.975,5.933)%
  --(9.978,5.935)--(9.981,5.937)--(9.984,5.939)--(9.987,5.941)--(9.990,5.943)--(9.992,5.946)%
  --(9.995,5.948)--(9.998,5.950)--(10.001,5.952)--(10.004,5.954)--(10.007,5.956)--(10.010,5.958)%
  --(10.013,5.960)--(10.016,5.962)--(10.019,5.964)--(10.022,5.966)--(10.025,5.968)--(10.028,5.970)%
  --(10.031,5.972)--(10.034,5.975)--(10.037,5.977)--(10.040,5.979)--(10.043,5.981)--(10.046,5.983)%
  --(10.049,5.985)--(10.052,5.987)--(10.055,5.989)--(10.058,5.991)--(10.061,5.993)--(10.064,5.995)%
  --(10.067,5.997)--(10.070,5.999)--(10.073,6.002)--(10.076,6.004)--(10.079,6.006)--(10.082,6.008)%
  --(10.085,6.010)--(10.088,6.012)--(10.091,6.014)--(10.094,6.016)--(10.097,6.018)--(10.100,6.020)%
  --(10.103,6.022)--(10.106,6.024)--(10.109,6.026)--(10.112,6.029)--(10.115,6.031)--(10.118,6.033)%
  --(10.121,6.035)--(10.124,6.037)--(10.127,6.039)--(10.130,6.041)--(10.133,6.043)--(10.136,6.045)%
  --(10.139,6.047)--(10.142,6.049)--(10.145,6.051)--(10.148,6.054)--(10.151,6.056)--(10.154,6.058)%
  --(10.157,6.060)--(10.160,6.062)--(10.163,6.064)--(10.166,6.066)--(10.169,6.068)--(10.172,6.070)%
  --(10.175,6.072)--(10.178,6.074)--(10.181,6.076)--(10.184,6.078)--(10.187,6.081)--(10.190,6.083)%
  --(10.193,6.085)--(10.196,6.087)--(10.199,6.089)--(10.201,6.091)--(10.204,6.093)--(10.207,6.095)%
  --(10.210,6.097)--(10.213,6.099)--(10.216,6.101)--(10.219,6.103)--(10.222,6.105)--(10.225,6.108)%
  --(10.228,6.110)--(10.231,6.112)--(10.234,6.114)--(10.237,6.116)--(10.240,6.118)--(10.243,6.120)%
  --(10.246,6.122)--(10.249,6.124)--(10.252,6.126)--(10.255,6.128)--(10.258,6.130)--(10.261,6.133)%
  --(10.264,6.135)--(10.267,6.137)--(10.270,6.139)--(10.273,6.141)--(10.276,6.143)--(10.279,6.145)%
  --(10.282,6.147)--(10.285,6.149)--(10.288,6.151)--(10.291,6.153)--(10.294,6.155)--(10.297,6.157)%
  --(10.300,6.160)--(10.303,6.162)--(10.306,6.164)--(10.309,6.166)--(10.312,6.168)--(10.315,6.170)%
  --(10.318,6.172)--(10.321,6.174)--(10.324,6.176)--(10.327,6.178)--(10.330,6.180)--(10.333,6.182)%
  --(10.336,6.185)--(10.339,6.187)--(10.342,6.189)--(10.345,6.191)--(10.348,6.193)--(10.351,6.195)%
  --(10.354,6.197)--(10.357,6.199)--(10.360,6.201)--(10.363,6.203)--(10.366,6.205)--(10.369,6.207)%
  --(10.372,6.209)--(10.375,6.212)--(10.378,6.214)--(10.381,6.216)--(10.384,6.218)--(10.387,6.220)%
  --(10.390,6.222)--(10.393,6.224)--(10.396,6.226)--(10.399,6.228)--(10.402,6.230)--(10.405,6.232)%
  --(10.408,6.234)--(10.410,6.237)--(10.413,6.239)--(10.416,6.241)--(10.419,6.243)--(10.422,6.245)%
  --(10.425,6.247)--(10.428,6.249)--(10.431,6.251)--(10.434,6.253)--(10.437,6.255)--(10.440,6.257)%
  --(10.443,6.259)--(10.446,6.262)--(10.449,6.264)--(10.452,6.266)--(10.455,6.268)--(10.458,6.270)%
  --(10.461,6.272)--(10.464,6.274)--(10.467,6.276)--(10.470,6.278)--(10.473,6.280)--(10.476,6.282)%
  --(10.479,6.284)--(10.482,6.287)--(10.485,6.289)--(10.488,6.291)--(10.491,6.293)--(10.494,6.295)%
  --(10.497,6.297)--(10.500,6.299)--(10.503,6.301)--(10.506,6.303)--(10.509,6.305)--(10.512,6.307)%
  --(10.515,6.309)--(10.518,6.312)--(10.521,6.314)--(10.524,6.316)--(10.527,6.318)--(10.530,6.320)%
  --(10.533,6.322)--(10.536,6.324)--(10.539,6.326)--(10.542,6.328)--(10.545,6.330)--(10.548,6.332)%
  --(10.551,6.334)--(10.554,6.337)--(10.557,6.339)--(10.560,6.341)--(10.563,6.343)--(10.566,6.345)%
  --(10.569,6.347)--(10.572,6.349)--(10.575,6.351)--(10.578,6.353)--(10.581,6.355)--(10.584,6.357)%
  --(10.587,6.359)--(10.590,6.362)--(10.593,6.364)--(10.596,6.366)--(10.599,6.368)--(10.602,6.370)%
  --(10.605,6.372)--(10.608,6.374)--(10.611,6.376)--(10.614,6.378)--(10.617,6.380)--(10.619,6.382)%
  --(10.622,6.384)--(10.625,6.387)--(10.628,6.389)--(10.631,6.391)--(10.634,6.393)--(10.637,6.395)%
  --(10.640,6.397)--(10.643,6.399)--(10.646,6.401)--(10.649,6.403)--(10.652,6.405)--(10.655,6.407)%
  --(10.658,6.409)--(10.661,6.412)--(10.664,6.414)--(10.667,6.416)--(10.670,6.418)--(10.673,6.420)%
  --(10.676,6.422)--(10.679,6.424)--(10.682,6.426)--(10.685,6.428)--(10.688,6.430)--(10.691,6.432)%
  --(10.694,6.434)--(10.697,6.437)--(10.700,6.439)--(10.703,6.441)--(10.706,6.443)--(10.709,6.445)%
  --(10.712,6.447)--(10.715,6.449)--(10.718,6.451)--(10.721,6.453)--(10.724,6.455)--(10.727,6.457)%
  --(10.730,6.460)--(10.733,6.462)--(10.736,6.464)--(10.739,6.466)--(10.742,6.468)--(10.745,6.470)%
  --(10.748,6.472)--(10.751,6.474)--(10.754,6.476)--(10.757,6.478)--(10.760,6.480)--(10.763,6.482)%
  --(10.766,6.485)--(10.769,6.487)--(10.772,6.489)--(10.775,6.491)--(10.778,6.493)--(10.781,6.495)%
  --(10.784,6.497)--(10.787,6.499)--(10.790,6.501)--(10.793,6.503)--(10.796,6.505)--(10.799,6.507)%
  --(10.802,6.510)--(10.805,6.512)--(10.808,6.514)--(10.811,6.516)--(10.814,6.518)--(10.817,6.520)%
  --(10.820,6.522)--(10.823,6.524)--(10.826,6.526)--(10.828,6.528)--(10.831,6.530)--(10.834,6.533)%
  --(10.837,6.535)--(10.840,6.537)--(10.843,6.539)--(10.846,6.541)--(10.849,6.543)--(10.852,6.545)%
  --(10.855,6.547)--(10.858,6.549)--(10.861,6.551)--(10.864,6.553)--(10.867,6.555)--(10.870,6.558)%
  --(10.873,6.560)--(10.876,6.562)--(10.879,6.564)--(10.882,6.566)--(10.885,6.568)--(10.888,6.570)%
  --(10.891,6.572)--(10.894,6.574)--(10.897,6.576)--(10.900,6.578)--(10.903,6.581)--(10.906,6.583)%
  --(10.909,6.585)--(10.912,6.587)--(10.915,6.589)--(10.918,6.591)--(10.921,6.593)--(10.924,6.595)%
  --(10.927,6.597)--(10.930,6.599)--(10.933,6.601)--(10.936,6.604)--(10.939,6.606)--(10.942,6.608)%
  --(10.945,6.610)--(10.948,6.612)--(10.951,6.614)--(10.954,6.616)--(10.957,6.618)--(10.960,6.620)%
  --(10.963,6.622)--(10.966,6.624)--(10.969,6.626)--(10.972,6.629)--(10.975,6.631)--(10.978,6.633)%
  --(10.981,6.635)--(10.984,6.637)--(10.987,6.639)--(10.990,6.641)--(10.993,6.643)--(10.996,6.645)%
  --(10.999,6.647)--(11.002,6.649)--(11.005,6.652)--(11.008,6.654)--(11.011,6.656)--(11.014,6.658)%
  --(11.017,6.660)--(11.020,6.662)--(11.023,6.664)--(11.026,6.666)--(11.029,6.668)--(11.032,6.670)%
  --(11.035,6.672)--(11.037,6.675)--(11.040,6.677)--(11.043,6.679)--(11.046,6.681)--(11.049,6.683)%
  --(11.052,6.685)--(11.055,6.687)--(11.058,6.689)--(11.061,6.691)--(11.064,6.693)--(11.067,6.695)%
  --(11.070,6.697)--(11.073,6.700)--(11.076,6.702)--(11.079,6.704)--(11.082,6.706)--(11.085,6.708)%
  --(11.088,6.710)--(11.091,6.712)--(11.094,6.714)--(11.097,6.716)--(11.100,6.718)--(11.103,6.720)%
  --(11.106,6.723)--(11.109,6.725)--(11.112,6.727)--(11.115,6.729)--(11.118,6.731)--(11.121,6.733)%
  --(11.124,6.735)--(11.127,6.737)--(11.130,6.739)--(11.133,6.741)--(11.136,6.743)--(11.139,6.746)%
  --(11.142,6.748)--(11.145,6.750)--(11.148,6.752)--(11.151,6.754)--(11.154,6.756)--(11.157,6.758)%
  --(11.160,6.760)--(11.163,6.762)--(11.166,6.764)--(11.169,6.766)--(11.172,6.769)--(11.175,6.771)%
  --(11.178,6.773)--(11.181,6.775)--(11.184,6.777)--(11.187,6.779)--(11.190,6.781)--(11.193,6.783)%
  --(11.196,6.785)--(11.199,6.787)--(11.202,6.789)--(11.205,6.792)--(11.208,6.794)--(11.211,6.796)%
  --(11.214,6.798)--(11.217,6.800)--(11.220,6.802)--(11.223,6.804)--(11.226,6.806)--(11.229,6.808)%
  --(11.232,6.810)--(11.235,6.812)--(11.238,6.815)--(11.241,6.817)--(11.244,6.819)--(11.247,6.821)%
  --(11.249,6.823)--(11.252,6.825)--(11.255,6.827)--(11.258,6.829)--(11.261,6.831)--(11.264,6.833)%
  --(11.267,6.835)--(11.270,6.838)--(11.273,6.840)--(11.276,6.842)--(11.279,6.844)--(11.282,6.846)%
  --(11.285,6.848)--(11.288,6.850)--(11.291,6.852)--(11.294,6.854)--(11.297,6.856)--(11.300,6.858)%
  --(11.303,6.861)--(11.306,6.863)--(11.309,6.865)--(11.312,6.867)--(11.315,6.869)--(11.318,6.871)%
  --(11.321,6.873)--(11.324,6.875)--(11.327,6.877)--(11.330,6.879)--(11.333,6.881)--(11.336,6.884)%
  --(11.339,6.886)--(11.342,6.888)--(11.345,6.890)--(11.348,6.892)--(11.351,6.894)--(11.354,6.896)%
  --(11.357,6.898)--(11.360,6.900)--(11.363,6.902)--(11.366,6.904)--(11.369,6.907)--(11.372,6.909)%
  --(11.375,6.911)--(11.378,6.913)--(11.381,6.915)--(11.384,6.917)--(11.387,6.919)--(11.390,6.921)%
  --(11.393,6.923)--(11.396,6.925)--(11.399,6.928)--(11.402,6.930)--(11.405,6.932)--(11.408,6.934)%
  --(11.411,6.936)--(11.414,6.938)--(11.417,6.940)--(11.420,6.942)--(11.423,6.944)--(11.426,6.946)%
  --(11.429,6.948)--(11.432,6.951)--(11.435,6.953)--(11.438,6.955)--(11.441,6.957)--(11.444,6.959)%
  --(11.447,6.961)--(11.450,6.963)--(11.453,6.965)--(11.456,6.967)--(11.458,6.969)--(11.461,6.971)%
  --(11.464,6.974)--(11.467,6.976)--(11.470,6.978)--(11.473,6.980)--(11.476,6.982)--(11.479,6.984)%
  --(11.482,6.986)--(11.485,6.988)--(11.488,6.990)--(11.491,6.992)--(11.494,6.994)--(11.497,6.997)%
  --(11.500,6.999)--(11.503,7.001)--(11.506,7.003)--(11.509,7.005)--(11.512,7.007)--(11.515,7.009)%
  --(11.518,7.011)--(11.521,7.013)--(11.524,7.015)--(11.527,7.018)--(11.530,7.020)--(11.533,7.022)%
  --(11.536,7.024)--(11.539,7.026)--(11.542,7.028)--(11.545,7.030)--(11.548,7.032)--(11.551,7.034)%
  --(11.554,7.036)--(11.557,7.038)--(11.560,7.041)--(11.563,7.043)--(11.566,7.045)--(11.569,7.047)%
  --(11.572,7.049)--(11.575,7.051)--(11.578,7.053)--(11.581,7.055)--(11.584,7.057)--(11.587,7.059)%
  --(11.590,7.061)--(11.593,7.064)--(11.596,7.066)--(11.599,7.068)--(11.602,7.070)--(11.605,7.072)%
  --(11.608,7.074)--(11.611,7.076)--(11.614,7.078)--(11.617,7.080)--(11.620,7.082)--(11.623,7.085)%
  --(11.626,7.087)--(11.629,7.089)--(11.632,7.091)--(11.635,7.093)--(11.638,7.095)--(11.641,7.097)%
  --(11.644,7.099)--(11.647,7.101)--(11.650,7.103)--(11.653,7.105)--(11.656,7.108)--(11.659,7.110)%
  --(11.662,7.112)--(11.665,7.114)--(11.667,7.116)--(11.670,7.118)--(11.673,7.120)--(11.676,7.122)%
  --(11.679,7.124)--(11.682,7.126)--(11.685,7.129)--(11.688,7.131)--(11.691,7.133)--(11.694,7.135)%
  --(11.697,7.137)--(11.700,7.139)--(11.703,7.141)--(11.706,7.143)--(11.709,7.145)--(11.712,7.147)%
  --(11.715,7.149)--(11.718,7.152)--(11.721,7.154)--(11.724,7.156)--(11.727,7.158)--(11.730,7.160)%
  --(11.733,7.162)--(11.736,7.164)--(11.739,7.166)--(11.742,7.168)--(11.745,7.170)--(11.748,7.173)%
  --(11.751,7.175)--(11.754,7.177)--(11.757,7.179)--(11.760,7.181)--(11.763,7.183)--(11.766,7.185)%
  --(11.769,7.187)--(11.772,7.189)--(11.775,7.191)--(11.778,7.193)--(11.781,7.196)--(11.784,7.198)%
  --(11.787,7.200)--(11.790,7.202)--(11.793,7.204)--(11.796,7.206)--(11.799,7.208)--(11.802,7.210)%
  --(11.805,7.212)--(11.808,7.214)--(11.811,7.217)--(11.814,7.219)--(11.817,7.221)--(11.820,7.223)%
  --(11.823,7.225)--(11.826,7.227)--(11.829,7.229)--(11.832,7.231)--(11.835,7.233)--(11.838,7.235)%
  --(11.841,7.237)--(11.844,7.240)--(11.847,7.242)--(11.850,7.244)--(11.853,7.246)--(11.856,7.248)%
  --(11.859,7.250)--(11.862,7.252)--(11.865,7.254)--(11.868,7.256)--(11.871,7.258)--(11.874,7.261)%
  --(11.876,7.263)--(11.879,7.265)--(11.882,7.267)--(11.885,7.269)--(11.888,7.271)--(11.891,7.273)%
  --(11.894,7.275)--(11.897,7.277)--(11.900,7.279)--(11.903,7.281)--(11.906,7.284)--(11.909,7.286)%
  --(11.912,7.288)--(11.915,7.290)--(11.918,7.292)--(11.921,7.294)--(11.924,7.296)--(11.927,7.298)%
  --(11.930,7.300)--(11.933,7.302)--(11.936,7.305)--(11.939,7.307)--(11.942,7.309)--(11.945,7.311)%
  --(11.948,7.313)--(11.951,7.315)--(11.954,7.317)--(11.957,7.319)--(11.960,7.321)--(11.963,7.323)%
  --(11.966,7.326)--(11.969,7.328)--(11.972,7.330)--(11.975,7.332)--(11.978,7.334)--(11.981,7.336)%
  --(11.984,7.338)--(11.987,7.340)--(11.990,7.342)--(11.993,7.344)--(11.996,7.347)--(11.999,7.349)%
  --(12.002,7.351)--(12.005,7.353)--(12.008,7.355)--(12.011,7.357)--(12.014,7.359)--(12.017,7.361)%
  --(12.020,7.363)--(12.023,7.365)--(12.026,7.367)--(12.029,7.370)--(12.032,7.372)--(12.035,7.374)%
  --(12.038,7.376)--(12.041,7.378)--(12.044,7.380)--(12.047,7.382)--(12.050,7.384)--(12.053,7.386)%
  --(12.056,7.388)--(12.059,7.391)--(12.062,7.393)--(12.065,7.395)--(12.068,7.397)--(12.071,7.399)%
  --(12.074,7.401)--(12.077,7.403)--(12.080,7.405)--(12.083,7.407)--(12.085,7.409)--(12.088,7.412)%
  --(12.091,7.414)--(12.094,7.416)--(12.097,7.418)--(12.100,7.420)--(12.103,7.422)--(12.106,7.424)%
  --(12.109,7.426)--(12.112,7.428)--(12.115,7.430)--(12.118,7.433)--(12.121,7.435)--(12.124,7.437)%
  --(12.127,7.439)--(12.130,7.441)--(12.133,7.443)--(12.136,7.445)--(12.139,7.447)--(12.142,7.449)%
  --(12.145,7.451)--(12.148,7.453)--(12.151,7.456)--(12.154,7.458)--(12.157,7.460)--(12.160,7.462)%
  --(12.163,7.464)--(12.166,7.466)--(12.169,7.468)--(12.172,7.470)--(12.175,7.472)--(12.178,7.474)%
  --(12.181,7.477)--(12.184,7.479)--(12.187,7.481)--(12.190,7.483)--(12.193,7.485)--(12.196,7.487)%
  --(12.199,7.489)--(12.202,7.491)--(12.205,7.493)--(12.208,7.495)--(12.211,7.498)--(12.214,7.500)%
  --(12.217,7.502)--(12.220,7.504)--(12.223,7.506)--(12.226,7.508)--(12.229,7.510)--(12.232,7.512)%
  --(12.235,7.514)--(12.238,7.516)--(12.241,7.519)--(12.244,7.521)--(12.247,7.523)--(12.250,7.525)%
  --(12.253,7.527)--(12.256,7.529)--(12.259,7.531)--(12.262,7.533)--(12.265,7.535)--(12.268,7.537)%
  --(12.271,7.540)--(12.274,7.542)--(12.277,7.544)--(12.280,7.546)--(12.283,7.548)--(12.286,7.550)%
  --(12.289,7.552)--(12.292,7.554)--(12.295,7.556)--(12.297,7.558)--(12.300,7.561)--(12.303,7.563)%
  --(12.306,7.565)--(12.309,7.567)--(12.312,7.569)--(12.315,7.571)--(12.318,7.573)--(12.321,7.575)%
  --(12.324,7.577)--(12.327,7.579)--(12.330,7.582)--(12.333,7.584)--(12.336,7.586)--(12.339,7.588)%
  --(12.342,7.590)--(12.345,7.592)--(12.348,7.594)--(12.351,7.596)--(12.354,7.598)--(12.357,7.600)%
  --(12.360,7.603)--(12.363,7.605)--(12.366,7.607)--(12.369,7.609)--(12.372,7.611)--(12.375,7.613)%
  --(12.378,7.615)--(12.381,7.617)--(12.384,7.619)--(12.387,7.621)--(12.390,7.624)--(12.393,7.626)%
  --(12.396,7.628)--(12.399,7.630)--(12.402,7.632)--(12.405,7.634)--(12.408,7.636)--(12.411,7.638)%
  --(12.414,7.640)--(12.417,7.642)--(12.420,7.645)--(12.423,7.647)--(12.426,7.649)--(12.429,7.651)%
  --(12.432,7.653)--(12.435,7.655)--(12.438,7.657)--(12.441,7.659)--(12.444,7.661)--(12.447,7.663)%
  --(12.450,7.666)--(12.453,7.668)--(12.456,7.670)--(12.459,7.672)--(12.462,7.674)--(12.465,7.676)%
  --(12.468,7.678)--(12.471,7.680)--(12.474,7.682)--(12.477,7.684)--(12.480,7.687)--(12.483,7.689)%
  --(12.486,7.691)--(12.489,7.693)--(12.492,7.695)--(12.495,7.697)--(12.498,7.699)--(12.501,7.701)%
  --(12.504,7.703)--(12.506,7.705)--(12.509,7.708)--(12.512,7.710)--(12.515,7.712)--(12.518,7.714)%
  --(12.521,7.716)--(12.524,7.718)--(12.527,7.720)--(12.530,7.722)--(12.533,7.724)--(12.536,7.726)%
  --(12.539,7.729)--(12.542,7.731)--(12.545,7.733)--(12.548,7.735)--(12.551,7.737)--(12.554,7.739)%
  --(12.557,7.741)--(12.560,7.743)--(12.563,7.745)--(12.566,7.747)--(12.569,7.750)--(12.572,7.752)%
  --(12.575,7.754)--(12.578,7.756)--(12.581,7.758)--(12.584,7.760)--(12.587,7.762)--(12.590,7.764)%
  --(12.593,7.766)--(12.596,7.768)--(12.599,7.771)--(12.602,7.773)--(12.605,7.775)--(12.608,7.777)%
  --(12.611,7.779)--(12.614,7.781)--(12.617,7.783)--(12.620,7.785)--(12.623,7.787)--(12.626,7.789)%
  --(12.629,7.792)--(12.632,7.794)--(12.635,7.796)--(12.638,7.798)--(12.641,7.800)--(12.644,7.802)%
  --(12.647,7.804)--(12.650,7.806)--(12.653,7.808)--(12.656,7.810)--(12.659,7.813)--(12.662,7.815)%
  --(12.665,7.817)--(12.668,7.819)--(12.671,7.821)--(12.674,7.823)--(12.677,7.825)--(12.680,7.827)%
  --(12.683,7.829)--(12.686,7.832)--(12.689,7.834)--(12.692,7.836)--(12.695,7.838)--(12.698,7.840)%
  --(12.701,7.842)--(12.704,7.844)--(12.707,7.846)--(12.710,7.848)--(12.713,7.850)--(12.715,7.853)%
  --(12.718,7.855)--(12.721,7.857)--(12.724,7.859)--(12.727,7.861)--(12.730,7.863)--(12.733,7.865)%
  --(12.736,7.867)--(12.739,7.869)--(12.742,7.871)--(12.745,7.874)--(12.748,7.876)--(12.751,7.878)%
  --(12.754,7.880)--(12.757,7.882)--(12.760,7.884)--(12.763,7.886)--(12.766,7.888)--(12.769,7.890)%
  --(12.772,7.892)--(12.775,7.895)--(12.778,7.897)--(12.781,7.899)--(12.784,7.901)--(12.787,7.903)%
  --(12.790,7.905)--(12.793,7.907)--(12.796,7.909)--(12.799,7.911)--(12.802,7.913)--(12.805,7.916)%
  --(12.808,7.918)--(12.811,7.920)--(12.814,7.922)--(12.817,7.924)--(12.820,7.926)--(12.823,7.928)%
  --(12.826,7.930)--(12.829,7.932)--(12.832,7.935)--(12.835,7.937)--(12.838,7.939)--(12.841,7.941)%
  --(12.844,7.943)--(12.847,7.945)--(12.850,7.947)--(12.853,7.949)--(12.856,7.951)--(12.859,7.953)%
  --(12.862,7.956)--(12.865,7.958)--(12.868,7.960)--(12.871,7.962)--(12.874,7.964)--(12.877,7.966)%
  --(12.880,7.968)--(12.883,7.970)--(12.886,7.972)--(12.889,7.974)--(12.892,7.977)--(12.895,7.979)%
  --(12.898,7.981)--(12.901,7.983)--(12.904,7.985)--(12.907,7.987)--(12.910,7.989)--(12.913,7.991)%
  --(12.916,7.993)--(12.919,7.995)--(12.922,7.998)--(12.924,8.000)--(12.927,8.002)--(12.930,8.004)%
  --(12.933,8.006)--(12.936,8.008)--(12.939,8.010)--(12.942,8.012)--(12.945,8.014)--(12.948,8.017)%
  --(12.951,8.019)--(12.954,8.021)--(12.957,8.023)--(12.960,8.025)--(12.963,8.027)--(12.966,8.029)%
  --(12.969,8.031)--(12.972,8.033)--(12.975,8.035)--(12.978,8.038)--(12.981,8.040)--(12.984,8.042)%
  --(12.987,8.044)--(12.990,8.046)--(12.993,8.048)--(12.996,8.050)--(12.999,8.052)--(13.002,8.054)%
  --(13.005,8.056)--(13.008,8.059)--(13.011,8.061)--(13.014,8.063)--(13.017,8.065)--(13.020,8.067)%
  --(13.023,8.069)--(13.026,8.071)--(13.029,8.073)--(13.032,8.075)--(13.035,8.078)--(13.038,8.080)%
  --(13.041,8.082)--(13.044,8.084)--(13.047,8.086)--(13.050,8.088)--(13.053,8.090)--(13.056,8.092)%
  --(13.059,8.094)--(13.062,8.096)--(13.065,8.099)--(13.068,8.101)--(13.071,8.103)--(13.074,8.105)%
  --(13.077,8.107)--(13.080,8.109)--(13.083,8.111)--(13.086,8.113)--(13.089,8.115)--(13.092,8.117)%
  --(13.095,8.120)--(13.098,8.122)--(13.101,8.124)--(13.104,8.126)--(13.107,8.128)--(13.110,8.130)%
  --(13.113,8.132)--(13.116,8.134)--(13.119,8.136)--(13.122,8.139)--(13.125,8.141)--(13.128,8.143)%
  --(13.131,8.145)--(13.133,8.147)--(13.136,8.149)--(13.139,8.151)--(13.142,8.153)--(13.145,8.155)%
  --(13.148,8.157)--(13.151,8.160)--(13.154,8.162)--(13.157,8.164)--(13.160,8.166)--(13.163,8.168)%
  --(13.166,8.170)--(13.169,8.172)--(13.172,8.174)--(13.175,8.176)--(13.178,8.179)--(13.181,8.181)%
  --(13.184,8.183)--(13.187,8.185)--(13.190,8.187)--(13.193,8.189)--(13.196,8.191)--(13.199,8.193)%
  --(13.202,8.195)--(13.205,8.197)--(13.208,8.200)--(13.211,8.202)--(13.214,8.204)--(13.217,8.206)%
  --(13.220,8.208)--(13.223,8.210)--(13.226,8.212)--(13.229,8.214)--(13.232,8.216)--(13.235,8.218)%
  --(13.238,8.221)--(13.241,8.223)--(13.244,8.225)--(13.247,8.227)--(13.250,8.229)--(13.253,8.231)%
  --(13.256,8.233)--(13.259,8.235)--(13.262,8.237)--(13.265,8.240)--(13.268,8.242)--(13.271,8.244)%
  --(13.274,8.246)--(13.277,8.248)--(13.280,8.250)--(13.283,8.252)--(13.286,8.254)--(13.289,8.256)%
  --(13.292,8.258)--(13.295,8.261)--(13.298,8.263)--(13.301,8.265)--(13.304,8.267)--(13.307,8.269)%
  --(13.310,8.271)--(13.313,8.273)--(13.316,8.275)--(13.319,8.277)--(13.322,8.280)--(13.325,8.282)%
  --(13.328,8.284)--(13.331,8.286)--(13.334,8.288)--(13.337,8.290)--(13.340,8.292)--(13.342,8.294)%
  --(13.345,8.296)--(13.348,8.298)--(13.351,8.301)--(13.354,8.303)--(13.357,8.305)--(13.360,8.307)%
  --(13.363,8.309)--(13.366,8.311)--(13.369,8.313)--(13.372,8.315)--(13.375,8.317)--(13.378,8.320)%
  --(13.381,8.322)--(13.384,8.324)--(13.387,8.326)--(13.390,8.328)--(13.393,8.330)--(13.396,8.332)%
  --(13.399,8.334)--(13.402,8.336)--(13.405,8.338)--(13.408,8.341)--(13.411,8.343)--(13.414,8.345)%
  --(13.417,8.347)--(13.420,8.349)--(13.423,8.351)--(13.426,8.353)--(13.429,8.355)--(13.432,8.357)%
  --(13.435,8.360)--(13.438,8.362)--(13.441,8.364)--(13.444,8.366);
\gpcolor{rgb color={0.941,0.894,0.259}}
\draw[gp path] (1.504,3.534)--(1.625,3.534)--(1.745,3.534)--(1.866,3.534)--(1.986,3.534)%
  --(2.107,3.534)--(2.228,3.534)--(2.348,3.534)--(2.469,3.534)--(2.589,3.534)--(2.710,3.534)%
  --(2.831,3.534)--(2.951,3.534)--(3.072,3.534)--(3.192,3.534)--(3.313,3.534)--(3.434,3.534)%
  --(3.554,3.534)--(3.675,3.534)--(3.796,3.534)--(3.916,3.534)--(4.037,3.534)--(4.157,3.534)%
  --(4.278,3.534)--(4.399,3.534)--(4.519,3.534)--(4.640,3.534)--(4.760,3.534)--(4.881,3.534)%
  --(5.002,3.534)--(5.122,3.534)--(5.243,3.534)--(5.363,3.534)--(5.484,3.534)--(5.605,3.534)%
  --(5.725,3.534)--(5.846,3.534)--(5.966,3.534)--(6.087,3.534)--(6.208,3.534)--(6.328,3.534)%
  --(6.449,3.534)--(6.569,3.534)--(6.690,3.534)--(6.811,3.534)--(6.931,3.534)--(7.052,3.534)%
  --(7.172,3.534)--(7.293,3.534)--(7.414,3.534)--(7.534,3.534)--(7.655,3.534)--(7.776,3.534)%
  --(7.896,3.534)--(8.017,3.534)--(8.137,3.534)--(8.258,3.534)--(8.379,3.534)--(8.499,3.534)%
  --(8.620,3.534)--(8.740,3.534)--(8.861,3.534)--(8.982,3.534)--(9.102,3.534)--(9.223,3.534)%
  --(9.343,3.534)--(9.464,3.534)--(9.585,3.534)--(9.705,3.534)--(9.826,3.534)--(9.946,3.534)%
  --(10.067,3.534)--(10.188,3.534)--(10.308,3.534)--(10.429,3.534)--(10.549,3.534)--(10.670,3.534)%
  --(10.791,3.534)--(10.911,3.534)--(11.032,3.534)--(11.152,3.534)--(11.273,3.534)--(11.394,3.534)%
  --(11.514,3.534)--(11.635,3.534)--(11.756,3.534)--(11.876,3.534)--(11.997,3.534)--(12.117,3.534)%
  --(12.238,3.534)--(12.359,3.534)--(12.479,3.534)--(12.600,3.534)--(12.720,3.534)--(12.841,3.534)%
  --(12.962,3.534)--(13.082,3.534)--(13.203,3.534)--(13.323,3.534)--(13.444,3.534);
\gpcolor{color=gp lt color border}
\draw[gp path] (1.504,8.631)--(1.504,0.985)--(13.447,0.985)--(13.447,8.631)--cycle;
%% coordinates of the plot area
\gpdefrectangularnode{gp plot 1}{\pgfpoint{1.504cm}{0.985cm}}{\pgfpoint{13.447cm}{8.631cm}}
\end{tikzpicture}
%% gnuplot variables

	\caption{Exemplos de formas diferentes para $f(M)$ para valores diferentes de $\rho$. Os momentos de Fermi foram determinados a partir de \eqref{Eq:Mom_Fermi_a_partir_de_rho} com fração de prótons 1/2. \protect[Parameters: eNJL1m; Proton fraction: 1/2] }
	\label{Fig:Gap_zero_graph_eNJL1m}
\end{figure*}

\begin{figure*}
	\begin{tikzpicture}[gnuplot]
%% generated with GNUPLOT 5.0p2 (Lua 5.2; terminal rev. 99, script rev. 100)
%% Mon Mar  7 16:11:42 2016
\path (0.000,0.000) rectangle (14.000,9.000);
\gpcolor{color=gp lt color border}
\gpsetlinetype{gp lt border}
\gpsetdashtype{gp dt solid}
\gpsetlinewidth{1.00}
\draw[gp path] (1.504,0.985)--(1.684,0.985);
\draw[gp path] (13.447,0.985)--(13.267,0.985);
\node[gp node right] at (1.320,0.985) {$0$};
\draw[gp path] (1.504,1.750)--(1.684,1.750);
\draw[gp path] (13.447,1.750)--(13.267,1.750);
\node[gp node right] at (1.320,1.750) {$100$};
\draw[gp path] (1.504,2.514)--(1.684,2.514);
\draw[gp path] (13.447,2.514)--(13.267,2.514);
\node[gp node right] at (1.320,2.514) {$200$};
\draw[gp path] (1.504,3.279)--(1.684,3.279);
\draw[gp path] (13.447,3.279)--(13.267,3.279);
\node[gp node right] at (1.320,3.279) {$300$};
\draw[gp path] (1.504,4.043)--(1.684,4.043);
\draw[gp path] (13.447,4.043)--(13.267,4.043);
\node[gp node right] at (1.320,4.043) {$400$};
\draw[gp path] (1.504,4.808)--(1.684,4.808);
\draw[gp path] (13.447,4.808)--(13.267,4.808);
\node[gp node right] at (1.320,4.808) {$500$};
\draw[gp path] (1.504,5.573)--(1.684,5.573);
\draw[gp path] (13.447,5.573)--(13.267,5.573);
\node[gp node right] at (1.320,5.573) {$600$};
\draw[gp path] (1.504,6.337)--(1.684,6.337);
\draw[gp path] (13.447,6.337)--(13.267,6.337);
\node[gp node right] at (1.320,6.337) {$700$};
\draw[gp path] (1.504,7.102)--(1.684,7.102);
\draw[gp path] (13.447,7.102)--(13.267,7.102);
\node[gp node right] at (1.320,7.102) {$800$};
\draw[gp path] (1.504,7.866)--(1.684,7.866);
\draw[gp path] (13.447,7.866)--(13.267,7.866);
\node[gp node right] at (1.320,7.866) {$900$};
\draw[gp path] (1.504,8.631)--(1.684,8.631);
\draw[gp path] (13.447,8.631)--(13.267,8.631);
\node[gp node right] at (1.320,8.631) {$1000$};
\draw[gp path] (1.504,0.985)--(1.504,1.165);
\draw[gp path] (1.504,8.631)--(1.504,8.451);
\node[gp node center] at (1.504,0.677) {$0$};
\draw[gp path] (3.210,0.985)--(3.210,1.165);
\draw[gp path] (3.210,8.631)--(3.210,8.451);
\node[gp node center] at (3.210,0.677) {$0.5$};
\draw[gp path] (4.916,0.985)--(4.916,1.165);
\draw[gp path] (4.916,8.631)--(4.916,8.451);
\node[gp node center] at (4.916,0.677) {$1$};
\draw[gp path] (6.622,0.985)--(6.622,1.165);
\draw[gp path] (6.622,8.631)--(6.622,8.451);
\node[gp node center] at (6.622,0.677) {$1.5$};
\draw[gp path] (8.329,0.985)--(8.329,1.165);
\draw[gp path] (8.329,8.631)--(8.329,8.451);
\node[gp node center] at (8.329,0.677) {$2$};
\draw[gp path] (10.035,0.985)--(10.035,1.165);
\draw[gp path] (10.035,8.631)--(10.035,8.451);
\node[gp node center] at (10.035,0.677) {$2.5$};
\draw[gp path] (11.741,0.985)--(11.741,1.165);
\draw[gp path] (11.741,8.631)--(11.741,8.451);
\node[gp node center] at (11.741,0.677) {$3$};
\draw[gp path] (13.447,0.985)--(13.447,1.165);
\draw[gp path] (13.447,8.631)--(13.447,8.451);
\node[gp node center] at (13.447,0.677) {$3.5$};
\draw[gp path] (1.504,8.631)--(1.504,0.985)--(13.447,0.985)--(13.447,8.631)--cycle;
\node[gp node center,rotate=-270] at (0.246,4.808) {$M$ (MeV)};
\node[gp node center] at (7.475,0.215) {$\rho$ ($\rm{fm}^{-3}$)};
\gpcolor{rgb color={0.580,0.000,0.827}}
\draw[gp path] (1.548,8.107)--(1.559,8.096)--(1.569,8.082)--(1.579,8.071)--(1.589,8.059)%
  --(1.599,8.045)--(1.610,8.034)--(1.620,8.023)--(1.630,8.012)--(1.640,8.001)--(1.650,7.989)%
  --(1.661,7.978)--(1.671,7.967)--(1.681,7.956)--(1.691,7.945)--(1.702,7.933)--(1.712,7.925)%
  --(1.722,7.914)--(1.732,7.903)--(1.742,7.894)--(1.753,7.883)--(1.763,7.875)--(1.773,7.863)%
  --(1.783,7.855)--(1.793,7.844)--(1.804,7.835)--(1.814,7.827)--(1.824,7.819)--(1.834,7.807)%
  --(1.845,7.799)--(1.855,7.791)--(1.865,7.782)--(1.875,7.774)--(1.885,7.765)--(1.896,7.757)%
  --(1.906,7.749)--(1.916,7.740)--(1.926,7.732)--(1.936,7.723)--(1.947,7.715)--(1.957,7.709)%
  --(1.967,7.701)--(1.977,7.693)--(1.987,7.684)--(1.998,7.679)--(2.008,7.670)--(2.018,7.665)%
  --(2.028,7.656)--(2.039,7.648)--(2.049,7.642)--(2.059,7.637)--(2.069,7.628)--(2.079,7.623)%
  --(2.090,7.614)--(2.100,7.609)--(2.110,7.603)--(2.120,7.595)--(2.130,7.589)--(2.141,7.583)%
  --(2.151,7.575)--(2.161,7.569)--(2.171,7.564)--(2.182,7.558)--(2.192,7.553)--(2.202,7.547)%
  --(2.212,7.539)--(2.222,7.533)--(2.233,7.527)--(2.243,7.522)--(2.253,7.516)--(2.263,7.511)%
  --(2.273,7.505)--(2.284,7.499)--(2.294,7.494)--(2.304,7.488)--(2.314,7.483)--(2.325,7.477)%
  --(2.335,7.471)--(2.345,7.469)--(2.355,7.463)--(2.365,7.457)--(2.376,7.452)--(2.386,7.446)%
  --(2.396,7.441)--(2.406,7.435)--(2.416,7.432)--(2.427,7.427)--(2.437,7.421)--(2.447,7.415)%
  --(2.457,7.413)--(2.468,7.407)--(2.478,7.401)--(2.488,7.396)--(2.498,7.390)--(2.508,7.387)%
  --(2.519,7.382)--(2.529,7.376)--(2.539,7.373)--(2.549,7.368)--(2.559,7.362)--(2.570,7.357)%
  --(2.580,7.354)--(2.590,7.348)--(2.600,7.343)--(2.610,7.337)--(2.621,7.334)--(2.631,7.329)%
  --(2.641,7.323)--(2.651,7.320)--(2.662,7.315)--(2.672,7.309)--(2.682,7.303)--(2.692,7.301)%
  --(2.702,7.295)--(2.713,7.289)--(2.723,7.287)--(2.733,7.281)--(2.743,7.275)--(2.753,7.270)%
  --(2.764,7.267)--(2.774,7.261)--(2.784,7.256)--(2.794,7.250)--(2.805,7.245)--(2.815,7.242)%
  --(2.825,7.236)--(2.835,7.231)--(2.845,7.225)--(2.856,7.219)--(2.866,7.214)--(2.876,7.208)%
  --(2.886,7.205)--(2.896,7.200)--(2.907,7.194)--(2.917,7.189)--(2.927,7.183)--(2.937,7.177)%
  --(2.948,7.172)--(2.958,7.166)--(2.968,7.161)--(2.978,7.155)--(2.988,7.149)--(2.999,7.144)%
  --(3.009,7.138)--(3.019,7.133)--(3.029,7.127)--(3.039,7.119)--(3.050,7.113)--(3.060,7.107)%
  --(3.070,7.102)--(3.080,7.096)--(3.090,7.088)--(3.101,7.082)--(3.111,7.077)--(3.121,7.071)%
  --(3.131,7.063)--(3.142,7.057)--(3.152,7.049)--(3.162,7.043)--(3.172,7.037)--(3.182,7.029)%
  --(3.193,7.023)--(3.203,7.015)--(3.213,7.007)--(3.223,7.001)--(3.233,6.993)--(3.244,6.987)%
  --(3.254,6.979)--(3.264,6.970)--(3.274,6.965)--(3.285,6.956)--(3.295,6.948)--(3.305,6.939)%
  --(3.315,6.931)--(3.325,6.923)--(3.336,6.914)--(3.346,6.906)--(3.356,6.897)--(3.366,6.889)%
  --(3.376,6.881)--(3.387,6.872)--(3.397,6.864)--(3.407,6.855)--(3.417,6.847)--(3.428,6.836)%
  --(3.438,6.827)--(3.448,6.819)--(3.458,6.808)--(3.468,6.799)--(3.479,6.791)--(3.489,6.780)%
  --(3.499,6.771)--(3.509,6.760)--(3.519,6.749)--(3.530,6.741)--(3.540,6.729)--(3.550,6.718)%
  --(3.560,6.710)--(3.570,6.699)--(3.581,6.687)--(3.591,6.676)--(3.601,6.665)--(3.611,6.654)%
  --(3.622,6.643)--(3.632,6.631)--(3.642,6.620)--(3.652,6.606)--(3.662,6.595)--(3.673,6.584)%
  --(3.683,6.573)--(3.693,6.559)--(3.703,6.547)--(3.713,6.533)--(3.724,6.522)--(3.734,6.508)%
  --(3.744,6.497)--(3.754,6.483)--(3.765,6.469)--(3.775,6.455)--(3.785,6.444)--(3.795,6.430)%
  --(3.805,6.416)--(3.816,6.402)--(3.826,6.388)--(3.836,6.374)--(3.846,6.357)--(3.856,6.343)%
  --(3.867,6.329)--(3.877,6.315)--(3.887,6.298)--(3.897,6.284)--(3.908,6.270)--(3.918,6.253)%
  --(3.928,6.236)--(3.938,6.222)--(3.948,6.206)--(3.959,6.189)--(3.969,6.175)--(3.979,6.158)%
  --(3.989,6.141)--(3.999,6.124)--(4.010,6.108)--(4.020,6.091)--(4.030,6.074)--(4.040,6.057)%
  --(4.051,6.038)--(4.061,6.021)--(4.071,6.004)--(4.081,5.984)--(4.091,5.968)--(4.102,5.948)%
  --(4.112,5.931)--(4.122,5.912)--(4.132,5.892)--(4.142,5.875)--(4.153,5.856)--(4.163,5.836)%
  --(4.173,5.817)--(4.183,5.798)--(4.193,5.778)--(4.204,5.758)--(4.214,5.737)--(4.224,5.718)%
  --(4.234,5.698)--(4.245,5.677)--(4.255,5.656)--(4.265,5.635)--(4.275,5.616)--(4.285,5.595)%
  --(4.296,5.572)--(4.306,5.551)--(4.316,5.530)--(4.326,5.508)--(4.336,5.487)--(4.347,5.464)%
  --(4.357,5.443)--(4.367,5.421)--(4.377,5.399)--(4.388,5.376)--(4.398,5.354)--(4.408,5.331)%
  --(4.418,5.308)--(4.428,5.285)--(4.439,5.261)--(4.449,5.239)--(4.459,5.215)--(4.469,5.191)%
  --(4.479,5.169)--(4.490,5.145)--(4.500,5.121)--(4.510,5.098)--(4.520,5.074)--(4.531,5.049)%
  --(4.541,5.025)--(4.551,5.001)--(4.561,4.976)--(4.571,4.952)--(4.582,4.927)--(4.592,4.903)%
  --(4.602,4.878)--(4.612,4.853)--(4.622,4.829)--(4.633,4.804)--(4.643,4.778)--(4.653,4.753)%
  --(4.663,4.728)--(4.673,4.703)--(4.684,4.678)--(4.694,4.652)--(4.704,4.627)--(4.714,4.602)%
  --(4.725,4.577)--(4.735,4.552)--(4.745,4.526)--(4.755,4.501)--(4.765,4.476)--(4.776,4.449)%
  --(4.786,4.424)--(4.796,4.399)--(4.806,4.374)--(4.816,4.349)--(4.827,4.323)--(4.837,4.298)%
  --(4.847,4.273)--(4.857,4.248)--(4.868,4.223)--(4.878,4.197)--(4.888,4.172)--(4.898,4.147)%
  --(4.908,4.122)--(4.919,4.097)--(4.929,4.071)--(4.939,4.046)--(4.949,4.022)--(4.959,3.997)%
  --(4.970,3.972)--(4.980,3.948)--(4.990,3.924)--(5.000,3.899)--(5.011,3.875)--(5.021,3.852)%
  --(5.031,3.826)--(5.041,3.803)--(5.051,3.779)--(5.062,3.756)--(5.072,3.733)--(5.082,3.709)%
  --(5.092,3.685)--(5.102,3.663)--(5.113,3.639)--(5.123,3.616)--(5.133,3.594)--(5.143,3.572)%
  --(5.154,3.549)--(5.164,3.527)--(5.174,3.504)--(5.184,3.482)--(5.194,3.461)--(5.205,3.439)%
  --(5.215,3.418)--(5.225,3.397)--(5.235,3.376)--(5.245,3.355)--(5.256,3.334)--(5.266,3.313)%
  --(5.276,3.293)--(5.286,3.272)--(5.296,3.252)--(5.307,3.231)--(5.317,3.212)--(5.327,3.192)%
  --(5.337,3.173)--(5.348,3.154)--(5.358,3.135)--(5.368,3.117)--(5.378,3.097)--(5.388,3.079)%
  --(5.399,3.061)--(5.409,3.042)--(5.419,3.024)--(5.429,3.006)--(5.439,2.989)--(5.450,2.971)%
  --(5.460,2.954)--(5.470,2.936)--(5.480,2.919)--(5.491,2.903)--(5.501,2.886)--(5.511,2.870)%
  --(5.521,2.853)--(5.531,2.837)--(5.542,2.821)--(5.552,2.805)--(5.562,2.789)--(5.572,2.774)%
  --(5.582,2.758)--(5.593,2.744)--(5.603,2.728)--(5.613,2.714)--(5.623,2.699)--(5.634,2.684)%
  --(5.644,2.670)--(5.654,2.656)--(5.664,2.642)--(5.674,2.628)--(5.685,2.614)--(5.695,2.600)%
  --(5.705,2.586)--(5.715,2.573)--(5.725,2.560)--(5.736,2.547)--(5.746,2.534)--(5.756,2.521)%
  --(5.766,2.509)--(5.776,2.496)--(5.787,2.483)--(5.797,2.471)--(5.807,2.459)--(5.817,2.447)%
  --(5.828,2.435)--(5.838,2.423)--(5.848,2.412)--(5.858,2.400)--(5.868,2.389)--(5.879,2.378)%
  --(5.889,2.366)--(5.899,2.355)--(5.909,2.344)--(5.919,2.334)--(5.930,2.323)--(5.940,2.313)%
  --(5.950,2.302)--(5.960,2.292)--(5.971,2.281)--(5.981,2.271)--(5.991,2.261)--(6.001,2.251)%
  --(6.011,2.241)--(6.022,2.231)--(6.032,2.222)--(6.042,2.212)--(6.052,2.203)--(6.062,2.194)%
  --(6.073,2.184)--(6.083,2.175)--(6.093,2.166)--(6.103,2.157)--(6.114,2.148)--(6.124,2.140)%
  --(6.134,2.131)--(6.144,2.122)--(6.154,2.114)--(6.165,2.105)--(6.175,2.097)--(6.185,2.089)%
  --(6.195,2.080)--(6.205,2.072)--(6.216,2.064)--(6.226,2.056)--(6.236,2.049)--(6.246,2.041)%
  --(6.257,2.033)--(6.267,2.026)--(6.277,2.018)--(6.287,2.010)--(6.297,2.003)--(6.308,1.995)%
  --(6.318,1.988)--(6.328,1.981)--(6.338,1.974)--(6.348,1.967)--(6.359,1.960)--(6.369,1.953)%
  --(6.379,1.946)--(6.389,1.939)--(6.399,1.932)--(6.410,1.926)--(6.420,1.919)--(6.430,1.913)%
  --(6.440,1.907)--(6.451,1.900)--(6.461,1.893)--(6.471,1.887)--(6.481,1.881)--(6.491,1.875)%
  --(6.502,1.869)--(6.512,1.862)--(6.522,1.857)--(6.532,1.851)--(6.542,1.845)--(6.553,1.839)%
  --(6.563,1.833)--(6.573,1.827)--(6.583,1.822)--(6.594,1.816)--(6.604,1.811)--(6.614,1.805)%
  --(6.624,1.799)--(6.634,1.794)--(6.645,1.789)--(6.655,1.783)--(6.665,1.778)--(6.675,1.773)%
  --(6.685,1.768)--(6.696,1.762)--(6.706,1.757)--(6.716,1.753)--(6.726,1.748)--(6.737,1.743)%
  --(6.747,1.738)--(6.757,1.733)--(6.767,1.728)--(6.777,1.723)--(6.788,1.718)--(6.798,1.713)%
  --(6.808,1.709)--(6.818,1.704)--(6.828,1.699)--(6.839,1.695)--(6.849,1.691)--(6.859,1.686)%
  --(6.869,1.682)--(6.879,1.678)--(6.890,1.673)--(6.900,1.669)--(6.910,1.664)--(6.920,1.660)%
  --(6.931,1.656)--(6.941,1.652)--(6.951,1.648)--(6.961,1.643)--(6.971,1.639)--(6.982,1.636)%
  --(6.992,1.631)--(7.002,1.627)--(7.012,1.624)--(7.022,1.620)--(7.033,1.616)--(7.043,1.612)%
  --(7.053,1.608)--(7.063,1.604)--(7.074,1.600)--(7.084,1.597)--(7.094,1.593)--(7.104,1.590)%
  --(7.114,1.586)--(7.125,1.582)--(7.135,1.579)--(7.145,1.575)--(7.155,1.572)--(7.165,1.568)%
  --(7.176,1.565)--(7.186,1.561)--(7.196,1.558)--(7.206,1.555)--(7.217,1.551)--(7.227,1.548)%
  --(7.237,1.544)--(7.247,1.541)--(7.257,1.538)--(7.268,1.535)--(7.278,1.532)--(7.288,1.529)%
  --(7.298,1.526)--(7.308,1.522)--(7.319,1.519)--(7.329,1.516)--(7.339,1.513)--(7.349,1.510)%
  --(7.360,1.507)--(7.370,1.504)--(7.380,1.501)--(7.390,1.498)--(7.400,1.495)--(7.411,1.492)%
  --(7.421,1.490)--(7.431,1.487)--(7.441,1.484)--(7.451,1.481)--(7.462,1.478)--(7.472,1.476)%
  --(7.482,1.473)--(7.492,1.470)--(7.502,1.467)--(7.513,1.465)--(7.523,1.462)--(7.533,1.459)%
  --(7.543,1.457)--(7.554,1.454)--(7.564,1.452)--(7.574,1.449)--(7.584,1.447)--(7.594,1.444)%
  --(7.605,1.442)--(7.615,1.439)--(7.625,1.437)--(7.635,1.434)--(7.645,1.432)--(7.656,1.429)%
  --(7.666,1.427)--(7.676,1.425)--(7.686,1.422)--(7.697,1.420)--(7.707,1.418)--(7.717,1.415)%
  --(7.727,1.413)--(7.737,1.411)--(7.748,1.409)--(7.758,1.406)--(7.768,1.404)--(7.778,1.402)%
  --(7.788,1.400)--(7.799,1.397)--(7.809,1.395)--(7.819,1.393)--(7.829,1.391)--(7.840,1.389)%
  --(7.850,1.387)--(7.860,1.385)--(7.870,1.383)--(7.880,1.381)--(7.891,1.379)--(7.901,1.376)%
  --(7.911,1.374)--(7.921,1.373)--(7.931,1.371)--(7.942,1.368)--(7.952,1.367)--(7.962,1.365)%
  --(7.972,1.363)--(7.982,1.361)--(7.993,1.359)--(8.003,1.357)--(8.013,1.355)--(8.023,1.353)%
  --(8.034,1.351)--(8.044,1.350)--(8.054,1.348)--(8.064,1.346)--(8.074,1.344)--(8.085,1.342)%
  --(8.095,1.340)--(8.105,1.339)--(8.115,1.337)--(8.125,1.335)--(8.136,1.333)--(8.146,1.332)%
  --(8.156,1.330)--(8.166,1.329)--(8.177,1.327)--(8.187,1.325)--(8.197,1.323)--(8.207,1.322)%
  --(8.217,1.320)--(8.228,1.318)--(8.238,1.317)--(8.248,1.315)--(8.258,1.313)--(8.268,1.312)%
  --(8.279,1.310)--(8.289,1.309)--(8.299,1.307)--(8.309,1.306)--(8.320,1.304)--(8.330,1.303)%
  --(8.340,1.301)--(8.350,1.299)--(8.360,1.298)--(8.371,1.297)--(8.381,1.295)--(8.391,1.294)%
  --(8.401,1.292)--(8.411,1.291)--(8.422,1.289)--(8.432,1.288)--(8.442,1.287)--(8.452,1.285)%
  --(8.463,1.284)--(8.473,1.282)--(8.483,1.281)--(8.493,1.280)--(8.503,1.278)--(8.514,1.277)%
  --(8.524,1.275)--(8.534,1.274)--(8.544,1.273)--(8.554,1.271)--(8.565,1.270)--(8.575,1.269)%
  --(8.585,1.267)--(8.595,1.266)--(8.605,1.265)--(8.616,1.263)--(8.626,1.262)--(8.636,1.261)%
  --(8.646,1.260)--(8.657,1.259)--(8.667,1.257)--(8.677,1.256)--(8.687,1.255)--(8.697,1.254)%
  --(8.708,1.252)--(8.718,1.251)--(8.728,1.250)--(8.738,1.249)--(8.748,1.248)--(8.759,1.246)%
  --(8.769,1.245)--(8.779,1.244)--(8.789,1.243)--(8.800,1.242)--(8.810,1.241)--(8.820,1.240)%
  --(8.830,1.238)--(8.840,1.237)--(8.851,1.236)--(8.861,1.235)--(8.871,1.234)--(8.881,1.233)%
  --(8.891,1.232)--(8.902,1.231)--(8.912,1.229)--(8.922,1.228)--(8.932,1.227)--(8.943,1.226)%
  --(8.953,1.225)--(8.963,1.224)--(8.973,1.223)--(8.983,1.222)--(8.994,1.221)--(9.004,1.220)%
  --(9.014,1.219)--(9.024,1.218)--(9.034,1.217)--(9.045,1.216)--(9.055,1.215)--(9.065,1.214)%
  --(9.075,1.213)--(9.085,1.212)--(9.096,1.211)--(9.106,1.210)--(9.116,1.210)--(9.126,1.208)%
  --(9.137,1.207)--(9.147,1.206)--(9.157,1.206)--(9.167,1.205)--(9.177,1.204)--(9.188,1.203)%
  --(9.198,1.202)--(9.208,1.201)--(9.218,1.200)--(9.228,1.199)--(9.239,1.198)--(9.249,1.197)%
  --(9.259,1.197)--(9.269,1.196)--(9.280,1.195)--(9.290,1.194)--(9.300,1.193)--(9.310,1.192)%
  --(9.320,1.191)--(9.331,1.191)--(9.341,1.190)--(9.351,1.189)--(9.361,1.188)--(9.371,1.187)%
  --(9.382,1.186)--(9.392,1.185)--(9.402,1.185)--(9.412,1.184)--(9.423,1.183)--(9.433,1.182)%
  --(9.443,1.182)--(9.453,1.180)--(9.463,1.180)--(9.474,1.179)--(9.484,1.178)--(9.494,1.178)%
  --(9.504,1.177)--(9.514,1.176)--(9.525,1.175)--(9.535,1.175)--(9.545,1.174)--(9.555,1.173)%
  --(9.565,1.172)--(9.576,1.171)--(9.586,1.171)--(9.596,1.170)--(9.606,1.169)--(9.617,1.169)%
  --(9.627,1.168)--(9.637,1.167)--(9.647,1.166)--(9.657,1.166)--(9.668,1.165)--(9.678,1.164)%
  --(9.688,1.163)--(9.698,1.163)--(9.708,1.162)--(9.719,1.162)--(9.729,1.161)--(9.739,1.160)%
  --(9.749,1.159)--(9.760,1.159)--(9.770,1.158)--(9.780,1.157)--(9.790,1.157)--(9.800,1.156)%
  --(9.811,1.155)--(9.821,1.155)--(9.831,1.154)--(9.841,1.154)--(9.851,1.153)--(9.862,1.152)%
  --(9.872,1.151)--(9.882,1.151)--(9.892,1.150)--(9.903,1.150)--(9.913,1.149)--(9.923,1.148)%
  --(9.933,1.148)--(9.943,1.147)--(9.954,1.147)--(9.964,1.146)--(9.974,1.145)--(9.984,1.145)%
  --(9.994,1.144)--(10.005,1.143)--(10.015,1.143)--(10.025,1.142)--(10.035,1.142)--(10.046,1.141)%
  --(10.056,1.141)--(10.066,1.140)--(10.076,1.140)--(10.086,1.139)--(10.097,1.138)--(10.107,1.138)%
  --(10.117,1.137)--(10.127,1.136)--(10.137,1.136)--(10.148,1.135)--(10.158,1.135)--(10.168,1.134)%
  --(10.178,1.134)--(10.188,1.133)--(10.199,1.133)--(10.209,1.132)--(10.219,1.131)--(10.229,1.131)%
  --(10.240,1.130)--(10.250,1.130)--(10.260,1.129)--(10.270,1.129)--(10.280,1.128)--(10.291,1.128)%
  --(10.301,1.127)--(10.311,1.127)--(10.321,1.126)--(10.331,1.126)--(10.342,1.125)--(10.352,1.125)%
  --(10.362,1.124)--(10.372,1.124)--(10.383,1.123)--(10.393,1.123)--(10.403,1.122)--(10.413,1.122)%
  --(10.423,1.121)--(10.434,1.121)--(10.444,1.120)--(10.454,1.120)--(10.464,1.120)--(10.474,1.119)%
  --(10.485,1.119)--(10.495,1.118)--(10.505,1.117)--(10.515,1.117)--(10.526,1.117)--(10.536,1.116)%
  --(10.546,1.116)--(10.556,1.115)--(10.566,1.115)--(10.577,1.114)--(10.587,1.114)--(10.597,1.113)%
  --(10.607,1.113)--(10.617,1.113)--(10.628,1.112)--(10.638,1.112)--(10.648,1.111)--(10.658,1.111)%
  --(10.668,1.110)--(10.679,1.110)--(10.689,1.109)--(10.699,1.109)--(10.709,1.109)--(10.720,1.108)%
  --(10.730,1.108)--(10.740,1.107)--(10.750,1.107)--(10.760,1.107)--(10.771,1.106)--(10.781,1.106)%
  --(10.791,1.105)--(10.801,1.105)--(10.811,1.105)--(10.822,1.104)--(10.832,1.103)--(10.842,1.103)%
  --(10.852,1.103)--(10.863,1.102)--(10.873,1.102)--(10.883,1.102)--(10.893,1.101)--(10.903,1.101)%
  --(10.914,1.100)--(10.924,1.100)--(10.934,1.100)--(10.944,1.099)--(10.954,1.099)--(10.965,1.099)%
  --(10.975,1.098)--(10.985,1.098)--(10.995,1.098)--(11.006,1.097)--(11.016,1.097)--(11.026,1.096)%
  --(11.036,1.096)--(11.046,1.095)--(11.057,1.095)--(11.067,1.095)--(11.077,1.094)--(11.087,1.094)%
  --(11.097,1.094)--(11.108,1.093)--(11.118,1.093)--(11.128,1.093)--(11.138,1.092)--(11.149,1.092)%
  --(11.159,1.092)--(11.169,1.091)--(11.179,1.091)--(11.189,1.091)--(11.200,1.090)--(11.210,1.090)%
  --(11.220,1.089)--(11.230,1.089)--(11.240,1.089)--(11.251,1.088)--(11.261,1.088)--(11.271,1.088)%
  --(11.281,1.088)--(11.291,1.087)--(11.302,1.087)--(11.312,1.087)--(11.322,1.086)--(11.332,1.086)%
  --(11.343,1.086)--(11.353,1.085)--(11.363,1.085)--(11.373,1.085)--(11.383,1.084)--(11.394,1.084)%
  --(11.404,1.084)--(11.414,1.084)--(11.424,1.083)--(11.434,1.083)--(11.445,1.082)--(11.455,1.082)%
  --(11.465,1.082)--(11.475,1.081)--(11.486,1.081)--(11.496,1.081)--(11.506,1.081)--(11.516,1.080)%
  --(11.526,1.080)--(11.537,1.080)--(11.547,1.079)--(11.557,1.079)--(11.567,1.079)--(11.577,1.079)%
  --(11.588,1.078)--(11.598,1.078)--(11.608,1.078)--(11.618,1.077)--(11.629,1.077)--(11.639,1.077)%
  --(11.649,1.077)--(11.659,1.076)--(11.669,1.076)--(11.680,1.076)--(11.690,1.075)--(11.700,1.075)%
  --(11.710,1.075)--(11.720,1.074)--(11.731,1.074)--(11.741,1.074)--(11.751,1.074);
\gpcolor{color=gp lt color border}
\draw[gp path] (1.504,8.631)--(1.504,0.985)--(13.447,0.985)--(13.447,8.631)--cycle;
%% coordinates of the plot area
\gpdefrectangularnode{gp plot 1}{\pgfpoint{1.504cm}{0.985cm}}{\pgfpoint{13.447cm}{8.631cm}}
\end{tikzpicture}
%% gnuplot variables

	\caption{Gráfico mostrando a massa em função da densidade bariônica para fração de prótons 1/2. Note que $M$ diminui até zero em $\rho \approx 0.35$. Nesse ponto ocorre a restauração da simetria quiral. \protect[Parameters: eNJL1m; Proton fraction: 1/2] }
	\label{Fig:mass_graph_eNJL1m}
\end{figure*}

\begin{figure*}
	\begin{tikzpicture}[gnuplot]
%% generated with GNUPLOT 5.0p2 (Lua 5.2; terminal rev. 99, script rev. 100)
%% Mon Mar  7 16:11:42 2016
\path (0.000,0.000) rectangle (14.000,9.000);
\gpcolor{color=gp lt color border}
\gpsetlinetype{gp lt border}
\gpsetdashtype{gp dt solid}
\gpsetlinewidth{1.00}
\draw[gp path] (1.504,0.985)--(1.684,0.985);
\draw[gp path] (13.447,0.985)--(13.267,0.985);
\node[gp node right] at (1.320,0.985) {$-0.9$};
\draw[gp path] (1.504,1.680)--(1.684,1.680);
\draw[gp path] (13.447,1.680)--(13.267,1.680);
\node[gp node right] at (1.320,1.680) {$-0.8$};
\draw[gp path] (1.504,2.375)--(1.684,2.375);
\draw[gp path] (13.447,2.375)--(13.267,2.375);
\node[gp node right] at (1.320,2.375) {$-0.7$};
\draw[gp path] (1.504,3.070)--(1.684,3.070);
\draw[gp path] (13.447,3.070)--(13.267,3.070);
\node[gp node right] at (1.320,3.070) {$-0.6$};
\draw[gp path] (1.504,3.765)--(1.684,3.765);
\draw[gp path] (13.447,3.765)--(13.267,3.765);
\node[gp node right] at (1.320,3.765) {$-0.5$};
\draw[gp path] (1.504,4.460)--(1.684,4.460);
\draw[gp path] (13.447,4.460)--(13.267,4.460);
\node[gp node right] at (1.320,4.460) {$-0.4$};
\draw[gp path] (1.504,5.156)--(1.684,5.156);
\draw[gp path] (13.447,5.156)--(13.267,5.156);
\node[gp node right] at (1.320,5.156) {$-0.3$};
\draw[gp path] (1.504,5.851)--(1.684,5.851);
\draw[gp path] (13.447,5.851)--(13.267,5.851);
\node[gp node right] at (1.320,5.851) {$-0.2$};
\draw[gp path] (1.504,6.546)--(1.684,6.546);
\draw[gp path] (13.447,6.546)--(13.267,6.546);
\node[gp node right] at (1.320,6.546) {$-0.1$};
\draw[gp path] (1.504,7.241)--(1.684,7.241);
\draw[gp path] (13.447,7.241)--(13.267,7.241);
\node[gp node right] at (1.320,7.241) {$0$};
\draw[gp path] (1.504,7.936)--(1.684,7.936);
\draw[gp path] (13.447,7.936)--(13.267,7.936);
\node[gp node right] at (1.320,7.936) {$0.1$};
\draw[gp path] (1.504,8.631)--(1.684,8.631);
\draw[gp path] (13.447,8.631)--(13.267,8.631);
\node[gp node right] at (1.320,8.631) {$0.2$};
\draw[gp path] (1.504,0.985)--(1.504,1.165);
\draw[gp path] (1.504,8.631)--(1.504,8.451);
\node[gp node center] at (1.504,0.677) {$0$};
\draw[gp path] (3.210,0.985)--(3.210,1.165);
\draw[gp path] (3.210,8.631)--(3.210,8.451);
\node[gp node center] at (3.210,0.677) {$0.5$};
\draw[gp path] (4.916,0.985)--(4.916,1.165);
\draw[gp path] (4.916,8.631)--(4.916,8.451);
\node[gp node center] at (4.916,0.677) {$1$};
\draw[gp path] (6.622,0.985)--(6.622,1.165);
\draw[gp path] (6.622,8.631)--(6.622,8.451);
\node[gp node center] at (6.622,0.677) {$1.5$};
\draw[gp path] (8.329,0.985)--(8.329,1.165);
\draw[gp path] (8.329,8.631)--(8.329,8.451);
\node[gp node center] at (8.329,0.677) {$2$};
\draw[gp path] (10.035,0.985)--(10.035,1.165);
\draw[gp path] (10.035,8.631)--(10.035,8.451);
\node[gp node center] at (10.035,0.677) {$2.5$};
\draw[gp path] (11.741,0.985)--(11.741,1.165);
\draw[gp path] (11.741,8.631)--(11.741,8.451);
\node[gp node center] at (11.741,0.677) {$3$};
\draw[gp path] (13.447,0.985)--(13.447,1.165);
\draw[gp path] (13.447,8.631)--(13.447,8.451);
\node[gp node center] at (13.447,0.677) {$3.5$};
\draw[gp path] (1.504,8.631)--(1.504,0.985)--(13.447,0.985)--(13.447,8.631)--cycle;
\node[gp node center,rotate=-270] at (0.246,4.808) {$\rho_s$ ($\rm{fm}^{-3}$)};
\node[gp node center] at (7.475,0.215) {$\rho$ ($\rm{fm}^{-3}$)};
\gpcolor{rgb color={0.580,0.000,0.827}}
\draw[gp path] (1.548,1.111)--(1.559,1.133)--(1.569,1.155)--(1.579,1.177)--(1.589,1.199)%
  --(1.599,1.221)--(1.610,1.243)--(1.620,1.265)--(1.630,1.287)--(1.640,1.309)--(1.650,1.330)%
  --(1.661,1.352)--(1.671,1.374)--(1.681,1.396)--(1.691,1.417)--(1.702,1.439)--(1.712,1.461)%
  --(1.722,1.482)--(1.732,1.504)--(1.742,1.525)--(1.753,1.547)--(1.763,1.568)--(1.773,1.590)%
  --(1.783,1.611)--(1.793,1.633)--(1.804,1.654)--(1.814,1.675)--(1.824,1.697)--(1.834,1.718)%
  --(1.845,1.740)--(1.855,1.761)--(1.865,1.782)--(1.875,1.803)--(1.885,1.824)--(1.896,1.846)%
  --(1.906,1.867)--(1.916,1.888)--(1.926,1.909)--(1.936,1.930)--(1.947,1.951)--(1.957,1.972)%
  --(1.967,1.993)--(1.977,2.014)--(1.987,2.035)--(1.998,2.056)--(2.008,2.077)--(2.018,2.098)%
  --(2.028,2.119)--(2.039,2.140)--(2.049,2.160)--(2.059,2.181)--(2.069,2.202)--(2.079,2.222)%
  --(2.090,2.243)--(2.100,2.264)--(2.110,2.285)--(2.120,2.305)--(2.130,2.326)--(2.141,2.347)%
  --(2.151,2.367)--(2.161,2.388)--(2.171,2.408)--(2.182,2.429)--(2.192,2.449)--(2.202,2.470)%
  --(2.212,2.491)--(2.222,2.511)--(2.233,2.532)--(2.243,2.552)--(2.253,2.572)--(2.263,2.593)%
  --(2.273,2.613)--(2.284,2.634)--(2.294,2.654)--(2.304,2.674)--(2.314,2.695)--(2.325,2.715)%
  --(2.335,2.735)--(2.345,2.755)--(2.355,2.776)--(2.365,2.796)--(2.376,2.816)--(2.386,2.836)%
  --(2.396,2.857)--(2.406,2.877)--(2.416,2.897)--(2.427,2.917)--(2.437,2.937)--(2.447,2.957)%
  --(2.457,2.977)--(2.468,2.997)--(2.478,3.017)--(2.488,3.038)--(2.498,3.058)--(2.508,3.078)%
  --(2.519,3.098)--(2.529,3.118)--(2.539,3.137)--(2.549,3.158)--(2.559,3.178)--(2.570,3.198)%
  --(2.580,3.217)--(2.590,3.237)--(2.600,3.257)--(2.610,3.277)--(2.621,3.297)--(2.631,3.317)%
  --(2.641,3.337)--(2.651,3.357)--(2.662,3.377)--(2.672,3.397)--(2.682,3.417)--(2.692,3.436)%
  --(2.702,3.456)--(2.713,3.476)--(2.723,3.495)--(2.733,3.515)--(2.743,3.535)--(2.753,3.555)%
  --(2.764,3.575)--(2.774,3.594)--(2.784,3.614)--(2.794,3.634)--(2.805,3.654)--(2.815,3.673)%
  --(2.825,3.693)--(2.835,3.713)--(2.845,3.733)--(2.856,3.752)--(2.866,3.772)--(2.876,3.792)%
  --(2.886,3.811)--(2.896,3.831)--(2.907,3.850)--(2.917,3.870)--(2.927,3.890)--(2.937,3.909)%
  --(2.948,3.929)--(2.958,3.949)--(2.968,3.968)--(2.978,3.988)--(2.988,4.007)--(2.999,4.027)%
  --(3.009,4.047)--(3.019,4.066)--(3.029,4.086)--(3.039,4.106)--(3.050,4.125)--(3.060,4.145)%
  --(3.070,4.164)--(3.080,4.183)--(3.090,4.203)--(3.101,4.223)--(3.111,4.242)--(3.121,4.262)%
  --(3.131,4.281)--(3.142,4.301)--(3.152,4.320)--(3.162,4.340)--(3.172,4.359)--(3.182,4.379)%
  --(3.193,4.398)--(3.203,4.418)--(3.213,4.437)--(3.223,4.457)--(3.233,4.476)--(3.244,4.495)%
  --(3.254,4.515)--(3.264,4.535)--(3.274,4.554)--(3.285,4.573)--(3.295,4.593)--(3.305,4.612)%
  --(3.315,4.632)--(3.325,4.651)--(3.336,4.671)--(3.346,4.690)--(3.356,4.709)--(3.366,4.729)%
  --(3.376,4.748)--(3.387,4.768)--(3.397,4.787)--(3.407,4.806)--(3.417,4.826)--(3.428,4.845)%
  --(3.438,4.864)--(3.448,4.884)--(3.458,4.903)--(3.468,4.923)--(3.479,4.942)--(3.489,4.961)%
  --(3.499,4.980)--(3.509,5.000)--(3.519,5.019)--(3.530,5.038)--(3.540,5.058)--(3.550,5.077)%
  --(3.560,5.096)--(3.570,5.116)--(3.581,5.135)--(3.591,5.154)--(3.601,5.174)--(3.611,5.193)%
  --(3.622,5.212)--(3.632,5.231)--(3.642,5.251)--(3.652,5.270)--(3.662,5.289)--(3.673,5.308)%
  --(3.683,5.327)--(3.693,5.347)--(3.703,5.366)--(3.713,5.385)--(3.724,5.404)--(3.734,5.424)%
  --(3.744,5.443)--(3.754,5.462)--(3.765,5.481)--(3.775,5.500)--(3.785,5.519)--(3.795,5.538)%
  --(3.805,5.557)--(3.816,5.576)--(3.826,5.595)--(3.836,5.615)--(3.846,5.634)--(3.856,5.653)%
  --(3.867,5.672)--(3.877,5.691)--(3.887,5.710)--(3.897,5.729)--(3.908,5.748)--(3.918,5.767)%
  --(3.928,5.786)--(3.938,5.805)--(3.948,5.823)--(3.959,5.842)--(3.969,5.861)--(3.979,5.880)%
  --(3.989,5.899)--(3.999,5.918)--(4.010,5.937)--(4.020,5.955)--(4.030,5.974)--(4.040,5.993)%
  --(4.051,6.012)--(4.061,6.030)--(4.071,6.049)--(4.081,6.068)--(4.091,6.086)--(4.102,6.105)%
  --(4.112,6.123)--(4.122,6.142)--(4.132,6.161)--(4.142,6.179)--(4.153,6.197)--(4.163,6.216)%
  --(4.173,6.234)--(4.183,6.253)--(4.193,6.271)--(4.204,6.289)--(4.214,6.308)--(4.224,6.326)%
  --(4.234,6.344)--(4.245,6.362)--(4.255,6.381)--(4.265,6.399)--(4.275,6.417)--(4.285,6.435)%
  --(4.296,6.453)--(4.306,6.471)--(4.316,6.489)--(4.326,6.507)--(4.336,6.524)--(4.347,6.542)%
  --(4.357,6.560)--(4.367,6.577)--(4.377,6.595)--(4.388,6.613)--(4.398,6.630)--(4.408,6.648)%
  --(4.418,6.665)--(4.428,6.682)--(4.439,6.700)--(4.449,6.717)--(4.459,6.734)--(4.469,6.751)%
  --(4.479,6.768)--(4.490,6.785)--(4.500,6.802)--(4.510,6.819)--(4.520,6.836)--(4.531,6.853)%
  --(4.541,6.869)--(4.551,6.886)--(4.561,6.902)--(4.571,6.919)--(4.582,6.935)--(4.592,6.951)%
  --(4.602,6.967)--(4.612,6.983)--(4.622,6.999)--(4.633,7.015)--(4.643,7.031)--(4.653,7.047)%
  --(4.663,7.062)--(4.673,7.078)--(4.684,7.093)--(4.694,7.109)--(4.704,7.124)--(4.714,7.139)%
  --(4.725,7.154)--(4.735,7.169)--(4.745,7.183)--(4.755,7.198)--(4.765,7.213)--(4.776,7.227)%
  --(4.786,7.241)--(4.796,7.256)--(4.806,7.270)--(4.816,7.284)--(4.827,7.297)--(4.837,7.311)%
  --(4.847,7.325)--(4.857,7.338)--(4.868,7.351)--(4.878,7.365)--(4.888,7.378)--(4.898,7.391)%
  --(4.908,7.403)--(4.919,7.416)--(4.929,7.428)--(4.939,7.441)--(4.949,7.453)--(4.959,7.465)%
  --(4.970,7.477)--(4.980,7.489)--(4.990,7.500)--(5.000,7.512)--(5.011,7.523)--(5.021,7.534)%
  --(5.031,7.545)--(5.041,7.556)--(5.051,7.567)--(5.062,7.578)--(5.072,7.588)--(5.082,7.598)%
  --(5.092,7.608)--(5.102,7.618)--(5.113,7.628)--(5.123,7.638)--(5.133,7.647)--(5.143,7.657)%
  --(5.154,7.666)--(5.164,7.675)--(5.174,7.684)--(5.184,7.693)--(5.194,7.701)--(5.205,7.710)%
  --(5.215,7.718)--(5.225,7.726)--(5.235,7.735)--(5.245,7.742)--(5.256,7.750)--(5.266,7.758)%
  --(5.276,7.765)--(5.286,7.772)--(5.296,7.780)--(5.307,7.787)--(5.317,7.793)--(5.327,7.800)%
  --(5.337,7.807)--(5.348,7.813)--(5.358,7.820)--(5.368,7.826)--(5.378,7.832)--(5.388,7.838)%
  --(5.399,7.844)--(5.409,7.849)--(5.419,7.855)--(5.429,7.860)--(5.439,7.866)--(5.450,7.871)%
  --(5.460,7.876)--(5.470,7.881)--(5.480,7.886)--(5.491,7.891)--(5.501,7.895)--(5.511,7.900)%
  --(5.521,7.904)--(5.531,7.908)--(5.542,7.912)--(5.552,7.917)--(5.562,7.920)--(5.572,7.924)%
  --(5.582,7.928)--(5.593,7.932)--(5.603,7.935)--(5.613,7.939)--(5.623,7.942)--(5.634,7.945)%
  --(5.644,7.948)--(5.654,7.952)--(5.664,7.955)--(5.674,7.958)--(5.685,7.960)--(5.695,7.963)%
  --(5.705,7.966)--(5.715,7.968)--(5.725,7.971)--(5.736,7.973)--(5.746,7.975)--(5.756,7.978)%
  --(5.766,7.980)--(5.776,7.982)--(5.787,7.984)--(5.797,7.986)--(5.807,7.988)--(5.817,7.990)%
  --(5.828,7.991)--(5.838,7.993)--(5.848,7.995)--(5.858,7.996)--(5.868,7.998)--(5.879,7.999)%
  --(5.889,8.001)--(5.899,8.002)--(5.909,8.003)--(5.919,8.004)--(5.930,8.006)--(5.940,8.007)%
  --(5.950,8.008)--(5.960,8.009)--(5.971,8.010)--(5.981,8.011)--(5.991,8.011)--(6.001,8.012)%
  --(6.011,8.013)--(6.022,8.013)--(6.032,8.014)--(6.042,8.015)--(6.052,8.015)--(6.062,8.016)%
  --(6.073,8.016)--(6.083,8.017)--(6.093,8.017)--(6.103,8.017)--(6.114,8.018)--(6.124,8.018)%
  --(6.134,8.018)--(6.144,8.018)--(6.154,8.019)--(6.165,8.019)--(6.175,8.019)--(6.185,8.019)%
  --(6.195,8.019)--(6.205,8.019)--(6.216,8.018)--(6.226,8.019)--(6.236,8.018)--(6.246,8.018)%
  --(6.257,8.018)--(6.267,8.018)--(6.277,8.018)--(6.287,8.017)--(6.297,8.017)--(6.308,8.017)%
  --(6.318,8.016)--(6.328,8.016)--(6.338,8.015)--(6.348,8.015)--(6.359,8.015)--(6.369,8.014)%
  --(6.379,8.013)--(6.389,8.013)--(6.399,8.013)--(6.410,8.012)--(6.420,8.011)--(6.430,8.011)%
  --(6.440,8.010)--(6.451,8.009)--(6.461,8.009)--(6.471,8.008)--(6.481,8.007)--(6.491,8.007)%
  --(6.502,8.006)--(6.512,8.005)--(6.522,8.005)--(6.532,8.004)--(6.542,8.003)--(6.553,8.002)%
  --(6.563,8.001)--(6.573,8.001)--(6.583,8.000)--(6.594,7.999)--(6.604,7.998)--(6.614,7.997)%
  --(6.624,7.996)--(6.634,7.995)--(6.645,7.994)--(6.655,7.993)--(6.665,7.993)--(6.675,7.991)%
  --(6.685,7.991)--(6.696,7.989)--(6.706,7.988)--(6.716,7.987)--(6.726,7.986)--(6.737,7.985)%
  --(6.747,7.984)--(6.757,7.983)--(6.767,7.982)--(6.777,7.981)--(6.788,7.980)--(6.798,7.978)%
  --(6.808,7.978)--(6.818,7.976)--(6.828,7.975)--(6.839,7.974)--(6.849,7.973)--(6.859,7.972)%
  --(6.869,7.971)--(6.879,7.970)--(6.890,7.968)--(6.900,7.967)--(6.910,7.966)--(6.920,7.965)%
  --(6.931,7.964)--(6.941,7.962)--(6.951,7.961)--(6.961,7.960)--(6.971,7.958)--(6.982,7.958)%
  --(6.992,7.956)--(7.002,7.955)--(7.012,7.954)--(7.022,7.953)--(7.033,7.952)--(7.043,7.950)%
  --(7.053,7.949)--(7.063,7.948)--(7.074,7.946)--(7.084,7.945)--(7.094,7.944)--(7.104,7.943)%
  --(7.114,7.941)--(7.125,7.940)--(7.135,7.939)--(7.145,7.937)--(7.155,7.936)--(7.165,7.935)%
  --(7.176,7.934)--(7.186,7.932)--(7.196,7.931)--(7.206,7.930)--(7.217,7.928)--(7.227,7.927)%
  --(7.237,7.925)--(7.247,7.924)--(7.257,7.923)--(7.268,7.922)--(7.278,7.921)--(7.288,7.919)%
  --(7.298,7.918)--(7.308,7.917)--(7.319,7.915)--(7.329,7.914)--(7.339,7.912)--(7.349,7.911)%
  --(7.360,7.910)--(7.370,7.909)--(7.380,7.907)--(7.390,7.906)--(7.400,7.905)--(7.411,7.903)%
  --(7.421,7.902)--(7.431,7.900)--(7.441,7.899)--(7.451,7.898)--(7.462,7.896)--(7.472,7.895)%
  --(7.482,7.894)--(7.492,7.893)--(7.502,7.891)--(7.513,7.890)--(7.523,7.889)--(7.533,7.887)%
  --(7.543,7.886)--(7.554,7.884)--(7.564,7.883)--(7.574,7.882)--(7.584,7.881)--(7.594,7.879)%
  --(7.605,7.878)--(7.615,7.876)--(7.625,7.875)--(7.635,7.874)--(7.645,7.872)--(7.656,7.871)%
  --(7.666,7.870)--(7.676,7.869)--(7.686,7.867)--(7.697,7.866)--(7.707,7.865)--(7.717,7.863)%
  --(7.727,7.862)--(7.737,7.861)--(7.748,7.860)--(7.758,7.858)--(7.768,7.857)--(7.778,7.855)%
  --(7.788,7.854)--(7.799,7.853)--(7.809,7.852)--(7.819,7.850)--(7.829,7.849)--(7.840,7.848)%
  --(7.850,7.846)--(7.860,7.845)--(7.870,7.844)--(7.880,7.843)--(7.891,7.841)--(7.901,7.840)%
  --(7.911,7.838)--(7.921,7.837)--(7.931,7.836)--(7.942,7.835)--(7.952,7.834)--(7.962,7.832)%
  --(7.972,7.831)--(7.982,7.830)--(7.993,7.828)--(8.003,7.827)--(8.013,7.826)--(8.023,7.825)%
  --(8.034,7.823)--(8.044,7.822)--(8.054,7.821)--(8.064,7.820)--(8.074,7.818)--(8.085,7.817)%
  --(8.095,7.816)--(8.105,7.814)--(8.115,7.813)--(8.125,7.812)--(8.136,7.811)--(8.146,7.809)%
  --(8.156,7.808)--(8.166,7.807)--(8.177,7.806)--(8.187,7.805)--(8.197,7.803)--(8.207,7.802)%
  --(8.217,7.801)--(8.228,7.800)--(8.238,7.798)--(8.248,7.797)--(8.258,7.796)--(8.268,7.795)%
  --(8.279,7.793)--(8.289,7.793)--(8.299,7.791)--(8.309,7.790)--(8.320,7.788)--(8.330,7.787)%
  --(8.340,7.787)--(8.350,7.785)--(8.360,7.784)--(8.371,7.783)--(8.381,7.782)--(8.391,7.780)%
  --(8.401,7.779)--(8.411,7.778)--(8.422,7.777)--(8.432,7.776)--(8.442,7.775)--(8.452,7.774)%
  --(8.463,7.772)--(8.473,7.771)--(8.483,7.770)--(8.493,7.769)--(8.503,7.768)--(8.514,7.766)%
  --(8.524,7.765)--(8.534,7.764)--(8.544,7.763)--(8.554,7.762)--(8.565,7.761)--(8.575,7.759)%
  --(8.585,7.758)--(8.595,7.757)--(8.605,7.756)--(8.616,7.755)--(8.626,7.754)--(8.636,7.753)%
  --(8.646,7.751)--(8.657,7.751)--(8.667,7.749)--(8.677,7.748)--(8.687,7.747)--(8.697,7.746)%
  --(8.708,7.745)--(8.718,7.744)--(8.728,7.742)--(8.738,7.742)--(8.748,7.741)--(8.759,7.739)%
  --(8.769,7.738)--(8.779,7.738)--(8.789,7.736)--(8.800,7.735)--(8.810,7.734)--(8.820,7.733)%
  --(8.830,7.732)--(8.840,7.731)--(8.851,7.730)--(8.861,7.729)--(8.871,7.728)--(8.881,7.727)%
  --(8.891,7.725)--(8.902,7.724)--(8.912,7.723)--(8.922,7.722)--(8.932,7.721)--(8.943,7.720)%
  --(8.953,7.719)--(8.963,7.718)--(8.973,7.717)--(8.983,7.716)--(8.994,7.715)--(9.004,7.714)%
  --(9.014,7.713)--(9.024,7.712)--(9.034,7.711)--(9.045,7.710)--(9.055,7.709)--(9.065,7.708)%
  --(9.075,7.707)--(9.085,7.706)--(9.096,7.705)--(9.106,7.704)--(9.116,7.703)--(9.126,7.702)%
  --(9.137,7.701)--(9.147,7.700)--(9.157,7.699)--(9.167,7.698)--(9.177,7.697)--(9.188,7.696)%
  --(9.198,7.695)--(9.208,7.694)--(9.218,7.693)--(9.228,7.692)--(9.239,7.691)--(9.249,7.690)%
  --(9.259,7.689)--(9.269,7.688)--(9.280,7.688)--(9.290,7.686)--(9.300,7.686)--(9.310,7.684)%
  --(9.320,7.684)--(9.331,7.683)--(9.341,7.682)--(9.351,7.681)--(9.361,7.680)--(9.371,7.679)%
  --(9.382,7.678)--(9.392,7.677)--(9.402,7.676)--(9.412,7.675)--(9.423,7.674)--(9.433,7.673)%
  --(9.443,7.673)--(9.453,7.671)--(9.463,7.670)--(9.474,7.670)--(9.484,7.669)--(9.494,7.668)%
  --(9.504,7.667)--(9.514,7.666)--(9.525,7.665)--(9.535,7.665)--(9.545,7.664)--(9.555,7.662)%
  --(9.565,7.661)--(9.576,7.661)--(9.586,7.660)--(9.596,7.659)--(9.606,7.658)--(9.617,7.657)%
  --(9.627,7.657)--(9.637,7.656)--(9.647,7.655)--(9.657,7.654)--(9.668,7.653)--(9.678,7.652)%
  --(9.688,7.651)--(9.698,7.650)--(9.708,7.650)--(9.719,7.649)--(9.729,7.648)--(9.739,7.647)%
  --(9.749,7.646)--(9.760,7.645)--(9.770,7.645)--(9.780,7.644)--(9.790,7.643)--(9.800,7.642)%
  --(9.811,7.641)--(9.821,7.640)--(9.831,7.639)--(9.841,7.639)--(9.851,7.638)--(9.862,7.637)%
  --(9.872,7.636)--(9.882,7.635)--(9.892,7.635)--(9.903,7.634)--(9.913,7.633)--(9.923,7.632)%
  --(9.933,7.631)--(9.943,7.631)--(9.954,7.630)--(9.964,7.629)--(9.974,7.628)--(9.984,7.628)%
  --(9.994,7.626)--(10.005,7.625)--(10.015,7.625)--(10.025,7.624)--(10.035,7.623)--(10.046,7.622)%
  --(10.056,7.622)--(10.066,7.621)--(10.076,7.621)--(10.086,7.619)--(10.097,7.618)--(10.107,7.618)%
  --(10.117,7.617)--(10.127,7.616)--(10.137,7.616)--(10.148,7.615)--(10.158,7.614)--(10.168,7.613)%
  --(10.178,7.612)--(10.188,7.612)--(10.199,7.611)--(10.209,7.610)--(10.219,7.609)--(10.229,7.609)%
  --(10.240,7.608)--(10.250,7.608)--(10.260,7.606)--(10.270,7.606)--(10.280,7.605)--(10.291,7.605)%
  --(10.301,7.603)--(10.311,7.603)--(10.321,7.602)--(10.331,7.602)--(10.342,7.600)--(10.352,7.600)%
  --(10.362,7.600)--(10.372,7.598)--(10.383,7.598)--(10.393,7.597)--(10.403,7.597)--(10.413,7.595)%
  --(10.423,7.595)--(10.434,7.595)--(10.444,7.593)--(10.454,7.593)--(10.464,7.593)--(10.474,7.591)%
  --(10.485,7.591)--(10.495,7.591)--(10.505,7.589)--(10.515,7.589)--(10.526,7.588)--(10.536,7.587)%
  --(10.546,7.587)--(10.556,7.586)--(10.566,7.585)--(10.577,7.585)--(10.587,7.584)--(10.597,7.583)%
  --(10.607,7.582)--(10.617,7.582)--(10.628,7.582)--(10.638,7.580)--(10.648,7.580)--(10.658,7.579)%
  --(10.668,7.579)--(10.679,7.578)--(10.689,7.577)--(10.699,7.577)--(10.709,7.576)--(10.720,7.575)%
  --(10.730,7.574)--(10.740,7.574)--(10.750,7.573)--(10.760,7.573)--(10.771,7.572)--(10.781,7.571)%
  --(10.791,7.571)--(10.801,7.570)--(10.811,7.570)--(10.822,7.569)--(10.832,7.568)--(10.842,7.567)%
  --(10.852,7.567)--(10.863,7.566)--(10.873,7.566)--(10.883,7.565)--(10.893,7.565)--(10.903,7.564)%
  --(10.914,7.563)--(10.924,7.562)--(10.934,7.562)--(10.944,7.561)--(10.954,7.561)--(10.965,7.560)%
  --(10.975,7.560)--(10.985,7.559)--(10.995,7.559)--(11.006,7.558)--(11.016,7.557)--(11.026,7.556)%
  --(11.036,7.555)--(11.046,7.555)--(11.057,7.554)--(11.067,7.554)--(11.077,7.553)--(11.087,7.553)%
  --(11.097,7.552)--(11.108,7.551)--(11.118,7.551)--(11.128,7.550)--(11.138,7.550)--(11.149,7.549)%
  --(11.159,7.549)--(11.169,7.548)--(11.179,7.547)--(11.189,7.547)--(11.200,7.546)--(11.210,7.546)%
  --(11.220,7.545)--(11.230,7.544)--(11.240,7.544)--(11.251,7.543)--(11.261,7.543)--(11.271,7.542)%
  --(11.281,7.542)--(11.291,7.542)--(11.302,7.541)--(11.312,7.540)--(11.322,7.540)--(11.332,7.539)%
  --(11.343,7.539)--(11.353,7.538)--(11.363,7.537)--(11.373,7.537)--(11.383,7.536)--(11.394,7.535)%
  --(11.404,7.535)--(11.414,7.535)--(11.424,7.534)--(11.434,7.534)--(11.445,7.533)--(11.455,7.532)%
  --(11.465,7.532)--(11.475,7.531)--(11.486,7.530)--(11.496,7.531)--(11.506,7.530)--(11.516,7.529)%
  --(11.526,7.529)--(11.537,7.528)--(11.547,7.527)--(11.557,7.527)--(11.567,7.527)--(11.577,7.526)%
  --(11.588,7.526)--(11.598,7.525)--(11.608,7.524)--(11.618,7.523)--(11.629,7.524)--(11.639,7.523)%
  --(11.649,7.522)--(11.659,7.522)--(11.669,7.521)--(11.680,7.521)--(11.690,7.521)--(11.700,7.520)%
  --(11.710,7.519)--(11.720,7.518)--(11.731,7.519)--(11.741,7.518)--(11.751,7.517);
\gpcolor{color=gp lt color border}
\draw[gp path] (1.504,8.631)--(1.504,0.985)--(13.447,0.985)--(13.447,8.631)--cycle;
%% coordinates of the plot area
\gpdefrectangularnode{gp plot 1}{\pgfpoint{1.504cm}{0.985cm}}{\pgfpoint{13.447cm}{8.631cm}}
\end{tikzpicture}
%% gnuplot variables

	\caption{Gráfico da densidade escalar em função da densidade bariônica para fração de prótons 1/2. O resultado mostrado nesse gráfico é obtido juntamente com os resultados mostrados na Fig.~\ref{Fig:mass_graph} através da solução da Equação~\ref{Eq:Gap_zero}. \protect[Parameters: eNJL1m; Proton fraction: 1/2] }
	\label{Fig:scalar_density_graph_eNJL1m}
\end{figure*}

\begin{figure*}
	\begin{tikzpicture}[gnuplot]
%% generated with GNUPLOT 5.0p2 (Lua 5.2; terminal rev. 99, script rev. 100)
%% Fri Mar  4 16:19:36 2016
\path (0.000,0.000) rectangle (14.000,9.000);
\gpcolor{color=gp lt color border}
\gpsetlinetype{gp lt border}
\gpsetdashtype{gp dt solid}
\gpsetlinewidth{1.00}
\draw[gp path] (1.504,0.985)--(1.684,0.985);
\draw[gp path] (13.447,0.985)--(13.267,0.985);
\node[gp node right] at (1.320,0.985) {$800$};
\draw[gp path] (1.504,1.680)--(1.684,1.680);
\draw[gp path] (13.447,1.680)--(13.267,1.680);
\node[gp node right] at (1.320,1.680) {$1000$};
\draw[gp path] (1.504,2.375)--(1.684,2.375);
\draw[gp path] (13.447,2.375)--(13.267,2.375);
\node[gp node right] at (1.320,2.375) {$1200$};
\draw[gp path] (1.504,3.070)--(1.684,3.070);
\draw[gp path] (13.447,3.070)--(13.267,3.070);
\node[gp node right] at (1.320,3.070) {$1400$};
\draw[gp path] (1.504,3.765)--(1.684,3.765);
\draw[gp path] (13.447,3.765)--(13.267,3.765);
\node[gp node right] at (1.320,3.765) {$1600$};
\draw[gp path] (1.504,4.460)--(1.684,4.460);
\draw[gp path] (13.447,4.460)--(13.267,4.460);
\node[gp node right] at (1.320,4.460) {$1800$};
\draw[gp path] (1.504,5.156)--(1.684,5.156);
\draw[gp path] (13.447,5.156)--(13.267,5.156);
\node[gp node right] at (1.320,5.156) {$2000$};
\draw[gp path] (1.504,5.851)--(1.684,5.851);
\draw[gp path] (13.447,5.851)--(13.267,5.851);
\node[gp node right] at (1.320,5.851) {$2200$};
\draw[gp path] (1.504,6.546)--(1.684,6.546);
\draw[gp path] (13.447,6.546)--(13.267,6.546);
\node[gp node right] at (1.320,6.546) {$2400$};
\draw[gp path] (1.504,7.241)--(1.684,7.241);
\draw[gp path] (13.447,7.241)--(13.267,7.241);
\node[gp node right] at (1.320,7.241) {$2600$};
\draw[gp path] (1.504,7.936)--(1.684,7.936);
\draw[gp path] (13.447,7.936)--(13.267,7.936);
\node[gp node right] at (1.320,7.936) {$2800$};
\draw[gp path] (1.504,8.631)--(1.684,8.631);
\draw[gp path] (13.447,8.631)--(13.267,8.631);
\node[gp node right] at (1.320,8.631) {$3000$};
\draw[gp path] (1.504,0.985)--(1.504,1.165);
\draw[gp path] (1.504,8.631)--(1.504,8.451);
\node[gp node center] at (1.504,0.677) {$0$};
\draw[gp path] (3.210,0.985)--(3.210,1.165);
\draw[gp path] (3.210,8.631)--(3.210,8.451);
\node[gp node center] at (3.210,0.677) {$0.5$};
\draw[gp path] (4.916,0.985)--(4.916,1.165);
\draw[gp path] (4.916,8.631)--(4.916,8.451);
\node[gp node center] at (4.916,0.677) {$1$};
\draw[gp path] (6.622,0.985)--(6.622,1.165);
\draw[gp path] (6.622,8.631)--(6.622,8.451);
\node[gp node center] at (6.622,0.677) {$1.5$};
\draw[gp path] (8.329,0.985)--(8.329,1.165);
\draw[gp path] (8.329,8.631)--(8.329,8.451);
\node[gp node center] at (8.329,0.677) {$2$};
\draw[gp path] (10.035,0.985)--(10.035,1.165);
\draw[gp path] (10.035,8.631)--(10.035,8.451);
\node[gp node center] at (10.035,0.677) {$2.5$};
\draw[gp path] (11.741,0.985)--(11.741,1.165);
\draw[gp path] (11.741,8.631)--(11.741,8.451);
\node[gp node center] at (11.741,0.677) {$3$};
\draw[gp path] (13.447,0.985)--(13.447,1.165);
\draw[gp path] (13.447,8.631)--(13.447,8.451);
\node[gp node center] at (13.447,0.677) {$3.5$};
\draw[gp path] (1.504,8.631)--(1.504,0.985)--(13.447,0.985)--(13.447,8.631)--cycle;
\node[gp node center,rotate=-270] at (0.246,4.808) {$\mu$ (MeV)};
\node[gp node center] at (7.475,0.215) {$\rho$ ($\rm{fm}^{-3}$)};
\node[gp node left] at (2.972,8.297) {$\mu_p$};
\gpcolor{rgb color={0.580,0.000,0.827}}
\draw[gp path] (1.872,8.297)--(2.788,8.297);
\draw[gp path] (1.559,1.454)--(1.569,1.449)--(1.579,1.446)--(1.589,1.443)--(1.599,1.438)%
  --(1.610,1.435)--(1.620,1.431)--(1.630,1.428)--(1.640,1.425)--(1.650,1.422)--(1.661,1.419)%
  --(1.671,1.416)--(1.681,1.413)--(1.691,1.411)--(1.702,1.408)--(1.712,1.407)--(1.722,1.404)%
  --(1.732,1.402)--(1.742,1.401)--(1.753,1.399)--(1.763,1.398)--(1.773,1.396)--(1.783,1.395)%
  --(1.793,1.393)--(1.804,1.393)--(1.814,1.392)--(1.824,1.392)--(1.834,1.390)--(1.845,1.390)%
  --(1.855,1.390)--(1.865,1.390)--(1.875,1.390)--(1.885,1.391)--(1.896,1.391)--(1.906,1.391)%
  --(1.916,1.392)--(1.926,1.392)--(1.936,1.393)--(1.947,1.394)--(1.957,1.396)--(1.967,1.397)%
  --(1.977,1.398)--(1.987,1.399)--(1.998,1.401)--(2.008,1.403)--(2.018,1.405)--(2.028,1.407)%
  --(2.039,1.408)--(2.049,1.411)--(2.059,1.414)--(2.069,1.416)--(2.079,1.419)--(2.090,1.421)%
  --(2.100,1.424)--(2.110,1.428)--(2.120,1.430)--(2.130,1.433)--(2.141,1.437)--(2.151,1.439)%
  --(2.161,1.443)--(2.171,1.447)--(2.182,1.451)--(2.192,1.455)--(2.202,1.459)--(2.212,1.462)%
  --(2.222,1.466)--(2.233,1.470)--(2.243,1.474)--(2.253,1.479)--(2.263,1.483)--(2.273,1.488)%
  --(2.284,1.492)--(2.294,1.497)--(2.304,1.502)--(2.314,1.507)--(2.325,1.512)--(2.335,1.517)%
  --(2.345,1.523)--(2.355,1.528)--(2.365,1.533)--(2.376,1.538)--(2.386,1.544)--(2.396,1.549)%
  --(2.406,1.554)--(2.416,1.561)--(2.427,1.567)--(2.437,1.572)--(2.447,1.578)--(2.457,1.585)%
  --(2.468,1.591)--(2.478,1.597)--(2.488,1.603)--(2.498,1.609)--(2.508,1.616)--(2.519,1.622)%
  --(2.529,1.628)--(2.539,1.635)--(2.549,1.642)--(2.559,1.648)--(2.570,1.654)--(2.580,1.662)%
  --(2.590,1.668)--(2.600,1.675)--(2.610,1.682)--(2.621,1.689)--(2.631,1.696)--(2.641,1.703)%
  --(2.651,1.711)--(2.662,1.718)--(2.672,1.725)--(2.682,1.732)--(2.692,1.740)--(2.702,1.747)%
  --(2.713,1.754)--(2.723,1.762)--(2.733,1.769)--(2.743,1.777)--(2.753,1.784)--(2.764,1.792)%
  --(2.774,1.800)--(2.784,1.807)--(2.794,1.815)--(2.805,1.822)--(2.815,1.831)--(2.825,1.838)%
  --(2.835,1.846)--(2.845,1.854)--(2.856,1.861)--(2.866,1.869)--(2.876,1.877)--(2.886,1.886)%
  --(2.896,1.894)--(2.907,1.902)--(2.917,1.910)--(2.927,1.918)--(2.937,1.926)--(2.948,1.934)%
  --(2.958,1.942)--(2.968,1.950)--(2.978,1.958)--(2.988,1.966)--(2.999,1.974)--(3.009,1.983)%
  --(3.019,1.991)--(3.029,1.999)--(3.039,2.006)--(3.050,2.015)--(3.060,2.023)--(3.070,2.032)%
  --(3.080,2.040)--(3.090,2.047)--(3.101,2.056)--(3.111,2.064)--(3.121,2.073)--(3.131,2.080)%
  --(3.142,2.089)--(3.152,2.097)--(3.162,2.105)--(3.172,2.114)--(3.182,2.121)--(3.193,2.130)%
  --(3.203,2.138)--(3.213,2.146)--(3.223,2.154)--(3.233,2.162)--(3.244,2.171)--(3.254,2.179)%
  --(3.264,2.186)--(3.274,2.195)--(3.285,2.203)--(3.295,2.211)--(3.305,2.219)--(3.315,2.226)%
  --(3.325,2.234)--(3.336,2.242)--(3.346,2.250)--(3.356,2.258)--(3.366,2.266)--(3.376,2.274)%
  --(3.387,2.282)--(3.397,2.290)--(3.407,2.298)--(3.417,2.306)--(3.428,2.313)--(3.438,2.321)%
  --(3.448,2.329)--(3.458,2.336)--(3.468,2.344)--(3.479,2.352)--(3.489,2.359)--(3.499,2.367)%
  --(3.509,2.374)--(3.519,2.381)--(3.530,2.389)--(3.540,2.396)--(3.550,2.403)--(3.560,2.411)%
  --(3.570,2.418)--(3.581,2.425)--(3.591,2.432)--(3.601,2.439)--(3.611,2.446)--(3.622,2.453)%
  --(3.632,2.460)--(3.642,2.467)--(3.652,2.473)--(3.662,2.480)--(3.673,2.487)--(3.683,2.494)%
  --(3.693,2.500)--(3.703,2.507)--(3.713,2.513)--(3.724,2.520)--(3.734,2.526)--(3.744,2.532)%
  --(3.754,2.538)--(3.765,2.544)--(3.775,2.550)--(3.785,2.557)--(3.795,2.563)--(3.805,2.569)%
  --(3.816,2.575)--(3.826,2.581)--(3.836,2.587)--(3.846,2.591)--(3.856,2.597)--(3.867,2.603)%
  --(3.877,2.609)--(3.887,2.614)--(3.897,2.620)--(3.908,2.625)--(3.918,2.630)--(3.928,2.635)%
  --(3.938,2.641)--(3.948,2.645)--(3.959,2.650)--(3.969,2.656)--(3.979,2.660)--(3.989,2.665)%
  --(3.999,2.670)--(4.010,2.674)--(4.020,2.679)--(4.030,2.684)--(4.040,2.688)--(4.051,2.692)%
  --(4.061,2.696)--(4.071,2.701)--(4.081,2.704)--(4.091,2.709)--(4.102,2.712)--(4.112,2.717)%
  --(4.122,2.720)--(4.132,2.724)--(4.142,2.728)--(4.153,2.731)--(4.163,2.735)--(4.173,2.738)%
  --(4.183,2.742)--(4.193,2.745)--(4.204,2.748)--(4.214,2.751)--(4.224,2.754)--(4.234,2.758)%
  --(4.245,2.760)--(4.255,2.763)--(4.265,2.766)--(4.275,2.769)--(4.285,2.772)--(4.296,2.774)%
  --(4.306,2.776)--(4.316,2.779)--(4.326,2.781)--(4.336,2.784)--(4.347,2.786)--(4.357,2.788)%
  --(4.367,2.790)--(4.377,2.792)--(4.388,2.794)--(4.398,2.796)--(4.408,2.798)--(4.418,2.800)%
  --(4.428,2.802)--(4.439,2.803)--(4.449,2.805)--(4.459,2.806)--(4.469,2.808)--(4.479,2.809)%
  --(4.490,2.811)--(4.500,2.812)--(4.510,2.813)--(4.520,2.815)--(4.531,2.816)--(4.541,2.817)%
  --(4.551,2.818)--(4.561,2.819)--(4.571,2.820)--(4.582,2.821)--(4.592,2.822)--(4.602,2.823)%
  --(4.612,2.824)--(4.622,2.825)--(4.633,2.826)--(4.643,2.826)--(4.653,2.827)--(4.663,2.828)%
  --(4.673,2.829)--(4.684,2.829)--(4.694,2.830)--(4.704,2.831)--(4.714,2.831)--(4.725,2.832)%
  --(4.735,2.833)--(4.745,2.833)--(4.755,2.834)--(4.765,2.835)--(4.776,2.835)--(4.786,2.836)%
  --(4.796,2.837)--(4.806,2.837)--(4.816,2.838)--(4.827,2.839)--(4.837,2.840)--(4.847,2.840)%
  --(4.857,2.841)--(4.868,2.842)--(4.878,2.843)--(4.888,2.843)--(4.898,2.844)--(4.908,2.845)%
  --(4.919,2.846)--(4.929,2.847)--(4.939,2.848)--(4.949,2.849)--(4.959,2.850)--(4.970,2.851)%
  --(4.980,2.852)--(4.990,2.854)--(5.000,2.855)--(5.011,2.856)--(5.021,2.857)--(5.031,2.859)%
  --(5.041,2.860)--(5.051,2.862)--(5.062,2.863)--(5.072,2.865)--(5.082,2.867)--(5.092,2.868)%
  --(5.102,2.870)--(5.113,2.872)--(5.123,2.874)--(5.133,2.876)--(5.143,2.878)--(5.154,2.881)%
  --(5.164,2.883)--(5.174,2.885)--(5.184,2.887)--(5.194,2.890)--(5.205,2.892)--(5.215,2.895)%
  --(5.225,2.898)--(5.235,2.901)--(5.245,2.903)--(5.256,2.906)--(5.266,2.909)--(5.276,2.912)%
  --(5.286,2.915)--(5.296,2.919)--(5.307,2.922)--(5.317,2.925)--(5.327,2.929)--(5.337,2.932)%
  --(5.348,2.936)--(5.358,2.939)--(5.368,2.943)--(5.378,2.947)--(5.388,2.951)--(5.399,2.955)%
  --(5.409,2.959)--(5.419,2.963)--(5.429,2.967)--(5.439,2.971)--(5.450,2.976)--(5.460,2.980)%
  --(5.470,2.984)--(5.480,2.989)--(5.491,2.994)--(5.501,2.998)--(5.511,3.003)--(5.521,3.008)%
  --(5.531,3.012)--(5.542,3.017)--(5.552,3.022)--(5.562,3.027)--(5.572,3.033)--(5.582,3.038)%
  --(5.593,3.043)--(5.603,3.048)--(5.613,3.054)--(5.623,3.059)--(5.634,3.064)--(5.644,3.070)%
  --(5.654,3.075)--(5.664,3.081)--(5.674,3.087)--(5.685,3.092)--(5.695,3.098)--(5.705,3.104)%
  --(5.715,3.110)--(5.725,3.116)--(5.736,3.122)--(5.746,3.128)--(5.756,3.134)--(5.766,3.140)%
  --(5.776,3.146)--(5.787,3.153)--(5.797,3.159)--(5.807,3.165)--(5.817,3.172)--(5.828,3.178)%
  --(5.838,3.184)--(5.848,3.191)--(5.858,3.197)--(5.868,3.204)--(5.879,3.211)--(5.889,3.217)%
  --(5.899,3.224)--(5.909,3.230)--(5.919,3.237)--(5.930,3.244)--(5.940,3.251)--(5.950,3.258)%
  --(5.960,3.265)--(5.971,3.272)--(5.981,3.278)--(5.991,3.285)--(6.001,3.292)--(6.011,3.300)%
  --(6.022,3.307)--(6.032,3.314)--(6.042,3.321)--(6.052,3.328)--(6.062,3.335)--(6.073,3.342)%
  --(6.083,3.350)--(6.093,3.357)--(6.103,3.364)--(6.114,3.372)--(6.124,3.379)--(6.134,3.386)%
  --(6.144,3.394)--(6.154,3.401)--(6.165,3.409)--(6.175,3.416)--(6.185,3.424)--(6.195,3.431)%
  --(6.205,3.439)--(6.216,3.446)--(6.226,3.454)--(6.236,3.461)--(6.246,3.469)--(6.257,3.477)%
  --(6.267,3.484)--(6.277,3.492)--(6.287,3.500)--(6.297,3.507)--(6.308,3.515)--(6.318,3.523)%
  --(6.328,3.531)--(6.338,3.538)--(6.348,3.546)--(6.359,3.554)--(6.369,3.562)--(6.379,3.570)%
  --(6.389,3.577)--(6.399,3.585)--(6.410,3.593)--(6.420,3.601)--(6.430,3.609)--(6.440,3.617)%
  --(6.451,3.625)--(6.461,3.633)--(6.471,3.641)--(6.481,3.649)--(6.491,3.657)--(6.502,3.665)%
  --(6.512,3.673)--(6.522,3.681)--(6.532,3.689)--(6.542,3.697)--(6.553,3.705)--(6.563,3.713)%
  --(6.573,3.721)--(6.583,3.729)--(6.594,3.737)--(6.604,3.745)--(6.614,3.754)--(6.624,3.762)%
  --(6.634,3.770)--(6.645,3.778)--(6.655,3.786)--(6.665,3.794)--(6.675,3.802)--(6.685,3.811)%
  --(6.696,3.819)--(6.706,3.827)--(6.716,3.835)--(6.726,3.843)--(6.737,3.852)--(6.747,3.860)%
  --(6.757,3.868)--(6.767,3.876)--(6.777,3.885)--(6.788,3.893)--(6.798,3.901)--(6.808,3.909)%
  --(6.818,3.918)--(6.828,3.926)--(6.839,3.934)--(6.849,3.943)--(6.859,3.951)--(6.869,3.959)%
  --(6.879,3.967)--(6.890,3.976)--(6.900,3.984)--(6.910,3.992)--(6.920,4.001)--(6.931,4.009)%
  --(6.941,4.017)--(6.951,4.026)--(6.961,4.034)--(6.971,4.043)--(6.982,4.051)--(6.992,4.059)%
  --(7.002,4.068)--(7.012,4.076)--(7.022,4.084)--(7.033,4.093)--(7.043,4.101)--(7.053,4.109)%
  --(7.063,4.118)--(7.074,4.126)--(7.084,4.135)--(7.094,4.143)--(7.104,4.151)--(7.114,4.160)%
  --(7.125,4.168)--(7.135,4.177)--(7.145,4.185)--(7.155,4.193)--(7.165,4.202)--(7.176,4.210)%
  --(7.186,4.219)--(7.196,4.227)--(7.206,4.236)--(7.217,4.244)--(7.227,4.252)--(7.237,4.261)%
  --(7.247,4.269)--(7.257,4.278)--(7.268,4.286)--(7.278,4.295)--(7.288,4.303)--(7.298,4.312)%
  --(7.308,4.320)--(7.319,4.328)--(7.329,4.337)--(7.339,4.345)--(7.349,4.354)--(7.360,4.362)%
  --(7.370,4.371)--(7.380,4.379)--(7.390,4.388)--(7.400,4.396)--(7.411,4.405)--(7.421,4.413)%
  --(7.431,4.422)--(7.441,4.430)--(7.451,4.439)--(7.462,4.447)--(7.472,4.455)--(7.482,4.464)%
  --(7.492,4.472)--(7.502,4.481)--(7.513,4.489)--(7.523,4.498)--(7.533,4.506)--(7.543,4.515)%
  --(7.554,4.523)--(7.564,4.532)--(7.574,4.540)--(7.584,4.549)--(7.594,4.557)--(7.605,4.566)%
  --(7.615,4.574)--(7.625,4.583)--(7.635,4.591)--(7.645,4.600)--(7.656,4.608)--(7.666,4.617)%
  --(7.676,4.625)--(7.686,4.634)--(7.697,4.642)--(7.707,4.650)--(7.717,4.659)--(7.727,4.668)%
  --(7.737,4.676)--(7.748,4.684)--(7.758,4.693)--(7.768,4.701)--(7.778,4.710)--(7.788,4.718)%
  --(7.799,4.727)--(7.809,4.735)--(7.819,4.744)--(7.829,4.752)--(7.840,4.761)--(7.850,4.769)%
  --(7.860,4.778)--(7.870,4.786)--(7.880,4.795)--(7.891,4.803)--(7.901,4.812)--(7.911,4.820)%
  --(7.921,4.829)--(7.931,4.837)--(7.942,4.846)--(7.952,4.854)--(7.962,4.863)--(7.972,4.871)%
  --(7.982,4.880)--(7.993,4.888)--(8.003,4.897)--(8.013,4.905)--(8.023,4.914)--(8.034,4.922)%
  --(8.044,4.931)--(8.054,4.939)--(8.064,4.947)--(8.074,4.956)--(8.085,4.964)--(8.095,4.973)%
  --(8.105,4.981)--(8.115,4.990)--(8.125,4.998)--(8.136,5.007)--(8.146,5.015)--(8.156,5.024)%
  --(8.166,5.032)--(8.177,5.041)--(8.187,5.049)--(8.197,5.058)--(8.207,5.066)--(8.217,5.075)%
  --(8.228,5.083)--(8.238,5.092)--(8.248,5.100)--(8.258,5.109)--(8.268,5.117)--(8.279,5.125)%
  --(8.289,5.134)--(8.299,5.142)--(8.309,5.151)--(8.320,5.159)--(8.330,5.168)--(8.340,5.176)%
  --(8.350,5.185)--(8.360,5.193)--(8.371,5.202)--(8.381,5.210)--(8.391,5.219)--(8.401,5.227)%
  --(8.411,5.235)--(8.422,5.244)--(8.432,5.252)--(8.442,5.261)--(8.452,5.269)--(8.463,5.278)%
  --(8.473,5.286)--(8.483,5.295)--(8.493,5.303)--(8.503,5.312)--(8.514,5.320)--(8.524,5.328)%
  --(8.534,5.337)--(8.544,5.345)--(8.554,5.354)--(8.565,5.362)--(8.575,5.371)--(8.585,5.379)%
  --(8.595,5.388)--(8.605,5.396)--(8.616,5.405)--(8.626,5.413)--(8.636,5.421)--(8.646,5.430)%
  --(8.657,5.438)--(8.667,5.447)--(8.677,5.455)--(8.687,5.464)--(8.697,5.472)--(8.708,5.480)%
  --(8.718,5.489)--(8.728,5.497)--(8.738,5.506)--(8.748,5.514)--(8.759,5.523)--(8.769,5.531)%
  --(8.779,5.539)--(8.789,5.548)--(8.800,5.556)--(8.810,5.565)--(8.820,5.573)--(8.830,5.582)%
  --(8.840,5.590)--(8.851,5.598)--(8.861,5.607)--(8.871,5.615)--(8.881,5.624)--(8.891,5.632)%
  --(8.902,5.640)--(8.912,5.649)--(8.922,5.657)--(8.932,5.666)--(8.943,5.674)--(8.953,5.683)%
  --(8.963,5.691)--(8.973,5.699)--(8.983,5.708)--(8.994,5.716)--(9.004,5.725)--(9.014,5.733)%
  --(9.024,5.741)--(9.034,5.750)--(9.045,5.758)--(9.055,5.767)--(9.065,5.775)--(9.075,5.783)%
  --(9.085,5.792)--(9.096,5.800)--(9.106,5.809)--(9.116,5.817)--(9.126,5.825)--(9.137,5.834)%
  --(9.147,5.842)--(9.157,5.850)--(9.167,5.859)--(9.177,5.867)--(9.188,5.876)--(9.198,5.884)%
  --(9.208,5.892)--(9.218,5.901)--(9.228,5.909)--(9.239,5.918)--(9.249,5.926)--(9.259,5.934)%
  --(9.269,5.943)--(9.280,5.951)--(9.290,5.959)--(9.300,5.968)--(9.310,5.976)--(9.320,5.985)%
  --(9.331,5.993)--(9.341,6.001)--(9.351,6.010)--(9.361,6.018)--(9.371,6.026)--(9.382,6.035)%
  --(9.392,6.043)--(9.402,6.051)--(9.412,6.060)--(9.423,6.068)--(9.433,6.077)--(9.443,6.085)%
  --(9.453,6.093)--(9.463,6.102)--(9.474,6.110)--(9.484,6.118)--(9.494,6.127)--(9.504,6.135)%
  --(9.514,6.143)--(9.525,6.152)--(9.535,6.160)--(9.545,6.168)--(9.555,6.177)--(9.565,6.185)%
  --(9.576,6.194)--(9.586,6.202)--(9.596,6.210)--(9.606,6.219)--(9.617,6.227)--(9.627,6.235)%
  --(9.637,6.244)--(9.647,6.252)--(9.657,6.260)--(9.668,6.269)--(9.678,6.277)--(9.688,6.285)%
  --(9.698,6.294)--(9.708,6.302)--(9.719,6.310)--(9.729,6.319)--(9.739,6.327)--(9.749,6.335)%
  --(9.760,6.344)--(9.770,6.352)--(9.780,6.360)--(9.790,6.369)--(9.800,6.377)--(9.811,6.385)%
  --(9.821,6.394)--(9.831,6.402)--(9.841,6.410)--(9.851,6.419)--(9.862,6.427)--(9.872,6.435)%
  --(9.882,6.444)--(9.892,6.452)--(9.903,6.460)--(9.913,6.468)--(9.923,6.477)--(9.933,6.485)%
  --(9.943,6.493)--(9.954,6.502)--(9.964,6.510)--(9.974,6.518)--(9.984,6.527)--(9.994,6.535)%
  --(10.005,6.543)--(10.015,6.551)--(10.025,6.560)--(10.035,6.568)--(10.046,6.577)--(10.056,6.585)%
  --(10.066,6.593)--(10.076,6.601)--(10.086,6.610)--(10.097,6.618)--(10.107,6.626)--(10.117,6.635)%
  --(10.127,6.643)--(10.137,6.651)--(10.148,6.659)--(10.158,6.668)--(10.168,6.676)--(10.178,6.684)%
  --(10.188,6.693)--(10.199,6.701)--(10.209,6.709)--(10.219,6.717)--(10.229,6.726)--(10.240,6.734)%
  --(10.250,6.742)--(10.260,6.751)--(10.270,6.759)--(10.280,6.767)--(10.291,6.775)--(10.301,6.784)%
  --(10.311,6.792)--(10.321,6.800)--(10.331,6.809)--(10.342,6.817)--(10.352,6.825)--(10.362,6.833)%
  --(10.372,6.842)--(10.383,6.850)--(10.393,6.858)--(10.403,6.866)--(10.413,6.875)--(10.423,6.883)%
  --(10.434,6.891)--(10.444,6.900)--(10.454,6.908)--(10.464,6.916)--(10.474,6.924)--(10.485,6.933)%
  --(10.495,6.941)--(10.505,6.949)--(10.515,6.957)--(10.526,6.966)--(10.536,6.974)--(10.546,6.982)%
  --(10.556,6.990)--(10.566,6.999)--(10.577,7.007)--(10.587,7.015)--(10.597,7.023)--(10.607,7.032)%
  --(10.617,7.040)--(10.628,7.048)--(10.638,7.056)--(10.648,7.065)--(10.658,7.073)--(10.668,7.081)%
  --(10.679,7.089)--(10.689,7.098)--(10.699,7.106)--(10.709,7.114)--(10.720,7.122)--(10.730,7.131)%
  --(10.740,7.139)--(10.750,7.147)--(10.760,7.155)--(10.771,7.164)--(10.781,7.172)--(10.791,7.180)%
  --(10.801,7.188)--(10.811,7.197)--(10.822,7.205)--(10.832,7.213)--(10.842,7.221)--(10.852,7.230)%
  --(10.863,7.238)--(10.873,7.246)--(10.883,7.254)--(10.893,7.262)--(10.903,7.271)--(10.914,7.279)%
  --(10.924,7.287)--(10.934,7.295)--(10.944,7.304)--(10.954,7.312)--(10.965,7.320)--(10.975,7.328)%
  --(10.985,7.336)--(10.995,7.345)--(11.006,7.353)--(11.016,7.361)--(11.026,7.369)--(11.036,7.378)%
  --(11.046,7.386)--(11.057,7.394)--(11.067,7.402)--(11.077,7.410)--(11.087,7.419)--(11.097,7.427)%
  --(11.108,7.435)--(11.118,7.443)--(11.128,7.452)--(11.138,7.460)--(11.149,7.468)--(11.159,7.476)%
  --(11.169,7.484)--(11.179,7.493)--(11.189,7.501)--(11.200,7.509)--(11.210,7.517)--(11.220,7.525)%
  --(11.230,7.534)--(11.240,7.542)--(11.251,7.550)--(11.261,7.558)--(11.271,7.566)--(11.281,7.575)%
  --(11.291,7.583)--(11.302,7.591)--(11.312,7.599)--(11.322,7.607)--(11.332,7.616)--(11.343,7.624)%
  --(11.353,7.632)--(11.363,7.640)--(11.373,7.648)--(11.383,7.657)--(11.394,7.665)--(11.404,7.673)%
  --(11.414,7.681)--(11.424,7.689)--(11.434,7.697)--(11.445,7.706)--(11.455,7.714)--(11.465,7.722)%
  --(11.475,7.730)--(11.486,7.738)--(11.496,7.747)--(11.506,7.755)--(11.516,7.763)--(11.526,7.771)%
  --(11.537,7.779)--(11.547,7.788)--(11.557,7.796)--(11.567,7.804)--(11.577,7.812)--(11.588,7.820)%
  --(11.598,7.828)--(11.608,7.837)--(11.618,7.845)--(11.629,7.853)--(11.639,7.861)--(11.649,7.869)%
  --(11.659,7.877)--(11.669,7.886)--(11.680,7.894)--(11.690,7.902)--(11.700,7.910)--(11.710,7.918)%
  --(11.720,7.927)--(11.731,7.935)--(11.741,7.943)--(11.751,7.951);
\gpcolor{color=gp lt color border}
\node[gp node left] at (2.972,7.989) {$\mu_n$};
\gpcolor{rgb color={0.000,0.620,0.451}}
\draw[gp path] (1.872,7.989)--(2.788,7.989);
\draw[gp path] (1.559,1.454)--(1.569,1.449)--(1.579,1.446)--(1.589,1.443)--(1.599,1.438)%
  --(1.610,1.435)--(1.620,1.431)--(1.630,1.428)--(1.640,1.425)--(1.650,1.422)--(1.661,1.419)%
  --(1.671,1.416)--(1.681,1.413)--(1.691,1.411)--(1.702,1.408)--(1.712,1.407)--(1.722,1.404)%
  --(1.732,1.402)--(1.742,1.401)--(1.753,1.399)--(1.763,1.398)--(1.773,1.396)--(1.783,1.395)%
  --(1.793,1.393)--(1.804,1.393)--(1.814,1.392)--(1.824,1.392)--(1.834,1.390)--(1.845,1.390)%
  --(1.855,1.390)--(1.865,1.390)--(1.875,1.390)--(1.885,1.391)--(1.896,1.391)--(1.906,1.391)%
  --(1.916,1.392)--(1.926,1.392)--(1.936,1.393)--(1.947,1.394)--(1.957,1.396)--(1.967,1.397)%
  --(1.977,1.398)--(1.987,1.399)--(1.998,1.401)--(2.008,1.403)--(2.018,1.405)--(2.028,1.407)%
  --(2.039,1.408)--(2.049,1.411)--(2.059,1.414)--(2.069,1.416)--(2.079,1.419)--(2.090,1.421)%
  --(2.100,1.424)--(2.110,1.428)--(2.120,1.430)--(2.130,1.433)--(2.141,1.437)--(2.151,1.439)%
  --(2.161,1.443)--(2.171,1.447)--(2.182,1.451)--(2.192,1.455)--(2.202,1.459)--(2.212,1.462)%
  --(2.222,1.466)--(2.233,1.470)--(2.243,1.474)--(2.253,1.479)--(2.263,1.483)--(2.273,1.488)%
  --(2.284,1.492)--(2.294,1.497)--(2.304,1.502)--(2.314,1.507)--(2.325,1.512)--(2.335,1.517)%
  --(2.345,1.523)--(2.355,1.528)--(2.365,1.533)--(2.376,1.538)--(2.386,1.544)--(2.396,1.549)%
  --(2.406,1.554)--(2.416,1.561)--(2.427,1.567)--(2.437,1.572)--(2.447,1.578)--(2.457,1.585)%
  --(2.468,1.591)--(2.478,1.597)--(2.488,1.603)--(2.498,1.609)--(2.508,1.616)--(2.519,1.622)%
  --(2.529,1.628)--(2.539,1.635)--(2.549,1.642)--(2.559,1.648)--(2.570,1.654)--(2.580,1.662)%
  --(2.590,1.668)--(2.600,1.675)--(2.610,1.682)--(2.621,1.689)--(2.631,1.696)--(2.641,1.703)%
  --(2.651,1.711)--(2.662,1.718)--(2.672,1.725)--(2.682,1.732)--(2.692,1.740)--(2.702,1.747)%
  --(2.713,1.754)--(2.723,1.762)--(2.733,1.769)--(2.743,1.777)--(2.753,1.784)--(2.764,1.792)%
  --(2.774,1.800)--(2.784,1.807)--(2.794,1.815)--(2.805,1.822)--(2.815,1.831)--(2.825,1.838)%
  --(2.835,1.846)--(2.845,1.854)--(2.856,1.861)--(2.866,1.869)--(2.876,1.877)--(2.886,1.886)%
  --(2.896,1.894)--(2.907,1.902)--(2.917,1.910)--(2.927,1.918)--(2.937,1.926)--(2.948,1.934)%
  --(2.958,1.942)--(2.968,1.950)--(2.978,1.958)--(2.988,1.966)--(2.999,1.974)--(3.009,1.983)%
  --(3.019,1.991)--(3.029,1.999)--(3.039,2.006)--(3.050,2.015)--(3.060,2.023)--(3.070,2.032)%
  --(3.080,2.040)--(3.090,2.047)--(3.101,2.056)--(3.111,2.064)--(3.121,2.073)--(3.131,2.080)%
  --(3.142,2.089)--(3.152,2.097)--(3.162,2.105)--(3.172,2.114)--(3.182,2.121)--(3.193,2.130)%
  --(3.203,2.138)--(3.213,2.146)--(3.223,2.154)--(3.233,2.162)--(3.244,2.171)--(3.254,2.179)%
  --(3.264,2.186)--(3.274,2.195)--(3.285,2.203)--(3.295,2.211)--(3.305,2.219)--(3.315,2.226)%
  --(3.325,2.234)--(3.336,2.242)--(3.346,2.250)--(3.356,2.258)--(3.366,2.266)--(3.376,2.274)%
  --(3.387,2.282)--(3.397,2.290)--(3.407,2.298)--(3.417,2.306)--(3.428,2.313)--(3.438,2.321)%
  --(3.448,2.329)--(3.458,2.336)--(3.468,2.344)--(3.479,2.352)--(3.489,2.359)--(3.499,2.367)%
  --(3.509,2.374)--(3.519,2.381)--(3.530,2.389)--(3.540,2.396)--(3.550,2.403)--(3.560,2.411)%
  --(3.570,2.418)--(3.581,2.425)--(3.591,2.432)--(3.601,2.439)--(3.611,2.446)--(3.622,2.453)%
  --(3.632,2.460)--(3.642,2.467)--(3.652,2.473)--(3.662,2.480)--(3.673,2.487)--(3.683,2.494)%
  --(3.693,2.500)--(3.703,2.507)--(3.713,2.513)--(3.724,2.520)--(3.734,2.526)--(3.744,2.532)%
  --(3.754,2.538)--(3.765,2.544)--(3.775,2.550)--(3.785,2.557)--(3.795,2.563)--(3.805,2.569)%
  --(3.816,2.575)--(3.826,2.581)--(3.836,2.587)--(3.846,2.591)--(3.856,2.597)--(3.867,2.603)%
  --(3.877,2.609)--(3.887,2.614)--(3.897,2.620)--(3.908,2.625)--(3.918,2.630)--(3.928,2.635)%
  --(3.938,2.641)--(3.948,2.645)--(3.959,2.650)--(3.969,2.656)--(3.979,2.660)--(3.989,2.665)%
  --(3.999,2.670)--(4.010,2.674)--(4.020,2.679)--(4.030,2.684)--(4.040,2.688)--(4.051,2.692)%
  --(4.061,2.696)--(4.071,2.701)--(4.081,2.704)--(4.091,2.709)--(4.102,2.712)--(4.112,2.717)%
  --(4.122,2.720)--(4.132,2.724)--(4.142,2.728)--(4.153,2.731)--(4.163,2.735)--(4.173,2.738)%
  --(4.183,2.742)--(4.193,2.745)--(4.204,2.748)--(4.214,2.751)--(4.224,2.754)--(4.234,2.758)%
  --(4.245,2.760)--(4.255,2.763)--(4.265,2.766)--(4.275,2.769)--(4.285,2.772)--(4.296,2.774)%
  --(4.306,2.776)--(4.316,2.779)--(4.326,2.781)--(4.336,2.784)--(4.347,2.786)--(4.357,2.788)%
  --(4.367,2.790)--(4.377,2.792)--(4.388,2.794)--(4.398,2.796)--(4.408,2.798)--(4.418,2.800)%
  --(4.428,2.802)--(4.439,2.803)--(4.449,2.805)--(4.459,2.806)--(4.469,2.808)--(4.479,2.809)%
  --(4.490,2.811)--(4.500,2.812)--(4.510,2.813)--(4.520,2.815)--(4.531,2.816)--(4.541,2.817)%
  --(4.551,2.818)--(4.561,2.819)--(4.571,2.820)--(4.582,2.821)--(4.592,2.822)--(4.602,2.823)%
  --(4.612,2.824)--(4.622,2.825)--(4.633,2.826)--(4.643,2.826)--(4.653,2.827)--(4.663,2.828)%
  --(4.673,2.829)--(4.684,2.829)--(4.694,2.830)--(4.704,2.831)--(4.714,2.831)--(4.725,2.832)%
  --(4.735,2.833)--(4.745,2.833)--(4.755,2.834)--(4.765,2.835)--(4.776,2.835)--(4.786,2.836)%
  --(4.796,2.837)--(4.806,2.837)--(4.816,2.838)--(4.827,2.839)--(4.837,2.840)--(4.847,2.840)%
  --(4.857,2.841)--(4.868,2.842)--(4.878,2.843)--(4.888,2.843)--(4.898,2.844)--(4.908,2.845)%
  --(4.919,2.846)--(4.929,2.847)--(4.939,2.848)--(4.949,2.849)--(4.959,2.850)--(4.970,2.851)%
  --(4.980,2.852)--(4.990,2.854)--(5.000,2.855)--(5.011,2.856)--(5.021,2.857)--(5.031,2.859)%
  --(5.041,2.860)--(5.051,2.862)--(5.062,2.863)--(5.072,2.865)--(5.082,2.867)--(5.092,2.868)%
  --(5.102,2.870)--(5.113,2.872)--(5.123,2.874)--(5.133,2.876)--(5.143,2.878)--(5.154,2.881)%
  --(5.164,2.883)--(5.174,2.885)--(5.184,2.887)--(5.194,2.890)--(5.205,2.892)--(5.215,2.895)%
  --(5.225,2.898)--(5.235,2.901)--(5.245,2.903)--(5.256,2.906)--(5.266,2.909)--(5.276,2.912)%
  --(5.286,2.915)--(5.296,2.919)--(5.307,2.922)--(5.317,2.925)--(5.327,2.929)--(5.337,2.932)%
  --(5.348,2.936)--(5.358,2.939)--(5.368,2.943)--(5.378,2.947)--(5.388,2.951)--(5.399,2.955)%
  --(5.409,2.959)--(5.419,2.963)--(5.429,2.967)--(5.439,2.971)--(5.450,2.976)--(5.460,2.980)%
  --(5.470,2.984)--(5.480,2.989)--(5.491,2.994)--(5.501,2.998)--(5.511,3.003)--(5.521,3.008)%
  --(5.531,3.012)--(5.542,3.017)--(5.552,3.022)--(5.562,3.027)--(5.572,3.033)--(5.582,3.038)%
  --(5.593,3.043)--(5.603,3.048)--(5.613,3.054)--(5.623,3.059)--(5.634,3.064)--(5.644,3.070)%
  --(5.654,3.075)--(5.664,3.081)--(5.674,3.087)--(5.685,3.092)--(5.695,3.098)--(5.705,3.104)%
  --(5.715,3.110)--(5.725,3.116)--(5.736,3.122)--(5.746,3.128)--(5.756,3.134)--(5.766,3.140)%
  --(5.776,3.146)--(5.787,3.153)--(5.797,3.159)--(5.807,3.165)--(5.817,3.172)--(5.828,3.178)%
  --(5.838,3.184)--(5.848,3.191)--(5.858,3.197)--(5.868,3.204)--(5.879,3.211)--(5.889,3.217)%
  --(5.899,3.224)--(5.909,3.230)--(5.919,3.237)--(5.930,3.244)--(5.940,3.251)--(5.950,3.258)%
  --(5.960,3.265)--(5.971,3.272)--(5.981,3.278)--(5.991,3.285)--(6.001,3.292)--(6.011,3.300)%
  --(6.022,3.307)--(6.032,3.314)--(6.042,3.321)--(6.052,3.328)--(6.062,3.335)--(6.073,3.342)%
  --(6.083,3.350)--(6.093,3.357)--(6.103,3.364)--(6.114,3.372)--(6.124,3.379)--(6.134,3.386)%
  --(6.144,3.394)--(6.154,3.401)--(6.165,3.409)--(6.175,3.416)--(6.185,3.424)--(6.195,3.431)%
  --(6.205,3.439)--(6.216,3.446)--(6.226,3.454)--(6.236,3.461)--(6.246,3.469)--(6.257,3.477)%
  --(6.267,3.484)--(6.277,3.492)--(6.287,3.500)--(6.297,3.507)--(6.308,3.515)--(6.318,3.523)%
  --(6.328,3.531)--(6.338,3.538)--(6.348,3.546)--(6.359,3.554)--(6.369,3.562)--(6.379,3.570)%
  --(6.389,3.577)--(6.399,3.585)--(6.410,3.593)--(6.420,3.601)--(6.430,3.609)--(6.440,3.617)%
  --(6.451,3.625)--(6.461,3.633)--(6.471,3.641)--(6.481,3.649)--(6.491,3.657)--(6.502,3.665)%
  --(6.512,3.673)--(6.522,3.681)--(6.532,3.689)--(6.542,3.697)--(6.553,3.705)--(6.563,3.713)%
  --(6.573,3.721)--(6.583,3.729)--(6.594,3.737)--(6.604,3.745)--(6.614,3.754)--(6.624,3.762)%
  --(6.634,3.770)--(6.645,3.778)--(6.655,3.786)--(6.665,3.794)--(6.675,3.802)--(6.685,3.811)%
  --(6.696,3.819)--(6.706,3.827)--(6.716,3.835)--(6.726,3.843)--(6.737,3.852)--(6.747,3.860)%
  --(6.757,3.868)--(6.767,3.876)--(6.777,3.885)--(6.788,3.893)--(6.798,3.901)--(6.808,3.909)%
  --(6.818,3.918)--(6.828,3.926)--(6.839,3.934)--(6.849,3.943)--(6.859,3.951)--(6.869,3.959)%
  --(6.879,3.967)--(6.890,3.976)--(6.900,3.984)--(6.910,3.992)--(6.920,4.001)--(6.931,4.009)%
  --(6.941,4.017)--(6.951,4.026)--(6.961,4.034)--(6.971,4.043)--(6.982,4.051)--(6.992,4.059)%
  --(7.002,4.068)--(7.012,4.076)--(7.022,4.084)--(7.033,4.093)--(7.043,4.101)--(7.053,4.109)%
  --(7.063,4.118)--(7.074,4.126)--(7.084,4.135)--(7.094,4.143)--(7.104,4.151)--(7.114,4.160)%
  --(7.125,4.168)--(7.135,4.177)--(7.145,4.185)--(7.155,4.193)--(7.165,4.202)--(7.176,4.210)%
  --(7.186,4.219)--(7.196,4.227)--(7.206,4.236)--(7.217,4.244)--(7.227,4.252)--(7.237,4.261)%
  --(7.247,4.269)--(7.257,4.278)--(7.268,4.286)--(7.278,4.295)--(7.288,4.303)--(7.298,4.312)%
  --(7.308,4.320)--(7.319,4.328)--(7.329,4.337)--(7.339,4.345)--(7.349,4.354)--(7.360,4.362)%
  --(7.370,4.371)--(7.380,4.379)--(7.390,4.388)--(7.400,4.396)--(7.411,4.405)--(7.421,4.413)%
  --(7.431,4.422)--(7.441,4.430)--(7.451,4.439)--(7.462,4.447)--(7.472,4.455)--(7.482,4.464)%
  --(7.492,4.472)--(7.502,4.481)--(7.513,4.489)--(7.523,4.498)--(7.533,4.506)--(7.543,4.515)%
  --(7.554,4.523)--(7.564,4.532)--(7.574,4.540)--(7.584,4.549)--(7.594,4.557)--(7.605,4.566)%
  --(7.615,4.574)--(7.625,4.583)--(7.635,4.591)--(7.645,4.600)--(7.656,4.608)--(7.666,4.617)%
  --(7.676,4.625)--(7.686,4.634)--(7.697,4.642)--(7.707,4.650)--(7.717,4.659)--(7.727,4.668)%
  --(7.737,4.676)--(7.748,4.684)--(7.758,4.693)--(7.768,4.701)--(7.778,4.710)--(7.788,4.718)%
  --(7.799,4.727)--(7.809,4.735)--(7.819,4.744)--(7.829,4.752)--(7.840,4.761)--(7.850,4.769)%
  --(7.860,4.778)--(7.870,4.786)--(7.880,4.795)--(7.891,4.803)--(7.901,4.812)--(7.911,4.820)%
  --(7.921,4.829)--(7.931,4.837)--(7.942,4.846)--(7.952,4.854)--(7.962,4.863)--(7.972,4.871)%
  --(7.982,4.880)--(7.993,4.888)--(8.003,4.897)--(8.013,4.905)--(8.023,4.914)--(8.034,4.922)%
  --(8.044,4.931)--(8.054,4.939)--(8.064,4.947)--(8.074,4.956)--(8.085,4.964)--(8.095,4.973)%
  --(8.105,4.981)--(8.115,4.990)--(8.125,4.998)--(8.136,5.007)--(8.146,5.015)--(8.156,5.024)%
  --(8.166,5.032)--(8.177,5.041)--(8.187,5.049)--(8.197,5.058)--(8.207,5.066)--(8.217,5.075)%
  --(8.228,5.083)--(8.238,5.092)--(8.248,5.100)--(8.258,5.109)--(8.268,5.117)--(8.279,5.125)%
  --(8.289,5.134)--(8.299,5.142)--(8.309,5.151)--(8.320,5.159)--(8.330,5.168)--(8.340,5.176)%
  --(8.350,5.185)--(8.360,5.193)--(8.371,5.202)--(8.381,5.210)--(8.391,5.219)--(8.401,5.227)%
  --(8.411,5.235)--(8.422,5.244)--(8.432,5.252)--(8.442,5.261)--(8.452,5.269)--(8.463,5.278)%
  --(8.473,5.286)--(8.483,5.295)--(8.493,5.303)--(8.503,5.312)--(8.514,5.320)--(8.524,5.328)%
  --(8.534,5.337)--(8.544,5.345)--(8.554,5.354)--(8.565,5.362)--(8.575,5.371)--(8.585,5.379)%
  --(8.595,5.388)--(8.605,5.396)--(8.616,5.405)--(8.626,5.413)--(8.636,5.421)--(8.646,5.430)%
  --(8.657,5.438)--(8.667,5.447)--(8.677,5.455)--(8.687,5.464)--(8.697,5.472)--(8.708,5.480)%
  --(8.718,5.489)--(8.728,5.497)--(8.738,5.506)--(8.748,5.514)--(8.759,5.523)--(8.769,5.531)%
  --(8.779,5.539)--(8.789,5.548)--(8.800,5.556)--(8.810,5.565)--(8.820,5.573)--(8.830,5.582)%
  --(8.840,5.590)--(8.851,5.598)--(8.861,5.607)--(8.871,5.615)--(8.881,5.624)--(8.891,5.632)%
  --(8.902,5.640)--(8.912,5.649)--(8.922,5.657)--(8.932,5.666)--(8.943,5.674)--(8.953,5.683)%
  --(8.963,5.691)--(8.973,5.699)--(8.983,5.708)--(8.994,5.716)--(9.004,5.725)--(9.014,5.733)%
  --(9.024,5.741)--(9.034,5.750)--(9.045,5.758)--(9.055,5.767)--(9.065,5.775)--(9.075,5.783)%
  --(9.085,5.792)--(9.096,5.800)--(9.106,5.809)--(9.116,5.817)--(9.126,5.825)--(9.137,5.834)%
  --(9.147,5.842)--(9.157,5.850)--(9.167,5.859)--(9.177,5.867)--(9.188,5.876)--(9.198,5.884)%
  --(9.208,5.892)--(9.218,5.901)--(9.228,5.909)--(9.239,5.918)--(9.249,5.926)--(9.259,5.934)%
  --(9.269,5.943)--(9.280,5.951)--(9.290,5.959)--(9.300,5.968)--(9.310,5.976)--(9.320,5.985)%
  --(9.331,5.993)--(9.341,6.001)--(9.351,6.010)--(9.361,6.018)--(9.371,6.026)--(9.382,6.035)%
  --(9.392,6.043)--(9.402,6.051)--(9.412,6.060)--(9.423,6.068)--(9.433,6.077)--(9.443,6.085)%
  --(9.453,6.093)--(9.463,6.102)--(9.474,6.110)--(9.484,6.118)--(9.494,6.127)--(9.504,6.135)%
  --(9.514,6.143)--(9.525,6.152)--(9.535,6.160)--(9.545,6.168)--(9.555,6.177)--(9.565,6.185)%
  --(9.576,6.194)--(9.586,6.202)--(9.596,6.210)--(9.606,6.219)--(9.617,6.227)--(9.627,6.235)%
  --(9.637,6.244)--(9.647,6.252)--(9.657,6.260)--(9.668,6.269)--(9.678,6.277)--(9.688,6.285)%
  --(9.698,6.294)--(9.708,6.302)--(9.719,6.310)--(9.729,6.319)--(9.739,6.327)--(9.749,6.335)%
  --(9.760,6.344)--(9.770,6.352)--(9.780,6.360)--(9.790,6.369)--(9.800,6.377)--(9.811,6.385)%
  --(9.821,6.394)--(9.831,6.402)--(9.841,6.410)--(9.851,6.419)--(9.862,6.427)--(9.872,6.435)%
  --(9.882,6.444)--(9.892,6.452)--(9.903,6.460)--(9.913,6.468)--(9.923,6.477)--(9.933,6.485)%
  --(9.943,6.493)--(9.954,6.502)--(9.964,6.510)--(9.974,6.518)--(9.984,6.527)--(9.994,6.535)%
  --(10.005,6.543)--(10.015,6.551)--(10.025,6.560)--(10.035,6.568)--(10.046,6.577)--(10.056,6.585)%
  --(10.066,6.593)--(10.076,6.601)--(10.086,6.610)--(10.097,6.618)--(10.107,6.626)--(10.117,6.635)%
  --(10.127,6.643)--(10.137,6.651)--(10.148,6.659)--(10.158,6.668)--(10.168,6.676)--(10.178,6.684)%
  --(10.188,6.693)--(10.199,6.701)--(10.209,6.709)--(10.219,6.717)--(10.229,6.726)--(10.240,6.734)%
  --(10.250,6.742)--(10.260,6.751)--(10.270,6.759)--(10.280,6.767)--(10.291,6.775)--(10.301,6.784)%
  --(10.311,6.792)--(10.321,6.800)--(10.331,6.809)--(10.342,6.817)--(10.352,6.825)--(10.362,6.833)%
  --(10.372,6.842)--(10.383,6.850)--(10.393,6.858)--(10.403,6.866)--(10.413,6.875)--(10.423,6.883)%
  --(10.434,6.891)--(10.444,6.900)--(10.454,6.908)--(10.464,6.916)--(10.474,6.924)--(10.485,6.933)%
  --(10.495,6.941)--(10.505,6.949)--(10.515,6.957)--(10.526,6.966)--(10.536,6.974)--(10.546,6.982)%
  --(10.556,6.990)--(10.566,6.999)--(10.577,7.007)--(10.587,7.015)--(10.597,7.023)--(10.607,7.032)%
  --(10.617,7.040)--(10.628,7.048)--(10.638,7.056)--(10.648,7.065)--(10.658,7.073)--(10.668,7.081)%
  --(10.679,7.089)--(10.689,7.098)--(10.699,7.106)--(10.709,7.114)--(10.720,7.122)--(10.730,7.131)%
  --(10.740,7.139)--(10.750,7.147)--(10.760,7.155)--(10.771,7.164)--(10.781,7.172)--(10.791,7.180)%
  --(10.801,7.188)--(10.811,7.197)--(10.822,7.205)--(10.832,7.213)--(10.842,7.221)--(10.852,7.230)%
  --(10.863,7.238)--(10.873,7.246)--(10.883,7.254)--(10.893,7.262)--(10.903,7.271)--(10.914,7.279)%
  --(10.924,7.287)--(10.934,7.295)--(10.944,7.304)--(10.954,7.312)--(10.965,7.320)--(10.975,7.328)%
  --(10.985,7.336)--(10.995,7.345)--(11.006,7.353)--(11.016,7.361)--(11.026,7.369)--(11.036,7.378)%
  --(11.046,7.386)--(11.057,7.394)--(11.067,7.402)--(11.077,7.410)--(11.087,7.419)--(11.097,7.427)%
  --(11.108,7.435)--(11.118,7.443)--(11.128,7.452)--(11.138,7.460)--(11.149,7.468)--(11.159,7.476)%
  --(11.169,7.484)--(11.179,7.493)--(11.189,7.501)--(11.200,7.509)--(11.210,7.517)--(11.220,7.525)%
  --(11.230,7.534)--(11.240,7.542)--(11.251,7.550)--(11.261,7.558)--(11.271,7.566)--(11.281,7.575)%
  --(11.291,7.583)--(11.302,7.591)--(11.312,7.599)--(11.322,7.607)--(11.332,7.616)--(11.343,7.624)%
  --(11.353,7.632)--(11.363,7.640)--(11.373,7.648)--(11.383,7.657)--(11.394,7.665)--(11.404,7.673)%
  --(11.414,7.681)--(11.424,7.689)--(11.434,7.697)--(11.445,7.706)--(11.455,7.714)--(11.465,7.722)%
  --(11.475,7.730)--(11.486,7.738)--(11.496,7.747)--(11.506,7.755)--(11.516,7.763)--(11.526,7.771)%
  --(11.537,7.779)--(11.547,7.788)--(11.557,7.796)--(11.567,7.804)--(11.577,7.812)--(11.588,7.820)%
  --(11.598,7.828)--(11.608,7.837)--(11.618,7.845)--(11.629,7.853)--(11.639,7.861)--(11.649,7.869)%
  --(11.659,7.877)--(11.669,7.886)--(11.680,7.894)--(11.690,7.902)--(11.700,7.910)--(11.710,7.918)%
  --(11.720,7.927)--(11.731,7.935)--(11.741,7.943)--(11.751,7.951);
\gpcolor{color=gp lt color border}
\draw[gp path] (1.504,8.631)--(1.504,0.985)--(13.447,0.985)--(13.447,8.631)--cycle;
%% coordinates of the plot area
\gpdefrectangularnode{gp plot 1}{\pgfpoint{1.504cm}{0.985cm}}{\pgfpoint{13.447cm}{8.631cm}}
\end{tikzpicture}
%% gnuplot variables

	\caption{Gráfico dos potenciais químicos de próton e nêutron calculados a partir da Equação~\eqref{Eq:Potenciais_Quimicos}. Como a fração de próton é 1/2, ambos os potenciais tem o mesmo valor. \protect[Parameters: eNJL1m; Proton fraction: 1/2] }
	\label{Fig:chemical_potential_graph_eNJL1m}
\end{figure*}

\begin{figure*}
	\begin{tikzpicture}[gnuplot]
%% generated with GNUPLOT 5.0p2 (Lua 5.2; terminal rev. 99, script rev. 100)
%% Mon Mar  7 16:11:42 2016
\path (0.000,0.000) rectangle (14.000,9.000);
\gpcolor{color=gp lt color border}
\gpsetlinetype{gp lt border}
\gpsetdashtype{gp dt solid}
\gpsetlinewidth{1.00}
\draw[gp path] (1.688,0.985)--(1.868,0.985);
\draw[gp path] (13.447,0.985)--(13.267,0.985);
\node[gp node right] at (1.504,0.985) {$-4500$};
\draw[gp path] (1.688,1.941)--(1.868,1.941);
\draw[gp path] (13.447,1.941)--(13.267,1.941);
\node[gp node right] at (1.504,1.941) {$-4000$};
\draw[gp path] (1.688,2.897)--(1.868,2.897);
\draw[gp path] (13.447,2.897)--(13.267,2.897);
\node[gp node right] at (1.504,2.897) {$-3500$};
\draw[gp path] (1.688,3.852)--(1.868,3.852);
\draw[gp path] (13.447,3.852)--(13.267,3.852);
\node[gp node right] at (1.504,3.852) {$-3000$};
\draw[gp path] (1.688,4.808)--(1.868,4.808);
\draw[gp path] (13.447,4.808)--(13.267,4.808);
\node[gp node right] at (1.504,4.808) {$-2500$};
\draw[gp path] (1.688,5.764)--(1.868,5.764);
\draw[gp path] (13.447,5.764)--(13.267,5.764);
\node[gp node right] at (1.504,5.764) {$-2000$};
\draw[gp path] (1.688,6.720)--(1.868,6.720);
\draw[gp path] (13.447,6.720)--(13.267,6.720);
\node[gp node right] at (1.504,6.720) {$-1500$};
\draw[gp path] (1.688,7.675)--(1.868,7.675);
\draw[gp path] (13.447,7.675)--(13.267,7.675);
\node[gp node right] at (1.504,7.675) {$-1000$};
\draw[gp path] (1.688,8.631)--(1.868,8.631);
\draw[gp path] (13.447,8.631)--(13.267,8.631);
\node[gp node right] at (1.504,8.631) {$-500$};
\draw[gp path] (1.688,0.985)--(1.688,1.165);
\draw[gp path] (1.688,8.631)--(1.688,8.451);
\node[gp node center] at (1.688,0.677) {$0$};
\draw[gp path] (3.368,0.985)--(3.368,1.165);
\draw[gp path] (3.368,8.631)--(3.368,8.451);
\node[gp node center] at (3.368,0.677) {$0.5$};
\draw[gp path] (5.048,0.985)--(5.048,1.165);
\draw[gp path] (5.048,8.631)--(5.048,8.451);
\node[gp node center] at (5.048,0.677) {$1$};
\draw[gp path] (6.728,0.985)--(6.728,1.165);
\draw[gp path] (6.728,8.631)--(6.728,8.451);
\node[gp node center] at (6.728,0.677) {$1.5$};
\draw[gp path] (8.407,0.985)--(8.407,1.165);
\draw[gp path] (8.407,8.631)--(8.407,8.451);
\node[gp node center] at (8.407,0.677) {$2$};
\draw[gp path] (10.087,0.985)--(10.087,1.165);
\draw[gp path] (10.087,8.631)--(10.087,8.451);
\node[gp node center] at (10.087,0.677) {$2.5$};
\draw[gp path] (11.767,0.985)--(11.767,1.165);
\draw[gp path] (11.767,8.631)--(11.767,8.451);
\node[gp node center] at (11.767,0.677) {$3$};
\draw[gp path] (13.447,0.985)--(13.447,1.165);
\draw[gp path] (13.447,8.631)--(13.447,8.451);
\node[gp node center] at (13.447,0.677) {$3.5$};
\draw[gp path] (1.688,8.631)--(1.688,0.985)--(13.447,0.985)--(13.447,8.631)--cycle;
\node[gp node center,rotate=-270] at (0.246,4.808) {$\omega$ ($\rm{MeV}/\rm{fm}^{3}$)};
\node[gp node center] at (7.567,0.215) {$\rho$ ($\rm{fm}^{-3}$)};
\gpcolor{rgb color={0.580,0.000,0.827}}
\draw[gp path] (1.732,8.151)--(1.742,8.151)--(1.752,8.151)--(1.762,8.151)--(1.772,8.151)%
  --(1.782,8.151)--(1.792,8.151)--(1.802,8.151)--(1.812,8.151)--(1.822,8.151)--(1.832,8.151)%
  --(1.842,8.151)--(1.852,8.151)--(1.862,8.151)--(1.872,8.152)--(1.882,8.152)--(1.893,8.152)%
  --(1.903,8.152)--(1.913,8.152)--(1.923,8.152)--(1.933,8.152)--(1.943,8.152)--(1.953,8.152)%
  --(1.963,8.152)--(1.973,8.152)--(1.983,8.152)--(1.993,8.152)--(2.003,8.152)--(2.013,8.152)%
  --(2.023,8.152)--(2.033,8.152)--(2.043,8.152)--(2.053,8.152)--(2.063,8.152)--(2.074,8.152)%
  --(2.084,8.152)--(2.094,8.152)--(2.104,8.152)--(2.114,8.152)--(2.124,8.152)--(2.134,8.152)%
  --(2.144,8.152)--(2.154,8.152)--(2.164,8.152)--(2.174,8.152)--(2.184,8.151)--(2.194,8.151)%
  --(2.204,8.151)--(2.214,8.151)--(2.224,8.151)--(2.234,8.150)--(2.244,8.150)--(2.255,8.150)%
  --(2.265,8.150)--(2.275,8.150)--(2.285,8.149)--(2.295,8.149)--(2.305,8.149)--(2.315,8.148)%
  --(2.325,8.148)--(2.335,8.148)--(2.345,8.147)--(2.355,8.147)--(2.365,8.146)--(2.375,8.146)%
  --(2.385,8.146)--(2.395,8.145)--(2.405,8.145)--(2.415,8.144)--(2.425,8.144)--(2.436,8.143)%
  --(2.446,8.142)--(2.456,8.142)--(2.466,8.141)--(2.476,8.141)--(2.486,8.140)--(2.496,8.139)%
  --(2.506,8.139)--(2.516,8.138)--(2.526,8.137)--(2.536,8.137)--(2.546,8.136)--(2.556,8.135)%
  --(2.566,8.134)--(2.576,8.133)--(2.586,8.132)--(2.596,8.132)--(2.606,8.131)--(2.617,8.130)%
  --(2.627,8.129)--(2.637,8.128)--(2.647,8.127)--(2.657,8.126)--(2.667,8.125)--(2.677,8.124)%
  --(2.687,8.123)--(2.697,8.122)--(2.707,8.121)--(2.717,8.120)--(2.727,8.119)--(2.737,8.118)%
  --(2.747,8.116)--(2.757,8.115)--(2.767,8.114)--(2.777,8.113)--(2.787,8.111)--(2.798,8.110)%
  --(2.808,8.109)--(2.818,8.108)--(2.828,8.106)--(2.838,8.105)--(2.848,8.104)--(2.858,8.102)%
  --(2.868,8.101)--(2.878,8.099)--(2.888,8.098)--(2.898,8.096)--(2.908,8.095)--(2.918,8.093)%
  --(2.928,8.092)--(2.938,8.090)--(2.948,8.089)--(2.958,8.087)--(2.968,8.086)--(2.979,8.084)%
  --(2.989,8.082)--(2.999,8.080)--(3.009,8.079)--(3.019,8.077)--(3.029,8.075)--(3.039,8.074)%
  --(3.049,8.072)--(3.059,8.070)--(3.069,8.068)--(3.079,8.066)--(3.089,8.065)--(3.099,8.063)%
  --(3.109,8.061)--(3.119,8.059)--(3.129,8.057)--(3.139,8.055)--(3.149,8.053)--(3.160,8.051)%
  --(3.170,8.049)--(3.180,8.047)--(3.190,8.045)--(3.200,8.043)--(3.210,8.041)--(3.220,8.039)%
  --(3.230,8.037)--(3.240,8.035)--(3.250,8.033)--(3.260,8.031)--(3.270,8.029)--(3.280,8.027)%
  --(3.290,8.025)--(3.300,8.022)--(3.310,8.020)--(3.320,8.018)--(3.330,8.016)--(3.341,8.014)%
  --(3.351,8.011)--(3.361,8.009)--(3.371,8.007)--(3.381,8.005)--(3.391,8.002)--(3.401,8.000)%
  --(3.411,7.998)--(3.421,7.996)--(3.431,7.993)--(3.441,7.991)--(3.451,7.989)--(3.461,7.986)%
  --(3.471,7.984)--(3.481,7.982)--(3.491,7.979)--(3.501,7.977)--(3.511,7.975)--(3.522,7.972)%
  --(3.532,7.970)--(3.542,7.968)--(3.552,7.965)--(3.562,7.963)--(3.572,7.960)--(3.582,7.958)%
  --(3.592,7.956)--(3.602,7.953)--(3.612,7.951)--(3.622,7.948)--(3.632,7.946)--(3.642,7.944)%
  --(3.652,7.941)--(3.662,7.939)--(3.672,7.937)--(3.682,7.934)--(3.692,7.932)--(3.703,7.929)%
  --(3.713,7.927)--(3.723,7.924)--(3.733,7.922)--(3.743,7.920)--(3.753,7.917)--(3.763,7.915)%
  --(3.773,7.913)--(3.783,7.910)--(3.793,7.908)--(3.803,7.906)--(3.813,7.903)--(3.823,7.901)%
  --(3.833,7.898)--(3.843,7.896)--(3.853,7.894)--(3.863,7.892)--(3.873,7.889)--(3.884,7.887)%
  --(3.894,7.885)--(3.904,7.882)--(3.914,7.880)--(3.924,7.878)--(3.934,7.876)--(3.944,7.873)%
  --(3.954,7.871)--(3.964,7.869)--(3.974,7.867)--(3.984,7.865)--(3.994,7.863)--(4.004,7.861)%
  --(4.014,7.858)--(4.024,7.856)--(4.034,7.854)--(4.044,7.852)--(4.054,7.850)--(4.065,7.848)%
  --(4.075,7.846)--(4.085,7.844)--(4.095,7.842)--(4.105,7.840)--(4.115,7.838)--(4.125,7.836)%
  --(4.135,7.834)--(4.145,7.832)--(4.155,7.830)--(4.165,7.829)--(4.175,7.827)--(4.185,7.825)%
  --(4.195,7.823)--(4.205,7.822)--(4.215,7.820)--(4.225,7.818)--(4.235,7.816)--(4.246,7.815)%
  --(4.256,7.813)--(4.266,7.812)--(4.276,7.810)--(4.286,7.808)--(4.296,7.807)--(4.306,7.805)%
  --(4.316,7.804)--(4.326,7.802)--(4.336,7.801)--(4.346,7.800)--(4.356,7.798)--(4.366,7.797)%
  --(4.376,7.795)--(4.386,7.794)--(4.396,7.793)--(4.406,7.792)--(4.416,7.790)--(4.427,7.789)%
  --(4.437,7.788)--(4.447,7.787)--(4.457,7.786)--(4.467,7.785)--(4.477,7.784)--(4.487,7.783)%
  --(4.497,7.782)--(4.507,7.781)--(4.517,7.780)--(4.527,7.779)--(4.537,7.778)--(4.547,7.777)%
  --(4.557,7.776)--(4.567,7.775)--(4.577,7.775)--(4.587,7.774)--(4.597,7.773)--(4.608,7.773)%
  --(4.618,7.772)--(4.628,7.771)--(4.638,7.770)--(4.648,7.770)--(4.658,7.769)--(4.668,7.769)%
  --(4.678,7.768)--(4.688,7.767)--(4.698,7.767)--(4.708,7.766)--(4.718,7.766)--(4.728,7.765)%
  --(4.738,7.765)--(4.748,7.765)--(4.758,7.764)--(4.768,7.764)--(4.778,7.763)--(4.789,7.763)%
  --(4.799,7.763)--(4.809,7.762)--(4.819,7.762)--(4.829,7.762)--(4.839,7.761)--(4.849,7.761)%
  --(4.859,7.760)--(4.869,7.760)--(4.879,7.760)--(4.889,7.759)--(4.899,7.759)--(4.909,7.759)%
  --(4.919,7.758)--(4.929,7.758)--(4.939,7.758)--(4.949,7.757)--(4.960,7.757)--(4.970,7.757)%
  --(4.980,7.756)--(4.990,7.756)--(5.000,7.755)--(5.010,7.755)--(5.020,7.754)--(5.030,7.754)%
  --(5.040,7.754)--(5.050,7.753)--(5.060,7.753)--(5.070,7.752)--(5.080,7.751)--(5.090,7.751)%
  --(5.100,7.750)--(5.110,7.750)--(5.120,7.749)--(5.130,7.748)--(5.141,7.747)--(5.151,7.747)%
  --(5.161,7.746)--(5.171,7.745)--(5.181,7.744)--(5.191,7.743)--(5.201,7.742)--(5.211,7.741)%
  --(5.221,7.740)--(5.231,7.739)--(5.241,7.738)--(5.251,7.737)--(5.261,7.736)--(5.271,7.735)%
  --(5.281,7.733)--(5.291,7.732)--(5.301,7.731)--(5.311,7.729)--(5.322,7.728)--(5.332,7.726)%
  --(5.342,7.725)--(5.352,7.723)--(5.362,7.721)--(5.372,7.720)--(5.382,7.718)--(5.392,7.716)%
  --(5.402,7.714)--(5.412,7.712)--(5.422,7.710)--(5.432,7.708)--(5.442,7.706)--(5.452,7.704)%
  --(5.462,7.702)--(5.472,7.700)--(5.482,7.698)--(5.492,7.695)--(5.503,7.693)--(5.513,7.690)%
  --(5.523,7.688)--(5.533,7.685)--(5.543,7.683)--(5.553,7.680)--(5.563,7.678)--(5.573,7.675)%
  --(5.583,7.672)--(5.593,7.669)--(5.603,7.666)--(5.613,7.663)--(5.623,7.660)--(5.633,7.657)%
  --(5.643,7.654)--(5.653,7.651)--(5.663,7.648)--(5.673,7.645)--(5.684,7.641)--(5.694,7.638)%
  --(5.704,7.635)--(5.714,7.631)--(5.724,7.628)--(5.734,7.624)--(5.744,7.621)--(5.754,7.617)%
  --(5.764,7.613)--(5.774,7.610)--(5.784,7.606)--(5.794,7.602)--(5.804,7.598)--(5.814,7.594)%
  --(5.824,7.590)--(5.834,7.586)--(5.844,7.582)--(5.854,7.578)--(5.865,7.574)--(5.875,7.570)%
  --(5.885,7.566)--(5.895,7.561)--(5.905,7.557)--(5.915,7.553)--(5.925,7.548)--(5.935,7.544)%
  --(5.945,7.540)--(5.955,7.535)--(5.965,7.531)--(5.975,7.526)--(5.985,7.521)--(5.995,7.517)%
  --(6.005,7.512)--(6.015,7.507)--(6.025,7.503)--(6.035,7.498)--(6.046,7.493)--(6.056,7.488)%
  --(6.066,7.483)--(6.076,7.478)--(6.086,7.473)--(6.096,7.468)--(6.106,7.463)--(6.116,7.458)%
  --(6.126,7.453)--(6.136,7.448)--(6.146,7.443)--(6.156,7.437)--(6.166,7.432)--(6.176,7.427)%
  --(6.186,7.422)--(6.196,7.416)--(6.206,7.411)--(6.216,7.406)--(6.227,7.400)--(6.237,7.395)%
  --(6.247,7.389)--(6.257,7.384)--(6.267,7.378)--(6.277,7.372)--(6.287,7.367)--(6.297,7.361)%
  --(6.307,7.356)--(6.317,7.350)--(6.327,7.344)--(6.337,7.338)--(6.347,7.333)--(6.357,7.327)%
  --(6.367,7.321)--(6.377,7.315)--(6.387,7.309)--(6.397,7.303)--(6.408,7.297)--(6.418,7.291)%
  --(6.428,7.285)--(6.438,7.279)--(6.448,7.273)--(6.458,7.267)--(6.468,7.261)--(6.478,7.255)%
  --(6.488,7.249)--(6.498,7.243)--(6.508,7.236)--(6.518,7.230)--(6.528,7.224)--(6.538,7.218)%
  --(6.548,7.211)--(6.558,7.205)--(6.568,7.199)--(6.578,7.192)--(6.589,7.186)--(6.599,7.179)%
  --(6.609,7.173)--(6.619,7.166)--(6.629,7.160)--(6.639,7.153)--(6.649,7.147)--(6.659,7.140)%
  --(6.669,7.134)--(6.679,7.127)--(6.689,7.121)--(6.699,7.114)--(6.709,7.107)--(6.719,7.101)%
  --(6.729,7.094)--(6.739,7.087)--(6.749,7.080)--(6.759,7.074)--(6.770,7.067)--(6.780,7.060)%
  --(6.790,7.053)--(6.800,7.046)--(6.810,7.040)--(6.820,7.033)--(6.830,7.026)--(6.840,7.019)%
  --(6.850,7.012)--(6.860,7.005)--(6.870,6.998)--(6.880,6.991)--(6.890,6.984)--(6.900,6.977)%
  --(6.910,6.970)--(6.920,6.963)--(6.930,6.956)--(6.940,6.949)--(6.951,6.941)--(6.961,6.934)%
  --(6.971,6.927)--(6.981,6.920)--(6.991,6.913)--(7.001,6.905)--(7.011,6.898)--(7.021,6.891)%
  --(7.031,6.884)--(7.041,6.876)--(7.051,6.869)--(7.061,6.862)--(7.071,6.854)--(7.081,6.847)%
  --(7.091,6.840)--(7.101,6.832)--(7.111,6.825)--(7.121,6.817)--(7.132,6.810)--(7.142,6.802)%
  --(7.152,6.795)--(7.162,6.787)--(7.172,6.780)--(7.182,6.772)--(7.192,6.765)--(7.202,6.757)%
  --(7.212,6.749)--(7.222,6.742)--(7.232,6.734)--(7.242,6.727)--(7.252,6.719)--(7.262,6.711)%
  --(7.272,6.704)--(7.282,6.696)--(7.292,6.688)--(7.302,6.680)--(7.313,6.673)--(7.323,6.665)%
  --(7.333,6.657)--(7.343,6.649)--(7.353,6.641)--(7.363,6.634)--(7.373,6.626)--(7.383,6.618)%
  --(7.393,6.610)--(7.403,6.602)--(7.413,6.594)--(7.423,6.586)--(7.433,6.578)--(7.443,6.570)%
  --(7.453,6.562)--(7.463,6.554)--(7.473,6.546)--(7.483,6.538)--(7.494,6.530)--(7.504,6.522)%
  --(7.514,6.514)--(7.524,6.506)--(7.534,6.498)--(7.544,6.490)--(7.554,6.482)--(7.564,6.474)%
  --(7.574,6.465)--(7.584,6.457)--(7.594,6.449)--(7.604,6.441)--(7.614,6.433)--(7.624,6.424)%
  --(7.634,6.416)--(7.644,6.408)--(7.654,6.400)--(7.664,6.391)--(7.675,6.383)--(7.685,6.375)%
  --(7.695,6.366)--(7.705,6.358)--(7.715,6.350)--(7.725,6.341)--(7.735,6.333)--(7.745,6.325)%
  --(7.755,6.316)--(7.765,6.308)--(7.775,6.299)--(7.785,6.291)--(7.795,6.282)--(7.805,6.274)%
  --(7.815,6.265)--(7.825,6.257)--(7.835,6.248)--(7.845,6.240)--(7.856,6.231)--(7.866,6.223)%
  --(7.876,6.214)--(7.886,6.205)--(7.896,6.197)--(7.906,6.188)--(7.916,6.180)--(7.926,6.171)%
  --(7.936,6.162)--(7.946,6.154)--(7.956,6.145)--(7.966,6.136)--(7.976,6.127)--(7.986,6.119)%
  --(7.996,6.110)--(8.006,6.101)--(8.016,6.092)--(8.026,6.083)--(8.037,6.075)--(8.047,6.066)%
  --(8.057,6.057)--(8.067,6.048)--(8.077,6.039)--(8.087,6.030)--(8.097,6.022)--(8.107,6.013)%
  --(8.117,6.004)--(8.127,5.995)--(8.137,5.986)--(8.147,5.977)--(8.157,5.968)--(8.167,5.959)%
  --(8.177,5.950)--(8.187,5.941)--(8.197,5.932)--(8.207,5.923)--(8.218,5.914)--(8.228,5.905)%
  --(8.238,5.896)--(8.248,5.887)--(8.258,5.877)--(8.268,5.868)--(8.278,5.859)--(8.288,5.850)%
  --(8.298,5.841)--(8.308,5.832)--(8.318,5.822)--(8.328,5.813)--(8.338,5.804)--(8.348,5.795)%
  --(8.358,5.786)--(8.368,5.776)--(8.378,5.767)--(8.388,5.758)--(8.399,5.748)--(8.409,5.739)%
  --(8.419,5.730)--(8.429,5.721)--(8.439,5.711)--(8.449,5.702)--(8.459,5.693)--(8.469,5.683)%
  --(8.479,5.674)--(8.489,5.664)--(8.499,5.655)--(8.509,5.645)--(8.519,5.636)--(8.529,5.627)%
  --(8.539,5.617)--(8.549,5.608)--(8.559,5.598)--(8.569,5.589)--(8.580,5.579)--(8.590,5.569)%
  --(8.600,5.560)--(8.610,5.550)--(8.620,5.541)--(8.630,5.531)--(8.640,5.522)--(8.650,5.512)%
  --(8.660,5.502)--(8.670,5.493)--(8.680,5.483)--(8.690,5.473)--(8.700,5.464)--(8.710,5.454)%
  --(8.720,5.444)--(8.730,5.435)--(8.740,5.425)--(8.750,5.415)--(8.761,5.405)--(8.771,5.396)%
  --(8.781,5.386)--(8.791,5.376)--(8.801,5.366)--(8.811,5.356)--(8.821,5.347)--(8.831,5.337)%
  --(8.841,5.327)--(8.851,5.317)--(8.861,5.307)--(8.871,5.297)--(8.881,5.287)--(8.891,5.277)%
  --(8.901,5.267)--(8.911,5.257)--(8.921,5.247)--(8.931,5.238)--(8.942,5.228)--(8.952,5.218)%
  --(8.962,5.207)--(8.972,5.197)--(8.982,5.187)--(8.992,5.177)--(9.002,5.167)--(9.012,5.157)%
  --(9.022,5.147)--(9.032,5.137)--(9.042,5.127)--(9.052,5.117)--(9.062,5.107)--(9.072,5.096)%
  --(9.082,5.086)--(9.092,5.076)--(9.102,5.066)--(9.112,5.056)--(9.123,5.046)--(9.133,5.035)%
  --(9.143,5.025)--(9.153,5.015)--(9.163,5.005)--(9.173,4.994)--(9.183,4.984)--(9.193,4.974)%
  --(9.203,4.963)--(9.213,4.953)--(9.223,4.943)--(9.233,4.932)--(9.243,4.922)--(9.253,4.912)%
  --(9.263,4.901)--(9.273,4.891)--(9.283,4.880)--(9.293,4.870)--(9.304,4.860)--(9.314,4.849)%
  --(9.324,4.839)--(9.334,4.828)--(9.344,4.818)--(9.354,4.807)--(9.364,4.797)--(9.374,4.786)%
  --(9.384,4.776)--(9.394,4.765)--(9.404,4.754)--(9.414,4.744)--(9.424,4.733)--(9.434,4.723)%
  --(9.444,4.712)--(9.454,4.701)--(9.464,4.691)--(9.474,4.680)--(9.485,4.669)--(9.495,4.659)%
  --(9.505,4.648)--(9.515,4.637)--(9.525,4.627)--(9.535,4.616)--(9.545,4.605)--(9.555,4.595)%
  --(9.565,4.584)--(9.575,4.573)--(9.585,4.562)--(9.595,4.551)--(9.605,4.541)--(9.615,4.530)%
  --(9.625,4.519)--(9.635,4.508)--(9.645,4.497)--(9.655,4.486)--(9.666,4.475)--(9.676,4.464)%
  --(9.686,4.454)--(9.696,4.443)--(9.706,4.432)--(9.716,4.421)--(9.726,4.410)--(9.736,4.399)%
  --(9.746,4.388)--(9.756,4.377)--(9.766,4.366)--(9.776,4.355)--(9.786,4.344)--(9.796,4.333)%
  --(9.806,4.322)--(9.816,4.311)--(9.826,4.299)--(9.836,4.288)--(9.847,4.277)--(9.857,4.266)%
  --(9.867,4.255)--(9.877,4.244)--(9.887,4.233)--(9.897,4.221)--(9.907,4.210)--(9.917,4.199)%
  --(9.927,4.188)--(9.937,4.177)--(9.947,4.165)--(9.957,4.154)--(9.967,4.143)--(9.977,4.132)%
  --(9.987,4.120)--(9.997,4.109)--(10.007,4.098)--(10.017,4.086)--(10.028,4.075)--(10.038,4.064)%
  --(10.048,4.052)--(10.058,4.041)--(10.068,4.030)--(10.078,4.018)--(10.088,4.007)--(10.098,3.995)%
  --(10.108,3.984)--(10.118,3.972)--(10.128,3.961)--(10.138,3.950)--(10.148,3.938)--(10.158,3.927)%
  --(10.168,3.915)--(10.178,3.903)--(10.188,3.892)--(10.198,3.880)--(10.209,3.869)--(10.219,3.857)%
  --(10.229,3.846)--(10.239,3.834)--(10.249,3.822)--(10.259,3.811)--(10.269,3.799)--(10.279,3.788)%
  --(10.289,3.776)--(10.299,3.764)--(10.309,3.753)--(10.319,3.741)--(10.329,3.729)--(10.339,3.717)%
  --(10.349,3.706)--(10.359,3.694)--(10.369,3.682)--(10.379,3.670)--(10.390,3.659)--(10.400,3.647)%
  --(10.410,3.635)--(10.420,3.623)--(10.430,3.611)--(10.440,3.599)--(10.450,3.588)--(10.460,3.576)%
  --(10.470,3.564)--(10.480,3.552)--(10.490,3.540)--(10.500,3.528)--(10.510,3.516)--(10.520,3.504)%
  --(10.530,3.492)--(10.540,3.481)--(10.550,3.468)--(10.560,3.456)--(10.571,3.445)--(10.581,3.432)%
  --(10.591,3.420)--(10.601,3.408)--(10.611,3.396)--(10.621,3.384)--(10.631,3.372)--(10.641,3.360)%
  --(10.651,3.348)--(10.661,3.336)--(10.671,3.324)--(10.681,3.312)--(10.691,3.299)--(10.701,3.287)%
  --(10.711,3.275)--(10.721,3.263)--(10.731,3.251)--(10.741,3.239)--(10.752,3.226)--(10.762,3.214)%
  --(10.772,3.202)--(10.782,3.189)--(10.792,3.177)--(10.802,3.165)--(10.812,3.153)--(10.822,3.140)%
  --(10.832,3.128)--(10.842,3.116)--(10.852,3.103)--(10.862,3.091)--(10.872,3.078)--(10.882,3.066)%
  --(10.892,3.054)--(10.902,3.041)--(10.912,3.029)--(10.922,3.017)--(10.933,3.004)--(10.943,2.992)%
  --(10.953,2.979)--(10.963,2.967)--(10.973,2.954)--(10.983,2.942)--(10.993,2.929)--(11.003,2.917)%
  --(11.013,2.904)--(11.023,2.892)--(11.033,2.879)--(11.043,2.866)--(11.053,2.854)--(11.063,2.841)%
  --(11.073,2.828)--(11.083,2.816)--(11.093,2.803)--(11.103,2.791)--(11.114,2.778)--(11.124,2.765)%
  --(11.134,2.752)--(11.144,2.740)--(11.154,2.727)--(11.164,2.714)--(11.174,2.702)--(11.184,2.689)%
  --(11.194,2.676)--(11.204,2.663)--(11.214,2.650)--(11.224,2.638)--(11.234,2.625)--(11.244,2.612)%
  --(11.254,2.599)--(11.264,2.586)--(11.274,2.573)--(11.284,2.561)--(11.295,2.548)--(11.305,2.535)%
  --(11.315,2.522)--(11.325,2.509)--(11.335,2.496)--(11.345,2.483)--(11.355,2.470)--(11.365,2.457)%
  --(11.375,2.444)--(11.385,2.431)--(11.395,2.418)--(11.405,2.405)--(11.415,2.392)--(11.425,2.379)%
  --(11.435,2.366)--(11.445,2.353)--(11.455,2.340)--(11.465,2.327)--(11.476,2.314)--(11.486,2.300)%
  --(11.496,2.287)--(11.506,2.274)--(11.516,2.261)--(11.526,2.248)--(11.536,2.235)--(11.546,2.222)%
  --(11.556,2.208)--(11.566,2.195)--(11.576,2.182)--(11.586,2.168)--(11.596,2.155)--(11.606,2.142)%
  --(11.616,2.129)--(11.626,2.115)--(11.636,2.102)--(11.646,2.089)--(11.657,2.076)--(11.667,2.062)%
  --(11.677,2.049)--(11.687,2.035)--(11.697,2.022)--(11.707,2.009)--(11.717,1.995)--(11.727,1.982)%
  --(11.737,1.968)--(11.747,1.955)--(11.757,1.942)--(11.767,1.928)--(11.777,1.914);
\gpcolor{color=gp lt color border}
\draw[gp path] (1.688,8.631)--(1.688,0.985)--(13.447,0.985)--(13.447,8.631)--cycle;
%% coordinates of the plot area
\gpdefrectangularnode{gp plot 1}{\pgfpoint{1.688cm}{0.985cm}}{\pgfpoint{13.447cm}{8.631cm}}
\end{tikzpicture}
%% gnuplot variables

	\caption{Gráfico do potencial grand canônico por unidade de volume obtido através da Equação~\eqref{Eq:potencial_termodinamico}. \protect[Parameters: eNJL1m; Proton fraction: 1/2] }
	\label{Fig:thermodynamic_potential_graph_eNJL1m}
\end{figure*}

\begin{figure*}
	\begin{tikzpicture}[gnuplot]
%% generated with GNUPLOT 5.0p2 (Lua 5.2; terminal rev. 99, script rev. 100)
%% Fri Mar  4 16:19:36 2016
\path (0.000,0.000) rectangle (14.000,9.000);
\gpcolor{color=gp lt color border}
\gpsetlinetype{gp lt border}
\gpsetdashtype{gp dt solid}
\gpsetlinewidth{1.00}
\draw[gp path] (1.504,0.985)--(1.684,0.985);
\draw[gp path] (13.447,0.985)--(13.267,0.985);
\node[gp node right] at (1.320,0.985) {$0$};
\draw[gp path] (1.504,1.941)--(1.684,1.941);
\draw[gp path] (13.447,1.941)--(13.267,1.941);
\node[gp node right] at (1.320,1.941) {$500$};
\draw[gp path] (1.504,2.897)--(1.684,2.897);
\draw[gp path] (13.447,2.897)--(13.267,2.897);
\node[gp node right] at (1.320,2.897) {$1000$};
\draw[gp path] (1.504,3.852)--(1.684,3.852);
\draw[gp path] (13.447,3.852)--(13.267,3.852);
\node[gp node right] at (1.320,3.852) {$1500$};
\draw[gp path] (1.504,4.808)--(1.684,4.808);
\draw[gp path] (13.447,4.808)--(13.267,4.808);
\node[gp node right] at (1.320,4.808) {$2000$};
\draw[gp path] (1.504,5.764)--(1.684,5.764);
\draw[gp path] (13.447,5.764)--(13.267,5.764);
\node[gp node right] at (1.320,5.764) {$2500$};
\draw[gp path] (1.504,6.720)--(1.684,6.720);
\draw[gp path] (13.447,6.720)--(13.267,6.720);
\node[gp node right] at (1.320,6.720) {$3000$};
\draw[gp path] (1.504,7.675)--(1.684,7.675);
\draw[gp path] (13.447,7.675)--(13.267,7.675);
\node[gp node right] at (1.320,7.675) {$3500$};
\draw[gp path] (1.504,8.631)--(1.684,8.631);
\draw[gp path] (13.447,8.631)--(13.267,8.631);
\node[gp node right] at (1.320,8.631) {$4000$};
\draw[gp path] (1.504,0.985)--(1.504,1.165);
\draw[gp path] (1.504,8.631)--(1.504,8.451);
\node[gp node center] at (1.504,0.677) {$0$};
\draw[gp path] (3.210,0.985)--(3.210,1.165);
\draw[gp path] (3.210,8.631)--(3.210,8.451);
\node[gp node center] at (3.210,0.677) {$0.5$};
\draw[gp path] (4.916,0.985)--(4.916,1.165);
\draw[gp path] (4.916,8.631)--(4.916,8.451);
\node[gp node center] at (4.916,0.677) {$1$};
\draw[gp path] (6.622,0.985)--(6.622,1.165);
\draw[gp path] (6.622,8.631)--(6.622,8.451);
\node[gp node center] at (6.622,0.677) {$1.5$};
\draw[gp path] (8.329,0.985)--(8.329,1.165);
\draw[gp path] (8.329,8.631)--(8.329,8.451);
\node[gp node center] at (8.329,0.677) {$2$};
\draw[gp path] (10.035,0.985)--(10.035,1.165);
\draw[gp path] (10.035,8.631)--(10.035,8.451);
\node[gp node center] at (10.035,0.677) {$2.5$};
\draw[gp path] (11.741,0.985)--(11.741,1.165);
\draw[gp path] (11.741,8.631)--(11.741,8.451);
\node[gp node center] at (11.741,0.677) {$3$};
\draw[gp path] (13.447,0.985)--(13.447,1.165);
\draw[gp path] (13.447,8.631)--(13.447,8.451);
\node[gp node center] at (13.447,0.677) {$3.5$};
\draw[gp path] (1.504,8.631)--(1.504,0.985)--(13.447,0.985)--(13.447,8.631)--cycle;
\node[gp node center,rotate=-270] at (0.246,4.808) {$P$ ($\rm{MeV}/\rm{fm}^3$)};
\node[gp node center] at (7.475,0.215) {$\rho$ ($\rm{fm}^{-3}$)};
\gpcolor{rgb color={0.580,0.000,0.827}}
\draw[gp path] (1.548,0.997)--(1.559,1.000)--(1.569,1.002)--(1.579,1.005)--(1.589,1.008)%
  --(1.599,1.010)--(1.610,1.013)--(1.620,1.016)--(1.630,1.018)--(1.640,1.021)--(1.650,1.023)%
  --(1.661,1.026)--(1.671,1.029)--(1.681,1.031)--(1.691,1.034)--(1.702,1.037)--(1.712,1.039)%
  --(1.722,1.042)--(1.732,1.044)--(1.742,1.047)--(1.753,1.050)--(1.763,1.052)--(1.773,1.055)%
  --(1.783,1.057)--(1.793,1.060)--(1.804,1.063)--(1.814,1.065)--(1.824,1.068)--(1.834,1.070)%
  --(1.845,1.073)--(1.855,1.076)--(1.865,1.078)--(1.875,1.081)--(1.885,1.084)--(1.896,1.086)%
  --(1.906,1.089)--(1.916,1.091)--(1.926,1.094)--(1.936,1.097)--(1.947,1.099)--(1.957,1.102)%
  --(1.967,1.105)--(1.977,1.108)--(1.987,1.110)--(1.998,1.113)--(2.008,1.116)--(2.018,1.118)%
  --(2.028,1.121)--(2.039,1.124)--(2.049,1.127)--(2.059,1.130)--(2.069,1.132)--(2.079,1.135)%
  --(2.090,1.138)--(2.100,1.141)--(2.110,1.144)--(2.120,1.146)--(2.130,1.149)--(2.141,1.152)%
  --(2.151,1.155)--(2.161,1.158)--(2.171,1.161)--(2.182,1.164)--(2.192,1.167)--(2.202,1.170)%
  --(2.212,1.173)--(2.222,1.176)--(2.233,1.179)--(2.243,1.182)--(2.253,1.185)--(2.263,1.188)%
  --(2.273,1.191)--(2.284,1.194)--(2.294,1.197)--(2.304,1.200)--(2.314,1.204)--(2.325,1.207)%
  --(2.335,1.210)--(2.345,1.213)--(2.355,1.216)--(2.365,1.220)--(2.376,1.223)--(2.386,1.226)%
  --(2.396,1.229)--(2.406,1.233)--(2.416,1.236)--(2.427,1.239)--(2.437,1.243)--(2.447,1.246)%
  --(2.457,1.250)--(2.468,1.253)--(2.478,1.256)--(2.488,1.260)--(2.498,1.263)--(2.508,1.267)%
  --(2.519,1.270)--(2.529,1.274)--(2.539,1.278)--(2.549,1.281)--(2.559,1.285)--(2.570,1.288)%
  --(2.580,1.292)--(2.590,1.296)--(2.600,1.299)--(2.610,1.303)--(2.621,1.307)--(2.631,1.310)%
  --(2.641,1.314)--(2.651,1.318)--(2.662,1.322)--(2.672,1.325)--(2.682,1.329)--(2.692,1.333)%
  --(2.702,1.337)--(2.713,1.341)--(2.723,1.345)--(2.733,1.349)--(2.743,1.353)--(2.753,1.357)%
  --(2.764,1.361)--(2.774,1.365)--(2.784,1.369)--(2.794,1.373)--(2.805,1.377)--(2.815,1.381)%
  --(2.825,1.385)--(2.835,1.389)--(2.845,1.393)--(2.856,1.397)--(2.866,1.401)--(2.876,1.406)%
  --(2.886,1.410)--(2.896,1.414)--(2.907,1.418)--(2.917,1.423)--(2.927,1.427)--(2.937,1.431)%
  --(2.948,1.435)--(2.958,1.440)--(2.968,1.444)--(2.978,1.448)--(2.988,1.453)--(2.999,1.457)%
  --(3.009,1.462)--(3.019,1.466)--(3.029,1.470)--(3.039,1.475)--(3.050,1.479)--(3.060,1.484)%
  --(3.070,1.488)--(3.080,1.493)--(3.090,1.497)--(3.101,1.502)--(3.111,1.506)--(3.121,1.511)%
  --(3.131,1.515)--(3.142,1.520)--(3.152,1.524)--(3.162,1.529)--(3.172,1.534)--(3.182,1.538)%
  --(3.193,1.543)--(3.203,1.548)--(3.213,1.552)--(3.223,1.557)--(3.233,1.562)--(3.244,1.566)%
  --(3.254,1.571)--(3.264,1.576)--(3.274,1.581)--(3.285,1.585)--(3.295,1.590)--(3.305,1.594)%
  --(3.315,1.599)--(3.325,1.604)--(3.336,1.609)--(3.346,1.613)--(3.356,1.618)--(3.366,1.623)%
  --(3.376,1.628)--(3.387,1.632)--(3.397,1.637)--(3.407,1.642)--(3.417,1.647)--(3.428,1.652)%
  --(3.438,1.656)--(3.448,1.661)--(3.458,1.666)--(3.468,1.671)--(3.479,1.676)--(3.489,1.680)%
  --(3.499,1.685)--(3.509,1.690)--(3.519,1.695)--(3.530,1.700)--(3.540,1.704)--(3.550,1.709)%
  --(3.560,1.714)--(3.570,1.719)--(3.581,1.723)--(3.591,1.728)--(3.601,1.733)--(3.611,1.738)%
  --(3.622,1.742)--(3.632,1.747)--(3.642,1.752)--(3.652,1.757)--(3.662,1.761)--(3.673,1.766)%
  --(3.683,1.771)--(3.693,1.775)--(3.703,1.780)--(3.713,1.785)--(3.724,1.790)--(3.734,1.794)%
  --(3.744,1.799)--(3.754,1.804)--(3.765,1.808)--(3.775,1.813)--(3.785,1.818)--(3.795,1.822)%
  --(3.805,1.827)--(3.816,1.831)--(3.826,1.836)--(3.836,1.840)--(3.846,1.844)--(3.856,1.849)%
  --(3.867,1.854)--(3.877,1.858)--(3.887,1.862)--(3.897,1.867)--(3.908,1.871)--(3.918,1.876)%
  --(3.928,1.880)--(3.938,1.884)--(3.948,1.889)--(3.959,1.893)--(3.969,1.897)--(3.979,1.902)%
  --(3.989,1.906)--(3.999,1.910)--(4.010,1.914)--(4.020,1.918)--(4.030,1.923)--(4.040,1.927)%
  --(4.051,1.931)--(4.061,1.935)--(4.071,1.939)--(4.081,1.943)--(4.091,1.947)--(4.102,1.951)%
  --(4.112,1.955)--(4.122,1.959)--(4.132,1.962)--(4.142,1.966)--(4.153,1.970)--(4.163,1.974)%
  --(4.173,1.978)--(4.183,1.981)--(4.193,1.985)--(4.204,1.989)--(4.214,1.992)--(4.224,1.996)%
  --(4.234,2.000)--(4.245,2.003)--(4.255,2.007)--(4.265,2.010)--(4.275,2.014)--(4.285,2.017)%
  --(4.296,2.020)--(4.306,2.024)--(4.316,2.027)--(4.326,2.030)--(4.336,2.034)--(4.347,2.037)%
  --(4.357,2.040)--(4.367,2.043)--(4.377,2.046)--(4.388,2.050)--(4.398,2.053)--(4.408,2.056)%
  --(4.418,2.059)--(4.428,2.062)--(4.439,2.064)--(4.449,2.067)--(4.459,2.070)--(4.469,2.073)%
  --(4.479,2.076)--(4.490,2.079)--(4.500,2.081)--(4.510,2.084)--(4.520,2.087)--(4.531,2.089)%
  --(4.541,2.092)--(4.551,2.095)--(4.561,2.097)--(4.571,2.100)--(4.582,2.102)--(4.592,2.105)%
  --(4.602,2.107)--(4.612,2.110)--(4.622,2.112)--(4.633,2.114)--(4.643,2.117)--(4.653,2.119)%
  --(4.663,2.121)--(4.673,2.124)--(4.684,2.126)--(4.694,2.128)--(4.704,2.130)--(4.714,2.133)%
  --(4.725,2.135)--(4.735,2.137)--(4.745,2.139)--(4.755,2.141)--(4.765,2.144)--(4.776,2.146)%
  --(4.786,2.148)--(4.796,2.150)--(4.806,2.152)--(4.816,2.154)--(4.827,2.156)--(4.837,2.158)%
  --(4.847,2.160)--(4.857,2.162)--(4.868,2.165)--(4.878,2.167)--(4.888,2.169)--(4.898,2.171)%
  --(4.908,2.173)--(4.919,2.175)--(4.929,2.177)--(4.939,2.179)--(4.949,2.181)--(4.959,2.183)%
  --(4.970,2.185)--(4.980,2.187)--(4.990,2.189)--(5.000,2.191)--(5.011,2.194)--(5.021,2.196)%
  --(5.031,2.198)--(5.041,2.200)--(5.051,2.202)--(5.062,2.205)--(5.072,2.207)--(5.082,2.209)%
  --(5.092,2.211)--(5.102,2.214)--(5.113,2.216)--(5.123,2.218)--(5.133,2.221)--(5.143,2.223)%
  --(5.154,2.226)--(5.164,2.228)--(5.174,2.230)--(5.184,2.233)--(5.194,2.235)--(5.205,2.238)%
  --(5.215,2.241)--(5.225,2.243)--(5.235,2.246)--(5.245,2.249)--(5.256,2.251)--(5.266,2.254)%
  --(5.276,2.257)--(5.286,2.260)--(5.296,2.262)--(5.307,2.265)--(5.317,2.268)--(5.327,2.271)%
  --(5.337,2.274)--(5.348,2.277)--(5.358,2.280)--(5.368,2.283)--(5.378,2.286)--(5.388,2.290)%
  --(5.399,2.293)--(5.409,2.296)--(5.419,2.299)--(5.429,2.303)--(5.439,2.306)--(5.450,2.309)%
  --(5.460,2.313)--(5.470,2.316)--(5.480,2.320)--(5.491,2.323)--(5.501,2.327)--(5.511,2.330)%
  --(5.521,2.334)--(5.531,2.338)--(5.542,2.341)--(5.552,2.345)--(5.562,2.349)--(5.572,2.353)%
  --(5.582,2.357)--(5.593,2.360)--(5.603,2.364)--(5.613,2.368)--(5.623,2.372)--(5.634,2.376)%
  --(5.644,2.380)--(5.654,2.385)--(5.664,2.389)--(5.674,2.393)--(5.685,2.397)--(5.695,2.401)%
  --(5.705,2.406)--(5.715,2.410)--(5.725,2.414)--(5.736,2.419)--(5.746,2.423)--(5.756,2.427)%
  --(5.766,2.432)--(5.776,2.436)--(5.787,2.441)--(5.797,2.446)--(5.807,2.450)--(5.817,2.455)%
  --(5.828,2.459)--(5.838,2.464)--(5.848,2.469)--(5.858,2.474)--(5.868,2.478)--(5.879,2.483)%
  --(5.889,2.488)--(5.899,2.493)--(5.909,2.498)--(5.919,2.503)--(5.930,2.508)--(5.940,2.513)%
  --(5.950,2.518)--(5.960,2.523)--(5.971,2.528)--(5.981,2.533)--(5.991,2.538)--(6.001,2.543)%
  --(6.011,2.549)--(6.022,2.554)--(6.032,2.559)--(6.042,2.564)--(6.052,2.570)--(6.062,2.575)%
  --(6.073,2.580)--(6.083,2.586)--(6.093,2.591)--(6.103,2.597)--(6.114,2.602)--(6.124,2.608)%
  --(6.134,2.613)--(6.144,2.619)--(6.154,2.624)--(6.165,2.630)--(6.175,2.636)--(6.185,2.641)%
  --(6.195,2.647)--(6.205,2.653)--(6.216,2.658)--(6.226,2.664)--(6.236,2.670)--(6.246,2.676)%
  --(6.257,2.681)--(6.267,2.687)--(6.277,2.693)--(6.287,2.699)--(6.297,2.705)--(6.308,2.711)%
  --(6.318,2.717)--(6.328,2.723)--(6.338,2.729)--(6.348,2.735)--(6.359,2.741)--(6.369,2.747)%
  --(6.379,2.753)--(6.389,2.759)--(6.399,2.765)--(6.410,2.772)--(6.420,2.778)--(6.430,2.784)%
  --(6.440,2.790)--(6.451,2.796)--(6.461,2.803)--(6.471,2.809)--(6.481,2.815)--(6.491,2.822)%
  --(6.502,2.828)--(6.512,2.834)--(6.522,2.841)--(6.532,2.847)--(6.542,2.854)--(6.553,2.860)%
  --(6.563,2.866)--(6.573,2.873)--(6.583,2.879)--(6.594,2.886)--(6.604,2.893)--(6.614,2.899)%
  --(6.624,2.906)--(6.634,2.912)--(6.645,2.919)--(6.655,2.926)--(6.665,2.932)--(6.675,2.939)%
  --(6.685,2.946)--(6.696,2.952)--(6.706,2.959)--(6.716,2.966)--(6.726,2.973)--(6.737,2.979)%
  --(6.747,2.986)--(6.757,2.993)--(6.767,3.000)--(6.777,3.007)--(6.788,3.014)--(6.798,3.021)%
  --(6.808,3.028)--(6.818,3.034)--(6.828,3.041)--(6.839,3.048)--(6.849,3.055)--(6.859,3.062)%
  --(6.869,3.069)--(6.879,3.077)--(6.890,3.084)--(6.900,3.091)--(6.910,3.098)--(6.920,3.105)%
  --(6.931,3.112)--(6.941,3.119)--(6.951,3.126)--(6.961,3.134)--(6.971,3.141)--(6.982,3.148)%
  --(6.992,3.155)--(7.002,3.162)--(7.012,3.170)--(7.022,3.177)--(7.033,3.184)--(7.043,3.192)%
  --(7.053,3.199)--(7.063,3.206)--(7.074,3.214)--(7.084,3.221)--(7.094,3.228)--(7.104,3.236)%
  --(7.114,3.243)--(7.125,3.251)--(7.135,3.258)--(7.145,3.266)--(7.155,3.273)--(7.165,3.281)%
  --(7.176,3.288)--(7.186,3.296)--(7.196,3.303)--(7.206,3.311)--(7.217,3.319)--(7.227,3.326)%
  --(7.237,3.334)--(7.247,3.341)--(7.257,3.349)--(7.268,3.357)--(7.278,3.364)--(7.288,3.372)%
  --(7.298,3.380)--(7.308,3.388)--(7.319,3.395)--(7.329,3.403)--(7.339,3.411)--(7.349,3.419)%
  --(7.360,3.427)--(7.370,3.434)--(7.380,3.442)--(7.390,3.450)--(7.400,3.458)--(7.411,3.466)%
  --(7.421,3.474)--(7.431,3.482)--(7.441,3.490)--(7.451,3.498)--(7.462,3.505)--(7.472,3.513)%
  --(7.482,3.521)--(7.492,3.529)--(7.502,3.537)--(7.513,3.546)--(7.523,3.554)--(7.533,3.562)%
  --(7.543,3.570)--(7.554,3.578)--(7.564,3.586)--(7.574,3.594)--(7.584,3.602)--(7.594,3.610)%
  --(7.605,3.619)--(7.615,3.627)--(7.625,3.635)--(7.635,3.643)--(7.645,3.651)--(7.656,3.660)%
  --(7.666,3.668)--(7.676,3.676)--(7.686,3.684)--(7.697,3.693)--(7.707,3.701)--(7.717,3.709)%
  --(7.727,3.718)--(7.737,3.726)--(7.748,3.734)--(7.758,3.743)--(7.768,3.751)--(7.778,3.760)%
  --(7.788,3.768)--(7.799,3.776)--(7.809,3.785)--(7.819,3.793)--(7.829,3.802)--(7.840,3.810)%
  --(7.850,3.819)--(7.860,3.827)--(7.870,3.836)--(7.880,3.845)--(7.891,3.853)--(7.901,3.862)%
  --(7.911,3.870)--(7.921,3.879)--(7.931,3.888)--(7.942,3.896)--(7.952,3.905)--(7.962,3.913)%
  --(7.972,3.922)--(7.982,3.931)--(7.993,3.940)--(8.003,3.948)--(8.013,3.957)--(8.023,3.966)%
  --(8.034,3.975)--(8.044,3.983)--(8.054,3.992)--(8.064,4.001)--(8.074,4.010)--(8.085,4.019)%
  --(8.095,4.027)--(8.105,4.036)--(8.115,4.045)--(8.125,4.054)--(8.136,4.063)--(8.146,4.072)%
  --(8.156,4.081)--(8.166,4.090)--(8.177,4.099)--(8.187,4.108)--(8.197,4.117)--(8.207,4.126)%
  --(8.217,4.135)--(8.228,4.144)--(8.238,4.153)--(8.248,4.162)--(8.258,4.171)--(8.268,4.180)%
  --(8.279,4.189)--(8.289,4.198)--(8.299,4.207)--(8.309,4.216)--(8.320,4.225)--(8.330,4.235)%
  --(8.340,4.244)--(8.350,4.253)--(8.360,4.262)--(8.371,4.271)--(8.381,4.281)--(8.391,4.290)%
  --(8.401,4.299)--(8.411,4.308)--(8.422,4.318)--(8.432,4.327)--(8.442,4.336)--(8.452,4.346)%
  --(8.463,4.355)--(8.473,4.364)--(8.483,4.374)--(8.493,4.383)--(8.503,4.392)--(8.514,4.402)%
  --(8.524,4.411)--(8.534,4.421)--(8.544,4.430)--(8.554,4.439)--(8.565,4.449)--(8.575,4.458)%
  --(8.585,4.468)--(8.595,4.477)--(8.605,4.487)--(8.616,4.496)--(8.626,4.506)--(8.636,4.516)%
  --(8.646,4.525)--(8.657,4.535)--(8.667,4.544)--(8.677,4.554)--(8.687,4.563)--(8.697,4.573)%
  --(8.708,4.583)--(8.718,4.592)--(8.728,4.602)--(8.738,4.612)--(8.748,4.621)--(8.759,4.631)%
  --(8.769,4.641)--(8.779,4.651)--(8.789,4.660)--(8.800,4.670)--(8.810,4.680)--(8.820,4.690)%
  --(8.830,4.700)--(8.840,4.709)--(8.851,4.719)--(8.861,4.729)--(8.871,4.739)--(8.881,4.749)%
  --(8.891,4.759)--(8.902,4.769)--(8.912,4.778)--(8.922,4.788)--(8.932,4.798)--(8.943,4.808)%
  --(8.953,4.818)--(8.963,4.828)--(8.973,4.838)--(8.983,4.848)--(8.994,4.858)--(9.004,4.868)%
  --(9.014,4.878)--(9.024,4.888)--(9.034,4.898)--(9.045,4.908)--(9.055,4.919)--(9.065,4.929)%
  --(9.075,4.939)--(9.085,4.949)--(9.096,4.959)--(9.106,4.969)--(9.116,4.979)--(9.126,4.990)%
  --(9.137,5.000)--(9.147,5.010)--(9.157,5.020)--(9.167,5.030)--(9.177,5.041)--(9.188,5.051)%
  --(9.198,5.061)--(9.208,5.071)--(9.218,5.082)--(9.228,5.092)--(9.239,5.102)--(9.249,5.113)%
  --(9.259,5.123)--(9.269,5.133)--(9.280,5.144)--(9.290,5.154)--(9.300,5.165)--(9.310,5.175)%
  --(9.320,5.185)--(9.331,5.196)--(9.341,5.206)--(9.351,5.217)--(9.361,5.227)--(9.371,5.238)%
  --(9.382,5.248)--(9.392,5.259)--(9.402,5.269)--(9.412,5.280)--(9.423,5.290)--(9.433,5.301)%
  --(9.443,5.311)--(9.453,5.322)--(9.463,5.333)--(9.474,5.343)--(9.484,5.354)--(9.494,5.365)%
  --(9.504,5.375)--(9.514,5.386)--(9.525,5.397)--(9.535,5.407)--(9.545,5.418)--(9.555,5.429)%
  --(9.565,5.439)--(9.576,5.450)--(9.586,5.461)--(9.596,5.472)--(9.606,5.482)--(9.617,5.493)%
  --(9.627,5.504)--(9.637,5.515)--(9.647,5.526)--(9.657,5.537)--(9.668,5.547)--(9.678,5.558)%
  --(9.688,5.569)--(9.698,5.580)--(9.708,5.591)--(9.719,5.602)--(9.729,5.613)--(9.739,5.624)%
  --(9.749,5.635)--(9.760,5.646)--(9.770,5.657)--(9.780,5.668)--(9.790,5.679)--(9.800,5.690)%
  --(9.811,5.701)--(9.821,5.712)--(9.831,5.723)--(9.841,5.734)--(9.851,5.745)--(9.862,5.756)%
  --(9.872,5.767)--(9.882,5.778)--(9.892,5.789)--(9.903,5.801)--(9.913,5.812)--(9.923,5.823)%
  --(9.933,5.834)--(9.943,5.845)--(9.954,5.857)--(9.964,5.868)--(9.974,5.879)--(9.984,5.890)%
  --(9.994,5.902)--(10.005,5.913)--(10.015,5.924)--(10.025,5.935)--(10.035,5.947)--(10.046,5.958)%
  --(10.056,5.969)--(10.066,5.981)--(10.076,5.992)--(10.086,6.004)--(10.097,6.015)--(10.107,6.026)%
  --(10.117,6.038)--(10.127,6.049)--(10.137,6.061)--(10.148,6.072)--(10.158,6.083)--(10.168,6.095)%
  --(10.178,6.107)--(10.188,6.118)--(10.199,6.130)--(10.209,6.141)--(10.219,6.153)--(10.229,6.164)%
  --(10.240,6.176)--(10.250,6.187)--(10.260,6.199)--(10.270,6.210)--(10.280,6.222)--(10.291,6.234)%
  --(10.301,6.245)--(10.311,6.257)--(10.321,6.269)--(10.331,6.280)--(10.342,6.292)--(10.352,6.304)%
  --(10.362,6.315)--(10.372,6.327)--(10.383,6.339)--(10.393,6.351)--(10.403,6.362)--(10.413,6.374)%
  --(10.423,6.386)--(10.434,6.398)--(10.444,6.410)--(10.454,6.422)--(10.464,6.433)--(10.474,6.445)%
  --(10.485,6.457)--(10.495,6.469)--(10.505,6.481)--(10.515,6.493)--(10.526,6.505)--(10.536,6.517)%
  --(10.546,6.529)--(10.556,6.540)--(10.566,6.553)--(10.577,6.564)--(10.587,6.576)--(10.597,6.588)%
  --(10.607,6.600)--(10.617,6.612)--(10.628,6.625)--(10.638,6.637)--(10.648,6.649)--(10.658,6.661)%
  --(10.668,6.673)--(10.679,6.685)--(10.689,6.697)--(10.699,6.709)--(10.709,6.721)--(10.720,6.734)%
  --(10.730,6.746)--(10.740,6.758)--(10.750,6.770)--(10.760,6.782)--(10.771,6.795)--(10.781,6.807)%
  --(10.791,6.819)--(10.801,6.831)--(10.811,6.843)--(10.822,6.856)--(10.832,6.868)--(10.842,6.880)%
  --(10.852,6.893)--(10.863,6.905)--(10.873,6.917)--(10.883,6.930)--(10.893,6.942)--(10.903,6.954)%
  --(10.914,6.967)--(10.924,6.979)--(10.934,6.992)--(10.944,7.004)--(10.954,7.017)--(10.965,7.029)%
  --(10.975,7.042)--(10.985,7.054)--(10.995,7.066)--(11.006,7.079)--(11.016,7.092)--(11.026,7.104)%
  --(11.036,7.117)--(11.046,7.129)--(11.057,7.142)--(11.067,7.154)--(11.077,7.167)--(11.087,7.180)%
  --(11.097,7.192)--(11.108,7.205)--(11.118,7.217)--(11.128,7.230)--(11.138,7.243)--(11.149,7.255)%
  --(11.159,7.268)--(11.169,7.281)--(11.179,7.294)--(11.189,7.306)--(11.200,7.319)--(11.210,7.332)%
  --(11.220,7.345)--(11.230,7.357)--(11.240,7.370)--(11.251,7.383)--(11.261,7.396)--(11.271,7.409)%
  --(11.281,7.421)--(11.291,7.434)--(11.302,7.447)--(11.312,7.460)--(11.322,7.473)--(11.332,7.486)%
  --(11.343,7.499)--(11.353,7.512)--(11.363,7.525)--(11.373,7.538)--(11.383,7.551)--(11.394,7.564)%
  --(11.404,7.577)--(11.414,7.590)--(11.424,7.603)--(11.434,7.616)--(11.445,7.629)--(11.455,7.642)%
  --(11.465,7.655)--(11.475,7.668)--(11.486,7.681)--(11.496,7.694)--(11.506,7.707)--(11.516,7.720)%
  --(11.526,7.734)--(11.537,7.747)--(11.547,7.760)--(11.557,7.773)--(11.567,7.786)--(11.577,7.799)%
  --(11.588,7.813)--(11.598,7.826)--(11.608,7.839)--(11.618,7.852)--(11.629,7.866)--(11.639,7.879)%
  --(11.649,7.892)--(11.659,7.906)--(11.669,7.919)--(11.680,7.932)--(11.690,7.946)--(11.700,7.959)%
  --(11.710,7.972)--(11.720,7.986)--(11.731,7.999)--(11.741,8.012)--(11.751,8.026);
\gpcolor{color=gp lt color border}
\draw[gp path] (1.504,8.631)--(1.504,0.985)--(13.447,0.985)--(13.447,8.631)--cycle;
%% coordinates of the plot area
\gpdefrectangularnode{gp plot 1}{\pgfpoint{1.504cm}{0.985cm}}{\pgfpoint{13.447cm}{8.631cm}}
\end{tikzpicture}
%% gnuplot variables

	\caption{Gráfico da pressão obtido através da Equação~\eqref{Eq:Pressao}. Devido ao fato de que estamos usando $\varepsilon_o = 0$ por enquanto, a escala vertical do gráfico está deslocada no sentido positivo. \protect[Parameters: eNJL1m; Proton fraction: 1/2] }
	\label{Fig:pressure_graph_eNJL1m}
\end{figure*}

\begin{figure*}
	\begin{tikzpicture}[gnuplot]
%% generated with GNUPLOT 5.0p2 (Lua 5.2; terminal rev. 99, script rev. 100)
%% Mon Mar  7 16:11:42 2016
\path (0.000,0.000) rectangle (14.000,9.000);
\gpcolor{color=gp lt color border}
\gpsetlinetype{gp lt border}
\gpsetdashtype{gp dt solid}
\gpsetlinewidth{1.00}
\draw[gp path] (1.872,0.985)--(2.052,0.985);
\draw[gp path] (13.447,0.985)--(13.267,0.985);
\node[gp node right] at (1.688,0.985) {$-35000$};
\draw[gp path] (1.872,1.941)--(2.052,1.941);
\draw[gp path] (13.447,1.941)--(13.267,1.941);
\node[gp node right] at (1.688,1.941) {$-30000$};
\draw[gp path] (1.872,2.897)--(2.052,2.897);
\draw[gp path] (13.447,2.897)--(13.267,2.897);
\node[gp node right] at (1.688,2.897) {$-25000$};
\draw[gp path] (1.872,3.852)--(2.052,3.852);
\draw[gp path] (13.447,3.852)--(13.267,3.852);
\node[gp node right] at (1.688,3.852) {$-20000$};
\draw[gp path] (1.872,4.808)--(2.052,4.808);
\draw[gp path] (13.447,4.808)--(13.267,4.808);
\node[gp node right] at (1.688,4.808) {$-15000$};
\draw[gp path] (1.872,5.764)--(2.052,5.764);
\draw[gp path] (13.447,5.764)--(13.267,5.764);
\node[gp node right] at (1.688,5.764) {$-10000$};
\draw[gp path] (1.872,6.720)--(2.052,6.720);
\draw[gp path] (13.447,6.720)--(13.267,6.720);
\node[gp node right] at (1.688,6.720) {$-5000$};
\draw[gp path] (1.872,7.675)--(2.052,7.675);
\draw[gp path] (13.447,7.675)--(13.267,7.675);
\node[gp node right] at (1.688,7.675) {$0$};
\draw[gp path] (1.872,8.631)--(2.052,8.631);
\draw[gp path] (13.447,8.631)--(13.267,8.631);
\node[gp node right] at (1.688,8.631) {$5000$};
\draw[gp path] (1.872,0.985)--(1.872,1.165);
\draw[gp path] (1.872,8.631)--(1.872,8.451);
\node[gp node center] at (1.872,0.677) {$0$};
\draw[gp path] (3.526,0.985)--(3.526,1.165);
\draw[gp path] (3.526,8.631)--(3.526,8.451);
\node[gp node center] at (3.526,0.677) {$0.5$};
\draw[gp path] (5.179,0.985)--(5.179,1.165);
\draw[gp path] (5.179,8.631)--(5.179,8.451);
\node[gp node center] at (5.179,0.677) {$1$};
\draw[gp path] (6.833,0.985)--(6.833,1.165);
\draw[gp path] (6.833,8.631)--(6.833,8.451);
\node[gp node center] at (6.833,0.677) {$1.5$};
\draw[gp path] (8.486,0.985)--(8.486,1.165);
\draw[gp path] (8.486,8.631)--(8.486,8.451);
\node[gp node center] at (8.486,0.677) {$2$};
\draw[gp path] (10.140,0.985)--(10.140,1.165);
\draw[gp path] (10.140,8.631)--(10.140,8.451);
\node[gp node center] at (10.140,0.677) {$2.5$};
\draw[gp path] (11.793,0.985)--(11.793,1.165);
\draw[gp path] (11.793,8.631)--(11.793,8.451);
\node[gp node center] at (11.793,0.677) {$3$};
\draw[gp path] (13.447,0.985)--(13.447,1.165);
\draw[gp path] (13.447,8.631)--(13.447,8.451);
\node[gp node center] at (13.447,0.677) {$3.5$};
\draw[gp path] (1.872,8.631)--(1.872,0.985)--(13.447,0.985)--(13.447,8.631)--cycle;
\node[gp node center,rotate=-270] at (0.246,4.808) {$E/A = \varepsilon/\rho$ (MeV)};
\node[gp node center] at (7.659,0.215) {$\rho$ $\rm{fm}^{-3}$};
\gpcolor{rgb color={0.580,0.000,0.827}}
\draw[gp path] (1.915,1.925)--(1.925,3.035)--(1.935,3.795)--(1.945,4.348)--(1.955,4.768)%
  --(1.964,5.098)--(1.974,5.365)--(1.984,5.584)--(1.994,5.768)--(2.004,5.924)--(2.014,6.059)%
  --(2.024,6.175)--(2.034,6.278)--(2.044,6.369)--(2.054,6.450)--(2.063,6.522)--(2.073,6.587)%
  --(2.083,6.647)--(2.093,6.701)--(2.103,6.750)--(2.113,6.795)--(2.123,6.837)--(2.133,6.875)%
  --(2.143,6.911)--(2.153,6.944)--(2.162,6.975)--(2.172,7.004)--(2.182,7.031)--(2.192,7.056)%
  --(2.202,7.080)--(2.212,7.102)--(2.222,7.124)--(2.232,7.144)--(2.242,7.163)--(2.252,7.181)%
  --(2.261,7.198)--(2.271,7.214)--(2.281,7.229)--(2.291,7.244)--(2.301,7.258)--(2.311,7.271)%
  --(2.321,7.284)--(2.331,7.296)--(2.341,7.308)--(2.350,7.319)--(2.360,7.330)--(2.370,7.340)%
  --(2.380,7.350)--(2.390,7.360)--(2.400,7.369)--(2.410,7.378)--(2.420,7.386)--(2.430,7.395)%
  --(2.440,7.403)--(2.449,7.410)--(2.459,7.418)--(2.469,7.425)--(2.479,7.432)--(2.489,7.439)%
  --(2.499,7.445)--(2.509,7.452)--(2.519,7.458)--(2.529,7.464)--(2.539,7.470)--(2.548,7.475)%
  --(2.558,7.481)--(2.568,7.486)--(2.578,7.491)--(2.588,7.496)--(2.598,7.501)--(2.608,7.506)%
  --(2.618,7.511)--(2.628,7.515)--(2.638,7.519)--(2.647,7.524)--(2.657,7.528)--(2.667,7.532)%
  --(2.677,7.536)--(2.687,7.540)--(2.697,7.544)--(2.707,7.547)--(2.717,7.551)--(2.727,7.555)%
  --(2.737,7.558)--(2.746,7.562)--(2.756,7.565)--(2.766,7.568)--(2.776,7.571)--(2.786,7.575)%
  --(2.796,7.578)--(2.806,7.581)--(2.816,7.584)--(2.826,7.586)--(2.836,7.589)--(2.845,7.592)%
  --(2.855,7.595)--(2.865,7.597)--(2.875,7.600)--(2.885,7.603)--(2.895,7.605)--(2.905,7.608)%
  --(2.915,7.610)--(2.925,7.613)--(2.934,7.615)--(2.944,7.617)--(2.954,7.619)--(2.964,7.622)%
  --(2.974,7.624)--(2.984,7.626)--(2.994,7.628)--(3.004,7.630)--(3.014,7.632)--(3.024,7.634)%
  --(3.033,7.636)--(3.043,7.638)--(3.053,7.640)--(3.063,7.642)--(3.073,7.644)--(3.083,7.646)%
  --(3.093,7.648)--(3.103,7.650)--(3.113,7.651)--(3.123,7.653)--(3.132,7.655)--(3.142,7.657)%
  --(3.152,7.658)--(3.162,7.660)--(3.172,7.662)--(3.182,7.663)--(3.192,7.665)--(3.202,7.666)%
  --(3.212,7.668)--(3.222,7.670)--(3.231,7.671)--(3.241,7.673)--(3.251,7.674)--(3.261,7.676)%
  --(3.271,7.677)--(3.281,7.678)--(3.291,7.680)--(3.301,7.681)--(3.311,7.683)--(3.321,7.684)%
  --(3.330,7.685)--(3.340,7.687)--(3.350,7.688)--(3.360,7.689)--(3.370,7.691)--(3.380,7.692)%
  --(3.390,7.693)--(3.400,7.694)--(3.410,7.696)--(3.420,7.697)--(3.429,7.698)--(3.439,7.699)%
  --(3.449,7.700)--(3.459,7.702)--(3.469,7.703)--(3.479,7.704)--(3.489,7.705)--(3.499,7.706)%
  --(3.509,7.707)--(3.518,7.708)--(3.528,7.709)--(3.538,7.711)--(3.548,7.712)--(3.558,7.713)%
  --(3.568,7.714)--(3.578,7.715)--(3.588,7.716)--(3.598,7.717)--(3.608,7.718)--(3.617,7.719)%
  --(3.627,7.720)--(3.637,7.721)--(3.647,7.722)--(3.657,7.723)--(3.667,7.724)--(3.677,7.725)%
  --(3.687,7.726)--(3.697,7.727)--(3.707,7.728)--(3.716,7.729)--(3.726,7.729)--(3.736,7.730)%
  --(3.746,7.731)--(3.756,7.732)--(3.766,7.733)--(3.776,7.734)--(3.786,7.735)--(3.796,7.736)%
  --(3.806,7.737)--(3.815,7.737)--(3.825,7.738)--(3.835,7.739)--(3.845,7.740)--(3.855,7.741)%
  --(3.865,7.742)--(3.875,7.742)--(3.885,7.743)--(3.895,7.744)--(3.905,7.745)--(3.914,7.746)%
  --(3.924,7.746)--(3.934,7.747)--(3.944,7.748)--(3.954,7.749)--(3.964,7.749)--(3.974,7.750)%
  --(3.984,7.751)--(3.994,7.752)--(4.004,7.752)--(4.013,7.753)--(4.023,7.754)--(4.033,7.755)%
  --(4.043,7.755)--(4.053,7.756)--(4.063,7.757)--(4.073,7.758)--(4.083,7.758)--(4.093,7.759)%
  --(4.102,7.760)--(4.112,7.760)--(4.122,7.761)--(4.132,7.762)--(4.142,7.762)--(4.152,7.763)%
  --(4.162,7.764)--(4.172,7.764)--(4.182,7.765)--(4.192,7.766)--(4.201,7.766)--(4.211,7.767)%
  --(4.221,7.768)--(4.231,7.768)--(4.241,7.769)--(4.251,7.769)--(4.261,7.770)--(4.271,7.771)%
  --(4.281,7.771)--(4.291,7.772)--(4.300,7.773)--(4.310,7.773)--(4.320,7.774)--(4.330,7.774)%
  --(4.340,7.775)--(4.350,7.776)--(4.360,7.776)--(4.370,7.777)--(4.380,7.777)--(4.390,7.778)%
  --(4.399,7.778)--(4.409,7.779)--(4.419,7.780)--(4.429,7.780)--(4.439,7.781)--(4.449,7.781)%
  --(4.459,7.782)--(4.469,7.782)--(4.479,7.783)--(4.489,7.783)--(4.498,7.784)--(4.508,7.784)%
  --(4.518,7.785)--(4.528,7.786)--(4.538,7.786)--(4.548,7.787)--(4.558,7.787)--(4.568,7.788)%
  --(4.578,7.788)--(4.587,7.789)--(4.597,7.789)--(4.607,7.790)--(4.617,7.790)--(4.627,7.791)%
  --(4.637,7.791)--(4.647,7.792)--(4.657,7.792)--(4.667,7.793)--(4.677,7.793)--(4.686,7.793)%
  --(4.696,7.794)--(4.706,7.794)--(4.716,7.795)--(4.726,7.795)--(4.736,7.796)--(4.746,7.796)%
  --(4.756,7.797)--(4.766,7.797)--(4.776,7.798)--(4.785,7.798)--(4.795,7.798)--(4.805,7.799)%
  --(4.815,7.799)--(4.825,7.800)--(4.835,7.800)--(4.845,7.801)--(4.855,7.801)--(4.865,7.802)%
  --(4.875,7.802)--(4.884,7.802)--(4.894,7.803)--(4.904,7.803)--(4.914,7.804)--(4.924,7.804)%
  --(4.934,7.804)--(4.944,7.805)--(4.954,7.805)--(4.964,7.806)--(4.974,7.806)--(4.983,7.806)%
  --(4.993,7.807)--(5.003,7.807)--(5.013,7.808)--(5.023,7.808)--(5.033,7.808)--(5.043,7.809)%
  --(5.053,7.809)--(5.063,7.809)--(5.073,7.810)--(5.082,7.810)--(5.092,7.811)--(5.102,7.811)%
  --(5.112,7.811)--(5.122,7.812)--(5.132,7.812)--(5.142,7.812)--(5.152,7.813)--(5.162,7.813)%
  --(5.171,7.813)--(5.181,7.814)--(5.191,7.814)--(5.201,7.814)--(5.211,7.815)--(5.221,7.815)%
  --(5.231,7.816)--(5.241,7.816)--(5.251,7.816)--(5.261,7.817)--(5.270,7.817)--(5.280,7.817)%
  --(5.290,7.818)--(5.300,7.818)--(5.310,7.818)--(5.320,7.819)--(5.330,7.819)--(5.340,7.819)%
  --(5.350,7.819)--(5.360,7.820)--(5.369,7.820)--(5.379,7.820)--(5.389,7.821)--(5.399,7.821)%
  --(5.409,7.821)--(5.419,7.822)--(5.429,7.822)--(5.439,7.822)--(5.449,7.823)--(5.459,7.823)%
  --(5.468,7.823)--(5.478,7.823)--(5.488,7.824)--(5.498,7.824)--(5.508,7.824)--(5.518,7.825)%
  --(5.528,7.825)--(5.538,7.825)--(5.548,7.826)--(5.558,7.826)--(5.567,7.826)--(5.577,7.826)%
  --(5.587,7.827)--(5.597,7.827)--(5.607,7.827)--(5.617,7.828)--(5.627,7.828)--(5.637,7.828)%
  --(5.647,7.828)--(5.657,7.829)--(5.666,7.829)--(5.676,7.829)--(5.686,7.830)--(5.696,7.830)%
  --(5.706,7.830)--(5.716,7.830)--(5.726,7.831)--(5.736,7.831)--(5.746,7.831)--(5.755,7.832)%
  --(5.765,7.832)--(5.775,7.832)--(5.785,7.832)--(5.795,7.833)--(5.805,7.833)--(5.815,7.833)%
  --(5.825,7.833)--(5.835,7.834)--(5.845,7.834)--(5.854,7.834)--(5.864,7.835)--(5.874,7.835)%
  --(5.884,7.835)--(5.894,7.835)--(5.904,7.836)--(5.914,7.836)--(5.924,7.836)--(5.934,7.836)%
  --(5.944,7.837)--(5.953,7.837)--(5.963,7.837)--(5.973,7.837)--(5.983,7.838)--(5.993,7.838)%
  --(6.003,7.838)--(6.013,7.838)--(6.023,7.839)--(6.033,7.839)--(6.043,7.839)--(6.052,7.839)%
  --(6.062,7.840)--(6.072,7.840)--(6.082,7.840)--(6.092,7.841)--(6.102,7.841)--(6.112,7.841)%
  --(6.122,7.841)--(6.132,7.842)--(6.142,7.842)--(6.151,7.842)--(6.161,7.842)--(6.171,7.843)%
  --(6.181,7.843)--(6.191,7.843)--(6.201,7.843)--(6.211,7.844)--(6.221,7.844)--(6.231,7.844)%
  --(6.241,7.844)--(6.250,7.845)--(6.260,7.845)--(6.270,7.845)--(6.280,7.845)--(6.290,7.846)%
  --(6.300,7.846)--(6.310,7.846)--(6.320,7.846)--(6.330,7.847)--(6.339,7.847)--(6.349,7.847)%
  --(6.359,7.847)--(6.369,7.848)--(6.379,7.848)--(6.389,7.848)--(6.399,7.848)--(6.409,7.849)%
  --(6.419,7.849)--(6.429,7.849)--(6.438,7.849)--(6.448,7.850)--(6.458,7.850)--(6.468,7.850)%
  --(6.478,7.850)--(6.488,7.851)--(6.498,7.851)--(6.508,7.851)--(6.518,7.851)--(6.528,7.852)%
  --(6.537,7.852)--(6.547,7.852)--(6.557,7.852)--(6.567,7.853)--(6.577,7.853)--(6.587,7.853)%
  --(6.597,7.853)--(6.607,7.854)--(6.617,7.854)--(6.627,7.854)--(6.636,7.854)--(6.646,7.854)%
  --(6.656,7.855)--(6.666,7.855)--(6.676,7.855)--(6.686,7.855)--(6.696,7.856)--(6.706,7.856)%
  --(6.716,7.856)--(6.726,7.856)--(6.735,7.857)--(6.745,7.857)--(6.755,7.857)--(6.765,7.857)%
  --(6.775,7.858)--(6.785,7.858)--(6.795,7.858)--(6.805,7.858)--(6.815,7.859)--(6.825,7.859)%
  --(6.834,7.859)--(6.844,7.859)--(6.854,7.860)--(6.864,7.860)--(6.874,7.860)--(6.884,7.860)%
  --(6.894,7.861)--(6.904,7.861)--(6.914,7.861)--(6.923,7.861)--(6.933,7.862)--(6.943,7.862)%
  --(6.953,7.862)--(6.963,7.862)--(6.973,7.863)--(6.983,7.863)--(6.993,7.863)--(7.003,7.863)%
  --(7.013,7.863)--(7.022,7.864)--(7.032,7.864)--(7.042,7.864)--(7.052,7.864)--(7.062,7.865)%
  --(7.072,7.865)--(7.082,7.865)--(7.092,7.865)--(7.102,7.866)--(7.112,7.866)--(7.121,7.866)%
  --(7.131,7.866)--(7.141,7.867)--(7.151,7.867)--(7.161,7.867)--(7.171,7.867)--(7.181,7.868)%
  --(7.191,7.868)--(7.201,7.868)--(7.211,7.868)--(7.220,7.869)--(7.230,7.869)--(7.240,7.869)%
  --(7.250,7.869)--(7.260,7.870)--(7.270,7.870)--(7.280,7.870)--(7.290,7.870)--(7.300,7.870)%
  --(7.310,7.871)--(7.319,7.871)--(7.329,7.871)--(7.339,7.871)--(7.349,7.872)--(7.359,7.872)%
  --(7.369,7.872)--(7.379,7.872)--(7.389,7.873)--(7.399,7.873)--(7.409,7.873)--(7.418,7.873)%
  --(7.428,7.874)--(7.438,7.874)--(7.448,7.874)--(7.458,7.874)--(7.468,7.875)--(7.478,7.875)%
  --(7.488,7.875)--(7.498,7.875)--(7.507,7.876)--(7.517,7.876)--(7.527,7.876)--(7.537,7.876)%
  --(7.547,7.876)--(7.557,7.877)--(7.567,7.877)--(7.577,7.877)--(7.587,7.877)--(7.597,7.878)%
  --(7.606,7.878)--(7.616,7.878)--(7.626,7.878)--(7.636,7.879)--(7.646,7.879)--(7.656,7.879)%
  --(7.666,7.879)--(7.676,7.880)--(7.686,7.880)--(7.696,7.880)--(7.705,7.880)--(7.715,7.881)%
  --(7.725,7.881)--(7.735,7.881)--(7.745,7.881)--(7.755,7.881)--(7.765,7.882)--(7.775,7.882)%
  --(7.785,7.882)--(7.795,7.882)--(7.804,7.883)--(7.814,7.883)--(7.824,7.883)--(7.834,7.883)%
  --(7.844,7.884)--(7.854,7.884)--(7.864,7.884)--(7.874,7.884)--(7.884,7.885)--(7.894,7.885)%
  --(7.903,7.885)--(7.913,7.885)--(7.923,7.886)--(7.933,7.886)--(7.943,7.886)--(7.953,7.886)%
  --(7.963,7.887)--(7.973,7.887)--(7.983,7.887)--(7.992,7.887)--(8.002,7.887)--(8.012,7.888)%
  --(8.022,7.888)--(8.032,7.888)--(8.042,7.888)--(8.052,7.889)--(8.062,7.889)--(8.072,7.889)%
  --(8.082,7.889)--(8.091,7.890)--(8.101,7.890)--(8.111,7.890)--(8.121,7.890)--(8.131,7.891)%
  --(8.141,7.891)--(8.151,7.891)--(8.161,7.891)--(8.171,7.892)--(8.181,7.892)--(8.190,7.892)%
  --(8.200,7.892)--(8.210,7.892)--(8.220,7.893)--(8.230,7.893)--(8.240,7.893)--(8.250,7.893)%
  --(8.260,7.894)--(8.270,7.894)--(8.280,7.894)--(8.289,7.894)--(8.299,7.895)--(8.309,7.895)%
  --(8.319,7.895)--(8.329,7.895)--(8.339,7.896)--(8.349,7.896)--(8.359,7.896)--(8.369,7.896)%
  --(8.379,7.897)--(8.388,7.897)--(8.398,7.897)--(8.408,7.897)--(8.418,7.897)--(8.428,7.898)%
  --(8.438,7.898)--(8.448,7.898)--(8.458,7.898)--(8.468,7.899)--(8.478,7.899)--(8.487,7.899)%
  --(8.497,7.899)--(8.507,7.900)--(8.517,7.900)--(8.527,7.900)--(8.537,7.900)--(8.547,7.901)%
  --(8.557,7.901)--(8.567,7.901)--(8.576,7.901)--(8.586,7.902)--(8.596,7.902)--(8.606,7.902)%
  --(8.616,7.902)--(8.626,7.902)--(8.636,7.903)--(8.646,7.903)--(8.656,7.903)--(8.666,7.903)%
  --(8.675,7.904)--(8.685,7.904)--(8.695,7.904)--(8.705,7.904)--(8.715,7.905)--(8.725,7.905)%
  --(8.735,7.905)--(8.745,7.905)--(8.755,7.906)--(8.765,7.906)--(8.774,7.906)--(8.784,7.906)%
  --(8.794,7.907)--(8.804,7.907)--(8.814,7.907)--(8.824,7.907)--(8.834,7.907)--(8.844,7.908)%
  --(8.854,7.908)--(8.864,7.908)--(8.873,7.908)--(8.883,7.909)--(8.893,7.909)--(8.903,7.909)%
  --(8.913,7.909)--(8.923,7.910)--(8.933,7.910)--(8.943,7.910)--(8.953,7.910)--(8.963,7.911)%
  --(8.972,7.911)--(8.982,7.911)--(8.992,7.911)--(9.002,7.912)--(9.012,7.912)--(9.022,7.912)%
  --(9.032,7.912)--(9.042,7.912)--(9.052,7.913)--(9.062,7.913)--(9.071,7.913)--(9.081,7.913)%
  --(9.091,7.914)--(9.101,7.914)--(9.111,7.914)--(9.121,7.914)--(9.131,7.915)--(9.141,7.915)%
  --(9.151,7.915)--(9.160,7.915)--(9.170,7.916)--(9.180,7.916)--(9.190,7.916)--(9.200,7.916)%
  --(9.210,7.916)--(9.220,7.917)--(9.230,7.917)--(9.240,7.917)--(9.250,7.917)--(9.259,7.918)%
  --(9.269,7.918)--(9.279,7.918)--(9.289,7.918)--(9.299,7.919)--(9.309,7.919)--(9.319,7.919)%
  --(9.329,7.919)--(9.339,7.920)--(9.349,7.920)--(9.358,7.920)--(9.368,7.920)--(9.378,7.921)%
  --(9.388,7.921)--(9.398,7.921)--(9.408,7.921)--(9.418,7.921)--(9.428,7.922)--(9.438,7.922)%
  --(9.448,7.922)--(9.457,7.922)--(9.467,7.923)--(9.477,7.923)--(9.487,7.923)--(9.497,7.923)%
  --(9.507,7.924)--(9.517,7.924)--(9.527,7.924)--(9.537,7.924)--(9.547,7.925)--(9.556,7.925)%
  --(9.566,7.925)--(9.576,7.925)--(9.586,7.925)--(9.596,7.926)--(9.606,7.926)--(9.616,7.926)%
  --(9.626,7.926)--(9.636,7.927)--(9.646,7.927)--(9.655,7.927)--(9.665,7.927)--(9.675,7.928)%
  --(9.685,7.928)--(9.695,7.928)--(9.705,7.928)--(9.715,7.929)--(9.725,7.929)--(9.735,7.929)%
  --(9.744,7.929)--(9.754,7.929)--(9.764,7.930)--(9.774,7.930)--(9.784,7.930)--(9.794,7.930)%
  --(9.804,7.931)--(9.814,7.931)--(9.824,7.931)--(9.834,7.931)--(9.843,7.932)--(9.853,7.932)%
  --(9.863,7.932)--(9.873,7.932)--(9.883,7.933)--(9.893,7.933)--(9.903,7.933)--(9.913,7.933)%
  --(9.923,7.933)--(9.933,7.934)--(9.942,7.934)--(9.952,7.934)--(9.962,7.934)--(9.972,7.935)%
  --(9.982,7.935)--(9.992,7.935)--(10.002,7.935)--(10.012,7.936)--(10.022,7.936)--(10.032,7.936)%
  --(10.041,7.936)--(10.051,7.937)--(10.061,7.937)--(10.071,7.937)--(10.081,7.937)--(10.091,7.938)%
  --(10.101,7.938)--(10.111,7.938)--(10.121,7.938)--(10.131,7.938)--(10.140,7.939)--(10.150,7.939)%
  --(10.160,7.939)--(10.170,7.939)--(10.180,7.940)--(10.190,7.940)--(10.200,7.940)--(10.210,7.940)%
  --(10.220,7.941)--(10.230,7.941)--(10.239,7.941)--(10.249,7.941)--(10.259,7.942)--(10.269,7.942)%
  --(10.279,7.942)--(10.289,7.942)--(10.299,7.942)--(10.309,7.943)--(10.319,7.943)--(10.328,7.943)%
  --(10.338,7.943)--(10.348,7.944)--(10.358,7.944)--(10.368,7.944)--(10.378,7.944)--(10.388,7.945)%
  --(10.398,7.945)--(10.408,7.945)--(10.418,7.945)--(10.427,7.945)--(10.437,7.946)--(10.447,7.946)%
  --(10.457,7.946)--(10.467,7.946)--(10.477,7.947)--(10.487,7.947)--(10.497,7.947)--(10.507,7.947)%
  --(10.517,7.948)--(10.526,7.948)--(10.536,7.948)--(10.546,7.948)--(10.556,7.949)--(10.566,7.949)%
  --(10.576,7.949)--(10.586,7.949)--(10.596,7.949)--(10.606,7.950)--(10.616,7.950)--(10.625,7.950)%
  --(10.635,7.950)--(10.645,7.951)--(10.655,7.951)--(10.665,7.951)--(10.675,7.951)--(10.685,7.952)%
  --(10.695,7.952)--(10.705,7.952)--(10.715,7.952)--(10.724,7.953)--(10.734,7.953)--(10.744,7.953)%
  --(10.754,7.953)--(10.764,7.953)--(10.774,7.954)--(10.784,7.954)--(10.794,7.954)--(10.804,7.954)%
  --(10.814,7.955)--(10.823,7.955)--(10.833,7.955)--(10.843,7.955)--(10.853,7.956)--(10.863,7.956)%
  --(10.873,7.956)--(10.883,7.956)--(10.893,7.957)--(10.903,7.957)--(10.912,7.957)--(10.922,7.957)%
  --(10.932,7.957)--(10.942,7.958)--(10.952,7.958)--(10.962,7.958)--(10.972,7.958)--(10.982,7.959)%
  --(10.992,7.959)--(11.002,7.959)--(11.011,7.959)--(11.021,7.960)--(11.031,7.960)--(11.041,7.960)%
  --(11.051,7.960)--(11.061,7.960)--(11.071,7.961)--(11.081,7.961)--(11.091,7.961)--(11.101,7.961)%
  --(11.110,7.962)--(11.120,7.962)--(11.130,7.962)--(11.140,7.962)--(11.150,7.963)--(11.160,7.963)%
  --(11.170,7.963)--(11.180,7.963)--(11.190,7.964)--(11.200,7.964)--(11.209,7.964)--(11.219,7.964)%
  --(11.229,7.964)--(11.239,7.965)--(11.249,7.965)--(11.259,7.965)--(11.269,7.965)--(11.279,7.966)%
  --(11.289,7.966)--(11.299,7.966)--(11.308,7.966)--(11.318,7.967)--(11.328,7.967)--(11.338,7.967)%
  --(11.348,7.967)--(11.358,7.967)--(11.368,7.968)--(11.378,7.968)--(11.388,7.968)--(11.397,7.968)%
  --(11.407,7.969)--(11.417,7.969)--(11.427,7.969)--(11.437,7.969)--(11.447,7.970)--(11.457,7.970)%
  --(11.467,7.970)--(11.477,7.970)--(11.487,7.971)--(11.496,7.971)--(11.506,7.971)--(11.516,7.971)%
  --(11.526,7.971)--(11.536,7.972)--(11.546,7.972)--(11.556,7.972)--(11.566,7.972)--(11.576,7.973)%
  --(11.586,7.973)--(11.595,7.973)--(11.605,7.973)--(11.615,7.974)--(11.625,7.974)--(11.635,7.974)%
  --(11.645,7.974)--(11.655,7.974)--(11.665,7.975)--(11.675,7.975)--(11.685,7.975)--(11.694,7.975)%
  --(11.704,7.976)--(11.714,7.976)--(11.724,7.976)--(11.734,7.976)--(11.744,7.977)--(11.754,7.977)%
  --(11.764,7.977)--(11.774,7.977)--(11.784,7.978)--(11.793,7.978)--(11.803,7.978);
\gpcolor{color=gp lt color border}
\draw[gp path] (1.872,8.631)--(1.872,0.985)--(13.447,0.985)--(13.447,8.631)--cycle;
%% coordinates of the plot area
\gpdefrectangularnode{gp plot 1}{\pgfpoint{1.872cm}{0.985cm}}{\pgfpoint{13.447cm}{8.631cm}}
\end{tikzpicture}
%% gnuplot variables

	\caption{Gráfico da razão entre a densidade de energia $\varepsilon$ e a densidade $\rho$. Note que o mínimo da energia ocorre em aproximadamente \np[fm^{-3}]{0.15}, que é a densidade de saturação da matéria nuclear. Para obtermos o valor de \np[MeV]{16} por nucleon, basta subtrairmos o valor da massa do nucleon. \protect[Parameters: eNJL1m; Proton fraction: 1/2] }
	\label{Fig:energy_by_nucleon_graph_eNJL1m}
\end{figure*}