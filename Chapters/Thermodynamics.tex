%%%%%%%%%%%%%%%%%%%%%%%
\chapter{Termodinâmica}
\label{Chap:Termo}
%%%%%%%%%%%%%%%%%%%%%%%

\begin{fullwidth}\it
Neste capítulo deduzimos as equações de estado a partir das densidades lagrangianas apresentadas no capítulo anterior.
\end{fullwidth}

%%%%%%%%%%%%%%%%%%%%
\section{Introdução}
%%%%%%%%%%%%%%%%%%%%

Como estamos interessados na obtenção da linha de coexistência de fases no diagrama de fases da QCD segundo o modelo NJL+eNJL e vamos utilizar as condições de Gibbs --~isto é, assumimos que na coexistência de fases temos igualdade de $\varepsilon$, $p$, $\mu$ em ambas as fases~--, precisamos determinar as expressões para essas grandezas.

A determinação pode ser feita de três maneiras:\footnote{Pelo que entendi, pelo menos.}
\begin{itemize}
	\item Através do tensor momento energia:\footnote{Nesses dois primeiros casos como se determinam $\mu$ e $M$? Como se adiciona a temperatura?}
		\begin{align}
			\varepsilon &= \mean{T_{00}} \\
			p &= \frac{1}{3} \mean{T_{ii}}
		\end{align}
	\item Através da Lagrangiana\cite{Glendenning}, que é um método similar ao anterior:
		\begin{align}
			\varepsilon &= -\mean{\mathcal{L}} + \mean{\bar{\psi}\gamma_0 k_0 \psi} \\
			p &= \frac{1}{3} \mean{\bar{\psi}\gamma_i k_i \psi}
		\end{align}
	\item Através da Termodinâmica:
		\begin{align}
			\omega(T, \mu) &= -\frac{T}{V}\ln \mathcal{Z} \\
			\mathcal{Z} &= \Tr\exp\left(-\frac{1}{T}\int d^3x (\mathcal{H} - \mu\mathcal{N})\right) \\
			p &= -\omega \\
			\varepsilon &= -p +Ts + \mu, 
		\end{align}
		onde $\tilde{\mu}$ e $M$ são determinados exigindo que $\partial \omega / \partial M = \partial \omega / \partial \tilde{\mu} = 0$.
\end{itemize}

A questão é que eu não sei fazer de nenhum dos três jeitos.

%%%%%%%%%%%%%%%%%%%%%%%%%%%%%%%%%%%%%%%%
\section{Obtenção da equações de estado}
%%%%%%%%%%%%%%%%%%%%%%%%%%%%%%%%%%%%%%%%

Nas próximas três seções, discutimos\footnote{Discutir da maneira mais genérica possível, já que podemos usar um método ou outro para cada fase e também existes os casos SU(2) e SU(3). Se for necessário particularizar para um caso ou outro, devemos fazer ao discutir a fase/grupo específica na seção apropriada.} meios equivalentes de se obter as equações de estado. 

%%%%%%%%%%%%%%%%%%%%%%%%%%%%%%%%%%%
\subsection{Tensor momento-energia}
%%%%%%%%%%%%%%%%%%%%%%%%%%%%%%%%%%%

Pelo que entendi, nesse método precisamos:
\begin{itemize}
	\item Calcular o tensor energia-momento a partir da lagrangiana;
	\item Calcular os valores esperados $\mean{T_{00}}$ e $\mean{T{ii}}$. Para isso teremos que calcular uma expressão do tipo $\mean{\psi^\dagger \vec{\alpha}\cdot\vec{p} + \beta m^*\psi}$;
	\item Essa expressão deve se separar em duas partes para a energia: $\mean{\psi^\dagger\vec{\alpha}\cdot\vec{p}\psi}$ e $m^*\mean{\psi^\dagger\psi}$. A primeira dá a parte cinética, a segunda é $m^* \rho_s$ ($\beta \equiv \gamma_0$). Como $\mean{\bar{q}q} = -(M - m)/(2G)$ (Buballa report), temos $m = m_0 - 2G_S\rho_s$ e substituindo no lugar do $m$, obtemos $m_0\rho_s - 2G_S\rho_s^2$. Então acho que obtemos no final das contas $\varepsilon = \varepsilon_{kin} + m_0\rho_s - 2G_S\rho_s^2$.
\end{itemize}

Devido a invariância translacional da lagrangiana, é possível mostrar\footnote{Referência. Melhor, fazer num apêndice. Pensando melhor, não estou vendo nada de interessante nisso. Por que anotei isso? Preciso mostrar que $\varepsilon$ e $p$ podem ser calculadas através de $\mean{T_{00}}$ e $\mean{T_{ii}}$.} (através do teorema de Noether) que existe um tensor tal que:
\begin{align}
	T^{\mu\nu} &= \frac{\partial \mathcal{L}}{\partial(\partial_\mu \phi_i)}\partial^\nu\phi_i - \eta^{\mu\nu}\mathcal{L} \\
	\partial_\mu T^{\mu\nu} &= 0,
\end{align}
%
onde a segunda equação é uma equação de continuidade. Integrando em uma superfície de forma que não haja corrente a atravessando, temos quatro cargas conservadas.

Como o momento canônico é dado por 
\begin{equation}
	\pi(x) = \frac{\partial \mathcal{L}}{\partial(\partial_0 \phi)}
\end{equation}
%
e
\begin{align}
	\mathcal{H} &= \int_V d\vec{r} T^{00} \\
	P^k &= \int_V d\vec{r} T^{0k}; \quad k= 1,2,3
\end{align}

Portanto, dada a invariância por translação, temos que a energia e o momento linear se conservam. Podemos definir o invariante
\begin{align}
	P_\mu P^\mu &= E^2 - \vec{P}^2 = M^2, \\
	P^\mu &= (E, \vec{P}).
\end{align}
 
Valor esperado de um operador no estado fundamental de um sistema de muitos nucleons\footnote{Também fazer uma explicação melhorzinha e colocar em um apêndice.}. Dado um operador $\Gamma$, temos que o valor esperado no estado fundamental de muitas partículas pode ser calculado através de
\begin{equation}
	\mean{\bar\psi\Gamma\psi} = \sum_\kappa \int \frac{d\vec{k}}{(2\pi)^3} (\bar\psi\Gamma\psi)_{\vec{k}\kappa} \theta[\mu-e(k)],
\end{equation}
%
Temos nesta expressão uma soma dos valores esperados do operador nos estados de uma partícula --~representado por $(\bar\psi\Gamma\psi)_{\vec{k}\kappa}$ acima~--em todos os estados ocupados: são varridos os índices de spins e isospin (no caso de nucleons), bem como em todos os valores de momento\footnote{Como o índice é contínuo, a soma é uma integral. A integral é em todos os valores de momento (de zero ao infinito), mas a função $\theta$ é zero se a energia do estado é maior que a energia do estado ocupado de mais alta energia, isto é, é maior que a energia de Fermi.}. 

Tal expressão não é útil, a não ser que saibamos como calcular $(\bar\psi\Gamma\psi)_{\vec{k}\kappa}$. No entanto,
\begin{equation}
	\frac{\partial}{\partial \zeta} (\psi^\dagger H_D\psi)_{\vec{k}\kappa} = (\psi^\dagger\frac{\partial H_D}{\partial\zeta}\psi)_{\vec{k}\kappa}
\end{equation}
%
e
\begin{equation}
	(\psi^\dagger H_D\psi)_{\vec{k}\kappa} = \underbrace{E(\vec{k}) + g_\omega\omega_0}_{K_0(\vec{k})}.
\end{equation}
%
Portanto,
\begin{equation}
	(\psi^\dagger\frac{\partial H_D}{\partial\zeta}\psi)_{\vec{k}\kappa} = \frac{\partial}{\partial\zeta}K(\vec{k}).
\end{equation}
%
Assim, se conhecemos $H_D$, podemos usar suas derivadas para conseguir o operador em que estamos interessados à esquerda, enquanto podemos calcular seu valor com a expressão à direita.\footnote{Fazer os cálculos para $\mean{\bar\psi\gamma_0k_0\psi}$ e $\mean{\bar\psi\vec{\gamma}\cdot\vec{k}\psi}$ utilizando esse método, seguindo o Glendenning.}

No programa que a Débora me passou como referência, o potencial termodinâmico tem uma forma análoga ao caso eNJL:
\begin{subequations}\label{Eq:Pot_termodinamico_2}
\begin{align}
	\varepsilon &= \tilde{\omega} + n_c\mu\rho_B = \varepsilon_{\rm{kin}} + m_0\rho_s - 2G_S\rho_s^2 \\
	\varepsilon_0 &= \tilde{\omega}_0 + n_c\mu\rho_B = \varepsilon_{\rm{kin}}^0 + m_0\rho_0^s - 2G_S(\rho_0^s)^2 \\
	\varepsilon_{\rm{kin}} &= -\frac{n_c}{\pi^2} [F_2(m, \Lambda) - F_2(m, p_F)] \\
	\varepsilon_{\rm{kin}}^0 &= -\frac{n_c}{\pi^2} [F_2(m, \Lambda) - F_2(m, 0)] \\
	\rho_0^s &= \rho_s|_{m = m_{\rm{vac}}}
\end{align}
\end{subequations}
%
onde, como na Equação~\eqref{Eq:Energia_kin},
\begin{equation}
	F_2(m, p) = \frac{1}{8}\Big((2p^3 - 3M_i^2p)\sqrt{p^2 + M_i^2} + 3M_i^4\ln\frac{p + \sqrt{p^2 + M_i^2}}{M_i}\Big).
\end{equation}
%
Isso se deve ao fato de que a densidade de energia pode ser calculada a partir do potencial termodinâmico através de mais que uma relação. Possivelmente essa esteja relacionada à derivada do potencial termodinâmico em relação à temperatura, ou ao inverso da temperatura. Ver Avancini~\cite{Avancini2004} e~\cite{Avancini2006}.

%%%%%%%%%%%%%%%%%%%%%%%%
\subsection{Lagrangiana}
%%%%%%%%%%%%%%%%%%%%%%%%

As equações de estado são dadas através da lagrangiana através de\cite{Glendenning1983}:
\begin{align}
	\varepsilon &= -\mean{\mathcal{L}} + \mean{\bar\psi\gamma_0k_0\psi} \\
	p &= \mean{\mathcal{L}} + \frac{1}{3} \mean{\bar\psi\vec{\gamma}\cdot\vec{k}\psi}.
\end{align}

%%%%%%%%%%%%%%%%%%%%%%%%%%%%%%%%%%%%
\subsection{Potencial termodinâmico}
%%%%%%%%%%%%%%%%%%%%%%%%%%%%%%%%%%%%

Sair da lagrangiana e obter a expressão para o potencial termodinâmico.\cite{Asakawa1989}.
\begin{align}
	\omega(T, \mu) &= -\frac{T}{V} \ln\mathcal{Z} \\
	&= -\frac{T}{V} \ln\Tr\exp\left(-\frac{1}{T} \int d^3 x(\mathcal{H} - \mu q^\dagger q)\right). \label{Eq:Pot_term_from_Hamiltonian}
\end{align}

Isso precisa constar nessa seção:
\begin{align}
	n_p(T, \mu) &= \frac{1}{1+\exp((E_p - \mu)/T)} \\
	\bar{n}_p(T, \mu) &= \frac{1}{1 + \exp((E_p + \mu)/T)}
\end{align}
\begin{align}\label{Eq:Def_m_mu_r}
	m &= m_0 - 2 G_s \phi \\
	\mu_R &= \mu - 2 G_v n.
\end{align}

%%%%%%%%%%%%%%%%%%%%%%%%%%%%%%%%%%%%
\section{Fase de quarks: NJL, SU(2)}
%%%%%%%%%%%%%%%%%%%%%%%%%%%%%%%%%%%%

De Buballa\cite{Buballa}\footnote{As variáveis foram alteradas para refletir a notação utilizada neste texto.}:
\begin{quote}
	[\dots] apart from constant (i.e., field independent) terms, which give trivial contributions to the r.h.s. of Eq.~[\eqref{Eq:Pot_term_from_Hamiltonian}], the problem is equivalent to a system o non-interacting particles with mass $m$ at chemical potential $\mu_R$. Hence, the mean-field thermodynamic potencial takes the form
	\begin{equation}\label{Eq:Pot_Termo_Temp_Finita}
		\omega(T, \mu; m, \mu_r) = \omega_M(T, \mu_R) + \frac{(m - m_0)^2}{4G_s} - \frac{(\mu - \mu_R)^2}{4G_v} + \textrm{const.},
	\end{equation}
	%
	with the free Fermi-gas contribution
	\begin{equation}
		\omega_M(T, \mu_R) = -T \sum_n \int \frac{d^3p}{(2\pi)^3} \Tr\ln\left(\frac{1}{T}S^{-1}(i\omega_n,\vec{p})\right).
	\end{equation}
	%
	[\dots]
	The further evaluation of $\omega_M$ can be found in textbooks. [\dots] One finally gets
	\begin{equation}
	\begin{split}\label{Eq:Por_Termo_Temp_Finita_Fermi_Gas_Contrib}
		\omega_M(T, \mu_R) = -2 n_f n_c \int \frac{d^3p}{(2\pi)^3} \{E_p &+ T\ln(1+e^{-(E_p-\mu_R)/T} \\
		&+ T\ln(1 + e^{-(E_p+\mu_R})/T)\}.
	\end{split}
	\end{equation}
\end{quote}
%
onde $E_p = \sqrt{p^2 + m^2}$.

A constante que aparece na definição do potencial não tem influência em nenhum resultado, por isso podemos a escolher\footnote{Mais detalhes na Seção~\ref{Sec:M_vac}.} de forma que o potencial termodinâmico seja nulo para $T = \mu = 0$, para o valor de $m = m_{\rm{vac}}$ que minimiza $\omega$. Por conveniência\footnote{Essa definição simplesmente calcula o valor do potencial termodinâmico mínimo no vácuo e subtrai tal valor do potencial como função de $T$ e $\mu$.}, definimos um potencial termodinâmico ``regularizado'' $\tilde\omega$ a partir do qual calcularemos as quantidades termodinâmicas em que estamos interessados, sendo que ele é dado por
\begin{equation}
	\tilde\omega(T, \mu; m, \mu_R) = \omega(T, \mu; m, \mu_R) - \omega(0, 0; m_{\rm{vac}}, 0).
\end{equation}

%%%%%%%%%%%%%%%%%%%%%%%%%%%%%%%%%%%%%%%%%%%%%%%%%%%%%%%%%%%%%%%%%%%%%%%%%%%%%%%%%%%%%%%%%%%%%
\subsection{Equações auto-consistentes para a massa e para o potencial químico renormalizado}
%%%%%%%%%%%%%%%%%%%%%%%%%%%%%%%%%%%%%%%%%%%%%%%%%%%%%%%%%%%%%%%%%%%%%%%%%%%%%%%%%%%%%%%%%%%%%

De Buballa\cite{Buballa}:
\begin{quote}
	Until this point, the result for $\omega$ depends on $m$ and $\mu_R$, i.e., on our choice of\footnote{Isso ainda não sei o que é, nem está definido no texto ($n$ parece o número de quarks, $\phi$ deve estar ligado à densidade escalar). Deve aparecer ao obter o potencial termodinâmico.} $\phi$ and $n$. On the other hand, in a thermodynamically consistent treatment, $\phi$ and $n$ should follow from $\omega$ as
	\begin{align}\label{Eq:Condicoes_phi_n}
		\phi &= \frac{\partial \omega}{\partial m_0} \\
		n &= -\frac{\partial \omega}{\partial \mu}.
	\end{align}
	Writing $m = m(m_0, T, \mu)$ e $\mu_R = \mu_R(m_0, T, \mu)$ and applying the chain rule we get from Eq.~[\eqref{Eq:Pot_Termo_Temp_Finita}]
	\begin{align}
		\frac{\partial \omega}{\partial m_0} &= \phi + \frac{\delta\omega}{\delta m}\frac{\partial m}{\partial m_0} + \frac{\delta \omega}{\delta\mu_R}\frac{\partial\mu_R}{\partial m_0} \\
		\frac{\partial\omega}{\partial \mu} &= -n + \frac{\delta \omega}{\delta m}\frac{\partial m}{\partial \mu} + \frac{\delta\omega}{\delta\mu_R}\frac{\partial\mu_R}{\partial\mu},
	\end{align}
	%
	where we have used Eq.~[\eqref{Eq:Def_m_mu_r}] to replace the \emph{explicit} derivatives by $\phi$ and $-n$, respectively. Thus, to be consistent with Eq.~[\eqref{Eq:Condicoes_phi_n}] the \emph{implicit} contributions have to vanish. This is obviously fulfilled if
	\begin{equation}
		\frac{\delta\omega}{\delta m} = \frac{\delta \omega}{\delta\mu_R} = 0,
	\end{equation}
	i.e., the stationary points of $\omega$ with respect to $m$ and $\mu_R$ are automatically thermodynamic consistent. Explicitly, one gets:
	\begin{align}
		\frac{\delta\omega}{\delta m} &= \frac{m-m_0}{2G_s} - 2 n_f n_c \int \frac{d^3p}{(2\pi)^3} \frac{m}{E_p} (1 - n_p(T, \mu_R) - \bar{n}_p(T, \mu_R)) = 0 \\
		\frac{\delta\omega}{\delta \mu_R} &= \frac{\mu - \mu_R}{2G_v} - 2 n_f n_c \int \frac{d^3p}{(2\pi)^3} (n_p(T, \mu_R) - \bar{n}_p(T, \mu_R)) = 0.
	\end{align}
	This is a coupled set of self-consistent equations for $m$ and $\mu_R$, [for the case] $G_v \neq 0$. [\dots] If there is more than one solution, the stable one is the solution which corresponds to the lowest value of $\omega$.
\end{quote}

Podemos reescrever as equações acima como
\begin{align}
	m &= m_0 + 4 G_s n_c n_f \int\frac{d^3p}{(2\pi)^3} \frac{m}{E_p} (1 - n_p(T, \mu_R) - \bar{n}_p(T, \mu_R)) \label{gap} \\
	\mu_R &= \mu - 4 G_v n_c n_f \int\frac{d^3p}{(2\pi)^3} (n_p(T, \mu_R) - \bar{n}_p(T, \mu_R)) \label{gap_mu_r},
\end{align}
%
que juntamente com a equação para a densidade bariônica
\begin{equation}\label{gap_rho}
	n_c \rho_B = n(T, \mu_R) = 2 n_f n_c\int\frac{d^3p}{(2\pi)^3}(n_p(T, \mu_R) - \bar{n}_p(T, \mu_R)) 
\end{equation}
%
formam dois conjuntos distintos de equações acopladas que devem ser resolvidas simultaneamente para encontrar $m$ e $\mu_R$: se utilizarmos $\rho_B$ e $T$ como parâmetros livres, devemos utilizar a primeira e a última; se utilizarmos $\mu$ e $T$, devemos utilizar as duas primeiras. 

No caso de tomarmos a segunda alternativa ($\mu$ e $T$ como variáveis), utilizamos a equação acima para obtermos $\rho_B$. No entanto, se utilizarmos a densidade bariônica como uma das variáveis, devemos calcular $\mu$. Isso é simples se notarmos que a integral que aparece na expressão para $\mu_R$ acima é a mesma que aparece na expressão para a densidade bariônica. Assim podemos escrever
\begin{equation}
	\mu = \mu_R + 4 n_f n_c G_v \int \frac{d^3p}{(2\pi)^3} (n_p(T, \mu_R) - \bar{n}_p(T, \mu_R))
\end{equation}
%
e
\begin{equation}
	\int\frac{d^3p}{(2\pi)^3} (n_p(T, \mu_R) - \bar{n}_p(T, \mu_R)) = \frac{\rho_B}{2 n_f}.
\end{equation}
%
Logo, substituindo a expressão acima na anterior,
\begin{equation}\label{Eq:Pot_quim_a_partir_de_mu_r_rho_bar}
	\mu = \mu_R + 2 n_c G_v \rho_B.
\end{equation}

%%%%%%%%%%%%%%%%%%%%%%%%%%%%%%%
\subsection{Equações de estado}
%%%%%%%%%%%%%%%%%%%%%%%%%%%%%%%

De Buballa~\cite{Buballa}:
\begin{quote}
	Having found a pair of solutions $m$ and $\mu_R$, other thermodynamic quantities can be obtained in the standard way. Since the system is uniform, pressure and energy density are given by
	\begin{align}
		p(T, \mu) &= -\omega(T, \mu; m, \mu_R) \label{Exp_pressao_T}\\
		\varepsilon(T, \mu) &= -p(T, \mu) + T s(T, \mu) + \mu n(T,\mu). \label{Exp_energia_T}
	\end{align}
\end{quote}
%
Portanto, a obtenção das equações de estado depende diretamente do potencial termodinâmico e indiretamente da massa e do potencial químico renormalizado. Verificamos que há uma dependência implicita em $\rho_B$ através de $\mu$, quando utilizamos a primeira como parâmetro ao invés da segunda. Verificamos também que a densidade de energia depende da densidade de entropia $s(T, \mu)$ que será discutida na próxima seção.

Para a obtenção de resultados numéricos devemos escolher a temperatura $T$ e mais uma variável ($\rho_B$ ou $\mu$) como sendo livres, determinando todas as demais quantidades. Ao varrermos as variáveis livres, obtemos um conjunto se soluções ``ponto a ponto'': se quisermos fazer um gráfico tipo $\omega \times m$ ou $\mu \times m$, basta escolhermos os pares de variáveis (por exemplo, $(m_1, \omega_1)$, $(m_2, \omega_2)$, \dots) \emph{de uma mesma solução} (isto é, para os mesmos valores das variáveis livres); assim, se estamos variando $\rho_B$ com um valor de $T$ qualquer, temos para $\omega \times m$ os pontos $(m_1, \omega_1)$ calculado à densidade $\rho_B^{(1)}$, $(m_2, \omega_2)$ calculado à densidade $\rho_B^{(2)}$, \dots.

%%%%%%%%%%%%%%%%%%%%
\section{Entropia}
\label{Sec:Entropia}
%%%%%%%%%%%%%%%%%%%%

Para calcularmos a densidade de energia $\varepsilon$, precisamos calcular a densidade de entropia $s$. Temos que a entropia é dada nesse caso\footnote{Não sei como chegar nessa expressão (peguei de V. Dexheimer, J.R. Torres, D.P. Menezes Eur. Phys. J. C (2013) 73:2569). No Buballa (Rep.) diz que é só calcular $s = -\partial\omega/\partial T$, mas a expressão que obtenho é diferente (veja abaixo). Ou meu cálculo está errado, ou as expressões são equivalentes (não consigo verificar isso).} pela expressão para um gás de Fermi livre:
\begin{equation}\label{Eq:Densidade_de_entropia}
\begin{split}
	s = -2 n_c \sum_i \int [&n_p^{(i)} \ln n_p^{(i)} + (1 - n_p^{(i)}) \ln(1-n_p^{(i)}) \\
	&+ \bar{n}_p^{(i)} \ln \bar{n}_p^{(i)} + (1-\bar{n}_p^{(i)})\ln(1-\bar{n}_p^{(i)})] \\
	&\times\theta(\Lambda^2 - p^2)\;\frac{d^3p}{(2\pi)^3}.
\end{split}
\end{equation}
%
Acredito que no nosso caso basta substituir a soma em $i = u, d$ pela constante $n_f \equiv 2$:
\begin{equation}
\begin{split}
	s = - \frac{n_c n_f}{\pi^2} \int [&n_p^{(i)} \ln n_p^{(i)} + (1 - n_p^{(i)}) \ln(1-n_p^{(i)}) \\
	&+ \bar{n}_p^{(i)} \ln \bar{n}_p^{(i)} + (1-\bar{n}_p^{(i)})\ln(1-\bar{n}_p^{(i)})] \\
	&\times\theta(\Lambda^2 - p^2) \; p^2 dp.
\end{split}
\end{equation}

%%%%%%%%%%%%%%%%%%%%%%%%%%%%%%%%%%%%%%
\textbf{Cálculo segundo a derivada}
%%%%%%%%%%%%%%%%%%%%%%%%%%%%%%%%%%%%%%

A entropia pode ser calculada através de
\begin{equation}
	s = -\frac{\partial \omega}{\partial T}.
\end{equation}
%
O único termo que contribui para a derivada é o dado pela Equação~\eqref{Eq:Por_Termo_Temp_Finita_Fermi_Gas_Contrib} para $\omega_M$. Utilizando\footnote{Assumindo que ela se aplique a derivadas parciais também.} a fórmula de Leibniz (Equação~\eqref{Eq:Form_Leibniz}), obtemos
\begin{equation}
	s = 2 n_f n_c \int\frac{d^3p}{(2\pi)^2} \frac{\partial}{\partial T} \left[E_p + T \ln(1 + e^{-\frac{E_p - \mu_p}{T}}) + T \ln(1 + e^{-\frac{E_p + \mu_p}{T}})\right].
\end{equation}
%
A derivada de $E_p$ é nula, enquanto\footnote{É fácil mostrar que $1 - n_p = 1/(1 + \exp(-(E_p - \mu_p)/T))$ e que $1 - \bar{n}_p = 1/(1 + \exp(-(E_p + \mu_p)/T))$.}
\begin{align}
	\frac{\partial}{\partial T}\left[T \ln(1 + e^{-\nicefrac{\xi}{T}})\right] &= \ln (1 + e^{-\nicefrac{\xi}{T}}) + \frac{\xi e^{-\nicefrac{\xi}{T}}}{T(1+e^{-\nicefrac{\xi}{T}})} \\
	&= - \ln(1-\hat{n}) + \frac{\xi}{T}\hat{n},
\end{align}
%
onde $\tilde{n}$ é dado por
\begin{equation}
	\hat{n} = \begin{cases} n_p(T, \mu_R), \quad \textrm{se}~ \xi = E_p - \mu \\ \bar{n}_p(T, \mu_R), \quad \textrm{se}~ \xi = E_p + \mu. \end{cases}
\end{equation}
%
Temos então
\begin{equation}\label{Eq:Densidade_de_entropia_deriv}
\begin{split}
	s = 2 n_f n_c \int \Big\{&-\ln(1 - n_p(T, \mu_R)) + \frac{E_p - \mu_R}{T} n_p(T, \mu_R) \\
	&- \ln (1 - \bar{n}_p(T, \mu_R)) + \frac{E_p + \mu_R}{T} \bar{n}_p(T, \mu_R)\Big\} \; \frac{d^3p}{(2\pi)^3},
\end{split}
\end{equation}
%
onde a integração deverá ser feita no intervalo $|\vec{p}| = [0;\Lambda]$:
\begin{equation}
\begin{split}
	s = \frac{n_f n_c}{\pi^2} \int_0^\Lambda \Big\{&-\ln(1 - n_p(T, \mu_R)) + \frac{E_p - \mu_R}{T} n_p(T, \mu_R) \\
	&- \ln (1 - \bar{n}_p(T, \mu_R)) + \frac{E_p + \mu_R}{T} \bar{n}_p(T, \mu_R)\Big\} \; p^2 dp,
\end{split}
\end{equation}

%%%%%%%%%%%%%%%%%%%%%%%%%%%%%%%%%%%%%%%%%%%%%%%%%%%%%%
\section{Fase de quarks: NJL, SU(2), temperatura zero}
%%%%%%%%%%%%%%%%%%%%%%%%%%%%%%%%%%%%%%%%%%%%%%%%%%%%%%

Tomando o caso especial\footnote{Por que essa simplificação? É um caso específico problemático de se calcular se não fizermos o limite aritmeticamente (devido à distrib. de Fermi-Dirac)? É um caso muito frequente e útil?} $T = 0$, temos uma simplificação substancial das Equações~\eqref{Eq:Pot_Termo_Temp_Finita}, \eqref{gap}, \eqref{gap_mu_r}, e~\eqref{gap_rho}. Verificaremos como obter as expressões nas próximas seções.

%%%%%%%%%%%%%%%%%%%%%%%%%%%%%%%%%%%%%%%%%%%%%%%%%%%%%%%%%%%%
\subsection{Limite $T \to 0$ da equação de gap para a massa}
%%%%%%%%%%%%%%%%%%%%%%%%%%%%%%%%%%%%%%%%%%%%%%%%%%%%%%%%%%%%

Na Equação~\eqref{gap}, temos a integral
\begin{equation}
	I = \int \frac{m}{E_p} (n_p(T, \mu_R) + \bar{n}_p(T, \mu_R)) d^3p.
\end{equation}
%
No limite $T \to 0$, $\bar{n}_p(T, \mu_R) \to 0$, mas o limite de $n_p$ depende so sinal de $E_p - \mu_R$. Se $E_p - \mu_R > 0$, $n_p \to 0$; Se $E_p - \mu_R \leqslant 0$, $n_p \to 1$. Logo, podemos considerar que no limite $T \to 0$ a distribuição de Fermi-Dirac pode ser substituida por uma função degrau $\theta(\mu_R - E_p)$. Isso significa que todos os estados estão ocupados até uma certa energia\footnote{Denominada \emph{energia de Fermi}.} $E_F$. Equivalentemente, podemos verificar que os estados ocupados estão contidos em uma esfera de raio $p_F$ no espaço de momento, onde $p_F$ é o momento\footnote{Denominado \emph{momento de Fermi}.} do estado com energia $E_F$. 

Podemos determinar uma relação entre $\mu_R$ e $p_F$ considerando que -- para os estados ocupados -- vale a desigualdade
\begin{equation}
	E_p = \sqrt{p^2 + m^2} \leqslant \mu_R.
\end{equation}
%
Para o último estado ocupado, em particular, temos
\begin{equation}
	E_p = \sqrt{p^2 + m^2} = \mu_R
\end{equation}
%
e, como nesse caso $p \equiv p_F$, temos
\begin{equation}
	\mu_R = \sqrt{p_F^2 + m^2}.
\end{equation}
%
Como podemos verificar de $E_p = \sqrt{p^2 + m^2}$, $E_F \equiv \sqrt{p_F^2 + m^2} = \mu_R$. Isto é, a energia de Fermi é o próprio valor do potencial químico renormalizado $\mu_R$.

Podemos, portanto, substituir a função degrau na energia por uma função degrau no momento, isto é, $\theta(p_F - p)$.\footnote{A alteração é basicamente uma mudança na ``ordem de varredura'' do índice da ``soma'': $\theta(\mu_R - E_p)$ é útil para uma soma em ordem crescente de energia, enquanto $\theta(p_F - p)$ é útil para uma soma em ordem crescente de momento; De qualquer forma, em ambos os casos são os mesmos estados.}

No limite $T \to 0$ temos então
\begin{equation}
	I = \int \frac{m}{E_p} \theta(p_F - p) d^3p.
\end{equation}

%%%%%%%%%%%%%%%%%%%%%%%%%%%%%%%%%%%%%%%%%%%%%%%%%%%%%%%%
\subsection{Limite $T \to 0$ do potencial termodinâmico}
%%%%%%%%%%%%%%%%%%%%%%%%%%%%%%%%%%%%%%%%%%%%%%%%%%%%%%%%

No caso do potencial termodinâmico, temos a integral
\begin{equation}
	I = \int [T \ln(1+e^{-\frac{E_p - \mu_R}{T}}) + T\ln(1+e^{-\frac{E_p + \mu_R}{T}})] d^3p.
\end{equation}
%
Assumindo\footnote{Se separarmos em dois casos ($\mu_R > 0$ e $\mu_R < 0$), provavelmente o que vai acontecer é que o primeiro termo será zero no limite, enquanto o segundo se encaixará na análise que segue. Talvez por isso no Buballa apareçam as funções degrau com os termos dos argumentos ao quadrado.} $\mu_R > 0$, temos que o segundo termo será zero no limite $T \to 0$, pois a exponencial tende a zero e $\ln(1) = 0$. Para o primeiro termo, se $E_p - \mu_R >$, temos o mesmo resultado. No entanto, se $E_p - \mu_R < 0$, temos
\begin{equation}
	\lim_{\substack{T \to 0 \\ E_p - \mu_R < 0}} T \ln(1+e^{-\frac{E_p - \mu_R}{T}}),
\end{equation}
%
onde a exponencial tende ao infinito. Tomando a expansão em série\footnote{Segundo o Wolfram alpha, \emph{expansão em série generalizada de Puiseux}.} de $\ln(1 + x)$ em torno de $x = \infty$
\begin{equation}
	\ln(1 + x) = -\ln\frac{1}{x} + \frac{1}{x} - \frac{1}{2x^2} + \frac{1}{3x^3} - \dots,
\end{equation}
%
verificamos que os termos $(n x^n)^{-1}$ vão a zero para $x \to \infty$. Substituindo o primeiro termo, temos
\begin{equation}
	\lim_{\substack{T \to 0 \\ E_p - \mu_R < 0}} T \ln(1+e^{-\frac{E_p - \mu_R}{T}}) = \lim_{T \to 0} \left[-T \frac{E_p - \mu_R}{T} \right] = \mu_R - E_p.
\end{equation}
%

Podemos descrever ambos os casos do primeiro termo utilizando uma função degrau $\theta(\mu_R - E_p)$
\begin{equation}
	\lim_{T \to 0} T \ln(1+e^{-\frac{E_p - \mu_R}{T}}) = (\mu_R - E_p)\theta(\mu_R - E_p).
\end{equation}
%
Como no caso anterior, podemos substituir a função degrau na energia por uma função degrau no momento. Obtemos portanto
\begin{equation}
	\lim_{T \to 0} I = \int (\mu_R - E_p)\theta(p_F - p) d^3p
\end{equation}
%

%%%%%%%%%%%%%%%%%%%%%%%%%%%%%%%%%%%%%%%%%%%%%%%%%%%%%%%%%%%%%%%%%%%%%%%%%%%%%%%%%%%%%%%%%%%%%%
\subsection{Limite $T \to 0$ da densidade e da equação para o potencial químico renormalizado}
%%%%%%%%%%%%%%%%%%%%%%%%%%%%%%%%%%%%%%%%%%%%%%%%%%%%%%%%%%%%%%%%%%%%%%%%%%%%%%%%%%%%%%%%%%%%%%

No caso do potencial químico renormalizado e da densidade, temos a seguinte integral:
\begin{equation}
	I = \int (n_p(T,\mu_R) - \bar{n}_p(T, \mu_R)) d^3p.
\end{equation}
%
Se tomarmos o limite $T \to 0$ de cada uma das expressões, temos
\begin{align}
	\lim_{T \to 0} \frac{1}{1+\exp((E_p - \mu_R)/T)} &= \begin{cases} 0, \quad \textrm{Se}~E_p - \mu_R > 0 \\ 1, \quad \textrm{Se}~E_p - \mu_R < 0 \end{cases} \\
	\lim_{T \to 0} \frac{1}{1+\exp((E_p + \mu_R)/T)} &= \begin{cases} 0, \quad \textrm{Se}~E_p + \mu_R > 0 \\ 1, \quad \textrm{Se}~E_p + \mu_R < 0. \end{cases}
\end{align}
%
Como temos\footnote{Novamente, talvez considerar o caso $\mu_R < 0$ explique o quadrado nos termos dos argumentos das funções degrau no Buballa.} que $E_p > 0$ e $\mu_R > 0$, o segundo termo será sempre nulo. Ambos os casos do primeiro podem ser descritos com o auxílio de uma função degrau $\theta(\mu_R - E_p)$, que pode ser trocada por uma função degrau $\theta(p_F - p)$. Logo,
\begin{equation}
	\lim_{T \to 0} I = \int \theta(p_F - p)d^3p.
\end{equation}
%
Realizando a integração angular, obtemos
\begin{equation}
	\lim_{T \to 0} I = 4\pi \int_0^\infty \theta(p_F - p) p^2 \;dp.
\end{equation}
%
Como o argumento da integral é zero a partir de $p_F$, os limites de integração estão limitados efetivamente:
\begin{align}
	\lim_{T \to 0} I &= 4\pi \int_0^{p_F} p^2 \;dp \\
	&= 4\pi \frac{p_F^3}{3}.
\end{align}

Substituindo esse resultado nas Equações~\eqref{gap_mu_r} e~\eqref{gap_rho}, obtemos as seguintes relações:
\begin{align}
	p_F &= \sqrt{\frac{3\pi^2\rho_B}{n_f}} \\
	\mu_R &= \mu - 2 n_c G_v \rho_B.
\end{align}	

%%%%%%%%%%%%%%%%%%%%%%%%%%%%%%%%%%%%%%%%
\section{Equações para temperatura zero}
%%%%%%%%%%%%%%%%%%%%%%%%%%%%%%%%%%%%%%%%

Para o limite de temperatura zero, temos\cite{Buballa1996}\footnote{Muito dessa seção é só uma particularização para temperatura zero. Como vamos discutir temperatura finita antes, a maioria dos conceitos tem que ser discutida antes, para $T \neq 0$, então vai ter que sair daqui para não ficar repetitivo.}:
\begin{quote}
Expanding $\bar{\psi}\psi$ and $\bar{\psi}\gamma^\mu\psi$ about their thermal expectation values we can derive the mean field thermodynamic potential at temperature $T$ and chemical potential $\mu$ [11]. We restrict ourselves to the Hartree approximation. Furthermore in this paper we only consider $T=0$ and $\mu \geqslant 0$. The result for the thermodynamic potential is
\begin{equation}\label{Eq:Def_pot_termo}
	\omega(0,\mu; m, \mu_R) = \omega_m^{\rm{(vac)}} + \omega_m^{\rm{(med)}}(\mu_R) + \frac{(m - m_0)^2}{4G_S} - \frac{(\mu - \mu_R)^2}{4G_V},
\end{equation}
%
with
\begin{align}
	\omega_m^{\rm{(vac)}} &= - (2n_fn_c) \int \frac{d^3p}{(2\pi)^3} E_p, \label{Eq:Def_omega_vac}\\
	\omega_m^{\rm{(med)}}(\mu_R) &= - (2n_fn_c) \int \frac{d^3p}{(2\pi)^3}(\mu_R - E_p)\theta(p_F - p), \label{Eq:Def_omega_med}
\end{align}
%
being the vacuum part ant the medium part of the thermodynamic potential (per volume) of a free fermion with mass $m$. [\dots] The vacuum part is strongly divergent and has to be regularized. For simplicity we use a sharp cut-off $\Lambda$ in the three momentum space.\footnote{Isso significa que todas as integrais no momento serão limitadas a $\Lambda$.}
\end{quote}
%
Ainda temos que
\begin{align}
	E_p &= \sqrt{m^2+p^2}, \label{Eq:Def_E}\\
	p_F &= \sqrt{\mu_R^2 - m^2}\theta(\mu_R^2 - m^2), \label{Eq:Rel_pot_quim_renorm_mom_fermi}
\end{align}
%
onde temos para $\mu_R$:
\begin{quote}
	In addition to the external\footnote{É nesse parâmetro que fazemos o \emph{loop}? Ou é uma constante?} parameter $\mu$, $\omega_{\rm{MF}}$ depends on two other parameters, the dynamical fermion mass $m$ and the renormalized chemical potential $\mu_R$, which are related to the scalar density $\langle\bar{\psi}\psi\rangle$ and the vector density $\langle\psi^\dagger\psi\rangle$ at the chemical potential $\mu$ by [11]:
	\begin{align}
		m &= m_0 - 2G_S\langle\bar{\psi}\psi\rangle, \label{Eq:Eq_Gap_Buballa_1}\\
		\mu_R &= \mu - 2G_V\langle\psi^\dagger\psi\rangle. \label{Eq:Eq_Gap_Pot_Quim_Renorm}
	\end{align}
	These parameter have to be determined self-consistently by calculating $\langle\bar{\psi}\psi\rangle$ and $\langle\psi^\dagger\psi\rangle$ from $\omega_{\rm{MF}}$. It can be shown that the self-consistent solution correspond to the extrema of $\omega_{\rm{MF}}$ as a function of $m$ and $\mu_R$. This leads to a set of coupled equations for $m$ and $\mu_R$ which reads for $T = 0$ and $\mu \geqslant 0$:
\begin{align}
	m &= m_0 + 2G_S(2n_fn_c)\left(\int \frac{d^3p}{(2\pi)^3} \frac{m}{E_p} - \int \frac{d^3p}{(2\pi)^3}\frac{m}{E_p}\theta(p_F - p)\right), \label{Eq:Eq_Gap_Buballa_2}\\
	\mu_R &= \mu - 2G_V(2n_fn_c)\int\frac{d^3p}{(2\pi)^3}\theta(p_F - p).
\end{align}
\end{quote}

Resolvendo a equação para $\mu_R$ acima, obtemos
\begin{align}
	\mu_R &= \mu - 2G_V(2n_fn_c)\int\frac{d^3p}{(2\pi)^3}\theta(p_F - p) \\
	&= \mu - \frac{2G_V(2n_fn_c)}{2\pi^2}\int_0^\Lambda p^2 \theta(p_F - p) dp \\
	&= \mu - \frac{2G_V(n_fn_c)}{3\pi^2}p_F^3 \label{Eq:Expr_pot_quim_renorm}
\end{align}
%
e é possível eliminar a dependência de $\omega$ na variável $\mu_R$, obtendo algo que depende somente de $m$ (para um dado $\mu$).\footnote{Na prática é mais simples calcular $\mu_R$ e usar o valor no cálculo de $\tilde{\omega}$. Além disso, ao se utilizar $\rho_B$ como variável livre, o próprio valor de $\mu$ passa a estar vinculado a $\rho_B$.}

Definindo então o potencial termodinâmico ``renormalizado'' $\tilde\omega$:
\begin{equation}\label{Eq:Def_omega_tilde}
	\tilde{\omega}(0,\mu;m,\mu_R) = \omega(0,\mu; m, \mu_R(\mu,m)) - \omega(0,0; m_{\rm{vac}}, 0),
\end{equation}
%
onde calculamos o valor de $\omega_{\rm{MF}}$ para $\mu = \mu_R = 0$ (o que implica\footnote{Isso pode ser visto na expressão \eqref{Eq:Expr_pot_quim_renorm} ou na expressão \eqref{Eq:Rel_pot_quim_renorm_mom_fermi}} em $p_F = 0$), obtendo uma constante que será diminuída do potencial, resultando em um potencial $\tilde\omega$ nulo quando $m = m_{\rm{vac}}$, $\mu = 0$. A obtenção de $m_{\rm{vac}}$ será discutida na próxima seção.

É possível se obter outras grandezas através das equações
\begin{align}
	\rho_B &= \frac{1}{n_c} \langle\psi^\dagger\psi\rangle = \frac{n_f}{3\pi^2}p_F^3, \label{Eq:Rel_Dens_Mom_Fermi_NJL}\\
	\varepsilon &= \tilde{\omega} + \mu n_c \rho_B, \label{Eq:Energia_omega_tilde}\\
	p &= - \tilde{\omega}, \label{Eq:Pressao_omega_tilde}
\end{align}
%
onde ``[\dots] for a given $\mu$ these formulas have to be evaluated for the values of $m$ and $\mu_R$ which minimize the thermodynamic potential.''\footnote{Isso é necessário se tivermos mais que uma solução. Nesse caso, devemos utilizar aquela que minimiza o potencial termodinâmico.}

%%%%%%%%%%%%%%%%%%%%%%%%%%%%%%%%%%%%%%%%%%%%%%%%%%%%%%%%%%%%%%
\section{Determinação da constante no potencial termodinâmico}
\label{Sec:M_vac}
%%%%%%%%%%%%%%%%%%%%%%%%%%%%%%%%%%%%%%%%%%%%%%%%%%%%%%%%%%%%%%

Para calcular as grandezas em que estamos interessados ($p$, $\varepsilon$), precisamos calcular $\tilde{\omega}$. Para calcular tal grandeza, precisamos calcular o valor de $m_{\rm{vac}}$, Eq.~\eqref{Eq:Def_omega_tilde}. Esse valor é dado de forma que\cite{Buballa1996}
\begin{quote}
	$m_{\rm{vac}}$ is the dynamical mass which minimizes the thermodynamic potential of the vacuum.
\end{quote}
%
isto é\footnote{ou acredito que seja}, $m_{\rm{vac}}$ deve minimizar o potencial definido pela Eq.~\ref{Eq:Def_pot_termo} para o caso do vácuo\footnote{No Report do Buballa, também há a menção a minimizar o potencial termodinâmico, porém não há o termo $\omega_m^{\rm{med}}$. Acredito que devido à função degrau, esse termo seja zero na forma que temos aqui.}. Podemos determinar o valor que minimiza o potencial tomando a derivada em relação a $m$ e igualando-a a zero:
\begin{align}
	\frac{d}{dm} \omega(T = 0,\mu = 0; m, \mu_R = 0) &= \frac{d}{dm}\omega_m^{\rm{(vac)}} \\
	&\phantom{=} + \frac{d}{dm}\omega_m^{\rm{(med)}}(\mu_R) \nonumber \\
	&\phantom{=} + \frac{d}{dm}\frac{(m - m_0)^2}{4G_S} \nonumber \\
	&= 0.\nonumber
\end{align}
%
Como $\mu_R = 0$ e $E = \sqrt{p^2+m^2}$, da expressão~\eqref{Eq:Def_omega_med} temos que o segundo termo é nulo (por $\mu_R$ ser zero e pelo fato de que a função degrau será zero, pois $p_F$ é zero se $\mu_R = 0$, Eq.~\eqref{Eq:Rel_pot_quim_renorm_mom_fermi}).

O primeiro termo pode ser calculado através da fórmula de Leibniz (Equação~\eqref{Eq:Form_Leibniz}) resultando em
\begin{align}
	\frac{d}{dm}\omega_m^{\rm{(vac)}} &= -2 n_f n_c\frac{d}{dm}\int \frac{d^3p}{(2\pi)^3} \sqrt{p^2 + m^2} \\
	&= - \frac{n_f n_c}{\pi^2} \int \frac{\partial}{\partial m} p^2\sqrt{p^2+m^2} dp \\
	&= - \frac{n_f n_c}{\pi^2} \int_0^\Lambda \frac{mp^2}{\sqrt{p^2+m^2}} dp \\
	&= - \frac{n_f n_c}{\pi^2} \;m [F_0(m, \Lambda) - F_0(m,0)]
\end{align}
%
onde usamos as relações~\eqref{Eq:Int_d3p_to_dp} e \eqref{Eq:Def_F0},~\eqref{Eq:Def_F0_integrado}.

Finalmente, o terceiro termo é dado por
\begin{equation}
	\frac{d}{dm} \frac{(m - m_0)^2}{4G_S} = \frac{2 (m - m_0)}{4G_S} = \frac{m - m_0}{2G_S}.
\end{equation}
%
Consequentemente,
\begin{align}
	\frac{d}{dm} \omega(T = 0, \mu = 0; m, \mu_R = 0) &= - \frac{n_f n_c}{\pi^2} \;m [F_0(m, \Lambda) - F_0(m,0)] \\
	&\phantom{=}~+ \frac{m - m_0}{2G_S} \nonumber \\
	&= 0.
\end{align}
%
multiplicando por $2G_S$, obtemos uma equação equivalente:
\begin{equation}\label{Eq:Calculo_m_vac}
	m - m_0 - 2G_S\frac{n_f n_c}{\pi^2} \;m [F_0(m, \Lambda) - F_0(m,0)] = 0.
\end{equation}
%
Devemos resolver tal equação através de um método para encontrar zeros de funções. O valor obtido será o valor de $m_{\rm{vac}}$, a partir do qual podemos calcular $\omega(0,0;m_{\rm{vac}}, 0)$.

%%%%%%%%%%%%%%%%%%%%%%%%%%%
\subsection{Equação do gap}
%%%%%%%%%%%%%%%%%%%%%%%%%%%

Precisamos resolver a Equação~\eqref{Eq:Eq_Gap_Buballa_1} para determinar a massa efetiva $m$ de maneira auto-consistente. Para $T=0$ e $\mu \geqslant 0$, $\mean{\bar\psi\psi}$ pode ser calculada a partir de $\omega_{\rm{MF}}$, resultando em ~\eqref{Eq:Eq_Gap_Buballa_2}.

Resolvendo as integrais, temos
\begin{align}
	\langle\bar{\psi}\psi\rangle &= - 2 n_f n_c \left(\int\frac{d^3p}{(2\pi)^3}\frac{m}{E_p} - \int\frac{d^3p}{(2\pi)^3} \frac{m}{E_p} \theta(p_F - p)\right) \\
	&= - 2 n_f n_c \left(\int_0^\Lambda\frac{dp}{2\pi^2} \frac{mp^2}{E_p} - \int_0^\Lambda \frac{dp}{2\pi^2} \frac{mp^2}{E_p} \theta(p_F - p)\right)
\end{align}
%
onde utilizamos a Eq.~\ref{Eq:Int_d3p_to_dp} e integramos o momento entre o valor mínimo (zero) e o valor do \emph{cutoff} $\Lambda$. As integrais podem ser realizadas utilizando a expressão~\eqref{Eq:Integ_momento_quad}, resultando em
\begin{align}
	\int_0^\Lambda \frac{dp}{2\pi^2} \frac{mp^2}{E_p} &= \frac{m}{2\pi^2} F_0(m,\Lambda)\\
	\int_0^\Lambda \frac{dp}{2\pi^2} \frac{mp^2}{E_p} \theta(p_F - p) &= \int_0^{p_F} \frac{dp}{2\pi^2} \frac{mp^2}{E_p} = \frac{m}{2\pi^2} F_0(m,p_F),
\end{align}
%
onde $F_0(m, p)$ é dada pela expressão~\eqref{Eq:Def_F0_integrado}. Com o auxílio dessas expressões, obtemos
\begin{align}
	\langle\bar{\psi}\psi\rangle &= - n_f n_c \frac{m}{\pi^2} (F_0(m,\Lambda) - F_0(m, p_F)) \label{Eq:Dens_Escalar_NJL_Gv_0}\\
	&= n_f n_c \frac{m}{\pi^2} (F_0(m,p_F) - F_0(m, \Lambda)), \\
	&\equiv \rho_s.
\end{align}
%
Portanto, precisamos resolver a seguinte equação\footnote{Equação do Gap}
\begin{equation}\label{Eq:Eq_Gap_NJL}
m = m_0 - 2 G_S\rho_s.
\end{equation}
%
Para isso, basta reescrevermos a expressão acima como
\begin{equation}
	m - m_0 + 2 G_S\rho_s = 0
\end{equation}
%
e utilizar um método para encontrar zeros de funções, variando $m$ em uma janela de valores que contenha o valor adequado da massa efetiva (entre algo próximo de zero e \np[GeV]{1}, por exemplo). Note que o valor de $\rho_s$ depende de $m$ e deve ser calculado para cada ``chute'' de $m$.

%%%%%%%%%%%%%%%%%%%%%%%%%%%%%%%%%%%%%%%%%%%%%%%%%%%%%%%%%%%%%%%%%%%
\subsection{Determinação de $\tilde{\omega}$ para o caso $G_V = 0$}
%%%%%%%%%%%%%%%%%%%%%%%%%%%%%%%%%%%%%%%%%%%%%%%%%%%%%%%%%%%%%%%%%%%

Se ignorarmos o termo em $G_V$, temos que 
\begin{equation}
	\mu_R = \mu
\end{equation}
%
e também que o termo $(\mu-\mu_r)^2/ 4G_V$ não contribui no potencial termodinâmico. Se escolhermos efetuar o cálculo variando $\rho_B$ arbitrariamente, podemos calcular as demais variáveis através de
\begin{align}
	p_F &= \sqrt[3]{3 \pi^2 \rho_B / N_f} \\
	\mu_R &= \sqrt{p_F^2 + m^2}
\end{align}

Uma vez calculados os valores de $m$, $m_{\rm{vac}}$ e $\mu_R$, precisamos calcular $\tilde\omega$ através de $\omega_{\rm{MF}}$. Temos, através das Equações~\eqref{Eq:Def_omega_tilde}, \eqref{Eq:Def_omega_vac}, e~\eqref{Eq:Def_omega_med}, que:
\begin{equation}\label{Eq:Pot_term_MF_sem_G_V}
\begin{split}
\omega_{\rm{MF}} =&~ -2 n_f n_c \left[\int\frac{d^3p}{(2\pi)^3} E_p + \int\frac{d^3p}{(2\pi)^3}(\mu_R - E_p)\theta(p_F - p)\right] \\
&~+ \frac{(m-m_0)^2}{4G_S}
\end{split}
\end{equation}
%
Utilizando a expressão~\eqref{Eq:Int_d3p_to_dp}, podemos reescrever tal expressão como (separando a integral de $\mu_R$ e explicitando os limites)
\begin{equation}
\begin{split}
\omega_{\rm{MF}} =&~ - \frac{n_f n_c}{\pi^2} \Big[\int_0^\Lambda p^2E_p \; dp - \int_0^\Lambda p^2 E_p \theta(p_F - p)\;dp \\
&~\phantom{- \frac{n_f n_c}{\pi^2} \Big[} + \int_0^\Lambda p^2 \mu_R \theta(p_F - p)\;dp \Big] + \frac{(m-m_0)^2}{4G_S}
\end{split}
\end{equation}

Na expressão acima, verificamos a existência de duas integrais idênticas --~a menos de uma função degrau $\theta$~-- sendo que estamos calculando a diferença entre elas. Tal resultado será zero, exceto para $p_F > p$, pois nesse caso a integral que contém a função degrau será zerada. O efeito líquido disso é o de calcularmos somente a integral sem a função $\theta$, porém no intervalo $[p_F, \Lambda]$. Assim,
\begin{equation}
\begin{split}
\omega_{\rm{MF}} =&~ - \frac{n_f n_c}{\pi^2} \left[\int_{p_F}^\Lambda p^2E_p \; dp + \int_0^\Lambda p^2 \mu_R \theta(p_F - p)\;dp \right] \\
&~ + \frac{(m-m_0)^2}{4G_S} \\
\end{split}
\end{equation}
%
Como $\mu_R$ não depende do momento $p$, temos
\begin{equation}
	\int_0^\Lambda \mu_R \theta(p_F - p) \; dp = \mu_R \int_{0}^{p_F} \; dp = \mu_R\frac{p_F^3}{3}.
\end{equation}
%
Portanto,
\begin{equation}
\omega_{\rm{MF}} = - \frac{n_f n_c}{\pi^2} \left[\int_{p_F}^\Lambda p^2E_p \; dp + \mu_R \frac{p_F^3}{3} \right] + \frac{(m-m_0)^2}{4G_S}.
\end{equation}

Utilizando a definição de $E_p$, Equação~\eqref{Eq:Def_E}, podemos calcular a integral utilizando a expressão \eqref{Eq:Def_F_E}, o que nos leva a
\begin{equation}\label{Eq:Pot_termodinamico_1}
\omega_{\rm{MF}} = - \frac{n_f n_c}{\pi^2} \left[[F_E(m,\Lambda) - F_E(m, p_F)] + \mu_R \frac{p_F^3}{3} \right] + \frac{(m-m_0)^2}{4G_S}.
\end{equation}

O resultado  para $\omega_{\rm{MF}}(0;m_{\rm{vac}},0)$ utilizando o valor para $m_{\rm{vac}}$ obtido através da Eq.~\eqref{Eq:Calculo_m_vac} será uma constante, o que pode ser entendido verificando que a expressão para o potencial termodinâmico nessas condições é calculada a partir de constantes. Além disso, como $\mu = \mu_R = 0$, temos que $p_F = 0$:
\begin{equation}\label{Eq:Pot_termodinamico_1_vacuo}
\omega_{\rm{MF}}(0;m_{\rm{vac}},0) = - \frac{n_f n_c}{\pi^2}[F_E(m_{\rm{vac}},\Lambda) - F_E(m_{\rm{vac}}, 0)] + \frac{(m-m_0)^2}{4G_S}.
\end{equation}

%
%Em especial, para o caso em que não temos a interação vetorial (o que equivale a dizer que $G_V = 0$, temos que o potencial químico renormalizado é igual ao potencial químico:
%\begin{equation}
%	\mu_R = \mu.
%\end{equation}
%
%Dessa forma, podemos solucionar a Eq.~\eqref{Eq:Eq_Gap_Buballa_1} se calcularmos o momento de Fermi através da expressão~\eqref{Eq:Rel_Dens_Mom_Fermi_NJL}, obtendo\footnote{Para calcular a densidade escalar, basta sabermos os valores de $p_F$ e de $\Lambda$.}
%\begin{equation}
%	p_F = \sqrt[3]{\frac{3\pi^2\rho_B}{n_f}}.
%\end{equation}
%
%Ao solucionarmos a Eq.~\eqref{Eq:Eq_Gap_Buballa_1}, obteremos o valor de $m$, porém não é possível determinar o valor de $\mu_R$.
%\begin{figure*}
%	\begin{tikzpicture}[gnuplot]
%% generated with GNUPLOT 5.0p2 (Lua 5.2; terminal rev. 99, script rev. 100)
%% Fri Mar 18 15:48:26 2016
\path (0.000,0.000) rectangle (14.000,9.000);
\gpcolor{color=gp lt color border}
\gpsetlinetype{gp lt border}
\gpsetdashtype{gp dt solid}
\gpsetlinewidth{1.00}
\draw[gp path] (1.320,1.680)--(1.500,1.680);
\draw[gp path] (13.447,1.680)--(13.267,1.680);
\node[gp node right] at (1.136,1.680) {$0$};
\draw[gp path] (1.320,3.070)--(1.500,3.070);
\draw[gp path] (13.447,3.070)--(13.267,3.070);
\node[gp node right] at (1.136,3.070) {$100$};
\draw[gp path] (1.320,4.460)--(1.500,4.460);
\draw[gp path] (13.447,4.460)--(13.267,4.460);
\node[gp node right] at (1.136,4.460) {$200$};
\draw[gp path] (1.320,5.851)--(1.500,5.851);
\draw[gp path] (13.447,5.851)--(13.267,5.851);
\node[gp node right] at (1.136,5.851) {$300$};
\draw[gp path] (1.320,7.241)--(1.500,7.241);
\draw[gp path] (13.447,7.241)--(13.267,7.241);
\node[gp node right] at (1.136,7.241) {$400$};
\draw[gp path] (1.320,8.631)--(1.500,8.631);
\draw[gp path] (13.447,8.631)--(13.267,8.631);
\node[gp node right] at (1.136,8.631) {$500$};
\draw[gp path] (1.320,0.985)--(1.320,1.165);
\draw[gp path] (1.320,8.631)--(1.320,8.451);
\node[gp node center] at (1.320,0.677) {$0$};
\draw[gp path] (3.745,0.985)--(3.745,1.165);
\draw[gp path] (3.745,8.631)--(3.745,8.451);
\node[gp node center] at (3.745,0.677) {$100$};
\draw[gp path] (6.171,0.985)--(6.171,1.165);
\draw[gp path] (6.171,8.631)--(6.171,8.451);
\node[gp node center] at (6.171,0.677) {$200$};
\draw[gp path] (8.596,0.985)--(8.596,1.165);
\draw[gp path] (8.596,8.631)--(8.596,8.451);
\node[gp node center] at (8.596,0.677) {$300$};
\draw[gp path] (11.022,0.985)--(11.022,1.165);
\draw[gp path] (11.022,8.631)--(11.022,8.451);
\node[gp node center] at (11.022,0.677) {$400$};
\draw[gp path] (13.447,0.985)--(13.447,1.165);
\draw[gp path] (13.447,8.631)--(13.447,8.451);
\node[gp node center] at (13.447,0.677) {$500$};
\draw[gp path] (1.320,8.631)--(1.320,0.985)--(13.447,0.985)--(13.447,8.631)--cycle;
\node[gp node center,rotate=-270] at (0.246,4.808) {$p_F$};
\node[gp node center] at (7.383,0.215) {$\mu_R$};
\node[gp node left] at (2.788,8.297) {$p_F = \sqrt{\mu_R^2 - m^2}\theta(\mu_R^2 - m^2), m = 100$};
\gpcolor{rgb color={0.580,0.000,0.827}}
\gpsetlinewidth{3.00}
\draw[gp path] (1.688,8.297)--(2.604,8.297);
\draw[gp path] (1.320,1.680)--(1.332,1.680)--(1.344,1.680)--(1.356,1.680)--(1.369,1.680)%
  --(1.381,1.680)--(1.393,1.680)--(1.405,1.680)--(1.417,1.680)--(1.429,1.680)--(1.441,1.680)%
  --(1.453,1.680)--(1.466,1.680)--(1.478,1.680)--(1.490,1.680)--(1.502,1.680)--(1.514,1.680)%
  --(1.526,1.680)--(1.538,1.680)--(1.550,1.680)--(1.563,1.680)--(1.575,1.680)--(1.587,1.680)%
  --(1.599,1.680)--(1.611,1.680)--(1.623,1.680)--(1.635,1.680)--(1.647,1.680)--(1.660,1.680)%
  --(1.672,1.680)--(1.684,1.680)--(1.696,1.680)--(1.708,1.680)--(1.720,1.680)--(1.732,1.680)%
  --(1.744,1.680)--(1.757,1.680)--(1.769,1.680)--(1.781,1.680)--(1.793,1.680)--(1.805,1.680)%
  --(1.817,1.680)--(1.829,1.680)--(1.841,1.680)--(1.854,1.680)--(1.866,1.680)--(1.878,1.680)%
  --(1.890,1.680)--(1.902,1.680)--(1.914,1.680)--(1.926,1.680)--(1.938,1.680)--(1.951,1.680)%
  --(1.963,1.680)--(1.975,1.680)--(1.987,1.680)--(1.999,1.680)--(2.011,1.680)--(2.023,1.680)%
  --(2.035,1.680)--(2.048,1.680)--(2.060,1.680)--(2.072,1.680)--(2.084,1.680)--(2.096,1.680)%
  --(2.108,1.680)--(2.120,1.680)--(2.133,1.680)--(2.145,1.680)--(2.157,1.680)--(2.169,1.680)%
  --(2.181,1.680)--(2.193,1.680)--(2.205,1.680)--(2.217,1.680)--(2.230,1.680)--(2.242,1.680)%
  --(2.254,1.680)--(2.266,1.680)--(2.278,1.680)--(2.290,1.680)--(2.302,1.680)--(2.314,1.680)%
  --(2.327,1.680)--(2.339,1.680)--(2.351,1.680)--(2.363,1.680)--(2.375,1.680)--(2.387,1.680)%
  --(2.399,1.680)--(2.411,1.680)--(2.424,1.680)--(2.436,1.680)--(2.448,1.680)--(2.460,1.680)%
  --(2.472,1.680)--(2.484,1.680)--(2.496,1.680)--(2.508,1.680)--(2.521,1.680)--(2.533,1.680)%
  --(2.545,1.680)--(2.557,1.680)--(2.569,1.680)--(2.581,1.680)--(2.593,1.680)--(2.605,1.680)%
  --(2.618,1.680)--(2.630,1.680)--(2.642,1.680)--(2.654,1.680)--(2.666,1.680)--(2.678,1.680)%
  --(2.690,1.680)--(2.702,1.680)--(2.715,1.680)--(2.727,1.680)--(2.739,1.680)--(2.751,1.680)%
  --(2.763,1.680)--(2.775,1.680)--(2.787,1.680)--(2.799,1.680)--(2.812,1.680)--(2.824,1.680)%
  --(2.836,1.680)--(2.848,1.680)--(2.860,1.680)--(2.872,1.680)--(2.884,1.680)--(2.897,1.680)%
  --(2.909,1.680)--(2.921,1.680)--(2.933,1.680)--(2.945,1.680)--(2.957,1.680)--(2.969,1.680)%
  --(2.981,1.680)--(2.994,1.680)--(3.006,1.680)--(3.018,1.680)--(3.030,1.680)--(3.042,1.680)%
  --(3.054,1.680)--(3.066,1.680)--(3.078,1.680)--(3.091,1.680)--(3.103,1.680)--(3.115,1.680)%
  --(3.127,1.680)--(3.139,1.680)--(3.151,1.680)--(3.163,1.680)--(3.175,1.680)--(3.188,1.680)%
  --(3.200,1.680)--(3.212,1.680)--(3.224,1.680)--(3.236,1.680)--(3.248,1.680)--(3.260,1.680)%
  --(3.272,1.680)--(3.285,1.680)--(3.297,1.680)--(3.309,1.680)--(3.321,1.680)--(3.333,1.680)%
  --(3.345,1.680)--(3.357,1.680)--(3.369,1.680)--(3.382,1.680)--(3.394,1.680)--(3.406,1.680)%
  --(3.418,1.680)--(3.430,1.680)--(3.442,1.680)--(3.454,1.680)--(3.466,1.680)--(3.479,1.680)%
  --(3.491,1.680)--(3.503,1.680)--(3.515,1.680)--(3.527,1.680)--(3.539,1.680)--(3.551,1.680)%
  --(3.563,1.680)--(3.576,1.680)--(3.588,1.680)--(3.600,1.680)--(3.612,1.680)--(3.624,1.680)%
  --(3.636,1.680)--(3.648,1.680)--(3.661,1.680)--(3.673,1.680)--(3.685,1.680)--(3.697,1.680)%
  --(3.709,1.680)--(3.721,1.680)--(3.733,1.680)--(3.745,1.680)--(3.758,1.819)--(3.770,1.877)%
  --(3.782,1.922)--(3.794,1.960)--(3.806,1.993)--(3.818,2.023)--(3.830,2.051)--(3.842,2.077)%
  --(3.855,2.102)--(3.867,2.125)--(3.879,2.147)--(3.891,2.169)--(3.903,2.189)--(3.915,2.209)%
  --(3.927,2.229)--(3.939,2.247)--(3.952,2.265)--(3.964,2.283)--(3.976,2.300)--(3.988,2.317)%
  --(4.000,2.334)--(4.012,2.350)--(4.024,2.366)--(4.036,2.381)--(4.049,2.397)--(4.061,2.412)%
  --(4.073,2.426)--(4.085,2.441)--(4.097,2.455)--(4.109,2.470)--(4.121,2.484)--(4.133,2.497)%
  --(4.146,2.511)--(4.158,2.524)--(4.170,2.538)--(4.182,2.551)--(4.194,2.564)--(4.206,2.577)%
  --(4.218,2.590)--(4.230,2.602)--(4.243,2.615)--(4.255,2.627)--(4.267,2.639)--(4.279,2.652)%
  --(4.291,2.664)--(4.303,2.676)--(4.315,2.688)--(4.327,2.699)--(4.340,2.711)--(4.352,2.723)%
  --(4.364,2.734)--(4.376,2.746)--(4.388,2.757)--(4.400,2.768)--(4.412,2.780)--(4.425,2.791)%
  --(4.437,2.802)--(4.449,2.813)--(4.461,2.824)--(4.473,2.835)--(4.485,2.846)--(4.497,2.857)%
  --(4.509,2.867)--(4.522,2.878)--(4.534,2.889)--(4.546,2.899)--(4.558,2.910)--(4.570,2.920)%
  --(4.582,2.931)--(4.594,2.941)--(4.606,2.951)--(4.619,2.961)--(4.631,2.972)--(4.643,2.982)%
  --(4.655,2.992)--(4.667,3.002)--(4.679,3.012)--(4.691,3.022)--(4.703,3.032)--(4.716,3.042)%
  --(4.728,3.052)--(4.740,3.062)--(4.752,3.072)--(4.764,3.082)--(4.776,3.091)--(4.788,3.101)%
  --(4.800,3.111)--(4.813,3.121)--(4.825,3.130)--(4.837,3.140)--(4.849,3.149)--(4.861,3.159)%
  --(4.873,3.168)--(4.885,3.178)--(4.897,3.187)--(4.910,3.197)--(4.922,3.206)--(4.934,3.216)%
  --(4.946,3.225)--(4.958,3.234)--(4.970,3.244)--(4.982,3.253)--(4.994,3.262)--(5.007,3.271)%
  --(5.019,3.281)--(5.031,3.290)--(5.043,3.299)--(5.055,3.308)--(5.067,3.317)--(5.079,3.326)%
  --(5.091,3.336)--(5.104,3.345)--(5.116,3.354)--(5.128,3.363)--(5.140,3.372)--(5.152,3.381)%
  --(5.164,3.390)--(5.176,3.399)--(5.189,3.408)--(5.201,3.416)--(5.213,3.425)--(5.225,3.434)%
  --(5.237,3.443)--(5.249,3.452)--(5.261,3.461)--(5.273,3.470)--(5.286,3.478)--(5.298,3.487)%
  --(5.310,3.496)--(5.322,3.505)--(5.334,3.513)--(5.346,3.522)--(5.358,3.531)--(5.370,3.539)%
  --(5.383,3.548)--(5.395,3.557)--(5.407,3.565)--(5.419,3.574)--(5.431,3.583)--(5.443,3.591)%
  --(5.455,3.600)--(5.467,3.608)--(5.480,3.617)--(5.492,3.626)--(5.504,3.634)--(5.516,3.643)%
  --(5.528,3.651)--(5.540,3.660)--(5.552,3.668)--(5.564,3.677)--(5.577,3.685)--(5.589,3.694)%
  --(5.601,3.702)--(5.613,3.710)--(5.625,3.719)--(5.637,3.727)--(5.649,3.736)--(5.661,3.744)%
  --(5.674,3.752)--(5.686,3.761)--(5.698,3.769)--(5.710,3.777)--(5.722,3.786)--(5.734,3.794)%
  --(5.746,3.802)--(5.758,3.811)--(5.771,3.819)--(5.783,3.827)--(5.795,3.836)--(5.807,3.844)%
  --(5.819,3.852)--(5.831,3.860)--(5.843,3.869)--(5.855,3.877)--(5.868,3.885)--(5.880,3.893)%
  --(5.892,3.901)--(5.904,3.910)--(5.916,3.918)--(5.928,3.926)--(5.940,3.934)--(5.953,3.942)%
  --(5.965,3.950)--(5.977,3.959)--(5.989,3.967)--(6.001,3.975)--(6.013,3.983)--(6.025,3.991)%
  --(6.037,3.999)--(6.050,4.007)--(6.062,4.015)--(6.074,4.024)--(6.086,4.032)--(6.098,4.040)%
  --(6.110,4.048)--(6.122,4.056)--(6.134,4.064)--(6.147,4.072)--(6.159,4.080)--(6.171,4.088)%
  --(6.183,4.096)--(6.195,4.104)--(6.207,4.112)--(6.219,4.120)--(6.231,4.128)--(6.244,4.136)%
  --(6.256,4.144)--(6.268,4.152)--(6.280,4.160)--(6.292,4.168)--(6.304,4.176)--(6.316,4.184)%
  --(6.328,4.192)--(6.341,4.200)--(6.353,4.208)--(6.365,4.216)--(6.377,4.223)--(6.389,4.231)%
  --(6.401,4.239)--(6.413,4.247)--(6.425,4.255)--(6.438,4.263)--(6.450,4.271)--(6.462,4.279)%
  --(6.474,4.287)--(6.486,4.295)--(6.498,4.302)--(6.510,4.310)--(6.522,4.318)--(6.535,4.326)%
  --(6.547,4.334)--(6.559,4.342)--(6.571,4.350)--(6.583,4.357)--(6.595,4.365)--(6.607,4.373)%
  --(6.619,4.381)--(6.632,4.389)--(6.644,4.396)--(6.656,4.404)--(6.668,4.412)--(6.680,4.420)%
  --(6.692,4.428)--(6.704,4.435)--(6.717,4.443)--(6.729,4.451)--(6.741,4.459)--(6.753,4.467)%
  --(6.765,4.474)--(6.777,4.482)--(6.789,4.490)--(6.801,4.498)--(6.814,4.505)--(6.826,4.513)%
  --(6.838,4.521)--(6.850,4.529)--(6.862,4.536)--(6.874,4.544)--(6.886,4.552)--(6.898,4.559)%
  --(6.911,4.567)--(6.923,4.575)--(6.935,4.583)--(6.947,4.590)--(6.959,4.598)--(6.971,4.606)%
  --(6.983,4.613)--(6.995,4.621)--(7.008,4.629)--(7.020,4.636)--(7.032,4.644)--(7.044,4.652)%
  --(7.056,4.660)--(7.068,4.667)--(7.080,4.675)--(7.092,4.682)--(7.105,4.690)--(7.117,4.698)%
  --(7.129,4.705)--(7.141,4.713)--(7.153,4.721)--(7.165,4.728)--(7.177,4.736)--(7.189,4.744)%
  --(7.202,4.751)--(7.214,4.759)--(7.226,4.767)--(7.238,4.774)--(7.250,4.782)--(7.262,4.789)%
  --(7.274,4.797)--(7.286,4.805)--(7.299,4.812)--(7.311,4.820)--(7.323,4.827)--(7.335,4.835)%
  --(7.347,4.843)--(7.359,4.850)--(7.371,4.858)--(7.384,4.865)--(7.396,4.873)--(7.408,4.881)%
  --(7.420,4.888)--(7.432,4.896)--(7.444,4.903)--(7.456,4.911)--(7.468,4.918)--(7.481,4.926)%
  --(7.493,4.934)--(7.505,4.941)--(7.517,4.949)--(7.529,4.956)--(7.541,4.964)--(7.553,4.971)%
  --(7.565,4.979)--(7.578,4.986)--(7.590,4.994)--(7.602,5.001)--(7.614,5.009)--(7.626,5.017)%
  --(7.638,5.024)--(7.650,5.032)--(7.662,5.039)--(7.675,5.047)--(7.687,5.054)--(7.699,5.062)%
  --(7.711,5.069)--(7.723,5.077)--(7.735,5.084)--(7.747,5.092)--(7.759,5.099)--(7.772,5.107)%
  --(7.784,5.114)--(7.796,5.122)--(7.808,5.129)--(7.820,5.137)--(7.832,5.144)--(7.844,5.152)%
  --(7.856,5.159)--(7.869,5.167)--(7.881,5.174)--(7.893,5.182)--(7.905,5.189)--(7.917,5.197)%
  --(7.929,5.204)--(7.941,5.212)--(7.953,5.219)--(7.966,5.226)--(7.978,5.234)--(7.990,5.241)%
  --(8.002,5.249)--(8.014,5.256)--(8.026,5.264)--(8.038,5.271)--(8.050,5.279)--(8.063,5.286)%
  --(8.075,5.294)--(8.087,5.301)--(8.099,5.308)--(8.111,5.316)--(8.123,5.323)--(8.135,5.331)%
  --(8.148,5.338)--(8.160,5.346)--(8.172,5.353)--(8.184,5.361)--(8.196,5.368)--(8.208,5.375)%
  --(8.220,5.383)--(8.232,5.390)--(8.245,5.398)--(8.257,5.405)--(8.269,5.412)--(8.281,5.420)%
  --(8.293,5.427)--(8.305,5.435)--(8.317,5.442)--(8.329,5.450)--(8.342,5.457)--(8.354,5.464)%
  --(8.366,5.472)--(8.378,5.479)--(8.390,5.487)--(8.402,5.494)--(8.414,5.501)--(8.426,5.509)%
  --(8.439,5.516)--(8.451,5.524)--(8.463,5.531)--(8.475,5.538)--(8.487,5.546)--(8.499,5.553)%
  --(8.511,5.560)--(8.523,5.568)--(8.536,5.575)--(8.548,5.583)--(8.560,5.590)--(8.572,5.597)%
  --(8.584,5.605)--(8.596,5.612)--(8.608,5.619)--(8.620,5.627)--(8.633,5.634)--(8.645,5.642)%
  --(8.657,5.649)--(8.669,5.656)--(8.681,5.664)--(8.693,5.671)--(8.705,5.678)--(8.717,5.686)%
  --(8.730,5.693)--(8.742,5.700)--(8.754,5.708)--(8.766,5.715)--(8.778,5.723)--(8.790,5.730)%
  --(8.802,5.737)--(8.814,5.745)--(8.827,5.752)--(8.839,5.759)--(8.851,5.767)--(8.863,5.774)%
  --(8.875,5.781)--(8.887,5.789)--(8.899,5.796)--(8.912,5.803)--(8.924,5.811)--(8.936,5.818)%
  --(8.948,5.825)--(8.960,5.833)--(8.972,5.840)--(8.984,5.847)--(8.996,5.855)--(9.009,5.862)%
  --(9.021,5.869)--(9.033,5.877)--(9.045,5.884)--(9.057,5.891)--(9.069,5.899)--(9.081,5.906)%
  --(9.093,5.913)--(9.106,5.921)--(9.118,5.928)--(9.130,5.935)--(9.142,5.942)--(9.154,5.950)%
  --(9.166,5.957)--(9.178,5.964)--(9.190,5.972)--(9.203,5.979)--(9.215,5.986)--(9.227,5.994)%
  --(9.239,6.001)--(9.251,6.008)--(9.263,6.016)--(9.275,6.023)--(9.287,6.030)--(9.300,6.037)%
  --(9.312,6.045)--(9.324,6.052)--(9.336,6.059)--(9.348,6.067)--(9.360,6.074)--(9.372,6.081)%
  --(9.384,6.088)--(9.397,6.096)--(9.409,6.103)--(9.421,6.110)--(9.433,6.118)--(9.445,6.125)%
  --(9.457,6.132)--(9.469,6.139)--(9.481,6.147)--(9.494,6.154)--(9.506,6.161)--(9.518,6.169)%
  --(9.530,6.176)--(9.542,6.183)--(9.554,6.190)--(9.566,6.198)--(9.578,6.205)--(9.591,6.212)%
  --(9.603,6.219)--(9.615,6.227)--(9.627,6.234)--(9.639,6.241)--(9.651,6.249)--(9.663,6.256)%
  --(9.676,6.263)--(9.688,6.270)--(9.700,6.278)--(9.712,6.285)--(9.724,6.292)--(9.736,6.299)%
  --(9.748,6.307)--(9.760,6.314)--(9.773,6.321)--(9.785,6.328)--(9.797,6.336)--(9.809,6.343)%
  --(9.821,6.350)--(9.833,6.357)--(9.845,6.365)--(9.857,6.372)--(9.870,6.379)--(9.882,6.386)%
  --(9.894,6.394)--(9.906,6.401)--(9.918,6.408)--(9.930,6.415)--(9.942,6.423)--(9.954,6.430)%
  --(9.967,6.437)--(9.979,6.444)--(9.991,6.452)--(10.003,6.459)--(10.015,6.466)--(10.027,6.473)%
  --(10.039,6.481)--(10.051,6.488)--(10.064,6.495)--(10.076,6.502)--(10.088,6.509)--(10.100,6.517)%
  --(10.112,6.524)--(10.124,6.531)--(10.136,6.538)--(10.148,6.546)--(10.161,6.553)--(10.173,6.560)%
  --(10.185,6.567)--(10.197,6.575)--(10.209,6.582)--(10.221,6.589)--(10.233,6.596)--(10.245,6.603)%
  --(10.258,6.611)--(10.270,6.618)--(10.282,6.625)--(10.294,6.632)--(10.306,6.640)--(10.318,6.647)%
  --(10.330,6.654)--(10.342,6.661)--(10.355,6.668)--(10.367,6.676)--(10.379,6.683)--(10.391,6.690)%
  --(10.403,6.697)--(10.415,6.704)--(10.427,6.712)--(10.440,6.719)--(10.452,6.726)--(10.464,6.733)%
  --(10.476,6.741)--(10.488,6.748)--(10.500,6.755)--(10.512,6.762)--(10.524,6.769)--(10.537,6.777)%
  --(10.549,6.784)--(10.561,6.791)--(10.573,6.798)--(10.585,6.805)--(10.597,6.813)--(10.609,6.820)%
  --(10.621,6.827)--(10.634,6.834)--(10.646,6.841)--(10.658,6.849)--(10.670,6.856)--(10.682,6.863)%
  --(10.694,6.870)--(10.706,6.877)--(10.718,6.885)--(10.731,6.892)--(10.743,6.899)--(10.755,6.906)%
  --(10.767,6.913)--(10.779,6.921)--(10.791,6.928)--(10.803,6.935)--(10.815,6.942)--(10.828,6.949)%
  --(10.840,6.956)--(10.852,6.964)--(10.864,6.971)--(10.876,6.978)--(10.888,6.985)--(10.900,6.992)%
  --(10.912,7.000)--(10.925,7.007)--(10.937,7.014)--(10.949,7.021)--(10.961,7.028)--(10.973,7.036)%
  --(10.985,7.043)--(10.997,7.050)--(11.009,7.057)--(11.022,7.064)--(11.034,7.071)--(11.046,7.079)%
  --(11.058,7.086)--(11.070,7.093)--(11.082,7.100)--(11.094,7.107)--(11.106,7.114)--(11.119,7.122)%
  --(11.131,7.129)--(11.143,7.136)--(11.155,7.143)--(11.167,7.150)--(11.179,7.158)--(11.191,7.165)%
  --(11.204,7.172)--(11.216,7.179)--(11.228,7.186)--(11.240,7.193)--(11.252,7.201)--(11.264,7.208)%
  --(11.276,7.215)--(11.288,7.222)--(11.301,7.229)--(11.313,7.236)--(11.325,7.244)--(11.337,7.251)%
  --(11.349,7.258)--(11.361,7.265)--(11.373,7.272)--(11.385,7.279)--(11.398,7.287)--(11.410,7.294)%
  --(11.422,7.301)--(11.434,7.308)--(11.446,7.315)--(11.458,7.322)--(11.470,7.329)--(11.482,7.337)%
  --(11.495,7.344)--(11.507,7.351)--(11.519,7.358)--(11.531,7.365)--(11.543,7.372)--(11.555,7.380)%
  --(11.567,7.387)--(11.579,7.394)--(11.592,7.401)--(11.604,7.408)--(11.616,7.415)--(11.628,7.422)%
  --(11.640,7.430)--(11.652,7.437)--(11.664,7.444)--(11.676,7.451)--(11.689,7.458)--(11.701,7.465)%
  --(11.713,7.473)--(11.725,7.480)--(11.737,7.487)--(11.749,7.494)--(11.761,7.501)--(11.773,7.508)%
  --(11.786,7.515)--(11.798,7.523)--(11.810,7.530)--(11.822,7.537)--(11.834,7.544)--(11.846,7.551)%
  --(11.858,7.558)--(11.870,7.565)--(11.883,7.573)--(11.895,7.580)--(11.907,7.587)--(11.919,7.594)%
  --(11.931,7.601)--(11.943,7.608)--(11.955,7.615)--(11.968,7.623)--(11.980,7.630)--(11.992,7.637)%
  --(12.004,7.644)--(12.016,7.651)--(12.028,7.658)--(12.040,7.665)--(12.052,7.673)--(12.065,7.680)%
  --(12.077,7.687)--(12.089,7.694)--(12.101,7.701)--(12.113,7.708)--(12.125,7.715)--(12.137,7.722)%
  --(12.149,7.730)--(12.162,7.737)--(12.174,7.744)--(12.186,7.751)--(12.198,7.758)--(12.210,7.765)%
  --(12.222,7.772)--(12.234,7.779)--(12.246,7.787)--(12.259,7.794)--(12.271,7.801)--(12.283,7.808)%
  --(12.295,7.815)--(12.307,7.822)--(12.319,7.829)--(12.331,7.837)--(12.343,7.844)--(12.356,7.851)%
  --(12.368,7.858)--(12.380,7.865)--(12.392,7.872)--(12.404,7.879)--(12.416,7.886)--(12.428,7.894)%
  --(12.440,7.901)--(12.453,7.908)--(12.465,7.915)--(12.477,7.922)--(12.489,7.929)--(12.501,7.936)%
  --(12.513,7.943)--(12.525,7.950)--(12.537,7.958)--(12.550,7.965)--(12.562,7.972)--(12.574,7.979)%
  --(12.586,7.986)--(12.598,7.993)--(12.610,8.000)--(12.622,8.007)--(12.634,8.015)--(12.647,8.022)%
  --(12.659,8.029)--(12.671,8.036)--(12.683,8.043)--(12.695,8.050)--(12.707,8.057)--(12.719,8.064)%
  --(12.732,8.071)--(12.744,8.079)--(12.756,8.086)--(12.768,8.093)--(12.780,8.100)--(12.792,8.107)%
  --(12.804,8.114)--(12.816,8.121)--(12.829,8.128)--(12.841,8.135)--(12.853,8.143)--(12.865,8.150)%
  --(12.877,8.157)--(12.889,8.164)--(12.901,8.171)--(12.913,8.178)--(12.926,8.185)--(12.938,8.192)%
  --(12.950,8.199)--(12.962,8.207)--(12.974,8.214)--(12.986,8.221)--(12.998,8.228)--(13.010,8.235)%
  --(13.023,8.242)--(13.035,8.249)--(13.047,8.256)--(13.059,8.263)--(13.071,8.270)--(13.083,8.278)%
  --(13.095,8.285)--(13.107,8.292)--(13.120,8.299)--(13.132,8.306)--(13.144,8.313)--(13.156,8.320)%
  --(13.168,8.327)--(13.180,8.334)--(13.192,8.342)--(13.204,8.349)--(13.217,8.356)--(13.229,8.363)%
  --(13.241,8.370)--(13.253,8.377)--(13.265,8.384)--(13.277,8.391)--(13.289,8.398)--(13.301,8.405)%
  --(13.314,8.413)--(13.326,8.420)--(13.338,8.427)--(13.350,8.434)--(13.362,8.441)--(13.374,8.448)%
  --(13.386,8.455)--(13.398,8.462)--(13.411,8.469)--(13.423,8.476)--(13.435,8.483)--(13.447,8.491);
\gpcolor{rgb color={0.000,0.620,0.451}}
\gpsetlinewidth{1.00}
\draw[gp path] (1.320,1.680)--(1.332,1.680)--(1.344,1.680)--(1.356,1.680)--(1.369,1.680)%
  --(1.381,1.680)--(1.393,1.680)--(1.405,1.680)--(1.417,1.680)--(1.429,1.680)--(1.441,1.680)%
  --(1.453,1.680)--(1.466,1.680)--(1.478,1.680)--(1.490,1.680)--(1.502,1.680)--(1.514,1.680)%
  --(1.526,1.680)--(1.538,1.680)--(1.550,1.680)--(1.563,1.680)--(1.575,1.680)--(1.587,1.680)%
  --(1.599,1.680)--(1.611,1.680)--(1.623,1.680)--(1.635,1.680)--(1.647,1.680)--(1.660,1.680)%
  --(1.672,1.680)--(1.684,1.680)--(1.696,1.680)--(1.708,1.680)--(1.720,1.680)--(1.732,1.680)%
  --(1.744,1.680)--(1.757,1.680)--(1.769,1.680)--(1.781,1.680)--(1.793,1.680)--(1.805,1.680)%
  --(1.817,1.680)--(1.829,1.680)--(1.841,1.680)--(1.854,1.680)--(1.866,1.680)--(1.878,1.680)%
  --(1.890,1.680)--(1.902,1.680)--(1.914,1.680)--(1.926,1.680)--(1.938,1.680)--(1.951,1.680)%
  --(1.963,1.680)--(1.975,1.680)--(1.987,1.680)--(1.999,1.680)--(2.011,1.680)--(2.023,1.680)%
  --(2.035,1.680)--(2.048,1.680)--(2.060,1.680)--(2.072,1.680)--(2.084,1.680)--(2.096,1.680)%
  --(2.108,1.680)--(2.120,1.680)--(2.133,1.680)--(2.145,1.680)--(2.157,1.680)--(2.169,1.680)%
  --(2.181,1.680)--(2.193,1.680)--(2.205,1.680)--(2.217,1.680)--(2.230,1.680)--(2.242,1.680)%
  --(2.254,1.680)--(2.266,1.680)--(2.278,1.680)--(2.290,1.680)--(2.302,1.680)--(2.314,1.680)%
  --(2.327,1.680)--(2.339,1.680)--(2.351,1.680)--(2.363,1.680)--(2.375,1.680)--(2.387,1.680)%
  --(2.399,1.680)--(2.411,1.680)--(2.424,1.680)--(2.436,1.680)--(2.448,1.680)--(2.460,1.680)%
  --(2.472,1.680)--(2.484,1.680)--(2.496,1.680)--(2.508,1.680)--(2.521,1.680)--(2.533,1.680)%
  --(2.545,1.680)--(2.557,1.680)--(2.569,1.680)--(2.581,1.680)--(2.593,1.680)--(2.605,1.680)%
  --(2.618,1.680)--(2.630,1.680)--(2.642,1.680)--(2.654,1.680)--(2.666,1.680)--(2.678,1.680)%
  --(2.690,1.680)--(2.702,1.680)--(2.715,1.680)--(2.727,1.680)--(2.739,1.680)--(2.751,1.680)%
  --(2.763,1.680)--(2.775,1.680)--(2.787,1.680)--(2.799,1.680)--(2.812,1.680)--(2.824,1.680)%
  --(2.836,1.680)--(2.848,1.680)--(2.860,1.680)--(2.872,1.680)--(2.884,1.680)--(2.897,1.680)%
  --(2.909,1.680)--(2.921,1.680)--(2.933,1.680)--(2.945,1.680)--(2.957,1.680)--(2.969,1.680)%
  --(2.981,1.680)--(2.994,1.680)--(3.006,1.680)--(3.018,1.680)--(3.030,1.680)--(3.042,1.680)%
  --(3.054,1.680)--(3.066,1.680)--(3.078,1.680)--(3.091,1.680)--(3.103,1.680)--(3.115,1.680)%
  --(3.127,1.680)--(3.139,1.680)--(3.151,1.680)--(3.163,1.680)--(3.175,1.680)--(3.188,1.680)%
  --(3.200,1.680)--(3.212,1.680)--(3.224,1.680)--(3.236,1.680)--(3.248,1.680)--(3.260,1.680)%
  --(3.272,1.680)--(3.285,1.680)--(3.297,1.680)--(3.309,1.680)--(3.321,1.680)--(3.333,1.680)%
  --(3.345,1.680)--(3.357,1.680)--(3.369,1.680)--(3.382,1.680)--(3.394,1.680)--(3.406,1.680)%
  --(3.418,1.680)--(3.430,1.680)--(3.442,1.680)--(3.454,1.680)--(3.466,1.680)--(3.479,1.680)%
  --(3.491,1.680)--(3.503,1.680)--(3.515,1.680)--(3.527,1.680)--(3.539,1.680)--(3.551,1.680)%
  --(3.563,1.680)--(3.576,1.680)--(3.588,1.680)--(3.600,1.680)--(3.612,1.680)--(3.624,1.680)%
  --(3.636,1.680)--(3.648,1.680)--(3.661,1.680)--(3.673,1.680)--(3.685,1.680)--(3.697,1.680)%
  --(3.709,1.680)--(3.721,1.680)--(3.733,1.680)--(3.745,1.680)--(3.758,1.680)--(3.770,1.680)%
  --(3.782,1.680)--(3.794,1.680)--(3.806,1.680)--(3.818,1.680)--(3.830,1.680)--(3.842,1.680)%
  --(3.855,1.680)--(3.867,1.680)--(3.879,1.680)--(3.891,1.680)--(3.903,1.680)--(3.915,1.680)%
  --(3.927,1.680)--(3.939,1.680)--(3.952,1.680)--(3.964,1.680)--(3.976,1.680)--(3.988,1.680)%
  --(4.000,1.680)--(4.012,1.680)--(4.024,1.680)--(4.036,1.680)--(4.049,1.680)--(4.061,1.680)%
  --(4.073,1.680)--(4.085,1.680)--(4.097,1.680)--(4.109,1.680)--(4.121,1.680)--(4.133,1.680)%
  --(4.146,1.680)--(4.158,1.680)--(4.170,1.680)--(4.182,1.680)--(4.194,1.680)--(4.206,1.680)%
  --(4.218,1.680)--(4.230,1.680)--(4.243,1.680)--(4.255,1.680)--(4.267,1.680)--(4.279,1.680)%
  --(4.291,1.680)--(4.303,1.680)--(4.315,1.680)--(4.327,1.680)--(4.340,1.680)--(4.352,1.680)%
  --(4.364,1.680)--(4.376,1.680)--(4.388,1.680)--(4.400,1.680)--(4.412,1.680)--(4.425,1.680)%
  --(4.437,1.680)--(4.449,1.680)--(4.461,1.680)--(4.473,1.680)--(4.485,1.680)--(4.497,1.680)%
  --(4.509,1.680)--(4.522,1.680)--(4.534,1.680)--(4.546,1.680)--(4.558,1.680)--(4.570,1.680)%
  --(4.582,1.680)--(4.594,1.680)--(4.606,1.680)--(4.619,1.680)--(4.631,1.680)--(4.643,1.680)%
  --(4.655,1.680)--(4.667,1.680)--(4.679,1.680)--(4.691,1.680)--(4.703,1.680)--(4.716,1.680)%
  --(4.728,1.680)--(4.740,1.680)--(4.752,1.680)--(4.764,1.680)--(4.776,1.680)--(4.788,1.680)%
  --(4.800,1.680)--(4.813,1.680)--(4.825,1.680)--(4.837,1.680)--(4.849,1.680)--(4.861,1.680)%
  --(4.873,1.680)--(4.885,1.680)--(4.897,1.680)--(4.910,1.680)--(4.922,1.680)--(4.934,1.680)%
  --(4.946,1.680)--(4.958,1.680)--(4.970,1.680)--(4.982,1.680)--(4.994,1.680)--(5.007,1.680)%
  --(5.019,1.680)--(5.031,1.680)--(5.043,1.680)--(5.055,1.680)--(5.067,1.680)--(5.079,1.680)%
  --(5.091,1.680)--(5.104,1.680)--(5.116,1.680)--(5.128,1.680)--(5.140,1.680)--(5.152,1.680)%
  --(5.164,1.680)--(5.176,1.680)--(5.189,1.680)--(5.201,1.680)--(5.213,1.680)--(5.225,1.680)%
  --(5.237,1.680)--(5.249,1.680)--(5.261,1.680)--(5.273,1.680)--(5.286,1.680)--(5.298,1.680)%
  --(5.310,1.680)--(5.322,1.680)--(5.334,1.680)--(5.346,1.680)--(5.358,1.680)--(5.370,1.680)%
  --(5.383,1.680)--(5.395,1.680)--(5.407,1.680)--(5.419,1.680)--(5.431,1.680)--(5.443,1.680)%
  --(5.455,1.680)--(5.467,1.680)--(5.480,1.680)--(5.492,1.680)--(5.504,1.680)--(5.516,1.680)%
  --(5.528,1.680)--(5.540,1.680)--(5.552,1.680)--(5.564,1.680)--(5.577,1.680)--(5.589,1.680)%
  --(5.601,1.680)--(5.613,1.680)--(5.625,1.680)--(5.637,1.680)--(5.649,1.680)--(5.661,1.680)%
  --(5.674,1.680)--(5.686,1.680)--(5.698,1.680)--(5.710,1.680)--(5.722,1.680)--(5.734,1.680)%
  --(5.746,1.680)--(5.758,1.680)--(5.771,1.680)--(5.783,1.680)--(5.795,1.680)--(5.807,1.680)%
  --(5.819,1.680)--(5.831,1.680)--(5.843,1.680)--(5.855,1.680)--(5.868,1.680)--(5.880,1.680)%
  --(5.892,1.680)--(5.904,1.680)--(5.916,1.680)--(5.928,1.680)--(5.940,1.680)--(5.953,1.680)%
  --(5.965,1.680)--(5.977,1.680)--(5.989,1.680)--(6.001,1.680)--(6.013,1.680)--(6.025,1.680)%
  --(6.037,1.680)--(6.050,1.680)--(6.062,1.680)--(6.074,1.680)--(6.086,1.680)--(6.098,1.680)%
  --(6.110,1.680)--(6.122,1.680)--(6.134,1.680)--(6.147,1.680)--(6.159,1.680)--(6.171,1.680)%
  --(6.183,1.680)--(6.195,1.680)--(6.207,1.680)--(6.219,1.680)--(6.231,1.680)--(6.244,1.680)%
  --(6.256,1.680)--(6.268,1.680)--(6.280,1.680)--(6.292,1.680)--(6.304,1.680)--(6.316,1.680)%
  --(6.328,1.680)--(6.341,1.680)--(6.353,1.680)--(6.365,1.680)--(6.377,1.680)--(6.389,1.680)%
  --(6.401,1.680)--(6.413,1.680)--(6.425,1.680)--(6.438,1.680)--(6.450,1.680)--(6.462,1.680)%
  --(6.474,1.680)--(6.486,1.680)--(6.498,1.680)--(6.510,1.680)--(6.522,1.680)--(6.535,1.680)%
  --(6.547,1.680)--(6.559,1.680)--(6.571,1.680)--(6.583,1.680)--(6.595,1.680)--(6.607,1.680)%
  --(6.619,1.680)--(6.632,1.680)--(6.644,1.680)--(6.656,1.680)--(6.668,1.680)--(6.680,1.680)%
  --(6.692,1.680)--(6.704,1.680)--(6.717,1.680)--(6.729,1.680)--(6.741,1.680)--(6.753,1.680)%
  --(6.765,1.680)--(6.777,1.680)--(6.789,1.680)--(6.801,1.680)--(6.814,1.680)--(6.826,1.680)%
  --(6.838,1.680)--(6.850,1.680)--(6.862,1.680)--(6.874,1.680)--(6.886,1.680)--(6.898,1.680)%
  --(6.911,1.680)--(6.923,1.680)--(6.935,1.680)--(6.947,1.680)--(6.959,1.680)--(6.971,1.680)%
  --(6.983,1.680)--(6.995,1.680)--(7.008,1.680)--(7.020,1.680)--(7.032,1.680)--(7.044,1.680)%
  --(7.056,1.680)--(7.068,1.680)--(7.080,1.680)--(7.092,1.680)--(7.105,1.680)--(7.117,1.680)%
  --(7.129,1.680)--(7.141,1.680)--(7.153,1.680)--(7.165,1.680)--(7.177,1.680)--(7.189,1.680)%
  --(7.202,1.680)--(7.214,1.680)--(7.226,1.680)--(7.238,1.680)--(7.250,1.680)--(7.262,1.680)%
  --(7.274,1.680)--(7.286,1.680)--(7.299,1.680)--(7.311,1.680)--(7.323,1.680)--(7.335,1.680)%
  --(7.347,1.680)--(7.359,1.680)--(7.371,1.680)--(7.384,1.680)--(7.396,1.680)--(7.408,1.680)%
  --(7.420,1.680)--(7.432,1.680)--(7.444,1.680)--(7.456,1.680)--(7.468,1.680)--(7.481,1.680)%
  --(7.493,1.680)--(7.505,1.680)--(7.517,1.680)--(7.529,1.680)--(7.541,1.680)--(7.553,1.680)%
  --(7.565,1.680)--(7.578,1.680)--(7.590,1.680)--(7.602,1.680)--(7.614,1.680)--(7.626,1.680)%
  --(7.638,1.680)--(7.650,1.680)--(7.662,1.680)--(7.675,1.680)--(7.687,1.680)--(7.699,1.680)%
  --(7.711,1.680)--(7.723,1.680)--(7.735,1.680)--(7.747,1.680)--(7.759,1.680)--(7.772,1.680)%
  --(7.784,1.680)--(7.796,1.680)--(7.808,1.680)--(7.820,1.680)--(7.832,1.680)--(7.844,1.680)%
  --(7.856,1.680)--(7.869,1.680)--(7.881,1.680)--(7.893,1.680)--(7.905,1.680)--(7.917,1.680)%
  --(7.929,1.680)--(7.941,1.680)--(7.953,1.680)--(7.966,1.680)--(7.978,1.680)--(7.990,1.680)%
  --(8.002,1.680)--(8.014,1.680)--(8.026,1.680)--(8.038,1.680)--(8.050,1.680)--(8.063,1.680)%
  --(8.075,1.680)--(8.087,1.680)--(8.099,1.680)--(8.111,1.680)--(8.123,1.680)--(8.135,1.680)%
  --(8.148,1.680)--(8.160,1.680)--(8.172,1.680)--(8.184,1.680)--(8.196,1.680)--(8.208,1.680)%
  --(8.220,1.680)--(8.232,1.680)--(8.245,1.680)--(8.257,1.680)--(8.269,1.680)--(8.281,1.680)%
  --(8.293,1.680)--(8.305,1.680)--(8.317,1.680)--(8.329,1.680)--(8.342,1.680)--(8.354,1.680)%
  --(8.366,1.680)--(8.378,1.680)--(8.390,1.680)--(8.402,1.680)--(8.414,1.680)--(8.426,1.680)%
  --(8.439,1.680)--(8.451,1.680)--(8.463,1.680)--(8.475,1.680)--(8.487,1.680)--(8.499,1.680)%
  --(8.511,1.680)--(8.523,1.680)--(8.536,1.680)--(8.548,1.680)--(8.560,1.680)--(8.572,1.680)%
  --(8.584,1.680)--(8.596,1.680)--(8.608,1.680)--(8.620,1.680)--(8.633,1.680)--(8.645,1.680)%
  --(8.657,1.680)--(8.669,1.680)--(8.681,1.680)--(8.693,1.680)--(8.705,1.680)--(8.717,1.680)%
  --(8.730,1.680)--(8.742,1.680)--(8.754,1.680)--(8.766,1.680)--(8.778,1.680)--(8.790,1.680)%
  --(8.802,1.680)--(8.814,1.680)--(8.827,1.680)--(8.839,1.680)--(8.851,1.680)--(8.863,1.680)%
  --(8.875,1.680)--(8.887,1.680)--(8.899,1.680)--(8.912,1.680)--(8.924,1.680)--(8.936,1.680)%
  --(8.948,1.680)--(8.960,1.680)--(8.972,1.680)--(8.984,1.680)--(8.996,1.680)--(9.009,1.680)%
  --(9.021,1.680)--(9.033,1.680)--(9.045,1.680)--(9.057,1.680)--(9.069,1.680)--(9.081,1.680)%
  --(9.093,1.680)--(9.106,1.680)--(9.118,1.680)--(9.130,1.680)--(9.142,1.680)--(9.154,1.680)%
  --(9.166,1.680)--(9.178,1.680)--(9.190,1.680)--(9.203,1.680)--(9.215,1.680)--(9.227,1.680)%
  --(9.239,1.680)--(9.251,1.680)--(9.263,1.680)--(9.275,1.680)--(9.287,1.680)--(9.300,1.680)%
  --(9.312,1.680)--(9.324,1.680)--(9.336,1.680)--(9.348,1.680)--(9.360,1.680)--(9.372,1.680)%
  --(9.384,1.680)--(9.397,1.680)--(9.409,1.680)--(9.421,1.680)--(9.433,1.680)--(9.445,1.680)%
  --(9.457,1.680)--(9.469,1.680)--(9.481,1.680)--(9.494,1.680)--(9.506,1.680)--(9.518,1.680)%
  --(9.530,1.680)--(9.542,1.680)--(9.554,1.680)--(9.566,1.680)--(9.578,1.680)--(9.591,1.680)%
  --(9.603,1.680)--(9.615,1.680)--(9.627,1.680)--(9.639,1.680)--(9.651,1.680)--(9.663,1.680)%
  --(9.676,1.680)--(9.688,1.680)--(9.700,1.680)--(9.712,1.680)--(9.724,1.680)--(9.736,1.680)%
  --(9.748,1.680)--(9.760,1.680)--(9.773,1.680)--(9.785,1.680)--(9.797,1.680)--(9.809,1.680)%
  --(9.821,1.680)--(9.833,1.680)--(9.845,1.680)--(9.857,1.680)--(9.870,1.680)--(9.882,1.680)%
  --(9.894,1.680)--(9.906,1.680)--(9.918,1.680)--(9.930,1.680)--(9.942,1.680)--(9.954,1.680)%
  --(9.967,1.680)--(9.979,1.680)--(9.991,1.680)--(10.003,1.680)--(10.015,1.680)--(10.027,1.680)%
  --(10.039,1.680)--(10.051,1.680)--(10.064,1.680)--(10.076,1.680)--(10.088,1.680)--(10.100,1.680)%
  --(10.112,1.680)--(10.124,1.680)--(10.136,1.680)--(10.148,1.680)--(10.161,1.680)--(10.173,1.680)%
  --(10.185,1.680)--(10.197,1.680)--(10.209,1.680)--(10.221,1.680)--(10.233,1.680)--(10.245,1.680)%
  --(10.258,1.680)--(10.270,1.680)--(10.282,1.680)--(10.294,1.680)--(10.306,1.680)--(10.318,1.680)%
  --(10.330,1.680)--(10.342,1.680)--(10.355,1.680)--(10.367,1.680)--(10.379,1.680)--(10.391,1.680)%
  --(10.403,1.680)--(10.415,1.680)--(10.427,1.680)--(10.440,1.680)--(10.452,1.680)--(10.464,1.680)%
  --(10.476,1.680)--(10.488,1.680)--(10.500,1.680)--(10.512,1.680)--(10.524,1.680)--(10.537,1.680)%
  --(10.549,1.680)--(10.561,1.680)--(10.573,1.680)--(10.585,1.680)--(10.597,1.680)--(10.609,1.680)%
  --(10.621,1.680)--(10.634,1.680)--(10.646,1.680)--(10.658,1.680)--(10.670,1.680)--(10.682,1.680)%
  --(10.694,1.680)--(10.706,1.680)--(10.718,1.680)--(10.731,1.680)--(10.743,1.680)--(10.755,1.680)%
  --(10.767,1.680)--(10.779,1.680)--(10.791,1.680)--(10.803,1.680)--(10.815,1.680)--(10.828,1.680)%
  --(10.840,1.680)--(10.852,1.680)--(10.864,1.680)--(10.876,1.680)--(10.888,1.680)--(10.900,1.680)%
  --(10.912,1.680)--(10.925,1.680)--(10.937,1.680)--(10.949,1.680)--(10.961,1.680)--(10.973,1.680)%
  --(10.985,1.680)--(10.997,1.680)--(11.009,1.680)--(11.022,1.680)--(11.034,1.680)--(11.046,1.680)%
  --(11.058,1.680)--(11.070,1.680)--(11.082,1.680)--(11.094,1.680)--(11.106,1.680)--(11.119,1.680)%
  --(11.131,1.680)--(11.143,1.680)--(11.155,1.680)--(11.167,1.680)--(11.179,1.680)--(11.191,1.680)%
  --(11.204,1.680)--(11.216,1.680)--(11.228,1.680)--(11.240,1.680)--(11.252,1.680)--(11.264,1.680)%
  --(11.276,1.680)--(11.288,1.680)--(11.301,1.680)--(11.313,1.680)--(11.325,1.680)--(11.337,1.680)%
  --(11.349,1.680)--(11.361,1.680)--(11.373,1.680)--(11.385,1.680)--(11.398,1.680)--(11.410,1.680)%
  --(11.422,1.680)--(11.434,1.680)--(11.446,1.680)--(11.458,1.680)--(11.470,1.680)--(11.482,1.680)%
  --(11.495,1.680)--(11.507,1.680)--(11.519,1.680)--(11.531,1.680)--(11.543,1.680)--(11.555,1.680)%
  --(11.567,1.680)--(11.579,1.680)--(11.592,1.680)--(11.604,1.680)--(11.616,1.680)--(11.628,1.680)%
  --(11.640,1.680)--(11.652,1.680)--(11.664,1.680)--(11.676,1.680)--(11.689,1.680)--(11.701,1.680)%
  --(11.713,1.680)--(11.725,1.680)--(11.737,1.680)--(11.749,1.680)--(11.761,1.680)--(11.773,1.680)%
  --(11.786,1.680)--(11.798,1.680)--(11.810,1.680)--(11.822,1.680)--(11.834,1.680)--(11.846,1.680)%
  --(11.858,1.680)--(11.870,1.680)--(11.883,1.680)--(11.895,1.680)--(11.907,1.680)--(11.919,1.680)%
  --(11.931,1.680)--(11.943,1.680)--(11.955,1.680)--(11.968,1.680)--(11.980,1.680)--(11.992,1.680)%
  --(12.004,1.680)--(12.016,1.680)--(12.028,1.680)--(12.040,1.680)--(12.052,1.680)--(12.065,1.680)%
  --(12.077,1.680)--(12.089,1.680)--(12.101,1.680)--(12.113,1.680)--(12.125,1.680)--(12.137,1.680)%
  --(12.149,1.680)--(12.162,1.680)--(12.174,1.680)--(12.186,1.680)--(12.198,1.680)--(12.210,1.680)%
  --(12.222,1.680)--(12.234,1.680)--(12.246,1.680)--(12.259,1.680)--(12.271,1.680)--(12.283,1.680)%
  --(12.295,1.680)--(12.307,1.680)--(12.319,1.680)--(12.331,1.680)--(12.343,1.680)--(12.356,1.680)%
  --(12.368,1.680)--(12.380,1.680)--(12.392,1.680)--(12.404,1.680)--(12.416,1.680)--(12.428,1.680)%
  --(12.440,1.680)--(12.453,1.680)--(12.465,1.680)--(12.477,1.680)--(12.489,1.680)--(12.501,1.680)%
  --(12.513,1.680)--(12.525,1.680)--(12.537,1.680)--(12.550,1.680)--(12.562,1.680)--(12.574,1.680)%
  --(12.586,1.680)--(12.598,1.680)--(12.610,1.680)--(12.622,1.680)--(12.634,1.680)--(12.647,1.680)%
  --(12.659,1.680)--(12.671,1.680)--(12.683,1.680)--(12.695,1.680)--(12.707,1.680)--(12.719,1.680)%
  --(12.732,1.680)--(12.744,1.680)--(12.756,1.680)--(12.768,1.680)--(12.780,1.680)--(12.792,1.680)%
  --(12.804,1.680)--(12.816,1.680)--(12.829,1.680)--(12.841,1.680)--(12.853,1.680)--(12.865,1.680)%
  --(12.877,1.680)--(12.889,1.680)--(12.901,1.680)--(12.913,1.680)--(12.926,1.680)--(12.938,1.680)%
  --(12.950,1.680)--(12.962,1.680)--(12.974,1.680)--(12.986,1.680)--(12.998,1.680)--(13.010,1.680)%
  --(13.023,1.680)--(13.035,1.680)--(13.047,1.680)--(13.059,1.680)--(13.071,1.680)--(13.083,1.680)%
  --(13.095,1.680)--(13.107,1.680)--(13.120,1.680)--(13.132,1.680)--(13.144,1.680)--(13.156,1.680)%
  --(13.168,1.680)--(13.180,1.680)--(13.192,1.680)--(13.204,1.680)--(13.217,1.680)--(13.229,1.680)%
  --(13.241,1.680)--(13.253,1.680)--(13.265,1.680)--(13.277,1.680)--(13.289,1.680)--(13.301,1.680)%
  --(13.314,1.680)--(13.326,1.680)--(13.338,1.680)--(13.350,1.680)--(13.362,1.680)--(13.374,1.680)%
  --(13.386,1.680)--(13.398,1.680)--(13.411,1.680)--(13.423,1.680)--(13.435,1.680)--(13.447,1.680);
\gpcolor{color=gp lt color border}
\draw[gp path] (1.320,8.631)--(1.320,0.985)--(13.447,0.985)--(13.447,8.631)--cycle;
%% coordinates of the plot area
\gpdefrectangularnode{gp plot 1}{\pgfpoint{1.320cm}{0.985cm}}{\pgfpoint{13.447cm}{8.631cm}}
\end{tikzpicture}
%% gnuplot variables

%	\caption{Gráfico da relação~\eqref{Eq:Rel_pot_quim_renorm_mom_fermi} para $m = 100$.}
%\end{figure*}
%Caso $\mu_R^2 > m^2$, temos uma relação simples para $\mu_R$:
%\begin{equation}
%	\mu_R = \sqrt{p_F^2 + m^2}, \quad \textrm{se}~ \mu_R^2 > m^2.
%\end{equation}
%
%No entanto, para qualquer valor de $\mu_R^2$ menor que $m^2$, não é possível inverter a relação~\eqref{Eq:Rel_pot_quim_renorm_mom_fermi}: todo valor de $\mu_R$ tal que $\mu_R^2 < m^2$ está associado a $p_F = 0$, e não é possível determinar uma inversa pois o elemento $0$ dos valores de $p_F$ levaria a diversos elementos dos valores de $\mu_R$ e, portanto, a relação inversa para $\mu_R^2 < m^2$ não é uma função em tal intervalo.\footnote{Uma interpretação para isso é a de que existe um valor mínimo $\mu_R = m$ para o potencial químico renormalizado.}

%Por outro lado, utilizando as Equações~\eqref{Eq:Rel_Dens_Mom_Fermi_NJL} e~\eqref{Eq:Rel_pot_quim_renorm_mom_fermi} podemos escrever
%\begin{equation}
%	\sqrt[3]{\frac{3\pi^2}{n_f} \rho_B} = \sqrt{\mu_R^2 - m^2}\theta(\mu_R^2 - m^2),
%\end{equation}
%
%onde $\theta(\mu_R^2 - m^2)$ garante que o lado direito seja estritamente positivo, ou seja
%\begin{equation}
%	\rho_B > 0,
%\end{equation}
%
%o que é perfeitamente razoável. Consequentemente temos que a expressão para $\mu_R$ é dada por
%\begin{equation}
%	\mu_R = \sqrt{p_F^2 + m^2}.
%\end{equation}
%
%Isto significa que, se rodarmos em $\mu_R$, temos que $p_F$ é zero para $\mu_R^2 < m^2$, o que resulta no mesmo valor para $m$ (resolvendo a Eq.~\eqref{Eq:Eq_Gap_Buballa_1}) para qualquer valor de $\mu_R$ em tal intervalo. Se acontecer o mesmo com o potencial termodinâmico\footnote{Ver se é esse o caso.} $\tilde{\omega}$, não tem problema algum, pois estaríamos calculando ``o mesmo ponto'' para as equações de estado.
%

%%%%%%%%%%%%%%%%%%%%%%%%%%%%%%%%%%%%%%%%%%%%%%%%%%%%%%%%%%%%%%%%%%%%
\subsection{Determinação de $\tilde\omega$ para o caso $G_V \neq 0$}
%%%%%%%%%%%%%%%%%%%%%%%%%%%%%%%%%%%%%%%%%%%%%%%%%%%%%%%%%%%%%%%%%%%%

Para o caso $G_V \neq 0$, adicionamos o termo 
\begin{equation}
	(\mu-\mu_R)^2 / 4G_V
\end{equation}
%
à Equação~\eqref{Eq:Pot_term_MF_sem_G_V}, o que resulta em 
\begin{equation}\label{Eq:Pot_termodinamico_1_G_V_neq_0}
\begin{split}
\omega_{\rm{MF}} =&~ - \frac{n_f n_c}{\pi^2} \left[[F_E(m,\Lambda) - F_E(m, p_F)] + \mu_R \frac{p_F^3}{3} \right] \\
&+ \frac{(m-m_0)^2}{4G_S} - \frac{(\mu - \mu_R)^2}{4G_V}.
\end{split}
\end{equation}
%
onde $\mu$ pode ser calculado invertendo a relação~\eqref{Eq:Eq_Gap_Pot_Quim_Renorm}, obtendo-se
\begin{equation}
	\mu = \mu_R + \frac{2G_V N_fN_c}{3\pi^2} \cdot \frac{1}{(\hbar c)^2} \cdot p_F^3.
\end{equation}
%
O potencial termodinâmico no vácuo ($\omega_{\rm{MF}}(0; m_{\rm{vac}}, 0)$) não é afetado pelo termo em $G_V$, o que pode ser visto da Equação~\eqref{Eq:Pot_termodinamico_1_G_V_neq_0} ao se tomar $\mu = \mu_R = 0$.

%%%%%%%%%%%%%%%%%%%%%%%%%%%%%%%%%%%%%%%%%%%%%%%%%%%%%%%%%
\section{Fase de Hádrons: NJL, SU(2), temperatura finita}
%%%%%%%%%%%%%%%%%%%%%%%%%%%%%%%%%%%%%%%%%%%%%%%%%%%%%%%%%

De uma forma geral, podemos escolher variar $\rho_B$ ou $\mu$ ao resolver o problema numericamente. Basicamente o que temos é uma solução de um sistema para um valor específico de densidade ou de potencial químico (não estamos calculando uma função da densidade, por exemplo, mas sim resolvendo o sistema para um valor específico de densidade). Como $\mu$ e $\rho$ estão ligados através das equações
\begin{subequations}
\begin{equation}\label{Eqs:sist_m_mu_t_finito}
	\mu_R = \mu - 4 n_f n_c G_V \int\frac{d^3}{(2\pi)^3}(n_p(T, \mu_R) - \bar{n}_p(T, \mu_R)
\end{equation}
%
e
\begin{equation}
	\rho_B = \frac{n(T,\mu_R)}{3} = 2 n_f n_c \int\frac{d^3}{(2\pi)^3}(n_p(T, \mu_R) - \bar{n}_p(T, \mu_R),
\end{equation}
\end{subequations}
%
já que estão ligados individualmente a $\mu_R$, podemos escolher um dos dois como variável e calcular o outro\footnote{Isso assume que ambos estão ligados de maneira única. Em Buballa (2005) -- p. 235, após a Eq. 2.56 -- diz que a função é estritamente crescente, o que garante isso.}. No cálculo do potencial termodinâmico, uma das variáveis pode ser então eliminada. Em geral se escolhe $\mu$ para ser variado, pelo que tenho percebido\footnote{Mas como comecei variando $\rho_B$, vou insistir. De qualquer forma, pelo menos para $T = 0$ é mais fácil.}.

%%%%%%%%%%%%%%%%%%%%%%%%%%%%%%%
\section{Fase de hádrons: eNJL} 
%%%%%%%%%%%%%%%%%%%%%%%%%%%%%%%

A partir da lagrangiana \eqref{Eq:Lagrangiana_eNLJ_Pais}, é possível determinar  o potencial termodinâmico\footnote{Potencial Grand-canônico, ou potencial de Landau.} por unidade de volume, dado por (em Tsue~\cite{japoneses}, o potencial é \emph{definido} como $\omega = \langle\langle \mathcal{H}^{MF}\rangle\rangle - \mu\langle\langle\mathcal{N}\rangle\rangle - \frac{1}{\beta}\langle\langle\mathcal{S}\rangle\rangle$\footnote{Os termos $\langle\langle \cdot \rangle\rangle$ são a Hamiltoniana de campo médio, a densidade bariônica, e algo que pode ser a entropia (eu acho). Isso tem que sair de alguma coisa da termodinâmica, tipo $\omega (T, m) = (T/V) \ln Z$, como poderia ser \emph{definido}?}, o que se traduz em uma expressão similar a essa que segue)
\begin{equation}\label{Eq:potencial_termodinamico}
\begin{split}
	\omega(\mu) =&~ \varepsilon_{\rm{kin}} + m\rho_s - G_s\rho_s^2 + G_v\rho^2 + G_{sv}\rho_s^2\rho^2 + G_\rho\rho_3^2 \\
	&+ G_{v\rho}\rho^2\rho_3^2 + G_{s\rho}\rho_s^2\rho_3^2 - \mu_p\rho_p - \mu_n\rho_n,
\end{split}
\end{equation}
%
onde
\begin{itemize}
	\item $\rho$ é a densidade bariônica, dada pela soma das densidades de nêutron e próton\footnote[][-1cm]{São densidades numéricas de partículas, ou seja, representam o número de partículas por unidade de volume.}:
	\begin{equation}
		\rho = \rho_p + \rho_n.
	\end{equation}

	\item As densidades bariônicas de próton e nêutron são dadas por (alguma indicação de onde isso sai é dada em Tsue~\cite{japoneses} (resumidamente: QFT)):\footnote{Explicar, explicar o momento de Fermi.}
	\begin{equation}
		\rho_i = \int_0^{k_F^i}\frac{dp}{\pi^2}p^2; \qquad i = p,n; \quad k_F^i = \textrm{momento de Fermi},
	\end{equation}
	%
	ou, caso $\rho_i$ sejam conhecidos
	\begin{equation}\label{Eq:Mom_Fermi_a_partir_de_rho}
		p_F^i = \sqrt[3]{3\pi^2\rho_i}.
	\end{equation}
	
	\item $\mu_p$ e $\mu_n$ representam os potenciais químicos de próton e nêutron, respectivamente.
\end{itemize}

O termo cinético na expressão acima pode ser calculado através de (primeiro termo da Eq. (1) em \cite{PRC_68_035804_2003}, o resto é energia potencial; Indicação de onde isso sai é dada em Tsue~\cite{japoneses}.)\footnote{Degenerescência: O 2 se refere às duas possibilidades de spin; Podemos ter um $N_f$ que representa o número de sabores.}
\begin{align}
	\varepsilon_{\rm{kin}} &= \langle\bar{\psi}(\vec{\gamma}\cdot\vec{p})\psi\rangle \\
	&= 2 N_c\sum_i \int \frac{d^3p}{(2\pi)^3}\frac{p^2 + m_i M_i}{E_i}(n_{i-}-n_{i+})\theta(\Lambda^2 - p^2),
\end{align}
%
onde
\begin{itemize}
	\item A soma se dá sobre as espécies de partículas;
	\item $N_c$ representa o número de cores\footnote{No nosso caso, 1?};
	\item $\theta$ é a função degrau, $\Lambda$ é o \emph{cutoff};
	\item $n_{i\pm}$ são as funções de distribuição de Fermi para estados de energia positiva e negativa (respectivamente), dados por
	\begin{equation}
		n_{i\pm} = \frac{1}{1 + \exp(\pm[\beta(E_i\mp\mu_i)])}
	\end{equation}
	%
	onde $i = p, n$ (no nosso caso, no artigo é $u, d, s$) e $\beta = T^{-1}$
	\item $M_i$ é a massa efetiva do nucleon em questão (quark, no artigo).
	\item $E_i = \sqrt{p^2 + M_i^2}$
	\item $m_i$ são as massas constituintes -- nuas, \emph{bare} --.
\end{itemize}

Se tomarmos $T \to 0$, temos que $n_{i-} \to 1$ e $n_{i+} \to 0$\footnote{Depende do sinal de $E - \mu$. Aparementemente isso dá $\theta(\mu-E)$}; Além disso, se o integrando só depende do módulo de $\vec{p}$, então (Glendenning\cite{Glendenning}, p. 92)
\begin{equation}\label{Eq:Int_d3p_to_dp}
	\int\frac{d^3p}{(2\pi)^3} \to \frac{1}{2\pi^2}\int p^2dp.
\end{equation}
%
Logo, temos
\begin{align}
	\varepsilon &= 2 N_c \frac{1}{2\pi^2}\sum_i \int p^2 dp \frac{p^2 + m_i M_i}{\sqrt{p^2 + M_i^2}} \theta(\Lambda^2 - p^2) \\
	&= \frac{N_c}{\pi^2}\sum_i\left[\int \frac{p^4dp}{\sqrt{p^2 + M_i^2}}\theta(\Lambda^2 - p^2) + \int m_i M_i \frac{p^2 dp}{\sqrt{p^2 + M_i^2}}\theta(\Lambda^2 - p^2)\right] \label{Eq:Engergia_cin_separada}
\end{align}
%
Podemos utilizar as relações (Glendenning\cite{Glendenning} p. 94\footnote{Na Ref. o primeiro termo da segunda expressão aparece sem o $k$ multiplicando, o que dimensionalmente está incorreto.})
\begin{align}
	\int \frac{k^4}{\sqrt{k^2 + m^2}} dk &= \frac{1}{4}\left[k^3\epsilon - \frac{3}{2} m^2k\epsilon + \frac{3}{2}m^4\ln\frac{\epsilon + k}{m} \right]\\
	\int \frac{k^2}{\sqrt{k^2 + m^2}} dk &= \frac{1}{2}\left[k\epsilon - m^2\ln\frac{\epsilon + k}{m}\right] \label{Eq:Integ_momento_quad}
\end{align}
%
onde $\epsilon = \sqrt{k^2+m^2}$. Tomando o caso $m_i \to 0$\footnote{No prog. \texttt{eos\_enjl1-dens-assym- clean-rho-vr.f}: $\varepsilon \propto [F_2(M, k_F^i) - F_2(M, \Lambda)]$ ao invés de $\varepsilon \propto [F_2(M_i, \Lambda) - F_2(M_i, 0)]$; Isso se deve à retirada da contribuição do vácuo. Vendo Lee \etal talvez seja o seguinte: $n_i^+$ não vai a zero, mas a uma função degrau que envolve o momento, de forma que entre zero e $p_F$ o resultado seja nulo. A mudança na ordem dos limites pode explicar algum sinal.}, obtemos
\begin{equation}\label{Eq:Energia_kin}
	\varepsilon_{\rm{kin}} = \frac{N_c}{\pi^2}\sum_i \Big[\underbrace{\frac{1}{8}\Big((2p^3 - 3M_i^2p)\sqrt{p^2 + M_i^2} + 3M_i^4\ln\frac{p + \sqrt{p^2 + M_i^2}}{M_i}\Big)}_{F_2(m,p)}\Big]_0^\Lambda
\end{equation}

A densidade escalar $\rho_s$ é dada por (a origem é indicada em Tsue~\cite{japoneses})
\begin{equation}\label{Eq:Dens_Escalar}
	\rho_s^i = \frac{M}{\pi^2}[F_0(M, p_F^i) - F_0(M, \Lambda)], \quad i = p, n,
\end{equation}
%
onde
\begin{equation}\label{Eq:Def_F0}
	F_0(M, x) = \int_0^x dp\frac{p^2}{\sqrt{M^2 + p^2}} dp.
\end{equation}
%
Utilizando a Equação~\eqref{Eq:Integ_momento_quad}, podemos reescrever a equação acima como
\begin{equation}\label{Eq:Def_F0_integrado}
	F_0(M, x) = \frac{1}{2}\left[x\sqrt{x^2+M^2} - M^2 \ln \frac{x + \sqrt{x^2+M^2}}{M}\right].
\end{equation}

A massa efetiva $M$ na equação acima é dada por\footnote{Essa equação é conhecida como \emph{gap equation}}
\begin{equation}\label{Eq:Gap}
	M = m - 2G_s\rho_s + 2G_{sv}\rho_s\rho^2 + 2 G_{s\rho}\rho_s\rho_3^2,
\end{equation}
%
com $\rho_s = \rho_s^p + \rho_s^n$. Temos, portanto, uma interdependência entre as equações. Para que seja possível solucionar tais equações, podemos definir uma função $f(M)$ de tal forma que
\begin{equation}\label{Eq:Gap_zero}
	f(M) = M - m + 2G_s\rho_s - 2G_{sv}\rho_s\rho^2 - 2 G_{s\rho}\rho_s\rho_3^2,.
\end{equation}
%
Para solucionarmos a equação acima, basta utilizarmos uma rotina para encontrar zeros de funções, por exemplo biseção ou Newton-Raphson, encontrando o valor de $M$ para o qual $f(M) = 0$. A densidade escalar $\rho_s$ pode ser calculada através da expressão~\eqref{Eq:Dens_Escalar}.

Os potenciais químicos são dados por\footnote{Como essas expressões são calculadas?}
\begin{equation}\label{Eq:Potenciais_Quimicos}
\begin{split}
	\mu_i =&~ E_{p_F}^i + 2G_v\rho + 2G_{sv}\rho\rho_s^2 \pm 2G_\rho\rho_3+2G_{v\rho}\rho_3^2\rho \\
	& \pm 2G_{v\rho}\rho^2\rho_3 \pm 2 G_{s\rho}\rho_3\rho_s^2,
\end{split}
\end{equation}
%
onde $i = p,n$, os sinais superiores se referem ao caso de prótons, e $E_{p_F}^i = \sqrt{M^2 + (p_F^i)^2}$.

As equações de estado para pressão $P$ e densidade de energia $\varepsilon$ são dadas por\footnote{Como são calculadas?}
\begin{align}
	P &= -\omega(\mu) + \epsilon_0 \label{Eq:Pressao}\\
	\varepsilon &= -P + \mu_p\rho_p + \mu_n\rho_n. \label{Eq:Densidade_energia}
\end{align}
%%%%%%%%%%%%%%%%%%%%%%%%%%%%%%%%%%%
%%%%%%%%%%%%%%%%%%%%%%%%%%%%%%%%%%%
%%%%%%%%%%%%%%%%%%%%%%%%%%%%%%%%%%%

%%%%%%%%%%%%%%%%%%
\section{Apêndice}
%%%%%%%%%%%%%%%%%%

Colocar expressões usadas no capítulo aqui, pra não poluir o texto.

trazer: Eq.~\ref{Eq:Int_d3p_to_dp}, \eqref{Eq:Integ_momento_quad}, \eqref{Eq:Def_F0_integrado}

Com $E = \sqrt{k^2 + m^2}$, 
\begin{align} \label{Eq:Def_F_E}
	\int k^2 \sqrt{k^2 + m^2} \;dk &= \frac{1}{4}\left[k E^3 - \frac{1}{2} m^2 k E - \frac{1}{2} m^4\ln\left(\frac{k+E}{m}\right)\right] \\
	&\equiv F_E(m, k),
\end{align}

Fórmula de Leibniz:\footnote{Se os limites são constantes, o lado direito se resume à integral.}
\begin{equation}\label{Eq:Form_Leibniz}
\begin{split}
	\frac{d}{dx} \left(\int_{a(x)}^{b(x)} f(x,t) dt\right) =&~ f(x, b(x))b'(x) - f(x, a(x))a'(x) \\
	&+ \int_{a(x)}^{b(x)}\frac{\partial}{\partial x}f(x,t) dt
\end{split}
\end{equation}

%%%%%%%%%%%%%%%%%%%%%%%%
\section{Próximas ações}
%%%%%%%%%%%%%%%%%%%%%%%%

\begin{itemize}
	\item Organizar o capítulo:
		\begin{itemize}
			\item Verificar se há detalhes de implementação para mover;
			\item Mover integrais para o capítulo de implementação;
			\item Rever a sequência, pois tem coisa do eNJL e do NJL; Por do NJL antes
		\end{itemize}
	\item Reproduzir cálculos de $p$ e $\varepsilon$ para NJL:
		\begin{itemize}
			\item Tentar seguir o que a Débora fez (solução pelo tensor);
				\begin{itemize}
					\item Calcular o tensor;
					\item Calcular $\mean{T_{00}}$
					\item Calcular $\mean{T_{ii}}$
					\item Calcular dens. energia, pressão
					\item Determinar a eq do gap, potencial químico renormalizado
					\item Det. $\mu$
					\item Como adicionar $T$?
					\item Por que precisamos calcular $\varepsilon_0$ e diminuir? (fazemos isso em $p$ tb?)
				\end{itemize}
			\item Verificar o cálculo dos termos $\bra{q}\alpha\cdot p\ket{q}$ e $\bra{q}\beta m^*\ket{q}$ no Walecka;
				\begin{itemize}
					\item Rep. cálculo de $\mean{\bar\psi\gamma_0 k_0\psi}$
					\item Rep. cálculo de $\mean{\bar\psi\gamma_i k_i\psi}$
					\item Calcular $\mean{\mathcal{L}}$
					\item Calcular $\varepsilon$, $p$
					\item Como adicionar $T$?
					\item Precisa calcular $\varepsilon_0$ e diminuir?
					\item Determinar a eq do gap, potencial químico renormalizado
				\end{itemize}
			\item Ver o cálculo a partir do potencial termodinâmico (ver Tese do James, tem a parte cinética pelo menos);
				\begin{itemize}
					\item[\checkmark] Alterar notas para refletir interpretação correta do cálculo de $\tilde{\omega}$;
					\item[\checkmark] Ler Asakawa \& Yazaki, Nuclear Physics A504 1989 668-684;
						\begin{itemize}
							\item ``é possível se mostrar que'' a solução auto-consistente (gap, $\mu_r$, $\rho_B$, $\rho_s$) dá extremos de $\omega$; devemos escolher a solução auto-consistente que minimiza $\omega$. Atualmente não estou minimizando nada pois nos casos que tratei só há uma solução.
						\end{itemize}
					\item Determinar Hamiltoniana;
					\item Determinar função partição;
					\item Determinar potencial termodinâmico;
					\item Determinar dens. de energia, pressão;
					\item Determinar a eq do gap, potencial químico renormalizado (fiz um \emph{quote} do Buballa, mas não sei exatamente o que é aquilo, de onde vem aquele cálculo.
				\end{itemize}
		\end{itemize}
	\item Dúvidas:
		\begin{itemize}
			\item Em Buballa\cite{Buballa1996}, $\mu$ é um ``parâmetro externo''. Que parâmetro é esse? Como ele é determinado? (Achei que fosse a partir do potencial termodinâmico, mas o potencial termodinâmico é calculado a partir dele! Só tem equação para o potencial químico efetivo $\mu_R$.)\footnote{Parece ser um parâmetro que vem dos dados experimentais de decaimento e \emph{branching ratio} do píon.}
		\end{itemize}
	\item Cálculo para $T = 0$
		\begin{itemize}
			\item deixar as integrais aqui, realizar as integrações angulares e explicitar os limites de zero ao \emph{cutoff};
			\item limpar itens que estejam repetidos por termos invertido a ordem para $T \neq 0$, $T = 0$.
		\end{itemize}
	\item Cálculo para $T \neq 0$
		\begin{itemize}
			\item digitar cálculo de entropia;
			\item Estou calculando como se todas as integrais precisassem ser regularizadas via cutoff. Não sei se isso é verdade.
		\end{itemize}
\end{itemize}
