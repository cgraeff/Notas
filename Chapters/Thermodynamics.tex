%%%%%%%%%%%%%%%%%%%%%%%
\chapter{Termodinâmica}
%%%%%%%%%%%%%%%%%%%%%%%

\begin{fullwidth}\it
Neste capítulo deduzimos as equações de estado a partir das densidades lagrangianas apresentadas no capítulo anterior.
\end{fullwidth}

%%%%%%%%%%%%%%%%%%%%
\section{Introdução}
%%%%%%%%%%%%%%%%%%%%

Como estamos interessados na obtenção da linha de coexistência de fases no diagrama de fases da QCD segundo o modelo NJL+eNJL e vamos utilizar as condições de Gibbs --~isto é, assumimos que na coexistência de fases temos igualdade de $\varepsilon$, $p$, $\mu$ em ambas as fases~--, precisamos determinar as expressões para essas grandezas.

A determinação pode ser feita de três maneiras:\footnote{Pelo que entendi, pelo menos.}
\begin{itemize}
	\item Através do tensor momento energia:\footnote{Nesses dois primeiros casos como se determinam $\mu$ e $M$? Como se adiciona a temperatura?}
		\begin{align}
			\varepsilon &= \mean{T_{00}} \\
			p &= \frac{1}{3} \mean{T_{ii}}
		\end{align}
	\item Através da Lagrangiana\cite{Glendenning}, que é um método similar ao anterior:
		\begin{align}
			\varepsilon &= -\mean{\mathcal{L}} + \mean{\bar{\psi}\gamma_0 k_0 \psi} \\
			p &= \frac{1}{3} \mean{\bar{\psi}\gamma_i k_i \psi}
		\end{align}
	\item Através da Termodinâmica:
		\begin{align}
			\omega(T, \mu) &= -\frac{T}{V}\ln \mathcal{Z} \\
			\mathcal{Z} &= \Tr\exp\left(-\frac{1}{T}\int d^3x (\mathcal{H} - \mu\mathcal{N})\right) \\
			p &= -\omega \\
			\varepsilon &= -p +Ts + \mu, 
		\end{align}
		onde $\tilde{\mu}$ e $M$ são determinados exigindo que $\partial \omega / \partial M = \partial \omega / \partial \tilde{\mu} = 0$.
\end{itemize}

A questão é que eu não sei fazer de nenhum dos três jeitos.

%%%%%%%%%%%%%%%%%%%%%%%%%%%%%
\section{Fase de quarks: NJL}
%%%%%%%%%%%%%%%%%%%%%%%%%%%%%

%%%%%%%%%%%%%%%%%%%%%%%%%%%%%%%%%%%%%%%%%%%%%%%%%%%%%%
\subsection{Solução através do tensor momento-energia}
%%%%%%%%%%%%%%%%%%%%%%%%%%%%%%%%%%%%%%%%%%%%%%%%%%%%%%

Pelo que entendi, nesse método precisamos:
\begin{itemize}
	\item Calcular o tensor energia-momento a partir da lagrangiana;
	\item Calcular os valores esperados $\mean{T_{00}}$ e $\mean{T{ii}}$. Para isso teremos que calcular uma expressão do tipo $\mean{\psi^\dagger \vec{\alpha}\cdot\vec{p} + \beta m^*\psi}$;
	\item Essa expressão deve se separar em duas partes para a energia: $\mean{\psi^\dagger\vec{\alpha}\cdot\vec{p}\psi}$ e $m^*\mean{\psi^\dagger\psi}$. A primeira dá a parte cinética, a segunda é $m^* \rho_s$ ($\beta \equiv \gamma_0$). Como $\mean{\bar{q}q} = -(M - m)/(2G)$ (Buballa report), temos $m = m_0 - 2G_S\rho_s$ e substituindo no lugar do $m$, obtemos $m_0\rho_s - 2G_S\rho_s^2$. Então acho que obtemos no final das contas $\varepsilon = \varepsilon_{kin} + m_0\rho_s - 2G_S\rho_s^2$.
\end{itemize}

Devido a invariância translacional da lagrangiana, é possível mostrar\footnote{Referência. Melhor, fazer num apêndice. Pensando melhor, não estou vendo nada de interessante nisso. Por que anotei isso? Preciso mostrar que $\varepsilon$ e $p$ podem ser calculadas através de $\mean{T_{00}}$ e $\mean{T_{ii}}$.} (através do teorema de Noether) que existe um tensor tal que:
\begin{align}
	T^{\mu\nu} &= \frac{\partial \mathcal{L}}{\partial(\partial_\mu \phi_i)}\partial^\nu\phi_i - \eta^{\mu\nu}\mathcal{L} \\
	\partial_\mu T^{\mu\nu} &= 0,
\end{align}
%
onde a segunda equação é uma equação de continuidade. Integrando em uma superfície de forma que não haja corrente a atravessando, temos quatro cargas conservadas.

Como o momento canônico é dado por 
\begin{equation}
	\pi(x) = \frac{\partial \mathcal{L}}{\partial(\partial_0 \phi)}
\end{equation}
%
e
\begin{align}
	\mathcal{H} &= \int_V d\vec{r} T^{00} \\
	P^k &= \int_V d\vec{r} T^{0k}; \quad k= 1,2,3
\end{align}

Portanto, dada a invariância por translação, temos que a energia e o momento linear se conservam. Podemos definir o invariante
\begin{align}
	P_\mu P^\mu &= E^2 - \vec{P}^2 = M^2, \\
	P^\mu &= (E, \vec{P}).
\end{align}
 
Valor esperado de um operador no estado fundamental de um sistema de muitos nucleons\footnote{Também fazer uma explicação melhorzinha e colocar em um apêndice.}. Dado um operador $\Gamma$, temos que o valor esperado no estado fundamental de muitas partículas pode ser calculado através de
\begin{equation}
	\mean{\bar\psi\Gamma\psi} = \sum_\kappa \int \frac{d\vec{k}}{(2\pi)^3} (\bar\psi\Gamma\psi)_{\vec{k}\kappa} \theta[\mu-e(k)],
\end{equation}
%
Temos nesta expressão uma soma dos valores esperados do operador nos estados de uma partícula --~representado por $(\bar\psi\Gamma\psi)_{\vec{k}\kappa}$ acima~--em todos os estados ocupados: são varridos os índices de spins e isospin (no caso de nucleons), bem como em todos os valores de momento\footnote{Como o índice é contínuo, a soma é uma integral. A integral é em todos os valores de momento (de zero ao infinito), mas a função $\theta$ é zero se a energia do estado é maior que a energia do estado ocupado de mais alta energia, isto é, é maior que a energia de Fermi.}. 

Tal expressão não é útil, a não ser que saibamos como calcular $(\bar\psi\Gamma\psi)_{\vec{k}\kappa}$. No entanto,
\begin{equation}
	\frac{\partial}{\partial \zeta} (\psi^\dagger H_D\psi)_{\vec{k}\kappa} = (\psi^\dagger\frac{\partial H_D}{\partial\zeta}\psi)_{\vec{k}\kappa}
\end{equation}
%
e
\begin{equation}
	(\psi^\dagger H_D\psi)_{\vec{k}\kappa} = \underbrace{E(\vec{k}) + g_\omega\omega_0}_{K_0(\vec{k})}.
\end{equation}
%
Portanto,
\begin{equation}
	(\psi^\dagger\frac{\partial H_D}{\partial\zeta}\psi)_{\vec{k}\kappa} = \frac{\partial}{\partial\zeta}K(\vec{k}).
\end{equation}
%
Assim, se conhecemos $H_D$, podemos usar suas derivadas para conseguir o operador em que estamos interessados à esquerda, enquanto podemos calcular seu valor com a expressão à direita.\footnote{Fazer os cálculos para $\mean{\bar\psi\gamma_0k_0\psi}$ e $\mean{\bar\psi\vec{\gamma}\cdot\vec{k}\psi}$ utilizando esse método, seguindo o Glendenning.}

No programa que a Débora me passou como referência, o potencial termodinâmico tem uma forma análoga ao caso eNJL:
\begin{subequations}\label{Eq:Pot_termodinamico_2}
\begin{align}
	\varepsilon &= \tilde{\omega} + n_c\mu\rho_B = \varepsilon_{\rm{kin}} + m_0\rho_s - 2G_S\rho_s^2 \\
	\varepsilon_0 &= \tilde{\omega}_0 + n_c\mu\rho_B = \varepsilon_{\rm{kin}}^0 + m_0\rho_0^s - 2G_S(\rho_0^s)^2 \\
	\varepsilon_{\rm{kin}} &= -\frac{n_c}{\pi^2} [F_2(m, \Lambda) - F_2(m, p_F)] \\
	\varepsilon_{\rm{kin}}^0 &= -\frac{n_c}{\pi^2} [F_2(m, \Lambda) - F_2(m, 0)] \\
	\rho_0^s &= \rho_s|_{m = m_{\rm{vac}}}
\end{align}
\end{subequations}
%
onde, como na Equação~\eqref{Eq:Energia_kin},
\begin{equation}
	F_2(m, p) = \frac{1}{8}\Big((2p^3 - 3M_i^2p)\sqrt{p^2 + M_i^2} + 3M_i^4\ln\frac{p + \sqrt{p^2 + M_i^2}}{M_i}\Big).
\end{equation}
%
Isso se deve ao fato de que a densidade de energia pode ser calculada a partir do potencial termodinâmico através de mais que uma relação. Possivelmente essa esteja relacionada à derivada do potencial termodinâmico em relação à temperatura, ou ao inverso da temperatura. Ver Avancini~\cite{Avancini2004} e~\cite{Avancini2006}.

%%%%%%%%%%%%%%%%%%%%%%%%%%%%%%%%%%%%%%%%%%%
\subsection{Solução através da lagrangiana}
%%%%%%%%%%%%%%%%%%%%%%%%%%%%%%%%%%%%%%%%%%%

As equações de estado são dadas através da lagrangiana através de\cite{Glendenning1983}:
\begin{align}
	\varepsilon &= -\mean{\mathcal{L}} + \mean{\bar\psi\gamma_0k_0\psi} \\
	p &= \mean{\mathcal{L}} + \frac{1}{3} \mean{\bar\psi\vec{\gamma}\cdot\vec{k}\psi}.
\end{align}

%%%%%%%%%%%%%%%%%%%%%%%%%%%%%%%%%%%%%%%%%%%%%%%%%%%%%%%
\subsection{Solução através do potencial termodinâmico}
%%%%%%%%%%%%%%%%%%%%%%%%%%%%%%%%%%%%%%%%%%%%%%%%%%%%%%%

Sair da lagrangiana e obter a expressão abaixo.\cite{Asakawa1989}.

\begin{quote}
Expanding $\bar{\psi}\psi$ and $\bar{\psi}\gamma^\mu\psi$ about their thermal expectation values we can derive the mean field thermodynamic potential at temperature $T$ and chemical potential $\mu$ [11]. We restrict ourselves to the Hartree approximation. Furthermore in this paper we only consider $T=0$ and $\mu \geqslant 0$. The result for the thermodynamic potential is
\begin{equation}\label{Eq:Def_pot_termo}
	\omega_{\rm{MF}}(\mu; m, \mu_R) = \omega_m^{\rm{(vac)}} + \omega_m^{\rm{(med)}}(\mu_R) + \frac{(m - m_0)^2}{4G_S} - \frac{(\mu - \mu_R)^2}{4G_V},
\end{equation}
%
with
\begin{align}
	\omega_m^{\rm{(vac)}} &= - (2n_fn_c) \int \frac{d^3p}{(2\pi)^3} E_p, \label{Eq:Def_omega_vac}\\
	\omega_m^{\rm{(med)}}(\mu_R) &= - (2n_fn_c) \int \frac{d^3p}{(2\pi)^3}(\mu_R - E_p)\theta(p_F - p), \label{Eq:Def_omega_med}
\end{align}
%
being the vacuum part ant the medium part of the thermodynamic potential (per volume) of a free fermion with mass $m$. [\dots] The vacuum part is strongly divergent and has to be regularized. For simplicity we use a sharp cut-off $\Lambda$ in the three momentum space.\footnote{Isso significa que todas as integrais no momento serão limitadas a $\Lambda$.}
\end{quote}
%
Ainda temos que
\begin{align}
	E_p &= \sqrt{m^2+p^2}, \label{Eq:Def_E}\\
	p_F &= \sqrt{\mu_R^2 - m^2}\theta(\mu_R^2 - m^2), \label{Eq:Rel_pot_quim_renorm_mom_fermi}
\end{align}
%
onde temos para $\mu_R$:
\begin{quote}
	In addition to the external\footnote{É nesse parâmetro que fazemos o \emph{loop}? Ou é uma constante?} parameter $\mu$, $\omega_{\rm{MF}}$ depends on two other parameters, the dynamical fermion mass $m$ and the renormalized chemical potential $\mu_R$, which are related to the scalar density $\langle\bar{\psi}\psi\rangle$ and the vector density $\langle\psi^\dagger\psi\rangle$ at the chemical potential $\mu$ by [11]:
	\begin{align}
		m &= m_0 - 2G_S\langle\bar{\psi}\psi\rangle, \label{Eq:Eq_Gap_Buballa_1}\\
		\mu_R &= \mu - 2G_V\langle\psi^\dagger\psi\rangle.
	\end{align}
	These parameter have to be determined self-consistently by calculating $\langle\bar{\psi}\psi\rangle$ and $\langle\psi^\dagger\psi\rangle$ from $\omega_{\rm{MF}}$. It can be shown that the self-consistent solution correspond to the extrema of $\omega_{\rm{MF}}$ as a function of $m$ and $\mu_R$. This leads to a set of coupled equations for $m$ and $\mu_R$ which reads for $T = 0$ and $\mu \geqslant 0$:
\begin{align}
	m &= m_0 + 2G_S(2n_fn_c)\left(\int \frac{d^3p}{(2\pi)^3} \frac{m}{E_p} - \int \frac{d^3p}{(2\pi)^3}\frac{m}{E_p}\theta(p_F - p)\right), \label{Eq:Eq_Gap_Buballa_2}\\
	\mu_R &= \mu - 2G_V(2n_fn_c)\int\frac{d^3p}{(2\pi)^3}\theta(p_F - p).
\end{align}
\end{quote}

Resolvendo a equação para $\mu_R$ acima, obtemos
\begin{align}
	\mu_R &= \mu - 2G_V(2n_fn_c)\int\frac{d^3p}{(2\pi)^3}\theta(p_F - p) \\
	&= \mu - \frac{2G_V(2n_fn_c)}{2\pi^2}\int_0^\Lambda p^2 \theta(p_F - p) dp \\
	&= \mu - \frac{2G_V(n_fn_c)}{3\pi^2}p_F^3 \label{Eq:Expr_pot_quim_renorm}
\end{align}
%
e é possível eliminar a dependência de $\omega_{\rm{MF}}$ na variável $\mu_R$, obtendo algo que depende somente de $m$ (para um dado $\mu$).

Definindo então o potencial termodinâmico ``renormalizado''\footnote{Eu estou chamando assim para diferenciar do $\omega$.} $\tilde\omega$:
\begin{equation}\label{Eq:Def_omega_tilde}
	\tilde{\omega}(\mu;m) = \omega_{\rm{MF}}(\mu; m, \mu_R(\mu,m)) - \omega_{\rm{MF}}(0; m_{\rm{vac}}, 0),
\end{equation}
%
onde calculamos o valor de $\omega_{\rm{MF}}$ para $\mu = \mu_R = 0$ (o que implica\footnote{Isso pode ser visto na expressão \eqref{Eq:Expr_pot_quim_renorm} ou na expressão \eqref{Eq:Rel_pot_quim_renorm_mom_fermi}} em $p_F = 0$), obtendo uma constante que será diminuída do potencial, resultando em um potencial $\tilde\omega$ nulo quando $m = m_{\rm{vac}}$, $\mu = 0$. A obtenção de $m_{\rm{vac}}$ será discutida adiante.

É possível se obter outras grandezas através das equações\footnote{Obter essas expressões.}
\begin{align}
	\rho_B &= \frac{1}{n_c} \langle\psi^\dagger\psi\rangle = \frac{n_f}{3\pi^2}p_F^3, \label{Eq:Rel_Dens_Mom_Fermi_NJL}\\
	\varepsilon &= \tilde{\omega} + \mu n_c \rho_B, \label{Eq:Energia_omega_tilde}\\
	p &= - \tilde{\omega}, \label{Eq:Pressao_omega_tilde}
\end{align}
%
onde ``[\dots] for a given $\mu$ these formulas have to be evaluated for the values of $m$ and $\mu_R$ which minimize the thermodynamic potential.''\footnote{É necessário fazer essa minimização para descobrir quais são esses valores antes?}

%%%%%%%%%%%%%%%%%%%%%%%%%%%
\subsection{Equação do gap}
%%%%%%%%%%%%%%%%%%%%%%%%%%%

Precisamos resolver a Equação~\eqref{Eq:Eq_Gap_Buballa_1} para determinar a massa efetiva $m$ de maneira auto-consistente. Para $T=0$ e $\mu \geqslant 0$, $\mean{\bar\psi\psi}$ pode ser calculada a partir de $\omega_{\rm{MF}}$, resultando em ~\eqref{Eq:Eq_Gap_Buballa_2}.

Resolvendo as integrais, temos
\begin{align}
	\langle\bar{\psi}\psi\rangle &= - 2 n_f n_c \left(\int\frac{d^3p}{(2\pi)^3}\frac{m}{E_p} - \int\frac{d^3p}{(2\pi)^3} \frac{m}{E_p} \theta(p_F - p)\right) \\
	&= - 2 n_f n_c \left(\int_0^\Lambda\frac{dp}{2\pi^2} \frac{mp^2}{E_p} - \int_0^\Lambda \frac{dp}{2\pi^2} \frac{mp^2}{E_p} \theta(p_F - p)\right)
\end{align}
%
onde utilizamos a Eq.~\ref{Eq:Int_d3p_to_dp} e integramos o momento entre o valor mínimo (zero) e o valor do \emph{cutoff} $\Lambda$. As integrais podem ser realizadas utilizando a expressão~\eqref{Eq:Integ_momento_quad}, resultando em
\begin{align}
	\int_0^\Lambda \frac{dp}{2\pi^2} \frac{mp^2}{E_p} &= \frac{m}{2\pi^2} F_0(m,\Lambda)\\
	\int_0^\Lambda \frac{dp}{2\pi^2} \frac{mp^2}{E_p} \theta(p_F - p) &= \int_0^{p_F} \frac{dp}{2\pi^2} \frac{mp^2}{E_p} = \frac{m}{2\pi^2} F_0(m,p_F),
\end{align}
%
onde $F_0(m, p)$ é dada pela expressão~\eqref{Eq:Def_F0_integrado}. Com o auxílio dessas expressões, obtemos
\begin{align}
	\langle\bar{\psi}\psi\rangle &= - n_f n_c \frac{m}{\pi^2} (F_0(m,\Lambda) - F_0(m, p_F)) \label{Eq:Dens_Escalar_NJL_Gv_0}\\
	&= n_f n_c \frac{m}{\pi^2} (F_0(m,p_F) - F_0(m, \Lambda)), \\
	&\equiv \rho_s.
\end{align}
%
Portanto, precisamos resolver a seguinte equação\footnote{Equação do Gap}
\begin{equation}\label{Eq:Eq_Gap_NJL}
m = m_0 - 2 G_S\rho_s.
\end{equation}
%
Para isso, basta reescrevermos a expressão acima como
\begin{equation}
	m - m_0 + 2 G_S\rho_s = 0
\end{equation}
%
e utilizar um método para encontrar zeros de funções, variando $m$ em uma janela de valores que contenha o valor adequado da massa efetiva (entre algo próximo de zero e \np[GeV]{1}, por exemplo). Note que o valor de $\rho_s$ depende de $m$ e deve ser calculado para cada ``chute'' de $m$.

%%%%%%%%%%%%%%%%%%%%%%%%%%%
\subsection{Massa no vácuo}
%%%%%%%%%%%%%%%%%%%%%%%%%%%

Para calcular as grandezas em que estamos interessados ($p$, $\varepsilon$), precisamos calcular $\tilde{\omega}$. Para calcular tal grandeza, precisamos calcular o valor de $m_{\rm{vac}}$, Eq.~\eqref{Eq:Def_omega_tilde}. Esse valor é dado de forma que\cite{Buballa1996}
\begin{quote}
	$m_{\rm{vac}}$ is the dynamical mass which minimizes the thermodynamic potential of the vacuum.
\end{quote}
%
isto é\footnote{ou acredito que seja}, $m_{\rm{vac}}$ deve minimizar o potencial definido pela Eq.~\ref{Eq:Def_pot_termo} para o caso do vácuo\footnote{No Report do Buballa, também há a menção a minimizar o potencial termodinâmico, porém não há o termo $\omega_m^{\rm{med}}$. Acredito que devido à função degrau, esse termo seja zero na forma que temos aqui.}. Podemos determinar o valor que minimiza o potencial tomando a derivada em relação a $m$ e igualando-a a zero:
\begin{align}
	\frac{d}{dm} \omega(\mu = 0; m, \mu_R = 0) &= \frac{d}{dm}\omega_m^{\rm{(vac)}} \\
	&\phantom{=} + \frac{d}{dm}\omega_m^{\rm{(med)}}(\mu_R) \nonumber \\
	&\phantom{=} + \frac{d}{dm}\frac{(m - m_0)^2}{4G_S} \nonumber \\
	&= 0.\nonumber
\end{align}
%
Como $\mu_R = 0$ e $E = \sqrt{p^2+m^2}$, da expressão~\eqref{Eq:Def_omega_med} temos que o segundo termo é nulo (por $\mu_R$ ser zero e pelo fato de que a função degrau será zero, pois $p_F$ é zero se $\mu_R$, Eq.~\eqref{Eq:Rel_pot_quim_renorm_mom_fermi}).

O primeiro termo pode ser calculado através da fórmula de Leibniz\footnote{Se os limites são constantes, o lado direito se resume à integral.}
\begin{equation}
\begin{split}
	\frac{d}{dx} \left(\int_{a(x)}^{b(x)} f(x,t) dt\right) =&~ f(x, b(x))b'(x) - f(x, a(x))a'(x) \\
	&+ \int_{a(x)}^{b(x)}\frac{\partial}{\partial x}f(x,t) dt
\end{split}
\end{equation}
%
resultando em
\begin{align}
	\frac{d}{dm}\omega_m^{\rm{(vac)}} &= -2 n_f n_c\frac{d}{dm}\int \frac{d^3p}{(2\pi)^3} \sqrt{p^2 + m^2} \\
	&= - \frac{n_f n_c}{\pi^2} \int \frac{\partial}{\partial m} p^2\sqrt{p^2+m^2} dp \\
	&= - \frac{n_f n_c}{\pi^2} \int_0^\Lambda \frac{mp^2}{\sqrt{p^2+m^2}} dp \\
	&= - \frac{n_f n_c}{\pi^2} \;m [F_0(m, \Lambda) - F_0(m,0)]
\end{align}
%
onde usamos as relações~\eqref{Eq:Int_d3p_to_dp} e \eqref{Eq:Def_F0},~\eqref{Eq:Def_F0_integrado}.

Finalmente, o terceiro termo é dado por
\begin{equation}
	\frac{d}{dm} \frac{(m - m_0)^2}{4G_S} = \frac{2 (m - m_0)}{4G_S} = \frac{m - m_0}{2G_S}.
\end{equation}
%
Consequentemente,
\begin{align}
	\frac{d}{dm} \omega(\mu = 0; m, \mu_R = 0) &= - \frac{n_f n_c}{\pi^2} \;m [F_0(m, \Lambda) - F_0(m,0)] \\
	&\phantom{=}~+ \frac{m - m_0}{2G_S} \nonumber \\
	&= 0.
\end{align}
%
multiplicando por $2G_S$, obtemos uma equação equivalente:
\begin{equation}\label{Eq:Calculo_m_vac}
	m - m_0 - 2G_S\frac{n_f n_c}{\pi^2} \;m [F_0(m, \Lambda) - F_0(m,0)] = 0.
\end{equation}
%
Devemos resolver tal equação através de um método para encontrar zeros de função, tal como é feito para a equação do gap.

%%%%%%%%%%%%%%%%%%%%%%%%%%%%%%%%%%%%%%%%%%%%%
\subsection{Determinação de $\tilde{\omega}$}
%%%%%%%%%%%%%%%%%%%%%%%%%%%%%%%%%%%%%%%%%%%%%

Uma vez calculados os valores de $m$, $m_{\rm{vac}}$ e $\mu_R$, precisamos calcular $\tilde\omega$ através de $\omega_{\rm{MF}}$. Temos, através das Equações~\eqref{Eq:Def_omega_tilde}, \eqref{Eq:Def_omega_vac}, e~\eqref{Eq:Def_omega_med}, que:
\begin{equation}
\begin{split}
\omega_{\rm{MF}} =&~ -2 n_f n_c \left[\int\frac{d^3p}{(2\pi)^3} E_p + \int\frac{d^3p}{(2\pi)^3}(\mu_R - E_p)\theta(p_F - p)\right] \\
&~+ \frac{(m-m_0)^2}{4G_S}
\end{split}
\end{equation}
%
Utilizando a expressão~\eqref{Eq:Int_d3p_to_dp}, podemos reescrever tal expressão como (separando a integral de $\mu_R$ e explicitando os limites)
\begin{equation}
\begin{split}
\omega_{\rm{MF}} =&~ - \frac{n_f n_c}{\pi^2} \Big[\int_0^\Lambda p^2E_p \; dp - \int_0^\Lambda p^2 E_p \theta(p_F - p)\;dp \\
&~\phantom{- \frac{n_f n_c}{\pi^2} \Big[} + \int_0^\Lambda p^2 \mu_R \theta(p_F - p)\;dp \Big] + \frac{(m-m_0)^2}{4G_S}
\end{split}
\end{equation}

Na expressão acima, verificamos a existência de duas integrais idênticas --~a menos de uma função degrau $\theta$~-- sendo que estamos calculando a diferença entre elas. Tal resultado será zero, exceto para $p_F > p$, pois nesse caso a integral que contém a função degrau será zerada. O efeito líquido disso é o de calcularmos somente a integral sem a função $\theta$, porém no intervalo $[p_F, \Lambda]$. Assim,
\begin{equation}
\begin{split}
\omega_{\rm{MF}} =&~ - \frac{n_f n_c}{\pi^2} \left[\int_{p_F}^\Lambda p^2E_p \; dp + \int_0^\Lambda p^2 \mu_R \theta(p_F - p)\;dp \right] \\
&~ + \frac{(m-m_0)^2}{4G_S} \\
\end{split}
\end{equation}
%
Como $\mu_R$ não depende do momento $p$, temos
\begin{equation}
	\int_0^\Lambda \mu_R \theta(p_F - p) \; dp = \mu_R \int_{0}^{p_F} \; dp = \mu_R\frac{p_F^3}{3}.
\end{equation}
%
Portanto,
\begin{equation}
\omega_{\rm{MF}} = - \frac{n_f n_c}{\pi^2} \left[\int_{p_F}^\Lambda p^2E_p \; dp + \mu_R \frac{p_F^3}{3} \right] + \frac{(m-m_0)^2}{4G_S}.
\end{equation}

Utilizando a definição de $E_p$, Equação~\eqref{Eq:Def_E}, podemos calcular a integral utilizando a expressão \eqref{Eq:Def_F_E}, o que nos leva a
\begin{equation}\label{Eq:Pot_termodinamico_1}
\omega_{\rm{MF}} = - \frac{n_f n_c}{\pi^2} \left[[F_E(m,\Lambda) - F_E(m, p_F)] + \mu_R \frac{p_F^3}{3} \right] + \frac{(m-m_0)^2}{4G_S}.
\end{equation}

O resultado  para $\omega_{\rm{MF}}(0;m_{\rm{vac}},0)$ utilizando o valor para $m_{\rm{vac}}$ obtido através da Eq.~\eqref{Eq:Calculo_m_vac} será uma constante, o que pode ser entendido verificando que a expressão para o potencial termodinâmico nessas condições é calculada a partir de constantes. Além disso, como $\mu = \mu_R = 0$, temos que $p_F = 0$:
\begin{equation}\label{Eq:Pot_termodinamico_1_vacuo}
\omega_{\rm{MF}}(0;m_{\rm{vac}},0) = - \frac{n_f n_c}{\pi^2}[F_E(m_{\rm{vac}},\Lambda) - F_E(m_{\rm{vac}}, 0)] + \frac{(m-m_0)^2}{4G_S}.
\end{equation}

%%%%%%%%%%%%%%%%%%%%%%%%%%%%%%%%%%%%%%%%
%\section{Solução para o caso $G_V = 0$}
%%%%%%%%%%%%%%%%%%%%%%%%%%%%%%%%%%%%%%%%
%
%Em especial, para o caso em que não temos a interação vetorial (o que equivale a dizer que $G_V = 0$, temos que o potencial químico renormalizado é igual ao potencial químico:
%\begin{equation}
%	\mu_R = \mu.
%\end{equation}
%
%Dessa forma, podemos solucionar a Eq.~\eqref{Eq:Eq_Gap_Buballa_1} se calcularmos o momento de Fermi através da expressão~\eqref{Eq:Rel_Dens_Mom_Fermi_NJL}, obtendo\footnote{Para calcular a densidade escalar, basta sabermos os valores de $p_F$ e de $\Lambda$.}
%\begin{equation}
%	p_F = \sqrt[3]{\frac{3\pi^2\rho_B}{n_f}}.
%\end{equation}
%
%Ao solucionarmos a Eq.~\eqref{Eq:Eq_Gap_Buballa_1}, obteremos o valor de $m$, porém não é possível determinar o valor de $\mu_R$.
%\begin{figure*}
%	\begin{tikzpicture}[gnuplot]
%% generated with GNUPLOT 5.0p1 (Lua 5.3; terminal rev. 99, script rev. 100)
%% 2016-03-01T23:40:23 BRT
\path (0.000,0.000) rectangle (14.000,9.000);
\gpcolor{color=gp lt color border}
\gpsetlinetype{gp lt border}
\gpsetdashtype{gp dt solid}
\gpsetlinewidth{1.00}
\draw[gp path] (1.320,0.985)--(1.500,0.985);
\draw[gp path] (13.447,0.985)--(13.267,0.985);
\node[gp node right] at (1.136,0.985) {$922$};
\draw[gp path] (1.320,1.750)--(1.500,1.750);
\draw[gp path] (13.447,1.750)--(13.267,1.750);
\node[gp node right] at (1.136,1.750) {$924$};
\draw[gp path] (1.320,2.514)--(1.500,2.514);
\draw[gp path] (13.447,2.514)--(13.267,2.514);
\node[gp node right] at (1.136,2.514) {$926$};
\draw[gp path] (1.320,3.279)--(1.500,3.279);
\draw[gp path] (13.447,3.279)--(13.267,3.279);
\node[gp node right] at (1.136,3.279) {$928$};
\draw[gp path] (1.320,4.043)--(1.500,4.043);
\draw[gp path] (13.447,4.043)--(13.267,4.043);
\node[gp node right] at (1.136,4.043) {$930$};
\draw[gp path] (1.320,4.808)--(1.500,4.808);
\draw[gp path] (13.447,4.808)--(13.267,4.808);
\node[gp node right] at (1.136,4.808) {$932$};
\draw[gp path] (1.320,5.573)--(1.500,5.573);
\draw[gp path] (13.447,5.573)--(13.267,5.573);
\node[gp node right] at (1.136,5.573) {$934$};
\draw[gp path] (1.320,6.337)--(1.500,6.337);
\draw[gp path] (13.447,6.337)--(13.267,6.337);
\node[gp node right] at (1.136,6.337) {$936$};
\draw[gp path] (1.320,7.102)--(1.500,7.102);
\draw[gp path] (13.447,7.102)--(13.267,7.102);
\node[gp node right] at (1.136,7.102) {$938$};
\draw[gp path] (1.320,7.866)--(1.500,7.866);
\draw[gp path] (13.447,7.866)--(13.267,7.866);
\node[gp node right] at (1.136,7.866) {$940$};
\draw[gp path] (1.320,8.631)--(1.500,8.631);
\draw[gp path] (13.447,8.631)--(13.267,8.631);
\node[gp node right] at (1.136,8.631) {$942$};
\draw[gp path] (1.320,0.985)--(1.320,1.165);
\draw[gp path] (1.320,8.631)--(1.320,8.451);
\node[gp node center] at (1.320,0.677) {$0$};
\draw[gp path] (2.836,0.985)--(2.836,1.165);
\draw[gp path] (2.836,8.631)--(2.836,8.451);
\node[gp node center] at (2.836,0.677) {$0.05$};
\draw[gp path] (4.352,0.985)--(4.352,1.165);
\draw[gp path] (4.352,8.631)--(4.352,8.451);
\node[gp node center] at (4.352,0.677) {$0.1$};
\draw[gp path] (5.868,0.985)--(5.868,1.165);
\draw[gp path] (5.868,8.631)--(5.868,8.451);
\node[gp node center] at (5.868,0.677) {$0.15$};
\draw[gp path] (7.384,0.985)--(7.384,1.165);
\draw[gp path] (7.384,8.631)--(7.384,8.451);
\node[gp node center] at (7.384,0.677) {$0.2$};
\draw[gp path] (8.899,0.985)--(8.899,1.165);
\draw[gp path] (8.899,8.631)--(8.899,8.451);
\node[gp node center] at (8.899,0.677) {$0.25$};
\draw[gp path] (10.415,0.985)--(10.415,1.165);
\draw[gp path] (10.415,8.631)--(10.415,8.451);
\node[gp node center] at (10.415,0.677) {$0.3$};
\draw[gp path] (11.931,0.985)--(11.931,1.165);
\draw[gp path] (11.931,8.631)--(11.931,8.451);
\node[gp node center] at (11.931,0.677) {$0.35$};
\draw[gp path] (13.447,0.985)--(13.447,1.165);
\draw[gp path] (13.447,8.631)--(13.447,8.451);
\node[gp node center] at (13.447,0.677) {$0.4$};
\draw[gp path] (1.320,8.631)--(1.320,0.985)--(13.447,0.985)--(13.447,8.631)--cycle;
\node[gp node center,rotate=-270] at (0.246,4.808) {$E/A = \varepsilon/\rho$ (MeV)};
\node[gp node center] at (7.383,0.215) {$\rho$ $\rm{fm}^{-3}$};
\gpcolor{rgb color={0.580,0.000,0.827}}
\draw[gp path] (1.633,7.149)--(1.644,7.125)--(1.654,7.101)--(1.664,7.077)--(1.675,7.053)%
  --(1.685,7.028)--(1.695,7.004)--(1.706,6.979)--(1.716,6.955)--(1.726,6.930)--(1.737,6.905)%
  --(1.747,6.880)--(1.757,6.855)--(1.768,6.829)--(1.778,6.804)--(1.788,6.779)--(1.799,6.753)%
  --(1.809,6.728)--(1.819,6.702)--(1.830,6.677)--(1.840,6.651)--(1.850,6.626)--(1.860,6.600)%
  --(1.871,6.574)--(1.881,6.549)--(1.891,6.523)--(1.902,6.497)--(1.912,6.472)--(1.922,6.446)%
  --(1.933,6.420)--(1.943,6.394)--(1.953,6.368)--(1.964,6.343)--(1.974,6.317)--(1.984,6.291)%
  --(1.995,6.265)--(2.005,6.240)--(2.015,6.214)--(2.026,6.188)--(2.036,6.163)--(2.046,6.137)%
  --(2.057,6.111)--(2.067,6.086)--(2.077,6.060)--(2.087,6.034)--(2.098,6.009)--(2.108,5.983)%
  --(2.118,5.958)--(2.129,5.932)--(2.139,5.907)--(2.149,5.881)--(2.160,5.856)--(2.170,5.831)%
  --(2.180,5.805)--(2.191,5.780)--(2.201,5.755)--(2.211,5.730)--(2.222,5.705)--(2.232,5.679)%
  --(2.242,5.654)--(2.253,5.629)--(2.263,5.604)--(2.273,5.579)--(2.284,5.555)--(2.294,5.530)%
  --(2.304,5.505)--(2.314,5.480)--(2.325,5.456)--(2.335,5.431)--(2.345,5.406)--(2.356,5.382)%
  --(2.366,5.357)--(2.376,5.333)--(2.387,5.309)--(2.397,5.284)--(2.407,5.260)--(2.418,5.236)%
  --(2.428,5.212)--(2.438,5.188)--(2.449,5.164)--(2.459,5.140)--(2.469,5.116)--(2.480,5.092)%
  --(2.490,5.068)--(2.500,5.044)--(2.511,5.021)--(2.521,4.997)--(2.531,4.974)--(2.542,4.950)%
  --(2.552,4.927)--(2.562,4.903)--(2.572,4.880)--(2.583,4.857)--(2.593,4.834)--(2.603,4.811)%
  --(2.614,4.788)--(2.624,4.765)--(2.634,4.742)--(2.645,4.719)--(2.655,4.696)--(2.665,4.674)%
  --(2.676,4.651)--(2.686,4.629)--(2.696,4.606)--(2.707,4.584)--(2.717,4.561)--(2.727,4.539)%
  --(2.738,4.517)--(2.748,4.495)--(2.758,4.473)--(2.769,4.451)--(2.779,4.429)--(2.789,4.407)%
  --(2.799,4.385)--(2.810,4.364)--(2.820,4.342)--(2.830,4.320)--(2.841,4.299)--(2.851,4.277)%
  --(2.861,4.256)--(2.872,4.235)--(2.882,4.214)--(2.892,4.193)--(2.903,4.171)--(2.913,4.150)%
  --(2.923,4.130)--(2.934,4.109)--(2.944,4.088)--(2.954,4.067)--(2.965,4.047)--(2.975,4.026)%
  --(2.985,4.006)--(2.996,3.985)--(3.006,3.965)--(3.016,3.945)--(3.026,3.925)--(3.037,3.905)%
  --(3.047,3.885)--(3.057,3.865)--(3.068,3.845)--(3.078,3.825)--(3.088,3.805)--(3.099,3.786)%
  --(3.109,3.766)--(3.119,3.747)--(3.130,3.727)--(3.140,3.708)--(3.150,3.689)--(3.161,3.669)%
  --(3.171,3.650)--(3.181,3.631)--(3.192,3.612)--(3.202,3.594)--(3.212,3.575)--(3.223,3.556)%
  --(3.233,3.537)--(3.243,3.519)--(3.253,3.500)--(3.264,3.482)--(3.274,3.464)--(3.284,3.445)%
  --(3.295,3.427)--(3.305,3.409)--(3.315,3.391)--(3.326,3.373)--(3.336,3.355)--(3.346,3.337)%
  --(3.357,3.320)--(3.367,3.302)--(3.377,3.284)--(3.388,3.267)--(3.398,3.250)--(3.408,3.232)%
  --(3.419,3.215)--(3.429,3.198)--(3.439,3.181)--(3.450,3.164)--(3.460,3.147)--(3.470,3.130)%
  --(3.480,3.113)--(3.491,3.096)--(3.501,3.080)--(3.511,3.063)--(3.522,3.047)--(3.532,3.030)%
  --(3.542,3.014)--(3.553,2.998)--(3.563,2.982)--(3.573,2.966)--(3.584,2.950)--(3.594,2.934)%
  --(3.604,2.918)--(3.615,2.902)--(3.625,2.886)--(3.635,2.871)--(3.646,2.855)--(3.656,2.840)%
  --(3.666,2.824)--(3.677,2.809)--(3.687,2.794)--(3.697,2.779)--(3.707,2.764)--(3.718,2.749)%
  --(3.728,2.734)--(3.738,2.719)--(3.749,2.704)--(3.759,2.689)--(3.769,2.675)--(3.780,2.660)%
  --(3.790,2.646)--(3.800,2.631)--(3.811,2.617)--(3.821,2.603)--(3.831,2.589)--(3.842,2.575)%
  --(3.852,2.561)--(3.862,2.547)--(3.873,2.533)--(3.883,2.519)--(3.893,2.505)--(3.904,2.492)%
  --(3.914,2.478)--(3.924,2.465)--(3.934,2.451)--(3.945,2.438)--(3.955,2.425)--(3.965,2.412)%
  --(3.976,2.399)--(3.986,2.386)--(3.996,2.373)--(4.007,2.360)--(4.017,2.347)--(4.027,2.335)%
  --(4.038,2.322)--(4.048,2.309)--(4.058,2.297)--(4.069,2.285)--(4.079,2.272)--(4.089,2.260)%
  --(4.100,2.248)--(4.110,2.236)--(4.120,2.224)--(4.131,2.212)--(4.141,2.200)--(4.151,2.188)%
  --(4.161,2.177)--(4.172,2.165)--(4.182,2.153)--(4.192,2.142)--(4.203,2.131)--(4.213,2.119)%
  --(4.223,2.108)--(4.234,2.097)--(4.244,2.086)--(4.254,2.075)--(4.265,2.064)--(4.275,2.053)%
  --(4.285,2.042)--(4.296,2.031)--(4.306,2.021)--(4.316,2.010)--(4.327,2.000)--(4.337,1.989)%
  --(4.347,1.979)--(4.358,1.969)--(4.368,1.959)--(4.378,1.948)--(4.388,1.938)--(4.399,1.928)%
  --(4.409,1.919)--(4.419,1.909)--(4.430,1.899)--(4.440,1.889)--(4.450,1.880)--(4.461,1.870)%
  --(4.471,1.861)--(4.481,1.851)--(4.492,1.842)--(4.502,1.833)--(4.512,1.824)--(4.523,1.815)%
  --(4.533,1.806)--(4.543,1.797)--(4.554,1.788)--(4.564,1.779)--(4.574,1.770)--(4.585,1.762)%
  --(4.595,1.753)--(4.605,1.745)--(4.615,1.736)--(4.626,1.728)--(4.636,1.720)--(4.646,1.712)%
  --(4.657,1.703)--(4.667,1.695)--(4.677,1.687)--(4.688,1.680)--(4.698,1.672)--(4.708,1.664)%
  --(4.719,1.656)--(4.729,1.649)--(4.739,1.641)--(4.750,1.634)--(4.760,1.626)--(4.770,1.619)%
  --(4.781,1.612)--(4.791,1.604)--(4.801,1.597)--(4.812,1.590)--(4.822,1.583)--(4.832,1.576)%
  --(4.842,1.570)--(4.853,1.563)--(4.863,1.556)--(4.873,1.550)--(4.884,1.543)--(4.894,1.537)%
  --(4.904,1.530)--(4.915,1.524)--(4.925,1.518)--(4.935,1.511)--(4.946,1.505)--(4.956,1.499)%
  --(4.966,1.493)--(4.977,1.487)--(4.987,1.482)--(4.997,1.476)--(5.008,1.470)--(5.018,1.464)%
  --(5.028,1.459)--(5.039,1.453)--(5.049,1.448)--(5.059,1.443)--(5.069,1.437)--(5.080,1.432)%
  --(5.090,1.427)--(5.100,1.422)--(5.111,1.417)--(5.121,1.412)--(5.131,1.407)--(5.142,1.402)%
  --(5.152,1.398)--(5.162,1.393)--(5.173,1.388)--(5.183,1.384)--(5.193,1.379)--(5.204,1.375)%
  --(5.214,1.371)--(5.224,1.366)--(5.235,1.362)--(5.245,1.358)--(5.255,1.354)--(5.266,1.350)%
  --(5.276,1.346)--(5.286,1.342)--(5.296,1.339)--(5.307,1.335)--(5.317,1.331)--(5.327,1.328)%
  --(5.338,1.324)--(5.348,1.321)--(5.358,1.317)--(5.369,1.314)--(5.379,1.311)--(5.389,1.308)%
  --(5.400,1.305)--(5.410,1.302)--(5.420,1.299)--(5.431,1.296)--(5.441,1.293)--(5.451,1.290)%
  --(5.462,1.287)--(5.472,1.285)--(5.482,1.282)--(5.493,1.280)--(5.503,1.277)--(5.513,1.275)%
  --(5.523,1.273)--(5.534,1.270)--(5.544,1.268)--(5.554,1.266)--(5.565,1.264)--(5.575,1.262)%
  --(5.585,1.260)--(5.596,1.258)--(5.606,1.256)--(5.616,1.255)--(5.627,1.253)--(5.637,1.251)%
  --(5.647,1.250)--(5.658,1.248)--(5.668,1.247)--(5.678,1.246)--(5.689,1.244)--(5.699,1.243)%
  --(5.709,1.242)--(5.720,1.241)--(5.730,1.240)--(5.740,1.239)--(5.750,1.238)--(5.761,1.237)%
  --(5.771,1.236)--(5.781,1.236)--(5.792,1.235)--(5.802,1.234)--(5.812,1.234)--(5.823,1.233)%
  --(5.833,1.233)--(5.843,1.233)--(5.854,1.232)--(5.864,1.232)--(5.874,1.232)--(5.885,1.232)%
  --(5.895,1.232)--(5.905,1.232)--(5.916,1.232)--(5.926,1.232)--(5.936,1.232)--(5.947,1.233)%
  --(5.957,1.233)--(5.967,1.233)--(5.977,1.234)--(5.988,1.234)--(5.998,1.235)--(6.008,1.236)%
  --(6.019,1.236)--(6.029,1.237)--(6.039,1.238)--(6.050,1.239)--(6.060,1.240)--(6.070,1.241)%
  --(6.081,1.242)--(6.091,1.243)--(6.101,1.244)--(6.112,1.245)--(6.122,1.247)--(6.132,1.248)%
  --(6.143,1.249)--(6.153,1.251)--(6.163,1.252)--(6.174,1.254)--(6.184,1.256)--(6.194,1.257)%
  --(6.204,1.259)--(6.215,1.261)--(6.225,1.263)--(6.235,1.265)--(6.246,1.267)--(6.256,1.269)%
  --(6.266,1.271)--(6.277,1.273)--(6.287,1.276)--(6.297,1.278)--(6.308,1.280)--(6.318,1.283)%
  --(6.328,1.285)--(6.339,1.288)--(6.349,1.290)--(6.359,1.293)--(6.370,1.295)--(6.380,1.298)%
  --(6.390,1.301)--(6.401,1.304)--(6.411,1.307)--(6.421,1.310)--(6.431,1.313)--(6.442,1.316)%
  --(6.452,1.319)--(6.462,1.322)--(6.473,1.326)--(6.483,1.329)--(6.493,1.332)--(6.504,1.336)%
  --(6.514,1.339)--(6.524,1.343)--(6.535,1.346)--(6.545,1.350)--(6.555,1.354)--(6.566,1.357)%
  --(6.576,1.361)--(6.586,1.365)--(6.597,1.369)--(6.607,1.373)--(6.617,1.377)--(6.628,1.381)%
  --(6.638,1.385)--(6.648,1.389)--(6.658,1.394)--(6.669,1.398)--(6.679,1.402)--(6.689,1.407)%
  --(6.700,1.411)--(6.710,1.416)--(6.720,1.420)--(6.731,1.425)--(6.741,1.429)--(6.751,1.434)%
  --(6.762,1.439)--(6.772,1.444)--(6.782,1.449)--(6.793,1.454)--(6.803,1.459)--(6.813,1.464)%
  --(6.824,1.469)--(6.834,1.474)--(6.844,1.479)--(6.855,1.484)--(6.865,1.490)--(6.875,1.495)%
  --(6.885,1.500)--(6.896,1.506)--(6.906,1.511)--(6.916,1.517)--(6.927,1.522)--(6.937,1.528)%
  --(6.947,1.534)--(6.958,1.540)--(6.968,1.545)--(6.978,1.551)--(6.989,1.557)--(6.999,1.563)%
  --(7.009,1.569)--(7.020,1.575)--(7.030,1.581)--(7.040,1.587)--(7.051,1.594)--(7.061,1.600)%
  --(7.071,1.606)--(7.082,1.613)--(7.092,1.619)--(7.102,1.625)--(7.112,1.632)--(7.123,1.638)%
  --(7.133,1.645)--(7.143,1.652)--(7.154,1.658)--(7.164,1.665)--(7.174,1.672)--(7.185,1.679)%
  --(7.195,1.686)--(7.205,1.693)--(7.216,1.700)--(7.226,1.707)--(7.236,1.714)--(7.247,1.721)%
  --(7.257,1.728)--(7.267,1.735)--(7.278,1.742)--(7.288,1.750)--(7.298,1.757)--(7.309,1.765)%
  --(7.319,1.772)--(7.329,1.780)--(7.339,1.787)--(7.350,1.795)--(7.360,1.802)--(7.370,1.810)%
  --(7.381,1.818)--(7.391,1.826)--(7.401,1.833)--(7.412,1.841)--(7.422,1.849)--(7.432,1.857)%
  --(7.443,1.865)--(7.453,1.873)--(7.463,1.881)--(7.474,1.889)--(7.484,1.898)--(7.494,1.906)%
  --(7.505,1.914)--(7.515,1.922)--(7.525,1.931)--(7.536,1.939)--(7.546,1.948)--(7.556,1.956)%
  --(7.566,1.965)--(7.577,1.973)--(7.587,1.982)--(7.597,1.991)--(7.608,1.999)--(7.618,2.008)%
  --(7.628,2.017)--(7.639,2.026)--(7.649,2.035)--(7.659,2.044)--(7.670,2.053)--(7.680,2.062)%
  --(7.690,2.071)--(7.701,2.080)--(7.711,2.089)--(7.721,2.098)--(7.732,2.107)--(7.742,2.117)%
  --(7.752,2.126)--(7.763,2.135)--(7.773,2.145)--(7.783,2.154)--(7.793,2.164)--(7.804,2.173)%
  --(7.814,2.183)--(7.824,2.192)--(7.835,2.202)--(7.845,2.212)--(7.855,2.221)--(7.866,2.231)%
  --(7.876,2.241)--(7.886,2.251)--(7.897,2.261)--(7.907,2.271)--(7.917,2.281)--(7.928,2.291)%
  --(7.938,2.301)--(7.948,2.311)--(7.959,2.321)--(7.969,2.331)--(7.979,2.341)--(7.990,2.351)%
  --(8.000,2.362)--(8.010,2.372)--(8.021,2.382)--(8.031,2.393)--(8.041,2.403)--(8.051,2.414)%
  --(8.062,2.424)--(8.072,2.435)--(8.082,2.445)--(8.093,2.456)--(8.103,2.467)--(8.113,2.477)%
  --(8.124,2.488)--(8.134,2.499)--(8.144,2.510)--(8.155,2.521)--(8.165,2.532)--(8.175,2.542)%
  --(8.186,2.553)--(8.196,2.564)--(8.206,2.575)--(8.217,2.587)--(8.227,2.598)--(8.237,2.609)%
  --(8.248,2.620)--(8.258,2.631)--(8.268,2.643)--(8.278,2.654)--(8.289,2.665)--(8.299,2.677)%
  --(8.309,2.688)--(8.320,2.699)--(8.330,2.711)--(8.340,2.722)--(8.351,2.734)--(8.361,2.745)%
  --(8.371,2.757)--(8.382,2.769)--(8.392,2.780)--(8.402,2.792)--(8.413,2.804)--(8.423,2.816)%
  --(8.433,2.828)--(8.444,2.839)--(8.454,2.851)--(8.464,2.863)--(8.475,2.875)--(8.485,2.887)%
  --(8.495,2.899)--(8.505,2.911)--(8.516,2.923)--(8.526,2.935)--(8.536,2.948)--(8.547,2.960)%
  --(8.557,2.972)--(8.567,2.984)--(8.578,2.997)--(8.588,3.009)--(8.598,3.021)--(8.609,3.034)%
  --(8.619,3.046)--(8.629,3.059)--(8.640,3.071)--(8.650,3.084)--(8.660,3.096)--(8.671,3.109)%
  --(8.681,3.121)--(8.691,3.134)--(8.702,3.147)--(8.712,3.159)--(8.722,3.172)--(8.732,3.185)%
  --(8.743,3.198)--(8.753,3.211)--(8.763,3.223)--(8.774,3.236)--(8.784,3.249)--(8.794,3.262)%
  --(8.805,3.275)--(8.815,3.288)--(8.825,3.301)--(8.836,3.314)--(8.846,3.327)--(8.856,3.340)%
  --(8.867,3.354)--(8.877,3.367)--(8.887,3.380)--(8.898,3.393)--(8.908,3.407)--(8.918,3.420)%
  --(8.929,3.433)--(8.939,3.447)--(8.949,3.460)--(8.959,3.473)--(8.970,3.487)--(8.980,3.500)%
  --(8.990,3.514)--(9.001,3.527)--(9.011,3.541)--(9.021,3.555)--(9.032,3.568)--(9.042,3.582)%
  --(9.052,3.595)--(9.063,3.609)--(9.073,3.623)--(9.083,3.637)--(9.094,3.650)--(9.104,3.664)%
  --(9.114,3.678)--(9.125,3.692)--(9.135,3.706)--(9.145,3.720)--(9.156,3.734)--(9.166,3.748)%
  --(9.176,3.762)--(9.186,3.776)--(9.197,3.790)--(9.207,3.804)--(9.217,3.818)--(9.228,3.832)%
  --(9.238,3.846)--(9.248,3.860)--(9.259,3.874)--(9.269,3.889)--(9.279,3.903)--(9.290,3.917)%
  --(9.300,3.931)--(9.310,3.946)--(9.321,3.960)--(9.331,3.974)--(9.341,3.989)--(9.352,4.003)%
  --(9.362,4.018)--(9.372,4.032)--(9.383,4.047)--(9.393,4.061)--(9.403,4.076)--(9.413,4.090)%
  --(9.424,4.105)--(9.434,4.119)--(9.444,4.134)--(9.455,4.149)--(9.465,4.163)--(9.475,4.178)%
  --(9.486,4.193)--(9.496,4.207)--(9.506,4.222)--(9.517,4.237)--(9.527,4.252)--(9.537,4.267)%
  --(9.548,4.281)--(9.558,4.296)--(9.568,4.311)--(9.579,4.326)--(9.589,4.341)--(9.599,4.356)%
  --(9.610,4.371)--(9.620,4.386)--(9.630,4.401)--(9.640,4.416)--(9.651,4.431)--(9.661,4.446)%
  --(9.671,4.461)--(9.682,4.476)--(9.692,4.491)--(9.702,4.507)--(9.713,4.522)--(9.723,4.537)%
  --(9.733,4.552)--(9.744,4.567)--(9.754,4.583)--(9.764,4.598)--(9.775,4.613)--(9.785,4.628)%
  --(9.795,4.644)--(9.806,4.659)--(9.816,4.675)--(9.826,4.690)--(9.837,4.705)--(9.847,4.721)%
  --(9.857,4.736)--(9.867,4.752)--(9.878,4.767)--(9.888,4.783)--(9.898,4.798)--(9.909,4.814)%
  --(9.919,4.829)--(9.929,4.845)--(9.940,4.860)--(9.950,4.876)--(9.960,4.891)--(9.971,4.907)%
  --(9.981,4.923)--(9.991,4.938)--(10.002,4.954)--(10.012,4.970)--(10.022,4.986)--(10.033,5.001)%
  --(10.043,5.017)--(10.053,5.033)--(10.064,5.049)--(10.074,5.064)--(10.084,5.080)--(10.094,5.096)%
  --(10.105,5.112)--(10.115,5.128)--(10.125,5.143)--(10.136,5.159)--(10.146,5.175)--(10.156,5.191)%
  --(10.167,5.207)--(10.177,5.223)--(10.187,5.239)--(10.198,5.255)--(10.208,5.271)--(10.218,5.287)%
  --(10.229,5.303)--(10.239,5.319)--(10.249,5.335)--(10.260,5.351)--(10.270,5.367)--(10.280,5.383)%
  --(10.291,5.399)--(10.301,5.415)--(10.311,5.432)--(10.321,5.448)--(10.332,5.464)--(10.342,5.480)%
  --(10.352,5.496)--(10.363,5.512)--(10.373,5.528)--(10.383,5.545)--(10.394,5.561)--(10.404,5.577)%
  --(10.414,5.593)--(10.425,5.610)--(10.435,5.626)--(10.445,5.642)--(10.456,5.658)--(10.466,5.675)%
  --(10.476,5.691)--(10.487,5.707)--(10.497,5.724)--(10.507,5.740)--(10.518,5.756)--(10.528,5.773)%
  --(10.538,5.789)--(10.548,5.806)--(10.559,5.822)--(10.569,5.838)--(10.579,5.855)--(10.590,5.871)%
  --(10.600,5.888)--(10.610,5.904)--(10.621,5.920)--(10.631,5.937)--(10.641,5.953)--(10.652,5.970)%
  --(10.662,5.986)--(10.672,6.003)--(10.683,6.019)--(10.693,6.036)--(10.703,6.052)--(10.714,6.069)%
  --(10.724,6.085)--(10.734,6.102)--(10.745,6.119)--(10.755,6.135)--(10.765,6.152)--(10.775,6.168)%
  --(10.786,6.185)--(10.796,6.201)--(10.806,6.218)--(10.817,6.235)--(10.827,6.251)--(10.837,6.268)%
  --(10.848,6.285)--(10.858,6.301)--(10.868,6.318)--(10.879,6.334)--(10.889,6.351)--(10.899,6.368)%
  --(10.910,6.384)--(10.920,6.401)--(10.930,6.418)--(10.941,6.434)--(10.951,6.451)--(10.961,6.468)%
  --(10.972,6.485)--(10.982,6.501)--(10.992,6.518)--(11.002,6.535)--(11.013,6.551)--(11.023,6.568)%
  --(11.033,6.585)--(11.044,6.602)--(11.054,6.618)--(11.064,6.635)--(11.075,6.652)--(11.085,6.669)%
  --(11.095,6.685)--(11.106,6.702)--(11.116,6.719)--(11.126,6.736)--(11.137,6.752)--(11.147,6.769)%
  --(11.157,6.786)--(11.168,6.803)--(11.178,6.819)--(11.188,6.836)--(11.199,6.853)--(11.209,6.870)%
  --(11.219,6.887)--(11.229,6.903)--(11.240,6.920)--(11.250,6.937)--(11.260,6.954)--(11.271,6.971)%
  --(11.281,6.987)--(11.291,7.004)--(11.302,7.021)--(11.312,7.038)--(11.322,7.055)--(11.333,7.071)%
  --(11.343,7.088)--(11.353,7.105)--(11.364,7.122)--(11.374,7.139)--(11.384,7.156)--(11.395,7.172)%
  --(11.405,7.189)--(11.415,7.206)--(11.426,7.223)--(11.436,7.240)--(11.446,7.256)--(11.456,7.273)%
  --(11.467,7.290)--(11.477,7.307)--(11.487,7.324)--(11.498,7.341)--(11.508,7.357)--(11.518,7.374)%
  --(11.529,7.391)--(11.539,7.408)--(11.549,7.425)--(11.560,7.441)--(11.570,7.458)--(11.580,7.475)%
  --(11.591,7.492)--(11.601,7.509)--(11.611,7.525)--(11.622,7.542)--(11.632,7.559)--(11.642,7.576)%
  --(11.653,7.593)--(11.663,7.609)--(11.673,7.626)--(11.683,7.643)--(11.694,7.660)--(11.704,7.677)%
  --(11.714,7.693)--(11.725,7.710)--(11.735,7.727)--(11.745,7.744)--(11.756,7.760)--(11.766,7.777)%
  --(11.776,7.794)--(11.787,7.811)--(11.797,7.827)--(11.807,7.844)--(11.818,7.861)--(11.828,7.878)%
  --(11.838,7.894)--(11.849,7.911)--(11.859,7.928)--(11.869,7.944)--(11.880,7.961)--(11.890,7.978)%
  --(11.900,7.995)--(11.910,8.011)--(11.921,8.028)--(11.931,8.045)--(11.941,8.061);
\gpcolor{color=gp lt color border}
\draw[gp path] (1.320,8.631)--(1.320,0.985)--(13.447,0.985)--(13.447,8.631)--cycle;
%% coordinates of the plot area
\gpdefrectangularnode{gp plot 1}{\pgfpoint{1.320cm}{0.985cm}}{\pgfpoint{13.447cm}{8.631cm}}
\end{tikzpicture}
%% gnuplot variables

%	\caption{Gráfico da relação~\eqref{Eq:Rel_pot_quim_renorm_mom_fermi} para $m = 100$.}
%\end{figure*}
%Caso $\mu_R^2 > m^2$, temos uma relação simples para $\mu_R$:
%\begin{equation}
%	\mu_R = \sqrt{p_F^2 + m^2}, \quad \textrm{se}~ \mu_R^2 > m^2.
%\end{equation}
%
%No entanto, para qualquer valor de $\mu_R^2$ menor que $m^2$, não é possível inverter a relação~\eqref{Eq:Rel_pot_quim_renorm_mom_fermi}: todo valor de $\mu_R$ tal que $\mu_R^2 < m^2$ está associado a $p_F = 0$, e não é possível determinar uma inversa pois o elemento $0$ dos valores de $p_F$ levaria a diversos elementos dos valores de $\mu_R$ e, portanto, a relação inversa para $\mu_R^2 < m^2$ não é uma função em tal intervalo.\footnote{Uma interpretação para isso é a de que existe um valor mínimo $\mu_R = m$ para o potencial químico renormalizado.}

%Por outro lado, utilizando as Equações~\eqref{Eq:Rel_Dens_Mom_Fermi_NJL} e~\eqref{Eq:Rel_pot_quim_renorm_mom_fermi} podemos escrever
%\begin{equation}
%	\sqrt[3]{\frac{3\pi^2}{n_f} \rho_B} = \sqrt{\mu_R^2 - m^2}\theta(\mu_R^2 - m^2),
%\end{equation}
%
%onde $\theta(\mu_R^2 - m^2)$ garante que o lado direito seja estritamente positivo, ou seja
%\begin{equation}
%	\rho_B > 0,
%\end{equation}
%
%o que é perfeitamente razoável. Consequentemente temos que a expressão para $\mu_R$ é dada por
%\begin{equation}
%	\mu_R = \sqrt{p_F^2 + m^2}.
%\end{equation}
%
%Isto significa que, se rodarmos em $\mu_R$, temos que $p_F$ é zero para $\mu_R^2 < m^2$, o que resulta no mesmo valor para $m$ (resolvendo a Eq.~\eqref{Eq:Eq_Gap_Buballa_1}) para qualquer valor de $\mu_R$ em tal intervalo. Se acontecer o mesmo com o potencial termodinâmico\footnote{Ver se é esse o caso.} $\tilde{\omega}$, não tem problema algum, pois estaríamos calculando ``o mesmo ponto'' para as equações de estado.
%

%%%%%%%%%%%%%%%%%%%%%%%%%%%%%%%%%%
\subsection{Fase de hádrons: eNJL} 
%%%%%%%%%%%%%%%%%%%%%%%%%%%%%%%%%%

A partir da lagrangiana \eqref{Eq:Lagrangiana_eNLJ_Pais}, é possível determinar  o potencial termodinâmico\footnote{Potencial Grand-canônico, ou potencial de Landau.} por unidade de volume, dado por (em Tsue~\cite{japoneses}, o potencial é \emph{definido} como $\omega = \langle\langle \mathcal{H}^{MF}\rangle\rangle - \mu\langle\langle\mathcal{N}\rangle\rangle - \frac{1}{\beta}\langle\langle\mathcal{S}\rangle\rangle$\footnote{Os termos $\langle\langle \cdot \rangle\rangle$ são a Hamiltoniana de campo médio, a densidade bariônica, e algo que pode ser a entropia (eu acho). Isso tem que sair de alguma coisa da termodinâmica, tipo $\omega (T, m) = (T/V) \ln Z$, como poderia ser \emph{definido}?}, o que se traduz em uma expressão similar a essa que segue)
\begin{equation}\label{Eq:potencial_termodinamico}
\begin{split}
	\omega(\mu) =&~ \varepsilon_{\rm{kin}} + m\rho_s - G_s\rho_s^2 + G_v\rho^2 + G_{sv}\rho_s^2\rho^2 + G_\rho\rho_3^2 \\
	&+ G_{v\rho}\rho^2\rho_3^2 + G_{s\rho}\rho_s^2\rho_3^2 - \mu_p\rho_p - \mu_n\rho_n,
\end{split}
\end{equation}
%
onde
\begin{itemize}
	\item $\rho$ é a densidade bariônica, dada pela soma das densidades de nêutron e próton\footnote[][-1cm]{São densidades numéricas de partículas, ou seja, representam o número de partículas por unidade de volume.}:
	\begin{equation}
		\rho = \rho_p + \rho_n.
	\end{equation}

	\item As densidades bariônicas de próton e nêutron são dadas por (alguma indicação de onde isso sai é dada em Tsue~\cite{japoneses} (resumidamente: QFT)):\footnote{Explicar, explicar o momento de Fermi.}
	\begin{equation}
		\rho_i = \int_0^{k_F^i}\frac{dp}{\pi^2}p^2; \qquad i = p,n; \quad k_F^i = \textrm{momento de Fermi},
	\end{equation}
	%
	ou, caso $\rho_i$ sejam conhecidos
	\begin{equation}\label{Eq:Mom_Fermi_a_partir_de_rho}
		p_F^i = \sqrt[3]{3\pi^2\rho_i}.
	\end{equation}
	
	\item $\mu_p$ e $\mu_n$ representam os potenciais químicos de próton e nêutron, respectivamente.
\end{itemize}

O termo cinético na expressão acima pode ser calculado através de (primeiro termo da Eq. (1) em \cite{PRC_68_035804_2003}, o resto é energia potencial; Indicação de onde isso sai é dada em Tsue~\cite{japoneses}.)\footnote{Degenerescência: O 2 se refere às duas possibilidades de spin; Podemos ter um $N_f$ que representa o número de sabores.}
\begin{align}
	\varepsilon_{\rm{kin}} &= \langle\bar{\psi}(\vec{\gamma}\cdot\vec{p})\psi\rangle \\
	&= 2 N_c\sum_i \int \frac{d^3p}{(2\pi)^3}\frac{p^2 + m_i M_i}{E_i}(n_{i-}-n_{i+})\theta(\Lambda^2 - p^2),
\end{align}
%
onde
\begin{itemize}
	\item A soma se dá sobre as espécies de partículas;
	\item $N_c$ representa o número de cores\footnote{No nosso caso, 1?};
	\item $\theta$ é a função degrau, $\Lambda$ é o \emph{cutoff};
	\item $n_{i\pm}$ são as funções de distribuição de Fermi para estados de energia positiva e negativa (respectivamente), dados por
	\begin{equation}
		n_{i\pm} = \frac{1}{1 + \exp(\pm[\beta(E_i\mp\mu_i)])}
	\end{equation}
	%
	onde $i = p, n$ (no nosso caso, no artigo é $u, d, s$) e $\beta = T^{-1}$
	\item $M_i$ é a massa efetiva do nucleon em questão (quark, no artigo).
	\item $E_i = \sqrt{p^2 + M_i^2}$
	\item $m_i$ são as massas constituintes -- nuas, \emph{bare} --.
\end{itemize}

Se tomarmos $T \to 0$, temos que $n_{i-} \to 1$ e $n_{i+} \to 0$\footnote{Depende do sinal de $E - \mu$. Aparementemente isso dá $\theta(\mu-E)$}; Além disso, se o integrando só depende do módulo de $\vec{p}$, então (Glendenning\cite{Glendenning}, p. 92)
\begin{equation}\label{Eq:Int_d3p_to_dp}
	\int\frac{d^3p}{(2\pi)^3} \to \frac{1}{2\pi^2}\int p^2dp.
\end{equation}
%
Logo, temos
\begin{align}
	\varepsilon &= 2 N_c \frac{1}{2\pi^2}\sum_i \int p^2 dp \frac{p^2 + m_i M_i}{\sqrt{p^2 + M_i^2}} \theta(\Lambda^2 - p^2) \\
	&= \frac{N_c}{\pi^2}\sum_i\left[\int \frac{p^4dp}{\sqrt{p^2 + M_i^2}}\theta(\Lambda^2 - p^2) + \int m_i M_i \frac{p^2 dp}{\sqrt{p^2 + M_i^2}}\theta(\Lambda^2 - p^2)\right] \label{Eq:Engergia_cin_separada}
\end{align}
%
Podemos utilizar as relações (Glendenning\cite{Glendenning} p. 94\footnote{Na Ref. o primeiro termo da segunda expressão aparece sem o $k$ multiplicando, o que dimensionalmente está incorreto.})
\begin{align}
	\int \frac{k^4}{\sqrt{k^2 + m^2}} dk &= \frac{1}{4}\left[k^3\epsilon - \frac{3}{2} m^2k\epsilon + \frac{3}{2}m^4\ln\frac{\epsilon + k}{m} \right]\\
	\int \frac{k^2}{\sqrt{k^2 + m^2}} dk &= \frac{1}{2}\left[k\epsilon - m^2\ln\frac{\epsilon + k}{m}\right] \label{Eq:Integ_momento_quad}
\end{align}
%
onde $\epsilon = \sqrt{k^2+m^2}$. Tomando o caso $m_i \to 0$\footnote{No prog. \texttt{eos\_enjl1-dens-assym- clean-rho-vr.f}: $\varepsilon \propto [F_2(M, k_F^i) - F_2(M, \Lambda)]$ ao invés de $\varepsilon \propto [F_2(M_i, \Lambda) - F_2(M_i, 0)]$; Isso se deve à retirada da contribuição do vácuo. Vendo Lee \etal talvez seja o seguinte: $n_i^+$ não vai a zero, mas a uma função degrau que envolve o momento, de forma que entre zero e $p_F$ o resultado seja nulo. A mudança na ordem dos limites pode explicar algum sinal.}, obtemos
\begin{equation}\label{Eq:Energia_kin}
	\varepsilon_{\rm{kin}} = \frac{N_c}{\pi^2}\sum_i \Big[\underbrace{\frac{1}{8}\Big((2p^3 - 3M_i^2p)\sqrt{p^2 + M_i^2} + 3M_i^4\ln\frac{p + \sqrt{p^2 + M_i^2}}{M_i}\Big)}_{F_2(m,p)}\Big]_0^\Lambda
\end{equation}

A densidade escalar $\rho_s$ é dada por (a origem é indicada em Tsue~\cite{japoneses})
\begin{equation}\label{Eq:Dens_Escalar}
	\rho_s^i = \frac{M}{\pi^2}[F_0(M, p_F^i) - F_0(M, \Lambda)], \quad i = p, n,
\end{equation}
%
onde
\begin{equation}\label{Eq:Def_F0}
	F_0(M, x) = \int_0^x dp\frac{p^2}{\sqrt{M^2 + p^2}} dp.
\end{equation}
%
Utilizando a Equação~\eqref{Eq:Integ_momento_quad}, podemos reescrever a equação acima como
\begin{equation}\label{Eq:Def_F0_integrado}
	F_0(M, x) = \frac{1}{2}\left[x\sqrt{x^2+M^2} - M^2 \ln \frac{x + \sqrt{x^2+M^2}}{M}\right].
\end{equation}

A massa efetiva $M$ na equação acima é dada por\footnote{Essa equação é conhecida como \emph{gap equation}}
\begin{equation}\label{Eq:Gap}
	M = m - 2G_s\rho_s + 2G_{sv}\rho_s\rho^2 + 2 G_{s\rho}\rho_s\rho_3^2,
\end{equation}
%
com $\rho_s = \rho_s^p + \rho_s^n$. Temos, portanto, uma interdependência entre as equações. Para que seja possível solucionar tais equações, podemos definir uma função $f(M)$ de tal forma que
\begin{equation}\label{Eq:Gap_zero}
	f(M) = M - m + 2G_s\rho_s - 2G_{sv}\rho_s\rho^2 - 2 G_{s\rho}\rho_s\rho_3^2,.
\end{equation}
%
Para solucionarmos a equação acima, basta utilizarmos uma rotina para encontrar zeros de funções, por exemplo biseção ou Newton-Raphson, encontrando o valor de $M$ para o qual $f(M) = 0$. A densidade escalar $\rho_s$ pode ser calculada através da expressão~\eqref{Eq:Dens_Escalar}.

Os potenciais químicos são dados por\footnote{Como essas expressões são calculadas?}
\begin{equation}\label{Eq:Potenciais_Quimicos}
\begin{split}
	\mu_i =&~ E_{p_F}^i + 2G_v\rho + 2G_{sv}\rho\rho_s^2 \pm 2G_\rho\rho_3+2G_{v\rho}\rho_3^2\rho \\
	& \pm 2G_{v\rho}\rho^2\rho_3 \pm 2 G_{s\rho}\rho_3\rho_s^2,
\end{split}
\end{equation}
%
onde $i = p,n$, os sinais superiores se referem ao caso de prótons, e $E_{p_F}^i = \sqrt{M^2 + (p_F^i)^2}$.

As equações de estado para pressão $P$ e densidade de energia $\varepsilon$ são dadas por\footnote{Como são calculadas?}
\begin{align}
	P &= -\omega(\mu) + \epsilon_0 \label{Eq:Pressao}\\
	\varepsilon &= -P + \mu_p\rho_p + \mu_n\rho_n. \label{Eq:Densidade_energia}
\end{align}
%%%%%%%%%%%%%%%%%%%%%%%%%%%%%%%%%%%
%%%%%%%%%%%%%%%%%%%%%%%%%%%%%%%%%%%
%%%%%%%%%%%%%%%%%%%%%%%%%%%%%%%%%%%

%%%%%%%%%%%%%%%%%%
\section{Apêndice}
%%%%%%%%%%%%%%%%%%

Colocar expressões usadas no capítulo aqui, pra não poluir o texto.

trazer: Eq.~\ref{Eq:Int_d3p_to_dp}, \eqref{Eq:Integ_momento_quad}, \eqref{Eq:Def_F0_integrado}

Com $E = \sqrt{k^2 + m^2}$, 
\begin{align} \label{Eq:Def_F_E}
	\int k^2 \sqrt{k^2 + m^2} \;dk &= \frac{1}{4}\left[k E^3 - \frac{1}{2} m^2 k E - \frac{1}{2} m^4\ln\left(\frac{k+E}{m}\right)\right] \\
	&\equiv F_E(m, k),
\end{align}

%%%%%%%%%%%%%%%%%%%%%%%%
\section{Próximas ações}
%%%%%%%%%%%%%%%%%%%%%%%%

\begin{itemize}
	\item Organizar o capítulo:
		\begin{itemize}
			\item Verificar se há detalhes de implementação para mover;
			\item Mover integrais para o capítulo de implementação;
			\item Rever a sequência, pois tem coisa do eNJL e do NJL; Por do NJL antes
		\end{itemize}
	\item Reproduzir cálculos de $p$ e $\varepsilon$ para NJL:
		\begin{itemize}
			\item Tentar seguir o que a Débora fez (solução pelo tensor);
				\begin{itemize}
					\item Calcular o tensor;
					\item Calcular $\mean{T_{00}}$
					\item Calcular $\mean{T_{ii}}$
					\item Calcular dens. energia, pressão
					\item Determinar a eq do gap
					\item Det. $\mu$
					\item Como adicionar $T$?
					\item Por que precisamos calcular $\varepsilon_0$ e diminuir? (fazemos isso em $p$ tb?)
				\end{itemize}
			\item Verificar o cálculo dos termos $\bra{q}\alpha\cdot p\ket{q}$ e $\bra{q}\beta m^*\ket{q}$ no Walecka;
				\begin{itemize}
					\item Rep. cálculo de $\mean{\bar\psi\gamma_0 k_0\psi}$
					\item Rep. cálculo de $\mean{\bar\psi\gamma_i k_i\psi}$
					\item Calcular $\mean{\mathcal{L}}$
					\item Calcular $\varepsilon$, $p$
					\item Como adicionar $T$?
					\item Precisa calcular $\varepsilon_0$ e diminuir?
				\end{itemize}
			\item Ver o cálculo a partir do potencial termodinâmico (ver Tese do James, tem a parte cinética pelo menos);
				\begin{itemize}
					\item[\checkmark] Alterar notas para refletir interpretação correta do cálculo de $\tilde{\omega}$;
					\item Ler Asakawa \& Yazaki, Nuclear Physics A504 1989 668-684;
					\item Determinar Hamiltoniana;
					\item Determinar função partição;
					\item Determinar potencial termodinâmico;
					\item Determinar dens. de energia, pressão;
					\item O pot. quim. e a massa efetiva são determinadas a partir das eq. do gap e a para o pot quim. efetivo, não precisa calcular depois (pelo que entendi)
				\end{itemize}
		\end{itemize}
	\item Dúvidas:
		\begin{itemize}
			\item Em Buballa\cite{Buballa1996}, $\mu$ é um ``parâmetro externo''. Que parâmetro é esse? Como ele é determinado? (Achei que fosse a partir do potencial termodinâmico, mas o potencial termodinâmico é calculado a partir dele! Só tem equação para o potencial químico efetivo $\mu_R$.)\footnote{É pra ser algo que se calcula do potencial termodinâmico, de alguma maneira ele tem que sair daí.}
		\end{itemize}
\end{itemize}
