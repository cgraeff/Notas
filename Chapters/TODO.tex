\chapter{TODO}

%%%%%%%%%%%%%%%%%%
\section{Programa}
%%%%%%%%%%%%%%%%%%

\begin{enumerate}
\item Fazer uma função para ler opções da linha de comando (devo ter isso meio feito em algum lugar).
	\begin{itemize}
	\item deve retornar uma struct com valores booleanos, double, int, strings
	\item
		\begin{description}
			\item[formatos:] Booleanos: -p, --parameters, -p = YES (y, yes, ...), --parameter = Yes (...);
			\item[Outros:] -p 1.3 --parameter 1.3, -p = 1.3, --parameter = 1.3
		\end{description}
	\item metadados: deve ser possível pelo menos perguntar se uma opção foi dada na linha de comando, caso não tenha sido, quem pode pergunta usar um valor padrão.
	\end{itemize}
\item Reorganizar EOS de forma a refletir a reorganização em quarks-EOS, incluindo
	\begin{itemize}
		\item Fazer uma função para escrever $f(M)$
		\item Contantes e parâmetros: Considerar como parâmetro tudo que por ventura possamos querer mudar; Os intervalos para procura de zeros de função devem ser considerados como parâmetros, não como constantes. As constantes de acoplamento devem ser consideradas como parâmetros, pois em geral usamos parametrizações diferentes; Considerar como constante só aquelas que nunca vão mudar, como $\hbar c$.
	\end{itemize}
\end{enumerate}

%%%%%%%%%%%%%%%%%%
\section{[Re]view}
%%%%%%%%%%%%%%%%%%

\begin{enumerate}
\item Ver sobre gás de Fermi, momento de Fermi, \dots
\item Adicionar esses links em algum lugar:
	\begin{itemize}
		\item \url{http://www.damtp.cam.ac.uk/user/tong/qftvids.html}
		\item \url{videos.if.usp.br}
		\item \url{https://www.physics.harvard.edu/events/videos/Phys253}
	\end{itemize}
\item Adicionar notas sobre teoria de grupos (e mais essas seções: representações, ações, produto de grupos)
\end{enumerate}
