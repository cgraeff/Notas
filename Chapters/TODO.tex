%%%%%%%%%%%%%%
\chapter{TODO}
%%%%%%%%%%%%%%

Mexer nisso aqui, tá meio inútil. Tentar uma ideia de fazer um ``próximas ações'' aqui. Posso fazer comentários na margem. Ainda falta um lugar para notas e possibilidades que ficam para o futuro (como as propostas de trabalho com o Juan). Basicamente tenho que implementar o GTD aqui, pois é específico para esse trabalho (indefinido, pois não se resume ao pós-doc; é a pesquisa, na verdade) \dots e é isso que eu sincronizo, o zim não (e não vale a pena o trabalho de fazer sincronizar).

Considerar a possibilidade de colocar NJL e eNJL em um capítulo só, mas sem chegar na termodinâmica. Fazer um capítulo com toda a termodinâmica. Fazer um capítulo com os resultados. Fazer um capítulo de motivação antes do projeto até, falando sobre estrelas e tudo mais (útil para futuros seminários na UTFPR).\footnote{Basicamente isso vai virar uma tese.}

%%%%%%%%%%%%%%%%%%
\section{Programa}
%%%%%%%%%%%%%%%%%%

\begin{enumerate}

\end{enumerate}

%%%%%%%%%%%%%%%%%%
\section{[Re]view}
%%%%%%%%%%%%%%%%%%

\begin{enumerate}
\item Ver sobre gás de Fermi, momento de Fermi, \dots
\item Adicionar esses links em algum lugar:
	\begin{itemize}
		\item \url{http://www.damtp.cam.ac.uk/user/tong/qftvids.html}
		\item \url{videos.if.usp.br}
		\item \url{https://www.physics.harvard.edu/events/videos/Phys253}
	\end{itemize}
\item Adicionar notas sobre teoria de grupos (e mais essas seções: representações, ações, produto de grupos)
\end{enumerate}

%%%%%%%%%%%%%%%%%%%%%%%%%%%%
\section{Trabalhos com Juan}
%%%%%%%%%%%%%%%%%%%%%%%%%%%%

Ideia inicial: calcular ambos os mínimos da função F, verificar quando eles são tais que o valor de F seja o mesmo em ambos os mínimos (ou seja, não há um mínimo global, mas dois mínimos equivalentes; existem então dois estados igualmente favoráveis energeticamente). 

\begin{itemize}
	\item Dar uma olhada nos códigos dele
	\item Reler o trabalho PRD
\end{itemize}