%%%%%%%%%%%%%%%%%%%%%%%%%%%%%
\chapter{NJL: Fase de quarks}
%%%%%%%%%%%%%%%%%%%%%%%%%%%%%

%%%%%%%%%%%%%%%%%%%%%%%%%%%%%%%%%%%
\section{Lagrangiana do modelo NJL}
%%%%%%%%%%%%%%%%%%%%%%%%%%%%%%%%%%%

Considerando a lagrangiana do tipo NJL em SU(2) dada por\cite{Buballa1996}
\begin{equation}
	\mathcal{L} = \bar{\psi}(i\gamma^\mu\partial_\mu - m_0)\psi + G_S[(\bar{\psi}\psi)^2 + (\bar{\psi}i\gamma_5\vec{\tau}\psi)^2] - G_V(\bar{\psi}\gamma^\mu \psi)^2,
\end{equation}
%
onde
\begin{itemize}
	\item ``Here $\psi$ is a fermion field with $n_f = 2$ flavors and $n_c$ colors. It may be interpreted [\dots] as a quark field ($n_c = 3$).''
	\item ``Apart from the bare mass $m_0$, the Lagrangian [above] is chirally symmetric.''
	\item $(\bar{\psi}\psi)^2$ e $(\bar{\psi}i\gamma_5\vec{\tau}\psi)^2$ são os canais scalar\footnote{scalar-isoscalar} e pseudoscalar-isovector, ambos com acoplamento $G_s$;
	\item $(\bar{\psi}\gamma^\mu \psi)^2$ é o canal vector-isoscalar. 
	\item ``It is known, e.g., from Walecka model [109], that this channel [(vector-isoscalar)] is quite important at non-zero densities.''; ``In principle we allow for further channels [\dots], which, however, do not contribute at mean-field level as long as we have only one common quark chemical potential.''\cite{Buballa}
\end{itemize}

%%%%%%%%%%%%%%%%%%%%%%%%%%%%%%%%%
\section{Potencial termodinâmico}
%%%%%%%%%%%%%%%%%%%%%%%%%%%%%%%%%

Obtenção do potencial termodinâmico
\begin{quote}
Expanding $\bar{\psi}\psi$ and $\bar{\psi}\gamma^\mu\psi$ about their thermal expectation values we can derive the mean field thermodynamic potential at temperature $T$ and chemical potential $\mu$ [11]. We restrict ourselves to the Hartree approximation. Furthermore in this paper we only consider $T=0$ and $\mu \geqslant 0$. The result for the thermodynamic potential is
\begin{equation}
	\omega_{\rm{MF}}(\mu; m, \mu_R) = \omega_m^{\rm{(vac)}} + \omega_m^{\rm{(med)}}(\mu_R) + \frac{(m - m_0)^2}{4G_S} - \frac{(\mu - \mu_R)^2}{4G_V},
\end{equation}
%
with
\begin{align}
	\omega_m^{\rm{(vac)}} &= - (2n_fn_c) \int \frac{d^3p}{(2\pi)^3} E_p, \\
	\omega_m^{\rm{(med)}}(\mu_R) &= - (2n_fn_c) \int \frac{d^3p}{(2\pi)^3}(\mu_R - E_p)\theta(p_F - p),
\end{align}
%
being the vacuum part ant the medium part of the thermodynamic potential (per volume) of a free fermion with mass $m$. [\dots] The vacuum part is strongly divergent and has to be regularized. For simplicity we use a sharp cut-off $\Lambda$ in the three momentum space.\footnote{Isso significa que todas as integrais no momento serão limitadas a $\Lambda$.}
\end{quote}
%
Ainda temos que
\begin{align}
	E_p &= \sqrt{m^2+p^2}, \\
	p_F &= \sqrt{\mu_R^2 - m^2}\theta(\mu_R^2 - m^2)
\end{align}

Sobre parâmetros:
\begin{quote}
	In addition to the external\footnote{É nesse parâmetro que fazemos o \emph{loop}? Ou é uma constante?} parameter $\mu$, $\omega_{\rm{MF}}$ depends on two other parameters, the dynamical fermion mass $m$ and the renormalized chemical potential $\mu_R$, which are related to the scalar density $\langle\bar{\psi}\psi\rangle$ and the vector density $\langle\psi^\dagger\psi\rangle$ at the chemical potential $\mu$ by [11]:
	\begin{align}
		m &= m_0 - 2G_S\langle\bar{\psi}\psi\rangle, \label{Eq:Eq_Gap_Buballa_1}\\
		\mu_R &= \mu - 2G_V\langle\psi^\dagger\psi\rangle.
	\end{align}
	These parameter have to be determined self-consistently by calculating $\langle\bar{\psi}\psi\rangle$ and $\langle\psi^\dagger\psi\rangle$ from $\omega_{\rm{MF}}$. It can be shown that the self-consistent solution correspond to the extrema of $\omega_{\rm{MF}}$ as a function of $m$ and $\mu_R$. This leads to a set of coupled equations for $m$ and $\mu_R$ which reads for $T = 0$ and $\mu \geqslant 0$:
\begin{align}
	m &= m_0 + 2G_S(2n_fn_c)\left(\int \frac{d^3p}{(2\pi)^3} \frac{m}{E_p} - \int \frac{d^3p}{(2\pi)^3}\frac{m}{E_p}\theta(p_F - p)\right), \label{Eq:Eq_Gap_Buballa_2}\\
	\mu_R &= \mu - 2G_V(2n_fn_c)\int\frac{d^3p}{(2\pi)^3}\theta(p_F - p).
\end{align}
\end{quote}

Resolvendo a equação para $\mu_R$ acima, obtemos
\begin{align}
	\mu_R &= \mu - 2G_V(2n_fn_c)\int\frac{d^3p}{(2\pi)^3}\theta(p_F - p) \\
	&= \mu - \frac{2G_V(2n_fn_c)}{2\pi^2}\int_0^\Lambda p^2 \theta(p_F - p) dp \\
	&= \mu - \frac{2G_V(n_fn_c)}{3\pi^2}p_F^3
\end{align}
%
e é possível eliminar a dependência de $\omega_{\rm{MF}}$ na variável $\mu_R$, obtendo algo que depende somente de $m$ (para um dado $\mu$). Definindo então
\begin{equation}
	\tilde{\omega}(\mu;m) = \omega_{\rm{MF}}(\mu; m, \mu_R(\mu,m)) - \omega_{\rm{MF}}(0; m_{\rm{vac}}, 0),
\end{equation}
%
é possível se obter outras grandezas através das equações\footnote{Se o \emph{loop} é em $\mu_R$, então deve ser possível escrever $\mu_R$ em função de $\rho_B$ através do momento de $p_F$, ou seja, podemos fazer um \emph{loop} em $\rho_B$. De onde vem essa densidade?}
\begin{align}
	\rho_B &= \frac{1}{n_c} \langle\psi^\dagger\psi\rangle = \frac{n_f}{3\pi^2}p_F^3, \\
	\varepsilon &= \tilde{\omega} + \mu n_c \rho_B, \\
	p &= - \tilde{\omega},
\end{align}
%
onde ``[\dots] for a given $\mu$ these formulas have to be evaluated for the values of $m$ and $\mu_R$ which minimize the thermodynamic potential.''\footnote{É necessário fazer essa minimização para descobrir quais são esses valores antes?}

Comparando as expressões~\eqref{Eq:Eq_Gap_Buballa_1} e~\eqref{Eq:Eq_Gap_Buballa_2} verificamos que
\begin{align}
	\langle\bar{\psi}\psi\rangle &= - 2 n_f n_c \left(\int\frac{d^3p}{(2\pi)^3}\frac{m}{E_p} - \int\frac{d^3p}{(2\pi)^3} \frac{m}{E_p} \theta(p_F - p)\right) \\
	&= - 2 n_f n_c \left(\int_0^\Lambda\frac{dp}{2\pi^2} \frac{mp^2}{E_p} - \int_0^\Lambda \frac{dp}{2\pi^2} \frac{mp^2}{E_p} \theta(p_F - p)\right)
\end{align}
%
onde utilizamos a Eq.~\ref{Eq:Int_d3p_to_dp} e integramos o momento entre o valor mínimo (zero) e o valor do \emph{cutoff} $\Lambda$. As integrais podem ser realizadas utilizando a expressão~\eqref{Eq:Integ_momento_quad}, resultando em
\begin{align}
	\int_0^\Lambda \frac{dp}{2\pi^2} \frac{mp^2}{E_p} &= \frac{m}{2\pi} F_0(m,\Lambda)\\
	\int_0^\Lambda \frac{dp}{2\pi^2} \frac{mp^2}{E_p} \theta(p_F - p) &= \int_0^{p_F} \frac{dp}{2\pi^2} \frac{mp^2}{E_p} = \frac{m}{2\pi} F_0(m,p_F),
\end{align}
%
onde $F_0(m, p)$ é dada pela expressão~\eqref{Eq:Def_F0_integrado}. Com o auxílio dessas expressões, obtemos
\begin{align}
	\langle\bar{\psi}\psi\rangle &= - n_f n_c \frac{m}{\pi} (F_0(m,\Lambda) - F_0(m, p_F)) \\
	&= n_f n_c \frac{m}{\pi} (F_0(m,p_F) - F_0(m, \Lambda)).
\end{align}
%
Tal expressão é a mesma que a dada pela Eq.~\eqref{Eq:Dens_Escalar}, se utilizarmos $n_f = 1$ $n_c = 1$ (caso de hádrons)\footnote{Acredito que para os hádrons temos dois \emph{flavors}, mas na expresssão~\eqref{Eq:Dens_Escalar} temos a densidade ou para prótons, ou para nêutrons, sendo que devemos somar as densidades resultantes, o que equivaleria a usar $n_f = 2$ caso ambas as partículas estivessem sujeitas às mesmas condições.}.

%%%%%%%%%%%%%%%%%%%%%%%%%%%%%%%%%%%%%%%
\section{Solução para o caso $G_V = 0$}
%%%%%%%%%%%%%%%%%%%%%%%%%%%%%%%%%%%%%%%

Em especial, para o caso em que não temos a interação vetorial (o que equivale a dizer que $G_V = 0$, temos que o potencial químico renormalizado $\mu_R$ é igual ao potencial químico $\mu$. Dessa forma, podemos relacionar o potencial químico à densidade bariônica através de\footnote{Não vejo como inverter essa relação para determinar $\mu$, pois precisamos de $m$ \dots mas podemos calcular $p_F$ a partir de $\rho_B$, calcular $m$ e então $\mu_R$, ou seja, dá pra rodar em $\rho_B$, mas depois de resolver a Eq. do Gap, precisamos calcular $\mu_R$ para poder calcular $\tilde{\omega}$.}
\begin{equation}
	\mu = \sqrt{(3\pi^2\rho_B/n_f)^{2/3} + m^2}.
\end{equation}

%%%%%%%%%%%%%%%%%%%%%%%%%%%%
\section{Programa Débora}
%%%%%%%%%%%%%%%%%%%%%%%%%%%%

\begin{itemize}
	\item Roda variando o potencial químico $\mu$. É igual em ambos os quarks?
	\item Usa $g = 3/\pi^2$
	\item $\rho_s$ é igual ao caso para hádrons
	\item $p_F^2 = \mu^2 - M^2$; Essa equação aparece no Buballa.
\end{itemize}
