%%%%%%%%%%%%%%%%%%%%%%%%%%%%%
\chapter{NJL: Fase de quarks}
%%%%%%%%%%%%%%%%%%%%%%%%%%%%%

%%%%%%%%%%%%%%%%%%%%%%%%%%%%%%%%%%%
\section{Lagrangiana do modelo NJL}
%%%%%%%%%%%%%%%%%%%%%%%%%%%%%%%%%%%

Considerando a lagrangiana do tipo NJL em SU(2) dada por Buballa\cite{Buballa1996}
\begin{equation}
	\mathcal{L} = \bar{\psi}(i\gamma^\mu\partial_\mu - m_0)\psi + G_S[(\bar{\psi}\psi)^2 + (\bar{\psi}i\gamma_5\vec{\tau}\psi)^2] - G_V(\bar{\psi}\gamma^\mu \psi)^2,
\end{equation}
%
onde
\begin{itemize}
	\item As constantes $G_S$ e $G_V$ na referência são dadas em unidades de $\rm{massa}^{-2}$. Estamos usando unidades de $\rm{fm}^2$ para tais constantes, portanto devemos multiplicar os valores contidos na referência por $(\hbar c)^2$ para manter a convenção adotada.
	\item ``Here $\psi$ is a fermion field with $n_f = 2$ flavors and $n_c$ colors. It may be interpreted [\dots] as a quark field ($n_c = 3$).''
	\item ``Apart from the bare mass $m_0$, the Lagrangian [above] is chirally symmetric.''
	\item $(\bar{\psi}\psi)^2$ e $(\bar{\psi}i\gamma_5\vec{\tau}\psi)^2$ são os canais scalar\footnote{scalar-isoscalar} e pseudoscalar-isovector, ambos com acoplamento $G_s$;
	\item $(\bar{\psi}\gamma^\mu \psi)^2$ é o canal vector-isoscalar. 
	\item ``It is known, e.g., from Walecka model [109], that this channel [(vector-isoscalar)] is quite important at non-zero densities.''; ``In principle we allow for further channels [\dots], which, however, do not contribute at mean-field level as long as we have only one common quark chemical potential.''\cite{Buballa}
\end{itemize}

%%%%%%%%%%%%%%%%%%%%%%%%%%%%%%%%%
\section{Potencial termodinâmico}
%%%%%%%%%%%%%%%%%%%%%%%%%%%%%%%%%

Obtenção do potencial termodinâmico
\begin{quote}
Expanding $\bar{\psi}\psi$ and $\bar{\psi}\gamma^\mu\psi$ about their thermal expectation values we can derive the mean field thermodynamic potential at temperature $T$ and chemical potential $\mu$ [11]. We restrict ourselves to the Hartree approximation. Furthermore in this paper we only consider $T=0$ and $\mu \geqslant 0$. The result for the thermodynamic potential is
\begin{equation}\label{Eq:Def_pot_termo}
	\omega_{\rm{MF}}(\mu; m, \mu_R) = \omega_m^{\rm{(vac)}} + \omega_m^{\rm{(med)}}(\mu_R) + \frac{(m - m_0)^2}{4G_S} - \frac{(\mu - \mu_R)^2}{4G_V},
\end{equation}
%
with
\begin{align}
	\omega_m^{\rm{(vac)}} &= - (2n_fn_c) \int \frac{d^3p}{(2\pi)^3} E_p, \label{Eq:Def_omega_vac}\\
	\omega_m^{\rm{(med)}}(\mu_R) &= - (2n_fn_c) \int \frac{d^3p}{(2\pi)^3}(\mu_R - E_p)\theta(p_F - p), \label{Eq:Def_omega_med}
\end{align}
%
being the vacuum part ant the medium part of the thermodynamic potential (per volume) of a free fermion with mass $m$. [\dots] The vacuum part is strongly divergent and has to be regularized. For simplicity we use a sharp cut-off $\Lambda$ in the three momentum space.\footnote{Isso significa que todas as integrais no momento serão limitadas a $\Lambda$.}
\end{quote}
%
Ainda temos que
\begin{align}
	E_p &= \sqrt{m^2+p^2}, \label{Eq:Def_E}\\
	p_F &= \sqrt{\mu_R^2 - m^2}\theta(\mu_R^2 - m^2) \label{Eq:Rel_pot_quim_renorm_mom_fermi}
\end{align}

Sobre parâmetros:
\begin{quote}
	In addition to the external\footnote{É nesse parâmetro que fazemos o \emph{loop}? Ou é uma constante?} parameter $\mu$, $\omega_{\rm{MF}}$ depends on two other parameters, the dynamical fermion mass $m$ and the renormalized chemical potential $\mu_R$, which are related to the scalar density $\langle\bar{\psi}\psi\rangle$ and the vector density $\langle\psi^\dagger\psi\rangle$ at the chemical potential $\mu$ by [11]:
	\begin{align}
		m &= m_0 - 2G_S\langle\bar{\psi}\psi\rangle, \label{Eq:Eq_Gap_Buballa_1}\\
		\mu_R &= \mu - 2G_V\langle\psi^\dagger\psi\rangle.
	\end{align}
	These parameter have to be determined self-consistently by calculating $\langle\bar{\psi}\psi\rangle$ and $\langle\psi^\dagger\psi\rangle$ from $\omega_{\rm{MF}}$. It can be shown that the self-consistent solution correspond to the extrema of $\omega_{\rm{MF}}$ as a function of $m$ and $\mu_R$. This leads to a set of coupled equations for $m$ and $\mu_R$ which reads for $T = 0$ and $\mu \geqslant 0$:
\begin{align}
	m &= m_0 + 2G_S(2n_fn_c)\left(\int \frac{d^3p}{(2\pi)^3} \frac{m}{E_p} - \int \frac{d^3p}{(2\pi)^3}\frac{m}{E_p}\theta(p_F - p)\right), \label{Eq:Eq_Gap_Buballa_2}\\
	\mu_R &= \mu - 2G_V(2n_fn_c)\int\frac{d^3p}{(2\pi)^3}\theta(p_F - p).
\end{align}
\end{quote}

Resolvendo a equação para $\mu_R$ acima, obtemos
\begin{align}
	\mu_R &= \mu - 2G_V(2n_fn_c)\int\frac{d^3p}{(2\pi)^3}\theta(p_F - p) \\
	&= \mu - \frac{2G_V(2n_fn_c)}{2\pi^2}\int_0^\Lambda p^2 \theta(p_F - p) dp \\
	&= \mu - \frac{2G_V(n_fn_c)}{3\pi^2}p_F^3
\end{align}
%
e é possível eliminar a dependência de $\omega_{\rm{MF}}$ na variável $\mu_R$, obtendo algo que depende somente de $m$ (para um dado $\mu$). Definindo então
\begin{equation}\label{Eq:Def_omega_tilde}
	\tilde{\omega}(\mu;m) = \omega_{\rm{MF}}(\mu; m, \mu_R(\mu,m)) - \omega_{\rm{MF}}(0; m_{\rm{vac}}, 0),
\end{equation}
%
é possível se obter outras grandezas através das equações\footnote{Se o \emph{loop} é em $\mu_R$, então deve ser possível escrever $\mu_R$ em função de $\rho_B$ através do momento de $p_F$, ou seja, podemos fazer um \emph{loop} em $\rho_B$. De onde vem essa densidade?}
\begin{align}
	\rho_B &= \frac{1}{n_c} \langle\psi^\dagger\psi\rangle = \frac{n_f}{3\pi^2}p_F^3, \label{Eq:Rel_Dens_Mom_Fermi_NJL}\\
	\varepsilon &= \tilde{\omega} + \mu n_c \rho_B, \label{Eq:Energia_omega_tilde}\\
	p &= - \tilde{\omega}, \label{Eq:Pressao_omega_tilde}
\end{align}
%
onde ``[\dots] for a given $\mu$ these formulas have to be evaluated for the values of $m$ and $\mu_R$ which minimize the thermodynamic potential.''\footnote{É necessário fazer essa minimização para descobrir quais são esses valores antes?}

Comparando as expressões~\eqref{Eq:Eq_Gap_Buballa_1} e~\eqref{Eq:Eq_Gap_Buballa_2} verificamos que
\begin{align}
	\langle\bar{\psi}\psi\rangle &= - 2 n_f n_c \left(\int\frac{d^3p}{(2\pi)^3}\frac{m}{E_p} - \int\frac{d^3p}{(2\pi)^3} \frac{m}{E_p} \theta(p_F - p)\right) \\
	&= - 2 n_f n_c \left(\int_0^\Lambda\frac{dp}{2\pi^2} \frac{mp^2}{E_p} - \int_0^\Lambda \frac{dp}{2\pi^2} \frac{mp^2}{E_p} \theta(p_F - p)\right)
\end{align}
%
onde utilizamos a Eq.~\ref{Eq:Int_d3p_to_dp} e integramos o momento entre o valor mínimo (zero) e o valor do \emph{cutoff} $\Lambda$. As integrais podem ser realizadas utilizando a expressão~\eqref{Eq:Integ_momento_quad}, resultando em
\begin{align}
	\int_0^\Lambda \frac{dp}{2\pi^2} \frac{mp^2}{E_p} &= \frac{m}{2\pi^2} F_0(m,\Lambda)\\
	\int_0^\Lambda \frac{dp}{2\pi^2} \frac{mp^2}{E_p} \theta(p_F - p) &= \int_0^{p_F} \frac{dp}{2\pi^2} \frac{mp^2}{E_p} = \frac{m}{2\pi^2} F_0(m,p_F),
\end{align}
%
onde $F_0(m, p)$ é dada pela expressão~\eqref{Eq:Def_F0_integrado}. Com o auxílio dessas expressões, obtemos
\begin{align}
	\langle\bar{\psi}\psi\rangle &= - n_f n_c \frac{m}{\pi^2} (F_0(m,\Lambda) - F_0(m, p_F)) \label{Eq:Dens_Escalar_NJL_Gv_0}\\
	&= n_f n_c \frac{m}{\pi^2} (F_0(m,p_F) - F_0(m, \Lambda)), \\
	&\equiv \rho_s.
\end{align}
%
Tal expressão é a mesma que a dada pela Eq.~\eqref{Eq:Dens_Escalar}, se utilizarmos $n_f = 1$ $n_c = 1$ (caso de hádrons)\footnote{Acredito que para os hádrons temos dois \emph{flavors}, mas na expresssão~\eqref{Eq:Dens_Escalar} temos a densidade ou para prótons, ou para nêutrons, sendo que devemos somar as densidades resultantes, o que equivaleria a usar $n_f = 2$ caso ambas as partículas estivessem sujeitas às mesmas condições.}. Portanto, precisamos resolver a seguinte equação\footnote{Equação do Gap}
\begin{equation}\label{Eq:Eq_Gap_NJL}
m = m_0 - 2 G_S\rho_s.
\end{equation}

%%%%%%%%%%%%%%%%%%%%%%%%%%%%%%%%%%%%%%%
\section{Solução para o caso $G_V = 0$}
%%%%%%%%%%%%%%%%%%%%%%%%%%%%%%%%%%%%%%%

Em especial, para o caso em que não temos a interação vetorial (o que equivale a dizer que $G_V = 0$, temos que o potencial químico renormalizado é igual ao potencial químico:
\begin{equation}
	\mu_R = \mu.
\end{equation}
%
Dessa forma, podemos solucionar a Eq.~\eqref{Eq:Eq_Gap_Buballa_1} se calcularmos o momento de Fermi através da expressão~\eqref{Eq:Rel_Dens_Mom_Fermi_NJL}, obtendo\footnote{Para calcular a densidade escalar, basta sabermos os valores de $p_F$ e de $\Lambda$.}
\begin{equation}
	p_F = \sqrt[3]{\frac{3\pi^2\rho_B}{n_f}}.
\end{equation}

Ao solucionarmos a Eq.~\eqref{Eq:Eq_Gap_Buballa_1}, obteremos o valor de $m$, porém não é possível determinar o valor de $\mu_R$.
\begin{figure*}
	\begin{tikzpicture}[gnuplot]
%% generated with GNUPLOT 5.0p2 (Lua 5.2; terminal rev. 99, script rev. 100)
%% Fri Mar 18 15:48:26 2016
\path (0.000,0.000) rectangle (14.000,9.000);
\gpcolor{color=gp lt color border}
\gpsetlinetype{gp lt border}
\gpsetdashtype{gp dt solid}
\gpsetlinewidth{1.00}
\draw[gp path] (1.320,1.680)--(1.500,1.680);
\draw[gp path] (13.447,1.680)--(13.267,1.680);
\node[gp node right] at (1.136,1.680) {$0$};
\draw[gp path] (1.320,3.070)--(1.500,3.070);
\draw[gp path] (13.447,3.070)--(13.267,3.070);
\node[gp node right] at (1.136,3.070) {$100$};
\draw[gp path] (1.320,4.460)--(1.500,4.460);
\draw[gp path] (13.447,4.460)--(13.267,4.460);
\node[gp node right] at (1.136,4.460) {$200$};
\draw[gp path] (1.320,5.851)--(1.500,5.851);
\draw[gp path] (13.447,5.851)--(13.267,5.851);
\node[gp node right] at (1.136,5.851) {$300$};
\draw[gp path] (1.320,7.241)--(1.500,7.241);
\draw[gp path] (13.447,7.241)--(13.267,7.241);
\node[gp node right] at (1.136,7.241) {$400$};
\draw[gp path] (1.320,8.631)--(1.500,8.631);
\draw[gp path] (13.447,8.631)--(13.267,8.631);
\node[gp node right] at (1.136,8.631) {$500$};
\draw[gp path] (1.320,0.985)--(1.320,1.165);
\draw[gp path] (1.320,8.631)--(1.320,8.451);
\node[gp node center] at (1.320,0.677) {$0$};
\draw[gp path] (3.745,0.985)--(3.745,1.165);
\draw[gp path] (3.745,8.631)--(3.745,8.451);
\node[gp node center] at (3.745,0.677) {$100$};
\draw[gp path] (6.171,0.985)--(6.171,1.165);
\draw[gp path] (6.171,8.631)--(6.171,8.451);
\node[gp node center] at (6.171,0.677) {$200$};
\draw[gp path] (8.596,0.985)--(8.596,1.165);
\draw[gp path] (8.596,8.631)--(8.596,8.451);
\node[gp node center] at (8.596,0.677) {$300$};
\draw[gp path] (11.022,0.985)--(11.022,1.165);
\draw[gp path] (11.022,8.631)--(11.022,8.451);
\node[gp node center] at (11.022,0.677) {$400$};
\draw[gp path] (13.447,0.985)--(13.447,1.165);
\draw[gp path] (13.447,8.631)--(13.447,8.451);
\node[gp node center] at (13.447,0.677) {$500$};
\draw[gp path] (1.320,8.631)--(1.320,0.985)--(13.447,0.985)--(13.447,8.631)--cycle;
\node[gp node center,rotate=-270] at (0.246,4.808) {$p_F$};
\node[gp node center] at (7.383,0.215) {$\mu_R$};
\node[gp node left] at (2.788,8.297) {$p_F = \sqrt{\mu_R^2 - m^2}\theta(\mu_R^2 - m^2), m = 100$};
\gpcolor{rgb color={0.580,0.000,0.827}}
\gpsetlinewidth{3.00}
\draw[gp path] (1.688,8.297)--(2.604,8.297);
\draw[gp path] (1.320,1.680)--(1.332,1.680)--(1.344,1.680)--(1.356,1.680)--(1.369,1.680)%
  --(1.381,1.680)--(1.393,1.680)--(1.405,1.680)--(1.417,1.680)--(1.429,1.680)--(1.441,1.680)%
  --(1.453,1.680)--(1.466,1.680)--(1.478,1.680)--(1.490,1.680)--(1.502,1.680)--(1.514,1.680)%
  --(1.526,1.680)--(1.538,1.680)--(1.550,1.680)--(1.563,1.680)--(1.575,1.680)--(1.587,1.680)%
  --(1.599,1.680)--(1.611,1.680)--(1.623,1.680)--(1.635,1.680)--(1.647,1.680)--(1.660,1.680)%
  --(1.672,1.680)--(1.684,1.680)--(1.696,1.680)--(1.708,1.680)--(1.720,1.680)--(1.732,1.680)%
  --(1.744,1.680)--(1.757,1.680)--(1.769,1.680)--(1.781,1.680)--(1.793,1.680)--(1.805,1.680)%
  --(1.817,1.680)--(1.829,1.680)--(1.841,1.680)--(1.854,1.680)--(1.866,1.680)--(1.878,1.680)%
  --(1.890,1.680)--(1.902,1.680)--(1.914,1.680)--(1.926,1.680)--(1.938,1.680)--(1.951,1.680)%
  --(1.963,1.680)--(1.975,1.680)--(1.987,1.680)--(1.999,1.680)--(2.011,1.680)--(2.023,1.680)%
  --(2.035,1.680)--(2.048,1.680)--(2.060,1.680)--(2.072,1.680)--(2.084,1.680)--(2.096,1.680)%
  --(2.108,1.680)--(2.120,1.680)--(2.133,1.680)--(2.145,1.680)--(2.157,1.680)--(2.169,1.680)%
  --(2.181,1.680)--(2.193,1.680)--(2.205,1.680)--(2.217,1.680)--(2.230,1.680)--(2.242,1.680)%
  --(2.254,1.680)--(2.266,1.680)--(2.278,1.680)--(2.290,1.680)--(2.302,1.680)--(2.314,1.680)%
  --(2.327,1.680)--(2.339,1.680)--(2.351,1.680)--(2.363,1.680)--(2.375,1.680)--(2.387,1.680)%
  --(2.399,1.680)--(2.411,1.680)--(2.424,1.680)--(2.436,1.680)--(2.448,1.680)--(2.460,1.680)%
  --(2.472,1.680)--(2.484,1.680)--(2.496,1.680)--(2.508,1.680)--(2.521,1.680)--(2.533,1.680)%
  --(2.545,1.680)--(2.557,1.680)--(2.569,1.680)--(2.581,1.680)--(2.593,1.680)--(2.605,1.680)%
  --(2.618,1.680)--(2.630,1.680)--(2.642,1.680)--(2.654,1.680)--(2.666,1.680)--(2.678,1.680)%
  --(2.690,1.680)--(2.702,1.680)--(2.715,1.680)--(2.727,1.680)--(2.739,1.680)--(2.751,1.680)%
  --(2.763,1.680)--(2.775,1.680)--(2.787,1.680)--(2.799,1.680)--(2.812,1.680)--(2.824,1.680)%
  --(2.836,1.680)--(2.848,1.680)--(2.860,1.680)--(2.872,1.680)--(2.884,1.680)--(2.897,1.680)%
  --(2.909,1.680)--(2.921,1.680)--(2.933,1.680)--(2.945,1.680)--(2.957,1.680)--(2.969,1.680)%
  --(2.981,1.680)--(2.994,1.680)--(3.006,1.680)--(3.018,1.680)--(3.030,1.680)--(3.042,1.680)%
  --(3.054,1.680)--(3.066,1.680)--(3.078,1.680)--(3.091,1.680)--(3.103,1.680)--(3.115,1.680)%
  --(3.127,1.680)--(3.139,1.680)--(3.151,1.680)--(3.163,1.680)--(3.175,1.680)--(3.188,1.680)%
  --(3.200,1.680)--(3.212,1.680)--(3.224,1.680)--(3.236,1.680)--(3.248,1.680)--(3.260,1.680)%
  --(3.272,1.680)--(3.285,1.680)--(3.297,1.680)--(3.309,1.680)--(3.321,1.680)--(3.333,1.680)%
  --(3.345,1.680)--(3.357,1.680)--(3.369,1.680)--(3.382,1.680)--(3.394,1.680)--(3.406,1.680)%
  --(3.418,1.680)--(3.430,1.680)--(3.442,1.680)--(3.454,1.680)--(3.466,1.680)--(3.479,1.680)%
  --(3.491,1.680)--(3.503,1.680)--(3.515,1.680)--(3.527,1.680)--(3.539,1.680)--(3.551,1.680)%
  --(3.563,1.680)--(3.576,1.680)--(3.588,1.680)--(3.600,1.680)--(3.612,1.680)--(3.624,1.680)%
  --(3.636,1.680)--(3.648,1.680)--(3.661,1.680)--(3.673,1.680)--(3.685,1.680)--(3.697,1.680)%
  --(3.709,1.680)--(3.721,1.680)--(3.733,1.680)--(3.745,1.680)--(3.758,1.819)--(3.770,1.877)%
  --(3.782,1.922)--(3.794,1.960)--(3.806,1.993)--(3.818,2.023)--(3.830,2.051)--(3.842,2.077)%
  --(3.855,2.102)--(3.867,2.125)--(3.879,2.147)--(3.891,2.169)--(3.903,2.189)--(3.915,2.209)%
  --(3.927,2.229)--(3.939,2.247)--(3.952,2.265)--(3.964,2.283)--(3.976,2.300)--(3.988,2.317)%
  --(4.000,2.334)--(4.012,2.350)--(4.024,2.366)--(4.036,2.381)--(4.049,2.397)--(4.061,2.412)%
  --(4.073,2.426)--(4.085,2.441)--(4.097,2.455)--(4.109,2.470)--(4.121,2.484)--(4.133,2.497)%
  --(4.146,2.511)--(4.158,2.524)--(4.170,2.538)--(4.182,2.551)--(4.194,2.564)--(4.206,2.577)%
  --(4.218,2.590)--(4.230,2.602)--(4.243,2.615)--(4.255,2.627)--(4.267,2.639)--(4.279,2.652)%
  --(4.291,2.664)--(4.303,2.676)--(4.315,2.688)--(4.327,2.699)--(4.340,2.711)--(4.352,2.723)%
  --(4.364,2.734)--(4.376,2.746)--(4.388,2.757)--(4.400,2.768)--(4.412,2.780)--(4.425,2.791)%
  --(4.437,2.802)--(4.449,2.813)--(4.461,2.824)--(4.473,2.835)--(4.485,2.846)--(4.497,2.857)%
  --(4.509,2.867)--(4.522,2.878)--(4.534,2.889)--(4.546,2.899)--(4.558,2.910)--(4.570,2.920)%
  --(4.582,2.931)--(4.594,2.941)--(4.606,2.951)--(4.619,2.961)--(4.631,2.972)--(4.643,2.982)%
  --(4.655,2.992)--(4.667,3.002)--(4.679,3.012)--(4.691,3.022)--(4.703,3.032)--(4.716,3.042)%
  --(4.728,3.052)--(4.740,3.062)--(4.752,3.072)--(4.764,3.082)--(4.776,3.091)--(4.788,3.101)%
  --(4.800,3.111)--(4.813,3.121)--(4.825,3.130)--(4.837,3.140)--(4.849,3.149)--(4.861,3.159)%
  --(4.873,3.168)--(4.885,3.178)--(4.897,3.187)--(4.910,3.197)--(4.922,3.206)--(4.934,3.216)%
  --(4.946,3.225)--(4.958,3.234)--(4.970,3.244)--(4.982,3.253)--(4.994,3.262)--(5.007,3.271)%
  --(5.019,3.281)--(5.031,3.290)--(5.043,3.299)--(5.055,3.308)--(5.067,3.317)--(5.079,3.326)%
  --(5.091,3.336)--(5.104,3.345)--(5.116,3.354)--(5.128,3.363)--(5.140,3.372)--(5.152,3.381)%
  --(5.164,3.390)--(5.176,3.399)--(5.189,3.408)--(5.201,3.416)--(5.213,3.425)--(5.225,3.434)%
  --(5.237,3.443)--(5.249,3.452)--(5.261,3.461)--(5.273,3.470)--(5.286,3.478)--(5.298,3.487)%
  --(5.310,3.496)--(5.322,3.505)--(5.334,3.513)--(5.346,3.522)--(5.358,3.531)--(5.370,3.539)%
  --(5.383,3.548)--(5.395,3.557)--(5.407,3.565)--(5.419,3.574)--(5.431,3.583)--(5.443,3.591)%
  --(5.455,3.600)--(5.467,3.608)--(5.480,3.617)--(5.492,3.626)--(5.504,3.634)--(5.516,3.643)%
  --(5.528,3.651)--(5.540,3.660)--(5.552,3.668)--(5.564,3.677)--(5.577,3.685)--(5.589,3.694)%
  --(5.601,3.702)--(5.613,3.710)--(5.625,3.719)--(5.637,3.727)--(5.649,3.736)--(5.661,3.744)%
  --(5.674,3.752)--(5.686,3.761)--(5.698,3.769)--(5.710,3.777)--(5.722,3.786)--(5.734,3.794)%
  --(5.746,3.802)--(5.758,3.811)--(5.771,3.819)--(5.783,3.827)--(5.795,3.836)--(5.807,3.844)%
  --(5.819,3.852)--(5.831,3.860)--(5.843,3.869)--(5.855,3.877)--(5.868,3.885)--(5.880,3.893)%
  --(5.892,3.901)--(5.904,3.910)--(5.916,3.918)--(5.928,3.926)--(5.940,3.934)--(5.953,3.942)%
  --(5.965,3.950)--(5.977,3.959)--(5.989,3.967)--(6.001,3.975)--(6.013,3.983)--(6.025,3.991)%
  --(6.037,3.999)--(6.050,4.007)--(6.062,4.015)--(6.074,4.024)--(6.086,4.032)--(6.098,4.040)%
  --(6.110,4.048)--(6.122,4.056)--(6.134,4.064)--(6.147,4.072)--(6.159,4.080)--(6.171,4.088)%
  --(6.183,4.096)--(6.195,4.104)--(6.207,4.112)--(6.219,4.120)--(6.231,4.128)--(6.244,4.136)%
  --(6.256,4.144)--(6.268,4.152)--(6.280,4.160)--(6.292,4.168)--(6.304,4.176)--(6.316,4.184)%
  --(6.328,4.192)--(6.341,4.200)--(6.353,4.208)--(6.365,4.216)--(6.377,4.223)--(6.389,4.231)%
  --(6.401,4.239)--(6.413,4.247)--(6.425,4.255)--(6.438,4.263)--(6.450,4.271)--(6.462,4.279)%
  --(6.474,4.287)--(6.486,4.295)--(6.498,4.302)--(6.510,4.310)--(6.522,4.318)--(6.535,4.326)%
  --(6.547,4.334)--(6.559,4.342)--(6.571,4.350)--(6.583,4.357)--(6.595,4.365)--(6.607,4.373)%
  --(6.619,4.381)--(6.632,4.389)--(6.644,4.396)--(6.656,4.404)--(6.668,4.412)--(6.680,4.420)%
  --(6.692,4.428)--(6.704,4.435)--(6.717,4.443)--(6.729,4.451)--(6.741,4.459)--(6.753,4.467)%
  --(6.765,4.474)--(6.777,4.482)--(6.789,4.490)--(6.801,4.498)--(6.814,4.505)--(6.826,4.513)%
  --(6.838,4.521)--(6.850,4.529)--(6.862,4.536)--(6.874,4.544)--(6.886,4.552)--(6.898,4.559)%
  --(6.911,4.567)--(6.923,4.575)--(6.935,4.583)--(6.947,4.590)--(6.959,4.598)--(6.971,4.606)%
  --(6.983,4.613)--(6.995,4.621)--(7.008,4.629)--(7.020,4.636)--(7.032,4.644)--(7.044,4.652)%
  --(7.056,4.660)--(7.068,4.667)--(7.080,4.675)--(7.092,4.682)--(7.105,4.690)--(7.117,4.698)%
  --(7.129,4.705)--(7.141,4.713)--(7.153,4.721)--(7.165,4.728)--(7.177,4.736)--(7.189,4.744)%
  --(7.202,4.751)--(7.214,4.759)--(7.226,4.767)--(7.238,4.774)--(7.250,4.782)--(7.262,4.789)%
  --(7.274,4.797)--(7.286,4.805)--(7.299,4.812)--(7.311,4.820)--(7.323,4.827)--(7.335,4.835)%
  --(7.347,4.843)--(7.359,4.850)--(7.371,4.858)--(7.384,4.865)--(7.396,4.873)--(7.408,4.881)%
  --(7.420,4.888)--(7.432,4.896)--(7.444,4.903)--(7.456,4.911)--(7.468,4.918)--(7.481,4.926)%
  --(7.493,4.934)--(7.505,4.941)--(7.517,4.949)--(7.529,4.956)--(7.541,4.964)--(7.553,4.971)%
  --(7.565,4.979)--(7.578,4.986)--(7.590,4.994)--(7.602,5.001)--(7.614,5.009)--(7.626,5.017)%
  --(7.638,5.024)--(7.650,5.032)--(7.662,5.039)--(7.675,5.047)--(7.687,5.054)--(7.699,5.062)%
  --(7.711,5.069)--(7.723,5.077)--(7.735,5.084)--(7.747,5.092)--(7.759,5.099)--(7.772,5.107)%
  --(7.784,5.114)--(7.796,5.122)--(7.808,5.129)--(7.820,5.137)--(7.832,5.144)--(7.844,5.152)%
  --(7.856,5.159)--(7.869,5.167)--(7.881,5.174)--(7.893,5.182)--(7.905,5.189)--(7.917,5.197)%
  --(7.929,5.204)--(7.941,5.212)--(7.953,5.219)--(7.966,5.226)--(7.978,5.234)--(7.990,5.241)%
  --(8.002,5.249)--(8.014,5.256)--(8.026,5.264)--(8.038,5.271)--(8.050,5.279)--(8.063,5.286)%
  --(8.075,5.294)--(8.087,5.301)--(8.099,5.308)--(8.111,5.316)--(8.123,5.323)--(8.135,5.331)%
  --(8.148,5.338)--(8.160,5.346)--(8.172,5.353)--(8.184,5.361)--(8.196,5.368)--(8.208,5.375)%
  --(8.220,5.383)--(8.232,5.390)--(8.245,5.398)--(8.257,5.405)--(8.269,5.412)--(8.281,5.420)%
  --(8.293,5.427)--(8.305,5.435)--(8.317,5.442)--(8.329,5.450)--(8.342,5.457)--(8.354,5.464)%
  --(8.366,5.472)--(8.378,5.479)--(8.390,5.487)--(8.402,5.494)--(8.414,5.501)--(8.426,5.509)%
  --(8.439,5.516)--(8.451,5.524)--(8.463,5.531)--(8.475,5.538)--(8.487,5.546)--(8.499,5.553)%
  --(8.511,5.560)--(8.523,5.568)--(8.536,5.575)--(8.548,5.583)--(8.560,5.590)--(8.572,5.597)%
  --(8.584,5.605)--(8.596,5.612)--(8.608,5.619)--(8.620,5.627)--(8.633,5.634)--(8.645,5.642)%
  --(8.657,5.649)--(8.669,5.656)--(8.681,5.664)--(8.693,5.671)--(8.705,5.678)--(8.717,5.686)%
  --(8.730,5.693)--(8.742,5.700)--(8.754,5.708)--(8.766,5.715)--(8.778,5.723)--(8.790,5.730)%
  --(8.802,5.737)--(8.814,5.745)--(8.827,5.752)--(8.839,5.759)--(8.851,5.767)--(8.863,5.774)%
  --(8.875,5.781)--(8.887,5.789)--(8.899,5.796)--(8.912,5.803)--(8.924,5.811)--(8.936,5.818)%
  --(8.948,5.825)--(8.960,5.833)--(8.972,5.840)--(8.984,5.847)--(8.996,5.855)--(9.009,5.862)%
  --(9.021,5.869)--(9.033,5.877)--(9.045,5.884)--(9.057,5.891)--(9.069,5.899)--(9.081,5.906)%
  --(9.093,5.913)--(9.106,5.921)--(9.118,5.928)--(9.130,5.935)--(9.142,5.942)--(9.154,5.950)%
  --(9.166,5.957)--(9.178,5.964)--(9.190,5.972)--(9.203,5.979)--(9.215,5.986)--(9.227,5.994)%
  --(9.239,6.001)--(9.251,6.008)--(9.263,6.016)--(9.275,6.023)--(9.287,6.030)--(9.300,6.037)%
  --(9.312,6.045)--(9.324,6.052)--(9.336,6.059)--(9.348,6.067)--(9.360,6.074)--(9.372,6.081)%
  --(9.384,6.088)--(9.397,6.096)--(9.409,6.103)--(9.421,6.110)--(9.433,6.118)--(9.445,6.125)%
  --(9.457,6.132)--(9.469,6.139)--(9.481,6.147)--(9.494,6.154)--(9.506,6.161)--(9.518,6.169)%
  --(9.530,6.176)--(9.542,6.183)--(9.554,6.190)--(9.566,6.198)--(9.578,6.205)--(9.591,6.212)%
  --(9.603,6.219)--(9.615,6.227)--(9.627,6.234)--(9.639,6.241)--(9.651,6.249)--(9.663,6.256)%
  --(9.676,6.263)--(9.688,6.270)--(9.700,6.278)--(9.712,6.285)--(9.724,6.292)--(9.736,6.299)%
  --(9.748,6.307)--(9.760,6.314)--(9.773,6.321)--(9.785,6.328)--(9.797,6.336)--(9.809,6.343)%
  --(9.821,6.350)--(9.833,6.357)--(9.845,6.365)--(9.857,6.372)--(9.870,6.379)--(9.882,6.386)%
  --(9.894,6.394)--(9.906,6.401)--(9.918,6.408)--(9.930,6.415)--(9.942,6.423)--(9.954,6.430)%
  --(9.967,6.437)--(9.979,6.444)--(9.991,6.452)--(10.003,6.459)--(10.015,6.466)--(10.027,6.473)%
  --(10.039,6.481)--(10.051,6.488)--(10.064,6.495)--(10.076,6.502)--(10.088,6.509)--(10.100,6.517)%
  --(10.112,6.524)--(10.124,6.531)--(10.136,6.538)--(10.148,6.546)--(10.161,6.553)--(10.173,6.560)%
  --(10.185,6.567)--(10.197,6.575)--(10.209,6.582)--(10.221,6.589)--(10.233,6.596)--(10.245,6.603)%
  --(10.258,6.611)--(10.270,6.618)--(10.282,6.625)--(10.294,6.632)--(10.306,6.640)--(10.318,6.647)%
  --(10.330,6.654)--(10.342,6.661)--(10.355,6.668)--(10.367,6.676)--(10.379,6.683)--(10.391,6.690)%
  --(10.403,6.697)--(10.415,6.704)--(10.427,6.712)--(10.440,6.719)--(10.452,6.726)--(10.464,6.733)%
  --(10.476,6.741)--(10.488,6.748)--(10.500,6.755)--(10.512,6.762)--(10.524,6.769)--(10.537,6.777)%
  --(10.549,6.784)--(10.561,6.791)--(10.573,6.798)--(10.585,6.805)--(10.597,6.813)--(10.609,6.820)%
  --(10.621,6.827)--(10.634,6.834)--(10.646,6.841)--(10.658,6.849)--(10.670,6.856)--(10.682,6.863)%
  --(10.694,6.870)--(10.706,6.877)--(10.718,6.885)--(10.731,6.892)--(10.743,6.899)--(10.755,6.906)%
  --(10.767,6.913)--(10.779,6.921)--(10.791,6.928)--(10.803,6.935)--(10.815,6.942)--(10.828,6.949)%
  --(10.840,6.956)--(10.852,6.964)--(10.864,6.971)--(10.876,6.978)--(10.888,6.985)--(10.900,6.992)%
  --(10.912,7.000)--(10.925,7.007)--(10.937,7.014)--(10.949,7.021)--(10.961,7.028)--(10.973,7.036)%
  --(10.985,7.043)--(10.997,7.050)--(11.009,7.057)--(11.022,7.064)--(11.034,7.071)--(11.046,7.079)%
  --(11.058,7.086)--(11.070,7.093)--(11.082,7.100)--(11.094,7.107)--(11.106,7.114)--(11.119,7.122)%
  --(11.131,7.129)--(11.143,7.136)--(11.155,7.143)--(11.167,7.150)--(11.179,7.158)--(11.191,7.165)%
  --(11.204,7.172)--(11.216,7.179)--(11.228,7.186)--(11.240,7.193)--(11.252,7.201)--(11.264,7.208)%
  --(11.276,7.215)--(11.288,7.222)--(11.301,7.229)--(11.313,7.236)--(11.325,7.244)--(11.337,7.251)%
  --(11.349,7.258)--(11.361,7.265)--(11.373,7.272)--(11.385,7.279)--(11.398,7.287)--(11.410,7.294)%
  --(11.422,7.301)--(11.434,7.308)--(11.446,7.315)--(11.458,7.322)--(11.470,7.329)--(11.482,7.337)%
  --(11.495,7.344)--(11.507,7.351)--(11.519,7.358)--(11.531,7.365)--(11.543,7.372)--(11.555,7.380)%
  --(11.567,7.387)--(11.579,7.394)--(11.592,7.401)--(11.604,7.408)--(11.616,7.415)--(11.628,7.422)%
  --(11.640,7.430)--(11.652,7.437)--(11.664,7.444)--(11.676,7.451)--(11.689,7.458)--(11.701,7.465)%
  --(11.713,7.473)--(11.725,7.480)--(11.737,7.487)--(11.749,7.494)--(11.761,7.501)--(11.773,7.508)%
  --(11.786,7.515)--(11.798,7.523)--(11.810,7.530)--(11.822,7.537)--(11.834,7.544)--(11.846,7.551)%
  --(11.858,7.558)--(11.870,7.565)--(11.883,7.573)--(11.895,7.580)--(11.907,7.587)--(11.919,7.594)%
  --(11.931,7.601)--(11.943,7.608)--(11.955,7.615)--(11.968,7.623)--(11.980,7.630)--(11.992,7.637)%
  --(12.004,7.644)--(12.016,7.651)--(12.028,7.658)--(12.040,7.665)--(12.052,7.673)--(12.065,7.680)%
  --(12.077,7.687)--(12.089,7.694)--(12.101,7.701)--(12.113,7.708)--(12.125,7.715)--(12.137,7.722)%
  --(12.149,7.730)--(12.162,7.737)--(12.174,7.744)--(12.186,7.751)--(12.198,7.758)--(12.210,7.765)%
  --(12.222,7.772)--(12.234,7.779)--(12.246,7.787)--(12.259,7.794)--(12.271,7.801)--(12.283,7.808)%
  --(12.295,7.815)--(12.307,7.822)--(12.319,7.829)--(12.331,7.837)--(12.343,7.844)--(12.356,7.851)%
  --(12.368,7.858)--(12.380,7.865)--(12.392,7.872)--(12.404,7.879)--(12.416,7.886)--(12.428,7.894)%
  --(12.440,7.901)--(12.453,7.908)--(12.465,7.915)--(12.477,7.922)--(12.489,7.929)--(12.501,7.936)%
  --(12.513,7.943)--(12.525,7.950)--(12.537,7.958)--(12.550,7.965)--(12.562,7.972)--(12.574,7.979)%
  --(12.586,7.986)--(12.598,7.993)--(12.610,8.000)--(12.622,8.007)--(12.634,8.015)--(12.647,8.022)%
  --(12.659,8.029)--(12.671,8.036)--(12.683,8.043)--(12.695,8.050)--(12.707,8.057)--(12.719,8.064)%
  --(12.732,8.071)--(12.744,8.079)--(12.756,8.086)--(12.768,8.093)--(12.780,8.100)--(12.792,8.107)%
  --(12.804,8.114)--(12.816,8.121)--(12.829,8.128)--(12.841,8.135)--(12.853,8.143)--(12.865,8.150)%
  --(12.877,8.157)--(12.889,8.164)--(12.901,8.171)--(12.913,8.178)--(12.926,8.185)--(12.938,8.192)%
  --(12.950,8.199)--(12.962,8.207)--(12.974,8.214)--(12.986,8.221)--(12.998,8.228)--(13.010,8.235)%
  --(13.023,8.242)--(13.035,8.249)--(13.047,8.256)--(13.059,8.263)--(13.071,8.270)--(13.083,8.278)%
  --(13.095,8.285)--(13.107,8.292)--(13.120,8.299)--(13.132,8.306)--(13.144,8.313)--(13.156,8.320)%
  --(13.168,8.327)--(13.180,8.334)--(13.192,8.342)--(13.204,8.349)--(13.217,8.356)--(13.229,8.363)%
  --(13.241,8.370)--(13.253,8.377)--(13.265,8.384)--(13.277,8.391)--(13.289,8.398)--(13.301,8.405)%
  --(13.314,8.413)--(13.326,8.420)--(13.338,8.427)--(13.350,8.434)--(13.362,8.441)--(13.374,8.448)%
  --(13.386,8.455)--(13.398,8.462)--(13.411,8.469)--(13.423,8.476)--(13.435,8.483)--(13.447,8.491);
\gpcolor{rgb color={0.000,0.620,0.451}}
\gpsetlinewidth{1.00}
\draw[gp path] (1.320,1.680)--(1.332,1.680)--(1.344,1.680)--(1.356,1.680)--(1.369,1.680)%
  --(1.381,1.680)--(1.393,1.680)--(1.405,1.680)--(1.417,1.680)--(1.429,1.680)--(1.441,1.680)%
  --(1.453,1.680)--(1.466,1.680)--(1.478,1.680)--(1.490,1.680)--(1.502,1.680)--(1.514,1.680)%
  --(1.526,1.680)--(1.538,1.680)--(1.550,1.680)--(1.563,1.680)--(1.575,1.680)--(1.587,1.680)%
  --(1.599,1.680)--(1.611,1.680)--(1.623,1.680)--(1.635,1.680)--(1.647,1.680)--(1.660,1.680)%
  --(1.672,1.680)--(1.684,1.680)--(1.696,1.680)--(1.708,1.680)--(1.720,1.680)--(1.732,1.680)%
  --(1.744,1.680)--(1.757,1.680)--(1.769,1.680)--(1.781,1.680)--(1.793,1.680)--(1.805,1.680)%
  --(1.817,1.680)--(1.829,1.680)--(1.841,1.680)--(1.854,1.680)--(1.866,1.680)--(1.878,1.680)%
  --(1.890,1.680)--(1.902,1.680)--(1.914,1.680)--(1.926,1.680)--(1.938,1.680)--(1.951,1.680)%
  --(1.963,1.680)--(1.975,1.680)--(1.987,1.680)--(1.999,1.680)--(2.011,1.680)--(2.023,1.680)%
  --(2.035,1.680)--(2.048,1.680)--(2.060,1.680)--(2.072,1.680)--(2.084,1.680)--(2.096,1.680)%
  --(2.108,1.680)--(2.120,1.680)--(2.133,1.680)--(2.145,1.680)--(2.157,1.680)--(2.169,1.680)%
  --(2.181,1.680)--(2.193,1.680)--(2.205,1.680)--(2.217,1.680)--(2.230,1.680)--(2.242,1.680)%
  --(2.254,1.680)--(2.266,1.680)--(2.278,1.680)--(2.290,1.680)--(2.302,1.680)--(2.314,1.680)%
  --(2.327,1.680)--(2.339,1.680)--(2.351,1.680)--(2.363,1.680)--(2.375,1.680)--(2.387,1.680)%
  --(2.399,1.680)--(2.411,1.680)--(2.424,1.680)--(2.436,1.680)--(2.448,1.680)--(2.460,1.680)%
  --(2.472,1.680)--(2.484,1.680)--(2.496,1.680)--(2.508,1.680)--(2.521,1.680)--(2.533,1.680)%
  --(2.545,1.680)--(2.557,1.680)--(2.569,1.680)--(2.581,1.680)--(2.593,1.680)--(2.605,1.680)%
  --(2.618,1.680)--(2.630,1.680)--(2.642,1.680)--(2.654,1.680)--(2.666,1.680)--(2.678,1.680)%
  --(2.690,1.680)--(2.702,1.680)--(2.715,1.680)--(2.727,1.680)--(2.739,1.680)--(2.751,1.680)%
  --(2.763,1.680)--(2.775,1.680)--(2.787,1.680)--(2.799,1.680)--(2.812,1.680)--(2.824,1.680)%
  --(2.836,1.680)--(2.848,1.680)--(2.860,1.680)--(2.872,1.680)--(2.884,1.680)--(2.897,1.680)%
  --(2.909,1.680)--(2.921,1.680)--(2.933,1.680)--(2.945,1.680)--(2.957,1.680)--(2.969,1.680)%
  --(2.981,1.680)--(2.994,1.680)--(3.006,1.680)--(3.018,1.680)--(3.030,1.680)--(3.042,1.680)%
  --(3.054,1.680)--(3.066,1.680)--(3.078,1.680)--(3.091,1.680)--(3.103,1.680)--(3.115,1.680)%
  --(3.127,1.680)--(3.139,1.680)--(3.151,1.680)--(3.163,1.680)--(3.175,1.680)--(3.188,1.680)%
  --(3.200,1.680)--(3.212,1.680)--(3.224,1.680)--(3.236,1.680)--(3.248,1.680)--(3.260,1.680)%
  --(3.272,1.680)--(3.285,1.680)--(3.297,1.680)--(3.309,1.680)--(3.321,1.680)--(3.333,1.680)%
  --(3.345,1.680)--(3.357,1.680)--(3.369,1.680)--(3.382,1.680)--(3.394,1.680)--(3.406,1.680)%
  --(3.418,1.680)--(3.430,1.680)--(3.442,1.680)--(3.454,1.680)--(3.466,1.680)--(3.479,1.680)%
  --(3.491,1.680)--(3.503,1.680)--(3.515,1.680)--(3.527,1.680)--(3.539,1.680)--(3.551,1.680)%
  --(3.563,1.680)--(3.576,1.680)--(3.588,1.680)--(3.600,1.680)--(3.612,1.680)--(3.624,1.680)%
  --(3.636,1.680)--(3.648,1.680)--(3.661,1.680)--(3.673,1.680)--(3.685,1.680)--(3.697,1.680)%
  --(3.709,1.680)--(3.721,1.680)--(3.733,1.680)--(3.745,1.680)--(3.758,1.680)--(3.770,1.680)%
  --(3.782,1.680)--(3.794,1.680)--(3.806,1.680)--(3.818,1.680)--(3.830,1.680)--(3.842,1.680)%
  --(3.855,1.680)--(3.867,1.680)--(3.879,1.680)--(3.891,1.680)--(3.903,1.680)--(3.915,1.680)%
  --(3.927,1.680)--(3.939,1.680)--(3.952,1.680)--(3.964,1.680)--(3.976,1.680)--(3.988,1.680)%
  --(4.000,1.680)--(4.012,1.680)--(4.024,1.680)--(4.036,1.680)--(4.049,1.680)--(4.061,1.680)%
  --(4.073,1.680)--(4.085,1.680)--(4.097,1.680)--(4.109,1.680)--(4.121,1.680)--(4.133,1.680)%
  --(4.146,1.680)--(4.158,1.680)--(4.170,1.680)--(4.182,1.680)--(4.194,1.680)--(4.206,1.680)%
  --(4.218,1.680)--(4.230,1.680)--(4.243,1.680)--(4.255,1.680)--(4.267,1.680)--(4.279,1.680)%
  --(4.291,1.680)--(4.303,1.680)--(4.315,1.680)--(4.327,1.680)--(4.340,1.680)--(4.352,1.680)%
  --(4.364,1.680)--(4.376,1.680)--(4.388,1.680)--(4.400,1.680)--(4.412,1.680)--(4.425,1.680)%
  --(4.437,1.680)--(4.449,1.680)--(4.461,1.680)--(4.473,1.680)--(4.485,1.680)--(4.497,1.680)%
  --(4.509,1.680)--(4.522,1.680)--(4.534,1.680)--(4.546,1.680)--(4.558,1.680)--(4.570,1.680)%
  --(4.582,1.680)--(4.594,1.680)--(4.606,1.680)--(4.619,1.680)--(4.631,1.680)--(4.643,1.680)%
  --(4.655,1.680)--(4.667,1.680)--(4.679,1.680)--(4.691,1.680)--(4.703,1.680)--(4.716,1.680)%
  --(4.728,1.680)--(4.740,1.680)--(4.752,1.680)--(4.764,1.680)--(4.776,1.680)--(4.788,1.680)%
  --(4.800,1.680)--(4.813,1.680)--(4.825,1.680)--(4.837,1.680)--(4.849,1.680)--(4.861,1.680)%
  --(4.873,1.680)--(4.885,1.680)--(4.897,1.680)--(4.910,1.680)--(4.922,1.680)--(4.934,1.680)%
  --(4.946,1.680)--(4.958,1.680)--(4.970,1.680)--(4.982,1.680)--(4.994,1.680)--(5.007,1.680)%
  --(5.019,1.680)--(5.031,1.680)--(5.043,1.680)--(5.055,1.680)--(5.067,1.680)--(5.079,1.680)%
  --(5.091,1.680)--(5.104,1.680)--(5.116,1.680)--(5.128,1.680)--(5.140,1.680)--(5.152,1.680)%
  --(5.164,1.680)--(5.176,1.680)--(5.189,1.680)--(5.201,1.680)--(5.213,1.680)--(5.225,1.680)%
  --(5.237,1.680)--(5.249,1.680)--(5.261,1.680)--(5.273,1.680)--(5.286,1.680)--(5.298,1.680)%
  --(5.310,1.680)--(5.322,1.680)--(5.334,1.680)--(5.346,1.680)--(5.358,1.680)--(5.370,1.680)%
  --(5.383,1.680)--(5.395,1.680)--(5.407,1.680)--(5.419,1.680)--(5.431,1.680)--(5.443,1.680)%
  --(5.455,1.680)--(5.467,1.680)--(5.480,1.680)--(5.492,1.680)--(5.504,1.680)--(5.516,1.680)%
  --(5.528,1.680)--(5.540,1.680)--(5.552,1.680)--(5.564,1.680)--(5.577,1.680)--(5.589,1.680)%
  --(5.601,1.680)--(5.613,1.680)--(5.625,1.680)--(5.637,1.680)--(5.649,1.680)--(5.661,1.680)%
  --(5.674,1.680)--(5.686,1.680)--(5.698,1.680)--(5.710,1.680)--(5.722,1.680)--(5.734,1.680)%
  --(5.746,1.680)--(5.758,1.680)--(5.771,1.680)--(5.783,1.680)--(5.795,1.680)--(5.807,1.680)%
  --(5.819,1.680)--(5.831,1.680)--(5.843,1.680)--(5.855,1.680)--(5.868,1.680)--(5.880,1.680)%
  --(5.892,1.680)--(5.904,1.680)--(5.916,1.680)--(5.928,1.680)--(5.940,1.680)--(5.953,1.680)%
  --(5.965,1.680)--(5.977,1.680)--(5.989,1.680)--(6.001,1.680)--(6.013,1.680)--(6.025,1.680)%
  --(6.037,1.680)--(6.050,1.680)--(6.062,1.680)--(6.074,1.680)--(6.086,1.680)--(6.098,1.680)%
  --(6.110,1.680)--(6.122,1.680)--(6.134,1.680)--(6.147,1.680)--(6.159,1.680)--(6.171,1.680)%
  --(6.183,1.680)--(6.195,1.680)--(6.207,1.680)--(6.219,1.680)--(6.231,1.680)--(6.244,1.680)%
  --(6.256,1.680)--(6.268,1.680)--(6.280,1.680)--(6.292,1.680)--(6.304,1.680)--(6.316,1.680)%
  --(6.328,1.680)--(6.341,1.680)--(6.353,1.680)--(6.365,1.680)--(6.377,1.680)--(6.389,1.680)%
  --(6.401,1.680)--(6.413,1.680)--(6.425,1.680)--(6.438,1.680)--(6.450,1.680)--(6.462,1.680)%
  --(6.474,1.680)--(6.486,1.680)--(6.498,1.680)--(6.510,1.680)--(6.522,1.680)--(6.535,1.680)%
  --(6.547,1.680)--(6.559,1.680)--(6.571,1.680)--(6.583,1.680)--(6.595,1.680)--(6.607,1.680)%
  --(6.619,1.680)--(6.632,1.680)--(6.644,1.680)--(6.656,1.680)--(6.668,1.680)--(6.680,1.680)%
  --(6.692,1.680)--(6.704,1.680)--(6.717,1.680)--(6.729,1.680)--(6.741,1.680)--(6.753,1.680)%
  --(6.765,1.680)--(6.777,1.680)--(6.789,1.680)--(6.801,1.680)--(6.814,1.680)--(6.826,1.680)%
  --(6.838,1.680)--(6.850,1.680)--(6.862,1.680)--(6.874,1.680)--(6.886,1.680)--(6.898,1.680)%
  --(6.911,1.680)--(6.923,1.680)--(6.935,1.680)--(6.947,1.680)--(6.959,1.680)--(6.971,1.680)%
  --(6.983,1.680)--(6.995,1.680)--(7.008,1.680)--(7.020,1.680)--(7.032,1.680)--(7.044,1.680)%
  --(7.056,1.680)--(7.068,1.680)--(7.080,1.680)--(7.092,1.680)--(7.105,1.680)--(7.117,1.680)%
  --(7.129,1.680)--(7.141,1.680)--(7.153,1.680)--(7.165,1.680)--(7.177,1.680)--(7.189,1.680)%
  --(7.202,1.680)--(7.214,1.680)--(7.226,1.680)--(7.238,1.680)--(7.250,1.680)--(7.262,1.680)%
  --(7.274,1.680)--(7.286,1.680)--(7.299,1.680)--(7.311,1.680)--(7.323,1.680)--(7.335,1.680)%
  --(7.347,1.680)--(7.359,1.680)--(7.371,1.680)--(7.384,1.680)--(7.396,1.680)--(7.408,1.680)%
  --(7.420,1.680)--(7.432,1.680)--(7.444,1.680)--(7.456,1.680)--(7.468,1.680)--(7.481,1.680)%
  --(7.493,1.680)--(7.505,1.680)--(7.517,1.680)--(7.529,1.680)--(7.541,1.680)--(7.553,1.680)%
  --(7.565,1.680)--(7.578,1.680)--(7.590,1.680)--(7.602,1.680)--(7.614,1.680)--(7.626,1.680)%
  --(7.638,1.680)--(7.650,1.680)--(7.662,1.680)--(7.675,1.680)--(7.687,1.680)--(7.699,1.680)%
  --(7.711,1.680)--(7.723,1.680)--(7.735,1.680)--(7.747,1.680)--(7.759,1.680)--(7.772,1.680)%
  --(7.784,1.680)--(7.796,1.680)--(7.808,1.680)--(7.820,1.680)--(7.832,1.680)--(7.844,1.680)%
  --(7.856,1.680)--(7.869,1.680)--(7.881,1.680)--(7.893,1.680)--(7.905,1.680)--(7.917,1.680)%
  --(7.929,1.680)--(7.941,1.680)--(7.953,1.680)--(7.966,1.680)--(7.978,1.680)--(7.990,1.680)%
  --(8.002,1.680)--(8.014,1.680)--(8.026,1.680)--(8.038,1.680)--(8.050,1.680)--(8.063,1.680)%
  --(8.075,1.680)--(8.087,1.680)--(8.099,1.680)--(8.111,1.680)--(8.123,1.680)--(8.135,1.680)%
  --(8.148,1.680)--(8.160,1.680)--(8.172,1.680)--(8.184,1.680)--(8.196,1.680)--(8.208,1.680)%
  --(8.220,1.680)--(8.232,1.680)--(8.245,1.680)--(8.257,1.680)--(8.269,1.680)--(8.281,1.680)%
  --(8.293,1.680)--(8.305,1.680)--(8.317,1.680)--(8.329,1.680)--(8.342,1.680)--(8.354,1.680)%
  --(8.366,1.680)--(8.378,1.680)--(8.390,1.680)--(8.402,1.680)--(8.414,1.680)--(8.426,1.680)%
  --(8.439,1.680)--(8.451,1.680)--(8.463,1.680)--(8.475,1.680)--(8.487,1.680)--(8.499,1.680)%
  --(8.511,1.680)--(8.523,1.680)--(8.536,1.680)--(8.548,1.680)--(8.560,1.680)--(8.572,1.680)%
  --(8.584,1.680)--(8.596,1.680)--(8.608,1.680)--(8.620,1.680)--(8.633,1.680)--(8.645,1.680)%
  --(8.657,1.680)--(8.669,1.680)--(8.681,1.680)--(8.693,1.680)--(8.705,1.680)--(8.717,1.680)%
  --(8.730,1.680)--(8.742,1.680)--(8.754,1.680)--(8.766,1.680)--(8.778,1.680)--(8.790,1.680)%
  --(8.802,1.680)--(8.814,1.680)--(8.827,1.680)--(8.839,1.680)--(8.851,1.680)--(8.863,1.680)%
  --(8.875,1.680)--(8.887,1.680)--(8.899,1.680)--(8.912,1.680)--(8.924,1.680)--(8.936,1.680)%
  --(8.948,1.680)--(8.960,1.680)--(8.972,1.680)--(8.984,1.680)--(8.996,1.680)--(9.009,1.680)%
  --(9.021,1.680)--(9.033,1.680)--(9.045,1.680)--(9.057,1.680)--(9.069,1.680)--(9.081,1.680)%
  --(9.093,1.680)--(9.106,1.680)--(9.118,1.680)--(9.130,1.680)--(9.142,1.680)--(9.154,1.680)%
  --(9.166,1.680)--(9.178,1.680)--(9.190,1.680)--(9.203,1.680)--(9.215,1.680)--(9.227,1.680)%
  --(9.239,1.680)--(9.251,1.680)--(9.263,1.680)--(9.275,1.680)--(9.287,1.680)--(9.300,1.680)%
  --(9.312,1.680)--(9.324,1.680)--(9.336,1.680)--(9.348,1.680)--(9.360,1.680)--(9.372,1.680)%
  --(9.384,1.680)--(9.397,1.680)--(9.409,1.680)--(9.421,1.680)--(9.433,1.680)--(9.445,1.680)%
  --(9.457,1.680)--(9.469,1.680)--(9.481,1.680)--(9.494,1.680)--(9.506,1.680)--(9.518,1.680)%
  --(9.530,1.680)--(9.542,1.680)--(9.554,1.680)--(9.566,1.680)--(9.578,1.680)--(9.591,1.680)%
  --(9.603,1.680)--(9.615,1.680)--(9.627,1.680)--(9.639,1.680)--(9.651,1.680)--(9.663,1.680)%
  --(9.676,1.680)--(9.688,1.680)--(9.700,1.680)--(9.712,1.680)--(9.724,1.680)--(9.736,1.680)%
  --(9.748,1.680)--(9.760,1.680)--(9.773,1.680)--(9.785,1.680)--(9.797,1.680)--(9.809,1.680)%
  --(9.821,1.680)--(9.833,1.680)--(9.845,1.680)--(9.857,1.680)--(9.870,1.680)--(9.882,1.680)%
  --(9.894,1.680)--(9.906,1.680)--(9.918,1.680)--(9.930,1.680)--(9.942,1.680)--(9.954,1.680)%
  --(9.967,1.680)--(9.979,1.680)--(9.991,1.680)--(10.003,1.680)--(10.015,1.680)--(10.027,1.680)%
  --(10.039,1.680)--(10.051,1.680)--(10.064,1.680)--(10.076,1.680)--(10.088,1.680)--(10.100,1.680)%
  --(10.112,1.680)--(10.124,1.680)--(10.136,1.680)--(10.148,1.680)--(10.161,1.680)--(10.173,1.680)%
  --(10.185,1.680)--(10.197,1.680)--(10.209,1.680)--(10.221,1.680)--(10.233,1.680)--(10.245,1.680)%
  --(10.258,1.680)--(10.270,1.680)--(10.282,1.680)--(10.294,1.680)--(10.306,1.680)--(10.318,1.680)%
  --(10.330,1.680)--(10.342,1.680)--(10.355,1.680)--(10.367,1.680)--(10.379,1.680)--(10.391,1.680)%
  --(10.403,1.680)--(10.415,1.680)--(10.427,1.680)--(10.440,1.680)--(10.452,1.680)--(10.464,1.680)%
  --(10.476,1.680)--(10.488,1.680)--(10.500,1.680)--(10.512,1.680)--(10.524,1.680)--(10.537,1.680)%
  --(10.549,1.680)--(10.561,1.680)--(10.573,1.680)--(10.585,1.680)--(10.597,1.680)--(10.609,1.680)%
  --(10.621,1.680)--(10.634,1.680)--(10.646,1.680)--(10.658,1.680)--(10.670,1.680)--(10.682,1.680)%
  --(10.694,1.680)--(10.706,1.680)--(10.718,1.680)--(10.731,1.680)--(10.743,1.680)--(10.755,1.680)%
  --(10.767,1.680)--(10.779,1.680)--(10.791,1.680)--(10.803,1.680)--(10.815,1.680)--(10.828,1.680)%
  --(10.840,1.680)--(10.852,1.680)--(10.864,1.680)--(10.876,1.680)--(10.888,1.680)--(10.900,1.680)%
  --(10.912,1.680)--(10.925,1.680)--(10.937,1.680)--(10.949,1.680)--(10.961,1.680)--(10.973,1.680)%
  --(10.985,1.680)--(10.997,1.680)--(11.009,1.680)--(11.022,1.680)--(11.034,1.680)--(11.046,1.680)%
  --(11.058,1.680)--(11.070,1.680)--(11.082,1.680)--(11.094,1.680)--(11.106,1.680)--(11.119,1.680)%
  --(11.131,1.680)--(11.143,1.680)--(11.155,1.680)--(11.167,1.680)--(11.179,1.680)--(11.191,1.680)%
  --(11.204,1.680)--(11.216,1.680)--(11.228,1.680)--(11.240,1.680)--(11.252,1.680)--(11.264,1.680)%
  --(11.276,1.680)--(11.288,1.680)--(11.301,1.680)--(11.313,1.680)--(11.325,1.680)--(11.337,1.680)%
  --(11.349,1.680)--(11.361,1.680)--(11.373,1.680)--(11.385,1.680)--(11.398,1.680)--(11.410,1.680)%
  --(11.422,1.680)--(11.434,1.680)--(11.446,1.680)--(11.458,1.680)--(11.470,1.680)--(11.482,1.680)%
  --(11.495,1.680)--(11.507,1.680)--(11.519,1.680)--(11.531,1.680)--(11.543,1.680)--(11.555,1.680)%
  --(11.567,1.680)--(11.579,1.680)--(11.592,1.680)--(11.604,1.680)--(11.616,1.680)--(11.628,1.680)%
  --(11.640,1.680)--(11.652,1.680)--(11.664,1.680)--(11.676,1.680)--(11.689,1.680)--(11.701,1.680)%
  --(11.713,1.680)--(11.725,1.680)--(11.737,1.680)--(11.749,1.680)--(11.761,1.680)--(11.773,1.680)%
  --(11.786,1.680)--(11.798,1.680)--(11.810,1.680)--(11.822,1.680)--(11.834,1.680)--(11.846,1.680)%
  --(11.858,1.680)--(11.870,1.680)--(11.883,1.680)--(11.895,1.680)--(11.907,1.680)--(11.919,1.680)%
  --(11.931,1.680)--(11.943,1.680)--(11.955,1.680)--(11.968,1.680)--(11.980,1.680)--(11.992,1.680)%
  --(12.004,1.680)--(12.016,1.680)--(12.028,1.680)--(12.040,1.680)--(12.052,1.680)--(12.065,1.680)%
  --(12.077,1.680)--(12.089,1.680)--(12.101,1.680)--(12.113,1.680)--(12.125,1.680)--(12.137,1.680)%
  --(12.149,1.680)--(12.162,1.680)--(12.174,1.680)--(12.186,1.680)--(12.198,1.680)--(12.210,1.680)%
  --(12.222,1.680)--(12.234,1.680)--(12.246,1.680)--(12.259,1.680)--(12.271,1.680)--(12.283,1.680)%
  --(12.295,1.680)--(12.307,1.680)--(12.319,1.680)--(12.331,1.680)--(12.343,1.680)--(12.356,1.680)%
  --(12.368,1.680)--(12.380,1.680)--(12.392,1.680)--(12.404,1.680)--(12.416,1.680)--(12.428,1.680)%
  --(12.440,1.680)--(12.453,1.680)--(12.465,1.680)--(12.477,1.680)--(12.489,1.680)--(12.501,1.680)%
  --(12.513,1.680)--(12.525,1.680)--(12.537,1.680)--(12.550,1.680)--(12.562,1.680)--(12.574,1.680)%
  --(12.586,1.680)--(12.598,1.680)--(12.610,1.680)--(12.622,1.680)--(12.634,1.680)--(12.647,1.680)%
  --(12.659,1.680)--(12.671,1.680)--(12.683,1.680)--(12.695,1.680)--(12.707,1.680)--(12.719,1.680)%
  --(12.732,1.680)--(12.744,1.680)--(12.756,1.680)--(12.768,1.680)--(12.780,1.680)--(12.792,1.680)%
  --(12.804,1.680)--(12.816,1.680)--(12.829,1.680)--(12.841,1.680)--(12.853,1.680)--(12.865,1.680)%
  --(12.877,1.680)--(12.889,1.680)--(12.901,1.680)--(12.913,1.680)--(12.926,1.680)--(12.938,1.680)%
  --(12.950,1.680)--(12.962,1.680)--(12.974,1.680)--(12.986,1.680)--(12.998,1.680)--(13.010,1.680)%
  --(13.023,1.680)--(13.035,1.680)--(13.047,1.680)--(13.059,1.680)--(13.071,1.680)--(13.083,1.680)%
  --(13.095,1.680)--(13.107,1.680)--(13.120,1.680)--(13.132,1.680)--(13.144,1.680)--(13.156,1.680)%
  --(13.168,1.680)--(13.180,1.680)--(13.192,1.680)--(13.204,1.680)--(13.217,1.680)--(13.229,1.680)%
  --(13.241,1.680)--(13.253,1.680)--(13.265,1.680)--(13.277,1.680)--(13.289,1.680)--(13.301,1.680)%
  --(13.314,1.680)--(13.326,1.680)--(13.338,1.680)--(13.350,1.680)--(13.362,1.680)--(13.374,1.680)%
  --(13.386,1.680)--(13.398,1.680)--(13.411,1.680)--(13.423,1.680)--(13.435,1.680)--(13.447,1.680);
\gpcolor{color=gp lt color border}
\draw[gp path] (1.320,8.631)--(1.320,0.985)--(13.447,0.985)--(13.447,8.631)--cycle;
%% coordinates of the plot area
\gpdefrectangularnode{gp plot 1}{\pgfpoint{1.320cm}{0.985cm}}{\pgfpoint{13.447cm}{8.631cm}}
\end{tikzpicture}
%% gnuplot variables

	\caption{Gráfico da relação~\eqref{Eq:Rel_pot_quim_renorm_mom_fermi} para $m = 100$.}
\end{figure*}
Caso $\mu_R^2 > m^2$, temos uma relação simples para $\mu_R$:
\begin{equation}
	\mu_R = \sqrt{p_F^2 + m^2}, \quad \textrm{se}~ \mu_R^2 > m^2.
\end{equation}
%
No entanto, para qualquer valor de $\mu_R^2$ menor que $m^2$, não é possível inverter a relação~\eqref{Eq:Rel_pot_quim_renorm_mom_fermi}: todo valor de $\mu_R$ tal que $\mu_R^2 < m^2$ está associado a $p_F = 0$, e não é possível determinar uma inversa pois o elemento $0$ dos valores de $p_F$ levaria a diversos elementos dos valores de $\mu_R$ e, portanto, a relação inversa para $\mu_R^2 < m^2$ não é uma função em tal intervalo.\footnote{Uma interpretação para isso é a de que existe um valor mínimo $\mu_R = m$ para o potencial químico renormalizado.}

Por outro lado, utilizando as Equações~\eqref{Eq:Rel_Dens_Mom_Fermi_NJL} e~\eqref{Eq:Rel_pot_quim_renorm_mom_fermi} podemos escrever
\begin{equation}
	\sqrt[3]{\frac{3\pi^2}{n_f} \rho_B} = \sqrt{\mu_R^2 - m^2}\theta(\mu_R^2 - m^2),
\end{equation}
%
onde $\theta(\mu_R^2 - m^2)$ garante que o lado direito seja estritamente positivo, ou seja
\begin{equation}
	\rho_B > 0,
\end{equation}
%
o que é perfeitamente razoável. Consequentemente temos que a expressão para $\mu_R$ é dada por
\begin{equation}
	\mu_R = \sqrt{p_F^2 + m^2}.
\end{equation}
%
Isto significa que, se rodarmos em $\mu_R$, temos que $p_F$ é zero para $\mu_R^2 < m^2$, o que resulta no mesmo valor para $m$ (resolvendo a Eq.~\eqref{Eq:Eq_Gap_Buballa_1}) para qualquer valor de $\mu_R$ em tal intervalo. Se acontecer o mesmo com o potencial termodinâmico\footnote{Ver se é esse o caso.} $\tilde{\omega}$, não tem problema algum, pois estaríamos calculando ``o mesmo ponto'' para as equações de estado.

%%%%%%%%%%%%%%%%%%%%%%%%%%%%%%%%%%%%
\subsection{Potencial termodinâmico}
%%%%%%%%%%%%%%%%%%%%%%%%%%%%%%%%%%%%

Para calcular as grandezas em que estamos interessados ($p$, $\varepsilon$), precisamos calcular $\tilde{\omega}$. Para calcular tal grandeza, precisamos calcular o valor de $m_{\rm{vac}}$, Eq.~\eqref{Eq:Def_omega_tilde}. Esse valor é dado de forma que\cite{Buballa1996}
\begin{quote}
	$m_{\rm{vac}}$ is the dynamical mass which minimizes the thermodynamic potential of the vacuum.
\end{quote}
%
isto é\footnote{ou acredito que seja}, $m_{\rm{vac}}$ deve minimizar o potencial definido pela Eq.~\ref{Eq:Def_pot_termo} para o caso do vácuo\footnote{No Report do Buballa, também há a menção a minimizar o potencial termodinâmico, porém não há o termo $\omega_m^{\rm{med}}$. Acredito que devido à função degrau, esse termo seja zero na forma que temos aqui.}. Podemos determinar o valor que minimiza o potencial tomando a derivada em relação a $m$ e igualando-a a zero:
\begin{align}
	\frac{d}{dm} \omega(\mu = 0; m, \mu_R = 0) &= \frac{d}{dm}\omega_m^{\rm{(vac)}} \\
	&\phantom{=} + \frac{d}{dm}\omega_m^{\rm{(med)}}(\mu_R) \nonumber \\
	&\phantom{=} + \frac{d}{dm}\frac{(m - m_0)^2}{4G_S} \nonumber \\
	&= 0.\nonumber
\end{align}
%
Como $\mu_R = 0$ e $E = \sqrt{p^2+m^2}$, da expressão~\eqref{Eq:Def_omega_med} temos que o segundo termo é nulo (por $\mu_R$ ser zero e pelo fato de que a função degrau será zero, pois $p_F$ é zero se $\mu_R$, Eq.~\eqref{Eq:Rel_pot_quim_renorm_mom_fermi}).

O primeiro termo pode ser calculado através da fórmula de Leibniz\footnote{Se os limites são constantes, o lado direito se resume à integral.}
\begin{equation}
\begin{split}
	\frac{d}{dx} \left(\int_{a(x)}^{b(x)} f(x,t) dt\right) =&~ f(x, b(x))b'(x) - f(x, a(x))a'(x) \\
	&+ \int_{a(x)}^{b(x)}\frac{\partial}{\partial x}f(x,t) dt
\end{split}
\end{equation}
%
resultando em
\begin{align}
	\frac{d}{dm}\omega_m^{\rm{(vac)}} &= -2 n_f n_c\frac{d}{dm}\int \frac{d^3p}{(2\pi)^3} \sqrt{p^2 + m^2} \\
	&= - \frac{n_f n_c}{\pi^2} \int \frac{\partial}{\partial m} p^2\sqrt{p^2+m^2} dp \\
	&= - \frac{n_f n_c}{\pi^2} \int_0^\Lambda \frac{mp^2}{\sqrt{p^2+m^2}} dp \\
	&= - \frac{n_f n_c}{\pi^2} \;m [F_0(m, \Lambda) - F_0(m,0)]
\end{align}
%
onde usamos as relações~\eqref{Eq:Int_d3p_to_dp} e \eqref{Eq:Def_F0},~\eqref{Eq:Def_F0_integrado}.

Finalmente, o terceiro termo é dado por
\begin{equation}
	\frac{d}{dm} \frac{(m - m_0)^2}{4G_S} = \frac{2 (m - m_0)}{4G_S} = \frac{m - m_0}{2G_S}.
\end{equation}
%
Consequentemente,
\begin{align}
	\frac{d}{dm} \omega(\mu = 0; m, \mu_R = 0) &= - \frac{n_f n_c}{\pi^2} \;m [F_0(m, \Lambda) - F_0(m,0)] \\
	&\phantom{=}~+ \frac{m - m_0}{2G_S} \nonumber \\
	&= 0.
\end{align}
%
multiplicando por $2G_S$, obtemos uma equação equivalente:
\begin{equation}\label{Eq:Calculo_m_vac}
	m - m_0 - 2G_S\frac{n_f n_c}{\pi^2} \;m [F_0(m, \Lambda) - F_0(m,0)] = 0.
\end{equation}
%
Devemos resolver tal equação através de um método para encontrar zeros de função, tal como é feito para a equação do gap.

%%%%%%%%%%%%%%%%%%%%%%%%%%%%%%%%%%%%%%%%%%%%%
\subsection{Determinação de $\tilde{\omega}$}
%%%%%%%%%%%%%%%%%%%%%%%%%%%%%%%%%%%%%%%%%%%%%

\begin{fullwidth}

Para que possamos calcular os valores da pressão $p$ e da energia $\varepsilon$ através das Equações~\eqref{Eq:Energia_omega_tilde} e~\eqref{Eq:Pressao_omega_tilde}, necessitamos de $\tilde{\omega}$. Utilizando as Equações~\eqref{Eq:Def_omega_tilde}, \eqref{Eq:Def_omega_vac}, e~\eqref{Eq:Def_omega_med}, temos:
\begin{equation}
\begin{split}
\tilde{\omega} =&~ \left\{-2 n_f n_c \left[\int\frac{d^3p}{(2\pi)^3} E_p + \int\frac{d^3p}{(2\pi)^3}(\mu_R - E_p)\theta(p_F - p)\right] + \frac{(m-m_0)^2}{4G_S}\right\} \\
&-\left\{-2 n_f n_c \left[\int\frac{d^3p}{(2\pi)^3}E_p|_{m=m_{\rm{vac}}} + \int\frac{d^3p}{(2\pi)^3}(\underbrace{\mu_R}_{0} - E_p|_{m = m_{\rm{vac}}}) \theta(p_F - p)\right] + \frac{(m_{\rm{vac}} - m_0)^2}{4G_S}\right\}
\end{split}
\end{equation}
%
Utilizando a expressão~\eqref{Eq:Int_d3p_to_dp}, podemos reescrever tal expressão como (separando a integral de $\mu_R$ e explicitando os limites)
\begin{equation}
\begin{split}
\tilde{\omega} =&~ \left\{- \frac{n_f n_c}{\pi^2} \left[\int_0^\Lambda p^2E_p \; dp - \int_0^\Lambda p^2 E_p \theta(p_F - p)\;dp + \int_0^\Lambda p^2 \mu_R \theta(p_F - p)\;dp \right] + \frac{(m-m_0)^2}{4G_S}\right\} \\
&-\left\{- \frac{n_f n_c}{\pi^2} \left[\int_0^\Lambda p^2 E_p|_{m=m_{\rm{vac}}} \; dp - \int_0^\Lambda p^2 E_p|_{m = m_{\rm{vac}}} \theta(p_F - p) \; dp\right] + \frac{(m_{\rm{vac}} - m_0)^2}{4G_S}\right\}
\end{split}
\end{equation}

Na expressão acima, verificamos a existência de dois pares de integrais idênticas --~a menos de uma função degrau $\theta$~-- sendo que estamos calculando a diferença entre elas. Tal resultado será, claramente, zero. Isso não será verdade, no entanto, para $p_F > p$, pois nesse caso a integral que contém a função degrau será zerada. O efeito líquido disso é o de calcularmos somente a integral sem a função $\theta$, porém no intervalo $[p_F, \Lambda]$. Assim,
\begin{equation}
\begin{split}
\tilde{\omega} =&~ \left\{- \frac{n_f n_c}{\pi^2} \left[\int_{p_F}^\Lambda p^2E_p \; dp + \int_0^\Lambda p^2 \mu_R \theta(p_F - p)\;dp \right] + \frac{(m-m_0)^2}{4G_S}\right\} \\
&-\left\{- \frac{n_f n_c}{\pi^2} \left[\int_{p_F}^\Lambda p^2 E_p|_{m=m_{\rm{vac}}} \; dp \right] + \frac{(m_{\rm{vac}} - m_0)^2}{4G_S}\right\}
\end{split}
\end{equation}
\end{fullwidth}
%
Utilizando a Equação~\eqref{Eq:Eq_Gap_NJL}, podemos escrever
\begin{equation}
	\frac{(m - m_0)^2}{4G_S} = G_S \rho_s^2
\end{equation}
%
e como $\mu_R$ não depende do momento $p$,
\begin{equation}
	\int_0^\Lambda \mu_R \theta(p_F - p) \; dp = \mu_R \int_{0}^{p_F} \; dp = \mu_R\frac{p_F^3}{3}.
\end{equation}
%
Portanto,
\begin{equation}
\begin{split}
\tilde{\omega} =&~ \left\{- \frac{n_f n_c}{\pi^2} \left[\int_{p_F}^\Lambda p^2E_p \; dp \right] - \frac{n_f n_c}{\pi^2} \frac{p_F^3}{3} + G_S\rho_s^2\right\} \\
&-\left\{- \frac{n_f n_c}{\pi^2} \left[\int_{p_F}^\Lambda p^2 E_p|_{m=m_{\rm{vac}}} \; dp \right] + G_S(\rho_s|_{m = m_{\rm{vac}}})^2\right\}.
\end{split}
\end{equation}

Utilizando a definição de $E_p$, Equação~\eqref{Eq:Def_E}, podemos calcular a integral utilizando
\begin{align}
	\int k^2 \sqrt{k^2 + m^2} \;dk &= \frac{1}{4}\left[k E^3 - \frac{1}{2} m^2 k E - \frac{1}{2} m^4\ln\left(\frac{k+E}{m}\right)\right] \\
	&\equiv F_E(m, k),
\end{align}
%
onde $E = \sqrt{k^2 + m^2}$, o que nos leva a
\begin{equation}\label{Eq:Pot_termodinamico_1}
\begin{split}
\tilde{\omega} =&~ \left\{- \frac{n_f n_c}{\pi^2} [F_E(m,\Lambda) - F_E(m, p_F)] - n_c\mu_R \rho_B + G_S\rho_s^2\right\} \\
&-\left\{- \frac{n_f n_c}{\pi^2} [F_E(m_{\rm{vac}}, \Lambda) - F_E(m_{\rm{vac}}, p_F)] + G_S(\rho_s|_{m = m_{\rm{vac}}})^2\right\},
\end{split}
\end{equation}
%
onde usamos a Equação~\eqref{Eq:Rel_Dens_Mom_Fermi_NJL} para escrever o termo envolvendo a densidade bariônica $\rho_B$.

O resultado  para $\omega_{\rm{MF}}(0;m_{\rm{vac}},0)$ utilizando o valor obtido através da Eq.~\eqref{Eq:Calculo_m_vac} será uma constante, o que pode ser entendido verificando que a expressão para o potencial termodinâmico nessas condições é calculada a partir de constantes:
\begin{equation}
	\omega_{\rm{MF}}(0;m_{\rm{vac}},0) = -\frac{n_f n_c}{\pi^2}\left[F_E(m_{\rm{vac}}, \Lambda) - F_E(m_{\rm{vac}}, p_F)\right] + G_S(\rho_s|_{m=m_{\rm{vac}}})^2.
\end{equation}

A expressão acima difere do programa que a Débora me passou como referência. Em tal programa, o potencial termodinâmico tem uma forma análoga ao caso eNJL:
\begin{subequations}\label{Eq:Pot_termodinamico_2}
\begin{align}
	\varepsilon &= \tilde{\omega} + n_c\mu\rho_B = \varepsilon_{\rm{kin}} + m_0\rho_s - 2G_S\rho_s^2 \\
	\varepsilon_0 &= \tilde{\omega}_0 + n_c\mu\rho_B = \varepsilon_{\rm{kin}}^0 + m_0\rho_0^s - 2G_S(\rho_0^s)^2 \\
	\varepsilon_{\rm{kin}} &= -\frac{n_c}{\pi^2} [F_2(m, \Lambda) - F_2(m, p_F)] \\
	\varepsilon_{\rm{kin}}^0 &= -\frac{n_c}{\pi^2} [F_2(m, \Lambda) - F_2(m, 0)] \\
	\rho_0^s &= \rho_s|_{m = m_{\rm{vac}}}
\end{align}
\end{subequations}
%
onde, como na Equação~\eqref{Eq:Energia_kin},
\begin{equation}
	F_2(m, p) = \frac{1}{8}\Big((2p^3 - 3M_i^2p)\sqrt{p^2 + M_i^2} + 3M_i^4\ln\frac{p + \sqrt{p^2 + M_i^2}}{M_i}\Big).
\end{equation}
%
Isso se deve ao fato de que a densidade de energia pode ser calculada a partir do potencial termodinâmico através de mais que uma relação. Possivelmente essa esteja relacionada à derivada do potencial termodinâmico em relação à temperatura, ou ao inverso da temperatura. Ver Avancini~\cite{Avancini2004} e~\cite{Avancini2006}.

%%%%%%%%%%%%%%%%%%%%%%%%%%%%%%%%
\subsection{Análise Dimensional}
%%%%%%%%%%%%%%%%%%%%%%%%%%%%%%%%

Assim como no caso eNJL, assumimos que
\begin{itemize}
	\item $[\rho_s] = [\rho_B] = \rm{fm}^{-3}$
	\item $[E] = [p] = [m] = [\mu] = p_F = \Lambda = \mu_R = \rm{Mev}$.
\end{itemize}
%
Verificando as dimensões na lagrangiana, temos
\begin{equation}
	\underbrace{\mathcal{L}}_{\mathcal{\rm{MeV}/\rm{fm}^3}} = \underbrace{\bar{\psi}}_{\rm{fm}^{3/2}}(i\gamma^\mu\underbrace{\partial_\mu}_{\rm{fm}^{-1}} - \underbrace{m_0}_{\rm{MeV}})\underbrace{\psi}_{\rm{fm}^{3/2}} + G_S [ \underbrace{(\bar{\psi}\psi)^2}_{\rm{fm}^{-6}} + \underbrace{(\bar{\psi}i\gamma_5\vec{\tau}\psi)^2}_{\rm{fm}^{-6}}].
\end{equation}
%
A constante $G_S$ tem dimensão de inverso de massa$^{-2}$ em Buballa\cite{Buballa1996}, porém adotamos que 
\begin{equation}
	[G_S] = \rm{fm}^{2}
\end{equation}
%
para que tenhamos uma dimensão igual ao caso eNJL. Assim o valor de $G_S$ da referência deve ser ajustado de acordo com
\begin{equation}
	\underbrace{G_S}_{\textrm{Buballa}} \cdot (\hbar c)^2 = \underbrace{G_S}_{\textrm{prog.}}.
\end{equation}
%
Além disso, para que dimensão da densidade lagrangiana seja adequada, temos
\begin{equation}
	\mathcal{L} = \bar{\psi}(\hbar c \cdot i\gamma^\mu\partial_\mu - m_0)\psi + \hbar c \cdot G_S[(\bar{\psi}\psi)^2 + (\bar{\psi}i\gamma_5\vec{\tau}\psi)^2].
\end{equation}
%
A relação entre a densidade bariônica e o momento de Fermi também precisa ter sua dimensão corrigida:
\begin{equation}
	\rho_B = (\hbar c)^{-3} \cdot \frac{n_f}{3\pi^2}p_F^3.
\end{equation}

Temos também que para que a expressão~\eqref{Eq:Dens_Escalar_NJL_Gv_0} para a densidade escalar tenha as dimensões corretas, devemos multiplicá-la por $(\hbar c)^{-3}$. Além disso, na Eq.~\eqref{Eq:Eq_Gap_Buballa_1} devemos multiplicar o termo proporcional a $G_S$ por $\hbar c$ de forma a obter a dimensão correta (MeV) para que possamos somar com a massa.

Para encontrar $m_{\rm{vac}}$, precisamos resolver
\begin{equation}
	m - m_0 - 2G_S\frac{n_f n_c}{\pi^2} \;m [F_0(m, \Lambda) - F_0(m,0)] = 0.
\end{equation}
%
Como $[m] = [m_0] = \rm{MeV}$, o último termo também deve estar em MeV. A dimensão de $F_0$ é $\rm{MeV}^2$, enquanto $[G_S] = \rm{fm}^2$, o que resulta -- levando em conta que o termo contém um $m$ no numerador -- em $\rm{MeV}^3\cdot\rm{fm}^2$. Para corrigir a dimensão, devemos multiplicar tal termo por $(\hbar c)^{-2}$:
\begin{equation}
	m - m_0 - 2G_S\frac{n_f n_c}{\pi^2} \;m [F_0(m, \Lambda) - F_0(m,0)] \cdot (\hbar c)^{-2} = 0.
\end{equation}

Para o cálculo do potencial termodinâmico, segundo a Equação~\ref{Eq:Pot_termodinamico_1}, temos que
\begin{align}
	[F_E] &= \left[\int_0^\Lambda p^2\sqrt{p^2+m^2} \;dp\right] = \rm{MeV}^2, \\
	[\omega] &= \rm{MeV}/\rm{fm}^3,
\end{align}
%
logo,
\begin{align}
	\tilde{\omega} &= \left\{- (\hbar c)^{-3} \cdot [F_E(m, \Lambda) - F_E(m, p_F)] - n_fn_c\mu_R\rho_B + (\hbar c) \cdot G_S\rho_S^2\right\} - \omega_0 \\
	\omega_0 &= - (\hbar c)^{-3} \cdot [F_E(m_{\rm{vac}}, \Lambda) - F_E(m_{\rm{vac}}, p_F)] - n_fn_c\mu_R\rho_B + (\hbar c) \cdot G_S(\rho_S|_{m = m_{\rm{vac}}})^2
\end{align}

Para o cálculo do potencial termodinâmico através das expressões~\eqref{Eq:Pot_termodinamico_2}, temos
\begin{subequations}
\begin{align}
	\varepsilon &= \tilde{\omega} + n_c\mu\rho_B = \varepsilon_{\rm{kin}} + m_0\rho_s - (\hbar c) \cdot 2G_S\rho_s^2 - \varepsilon_0\\
	\varepsilon_0 &= \tilde{\omega}_0 + n_c\mu\rho_B = \varepsilon_{\rm{kin}}^0 + m_0\rho_0^s - (\hbar c) \cdot 2G_S(\rho_0^s)^2 \\
	\varepsilon_{\rm{kin}} &= -(\hbar c)^{-3} \cdot \frac{n_c}{\pi^2} [F_2(m, \Lambda) - F_2(m, p_F)] \\
	\varepsilon_{\rm{kin}}^0 &= -(\hbar c)^{-3} \cdot \frac{n_c}{\pi^2} [F_2(m, \Lambda) - F_2(m, 0)]
\end{align}
\end{subequations}

%%%%%%%%%%%%%%%%%%%%
\section{Resultados}
%%%%%%%%%%%%%%%%%%%%

Os resultados abaixo aparentemente estão adequados até a determinação da massa, densidade escalar. Consigo reproduzir o valor da massa $m_{\rm{vac}}$ do Buballa~\cite{Buballa1996}, e os gráficos da massa efetiva e da densidade escalar se comportam de acordo com as expectativas (são parecidos com o caso de hádrons, quando $m_0 \neq 0$ a massa são vai a zero, e a densidade escalar tende a verticalizar próximo do valor da densidade máxima no caso $m_0 = 0$.).

No entanto, tudo o que é calculado após o potencial termodinâmico (incluindo o próprio potencial) estão estranhos. O valor de $\omega_0$ é muito grande e negativo, crescendo de maneira aproximadamente linear. O valor do próprio potencial não resulta em algo similar ao caso de hádrons. A energia por partícula está com uma escala estranha também, sem falar que o gráfico está muito estranho, com um mínimo local, um máximo e depois desanda a baixar.

Em tese, o que deveria ser revisto é o próprio potencial termodinâmico. Em primeiro lugar, ele não reproduz a Figura 1 do Buballa\cite{Buballa1996}. Isso seria o mínimo. De qualquer forma, o cálculo do Buballa varia $\mu_R$, acredito. Então fica até difícil de comparar. Portanto, como primeiro passo, o programa tem que passar a calcular usando o potencial químico (ou o que for), como no Buballa.

Na Figura~\ref{Fig:pot_term_analysys_NJL-Buballa_Set_1} mostramos o potencial termodinâmico calculado para $\mu = \mu_r = p_F = \rho_B = 0$, variando $m$ arbitrariamente. O resultado é exatamente igual ao apresentado no Buballa. No entanto, para $\mu \neq 0$, os resultados não conferem. Aparentemente, o cálculo do potencial termodinâmico não funciona direito pois a variação é muito maior do que deveria (o comportamento geral do cálculo parece correto, mas deveria ser menos sensível à variação do parâmetro). Testei com o sem o potencial termodinâmico do vácuo e o resultado na variação do valor do potencial para $m \to 0$ entre os casos $\mu = 0$ e $\mu = 350$ MeV é mais ou menos o mesmo (vi no gráfico, não verifiquei o valor numericamente, só estimei), então o problema deve estar no cálculo do termo que não é o vácuo.\footnote{Na verdade pode ser também, tinha um termo multiplicado por zero para testes. De qq forma, o negócio só tá funcionando direito pra zero.}

\begin{figure*}
	\begin{tikzpicture}[gnuplot]
%% generated with GNUPLOT 5.0p2 (Lua 5.2; terminal rev. 99, script rev. 100)
%% Wed Apr  6 19:50:08 2016
\path (0.000,0.000) rectangle (14.000,9.000);
\gpcolor{color=gp lt color border}
\gpsetlinetype{gp lt border}
\gpsetdashtype{gp dt solid}
\gpsetlinewidth{1.00}
\draw[gp path] (1.504,0.985)--(1.684,0.985);
\draw[gp path] (13.447,0.985)--(13.267,0.985);
\node[gp node right] at (1.320,0.985) {$0$};
\draw[gp path] (1.504,2.077)--(1.684,2.077);
\draw[gp path] (13.447,2.077)--(13.267,2.077);
\node[gp node right] at (1.320,2.077) {$200$};
\draw[gp path] (1.504,3.170)--(1.684,3.170);
\draw[gp path] (13.447,3.170)--(13.267,3.170);
\node[gp node right] at (1.320,3.170) {$400$};
\draw[gp path] (1.504,4.262)--(1.684,4.262);
\draw[gp path] (13.447,4.262)--(13.267,4.262);
\node[gp node right] at (1.320,4.262) {$600$};
\draw[gp path] (1.504,5.354)--(1.684,5.354);
\draw[gp path] (13.447,5.354)--(13.267,5.354);
\node[gp node right] at (1.320,5.354) {$800$};
\draw[gp path] (1.504,6.446)--(1.684,6.446);
\draw[gp path] (13.447,6.446)--(13.267,6.446);
\node[gp node right] at (1.320,6.446) {$1000$};
\draw[gp path] (1.504,7.539)--(1.684,7.539);
\draw[gp path] (13.447,7.539)--(13.267,7.539);
\node[gp node right] at (1.320,7.539) {$1200$};
\draw[gp path] (1.504,8.631)--(1.684,8.631);
\draw[gp path] (13.447,8.631)--(13.267,8.631);
\node[gp node right] at (1.320,8.631) {$1400$};
\draw[gp path] (1.504,0.985)--(1.504,1.165);
\draw[gp path] (1.504,8.631)--(1.504,8.451);
\node[gp node center] at (1.504,0.677) {$0$};
\draw[gp path] (2.698,0.985)--(2.698,1.165);
\draw[gp path] (2.698,8.631)--(2.698,8.451);
\node[gp node center] at (2.698,0.677) {$100$};
\draw[gp path] (3.893,0.985)--(3.893,1.165);
\draw[gp path] (3.893,8.631)--(3.893,8.451);
\node[gp node center] at (3.893,0.677) {$200$};
\draw[gp path] (5.087,0.985)--(5.087,1.165);
\draw[gp path] (5.087,8.631)--(5.087,8.451);
\node[gp node center] at (5.087,0.677) {$300$};
\draw[gp path] (6.281,0.985)--(6.281,1.165);
\draw[gp path] (6.281,8.631)--(6.281,8.451);
\node[gp node center] at (6.281,0.677) {$400$};
\draw[gp path] (7.476,0.985)--(7.476,1.165);
\draw[gp path] (7.476,8.631)--(7.476,8.451);
\node[gp node center] at (7.476,0.677) {$500$};
\draw[gp path] (8.670,0.985)--(8.670,1.165);
\draw[gp path] (8.670,8.631)--(8.670,8.451);
\node[gp node center] at (8.670,0.677) {$600$};
\draw[gp path] (9.864,0.985)--(9.864,1.165);
\draw[gp path] (9.864,8.631)--(9.864,8.451);
\node[gp node center] at (9.864,0.677) {$700$};
\draw[gp path] (11.058,0.985)--(11.058,1.165);
\draw[gp path] (11.058,8.631)--(11.058,8.451);
\node[gp node center] at (11.058,0.677) {$800$};
\draw[gp path] (12.253,0.985)--(12.253,1.165);
\draw[gp path] (12.253,8.631)--(12.253,8.451);
\node[gp node center] at (12.253,0.677) {$900$};
\draw[gp path] (13.447,0.985)--(13.447,1.165);
\draw[gp path] (13.447,8.631)--(13.447,8.451);
\node[gp node center] at (13.447,0.677) {$1000$};
\draw[gp path] (1.504,8.631)--(1.504,0.985)--(13.447,0.985)--(13.447,8.631)--cycle;
\node[gp node center,rotate=-270] at (0.246,4.808) {$\tilde{\omega} (\rm{MeV}/\rm{fm}^3)$};
\node[gp node center] at (7.475,0.215) {$m$ (MeV)};
\node[gp node left] at (2.972,8.297) {"therm1.dat"};
\gpcolor{rgb color={0.580,0.000,0.827}}
\gpsetpointsize{4.00}
\gppoint{gp mark 1}{(1.516,1.720)}
\gppoint{gp mark 1}{(1.528,1.720)}
\gppoint{gp mark 1}{(1.540,1.720)}
\gppoint{gp mark 1}{(1.552,1.720)}
\gppoint{gp mark 1}{(1.564,1.720)}
\gppoint{gp mark 1}{(1.576,1.720)}
\gppoint{gp mark 1}{(1.588,1.720)}
\gppoint{gp mark 1}{(1.600,1.720)}
\gppoint{gp mark 1}{(1.612,1.719)}
\gppoint{gp mark 1}{(1.624,1.719)}
\gppoint{gp mark 1}{(1.636,1.719)}
\gppoint{gp mark 1}{(1.647,1.719)}
\gppoint{gp mark 1}{(1.659,1.718)}
\gppoint{gp mark 1}{(1.671,1.718)}
\gppoint{gp mark 1}{(1.683,1.718)}
\gppoint{gp mark 1}{(1.695,1.717)}
\gppoint{gp mark 1}{(1.707,1.717)}
\gppoint{gp mark 1}{(1.719,1.716)}
\gppoint{gp mark 1}{(1.731,1.716)}
\gppoint{gp mark 1}{(1.743,1.715)}
\gppoint{gp mark 1}{(1.755,1.715)}
\gppoint{gp mark 1}{(1.767,1.714)}
\gppoint{gp mark 1}{(1.779,1.714)}
\gppoint{gp mark 1}{(1.791,1.713)}
\gppoint{gp mark 1}{(1.803,1.712)}
\gppoint{gp mark 1}{(1.815,1.712)}
\gppoint{gp mark 1}{(1.827,1.711)}
\gppoint{gp mark 1}{(1.839,1.710)}
\gppoint{gp mark 1}{(1.851,1.710)}
\gppoint{gp mark 1}{(1.863,1.709)}
\gppoint{gp mark 1}{(1.875,1.708)}
\gppoint{gp mark 1}{(1.887,1.707)}
\gppoint{gp mark 1}{(1.899,1.707)}
\gppoint{gp mark 1}{(1.910,1.706)}
\gppoint{gp mark 1}{(1.922,1.705)}
\gppoint{gp mark 1}{(1.934,1.704)}
\gppoint{gp mark 1}{(1.946,1.703)}
\gppoint{gp mark 1}{(1.958,1.702)}
\gppoint{gp mark 1}{(1.970,1.701)}
\gppoint{gp mark 1}{(1.982,1.700)}
\gppoint{gp mark 1}{(1.994,1.699)}
\gppoint{gp mark 1}{(2.006,1.698)}
\gppoint{gp mark 1}{(2.018,1.697)}
\gppoint{gp mark 1}{(2.030,1.696)}
\gppoint{gp mark 1}{(2.042,1.695)}
\gppoint{gp mark 1}{(2.054,1.694)}
\gppoint{gp mark 1}{(2.066,1.693)}
\gppoint{gp mark 1}{(2.078,1.692)}
\gppoint{gp mark 1}{(2.090,1.691)}
\gppoint{gp mark 1}{(2.102,1.689)}
\gppoint{gp mark 1}{(2.114,1.688)}
\gppoint{gp mark 1}{(2.126,1.687)}
\gppoint{gp mark 1}{(2.138,1.686)}
\gppoint{gp mark 1}{(2.150,1.684)}
\gppoint{gp mark 1}{(2.162,1.683)}
\gppoint{gp mark 1}{(2.173,1.682)}
\gppoint{gp mark 1}{(2.185,1.680)}
\gppoint{gp mark 1}{(2.197,1.679)}
\gppoint{gp mark 1}{(2.209,1.678)}
\gppoint{gp mark 1}{(2.221,1.676)}
\gppoint{gp mark 1}{(2.233,1.675)}
\gppoint{gp mark 1}{(2.245,1.673)}
\gppoint{gp mark 1}{(2.257,1.672)}
\gppoint{gp mark 1}{(2.269,1.670)}
\gppoint{gp mark 1}{(2.281,1.669)}
\gppoint{gp mark 1}{(2.293,1.667)}
\gppoint{gp mark 1}{(2.305,1.666)}
\gppoint{gp mark 1}{(2.317,1.664)}
\gppoint{gp mark 1}{(2.329,1.663)}
\gppoint{gp mark 1}{(2.341,1.661)}
\gppoint{gp mark 1}{(2.353,1.659)}
\gppoint{gp mark 1}{(2.365,1.658)}
\gppoint{gp mark 1}{(2.377,1.656)}
\gppoint{gp mark 1}{(2.389,1.654)}
\gppoint{gp mark 1}{(2.401,1.653)}
\gppoint{gp mark 1}{(2.413,1.651)}
\gppoint{gp mark 1}{(2.425,1.649)}
\gppoint{gp mark 1}{(2.436,1.647)}
\gppoint{gp mark 1}{(2.448,1.646)}
\gppoint{gp mark 1}{(2.460,1.644)}
\gppoint{gp mark 1}{(2.472,1.642)}
\gppoint{gp mark 1}{(2.484,1.640)}
\gppoint{gp mark 1}{(2.496,1.638)}
\gppoint{gp mark 1}{(2.508,1.637)}
\gppoint{gp mark 1}{(2.520,1.635)}
\gppoint{gp mark 1}{(2.532,1.633)}
\gppoint{gp mark 1}{(2.544,1.631)}
\gppoint{gp mark 1}{(2.556,1.629)}
\gppoint{gp mark 1}{(2.568,1.627)}
\gppoint{gp mark 1}{(2.580,1.625)}
\gppoint{gp mark 1}{(2.592,1.623)}
\gppoint{gp mark 1}{(2.604,1.621)}
\gppoint{gp mark 1}{(2.616,1.619)}
\gppoint{gp mark 1}{(2.628,1.617)}
\gppoint{gp mark 1}{(2.640,1.615)}
\gppoint{gp mark 1}{(2.652,1.613)}
\gppoint{gp mark 1}{(2.664,1.611)}
\gppoint{gp mark 1}{(2.676,1.609)}
\gppoint{gp mark 1}{(2.688,1.607)}
\gppoint{gp mark 1}{(2.699,1.605)}
\gppoint{gp mark 1}{(2.711,1.602)}
\gppoint{gp mark 1}{(2.723,1.600)}
\gppoint{gp mark 1}{(2.735,1.598)}
\gppoint{gp mark 1}{(2.747,1.596)}
\gppoint{gp mark 1}{(2.759,1.594)}
\gppoint{gp mark 1}{(2.771,1.591)}
\gppoint{gp mark 1}{(2.783,1.589)}
\gppoint{gp mark 1}{(2.795,1.587)}
\gppoint{gp mark 1}{(2.807,1.585)}
\gppoint{gp mark 1}{(2.819,1.583)}
\gppoint{gp mark 1}{(2.831,1.580)}
\gppoint{gp mark 1}{(2.843,1.578)}
\gppoint{gp mark 1}{(2.855,1.576)}
\gppoint{gp mark 1}{(2.867,1.573)}
\gppoint{gp mark 1}{(2.879,1.571)}
\gppoint{gp mark 1}{(2.891,1.569)}
\gppoint{gp mark 1}{(2.903,1.566)}
\gppoint{gp mark 1}{(2.915,1.564)}
\gppoint{gp mark 1}{(2.927,1.562)}
\gppoint{gp mark 1}{(2.939,1.559)}
\gppoint{gp mark 1}{(2.951,1.557)}
\gppoint{gp mark 1}{(2.963,1.554)}
\gppoint{gp mark 1}{(2.974,1.552)}
\gppoint{gp mark 1}{(2.986,1.550)}
\gppoint{gp mark 1}{(2.998,1.547)}
\gppoint{gp mark 1}{(3.010,1.545)}
\gppoint{gp mark 1}{(3.022,1.542)}
\gppoint{gp mark 1}{(3.034,1.540)}
\gppoint{gp mark 1}{(3.046,1.537)}
\gppoint{gp mark 1}{(3.058,1.535)}
\gppoint{gp mark 1}{(3.070,1.532)}
\gppoint{gp mark 1}{(3.082,1.530)}
\gppoint{gp mark 1}{(3.094,1.527)}
\gppoint{gp mark 1}{(3.106,1.525)}
\gppoint{gp mark 1}{(3.118,1.522)}
\gppoint{gp mark 1}{(3.130,1.520)}
\gppoint{gp mark 1}{(3.142,1.517)}
\gppoint{gp mark 1}{(3.154,1.514)}
\gppoint{gp mark 1}{(3.166,1.512)}
\gppoint{gp mark 1}{(3.178,1.509)}
\gppoint{gp mark 1}{(3.190,1.507)}
\gppoint{gp mark 1}{(3.202,1.504)}
\gppoint{gp mark 1}{(3.214,1.501)}
\gppoint{gp mark 1}{(3.226,1.499)}
\gppoint{gp mark 1}{(3.237,1.496)}
\gppoint{gp mark 1}{(3.249,1.494)}
\gppoint{gp mark 1}{(3.261,1.491)}
\gppoint{gp mark 1}{(3.273,1.488)}
\gppoint{gp mark 1}{(3.285,1.486)}
\gppoint{gp mark 1}{(3.297,1.483)}
\gppoint{gp mark 1}{(3.309,1.480)}
\gppoint{gp mark 1}{(3.321,1.478)}
\gppoint{gp mark 1}{(3.333,1.475)}
\gppoint{gp mark 1}{(3.345,1.472)}
\gppoint{gp mark 1}{(3.357,1.469)}
\gppoint{gp mark 1}{(3.369,1.467)}
\gppoint{gp mark 1}{(3.381,1.464)}
\gppoint{gp mark 1}{(3.393,1.461)}
\gppoint{gp mark 1}{(3.405,1.459)}
\gppoint{gp mark 1}{(3.417,1.456)}
\gppoint{gp mark 1}{(3.429,1.453)}
\gppoint{gp mark 1}{(3.441,1.450)}
\gppoint{gp mark 1}{(3.453,1.448)}
\gppoint{gp mark 1}{(3.465,1.445)}
\gppoint{gp mark 1}{(3.477,1.442)}
\gppoint{gp mark 1}{(3.489,1.439)}
\gppoint{gp mark 1}{(3.500,1.437)}
\gppoint{gp mark 1}{(3.512,1.434)}
\gppoint{gp mark 1}{(3.524,1.431)}
\gppoint{gp mark 1}{(3.536,1.428)}
\gppoint{gp mark 1}{(3.548,1.426)}
\gppoint{gp mark 1}{(3.560,1.423)}
\gppoint{gp mark 1}{(3.572,1.420)}
\gppoint{gp mark 1}{(3.584,1.417)}
\gppoint{gp mark 1}{(3.596,1.414)}
\gppoint{gp mark 1}{(3.608,1.412)}
\gppoint{gp mark 1}{(3.620,1.409)}
\gppoint{gp mark 1}{(3.632,1.406)}
\gppoint{gp mark 1}{(3.644,1.403)}
\gppoint{gp mark 1}{(3.656,1.400)}
\gppoint{gp mark 1}{(3.668,1.398)}
\gppoint{gp mark 1}{(3.680,1.395)}
\gppoint{gp mark 1}{(3.692,1.392)}
\gppoint{gp mark 1}{(3.704,1.389)}
\gppoint{gp mark 1}{(3.716,1.386)}
\gppoint{gp mark 1}{(3.728,1.384)}
\gppoint{gp mark 1}{(3.740,1.381)}
\gppoint{gp mark 1}{(3.752,1.378)}
\gppoint{gp mark 1}{(3.763,1.375)}
\gppoint{gp mark 1}{(3.775,1.372)}
\gppoint{gp mark 1}{(3.787,1.370)}
\gppoint{gp mark 1}{(3.799,1.367)}
\gppoint{gp mark 1}{(3.811,1.364)}
\gppoint{gp mark 1}{(3.823,1.361)}
\gppoint{gp mark 1}{(3.835,1.358)}
\gppoint{gp mark 1}{(3.847,1.355)}
\gppoint{gp mark 1}{(3.859,1.353)}
\gppoint{gp mark 1}{(3.871,1.350)}
\gppoint{gp mark 1}{(3.883,1.347)}
\gppoint{gp mark 1}{(3.895,1.344)}
\gppoint{gp mark 1}{(3.907,1.341)}
\gppoint{gp mark 1}{(3.919,1.339)}
\gppoint{gp mark 1}{(3.931,1.336)}
\gppoint{gp mark 1}{(3.943,1.333)}
\gppoint{gp mark 1}{(3.955,1.330)}
\gppoint{gp mark 1}{(3.967,1.327)}
\gppoint{gp mark 1}{(3.979,1.325)}
\gppoint{gp mark 1}{(3.991,1.322)}
\gppoint{gp mark 1}{(4.003,1.319)}
\gppoint{gp mark 1}{(4.015,1.316)}
\gppoint{gp mark 1}{(4.026,1.313)}
\gppoint{gp mark 1}{(4.038,1.311)}
\gppoint{gp mark 1}{(4.050,1.308)}
\gppoint{gp mark 1}{(4.062,1.305)}
\gppoint{gp mark 1}{(4.074,1.302)}
\gppoint{gp mark 1}{(4.086,1.299)}
\gppoint{gp mark 1}{(4.098,1.297)}
\gppoint{gp mark 1}{(4.110,1.294)}
\gppoint{gp mark 1}{(4.122,1.291)}
\gppoint{gp mark 1}{(4.134,1.288)}
\gppoint{gp mark 1}{(4.146,1.286)}
\gppoint{gp mark 1}{(4.158,1.283)}
\gppoint{gp mark 1}{(4.170,1.280)}
\gppoint{gp mark 1}{(4.182,1.277)}
\gppoint{gp mark 1}{(4.194,1.275)}
\gppoint{gp mark 1}{(4.206,1.272)}
\gppoint{gp mark 1}{(4.218,1.269)}
\gppoint{gp mark 1}{(4.230,1.266)}
\gppoint{gp mark 1}{(4.242,1.264)}
\gppoint{gp mark 1}{(4.254,1.261)}
\gppoint{gp mark 1}{(4.266,1.258)}
\gppoint{gp mark 1}{(4.278,1.256)}
\gppoint{gp mark 1}{(4.290,1.253)}
\gppoint{gp mark 1}{(4.301,1.250)}
\gppoint{gp mark 1}{(4.313,1.247)}
\gppoint{gp mark 1}{(4.325,1.245)}
\gppoint{gp mark 1}{(4.337,1.242)}
\gppoint{gp mark 1}{(4.349,1.239)}
\gppoint{gp mark 1}{(4.361,1.237)}
\gppoint{gp mark 1}{(4.373,1.234)}
\gppoint{gp mark 1}{(4.385,1.231)}
\gppoint{gp mark 1}{(4.397,1.229)}
\gppoint{gp mark 1}{(4.409,1.226)}
\gppoint{gp mark 1}{(4.421,1.224)}
\gppoint{gp mark 1}{(4.433,1.221)}
\gppoint{gp mark 1}{(4.445,1.218)}
\gppoint{gp mark 1}{(4.457,1.216)}
\gppoint{gp mark 1}{(4.469,1.213)}
\gppoint{gp mark 1}{(4.481,1.211)}
\gppoint{gp mark 1}{(4.493,1.208)}
\gppoint{gp mark 1}{(4.505,1.205)}
\gppoint{gp mark 1}{(4.517,1.203)}
\gppoint{gp mark 1}{(4.529,1.200)}
\gppoint{gp mark 1}{(4.541,1.198)}
\gppoint{gp mark 1}{(4.553,1.195)}
\gppoint{gp mark 1}{(4.564,1.193)}
\gppoint{gp mark 1}{(4.576,1.190)}
\gppoint{gp mark 1}{(4.588,1.188)}
\gppoint{gp mark 1}{(4.600,1.185)}
\gppoint{gp mark 1}{(4.612,1.183)}
\gppoint{gp mark 1}{(4.624,1.180)}
\gppoint{gp mark 1}{(4.636,1.178)}
\gppoint{gp mark 1}{(4.648,1.175)}
\gppoint{gp mark 1}{(4.660,1.173)}
\gppoint{gp mark 1}{(4.672,1.170)}
\gppoint{gp mark 1}{(4.684,1.168)}
\gppoint{gp mark 1}{(4.696,1.165)}
\gppoint{gp mark 1}{(4.708,1.163)}
\gppoint{gp mark 1}{(4.720,1.161)}
\gppoint{gp mark 1}{(4.732,1.158)}
\gppoint{gp mark 1}{(4.744,1.156)}
\gppoint{gp mark 1}{(4.756,1.154)}
\gppoint{gp mark 1}{(4.768,1.151)}
\gppoint{gp mark 1}{(4.780,1.149)}
\gppoint{gp mark 1}{(4.792,1.146)}
\gppoint{gp mark 1}{(4.804,1.144)}
\gppoint{gp mark 1}{(4.816,1.142)}
\gppoint{gp mark 1}{(4.827,1.140)}
\gppoint{gp mark 1}{(4.839,1.137)}
\gppoint{gp mark 1}{(4.851,1.135)}
\gppoint{gp mark 1}{(4.863,1.133)}
\gppoint{gp mark 1}{(4.875,1.130)}
\gppoint{gp mark 1}{(4.887,1.128)}
\gppoint{gp mark 1}{(4.899,1.126)}
\gppoint{gp mark 1}{(4.911,1.124)}
\gppoint{gp mark 1}{(4.923,1.122)}
\gppoint{gp mark 1}{(4.935,1.119)}
\gppoint{gp mark 1}{(4.947,1.117)}
\gppoint{gp mark 1}{(4.959,1.115)}
\gppoint{gp mark 1}{(4.971,1.113)}
\gppoint{gp mark 1}{(4.983,1.111)}
\gppoint{gp mark 1}{(4.995,1.109)}
\gppoint{gp mark 1}{(5.007,1.107)}
\gppoint{gp mark 1}{(5.019,1.104)}
\gppoint{gp mark 1}{(5.031,1.102)}
\gppoint{gp mark 1}{(5.043,1.100)}
\gppoint{gp mark 1}{(5.055,1.098)}
\gppoint{gp mark 1}{(5.067,1.096)}
\gppoint{gp mark 1}{(5.079,1.094)}
\gppoint{gp mark 1}{(5.090,1.092)}
\gppoint{gp mark 1}{(5.102,1.090)}
\gppoint{gp mark 1}{(5.114,1.088)}
\gppoint{gp mark 1}{(5.126,1.086)}
\gppoint{gp mark 1}{(5.138,1.084)}
\gppoint{gp mark 1}{(5.150,1.082)}
\gppoint{gp mark 1}{(5.162,1.081)}
\gppoint{gp mark 1}{(5.174,1.079)}
\gppoint{gp mark 1}{(5.186,1.077)}
\gppoint{gp mark 1}{(5.198,1.075)}
\gppoint{gp mark 1}{(5.210,1.073)}
\gppoint{gp mark 1}{(5.222,1.071)}
\gppoint{gp mark 1}{(5.234,1.069)}
\gppoint{gp mark 1}{(5.246,1.068)}
\gppoint{gp mark 1}{(5.258,1.066)}
\gppoint{gp mark 1}{(5.270,1.064)}
\gppoint{gp mark 1}{(5.282,1.062)}
\gppoint{gp mark 1}{(5.294,1.061)}
\gppoint{gp mark 1}{(5.306,1.059)}
\gppoint{gp mark 1}{(5.318,1.057)}
\gppoint{gp mark 1}{(5.330,1.055)}
\gppoint{gp mark 1}{(5.342,1.054)}
\gppoint{gp mark 1}{(5.353,1.052)}
\gppoint{gp mark 1}{(5.365,1.050)}
\gppoint{gp mark 1}{(5.377,1.049)}
\gppoint{gp mark 1}{(5.389,1.047)}
\gppoint{gp mark 1}{(5.401,1.046)}
\gppoint{gp mark 1}{(5.413,1.044)}
\gppoint{gp mark 1}{(5.425,1.043)}
\gppoint{gp mark 1}{(5.437,1.041)}
\gppoint{gp mark 1}{(5.449,1.039)}
\gppoint{gp mark 1}{(5.461,1.038)}
\gppoint{gp mark 1}{(5.473,1.037)}
\gppoint{gp mark 1}{(5.485,1.035)}
\gppoint{gp mark 1}{(5.497,1.034)}
\gppoint{gp mark 1}{(5.509,1.032)}
\gppoint{gp mark 1}{(5.521,1.031)}
\gppoint{gp mark 1}{(5.533,1.029)}
\gppoint{gp mark 1}{(5.545,1.028)}
\gppoint{gp mark 1}{(5.557,1.027)}
\gppoint{gp mark 1}{(5.569,1.025)}
\gppoint{gp mark 1}{(5.581,1.024)}
\gppoint{gp mark 1}{(5.593,1.023)}
\gppoint{gp mark 1}{(5.605,1.022)}
\gppoint{gp mark 1}{(5.617,1.020)}
\gppoint{gp mark 1}{(5.628,1.019)}
\gppoint{gp mark 1}{(5.640,1.018)}
\gppoint{gp mark 1}{(5.652,1.017)}
\gppoint{gp mark 1}{(5.664,1.015)}
\gppoint{gp mark 1}{(5.676,1.014)}
\gppoint{gp mark 1}{(5.688,1.013)}
\gppoint{gp mark 1}{(5.700,1.012)}
\gppoint{gp mark 1}{(5.712,1.011)}
\gppoint{gp mark 1}{(5.724,1.010)}
\gppoint{gp mark 1}{(5.736,1.009)}
\gppoint{gp mark 1}{(5.748,1.008)}
\gppoint{gp mark 1}{(5.760,1.007)}
\gppoint{gp mark 1}{(5.772,1.006)}
\gppoint{gp mark 1}{(5.784,1.005)}
\gppoint{gp mark 1}{(5.796,1.004)}
\gppoint{gp mark 1}{(5.808,1.003)}
\gppoint{gp mark 1}{(5.820,1.002)}
\gppoint{gp mark 1}{(5.832,1.001)}
\gppoint{gp mark 1}{(5.844,1.000)}
\gppoint{gp mark 1}{(5.856,1.000)}
\gppoint{gp mark 1}{(5.868,0.999)}
\gppoint{gp mark 1}{(5.880,0.998)}
\gppoint{gp mark 1}{(5.891,0.997)}
\gppoint{gp mark 1}{(5.903,0.997)}
\gppoint{gp mark 1}{(5.915,0.996)}
\gppoint{gp mark 1}{(5.927,0.995)}
\gppoint{gp mark 1}{(5.939,0.994)}
\gppoint{gp mark 1}{(5.951,0.994)}
\gppoint{gp mark 1}{(5.963,0.993)}
\gppoint{gp mark 1}{(5.975,0.993)}
\gppoint{gp mark 1}{(5.987,0.992)}
\gppoint{gp mark 1}{(5.999,0.991)}
\gppoint{gp mark 1}{(6.011,0.991)}
\gppoint{gp mark 1}{(6.023,0.990)}
\gppoint{gp mark 1}{(6.035,0.990)}
\gppoint{gp mark 1}{(6.047,0.989)}
\gppoint{gp mark 1}{(6.059,0.989)}
\gppoint{gp mark 1}{(6.071,0.989)}
\gppoint{gp mark 1}{(6.083,0.988)}
\gppoint{gp mark 1}{(6.095,0.988)}
\gppoint{gp mark 1}{(6.107,0.987)}
\gppoint{gp mark 1}{(6.119,0.987)}
\gppoint{gp mark 1}{(6.131,0.987)}
\gppoint{gp mark 1}{(6.143,0.986)}
\gppoint{gp mark 1}{(6.154,0.986)}
\gppoint{gp mark 1}{(6.166,0.986)}
\gppoint{gp mark 1}{(6.178,0.986)}
\gppoint{gp mark 1}{(6.190,0.986)}
\gppoint{gp mark 1}{(6.202,0.985)}
\gppoint{gp mark 1}{(6.214,0.985)}
\gppoint{gp mark 1}{(6.226,0.985)}
\gppoint{gp mark 1}{(6.238,0.985)}
\gppoint{gp mark 1}{(6.250,0.985)}
\gppoint{gp mark 1}{(6.262,0.985)}
\gppoint{gp mark 1}{(6.274,0.985)}
\gppoint{gp mark 1}{(6.286,0.985)}
\gppoint{gp mark 1}{(6.298,0.985)}
\gppoint{gp mark 1}{(6.310,0.985)}
\gppoint{gp mark 1}{(6.322,0.985)}
\gppoint{gp mark 1}{(6.334,0.985)}
\gppoint{gp mark 1}{(6.346,0.985)}
\gppoint{gp mark 1}{(6.358,0.986)}
\gppoint{gp mark 1}{(6.370,0.986)}
\gppoint{gp mark 1}{(6.382,0.986)}
\gppoint{gp mark 1}{(6.394,0.986)}
\gppoint{gp mark 1}{(6.406,0.987)}
\gppoint{gp mark 1}{(6.417,0.987)}
\gppoint{gp mark 1}{(6.429,0.987)}
\gppoint{gp mark 1}{(6.441,0.988)}
\gppoint{gp mark 1}{(6.453,0.988)}
\gppoint{gp mark 1}{(6.465,0.988)}
\gppoint{gp mark 1}{(6.477,0.989)}
\gppoint{gp mark 1}{(6.489,0.989)}
\gppoint{gp mark 1}{(6.501,0.990)}
\gppoint{gp mark 1}{(6.513,0.990)}
\gppoint{gp mark 1}{(6.525,0.991)}
\gppoint{gp mark 1}{(6.537,0.991)}
\gppoint{gp mark 1}{(6.549,0.992)}
\gppoint{gp mark 1}{(6.561,0.992)}
\gppoint{gp mark 1}{(6.573,0.993)}
\gppoint{gp mark 1}{(6.585,0.994)}
\gppoint{gp mark 1}{(6.597,0.995)}
\gppoint{gp mark 1}{(6.609,0.995)}
\gppoint{gp mark 1}{(6.621,0.996)}
\gppoint{gp mark 1}{(6.633,0.997)}
\gppoint{gp mark 1}{(6.645,0.998)}
\gppoint{gp mark 1}{(6.657,0.998)}
\gppoint{gp mark 1}{(6.669,0.999)}
\gppoint{gp mark 1}{(6.680,1.000)}
\gppoint{gp mark 1}{(6.692,1.001)}
\gppoint{gp mark 1}{(6.704,1.002)}
\gppoint{gp mark 1}{(6.716,1.003)}
\gppoint{gp mark 1}{(6.728,1.004)}
\gppoint{gp mark 1}{(6.740,1.005)}
\gppoint{gp mark 1}{(6.752,1.006)}
\gppoint{gp mark 1}{(6.764,1.007)}
\gppoint{gp mark 1}{(6.776,1.008)}
\gppoint{gp mark 1}{(6.788,1.009)}
\gppoint{gp mark 1}{(6.800,1.011)}
\gppoint{gp mark 1}{(6.812,1.012)}
\gppoint{gp mark 1}{(6.824,1.013)}
\gppoint{gp mark 1}{(6.836,1.014)}
\gppoint{gp mark 1}{(6.848,1.016)}
\gppoint{gp mark 1}{(6.860,1.017)}
\gppoint{gp mark 1}{(6.872,1.018)}
\gppoint{gp mark 1}{(6.884,1.020)}
\gppoint{gp mark 1}{(6.896,1.021)}
\gppoint{gp mark 1}{(6.908,1.023)}
\gppoint{gp mark 1}{(6.920,1.024)}
\gppoint{gp mark 1}{(6.932,1.025)}
\gppoint{gp mark 1}{(6.944,1.027)}
\gppoint{gp mark 1}{(6.955,1.029)}
\gppoint{gp mark 1}{(6.967,1.030)}
\gppoint{gp mark 1}{(6.979,1.032)}
\gppoint{gp mark 1}{(6.991,1.033)}
\gppoint{gp mark 1}{(7.003,1.035)}
\gppoint{gp mark 1}{(7.015,1.037)}
\gppoint{gp mark 1}{(7.027,1.039)}
\gppoint{gp mark 1}{(7.039,1.040)}
\gppoint{gp mark 1}{(7.051,1.042)}
\gppoint{gp mark 1}{(7.063,1.044)}
\gppoint{gp mark 1}{(7.075,1.046)}
\gppoint{gp mark 1}{(7.087,1.048)}
\gppoint{gp mark 1}{(7.099,1.050)}
\gppoint{gp mark 1}{(7.111,1.051)}
\gppoint{gp mark 1}{(7.123,1.053)}
\gppoint{gp mark 1}{(7.135,1.055)}
\gppoint{gp mark 1}{(7.147,1.058)}
\gppoint{gp mark 1}{(7.159,1.060)}
\gppoint{gp mark 1}{(7.171,1.062)}
\gppoint{gp mark 1}{(7.183,1.064)}
\gppoint{gp mark 1}{(7.195,1.066)}
\gppoint{gp mark 1}{(7.207,1.068)}
\gppoint{gp mark 1}{(7.218,1.070)}
\gppoint{gp mark 1}{(7.230,1.073)}
\gppoint{gp mark 1}{(7.242,1.075)}
\gppoint{gp mark 1}{(7.254,1.077)}
\gppoint{gp mark 1}{(7.266,1.080)}
\gppoint{gp mark 1}{(7.278,1.082)}
\gppoint{gp mark 1}{(7.290,1.084)}
\gppoint{gp mark 1}{(7.302,1.087)}
\gppoint{gp mark 1}{(7.314,1.089)}
\gppoint{gp mark 1}{(7.326,1.092)}
\gppoint{gp mark 1}{(7.338,1.094)}
\gppoint{gp mark 1}{(7.350,1.097)}
\gppoint{gp mark 1}{(7.362,1.100)}
\gppoint{gp mark 1}{(7.374,1.102)}
\gppoint{gp mark 1}{(7.386,1.105)}
\gppoint{gp mark 1}{(7.398,1.108)}
\gppoint{gp mark 1}{(7.410,1.110)}
\gppoint{gp mark 1}{(7.422,1.113)}
\gppoint{gp mark 1}{(7.434,1.116)}
\gppoint{gp mark 1}{(7.446,1.119)}
\gppoint{gp mark 1}{(7.458,1.122)}
\gppoint{gp mark 1}{(7.470,1.124)}
\gppoint{gp mark 1}{(7.481,1.127)}
\gppoint{gp mark 1}{(7.493,1.130)}
\gppoint{gp mark 1}{(7.505,1.133)}
\gppoint{gp mark 1}{(7.517,1.136)}
\gppoint{gp mark 1}{(7.529,1.139)}
\gppoint{gp mark 1}{(7.541,1.143)}
\gppoint{gp mark 1}{(7.553,1.146)}
\gppoint{gp mark 1}{(7.565,1.149)}
\gppoint{gp mark 1}{(7.577,1.152)}
\gppoint{gp mark 1}{(7.589,1.155)}
\gppoint{gp mark 1}{(7.601,1.158)}
\gppoint{gp mark 1}{(7.613,1.162)}
\gppoint{gp mark 1}{(7.625,1.165)}
\gppoint{gp mark 1}{(7.637,1.168)}
\gppoint{gp mark 1}{(7.649,1.172)}
\gppoint{gp mark 1}{(7.661,1.175)}
\gppoint{gp mark 1}{(7.673,1.179)}
\gppoint{gp mark 1}{(7.685,1.182)}
\gppoint{gp mark 1}{(7.697,1.186)}
\gppoint{gp mark 1}{(7.709,1.189)}
\gppoint{gp mark 1}{(7.721,1.193)}
\gppoint{gp mark 1}{(7.733,1.197)}
\gppoint{gp mark 1}{(7.744,1.200)}
\gppoint{gp mark 1}{(7.756,1.204)}
\gppoint{gp mark 1}{(7.768,1.208)}
\gppoint{gp mark 1}{(7.780,1.211)}
\gppoint{gp mark 1}{(7.792,1.215)}
\gppoint{gp mark 1}{(7.804,1.219)}
\gppoint{gp mark 1}{(7.816,1.223)}
\gppoint{gp mark 1}{(7.828,1.227)}
\gppoint{gp mark 1}{(7.840,1.231)}
\gppoint{gp mark 1}{(7.852,1.235)}
\gppoint{gp mark 1}{(7.864,1.239)}
\gppoint{gp mark 1}{(7.876,1.243)}
\gppoint{gp mark 1}{(7.888,1.247)}
\gppoint{gp mark 1}{(7.900,1.251)}
\gppoint{gp mark 1}{(7.912,1.255)}
\gppoint{gp mark 1}{(7.924,1.259)}
\gppoint{gp mark 1}{(7.936,1.263)}
\gppoint{gp mark 1}{(7.948,1.268)}
\gppoint{gp mark 1}{(7.960,1.272)}
\gppoint{gp mark 1}{(7.972,1.276)}
\gppoint{gp mark 1}{(7.984,1.281)}
\gppoint{gp mark 1}{(7.996,1.285)}
\gppoint{gp mark 1}{(8.007,1.289)}
\gppoint{gp mark 1}{(8.019,1.294)}
\gppoint{gp mark 1}{(8.031,1.298)}
\gppoint{gp mark 1}{(8.043,1.303)}
\gppoint{gp mark 1}{(8.055,1.308)}
\gppoint{gp mark 1}{(8.067,1.312)}
\gppoint{gp mark 1}{(8.079,1.317)}
\gppoint{gp mark 1}{(8.091,1.321)}
\gppoint{gp mark 1}{(8.103,1.326)}
\gppoint{gp mark 1}{(8.115,1.331)}
\gppoint{gp mark 1}{(8.127,1.336)}
\gppoint{gp mark 1}{(8.139,1.341)}
\gppoint{gp mark 1}{(8.151,1.345)}
\gppoint{gp mark 1}{(8.163,1.350)}
\gppoint{gp mark 1}{(8.175,1.355)}
\gppoint{gp mark 1}{(8.187,1.360)}
\gppoint{gp mark 1}{(8.199,1.365)}
\gppoint{gp mark 1}{(8.211,1.370)}
\gppoint{gp mark 1}{(8.223,1.375)}
\gppoint{gp mark 1}{(8.235,1.380)}
\gppoint{gp mark 1}{(8.247,1.385)}
\gppoint{gp mark 1}{(8.259,1.391)}
\gppoint{gp mark 1}{(8.271,1.396)}
\gppoint{gp mark 1}{(8.282,1.401)}
\gppoint{gp mark 1}{(8.294,1.406)}
\gppoint{gp mark 1}{(8.306,1.412)}
\gppoint{gp mark 1}{(8.318,1.417)}
\gppoint{gp mark 1}{(8.330,1.422)}
\gppoint{gp mark 1}{(8.342,1.428)}
\gppoint{gp mark 1}{(8.354,1.433)}
\gppoint{gp mark 1}{(8.366,1.439)}
\gppoint{gp mark 1}{(8.378,1.444)}
\gppoint{gp mark 1}{(8.390,1.450)}
\gppoint{gp mark 1}{(8.402,1.456)}
\gppoint{gp mark 1}{(8.414,1.461)}
\gppoint{gp mark 1}{(8.426,1.467)}
\gppoint{gp mark 1}{(8.438,1.473)}
\gppoint{gp mark 1}{(8.450,1.478)}
\gppoint{gp mark 1}{(8.462,1.484)}
\gppoint{gp mark 1}{(8.474,1.490)}
\gppoint{gp mark 1}{(8.486,1.496)}
\gppoint{gp mark 1}{(8.498,1.502)}
\gppoint{gp mark 1}{(8.510,1.508)}
\gppoint{gp mark 1}{(8.522,1.514)}
\gppoint{gp mark 1}{(8.534,1.520)}
\gppoint{gp mark 1}{(8.545,1.526)}
\gppoint{gp mark 1}{(8.557,1.532)}
\gppoint{gp mark 1}{(8.569,1.538)}
\gppoint{gp mark 1}{(8.581,1.544)}
\gppoint{gp mark 1}{(8.593,1.551)}
\gppoint{gp mark 1}{(8.605,1.557)}
\gppoint{gp mark 1}{(8.617,1.563)}
\gppoint{gp mark 1}{(8.629,1.569)}
\gppoint{gp mark 1}{(8.641,1.576)}
\gppoint{gp mark 1}{(8.653,1.582)}
\gppoint{gp mark 1}{(8.665,1.589)}
\gppoint{gp mark 1}{(8.677,1.595)}
\gppoint{gp mark 1}{(8.689,1.602)}
\gppoint{gp mark 1}{(8.701,1.608)}
\gppoint{gp mark 1}{(8.713,1.615)}
\gppoint{gp mark 1}{(8.725,1.621)}
\gppoint{gp mark 1}{(8.737,1.628)}
\gppoint{gp mark 1}{(8.749,1.635)}
\gppoint{gp mark 1}{(8.761,1.641)}
\gppoint{gp mark 1}{(8.773,1.648)}
\gppoint{gp mark 1}{(8.785,1.655)}
\gppoint{gp mark 1}{(8.797,1.662)}
\gppoint{gp mark 1}{(8.808,1.669)}
\gppoint{gp mark 1}{(8.820,1.676)}
\gppoint{gp mark 1}{(8.832,1.683)}
\gppoint{gp mark 1}{(8.844,1.690)}
\gppoint{gp mark 1}{(8.856,1.697)}
\gppoint{gp mark 1}{(8.868,1.704)}
\gppoint{gp mark 1}{(8.880,1.711)}
\gppoint{gp mark 1}{(8.892,1.718)}
\gppoint{gp mark 1}{(8.904,1.725)}
\gppoint{gp mark 1}{(8.916,1.733)}
\gppoint{gp mark 1}{(8.928,1.740)}
\gppoint{gp mark 1}{(8.940,1.747)}
\gppoint{gp mark 1}{(8.952,1.755)}
\gppoint{gp mark 1}{(8.964,1.762)}
\gppoint{gp mark 1}{(8.976,1.769)}
\gppoint{gp mark 1}{(8.988,1.777)}
\gppoint{gp mark 1}{(9.000,1.784)}
\gppoint{gp mark 1}{(9.012,1.792)}
\gppoint{gp mark 1}{(9.024,1.800)}
\gppoint{gp mark 1}{(9.036,1.807)}
\gppoint{gp mark 1}{(9.048,1.815)}
\gppoint{gp mark 1}{(9.060,1.823)}
\gppoint{gp mark 1}{(9.071,1.830)}
\gppoint{gp mark 1}{(9.083,1.838)}
\gppoint{gp mark 1}{(9.095,1.846)}
\gppoint{gp mark 1}{(9.107,1.854)}
\gppoint{gp mark 1}{(9.119,1.862)}
\gppoint{gp mark 1}{(9.131,1.870)}
\gppoint{gp mark 1}{(9.143,1.878)}
\gppoint{gp mark 1}{(9.155,1.886)}
\gppoint{gp mark 1}{(9.167,1.894)}
\gppoint{gp mark 1}{(9.179,1.902)}
\gppoint{gp mark 1}{(9.191,1.910)}
\gppoint{gp mark 1}{(9.203,1.918)}
\gppoint{gp mark 1}{(9.215,1.926)}
\gppoint{gp mark 1}{(9.227,1.935)}
\gppoint{gp mark 1}{(9.239,1.943)}
\gppoint{gp mark 1}{(9.251,1.951)}
\gppoint{gp mark 1}{(9.263,1.960)}
\gppoint{gp mark 1}{(9.275,1.968)}
\gppoint{gp mark 1}{(9.287,1.977)}
\gppoint{gp mark 1}{(9.299,1.985)}
\gppoint{gp mark 1}{(9.311,1.994)}
\gppoint{gp mark 1}{(9.323,2.002)}
\gppoint{gp mark 1}{(9.334,2.011)}
\gppoint{gp mark 1}{(9.346,2.020)}
\gppoint{gp mark 1}{(9.358,2.028)}
\gppoint{gp mark 1}{(9.370,2.037)}
\gppoint{gp mark 1}{(9.382,2.046)}
\gppoint{gp mark 1}{(9.394,2.055)}
\gppoint{gp mark 1}{(9.406,2.064)}
\gppoint{gp mark 1}{(9.418,2.072)}
\gppoint{gp mark 1}{(9.430,2.081)}
\gppoint{gp mark 1}{(9.442,2.090)}
\gppoint{gp mark 1}{(9.454,2.099)}
\gppoint{gp mark 1}{(9.466,2.108)}
\gppoint{gp mark 1}{(9.478,2.118)}
\gppoint{gp mark 1}{(9.490,2.127)}
\gppoint{gp mark 1}{(9.502,2.136)}
\gppoint{gp mark 1}{(9.514,2.145)}
\gppoint{gp mark 1}{(9.526,2.154)}
\gppoint{gp mark 1}{(9.538,2.164)}
\gppoint{gp mark 1}{(9.550,2.173)}
\gppoint{gp mark 1}{(9.562,2.183)}
\gppoint{gp mark 1}{(9.574,2.192)}
\gppoint{gp mark 1}{(9.586,2.201)}
\gppoint{gp mark 1}{(9.598,2.211)}
\gppoint{gp mark 1}{(9.609,2.221)}
\gppoint{gp mark 1}{(9.621,2.230)}
\gppoint{gp mark 1}{(9.633,2.240)}
\gppoint{gp mark 1}{(9.645,2.249)}
\gppoint{gp mark 1}{(9.657,2.259)}
\gppoint{gp mark 1}{(9.669,2.269)}
\gppoint{gp mark 1}{(9.681,2.279)}
\gppoint{gp mark 1}{(9.693,2.289)}
\gppoint{gp mark 1}{(9.705,2.298)}
\gppoint{gp mark 1}{(9.717,2.308)}
\gppoint{gp mark 1}{(9.729,2.318)}
\gppoint{gp mark 1}{(9.741,2.328)}
\gppoint{gp mark 1}{(9.753,2.338)}
\gppoint{gp mark 1}{(9.765,2.349)}
\gppoint{gp mark 1}{(9.777,2.359)}
\gppoint{gp mark 1}{(9.789,2.369)}
\gppoint{gp mark 1}{(9.801,2.379)}
\gppoint{gp mark 1}{(9.813,2.389)}
\gppoint{gp mark 1}{(9.825,2.400)}
\gppoint{gp mark 1}{(9.837,2.410)}
\gppoint{gp mark 1}{(9.849,2.420)}
\gppoint{gp mark 1}{(9.861,2.431)}
\gppoint{gp mark 1}{(9.872,2.441)}
\gppoint{gp mark 1}{(9.884,2.452)}
\gppoint{gp mark 1}{(9.896,2.462)}
\gppoint{gp mark 1}{(9.908,2.473)}
\gppoint{gp mark 1}{(9.920,2.484)}
\gppoint{gp mark 1}{(9.932,2.494)}
\gppoint{gp mark 1}{(9.944,2.505)}
\gppoint{gp mark 1}{(9.956,2.516)}
\gppoint{gp mark 1}{(9.968,2.527)}
\gppoint{gp mark 1}{(9.980,2.537)}
\gppoint{gp mark 1}{(9.992,2.548)}
\gppoint{gp mark 1}{(10.004,2.559)}
\gppoint{gp mark 1}{(10.016,2.570)}
\gppoint{gp mark 1}{(10.028,2.581)}
\gppoint{gp mark 1}{(10.040,2.592)}
\gppoint{gp mark 1}{(10.052,2.603)}
\gppoint{gp mark 1}{(10.064,2.614)}
\gppoint{gp mark 1}{(10.076,2.626)}
\gppoint{gp mark 1}{(10.088,2.637)}
\gppoint{gp mark 1}{(10.100,2.648)}
\gppoint{gp mark 1}{(10.112,2.659)}
\gppoint{gp mark 1}{(10.124,2.671)}
\gppoint{gp mark 1}{(10.135,2.682)}
\gppoint{gp mark 1}{(10.147,2.694)}
\gppoint{gp mark 1}{(10.159,2.705)}
\gppoint{gp mark 1}{(10.171,2.717)}
\gppoint{gp mark 1}{(10.183,2.728)}
\gppoint{gp mark 1}{(10.195,2.740)}
\gppoint{gp mark 1}{(10.207,2.751)}
\gppoint{gp mark 1}{(10.219,2.763)}
\gppoint{gp mark 1}{(10.231,2.775)}
\gppoint{gp mark 1}{(10.243,2.787)}
\gppoint{gp mark 1}{(10.255,2.798)}
\gppoint{gp mark 1}{(10.267,2.810)}
\gppoint{gp mark 1}{(10.279,2.822)}
\gppoint{gp mark 1}{(10.291,2.834)}
\gppoint{gp mark 1}{(10.303,2.846)}
\gppoint{gp mark 1}{(10.315,2.858)}
\gppoint{gp mark 1}{(10.327,2.870)}
\gppoint{gp mark 1}{(10.339,2.882)}
\gppoint{gp mark 1}{(10.351,2.895)}
\gppoint{gp mark 1}{(10.363,2.907)}
\gppoint{gp mark 1}{(10.375,2.919)}
\gppoint{gp mark 1}{(10.387,2.931)}
\gppoint{gp mark 1}{(10.398,2.944)}
\gppoint{gp mark 1}{(10.410,2.956)}
\gppoint{gp mark 1}{(10.422,2.968)}
\gppoint{gp mark 1}{(10.434,2.981)}
\gppoint{gp mark 1}{(10.446,2.993)}
\gppoint{gp mark 1}{(10.458,3.006)}
\gppoint{gp mark 1}{(10.470,3.019)}
\gppoint{gp mark 1}{(10.482,3.031)}
\gppoint{gp mark 1}{(10.494,3.044)}
\gppoint{gp mark 1}{(10.506,3.057)}
\gppoint{gp mark 1}{(10.518,3.069)}
\gppoint{gp mark 1}{(10.530,3.082)}
\gppoint{gp mark 1}{(10.542,3.095)}
\gppoint{gp mark 1}{(10.554,3.108)}
\gppoint{gp mark 1}{(10.566,3.121)}
\gppoint{gp mark 1}{(10.578,3.134)}
\gppoint{gp mark 1}{(10.590,3.147)}
\gppoint{gp mark 1}{(10.602,3.160)}
\gppoint{gp mark 1}{(10.614,3.173)}
\gppoint{gp mark 1}{(10.626,3.186)}
\gppoint{gp mark 1}{(10.638,3.199)}
\gppoint{gp mark 1}{(10.650,3.213)}
\gppoint{gp mark 1}{(10.661,3.226)}
\gppoint{gp mark 1}{(10.673,3.239)}
\gppoint{gp mark 1}{(10.685,3.253)}
\gppoint{gp mark 1}{(10.697,3.266)}
\gppoint{gp mark 1}{(10.709,3.279)}
\gppoint{gp mark 1}{(10.721,3.293)}
\gppoint{gp mark 1}{(10.733,3.306)}
\gppoint{gp mark 1}{(10.745,3.320)}
\gppoint{gp mark 1}{(10.757,3.334)}
\gppoint{gp mark 1}{(10.769,3.347)}
\gppoint{gp mark 1}{(10.781,3.361)}
\gppoint{gp mark 1}{(10.793,3.375)}
\gppoint{gp mark 1}{(10.805,3.389)}
\gppoint{gp mark 1}{(10.817,3.403)}
\gppoint{gp mark 1}{(10.829,3.416)}
\gppoint{gp mark 1}{(10.841,3.430)}
\gppoint{gp mark 1}{(10.853,3.444)}
\gppoint{gp mark 1}{(10.865,3.458)}
\gppoint{gp mark 1}{(10.877,3.472)}
\gppoint{gp mark 1}{(10.889,3.487)}
\gppoint{gp mark 1}{(10.901,3.501)}
\gppoint{gp mark 1}{(10.913,3.515)}
\gppoint{gp mark 1}{(10.925,3.529)}
\gppoint{gp mark 1}{(10.936,3.543)}
\gppoint{gp mark 1}{(10.948,3.558)}
\gppoint{gp mark 1}{(10.960,3.572)}
\gppoint{gp mark 1}{(10.972,3.587)}
\gppoint{gp mark 1}{(10.984,3.601)}
\gppoint{gp mark 1}{(10.996,3.616)}
\gppoint{gp mark 1}{(11.008,3.630)}
\gppoint{gp mark 1}{(11.020,3.645)}
\gppoint{gp mark 1}{(11.032,3.659)}
\gppoint{gp mark 1}{(11.044,3.674)}
\gppoint{gp mark 1}{(11.056,3.689)}
\gppoint{gp mark 1}{(11.068,3.704)}
\gppoint{gp mark 1}{(11.080,3.718)}
\gppoint{gp mark 1}{(11.092,3.733)}
\gppoint{gp mark 1}{(11.104,3.748)}
\gppoint{gp mark 1}{(11.116,3.763)}
\gppoint{gp mark 1}{(11.128,3.778)}
\gppoint{gp mark 1}{(11.140,3.793)}
\gppoint{gp mark 1}{(11.152,3.808)}
\gppoint{gp mark 1}{(11.164,3.823)}
\gppoint{gp mark 1}{(11.176,3.838)}
\gppoint{gp mark 1}{(11.188,3.854)}
\gppoint{gp mark 1}{(11.199,3.869)}
\gppoint{gp mark 1}{(11.211,3.884)}
\gppoint{gp mark 1}{(11.223,3.900)}
\gppoint{gp mark 1}{(11.235,3.915)}
\gppoint{gp mark 1}{(11.247,3.930)}
\gppoint{gp mark 1}{(11.259,3.946)}
\gppoint{gp mark 1}{(11.271,3.961)}
\gppoint{gp mark 1}{(11.283,3.977)}
\gppoint{gp mark 1}{(11.295,3.993)}
\gppoint{gp mark 1}{(11.307,4.008)}
\gppoint{gp mark 1}{(11.319,4.024)}
\gppoint{gp mark 1}{(11.331,4.040)}
\gppoint{gp mark 1}{(11.343,4.056)}
\gppoint{gp mark 1}{(11.355,4.071)}
\gppoint{gp mark 1}{(11.367,4.087)}
\gppoint{gp mark 1}{(11.379,4.103)}
\gppoint{gp mark 1}{(11.391,4.119)}
\gppoint{gp mark 1}{(11.403,4.135)}
\gppoint{gp mark 1}{(11.415,4.151)}
\gppoint{gp mark 1}{(11.427,4.167)}
\gppoint{gp mark 1}{(11.439,4.184)}
\gppoint{gp mark 1}{(11.451,4.200)}
\gppoint{gp mark 1}{(11.462,4.216)}
\gppoint{gp mark 1}{(11.474,4.232)}
\gppoint{gp mark 1}{(11.486,4.249)}
\gppoint{gp mark 1}{(11.498,4.265)}
\gppoint{gp mark 1}{(11.510,4.282)}
\gppoint{gp mark 1}{(11.522,4.298)}
\gppoint{gp mark 1}{(11.534,4.315)}
\gppoint{gp mark 1}{(11.546,4.331)}
\gppoint{gp mark 1}{(11.558,4.348)}
\gppoint{gp mark 1}{(11.570,4.364)}
\gppoint{gp mark 1}{(11.582,4.381)}
\gppoint{gp mark 1}{(11.594,4.398)}
\gppoint{gp mark 1}{(11.606,4.415)}
\gppoint{gp mark 1}{(11.618,4.431)}
\gppoint{gp mark 1}{(11.630,4.448)}
\gppoint{gp mark 1}{(11.642,4.465)}
\gppoint{gp mark 1}{(11.654,4.482)}
\gppoint{gp mark 1}{(11.666,4.499)}
\gppoint{gp mark 1}{(11.678,4.516)}
\gppoint{gp mark 1}{(11.690,4.533)}
\gppoint{gp mark 1}{(11.702,4.551)}
\gppoint{gp mark 1}{(11.714,4.568)}
\gppoint{gp mark 1}{(11.725,4.585)}
\gppoint{gp mark 1}{(11.737,4.602)}
\gppoint{gp mark 1}{(11.749,4.620)}
\gppoint{gp mark 1}{(11.761,4.637)}
\gppoint{gp mark 1}{(11.773,4.654)}
\gppoint{gp mark 1}{(11.785,4.672)}
\gppoint{gp mark 1}{(11.797,4.689)}
\gppoint{gp mark 1}{(11.809,4.707)}
\gppoint{gp mark 1}{(11.821,4.725)}
\gppoint{gp mark 1}{(11.833,4.742)}
\gppoint{gp mark 1}{(11.845,4.760)}
\gppoint{gp mark 1}{(11.857,4.778)}
\gppoint{gp mark 1}{(11.869,4.795)}
\gppoint{gp mark 1}{(11.881,4.813)}
\gppoint{gp mark 1}{(11.893,4.831)}
\gppoint{gp mark 1}{(11.905,4.849)}
\gppoint{gp mark 1}{(11.917,4.867)}
\gppoint{gp mark 1}{(11.929,4.885)}
\gppoint{gp mark 1}{(11.941,4.903)}
\gppoint{gp mark 1}{(11.953,4.921)}
\gppoint{gp mark 1}{(11.965,4.939)}
\gppoint{gp mark 1}{(11.977,4.958)}
\gppoint{gp mark 1}{(11.988,4.976)}
\gppoint{gp mark 1}{(12.000,4.994)}
\gppoint{gp mark 1}{(12.012,5.012)}
\gppoint{gp mark 1}{(12.024,5.031)}
\gppoint{gp mark 1}{(12.036,5.049)}
\gppoint{gp mark 1}{(12.048,5.068)}
\gppoint{gp mark 1}{(12.060,5.086)}
\gppoint{gp mark 1}{(12.072,5.105)}
\gppoint{gp mark 1}{(12.084,5.123)}
\gppoint{gp mark 1}{(12.096,5.142)}
\gppoint{gp mark 1}{(12.108,5.161)}
\gppoint{gp mark 1}{(12.120,5.180)}
\gppoint{gp mark 1}{(12.132,5.198)}
\gppoint{gp mark 1}{(12.144,5.217)}
\gppoint{gp mark 1}{(12.156,5.236)}
\gppoint{gp mark 1}{(12.168,5.255)}
\gppoint{gp mark 1}{(12.180,5.274)}
\gppoint{gp mark 1}{(12.192,5.293)}
\gppoint{gp mark 1}{(12.204,5.312)}
\gppoint{gp mark 1}{(12.216,5.331)}
\gppoint{gp mark 1}{(12.228,5.350)}
\gppoint{gp mark 1}{(12.240,5.370)}
\gppoint{gp mark 1}{(12.252,5.389)}
\gppoint{gp mark 1}{(12.263,5.408)}
\gppoint{gp mark 1}{(12.275,5.427)}
\gppoint{gp mark 1}{(12.287,5.447)}
\gppoint{gp mark 1}{(12.299,5.466)}
\gppoint{gp mark 1}{(12.311,5.486)}
\gppoint{gp mark 1}{(12.323,5.505)}
\gppoint{gp mark 1}{(12.335,5.525)}
\gppoint{gp mark 1}{(12.347,5.545)}
\gppoint{gp mark 1}{(12.359,5.564)}
\gppoint{gp mark 1}{(12.371,5.584)}
\gppoint{gp mark 1}{(12.383,5.604)}
\gppoint{gp mark 1}{(12.395,5.624)}
\gppoint{gp mark 1}{(12.407,5.643)}
\gppoint{gp mark 1}{(12.419,5.663)}
\gppoint{gp mark 1}{(12.431,5.683)}
\gppoint{gp mark 1}{(12.443,5.703)}
\gppoint{gp mark 1}{(12.455,5.723)}
\gppoint{gp mark 1}{(12.467,5.743)}
\gppoint{gp mark 1}{(12.479,5.763)}
\gppoint{gp mark 1}{(12.491,5.784)}
\gppoint{gp mark 1}{(12.503,5.804)}
\gppoint{gp mark 1}{(12.515,5.824)}
\gppoint{gp mark 1}{(12.526,5.844)}
\gppoint{gp mark 1}{(12.538,5.865)}
\gppoint{gp mark 1}{(12.550,5.885)}
\gppoint{gp mark 1}{(12.562,5.906)}
\gppoint{gp mark 1}{(12.574,5.926)}
\gppoint{gp mark 1}{(12.586,5.947)}
\gppoint{gp mark 1}{(12.598,5.967)}
\gppoint{gp mark 1}{(12.610,5.988)}
\gppoint{gp mark 1}{(12.622,6.009)}
\gppoint{gp mark 1}{(12.634,6.029)}
\gppoint{gp mark 1}{(12.646,6.050)}
\gppoint{gp mark 1}{(12.658,6.071)}
\gppoint{gp mark 1}{(12.670,6.092)}
\gppoint{gp mark 1}{(12.682,6.113)}
\gppoint{gp mark 1}{(12.694,6.134)}
\gppoint{gp mark 1}{(12.706,6.155)}
\gppoint{gp mark 1}{(12.718,6.176)}
\gppoint{gp mark 1}{(12.730,6.197)}
\gppoint{gp mark 1}{(12.742,6.218)}
\gppoint{gp mark 1}{(12.754,6.239)}
\gppoint{gp mark 1}{(12.766,6.261)}
\gppoint{gp mark 1}{(12.778,6.282)}
\gppoint{gp mark 1}{(12.789,6.303)}
\gppoint{gp mark 1}{(12.801,6.325)}
\gppoint{gp mark 1}{(12.813,6.346)}
\gppoint{gp mark 1}{(12.825,6.368)}
\gppoint{gp mark 1}{(12.837,6.389)}
\gppoint{gp mark 1}{(12.849,6.411)}
\gppoint{gp mark 1}{(12.861,6.432)}
\gppoint{gp mark 1}{(12.873,6.454)}
\gppoint{gp mark 1}{(12.885,6.476)}
\gppoint{gp mark 1}{(12.897,6.498)}
\gppoint{gp mark 1}{(12.909,6.519)}
\gppoint{gp mark 1}{(12.921,6.541)}
\gppoint{gp mark 1}{(12.933,6.563)}
\gppoint{gp mark 1}{(12.945,6.585)}
\gppoint{gp mark 1}{(12.957,6.607)}
\gppoint{gp mark 1}{(12.969,6.629)}
\gppoint{gp mark 1}{(12.981,6.651)}
\gppoint{gp mark 1}{(12.993,6.673)}
\gppoint{gp mark 1}{(13.005,6.696)}
\gppoint{gp mark 1}{(13.017,6.718)}
\gppoint{gp mark 1}{(13.029,6.740)}
\gppoint{gp mark 1}{(13.041,6.763)}
\gppoint{gp mark 1}{(13.052,6.785)}
\gppoint{gp mark 1}{(13.064,6.807)}
\gppoint{gp mark 1}{(13.076,6.830)}
\gppoint{gp mark 1}{(13.088,6.852)}
\gppoint{gp mark 1}{(13.100,6.875)}
\gppoint{gp mark 1}{(13.112,6.898)}
\gppoint{gp mark 1}{(13.124,6.920)}
\gppoint{gp mark 1}{(13.136,6.943)}
\gppoint{gp mark 1}{(13.148,6.966)}
\gppoint{gp mark 1}{(13.160,6.988)}
\gppoint{gp mark 1}{(13.172,7.011)}
\gppoint{gp mark 1}{(13.184,7.034)}
\gppoint{gp mark 1}{(13.196,7.057)}
\gppoint{gp mark 1}{(13.208,7.080)}
\gppoint{gp mark 1}{(13.220,7.103)}
\gppoint{gp mark 1}{(13.232,7.126)}
\gppoint{gp mark 1}{(13.244,7.149)}
\gppoint{gp mark 1}{(13.256,7.173)}
\gppoint{gp mark 1}{(13.268,7.196)}
\gppoint{gp mark 1}{(13.280,7.219)}
\gppoint{gp mark 1}{(13.292,7.242)}
\gppoint{gp mark 1}{(13.304,7.266)}
\gppoint{gp mark 1}{(13.315,7.289)}
\gppoint{gp mark 1}{(13.327,7.313)}
\gppoint{gp mark 1}{(13.339,7.336)}
\gppoint{gp mark 1}{(13.351,7.360)}
\gppoint{gp mark 1}{(13.363,7.383)}
\gppoint{gp mark 1}{(13.375,7.407)}
\gppoint{gp mark 1}{(13.387,7.431)}
\gppoint{gp mark 1}{(13.399,7.455)}
\gppoint{gp mark 1}{(13.411,7.478)}
\gppoint{gp mark 1}{(13.423,7.502)}
\gppoint{gp mark 1}{(13.435,7.526)}
\gppoint{gp mark 1}{(13.447,7.550)}
\gppoint{gp mark 1}{(2.330,8.297)}
\gpcolor{color=gp lt color border}
\draw[gp path] (1.504,8.631)--(1.504,0.985)--(13.447,0.985)--(13.447,8.631)--cycle;
%% coordinates of the plot area
\gpdefrectangularnode{gp plot 1}{\pgfpoint{1.504cm}{0.985cm}}{\pgfpoint{13.447cm}{8.631cm}}
\end{tikzpicture}
%% gnuplot variables

	\caption{Gráfico do potencial termodinâmico $\tilde{\omega}$ obtido variando $m$ arbitrariamente, para o valor $\mu = \mu_R = 0$. \protect[Parameters: NJL $\rm{D}_1$, $m_0 = \np[MeV]{5.6}$]}
	\label{Fig:pot_term_analysys_NJL-Buballa_Set_1}
\end{figure*}

\begin{figure*}
	\begin{tikzpicture}[gnuplot]
%% generated with GNUPLOT 5.0p2 (Lua 5.2; terminal rev. 99, script rev. 100)
%% Mon Apr 11 17:50:26 2016
\path (0.000,0.000) rectangle (14.000,9.000);
\gpcolor{color=gp lt color border}
\gpsetlinetype{gp lt border}
\gpsetdashtype{gp dt solid}
\gpsetlinewidth{1.00}
\draw[gp path] (1.320,0.985)--(1.500,0.985);
\draw[gp path] (13.447,0.985)--(13.267,0.985);
\node[gp node right] at (1.136,0.985) {$100$};
\draw[gp path] (1.320,1.835)--(1.500,1.835);
\draw[gp path] (13.447,1.835)--(13.267,1.835);
\node[gp node right] at (1.136,1.835) {$150$};
\draw[gp path] (1.320,2.684)--(1.500,2.684);
\draw[gp path] (13.447,2.684)--(13.267,2.684);
\node[gp node right] at (1.136,2.684) {$200$};
\draw[gp path] (1.320,3.534)--(1.500,3.534);
\draw[gp path] (13.447,3.534)--(13.267,3.534);
\node[gp node right] at (1.136,3.534) {$250$};
\draw[gp path] (1.320,4.383)--(1.500,4.383);
\draw[gp path] (13.447,4.383)--(13.267,4.383);
\node[gp node right] at (1.136,4.383) {$300$};
\draw[gp path] (1.320,5.233)--(1.500,5.233);
\draw[gp path] (13.447,5.233)--(13.267,5.233);
\node[gp node right] at (1.136,5.233) {$350$};
\draw[gp path] (1.320,6.082)--(1.500,6.082);
\draw[gp path] (13.447,6.082)--(13.267,6.082);
\node[gp node right] at (1.136,6.082) {$400$};
\draw[gp path] (1.320,6.932)--(1.500,6.932);
\draw[gp path] (13.447,6.932)--(13.267,6.932);
\node[gp node right] at (1.136,6.932) {$450$};
\draw[gp path] (1.320,7.781)--(1.500,7.781);
\draw[gp path] (13.447,7.781)--(13.267,7.781);
\node[gp node right] at (1.136,7.781) {$500$};
\draw[gp path] (1.320,8.631)--(1.500,8.631);
\draw[gp path] (13.447,8.631)--(13.267,8.631);
\node[gp node right] at (1.136,8.631) {$550$};
\draw[gp path] (1.320,0.985)--(1.320,1.165);
\draw[gp path] (1.320,8.631)--(1.320,8.451);
\node[gp node center] at (1.320,0.677) {$0$};
\draw[gp path] (3.341,0.985)--(3.341,1.165);
\draw[gp path] (3.341,8.631)--(3.341,8.451);
\node[gp node center] at (3.341,0.677) {$0.2$};
\draw[gp path] (5.362,0.985)--(5.362,1.165);
\draw[gp path] (5.362,8.631)--(5.362,8.451);
\node[gp node center] at (5.362,0.677) {$0.4$};
\draw[gp path] (7.384,0.985)--(7.384,1.165);
\draw[gp path] (7.384,8.631)--(7.384,8.451);
\node[gp node center] at (7.384,0.677) {$0.6$};
\draw[gp path] (9.405,0.985)--(9.405,1.165);
\draw[gp path] (9.405,8.631)--(9.405,8.451);
\node[gp node center] at (9.405,0.677) {$0.8$};
\draw[gp path] (11.426,0.985)--(11.426,1.165);
\draw[gp path] (11.426,8.631)--(11.426,8.451);
\node[gp node center] at (11.426,0.677) {$1$};
\draw[gp path] (13.447,0.985)--(13.447,1.165);
\draw[gp path] (13.447,8.631)--(13.447,8.451);
\node[gp node center] at (13.447,0.677) {$1.2$};
\draw[gp path] (1.320,8.631)--(1.320,0.985)--(13.447,0.985)--(13.447,8.631)--cycle;
\node[gp node center,rotate=-270] at (0.246,4.808) {$p_F$ (MeV)};
\node[gp node center] at (7.383,0.215) {$\rho$ ($\rm{fm}^{-3}$)};
\gpcolor{rgb color={0.580,0.000,0.827}}
\draw[gp path] (1.432,1.122)--(1.443,1.180)--(1.454,1.235)--(1.465,1.287)--(1.476,1.337)%
  --(1.487,1.384)--(1.498,1.429)--(1.509,1.472)--(1.520,1.514)--(1.531,1.554)--(1.542,1.593)%
  --(1.553,1.630)--(1.564,1.667)--(1.575,1.702)--(1.586,1.736)--(1.597,1.770)--(1.609,1.802)%
  --(1.620,1.834)--(1.631,1.865)--(1.642,1.895)--(1.653,1.924)--(1.664,1.953)--(1.675,1.981)%
  --(1.686,2.009)--(1.697,2.036)--(1.708,2.063)--(1.719,2.089)--(1.730,2.114)--(1.741,2.139)%
  --(1.752,2.164)--(1.763,2.188)--(1.774,2.212)--(1.785,2.236)--(1.796,2.259)--(1.807,2.282)%
  --(1.818,2.304)--(1.829,2.326)--(1.840,2.348)--(1.851,2.370)--(1.862,2.391)--(1.873,2.412)%
  --(1.884,2.432)--(1.895,2.453)--(1.906,2.473)--(1.917,2.493)--(1.928,2.512)--(1.939,2.532)%
  --(1.950,2.551)--(1.961,2.570)--(1.972,2.588)--(1.983,2.607)--(1.994,2.625)--(2.005,2.643)%
  --(2.016,2.661)--(2.028,2.679)--(2.039,2.696)--(2.050,2.714)--(2.061,2.731)--(2.072,2.748)%
  --(2.083,2.765)--(2.094,2.782)--(2.105,2.798)--(2.116,2.814)--(2.127,2.831)--(2.138,2.847)%
  --(2.149,2.863)--(2.160,2.878)--(2.171,2.894)--(2.182,2.910)--(2.193,2.925)--(2.204,2.940)%
  --(2.215,2.955)--(2.226,2.970)--(2.237,2.985)--(2.248,3.000)--(2.259,3.015)--(2.270,3.029)%
  --(2.281,3.044)--(2.292,3.058)--(2.303,3.072)--(2.314,3.086)--(2.325,3.100)--(2.336,3.114)%
  --(2.347,3.128)--(2.358,3.142)--(2.369,3.155)--(2.380,3.169)--(2.391,3.182)--(2.402,3.196)%
  --(2.413,3.209)--(2.424,3.222)--(2.435,3.235)--(2.447,3.248)--(2.458,3.261)--(2.469,3.274)%
  --(2.480,3.286)--(2.491,3.299)--(2.502,3.312)--(2.513,3.324)--(2.524,3.336)--(2.535,3.349)%
  --(2.546,3.361)--(2.557,3.373)--(2.568,3.385)--(2.579,3.397)--(2.590,3.409)--(2.601,3.421)%
  --(2.612,3.433)--(2.623,3.445)--(2.634,3.457)--(2.645,3.468)--(2.656,3.480)--(2.667,3.491)%
  --(2.678,3.503)--(2.689,3.514)--(2.700,3.525)--(2.711,3.537)--(2.722,3.548)--(2.733,3.559)%
  --(2.744,3.570)--(2.755,3.581)--(2.766,3.592)--(2.777,3.603)--(2.788,3.614)--(2.799,3.625)%
  --(2.810,3.635)--(2.821,3.646)--(2.832,3.657)--(2.843,3.667)--(2.854,3.678)--(2.866,3.688)%
  --(2.877,3.699)--(2.888,3.709)--(2.899,3.720)--(2.910,3.730)--(2.921,3.740)--(2.932,3.750)%
  --(2.943,3.761)--(2.954,3.771)--(2.965,3.781)--(2.976,3.791)--(2.987,3.801)--(2.998,3.811)%
  --(3.009,3.821)--(3.020,3.830)--(3.031,3.840)--(3.042,3.850)--(3.053,3.860)--(3.064,3.869)%
  --(3.075,3.879)--(3.086,3.889)--(3.097,3.898)--(3.108,3.908)--(3.119,3.917)--(3.130,3.927)%
  --(3.141,3.936)--(3.152,3.945)--(3.163,3.955)--(3.174,3.964)--(3.185,3.973)--(3.196,3.982)%
  --(3.207,3.992)--(3.218,4.001)--(3.229,4.010)--(3.240,4.019)--(3.251,4.028)--(3.262,4.037)%
  --(3.273,4.046)--(3.285,4.055)--(3.296,4.064)--(3.307,4.073)--(3.318,4.082)--(3.329,4.090)%
  --(3.340,4.099)--(3.351,4.108)--(3.362,4.117)--(3.373,4.125)--(3.384,4.134)--(3.395,4.143)%
  --(3.406,4.151)--(3.417,4.160)--(3.428,4.168)--(3.439,4.177)--(3.450,4.185)--(3.461,4.194)%
  --(3.472,4.202)--(3.483,4.210)--(3.494,4.219)--(3.505,4.227)--(3.516,4.235)--(3.527,4.244)%
  --(3.538,4.252)--(3.549,4.260)--(3.560,4.268)--(3.571,4.276)--(3.582,4.285)--(3.593,4.293)%
  --(3.604,4.301)--(3.615,4.309)--(3.626,4.317)--(3.637,4.325)--(3.648,4.333)--(3.659,4.341)%
  --(3.670,4.349)--(3.681,4.357)--(3.692,4.365)--(3.704,4.372)--(3.715,4.380)--(3.726,4.388)%
  --(3.737,4.396)--(3.748,4.404)--(3.759,4.411)--(3.770,4.419)--(3.781,4.427)--(3.792,4.434)%
  --(3.803,4.442)--(3.814,4.450)--(3.825,4.457)--(3.836,4.465)--(3.847,4.472)--(3.858,4.480)%
  --(3.869,4.487)--(3.880,4.495)--(3.891,4.502)--(3.902,4.510)--(3.913,4.517)--(3.924,4.525)%
  --(3.935,4.532)--(3.946,4.539)--(3.957,4.547)--(3.968,4.554)--(3.979,4.561)--(3.990,4.569)%
  --(4.001,4.576)--(4.012,4.583)--(4.023,4.590)--(4.034,4.598)--(4.045,4.605)--(4.056,4.612)%
  --(4.067,4.619)--(4.078,4.626)--(4.089,4.633)--(4.100,4.640)--(4.111,4.647)--(4.123,4.655)%
  --(4.134,4.662)--(4.145,4.669)--(4.156,4.676)--(4.167,4.683)--(4.178,4.689)--(4.189,4.696)%
  --(4.200,4.703)--(4.211,4.710)--(4.222,4.717)--(4.233,4.724)--(4.244,4.731)--(4.255,4.738)%
  --(4.266,4.745)--(4.277,4.751)--(4.288,4.758)--(4.299,4.765)--(4.310,4.772)--(4.321,4.778)%
  --(4.332,4.785)--(4.343,4.792)--(4.354,4.798)--(4.365,4.805)--(4.376,4.812)--(4.387,4.818)%
  --(4.398,4.825)--(4.409,4.832)--(4.420,4.838)--(4.431,4.845)--(4.442,4.851)--(4.453,4.858)%
  --(4.464,4.864)--(4.475,4.871)--(4.486,4.877)--(4.497,4.884)--(4.508,4.890)--(4.519,4.897)%
  --(4.530,4.903)--(4.542,4.910)--(4.553,4.916)--(4.564,4.923)--(4.575,4.929)--(4.586,4.935)%
  --(4.597,4.942)--(4.608,4.948)--(4.619,4.954)--(4.630,4.961)--(4.641,4.967)--(4.652,4.973)%
  --(4.663,4.979)--(4.674,4.986)--(4.685,4.992)--(4.696,4.998)--(4.707,5.004)--(4.718,5.011)%
  --(4.729,5.017)--(4.740,5.023)--(4.751,5.029)--(4.762,5.035)--(4.773,5.041)--(4.784,5.047)%
  --(4.795,5.054)--(4.806,5.060)--(4.817,5.066)--(4.828,5.072)--(4.839,5.078)--(4.850,5.084)%
  --(4.861,5.090)--(4.872,5.096)--(4.883,5.102)--(4.894,5.108)--(4.905,5.114)--(4.916,5.120)%
  --(4.927,5.126)--(4.938,5.132)--(4.950,5.138)--(4.961,5.144)--(4.972,5.150)--(4.983,5.155)%
  --(4.994,5.161)--(5.005,5.167)--(5.016,5.173)--(5.027,5.179)--(5.038,5.185)--(5.049,5.191)%
  --(5.060,5.196)--(5.071,5.202)--(5.082,5.208)--(5.093,5.214)--(5.104,5.220)--(5.115,5.225)%
  --(5.126,5.231)--(5.137,5.237)--(5.148,5.243)--(5.159,5.248)--(5.170,5.254)--(5.181,5.260)%
  --(5.192,5.265)--(5.203,5.271)--(5.214,5.277)--(5.225,5.282)--(5.236,5.288)--(5.247,5.294)%
  --(5.258,5.299)--(5.269,5.305)--(5.280,5.310)--(5.291,5.316)--(5.302,5.322)--(5.313,5.327)%
  --(5.324,5.333)--(5.335,5.338)--(5.346,5.344)--(5.357,5.349)--(5.369,5.355)--(5.380,5.360)%
  --(5.391,5.366)--(5.402,5.371)--(5.413,5.377)--(5.424,5.382)--(5.435,5.388)--(5.446,5.393)%
  --(5.457,5.399)--(5.468,5.404)--(5.479,5.409)--(5.490,5.415)--(5.501,5.420)--(5.512,5.426)%
  --(5.523,5.431)--(5.534,5.436)--(5.545,5.442)--(5.556,5.447)--(5.567,5.452)--(5.578,5.458)%
  --(5.589,5.463)--(5.600,5.468)--(5.611,5.474)--(5.622,5.479)--(5.633,5.484)--(5.644,5.490)%
  --(5.655,5.495)--(5.666,5.500)--(5.677,5.505)--(5.688,5.511)--(5.699,5.516)--(5.710,5.521)%
  --(5.721,5.526)--(5.732,5.531)--(5.743,5.537)--(5.754,5.542)--(5.765,5.547)--(5.776,5.552)%
  --(5.788,5.557)--(5.799,5.562)--(5.810,5.568)--(5.821,5.573)--(5.832,5.578)--(5.843,5.583)%
  --(5.854,5.588)--(5.865,5.593)--(5.876,5.598)--(5.887,5.603)--(5.898,5.609)--(5.909,5.614)%
  --(5.920,5.619)--(5.931,5.624)--(5.942,5.629)--(5.953,5.634)--(5.964,5.639)--(5.975,5.644)%
  --(5.986,5.649)--(5.997,5.654)--(6.008,5.659)--(6.019,5.664)--(6.030,5.669)--(6.041,5.674)%
  --(6.052,5.679)--(6.063,5.684)--(6.074,5.689)--(6.085,5.694)--(6.096,5.699)--(6.107,5.704)%
  --(6.118,5.708)--(6.129,5.713)--(6.140,5.718)--(6.151,5.723)--(6.162,5.728)--(6.173,5.733)%
  --(6.184,5.738)--(6.195,5.743)--(6.207,5.748)--(6.218,5.752)--(6.229,5.757)--(6.240,5.762)%
  --(6.251,5.767)--(6.262,5.772)--(6.273,5.777)--(6.284,5.781)--(6.295,5.786)--(6.306,5.791)%
  --(6.317,5.796)--(6.328,5.801)--(6.339,5.805)--(6.350,5.810)--(6.361,5.815)--(6.372,5.820)%
  --(6.383,5.824)--(6.394,5.829)--(6.405,5.834)--(6.416,5.839)--(6.427,5.843)--(6.438,5.848)%
  --(6.449,5.853)--(6.460,5.857)--(6.471,5.862)--(6.482,5.867)--(6.493,5.872)--(6.504,5.876)%
  --(6.515,5.881)--(6.526,5.886)--(6.537,5.890)--(6.548,5.895)--(6.559,5.899)--(6.570,5.904)%
  --(6.581,5.909)--(6.592,5.913)--(6.603,5.918)--(6.614,5.923)--(6.626,5.927)--(6.637,5.932)%
  --(6.648,5.936)--(6.659,5.941)--(6.670,5.946)--(6.681,5.950)--(6.692,5.955)--(6.703,5.959)%
  --(6.714,5.964)--(6.725,5.968)--(6.736,5.973)--(6.747,5.977)--(6.758,5.982)--(6.769,5.986)%
  --(6.780,5.991)--(6.791,5.996)--(6.802,6.000)--(6.813,6.005)--(6.824,6.009)--(6.835,6.014)%
  --(6.846,6.018)--(6.857,6.022)--(6.868,6.027)--(6.879,6.031)--(6.890,6.036)--(6.901,6.040)%
  --(6.912,6.045)--(6.923,6.049)--(6.934,6.054)--(6.945,6.058)--(6.956,6.062)--(6.967,6.067)%
  --(6.978,6.071)--(6.989,6.076)--(7.000,6.080)--(7.011,6.084)--(7.022,6.089)--(7.033,6.093)%
  --(7.045,6.098)--(7.056,6.102)--(7.067,6.106)--(7.078,6.111)--(7.089,6.115)--(7.100,6.119)%
  --(7.111,6.124)--(7.122,6.128)--(7.133,6.132)--(7.144,6.137)--(7.155,6.141)--(7.166,6.145)%
  --(7.177,6.150)--(7.188,6.154)--(7.199,6.158)--(7.210,6.163)--(7.221,6.167)--(7.232,6.171)%
  --(7.243,6.175)--(7.254,6.180)--(7.265,6.184)--(7.276,6.188)--(7.287,6.193)--(7.298,6.197)%
  --(7.309,6.201)--(7.320,6.205)--(7.331,6.210)--(7.342,6.214)--(7.353,6.218)--(7.364,6.222)%
  --(7.375,6.226)--(7.386,6.231)--(7.397,6.235)--(7.408,6.239)--(7.419,6.243)--(7.430,6.247)%
  --(7.441,6.252)--(7.452,6.256)--(7.464,6.260)--(7.475,6.264)--(7.486,6.268)--(7.497,6.272)%
  --(7.508,6.277)--(7.519,6.281)--(7.530,6.285)--(7.541,6.289)--(7.552,6.293)--(7.563,6.297)%
  --(7.574,6.301)--(7.585,6.306)--(7.596,6.310)--(7.607,6.314)--(7.618,6.318)--(7.629,6.322)%
  --(7.640,6.326)--(7.651,6.330)--(7.662,6.334)--(7.673,6.338)--(7.684,6.342)--(7.695,6.346)%
  --(7.706,6.351)--(7.717,6.355)--(7.728,6.359)--(7.739,6.363)--(7.750,6.367)--(7.761,6.371)%
  --(7.772,6.375)--(7.783,6.379)--(7.794,6.383)--(7.805,6.387)--(7.816,6.391)--(7.827,6.395)%
  --(7.838,6.399)--(7.849,6.403)--(7.860,6.407)--(7.871,6.411)--(7.883,6.415)--(7.894,6.419)%
  --(7.905,6.423)--(7.916,6.427)--(7.927,6.431)--(7.938,6.435)--(7.949,6.439)--(7.960,6.443)%
  --(7.971,6.447)--(7.982,6.451)--(7.993,6.455)--(8.004,6.459)--(8.015,6.463)--(8.026,6.467)%
  --(8.037,6.470)--(8.048,6.474)--(8.059,6.478)--(8.070,6.482)--(8.081,6.486)--(8.092,6.490)%
  --(8.103,6.494)--(8.114,6.498)--(8.125,6.502)--(8.136,6.506)--(8.147,6.510)--(8.158,6.513)%
  --(8.169,6.517)--(8.180,6.521)--(8.191,6.525)--(8.202,6.529)--(8.213,6.533)--(8.224,6.537)%
  --(8.235,6.541)--(8.246,6.544)--(8.257,6.548)--(8.268,6.552)--(8.279,6.556)--(8.290,6.560)%
  --(8.302,6.564)--(8.313,6.567)--(8.324,6.571)--(8.335,6.575)--(8.346,6.579)--(8.357,6.583)%
  --(8.368,6.587)--(8.379,6.590)--(8.390,6.594)--(8.401,6.598)--(8.412,6.602)--(8.423,6.606)%
  --(8.434,6.609)--(8.445,6.613)--(8.456,6.617)--(8.467,6.621)--(8.478,6.624)--(8.489,6.628)%
  --(8.500,6.632)--(8.511,6.636)--(8.522,6.639)--(8.533,6.643)--(8.544,6.647)--(8.555,6.651)%
  --(8.566,6.654)--(8.577,6.658)--(8.588,6.662)--(8.599,6.666)--(8.610,6.669)--(8.621,6.673)%
  --(8.632,6.677)--(8.643,6.681)--(8.654,6.684)--(8.665,6.688)--(8.676,6.692)--(8.687,6.695)%
  --(8.698,6.699)--(8.709,6.703)--(8.721,6.706)--(8.732,6.710)--(8.743,6.714)--(8.754,6.717)%
  --(8.765,6.721)--(8.776,6.725)--(8.787,6.728)--(8.798,6.732)--(8.809,6.736)--(8.820,6.739)%
  --(8.831,6.743)--(8.842,6.747)--(8.853,6.750)--(8.864,6.754)--(8.875,6.758)--(8.886,6.761)%
  --(8.897,6.765)--(8.908,6.769)--(8.919,6.772)--(8.930,6.776)--(8.941,6.779)--(8.952,6.783)%
  --(8.963,6.787)--(8.974,6.790)--(8.985,6.794)--(8.996,6.797)--(9.007,6.801)--(9.018,6.805)%
  --(9.029,6.808)--(9.040,6.812)--(9.051,6.815)--(9.062,6.819)--(9.073,6.822)--(9.084,6.826)%
  --(9.095,6.830)--(9.106,6.833)--(9.117,6.837)--(9.129,6.840)--(9.140,6.844)--(9.151,6.847)%
  --(9.162,6.851)--(9.173,6.855)--(9.184,6.858)--(9.195,6.862)--(9.206,6.865)--(9.217,6.869)%
  --(9.228,6.872)--(9.239,6.876)--(9.250,6.879)--(9.261,6.883)--(9.272,6.886)--(9.283,6.890)%
  --(9.294,6.893)--(9.305,6.897)--(9.316,6.900)--(9.327,6.904)--(9.338,6.907)--(9.349,6.911)%
  --(9.360,6.914)--(9.371,6.918)--(9.382,6.921)--(9.393,6.925)--(9.404,6.928)--(9.415,6.932)%
  --(9.426,6.935)--(9.437,6.939)--(9.448,6.942)--(9.459,6.946)--(9.470,6.949)--(9.481,6.952)%
  --(9.492,6.956)--(9.503,6.959)--(9.514,6.963)--(9.525,6.966)--(9.536,6.970)--(9.548,6.973)%
  --(9.559,6.977)--(9.570,6.980)--(9.581,6.983)--(9.592,6.987)--(9.603,6.990)--(9.614,6.994)%
  --(9.625,6.997)--(9.636,7.000)--(9.647,7.004)--(9.658,7.007)--(9.669,7.011)--(9.680,7.014)%
  --(9.691,7.017)--(9.702,7.021)--(9.713,7.024)--(9.724,7.028)--(9.735,7.031)--(9.746,7.034)%
  --(9.757,7.038)--(9.768,7.041)--(9.779,7.045)--(9.790,7.048)--(9.801,7.051)--(9.812,7.055)%
  --(9.823,7.058)--(9.834,7.061)--(9.845,7.065)--(9.856,7.068)--(9.867,7.071)--(9.878,7.075)%
  --(9.889,7.078)--(9.900,7.081)--(9.911,7.085)--(9.922,7.088)--(9.933,7.091)--(9.944,7.095)%
  --(9.955,7.098)--(9.967,7.101)--(9.978,7.105)--(9.989,7.108)--(10.000,7.111)--(10.011,7.115)%
  --(10.022,7.118)--(10.033,7.121)--(10.044,7.125)--(10.055,7.128)--(10.066,7.131)--(10.077,7.135)%
  --(10.088,7.138)--(10.099,7.141)--(10.110,7.144)--(10.121,7.148)--(10.132,7.151)--(10.143,7.154)%
  --(10.154,7.158)--(10.165,7.161)--(10.176,7.164)--(10.187,7.167)--(10.198,7.171)--(10.209,7.174)%
  --(10.220,7.177)--(10.231,7.180)--(10.242,7.184)--(10.253,7.187)--(10.264,7.190)--(10.275,7.193)%
  --(10.286,7.197)--(10.297,7.200)--(10.308,7.203)--(10.319,7.206)--(10.330,7.210)--(10.341,7.213)%
  --(10.352,7.216)--(10.363,7.219)--(10.374,7.222)--(10.386,7.226)--(10.397,7.229)--(10.408,7.232)%
  --(10.419,7.235)--(10.430,7.239)--(10.441,7.242)--(10.452,7.245)--(10.463,7.248)--(10.474,7.251)%
  --(10.485,7.255)--(10.496,7.258)--(10.507,7.261)--(10.518,7.264)--(10.529,7.267)--(10.540,7.271)%
  --(10.551,7.274)--(10.562,7.277)--(10.573,7.280)--(10.584,7.283)--(10.595,7.286)--(10.606,7.290)%
  --(10.617,7.293)--(10.628,7.296)--(10.639,7.299)--(10.650,7.302)--(10.661,7.305)--(10.672,7.309)%
  --(10.683,7.312)--(10.694,7.315)--(10.705,7.318)--(10.716,7.321)--(10.727,7.324)--(10.738,7.327)%
  --(10.749,7.331)--(10.760,7.334)--(10.771,7.337)--(10.782,7.340)--(10.793,7.343)--(10.805,7.346)%
  --(10.816,7.349)--(10.827,7.352)--(10.838,7.356)--(10.849,7.359)--(10.860,7.362)--(10.871,7.365)%
  --(10.882,7.368)--(10.893,7.371)--(10.904,7.374)--(10.915,7.377)--(10.926,7.380)--(10.937,7.383)%
  --(10.948,7.387)--(10.959,7.390)--(10.970,7.393)--(10.981,7.396)--(10.992,7.399)--(11.003,7.402)%
  --(11.014,7.405)--(11.025,7.408)--(11.036,7.411)--(11.047,7.414)--(11.058,7.417)--(11.069,7.420)%
  --(11.080,7.424)--(11.091,7.427)--(11.102,7.430)--(11.113,7.433)--(11.124,7.436)--(11.135,7.439)%
  --(11.146,7.442)--(11.157,7.445)--(11.168,7.448)--(11.179,7.451)--(11.190,7.454)--(11.201,7.457)%
  --(11.212,7.460)--(11.224,7.463)--(11.235,7.466)--(11.246,7.469)--(11.257,7.472)--(11.268,7.475)%
  --(11.279,7.478)--(11.290,7.481)--(11.301,7.484)--(11.312,7.487)--(11.323,7.490)--(11.334,7.493)%
  --(11.345,7.496)--(11.356,7.499)--(11.367,7.502)--(11.378,7.505)--(11.389,7.508)--(11.400,7.511)%
  --(11.411,7.514)--(11.422,7.517)--(11.433,7.520)--(11.444,7.523)--(11.455,7.526)--(11.466,7.529)%
  --(11.477,7.532)--(11.488,7.535)--(11.499,7.538)--(11.510,7.541)--(11.521,7.544)--(11.532,7.547)%
  --(11.543,7.550)--(11.554,7.553)--(11.565,7.556)--(11.576,7.559)--(11.587,7.562)--(11.598,7.565)%
  --(11.609,7.568)--(11.620,7.571)--(11.631,7.574)--(11.643,7.577)--(11.654,7.580)--(11.665,7.583)%
  --(11.676,7.586)--(11.687,7.589)--(11.698,7.592)--(11.709,7.595)--(11.720,7.598)--(11.731,7.600)%
  --(11.742,7.603)--(11.753,7.606)--(11.764,7.609)--(11.775,7.612)--(11.786,7.615)--(11.797,7.618)%
  --(11.808,7.621)--(11.819,7.624)--(11.830,7.627)--(11.841,7.630)--(11.852,7.633)--(11.863,7.636)%
  --(11.874,7.638)--(11.885,7.641)--(11.896,7.644)--(11.907,7.647)--(11.918,7.650)--(11.929,7.653)%
  --(11.940,7.656)--(11.951,7.659)--(11.962,7.662)--(11.973,7.665)--(11.984,7.667)--(11.995,7.670)%
  --(12.006,7.673)--(12.017,7.676)--(12.028,7.679)--(12.039,7.682)--(12.050,7.685)--(12.062,7.688)%
  --(12.073,7.690)--(12.084,7.693)--(12.095,7.696)--(12.106,7.699)--(12.117,7.702)--(12.128,7.705)%
  --(12.139,7.708)--(12.150,7.711)--(12.161,7.713)--(12.172,7.716)--(12.183,7.719)--(12.194,7.722)%
  --(12.205,7.725)--(12.216,7.728)--(12.227,7.731)--(12.238,7.733)--(12.249,7.736)--(12.260,7.739)%
  --(12.271,7.742)--(12.282,7.745)--(12.293,7.748)--(12.304,7.750)--(12.315,7.753)--(12.326,7.756)%
  --(12.337,7.759)--(12.348,7.762)--(12.359,7.765)--(12.370,7.767)--(12.381,7.770)--(12.392,7.773)%
  --(12.403,7.776)--(12.414,7.779)--(12.425,7.781)--(12.436,7.784)--(12.447,7.787);
\gpcolor{color=gp lt color border}
\draw[gp path] (1.320,8.631)--(1.320,0.985)--(13.447,0.985)--(13.447,8.631)--cycle;
%% coordinates of the plot area
\gpdefrectangularnode{gp plot 1}{\pgfpoint{1.320cm}{0.985cm}}{\pgfpoint{13.447cm}{8.631cm}}
\end{tikzpicture}
%% gnuplot variables

	\caption{Gráfico do momento de Fermi $p_F$. \protect[Parameters: NJL $\rm{D}_1$, $m_0 = \np[MeV]{5.6}$]}
	\label{Fig:fermi_momentum_NJL-Buballa_Set_1}
\end{figure*}

\begin{figure*}
	\begin{tikzpicture}[gnuplot]
%% generated with GNUPLOT 5.0p2 (Lua 5.2; terminal rev. 99, script rev. 100)
%% Tue Apr  5 14:26:13 2016
\path (0.000,0.000) rectangle (14.000,9.000);
\gpcolor{color=gp lt color border}
\gpsetlinetype{gp lt border}
\gpsetdashtype{gp dt solid}
\gpsetlinewidth{1.00}
\draw[gp path] (1.504,0.985)--(1.684,0.985);
\draw[gp path] (13.447,0.985)--(13.267,0.985);
\node[gp node right] at (1.320,0.985) {$-100$};
\draw[gp path] (1.504,2.259)--(1.684,2.259);
\draw[gp path] (13.447,2.259)--(13.267,2.259);
\node[gp node right] at (1.320,2.259) {$0$};
\draw[gp path] (1.504,3.534)--(1.684,3.534);
\draw[gp path] (13.447,3.534)--(13.267,3.534);
\node[gp node right] at (1.320,3.534) {$100$};
\draw[gp path] (1.504,4.808)--(1.684,4.808);
\draw[gp path] (13.447,4.808)--(13.267,4.808);
\node[gp node right] at (1.320,4.808) {$200$};
\draw[gp path] (1.504,6.082)--(1.684,6.082);
\draw[gp path] (13.447,6.082)--(13.267,6.082);
\node[gp node right] at (1.320,6.082) {$300$};
\draw[gp path] (1.504,7.357)--(1.684,7.357);
\draw[gp path] (13.447,7.357)--(13.267,7.357);
\node[gp node right] at (1.320,7.357) {$400$};
\draw[gp path] (1.504,8.631)--(1.684,8.631);
\draw[gp path] (13.447,8.631)--(13.267,8.631);
\node[gp node right] at (1.320,8.631) {$500$};
\draw[gp path] (1.504,0.985)--(1.504,1.165);
\draw[gp path] (1.504,8.631)--(1.504,8.451);
\node[gp node center] at (1.504,0.677) {$0$};
\draw[gp path] (2.698,0.985)--(2.698,1.165);
\draw[gp path] (2.698,8.631)--(2.698,8.451);
\node[gp node center] at (2.698,0.677) {$100$};
\draw[gp path] (3.893,0.985)--(3.893,1.165);
\draw[gp path] (3.893,8.631)--(3.893,8.451);
\node[gp node center] at (3.893,0.677) {$200$};
\draw[gp path] (5.087,0.985)--(5.087,1.165);
\draw[gp path] (5.087,8.631)--(5.087,8.451);
\node[gp node center] at (5.087,0.677) {$300$};
\draw[gp path] (6.281,0.985)--(6.281,1.165);
\draw[gp path] (6.281,8.631)--(6.281,8.451);
\node[gp node center] at (6.281,0.677) {$400$};
\draw[gp path] (7.476,0.985)--(7.476,1.165);
\draw[gp path] (7.476,8.631)--(7.476,8.451);
\node[gp node center] at (7.476,0.677) {$500$};
\draw[gp path] (8.670,0.985)--(8.670,1.165);
\draw[gp path] (8.670,8.631)--(8.670,8.451);
\node[gp node center] at (8.670,0.677) {$600$};
\draw[gp path] (9.864,0.985)--(9.864,1.165);
\draw[gp path] (9.864,8.631)--(9.864,8.451);
\node[gp node center] at (9.864,0.677) {$700$};
\draw[gp path] (11.058,0.985)--(11.058,1.165);
\draw[gp path] (11.058,8.631)--(11.058,8.451);
\node[gp node center] at (11.058,0.677) {$800$};
\draw[gp path] (12.253,0.985)--(12.253,1.165);
\draw[gp path] (12.253,8.631)--(12.253,8.451);
\node[gp node center] at (12.253,0.677) {$900$};
\draw[gp path] (13.447,0.985)--(13.447,1.165);
\draw[gp path] (13.447,8.631)--(13.447,8.451);
\node[gp node center] at (13.447,0.677) {$1000$};
\draw[gp path] (1.504,8.631)--(1.504,0.985)--(13.447,0.985)--(13.447,8.631)--cycle;
\node[gp node center,rotate=-270] at (0.246,4.808) {$F(m) = m - m_0 + 2G_S\rho_s$ (MeV)};
\node[gp node center] at (7.475,0.215) {$m$ (MeV)};
\gpcolor{rgb color={0.580,0.000,0.827}}
\draw[gp path] (1.516,2.255)--(1.528,2.252)--(1.540,2.248)--(1.552,2.244)--(1.564,2.240)%
  --(1.576,2.236)--(1.588,2.233)--(1.600,2.229)--(1.612,2.225)--(1.624,2.221)--(1.636,2.217)%
  --(1.647,2.214)--(1.659,2.210)--(1.671,2.206)--(1.683,2.202)--(1.695,2.199)--(1.707,2.195)%
  --(1.719,2.191)--(1.731,2.187)--(1.743,2.184)--(1.755,2.180)--(1.767,2.176)--(1.779,2.173)%
  --(1.791,2.169)--(1.803,2.165)--(1.815,2.162)--(1.827,2.158)--(1.839,2.155)--(1.851,2.151)%
  --(1.863,2.148)--(1.875,2.144)--(1.887,2.141)--(1.899,2.137)--(1.910,2.134)--(1.922,2.130)%
  --(1.934,2.127)--(1.946,2.123)--(1.958,2.120)--(1.970,2.117)--(1.982,2.113)--(1.994,2.110)%
  --(2.006,2.107)--(2.018,2.103)--(2.030,2.100)--(2.042,2.097)--(2.054,2.094)--(2.066,2.090)%
  --(2.078,2.087)--(2.090,2.084)--(2.102,2.081)--(2.114,2.078)--(2.126,2.075)--(2.138,2.072)%
  --(2.150,2.069)--(2.162,2.066)--(2.173,2.063)--(2.185,2.060)--(2.197,2.057)--(2.209,2.054)%
  --(2.221,2.051)--(2.233,2.048)--(2.245,2.045)--(2.257,2.042)--(2.269,2.040)--(2.281,2.037)%
  --(2.293,2.034)--(2.305,2.031)--(2.317,2.029)--(2.329,2.026)--(2.341,2.023)--(2.353,2.021)%
  --(2.365,2.018)--(2.377,2.016)--(2.389,2.013)--(2.401,2.011)--(2.413,2.008)--(2.425,2.006)%
  --(2.436,2.003)--(2.448,2.001)--(2.460,1.998)--(2.472,1.996)--(2.484,1.994)--(2.496,1.991)%
  --(2.508,1.989)--(2.520,1.987)--(2.532,1.985)--(2.544,1.983)--(2.556,1.980)--(2.568,1.978)%
  --(2.580,1.976)--(2.592,1.974)--(2.604,1.972)--(2.616,1.970)--(2.628,1.968)--(2.640,1.966)%
  --(2.652,1.964)--(2.664,1.962)--(2.676,1.961)--(2.688,1.959)--(2.699,1.957)--(2.711,1.955)%
  --(2.723,1.954)--(2.735,1.952)--(2.747,1.950)--(2.759,1.948)--(2.771,1.947)--(2.783,1.945)%
  --(2.795,1.944)--(2.807,1.942)--(2.819,1.941)--(2.831,1.939)--(2.843,1.938)--(2.855,1.936)%
  --(2.867,1.935)--(2.879,1.934)--(2.891,1.932)--(2.903,1.931)--(2.915,1.930)--(2.927,1.929)%
  --(2.939,1.927)--(2.951,1.926)--(2.963,1.925)--(2.974,1.924)--(2.986,1.923)--(2.998,1.922)%
  --(3.010,1.921)--(3.022,1.920)--(3.034,1.919)--(3.046,1.918)--(3.058,1.917)--(3.070,1.916)%
  --(3.082,1.916)--(3.094,1.915)--(3.106,1.914)--(3.118,1.913)--(3.130,1.913)--(3.142,1.912)%
  --(3.154,1.911)--(3.166,1.911)--(3.178,1.910)--(3.190,1.910)--(3.202,1.909)--(3.214,1.909)%
  --(3.226,1.908)--(3.237,1.908)--(3.249,1.908)--(3.261,1.907)--(3.273,1.907)--(3.285,1.907)%
  --(3.297,1.906)--(3.309,1.906)--(3.321,1.906)--(3.333,1.906)--(3.345,1.906)--(3.357,1.906)%
  --(3.369,1.906)--(3.381,1.906)--(3.393,1.906)--(3.405,1.906)--(3.417,1.906)--(3.429,1.906)%
  --(3.441,1.906)--(3.453,1.906)--(3.465,1.906)--(3.477,1.906)--(3.489,1.907)--(3.500,1.907)%
  --(3.512,1.907)--(3.524,1.908)--(3.536,1.908)--(3.548,1.908)--(3.560,1.909)--(3.572,1.909)%
  --(3.584,1.910)--(3.596,1.910)--(3.608,1.911)--(3.620,1.912)--(3.632,1.912)--(3.644,1.913)%
  --(3.656,1.914)--(3.668,1.914)--(3.680,1.915)--(3.692,1.916)--(3.704,1.917)--(3.716,1.917)%
  --(3.728,1.918)--(3.740,1.919)--(3.752,1.920)--(3.763,1.921)--(3.775,1.922)--(3.787,1.923)%
  --(3.799,1.924)--(3.811,1.925)--(3.823,1.926)--(3.835,1.928)--(3.847,1.929)--(3.859,1.930)%
  --(3.871,1.931)--(3.883,1.933)--(3.895,1.934)--(3.907,1.935)--(3.919,1.937)--(3.931,1.938)%
  --(3.943,1.939)--(3.955,1.941)--(3.967,1.942)--(3.979,1.944)--(3.991,1.945)--(4.003,1.947)%
  --(4.015,1.949)--(4.026,1.950)--(4.038,1.952)--(4.050,1.954)--(4.062,1.955)--(4.074,1.957)%
  --(4.086,1.959)--(4.098,1.961)--(4.110,1.963)--(4.122,1.965)--(4.134,1.966)--(4.146,1.968)%
  --(4.158,1.970)--(4.170,1.972)--(4.182,1.974)--(4.194,1.977)--(4.206,1.979)--(4.218,1.981)%
  --(4.230,1.983)--(4.242,1.985)--(4.254,1.987)--(4.266,1.990)--(4.278,1.992)--(4.290,1.994)%
  --(4.301,1.996)--(4.313,1.999)--(4.325,2.001)--(4.337,2.004)--(4.349,2.006)--(4.361,2.008)%
  --(4.373,2.011)--(4.385,2.013)--(4.397,2.016)--(4.409,2.019)--(4.421,2.021)--(4.433,2.024)%
  --(4.445,2.027)--(4.457,2.029)--(4.469,2.032)--(4.481,2.035)--(4.493,2.038)--(4.505,2.040)%
  --(4.517,2.043)--(4.529,2.046)--(4.541,2.049)--(4.553,2.052)--(4.564,2.055)--(4.576,2.058)%
  --(4.588,2.061)--(4.600,2.064)--(4.612,2.067)--(4.624,2.070)--(4.636,2.073)--(4.648,2.076)%
  --(4.660,2.080)--(4.672,2.083)--(4.684,2.086)--(4.696,2.089)--(4.708,2.092)--(4.720,2.096)%
  --(4.732,2.099)--(4.744,2.103)--(4.756,2.106)--(4.768,2.109)--(4.780,2.113)--(4.792,2.116)%
  --(4.804,2.120)--(4.816,2.123)--(4.827,2.127)--(4.839,2.130)--(4.851,2.134)--(4.863,2.138)%
  --(4.875,2.141)--(4.887,2.145)--(4.899,2.149)--(4.911,2.152)--(4.923,2.156)--(4.935,2.160)%
  --(4.947,2.164)--(4.959,2.168)--(4.971,2.172)--(4.983,2.175)--(4.995,2.179)--(5.007,2.183)%
  --(5.019,2.187)--(5.031,2.191)--(5.043,2.195)--(5.055,2.199)--(5.067,2.203)--(5.079,2.208)%
  --(5.090,2.212)--(5.102,2.216)--(5.114,2.220)--(5.126,2.224)--(5.138,2.228)--(5.150,2.233)%
  --(5.162,2.237)--(5.174,2.241)--(5.186,2.246)--(5.198,2.250)--(5.210,2.254)--(5.222,2.259)%
  --(5.234,2.263)--(5.246,2.268)--(5.258,2.272)--(5.270,2.277)--(5.282,2.281)--(5.294,2.286)%
  --(5.306,2.290)--(5.318,2.295)--(5.330,2.300)--(5.342,2.304)--(5.353,2.309)--(5.365,2.314)%
  --(5.377,2.318)--(5.389,2.323)--(5.401,2.328)--(5.413,2.333)--(5.425,2.337)--(5.437,2.342)%
  --(5.449,2.347)--(5.461,2.352)--(5.473,2.357)--(5.485,2.362)--(5.497,2.367)--(5.509,2.372)%
  --(5.521,2.377)--(5.533,2.382)--(5.545,2.387)--(5.557,2.392)--(5.569,2.397)--(5.581,2.402)%
  --(5.593,2.407)--(5.605,2.412)--(5.617,2.418)--(5.628,2.423)--(5.640,2.428)--(5.652,2.433)%
  --(5.664,2.439)--(5.676,2.444)--(5.688,2.449)--(5.700,2.455)--(5.712,2.460)--(5.724,2.465)%
  --(5.736,2.471)--(5.748,2.476)--(5.760,2.482)--(5.772,2.487)--(5.784,2.493)--(5.796,2.498)%
  --(5.808,2.504)--(5.820,2.509)--(5.832,2.515)--(5.844,2.520)--(5.856,2.526)--(5.868,2.532)%
  --(5.880,2.537)--(5.891,2.543)--(5.903,2.549)--(5.915,2.554)--(5.927,2.560)--(5.939,2.566)%
  --(5.951,2.572)--(5.963,2.578)--(5.975,2.583)--(5.987,2.589)--(5.999,2.595)--(6.011,2.601)%
  --(6.023,2.607)--(6.035,2.613)--(6.047,2.619)--(6.059,2.625)--(6.071,2.631)--(6.083,2.637)%
  --(6.095,2.643)--(6.107,2.649)--(6.119,2.655)--(6.131,2.661)--(6.143,2.667)--(6.154,2.673)%
  --(6.166,2.680)--(6.178,2.686)--(6.190,2.692)--(6.202,2.698)--(6.214,2.705)--(6.226,2.711)%
  --(6.238,2.717)--(6.250,2.723)--(6.262,2.730)--(6.274,2.736)--(6.286,2.742)--(6.298,2.749)%
  --(6.310,2.755)--(6.322,2.762)--(6.334,2.768)--(6.346,2.774)--(6.358,2.781)--(6.370,2.787)%
  --(6.382,2.794)--(6.394,2.800)--(6.406,2.807)--(6.417,2.814)--(6.429,2.820)--(6.441,2.827)%
  --(6.453,2.833)--(6.465,2.840)--(6.477,2.847)--(6.489,2.853)--(6.501,2.860)--(6.513,2.867)%
  --(6.525,2.874)--(6.537,2.880)--(6.549,2.887)--(6.561,2.894)--(6.573,2.901)--(6.585,2.908)%
  --(6.597,2.914)--(6.609,2.921)--(6.621,2.928)--(6.633,2.935)--(6.645,2.942)--(6.657,2.949)%
  --(6.669,2.956)--(6.680,2.963)--(6.692,2.970)--(6.704,2.977)--(6.716,2.984)--(6.728,2.991)%
  --(6.740,2.998)--(6.752,3.005)--(6.764,3.012)--(6.776,3.019)--(6.788,3.026)--(6.800,3.033)%
  --(6.812,3.041)--(6.824,3.048)--(6.836,3.055)--(6.848,3.062)--(6.860,3.069)--(6.872,3.077)%
  --(6.884,3.084)--(6.896,3.091)--(6.908,3.098)--(6.920,3.106)--(6.932,3.113)--(6.944,3.120)%
  --(6.955,3.128)--(6.967,3.135)--(6.979,3.142)--(6.991,3.150)--(7.003,3.157)--(7.015,3.165)%
  --(7.027,3.172)--(7.039,3.180)--(7.051,3.187)--(7.063,3.195)--(7.075,3.202)--(7.087,3.210)%
  --(7.099,3.217)--(7.111,3.225)--(7.123,3.232)--(7.135,3.240)--(7.147,3.248)--(7.159,3.255)%
  --(7.171,3.263)--(7.183,3.270)--(7.195,3.278)--(7.207,3.286)--(7.218,3.294)--(7.230,3.301)%
  --(7.242,3.309)--(7.254,3.317)--(7.266,3.324)--(7.278,3.332)--(7.290,3.340)--(7.302,3.348)%
  --(7.314,3.356)--(7.326,3.363)--(7.338,3.371)--(7.350,3.379)--(7.362,3.387)--(7.374,3.395)%
  --(7.386,3.403)--(7.398,3.411)--(7.410,3.419)--(7.422,3.427)--(7.434,3.435)--(7.446,3.443)%
  --(7.458,3.451)--(7.470,3.459)--(7.481,3.467)--(7.493,3.475)--(7.505,3.483)--(7.517,3.491)%
  --(7.529,3.499)--(7.541,3.507)--(7.553,3.515)--(7.565,3.523)--(7.577,3.531)--(7.589,3.539)%
  --(7.601,3.548)--(7.613,3.556)--(7.625,3.564)--(7.637,3.572)--(7.649,3.580)--(7.661,3.589)%
  --(7.673,3.597)--(7.685,3.605)--(7.697,3.613)--(7.709,3.622)--(7.721,3.630)--(7.733,3.638)%
  --(7.744,3.647)--(7.756,3.655)--(7.768,3.663)--(7.780,3.672)--(7.792,3.680)--(7.804,3.688)%
  --(7.816,3.697)--(7.828,3.705)--(7.840,3.714)--(7.852,3.722)--(7.864,3.731)--(7.876,3.739)%
  --(7.888,3.748)--(7.900,3.756)--(7.912,3.765)--(7.924,3.773)--(7.936,3.782)--(7.948,3.790)%
  --(7.960,3.799)--(7.972,3.807)--(7.984,3.816)--(7.996,3.824)--(8.007,3.833)--(8.019,3.842)%
  --(8.031,3.850)--(8.043,3.859)--(8.055,3.868)--(8.067,3.876)--(8.079,3.885)--(8.091,3.894)%
  --(8.103,3.902)--(8.115,3.911)--(8.127,3.920)--(8.139,3.928)--(8.151,3.937)--(8.163,3.946)%
  --(8.175,3.955)--(8.187,3.963)--(8.199,3.972)--(8.211,3.981)--(8.223,3.990)--(8.235,3.999)%
  --(8.247,4.008)--(8.259,4.016)--(8.271,4.025)--(8.282,4.034)--(8.294,4.043)--(8.306,4.052)%
  --(8.318,4.061)--(8.330,4.070)--(8.342,4.079)--(8.354,4.088)--(8.366,4.097)--(8.378,4.106)%
  --(8.390,4.115)--(8.402,4.123)--(8.414,4.132)--(8.426,4.142)--(8.438,4.151)--(8.450,4.160)%
  --(8.462,4.169)--(8.474,4.178)--(8.486,4.187)--(8.498,4.196)--(8.510,4.205)--(8.522,4.214)%
  --(8.534,4.223)--(8.545,4.232)--(8.557,4.241)--(8.569,4.250)--(8.581,4.260)--(8.593,4.269)%
  --(8.605,4.278)--(8.617,4.287)--(8.629,4.296)--(8.641,4.305)--(8.653,4.315)--(8.665,4.324)%
  --(8.677,4.333)--(8.689,4.342)--(8.701,4.352)--(8.713,4.361)--(8.725,4.370)--(8.737,4.379)%
  --(8.749,4.389)--(8.761,4.398)--(8.773,4.407)--(8.785,4.417)--(8.797,4.426)--(8.808,4.435)%
  --(8.820,4.445)--(8.832,4.454)--(8.844,4.463)--(8.856,4.473)--(8.868,4.482)--(8.880,4.492)%
  --(8.892,4.501)--(8.904,4.510)--(8.916,4.520)--(8.928,4.529)--(8.940,4.539)--(8.952,4.548)%
  --(8.964,4.558)--(8.976,4.567)--(8.988,4.577)--(9.000,4.586)--(9.012,4.596)--(9.024,4.605)%
  --(9.036,4.615)--(9.048,4.624)--(9.060,4.634)--(9.071,4.643)--(9.083,4.653)--(9.095,4.662)%
  --(9.107,4.672)--(9.119,4.681)--(9.131,4.691)--(9.143,4.701)--(9.155,4.710)--(9.167,4.720)%
  --(9.179,4.730)--(9.191,4.739)--(9.203,4.749)--(9.215,4.758)--(9.227,4.768)--(9.239,4.778)%
  --(9.251,4.787)--(9.263,4.797)--(9.275,4.807)--(9.287,4.817)--(9.299,4.826)--(9.311,4.836)%
  --(9.323,4.846)--(9.334,4.855)--(9.346,4.865)--(9.358,4.875)--(9.370,4.885)--(9.382,4.895)%
  --(9.394,4.904)--(9.406,4.914)--(9.418,4.924)--(9.430,4.934)--(9.442,4.944)--(9.454,4.953)%
  --(9.466,4.963)--(9.478,4.973)--(9.490,4.983)--(9.502,4.993)--(9.514,5.003)--(9.526,5.012)%
  --(9.538,5.022)--(9.550,5.032)--(9.562,5.042)--(9.574,5.052)--(9.586,5.062)--(9.598,5.072)%
  --(9.609,5.082)--(9.621,5.092)--(9.633,5.102)--(9.645,5.112)--(9.657,5.122)--(9.669,5.132)%
  --(9.681,5.142)--(9.693,5.152)--(9.705,5.162)--(9.717,5.172)--(9.729,5.182)--(9.741,5.192)%
  --(9.753,5.202)--(9.765,5.212)--(9.777,5.222)--(9.789,5.232)--(9.801,5.242)--(9.813,5.252)%
  --(9.825,5.262)--(9.837,5.272)--(9.849,5.282)--(9.861,5.292)--(9.872,5.302)--(9.884,5.312)%
  --(9.896,5.322)--(9.908,5.333)--(9.920,5.343)--(9.932,5.353)--(9.944,5.363)--(9.956,5.373)%
  --(9.968,5.383)--(9.980,5.393)--(9.992,5.404)--(10.004,5.414)--(10.016,5.424)--(10.028,5.434)%
  --(10.040,5.444)--(10.052,5.455)--(10.064,5.465)--(10.076,5.475)--(10.088,5.485)--(10.100,5.495)%
  --(10.112,5.506)--(10.124,5.516)--(10.135,5.526)--(10.147,5.536)--(10.159,5.547)--(10.171,5.557)%
  --(10.183,5.567)--(10.195,5.578)--(10.207,5.588)--(10.219,5.598)--(10.231,5.608)--(10.243,5.619)%
  --(10.255,5.629)--(10.267,5.639)--(10.279,5.650)--(10.291,5.660)--(10.303,5.670)--(10.315,5.681)%
  --(10.327,5.691)--(10.339,5.702)--(10.351,5.712)--(10.363,5.722)--(10.375,5.733)--(10.387,5.743)%
  --(10.398,5.753)--(10.410,5.764)--(10.422,5.774)--(10.434,5.785)--(10.446,5.795)--(10.458,5.806)%
  --(10.470,5.816)--(10.482,5.826)--(10.494,5.837)--(10.506,5.847)--(10.518,5.858)--(10.530,5.868)%
  --(10.542,5.879)--(10.554,5.889)--(10.566,5.900)--(10.578,5.910)--(10.590,5.921)--(10.602,5.931)%
  --(10.614,5.942)--(10.626,5.952)--(10.638,5.963)--(10.650,5.973)--(10.661,5.984)--(10.673,5.994)%
  --(10.685,6.005)--(10.697,6.015)--(10.709,6.026)--(10.721,6.037)--(10.733,6.047)--(10.745,6.058)%
  --(10.757,6.068)--(10.769,6.079)--(10.781,6.089)--(10.793,6.100)--(10.805,6.111)--(10.817,6.121)%
  --(10.829,6.132)--(10.841,6.143)--(10.853,6.153)--(10.865,6.164)--(10.877,6.174)--(10.889,6.185)%
  --(10.901,6.196)--(10.913,6.206)--(10.925,6.217)--(10.936,6.228)--(10.948,6.238)--(10.960,6.249)%
  --(10.972,6.260)--(10.984,6.270)--(10.996,6.281)--(11.008,6.292)--(11.020,6.302)--(11.032,6.313)%
  --(11.044,6.324)--(11.056,6.335)--(11.068,6.345)--(11.080,6.356)--(11.092,6.367)--(11.104,6.378)%
  --(11.116,6.388)--(11.128,6.399)--(11.140,6.410)--(11.152,6.421)--(11.164,6.431)--(11.176,6.442)%
  --(11.188,6.453)--(11.199,6.464)--(11.211,6.474)--(11.223,6.485)--(11.235,6.496)--(11.247,6.507)%
  --(11.259,6.518)--(11.271,6.528)--(11.283,6.539)--(11.295,6.550)--(11.307,6.561)--(11.319,6.572)%
  --(11.331,6.583)--(11.343,6.593)--(11.355,6.604)--(11.367,6.615)--(11.379,6.626)--(11.391,6.637)%
  --(11.403,6.648)--(11.415,6.659)--(11.427,6.669)--(11.439,6.680)--(11.451,6.691)--(11.462,6.702)%
  --(11.474,6.713)--(11.486,6.724)--(11.498,6.735)--(11.510,6.746)--(11.522,6.757)--(11.534,6.768)%
  --(11.546,6.779)--(11.558,6.789)--(11.570,6.800)--(11.582,6.811)--(11.594,6.822)--(11.606,6.833)%
  --(11.618,6.844)--(11.630,6.855)--(11.642,6.866)--(11.654,6.877)--(11.666,6.888)--(11.678,6.899)%
  --(11.690,6.910)--(11.702,6.921)--(11.714,6.932)--(11.725,6.943)--(11.737,6.954)--(11.749,6.965)%
  --(11.761,6.976)--(11.773,6.987)--(11.785,6.998)--(11.797,7.009)--(11.809,7.020)--(11.821,7.031)%
  --(11.833,7.042)--(11.845,7.053)--(11.857,7.064)--(11.869,7.075)--(11.881,7.086)--(11.893,7.097)%
  --(11.905,7.108)--(11.917,7.119)--(11.929,7.131)--(11.941,7.142)--(11.953,7.153)--(11.965,7.164)%
  --(11.977,7.175)--(11.988,7.186)--(12.000,7.197)--(12.012,7.208)--(12.024,7.219)--(12.036,7.230)%
  --(12.048,7.241)--(12.060,7.253)--(12.072,7.264)--(12.084,7.275)--(12.096,7.286)--(12.108,7.297)%
  --(12.120,7.308)--(12.132,7.319)--(12.144,7.330)--(12.156,7.342)--(12.168,7.353)--(12.180,7.364)%
  --(12.192,7.375)--(12.204,7.386)--(12.216,7.397)--(12.228,7.409)--(12.240,7.420)--(12.252,7.431)%
  --(12.263,7.442)--(12.275,7.453)--(12.287,7.464)--(12.299,7.476)--(12.311,7.487)--(12.323,7.498)%
  --(12.335,7.509)--(12.347,7.520)--(12.359,7.532)--(12.371,7.543)--(12.383,7.554)--(12.395,7.565)%
  --(12.407,7.577)--(12.419,7.588)--(12.431,7.599)--(12.443,7.610)--(12.455,7.622)--(12.467,7.633)%
  --(12.479,7.644)--(12.491,7.655)--(12.503,7.667)--(12.515,7.678)--(12.526,7.689)--(12.538,7.700)%
  --(12.550,7.712)--(12.562,7.723)--(12.574,7.734)--(12.586,7.745)--(12.598,7.757)--(12.610,7.768)%
  --(12.622,7.779)--(12.634,7.791)--(12.646,7.802)--(12.658,7.813)--(12.670,7.825)--(12.682,7.836)%
  --(12.694,7.847)--(12.706,7.858)--(12.718,7.870)--(12.730,7.881)--(12.742,7.892)--(12.754,7.904)%
  --(12.766,7.915)--(12.778,7.926)--(12.789,7.938)--(12.801,7.949)--(12.813,7.960)--(12.825,7.972)%
  --(12.837,7.983)--(12.849,7.995)--(12.861,8.006)--(12.873,8.017)--(12.885,8.029)--(12.897,8.040)%
  --(12.909,8.051)--(12.921,8.063)--(12.933,8.074)--(12.945,8.086)--(12.957,8.097)--(12.969,8.108)%
  --(12.981,8.120)--(12.993,8.131)--(13.005,8.143)--(13.017,8.154)--(13.029,8.165)--(13.041,8.177)%
  --(13.052,8.188)--(13.064,8.200)--(13.076,8.211)--(13.088,8.222)--(13.100,8.234)--(13.112,8.245)%
  --(13.124,8.257)--(13.136,8.268)--(13.148,8.280)--(13.160,8.291)--(13.172,8.302)--(13.184,8.314)%
  --(13.196,8.325)--(13.208,8.337)--(13.220,8.348)--(13.232,8.360)--(13.244,8.371)--(13.256,8.383)%
  --(13.268,8.394)--(13.280,8.406)--(13.292,8.417)--(13.304,8.429)--(13.315,8.440)--(13.327,8.452)%
  --(13.339,8.463)--(13.351,8.475)--(13.363,8.486)--(13.375,8.498)--(13.387,8.509)--(13.399,8.521)%
  --(13.411,8.532)--(13.423,8.544)--(13.435,8.555)--(13.447,8.567);
\gpcolor{rgb color={0.000,0.620,0.451}}
\draw[gp path] (1.504,2.259)--(1.625,2.259)--(1.745,2.259)--(1.866,2.259)--(1.987,2.259)%
  --(2.107,2.259)--(2.228,2.259)--(2.348,2.259)--(2.469,2.259)--(2.590,2.259)--(2.710,2.259)%
  --(2.831,2.259)--(2.952,2.259)--(3.072,2.259)--(3.193,2.259)--(3.314,2.259)--(3.434,2.259)%
  --(3.555,2.259)--(3.675,2.259)--(3.796,2.259)--(3.917,2.259)--(4.037,2.259)--(4.158,2.259)%
  --(4.279,2.259)--(4.399,2.259)--(4.520,2.259)--(4.641,2.259)--(4.761,2.259)--(4.882,2.259)%
  --(5.002,2.259)--(5.123,2.259)--(5.244,2.259)--(5.364,2.259)--(5.485,2.259)--(5.606,2.259)%
  --(5.726,2.259)--(5.847,2.259)--(5.968,2.259)--(6.088,2.259)--(6.209,2.259)--(6.329,2.259)%
  --(6.450,2.259)--(6.571,2.259)--(6.691,2.259)--(6.812,2.259)--(6.933,2.259)--(7.053,2.259)%
  --(7.174,2.259)--(7.295,2.259)--(7.415,2.259)--(7.536,2.259)--(7.656,2.259)--(7.777,2.259)%
  --(7.898,2.259)--(8.018,2.259)--(8.139,2.259)--(8.260,2.259)--(8.380,2.259)--(8.501,2.259)%
  --(8.622,2.259)--(8.742,2.259)--(8.863,2.259)--(8.983,2.259)--(9.104,2.259)--(9.225,2.259)%
  --(9.345,2.259)--(9.466,2.259)--(9.587,2.259)--(9.707,2.259)--(9.828,2.259)--(9.949,2.259)%
  --(10.069,2.259)--(10.190,2.259)--(10.310,2.259)--(10.431,2.259)--(10.552,2.259)--(10.672,2.259)%
  --(10.793,2.259)--(10.914,2.259)--(11.034,2.259)--(11.155,2.259)--(11.276,2.259)--(11.396,2.259)%
  --(11.517,2.259)--(11.637,2.259)--(11.758,2.259)--(11.879,2.259)--(11.999,2.259)--(12.120,2.259)%
  --(12.241,2.259)--(12.361,2.259)--(12.482,2.259)--(12.603,2.259)--(12.723,2.259)--(12.844,2.259)%
  --(12.964,2.259)--(13.085,2.259)--(13.206,2.259)--(13.326,2.259)--(13.447,2.259);
\gpcolor{color=gp lt color border}
\draw[gp path] (1.504,8.631)--(1.504,0.985)--(13.447,0.985)--(13.447,8.631)--cycle;
%% coordinates of the plot area
\gpdefrectangularnode{gp plot 1}{\pgfpoint{1.504cm}{0.985cm}}{\pgfpoint{13.447cm}{8.631cm}}
\end{tikzpicture}
%% gnuplot variables

	\caption{Gráfico da equação cuja raíz determina o valor de $m_{\rm{vac}}$. \protect[Parameters: NJL $\rm{D}_1$, $m_0 = \np[MeV]{5.6}$]}
	\label{Fig:vacuum_mass_equation_NJL-Buballa_Set_1}
\end{figure*}

\begin{figure*}
	\begin{tikzpicture}[gnuplot]
%% generated with GNUPLOT 5.0p2 (Lua 5.2; terminal rev. 99, script rev. 100)
%% Fri Mar  4 16:19:10 2016
\path (0.000,0.000) rectangle (14.000,9.000);
\gpcolor{color=gp lt color border}
\gpsetlinetype{gp lt border}
\gpsetdashtype{gp dt solid}
\gpsetlinewidth{1.00}
\draw[gp path] (1.504,0.985)--(1.684,0.985);
\draw[gp path] (13.447,0.985)--(13.267,0.985);
\node[gp node right] at (1.320,0.985) {$-500$};
\draw[gp path] (1.504,2.514)--(1.684,2.514);
\draw[gp path] (13.447,2.514)--(13.267,2.514);
\node[gp node right] at (1.320,2.514) {$0$};
\draw[gp path] (1.504,4.043)--(1.684,4.043);
\draw[gp path] (13.447,4.043)--(13.267,4.043);
\node[gp node right] at (1.320,4.043) {$500$};
\draw[gp path] (1.504,5.573)--(1.684,5.573);
\draw[gp path] (13.447,5.573)--(13.267,5.573);
\node[gp node right] at (1.320,5.573) {$1000$};
\draw[gp path] (1.504,7.102)--(1.684,7.102);
\draw[gp path] (13.447,7.102)--(13.267,7.102);
\node[gp node right] at (1.320,7.102) {$1500$};
\draw[gp path] (1.504,8.631)--(1.684,8.631);
\draw[gp path] (13.447,8.631)--(13.267,8.631);
\node[gp node right] at (1.320,8.631) {$2000$};
\draw[gp path] (1.504,0.985)--(1.504,1.165);
\draw[gp path] (1.504,8.631)--(1.504,8.451);
\node[gp node center] at (1.504,0.677) {$0$};
\draw[gp path] (2.698,0.985)--(2.698,1.165);
\draw[gp path] (2.698,8.631)--(2.698,8.451);
\node[gp node center] at (2.698,0.677) {$200$};
\draw[gp path] (3.893,0.985)--(3.893,1.165);
\draw[gp path] (3.893,8.631)--(3.893,8.451);
\node[gp node center] at (3.893,0.677) {$400$};
\draw[gp path] (5.087,0.985)--(5.087,1.165);
\draw[gp path] (5.087,8.631)--(5.087,8.451);
\node[gp node center] at (5.087,0.677) {$600$};
\draw[gp path] (6.281,0.985)--(6.281,1.165);
\draw[gp path] (6.281,8.631)--(6.281,8.451);
\node[gp node center] at (6.281,0.677) {$800$};
\draw[gp path] (7.476,0.985)--(7.476,1.165);
\draw[gp path] (7.476,8.631)--(7.476,8.451);
\node[gp node center] at (7.476,0.677) {$1000$};
\draw[gp path] (8.670,0.985)--(8.670,1.165);
\draw[gp path] (8.670,8.631)--(8.670,8.451);
\node[gp node center] at (8.670,0.677) {$1200$};
\draw[gp path] (9.864,0.985)--(9.864,1.165);
\draw[gp path] (9.864,8.631)--(9.864,8.451);
\node[gp node center] at (9.864,0.677) {$1400$};
\draw[gp path] (11.058,0.985)--(11.058,1.165);
\draw[gp path] (11.058,8.631)--(11.058,8.451);
\node[gp node center] at (11.058,0.677) {$1600$};
\draw[gp path] (12.253,0.985)--(12.253,1.165);
\draw[gp path] (12.253,8.631)--(12.253,8.451);
\node[gp node center] at (12.253,0.677) {$1800$};
\draw[gp path] (13.447,0.985)--(13.447,1.165);
\draw[gp path] (13.447,8.631)--(13.447,8.451);
\node[gp node center] at (13.447,0.677) {$2000$};
\draw[gp path] (1.504,8.631)--(1.504,0.985)--(13.447,0.985)--(13.447,8.631)--cycle;
\node[gp node center,rotate=-270] at (0.246,4.808) {$f(M) = M + G_s\rho_s - 2G_{sv}\rho_s\rho^2$ (MeV)};
\node[gp node center] at (7.475,0.215) {$M$ (MeV)};
\node[gp node left] at (2.972,8.297) {$\rho_{\rm{min}}$};
\gpcolor{rgb color={0.580,0.000,0.827}}
\draw[gp path] (1.872,8.297)--(2.788,8.297);
\draw[gp path] (1.507,2.510)--(1.510,2.506)--(1.513,2.503)--(1.516,2.499)--(1.519,2.495)%
  --(1.522,2.491)--(1.525,2.487)--(1.528,2.483)--(1.531,2.479)--(1.534,2.476)--(1.537,2.472)%
  --(1.540,2.468)--(1.543,2.464)--(1.546,2.460)--(1.549,2.456)--(1.552,2.452)--(1.555,2.449)%
  --(1.558,2.445)--(1.561,2.441)--(1.564,2.437)--(1.567,2.433)--(1.570,2.429)--(1.573,2.425)%
  --(1.576,2.422)--(1.579,2.418)--(1.582,2.414)--(1.585,2.410)--(1.588,2.406)--(1.591,2.402)%
  --(1.594,2.399)--(1.597,2.395)--(1.600,2.391)--(1.603,2.387)--(1.606,2.383)--(1.609,2.380)%
  --(1.611,2.376)--(1.614,2.372)--(1.617,2.368)--(1.620,2.364)--(1.623,2.360)--(1.626,2.357)%
  --(1.629,2.353)--(1.632,2.349)--(1.635,2.345)--(1.638,2.341)--(1.641,2.338)--(1.644,2.334)%
  --(1.647,2.330)--(1.650,2.326)--(1.653,2.323)--(1.656,2.319)--(1.659,2.315)--(1.662,2.311)%
  --(1.665,2.308)--(1.668,2.304)--(1.671,2.300)--(1.674,2.296)--(1.677,2.293)--(1.680,2.289)%
  --(1.683,2.285)--(1.686,2.281)--(1.689,2.278)--(1.692,2.274)--(1.695,2.270)--(1.698,2.266)%
  --(1.701,2.263)--(1.704,2.259)--(1.707,2.255)--(1.710,2.252)--(1.713,2.248)--(1.716,2.244)%
  --(1.719,2.241)--(1.722,2.237)--(1.725,2.233)--(1.728,2.230)--(1.731,2.226)--(1.734,2.222)%
  --(1.737,2.219)--(1.740,2.215)--(1.743,2.211)--(1.746,2.208)--(1.749,2.204)--(1.752,2.200)%
  --(1.755,2.197)--(1.758,2.193)--(1.761,2.190)--(1.764,2.186)--(1.767,2.182)--(1.770,2.179)%
  --(1.773,2.175)--(1.776,2.172)--(1.779,2.168)--(1.782,2.165)--(1.785,2.161)--(1.788,2.157)%
  --(1.791,2.154)--(1.794,2.150)--(1.797,2.147)--(1.800,2.143)--(1.803,2.140)--(1.806,2.136)%
  --(1.809,2.133)--(1.812,2.129)--(1.815,2.126)--(1.818,2.122)--(1.820,2.119)--(1.823,2.115)%
  --(1.826,2.112)--(1.829,2.108)--(1.832,2.105)--(1.835,2.101)--(1.838,2.098)--(1.841,2.095)%
  --(1.844,2.091)--(1.847,2.088)--(1.850,2.084)--(1.853,2.081)--(1.856,2.077)--(1.859,2.074)%
  --(1.862,2.071)--(1.865,2.067)--(1.868,2.064)--(1.871,2.061)--(1.874,2.057)--(1.877,2.054)%
  --(1.880,2.050)--(1.883,2.047)--(1.886,2.044)--(1.889,2.041)--(1.892,2.037)--(1.895,2.034)%
  --(1.898,2.031)--(1.901,2.027)--(1.904,2.024)--(1.907,2.021)--(1.910,2.017)--(1.913,2.014)%
  --(1.916,2.011)--(1.919,2.008)--(1.922,2.004)--(1.925,2.001)--(1.928,1.998)--(1.931,1.995)%
  --(1.934,1.992)--(1.937,1.988)--(1.940,1.985)--(1.943,1.982)--(1.946,1.979)--(1.949,1.976)%
  --(1.952,1.972)--(1.955,1.969)--(1.958,1.966)--(1.961,1.963)--(1.964,1.960)--(1.967,1.957)%
  --(1.970,1.954)--(1.973,1.950)--(1.976,1.947)--(1.979,1.944)--(1.982,1.941)--(1.985,1.938)%
  --(1.988,1.935)--(1.991,1.932)--(1.994,1.929)--(1.997,1.926)--(2.000,1.923)--(2.003,1.920)%
  --(2.006,1.917)--(2.009,1.914)--(2.012,1.911)--(2.015,1.908)--(2.018,1.905)--(2.021,1.902)%
  --(2.024,1.899)--(2.027,1.896)--(2.029,1.893)--(2.032,1.890)--(2.035,1.887)--(2.038,1.884)%
  --(2.041,1.881)--(2.044,1.878)--(2.047,1.875)--(2.050,1.873)--(2.053,1.870)--(2.056,1.867)%
  --(2.059,1.864)--(2.062,1.861)--(2.065,1.858)--(2.068,1.855)--(2.071,1.852)--(2.074,1.850)%
  --(2.077,1.847)--(2.080,1.844)--(2.083,1.841)--(2.086,1.838)--(2.089,1.836)--(2.092,1.833)%
  --(2.095,1.830)--(2.098,1.827)--(2.101,1.825)--(2.104,1.822)--(2.107,1.819)--(2.110,1.816)%
  --(2.113,1.814)--(2.116,1.811)--(2.119,1.808)--(2.122,1.805)--(2.125,1.803)--(2.128,1.800)%
  --(2.131,1.797)--(2.134,1.795)--(2.137,1.792)--(2.140,1.789)--(2.143,1.787)--(2.146,1.784)%
  --(2.149,1.782)--(2.152,1.779)--(2.155,1.776)--(2.158,1.774)--(2.161,1.771)--(2.164,1.769)%
  --(2.167,1.766)--(2.170,1.764)--(2.173,1.761)--(2.176,1.758)--(2.179,1.756)--(2.182,1.753)%
  --(2.185,1.751)--(2.188,1.748)--(2.191,1.746)--(2.194,1.743)--(2.197,1.741)--(2.200,1.738)%
  --(2.203,1.736)--(2.206,1.733)--(2.209,1.731)--(2.212,1.729)--(2.215,1.726)--(2.218,1.724)%
  --(2.221,1.721)--(2.224,1.719)--(2.227,1.716)--(2.230,1.714)--(2.233,1.712)--(2.236,1.709)%
  --(2.238,1.707)--(2.241,1.705)--(2.244,1.702)--(2.247,1.700)--(2.250,1.698)--(2.253,1.695)%
  --(2.256,1.693)--(2.259,1.691)--(2.262,1.688)--(2.265,1.686)--(2.268,1.684)--(2.271,1.682)%
  --(2.274,1.679)--(2.277,1.677)--(2.280,1.675)--(2.283,1.673)--(2.286,1.670)--(2.289,1.668)%
  --(2.292,1.666)--(2.295,1.664)--(2.298,1.661)--(2.301,1.659)--(2.304,1.657)--(2.307,1.655)%
  --(2.310,1.653)--(2.313,1.651)--(2.316,1.648)--(2.319,1.646)--(2.322,1.644)--(2.325,1.642)%
  --(2.328,1.640)--(2.331,1.638)--(2.334,1.636)--(2.337,1.634)--(2.340,1.632)--(2.343,1.629)%
  --(2.346,1.627)--(2.349,1.625)--(2.352,1.623)--(2.355,1.621)--(2.358,1.619)--(2.361,1.617)%
  --(2.364,1.615)--(2.367,1.613)--(2.370,1.611)--(2.373,1.609)--(2.376,1.607)--(2.379,1.605)%
  --(2.382,1.603)--(2.385,1.601)--(2.388,1.599)--(2.391,1.597)--(2.394,1.595)--(2.397,1.594)%
  --(2.400,1.592)--(2.403,1.590)--(2.406,1.588)--(2.409,1.586)--(2.412,1.584)--(2.415,1.582)%
  --(2.418,1.580)--(2.421,1.578)--(2.424,1.577)--(2.427,1.575)--(2.430,1.573)--(2.433,1.571)%
  --(2.436,1.569)--(2.439,1.567)--(2.442,1.566)--(2.445,1.564)--(2.447,1.562)--(2.450,1.560)%
  --(2.453,1.558)--(2.456,1.557)--(2.459,1.555)--(2.462,1.553)--(2.465,1.551)--(2.468,1.550)%
  --(2.471,1.548)--(2.474,1.546)--(2.477,1.544)--(2.480,1.543)--(2.483,1.541)--(2.486,1.539)%
  --(2.489,1.538)--(2.492,1.536)--(2.495,1.534)--(2.498,1.533)--(2.501,1.531)--(2.504,1.529)%
  --(2.507,1.528)--(2.510,1.526)--(2.513,1.524)--(2.516,1.523)--(2.519,1.521)--(2.522,1.520)%
  --(2.525,1.518)--(2.528,1.516)--(2.531,1.515)--(2.534,1.513)--(2.537,1.512)--(2.540,1.510)%
  --(2.543,1.509)--(2.546,1.507)--(2.549,1.506)--(2.552,1.504)--(2.555,1.502)--(2.558,1.501)%
  --(2.561,1.499)--(2.564,1.498)--(2.567,1.496)--(2.570,1.495)--(2.573,1.493)--(2.576,1.492)%
  --(2.579,1.491)--(2.582,1.489)--(2.585,1.488)--(2.588,1.486)--(2.591,1.485)--(2.594,1.483)%
  --(2.597,1.482)--(2.600,1.481)--(2.603,1.479)--(2.606,1.478)--(2.609,1.476)--(2.612,1.475)%
  --(2.615,1.474)--(2.618,1.472)--(2.621,1.471)--(2.624,1.469)--(2.627,1.468)--(2.630,1.467)%
  --(2.633,1.465)--(2.636,1.464)--(2.639,1.463)--(2.642,1.462)--(2.645,1.460)--(2.648,1.459)%
  --(2.651,1.458)--(2.654,1.456)--(2.656,1.455)--(2.659,1.454)--(2.662,1.453)--(2.665,1.451)%
  --(2.668,1.450)--(2.671,1.449)--(2.674,1.448)--(2.677,1.446)--(2.680,1.445)--(2.683,1.444)%
  --(2.686,1.443)--(2.689,1.441)--(2.692,1.440)--(2.695,1.439)--(2.698,1.438)--(2.701,1.437)%
  --(2.704,1.436)--(2.707,1.434)--(2.710,1.433)--(2.713,1.432)--(2.716,1.431)--(2.719,1.430)%
  --(2.722,1.429)--(2.725,1.428)--(2.728,1.426)--(2.731,1.425)--(2.734,1.424)--(2.737,1.423)%
  --(2.740,1.422)--(2.743,1.421)--(2.746,1.420)--(2.749,1.419)--(2.752,1.418)--(2.755,1.417)%
  --(2.758,1.416)--(2.761,1.415)--(2.764,1.414)--(2.767,1.412)--(2.770,1.411)--(2.773,1.410)%
  --(2.776,1.409)--(2.779,1.408)--(2.782,1.407)--(2.785,1.406)--(2.788,1.405)--(2.791,1.404)%
  --(2.794,1.403)--(2.797,1.402)--(2.800,1.402)--(2.803,1.401)--(2.806,1.400)--(2.809,1.399)%
  --(2.812,1.398)--(2.815,1.397)--(2.818,1.396)--(2.821,1.395)--(2.824,1.394)--(2.827,1.393)%
  --(2.830,1.392)--(2.833,1.391)--(2.836,1.390)--(2.839,1.390)--(2.842,1.389)--(2.845,1.388)%
  --(2.848,1.387)--(2.851,1.386)--(2.854,1.385)--(2.857,1.384)--(2.860,1.384)--(2.863,1.383)%
  --(2.866,1.382)--(2.868,1.381)--(2.871,1.380)--(2.874,1.379)--(2.877,1.379)--(2.880,1.378)%
  --(2.883,1.377)--(2.886,1.376)--(2.889,1.375)--(2.892,1.375)--(2.895,1.374)--(2.898,1.373)%
  --(2.901,1.372)--(2.904,1.372)--(2.907,1.371)--(2.910,1.370)--(2.913,1.369)--(2.916,1.369)%
  --(2.919,1.368)--(2.922,1.367)--(2.925,1.366)--(2.928,1.366)--(2.931,1.365)--(2.934,1.364)%
  --(2.937,1.364)--(2.940,1.363)--(2.943,1.362)--(2.946,1.362)--(2.949,1.361)--(2.952,1.360)%
  --(2.955,1.360)--(2.958,1.359)--(2.961,1.358)--(2.964,1.358)--(2.967,1.357)--(2.970,1.356)%
  --(2.973,1.356)--(2.976,1.355)--(2.979,1.355)--(2.982,1.354)--(2.985,1.353)--(2.988,1.353)%
  --(2.991,1.352)--(2.994,1.352)--(2.997,1.351)--(3.000,1.350)--(3.003,1.350)--(3.006,1.349)%
  --(3.009,1.349)--(3.012,1.348)--(3.015,1.348)--(3.018,1.347)--(3.021,1.347)--(3.024,1.346)%
  --(3.027,1.345)--(3.030,1.345)--(3.033,1.344)--(3.036,1.344)--(3.039,1.343)--(3.042,1.343)%
  --(3.045,1.342)--(3.048,1.342)--(3.051,1.341)--(3.054,1.341)--(3.057,1.340)--(3.060,1.340)%
  --(3.063,1.339)--(3.066,1.339)--(3.069,1.339)--(3.072,1.338)--(3.075,1.338)--(3.077,1.337)%
  --(3.080,1.337)--(3.083,1.336)--(3.086,1.336)--(3.089,1.335)--(3.092,1.335)--(3.095,1.335)%
  --(3.098,1.334)--(3.101,1.334)--(3.104,1.333)--(3.107,1.333)--(3.110,1.333)--(3.113,1.332)%
  --(3.116,1.332)--(3.119,1.331)--(3.122,1.331)--(3.125,1.331)--(3.128,1.330)--(3.131,1.330)%
  --(3.134,1.330)--(3.137,1.329)--(3.140,1.329)--(3.143,1.329)--(3.146,1.328)--(3.149,1.328)%
  --(3.152,1.328)--(3.155,1.327)--(3.158,1.327)--(3.161,1.327)--(3.164,1.326)--(3.167,1.326)%
  --(3.170,1.326)--(3.173,1.325)--(3.176,1.325)--(3.179,1.325)--(3.182,1.325)--(3.185,1.324)%
  --(3.188,1.324)--(3.191,1.324)--(3.194,1.324)--(3.197,1.323)--(3.200,1.323)--(3.203,1.323)%
  --(3.206,1.323)--(3.209,1.322)--(3.212,1.322)--(3.215,1.322)--(3.218,1.322)--(3.221,1.321)%
  --(3.224,1.321)--(3.227,1.321)--(3.230,1.321)--(3.233,1.321)--(3.236,1.320)--(3.239,1.320)%
  --(3.242,1.320)--(3.245,1.320)--(3.248,1.320)--(3.251,1.319)--(3.254,1.319)--(3.257,1.319)%
  --(3.260,1.319)--(3.263,1.319)--(3.266,1.319)--(3.269,1.318)--(3.272,1.318)--(3.275,1.318)%
  --(3.278,1.318)--(3.281,1.318)--(3.284,1.318)--(3.286,1.318)--(3.289,1.317)--(3.292,1.317)%
  --(3.295,1.317)--(3.298,1.317)--(3.301,1.317)--(3.304,1.317)--(3.307,1.317)--(3.310,1.317)%
  --(3.313,1.317)--(3.316,1.317)--(3.319,1.316)--(3.322,1.316)--(3.325,1.316)--(3.328,1.316)%
  --(3.331,1.316)--(3.334,1.316)--(3.337,1.316)--(3.340,1.316)--(3.343,1.316)--(3.346,1.316)%
  --(3.349,1.316)--(3.352,1.316)--(3.355,1.316)--(3.358,1.316)--(3.361,1.316)--(3.364,1.316)%
  --(3.367,1.316)--(3.370,1.316)--(3.373,1.316)--(3.376,1.316)--(3.379,1.316)--(3.382,1.316)%
  --(3.385,1.316)--(3.388,1.316)--(3.391,1.316)--(3.394,1.316)--(3.397,1.316)--(3.400,1.316)%
  --(3.403,1.316)--(3.406,1.316)--(3.409,1.316)--(3.412,1.316)--(3.415,1.316)--(3.418,1.316)%
  --(3.421,1.316)--(3.424,1.316)--(3.427,1.316)--(3.430,1.316)--(3.433,1.316)--(3.436,1.316)%
  --(3.439,1.316)--(3.442,1.316)--(3.445,1.316)--(3.448,1.317)--(3.451,1.317)--(3.454,1.317)%
  --(3.457,1.317)--(3.460,1.317)--(3.463,1.317)--(3.466,1.317)--(3.469,1.317)--(3.472,1.317)%
  --(3.475,1.317)--(3.478,1.318)--(3.481,1.318)--(3.484,1.318)--(3.487,1.318)--(3.490,1.318)%
  --(3.493,1.318)--(3.495,1.318)--(3.498,1.318)--(3.501,1.319)--(3.504,1.319)--(3.507,1.319)%
  --(3.510,1.319)--(3.513,1.319)--(3.516,1.319)--(3.519,1.319)--(3.522,1.320)--(3.525,1.320)%
  --(3.528,1.320)--(3.531,1.320)--(3.534,1.320)--(3.537,1.321)--(3.540,1.321)--(3.543,1.321)%
  --(3.546,1.321)--(3.549,1.321)--(3.552,1.321)--(3.555,1.322)--(3.558,1.322)--(3.561,1.322)%
  --(3.564,1.322)--(3.567,1.323)--(3.570,1.323)--(3.573,1.323)--(3.576,1.323)--(3.579,1.323)%
  --(3.582,1.324)--(3.585,1.324)--(3.588,1.324)--(3.591,1.324)--(3.594,1.325)--(3.597,1.325)%
  --(3.600,1.325)--(3.603,1.325)--(3.606,1.326)--(3.609,1.326)--(3.612,1.326)--(3.615,1.326)%
  --(3.618,1.327)--(3.621,1.327)--(3.624,1.327)--(3.627,1.327)--(3.630,1.328)--(3.633,1.328)%
  --(3.636,1.328)--(3.639,1.329)--(3.642,1.329)--(3.645,1.329)--(3.648,1.330)--(3.651,1.330)%
  --(3.654,1.330)--(3.657,1.330)--(3.660,1.331)--(3.663,1.331)--(3.666,1.331)--(3.669,1.332)%
  --(3.672,1.332)--(3.675,1.332)--(3.678,1.333)--(3.681,1.333)--(3.684,1.333)--(3.687,1.334)%
  --(3.690,1.334)--(3.693,1.334)--(3.696,1.335)--(3.699,1.335)--(3.702,1.335)--(3.704,1.336)%
  --(3.707,1.336)--(3.710,1.336)--(3.713,1.337)--(3.716,1.337)--(3.719,1.338)--(3.722,1.338)%
  --(3.725,1.338)--(3.728,1.339)--(3.731,1.339)--(3.734,1.339)--(3.737,1.340)--(3.740,1.340)%
  --(3.743,1.341)--(3.746,1.341)--(3.749,1.341)--(3.752,1.342)--(3.755,1.342)--(3.758,1.343)%
  --(3.761,1.343)--(3.764,1.343)--(3.767,1.344)--(3.770,1.344)--(3.773,1.345)--(3.776,1.345)%
  --(3.779,1.345)--(3.782,1.346)--(3.785,1.346)--(3.788,1.347)--(3.791,1.347)--(3.794,1.348)%
  --(3.797,1.348)--(3.800,1.348)--(3.803,1.349)--(3.806,1.349)--(3.809,1.350)--(3.812,1.350)%
  --(3.815,1.351)--(3.818,1.351)--(3.821,1.352)--(3.824,1.352)--(3.827,1.353)--(3.830,1.353)%
  --(3.833,1.353)--(3.836,1.354)--(3.839,1.354)--(3.842,1.355)--(3.845,1.355)--(3.848,1.356)%
  --(3.851,1.356)--(3.854,1.357)--(3.857,1.357)--(3.860,1.358)--(3.863,1.358)--(3.866,1.359)%
  --(3.869,1.359)--(3.872,1.360)--(3.875,1.360)--(3.878,1.361)--(3.881,1.361)--(3.884,1.362)%
  --(3.887,1.362)--(3.890,1.363)--(3.893,1.363)--(3.896,1.364)--(3.899,1.364)--(3.902,1.365)%
  --(3.905,1.365)--(3.908,1.366)--(3.911,1.366)--(3.914,1.367)--(3.916,1.368)--(3.919,1.368)%
  --(3.922,1.369)--(3.925,1.369)--(3.928,1.370)--(3.931,1.370)--(3.934,1.371)--(3.937,1.371)%
  --(3.940,1.372)--(3.943,1.372)--(3.946,1.373)--(3.949,1.374)--(3.952,1.374)--(3.955,1.375)%
  --(3.958,1.375)--(3.961,1.376)--(3.964,1.376)--(3.967,1.377)--(3.970,1.378)--(3.973,1.378)%
  --(3.976,1.379)--(3.979,1.379)--(3.982,1.380)--(3.985,1.380)--(3.988,1.381)--(3.991,1.382)%
  --(3.994,1.382)--(3.997,1.383)--(4.000,1.383)--(4.003,1.384)--(4.006,1.385)--(4.009,1.385)%
  --(4.012,1.386)--(4.015,1.386)--(4.018,1.387)--(4.021,1.388)--(4.024,1.388)--(4.027,1.389)%
  --(4.030,1.389)--(4.033,1.390)--(4.036,1.391)--(4.039,1.391)--(4.042,1.392)--(4.045,1.393)%
  --(4.048,1.393)--(4.051,1.394)--(4.054,1.395)--(4.057,1.395)--(4.060,1.396)--(4.063,1.396)%
  --(4.066,1.397)--(4.069,1.398)--(4.072,1.398)--(4.075,1.399)--(4.078,1.400)--(4.081,1.400)%
  --(4.084,1.401)--(4.087,1.402)--(4.090,1.402)--(4.093,1.403)--(4.096,1.404)--(4.099,1.404)%
  --(4.102,1.405)--(4.105,1.406)--(4.108,1.406)--(4.111,1.407)--(4.114,1.408)--(4.117,1.408)%
  --(4.120,1.409)--(4.123,1.410)--(4.125,1.410)--(4.128,1.411)--(4.131,1.412)--(4.134,1.412)%
  --(4.137,1.413)--(4.140,1.414)--(4.143,1.414)--(4.146,1.415)--(4.149,1.416)--(4.152,1.417)%
  --(4.155,1.417)--(4.158,1.418)--(4.161,1.419)--(4.164,1.419)--(4.167,1.420)--(4.170,1.421)%
  --(4.173,1.422)--(4.176,1.422)--(4.179,1.423)--(4.182,1.424)--(4.185,1.424)--(4.188,1.425)%
  --(4.191,1.426)--(4.194,1.427)--(4.197,1.427)--(4.200,1.428)--(4.203,1.429)--(4.206,1.429)%
  --(4.209,1.430)--(4.212,1.431)--(4.215,1.432)--(4.218,1.432)--(4.221,1.433)--(4.224,1.434)%
  --(4.227,1.435)--(4.230,1.435)--(4.233,1.436)--(4.236,1.437)--(4.239,1.438)--(4.242,1.438)%
  --(4.245,1.439)--(4.248,1.440)--(4.251,1.441)--(4.254,1.441)--(4.257,1.442)--(4.260,1.443)%
  --(4.263,1.444)--(4.266,1.444)--(4.269,1.445)--(4.272,1.446)--(4.275,1.447)--(4.278,1.448)%
  --(4.281,1.448)--(4.284,1.449)--(4.287,1.450)--(4.290,1.451)--(4.293,1.451)--(4.296,1.452)%
  --(4.299,1.453)--(4.302,1.454)--(4.305,1.455)--(4.308,1.455)--(4.311,1.456)--(4.314,1.457)%
  --(4.317,1.458)--(4.320,1.459)--(4.323,1.459)--(4.326,1.460)--(4.329,1.461)--(4.332,1.462)%
  --(4.334,1.463)--(4.337,1.463)--(4.340,1.464)--(4.343,1.465)--(4.346,1.466)--(4.349,1.467)%
  --(4.352,1.467)--(4.355,1.468)--(4.358,1.469)--(4.361,1.470)--(4.364,1.471)--(4.367,1.472)%
  --(4.370,1.472)--(4.373,1.473)--(4.376,1.474)--(4.379,1.475)--(4.382,1.476)--(4.385,1.477)%
  --(4.388,1.477)--(4.391,1.478)--(4.394,1.479)--(4.397,1.480)--(4.400,1.481)--(4.403,1.482)%
  --(4.406,1.482)--(4.409,1.483)--(4.412,1.484)--(4.415,1.485)--(4.418,1.486)--(4.421,1.487)%
  --(4.424,1.487)--(4.427,1.488)--(4.430,1.489)--(4.433,1.490)--(4.436,1.491)--(4.439,1.492)%
  --(4.442,1.493)--(4.445,1.493)--(4.448,1.494)--(4.451,1.495)--(4.454,1.496)--(4.457,1.497)%
  --(4.460,1.498)--(4.463,1.499)--(4.466,1.500)--(4.469,1.500)--(4.472,1.501)--(4.475,1.502)%
  --(4.478,1.503)--(4.481,1.504)--(4.484,1.505)--(4.487,1.506)--(4.490,1.507)--(4.493,1.507)%
  --(4.496,1.508)--(4.499,1.509)--(4.502,1.510)--(4.505,1.511)--(4.508,1.512)--(4.511,1.513)%
  --(4.514,1.514)--(4.517,1.515)--(4.520,1.515)--(4.523,1.516)--(4.526,1.517)--(4.529,1.518)%
  --(4.532,1.519)--(4.535,1.520)--(4.538,1.521)--(4.541,1.522)--(4.543,1.523)--(4.546,1.524)%
  --(4.549,1.524)--(4.552,1.525)--(4.555,1.526)--(4.558,1.527)--(4.561,1.528)--(4.564,1.529)%
  --(4.567,1.530)--(4.570,1.531)--(4.573,1.532)--(4.576,1.533)--(4.579,1.534)--(4.582,1.535)%
  --(4.585,1.535)--(4.588,1.536)--(4.591,1.537)--(4.594,1.538)--(4.597,1.539)--(4.600,1.540)%
  --(4.603,1.541)--(4.606,1.542)--(4.609,1.543)--(4.612,1.544)--(4.615,1.545)--(4.618,1.546)%
  --(4.621,1.547)--(4.624,1.548)--(4.627,1.549)--(4.630,1.550)--(4.633,1.550)--(4.636,1.551)%
  --(4.639,1.552)--(4.642,1.553)--(4.645,1.554)--(4.648,1.555)--(4.651,1.556)--(4.654,1.557)%
  --(4.657,1.558)--(4.660,1.559)--(4.663,1.560)--(4.666,1.561)--(4.669,1.562)--(4.672,1.563)%
  --(4.675,1.564)--(4.678,1.565)--(4.681,1.566)--(4.684,1.567)--(4.687,1.568)--(4.690,1.569)%
  --(4.693,1.570)--(4.696,1.571)--(4.699,1.572)--(4.702,1.573)--(4.705,1.573)--(4.708,1.574)%
  --(4.711,1.575)--(4.714,1.576)--(4.717,1.577)--(4.720,1.578)--(4.723,1.579)--(4.726,1.580)%
  --(4.729,1.581)--(4.732,1.582)--(4.735,1.583)--(4.738,1.584)--(4.741,1.585)--(4.744,1.586)%
  --(4.747,1.587)--(4.750,1.588)--(4.752,1.589)--(4.755,1.590)--(4.758,1.591)--(4.761,1.592)%
  --(4.764,1.593)--(4.767,1.594)--(4.770,1.595)--(4.773,1.596)--(4.776,1.597)--(4.779,1.598)%
  --(4.782,1.599)--(4.785,1.600)--(4.788,1.601)--(4.791,1.602)--(4.794,1.603)--(4.797,1.604)%
  --(4.800,1.605)--(4.803,1.606)--(4.806,1.607)--(4.809,1.608)--(4.812,1.609)--(4.815,1.610)%
  --(4.818,1.611)--(4.821,1.612)--(4.824,1.613)--(4.827,1.614)--(4.830,1.615)--(4.833,1.616)%
  --(4.836,1.617)--(4.839,1.618)--(4.842,1.619)--(4.845,1.620)--(4.848,1.622)--(4.851,1.623)%
  --(4.854,1.624)--(4.857,1.625)--(4.860,1.626)--(4.863,1.627)--(4.866,1.628)--(4.869,1.629)%
  --(4.872,1.630)--(4.875,1.631)--(4.878,1.632)--(4.881,1.633)--(4.884,1.634)--(4.887,1.635)%
  --(4.890,1.636)--(4.893,1.637)--(4.896,1.638)--(4.899,1.639)--(4.902,1.640)--(4.905,1.641)%
  --(4.908,1.642)--(4.911,1.643)--(4.914,1.644)--(4.917,1.645)--(4.920,1.646)--(4.923,1.647)%
  --(4.926,1.649)--(4.929,1.650)--(4.932,1.651)--(4.935,1.652)--(4.938,1.653)--(4.941,1.654)%
  --(4.944,1.655)--(4.947,1.656)--(4.950,1.657)--(4.953,1.658)--(4.956,1.659)--(4.959,1.660)%
  --(4.961,1.661)--(4.964,1.662)--(4.967,1.663)--(4.970,1.664)--(4.973,1.665)--(4.976,1.667)%
  --(4.979,1.668)--(4.982,1.669)--(4.985,1.670)--(4.988,1.671)--(4.991,1.672)--(4.994,1.673)%
  --(4.997,1.674)--(5.000,1.675)--(5.003,1.676)--(5.006,1.677)--(5.009,1.678)--(5.012,1.679)%
  --(5.015,1.681)--(5.018,1.682)--(5.021,1.683)--(5.024,1.684)--(5.027,1.685)--(5.030,1.686)%
  --(5.033,1.687)--(5.036,1.688)--(5.039,1.689)--(5.042,1.690)--(5.045,1.691)--(5.048,1.692)%
  --(5.051,1.694)--(5.054,1.695)--(5.057,1.696)--(5.060,1.697)--(5.063,1.698)--(5.066,1.699)%
  --(5.069,1.700)--(5.072,1.701)--(5.075,1.702)--(5.078,1.703)--(5.081,1.704)--(5.084,1.706)%
  --(5.087,1.707)--(5.090,1.708)--(5.093,1.709)--(5.096,1.710)--(5.099,1.711)--(5.102,1.712)%
  --(5.105,1.713)--(5.108,1.714)--(5.111,1.715)--(5.114,1.717)--(5.117,1.718)--(5.120,1.719)%
  --(5.123,1.720)--(5.126,1.721)--(5.129,1.722)--(5.132,1.723)--(5.135,1.724)--(5.138,1.725)%
  --(5.141,1.727)--(5.144,1.728)--(5.147,1.729)--(5.150,1.730)--(5.153,1.731)--(5.156,1.732)%
  --(5.159,1.733)--(5.162,1.734)--(5.165,1.736)--(5.168,1.737)--(5.171,1.738)--(5.173,1.739)%
  --(5.176,1.740)--(5.179,1.741)--(5.182,1.742)--(5.185,1.743)--(5.188,1.745)--(5.191,1.746)%
  --(5.194,1.747)--(5.197,1.748)--(5.200,1.749)--(5.203,1.750)--(5.206,1.751)--(5.209,1.752)%
  --(5.212,1.754)--(5.215,1.755)--(5.218,1.756)--(5.221,1.757)--(5.224,1.758)--(5.227,1.759)%
  --(5.230,1.760)--(5.233,1.762)--(5.236,1.763)--(5.239,1.764)--(5.242,1.765)--(5.245,1.766)%
  --(5.248,1.767)--(5.251,1.768)--(5.254,1.770)--(5.257,1.771)--(5.260,1.772)--(5.263,1.773)%
  --(5.266,1.774)--(5.269,1.775)--(5.272,1.776)--(5.275,1.778)--(5.278,1.779)--(5.281,1.780)%
  --(5.284,1.781)--(5.287,1.782)--(5.290,1.783)--(5.293,1.784)--(5.296,1.786)--(5.299,1.787)%
  --(5.302,1.788)--(5.305,1.789)--(5.308,1.790)--(5.311,1.791)--(5.314,1.793)--(5.317,1.794)%
  --(5.320,1.795)--(5.323,1.796)--(5.326,1.797)--(5.329,1.798)--(5.332,1.799)--(5.335,1.801)%
  --(5.338,1.802)--(5.341,1.803)--(5.344,1.804)--(5.347,1.805)--(5.350,1.806)--(5.353,1.808)%
  --(5.356,1.809)--(5.359,1.810)--(5.362,1.811)--(5.365,1.812)--(5.368,1.813)--(5.371,1.815)%
  --(5.374,1.816)--(5.377,1.817)--(5.380,1.818)--(5.382,1.819)--(5.385,1.821)--(5.388,1.822)%
  --(5.391,1.823)--(5.394,1.824)--(5.397,1.825)--(5.400,1.826)--(5.403,1.828)--(5.406,1.829)%
  --(5.409,1.830)--(5.412,1.831)--(5.415,1.832)--(5.418,1.833)--(5.421,1.835)--(5.424,1.836)%
  --(5.427,1.837)--(5.430,1.838)--(5.433,1.839)--(5.436,1.841)--(5.439,1.842)--(5.442,1.843)%
  --(5.445,1.844)--(5.448,1.845)--(5.451,1.847)--(5.454,1.848)--(5.457,1.849)--(5.460,1.850)%
  --(5.463,1.851)--(5.466,1.852)--(5.469,1.854)--(5.472,1.855)--(5.475,1.856)--(5.478,1.857)%
  --(5.481,1.858)--(5.484,1.860)--(5.487,1.861)--(5.490,1.862)--(5.493,1.863)--(5.496,1.864)%
  --(5.499,1.866)--(5.502,1.867)--(5.505,1.868)--(5.508,1.869)--(5.511,1.870)--(5.514,1.872)%
  --(5.517,1.873)--(5.520,1.874)--(5.523,1.875)--(5.526,1.876)--(5.529,1.878)--(5.532,1.879)%
  --(5.535,1.880)--(5.538,1.881)--(5.541,1.882)--(5.544,1.884)--(5.547,1.885)--(5.550,1.886)%
  --(5.553,1.887)--(5.556,1.888)--(5.559,1.890)--(5.562,1.891)--(5.565,1.892)--(5.568,1.893)%
  --(5.571,1.895)--(5.574,1.896)--(5.577,1.897)--(5.580,1.898)--(5.583,1.899)--(5.586,1.901)%
  --(5.589,1.902)--(5.591,1.903)--(5.594,1.904)--(5.597,1.905)--(5.600,1.907)--(5.603,1.908)%
  --(5.606,1.909)--(5.609,1.910)--(5.612,1.912)--(5.615,1.913)--(5.618,1.914)--(5.621,1.915)%
  --(5.624,1.916)--(5.627,1.918)--(5.630,1.919)--(5.633,1.920)--(5.636,1.921)--(5.639,1.923)%
  --(5.642,1.924)--(5.645,1.925)--(5.648,1.926)--(5.651,1.927)--(5.654,1.929)--(5.657,1.930)%
  --(5.660,1.931)--(5.663,1.932)--(5.666,1.934)--(5.669,1.935)--(5.672,1.936)--(5.675,1.937)%
  --(5.678,1.939)--(5.681,1.940)--(5.684,1.941)--(5.687,1.942)--(5.690,1.943)--(5.693,1.945)%
  --(5.696,1.946)--(5.699,1.947)--(5.702,1.948)--(5.705,1.950)--(5.708,1.951)--(5.711,1.952)%
  --(5.714,1.953)--(5.717,1.955)--(5.720,1.956)--(5.723,1.957)--(5.726,1.958)--(5.729,1.960)%
  --(5.732,1.961)--(5.735,1.962)--(5.738,1.963)--(5.741,1.965)--(5.744,1.966)--(5.747,1.967)%
  --(5.750,1.968)--(5.753,1.970)--(5.756,1.971)--(5.759,1.972)--(5.762,1.973)--(5.765,1.974)%
  --(5.768,1.976)--(5.771,1.977)--(5.774,1.978)--(5.777,1.979)--(5.780,1.981)--(5.783,1.982)%
  --(5.786,1.983)--(5.789,1.984)--(5.792,1.986)--(5.795,1.987)--(5.798,1.988)--(5.800,1.989)%
  --(5.803,1.991)--(5.806,1.992)--(5.809,1.993)--(5.812,1.995)--(5.815,1.996)--(5.818,1.997)%
  --(5.821,1.998)--(5.824,2.000)--(5.827,2.001)--(5.830,2.002)--(5.833,2.003)--(5.836,2.005)%
  --(5.839,2.006)--(5.842,2.007)--(5.845,2.008)--(5.848,2.010)--(5.851,2.011)--(5.854,2.012)%
  --(5.857,2.013)--(5.860,2.015)--(5.863,2.016)--(5.866,2.017)--(5.869,2.018)--(5.872,2.020)%
  --(5.875,2.021)--(5.878,2.022)--(5.881,2.023)--(5.884,2.025)--(5.887,2.026)--(5.890,2.027)%
  --(5.893,2.029)--(5.896,2.030)--(5.899,2.031)--(5.902,2.032)--(5.905,2.034)--(5.908,2.035)%
  --(5.911,2.036)--(5.914,2.037)--(5.917,2.039)--(5.920,2.040)--(5.923,2.041)--(5.926,2.043)%
  --(5.929,2.044)--(5.932,2.045)--(5.935,2.046)--(5.938,2.048)--(5.941,2.049)--(5.944,2.050)%
  --(5.947,2.051)--(5.950,2.053)--(5.953,2.054)--(5.956,2.055)--(5.959,2.057)--(5.962,2.058)%
  --(5.965,2.059)--(5.968,2.060)--(5.971,2.062)--(5.974,2.063)--(5.977,2.064)--(5.980,2.065)%
  --(5.983,2.067)--(5.986,2.068)--(5.989,2.069)--(5.992,2.071)--(5.995,2.072)--(5.998,2.073)%
  --(6.001,2.074)--(6.004,2.076)--(6.007,2.077)--(6.009,2.078)--(6.012,2.080)--(6.015,2.081)%
  --(6.018,2.082)--(6.021,2.083)--(6.024,2.085)--(6.027,2.086)--(6.030,2.087)--(6.033,2.089)%
  --(6.036,2.090)--(6.039,2.091)--(6.042,2.092)--(6.045,2.094)--(6.048,2.095)--(6.051,2.096)%
  --(6.054,2.098)--(6.057,2.099)--(6.060,2.100)--(6.063,2.101)--(6.066,2.103)--(6.069,2.104)%
  --(6.072,2.105)--(6.075,2.107)--(6.078,2.108)--(6.081,2.109)--(6.084,2.111)--(6.087,2.112)%
  --(6.090,2.113)--(6.093,2.114)--(6.096,2.116)--(6.099,2.117)--(6.102,2.118)--(6.105,2.120)%
  --(6.108,2.121)--(6.111,2.122)--(6.114,2.123)--(6.117,2.125)--(6.120,2.126)--(6.123,2.127)%
  --(6.126,2.129)--(6.129,2.130)--(6.132,2.131)--(6.135,2.133)--(6.138,2.134)--(6.141,2.135)%
  --(6.144,2.136)--(6.147,2.138)--(6.150,2.139)--(6.153,2.140)--(6.156,2.142)--(6.159,2.143)%
  --(6.162,2.144)--(6.165,2.146)--(6.168,2.147)--(6.171,2.148)--(6.174,2.150)--(6.177,2.151)%
  --(6.180,2.152)--(6.183,2.153)--(6.186,2.155)--(6.189,2.156)--(6.192,2.157)--(6.195,2.159)%
  --(6.198,2.160)--(6.201,2.161)--(6.204,2.163)--(6.207,2.164)--(6.210,2.165)--(6.213,2.167)%
  --(6.216,2.168)--(6.218,2.169)--(6.221,2.170)--(6.224,2.172)--(6.227,2.173)--(6.230,2.174)%
  --(6.233,2.176)--(6.236,2.177)--(6.239,2.178)--(6.242,2.180)--(6.245,2.181)--(6.248,2.182)%
  --(6.251,2.184)--(6.254,2.185)--(6.257,2.186)--(6.260,2.188)--(6.263,2.189)--(6.266,2.190)%
  --(6.269,2.191)--(6.272,2.193)--(6.275,2.194)--(6.278,2.195)--(6.281,2.197)--(6.284,2.198)%
  --(6.287,2.199)--(6.290,2.201)--(6.293,2.202)--(6.296,2.203)--(6.299,2.205)--(6.302,2.206)%
  --(6.305,2.207)--(6.308,2.209)--(6.311,2.210)--(6.314,2.211)--(6.317,2.213)--(6.320,2.214)%
  --(6.323,2.215)--(6.326,2.217)--(6.329,2.218)--(6.332,2.219)--(6.335,2.221)--(6.338,2.222)%
  --(6.341,2.223)--(6.344,2.224)--(6.347,2.226)--(6.350,2.227)--(6.353,2.228)--(6.356,2.230)%
  --(6.359,2.231)--(6.362,2.232)--(6.365,2.234)--(6.368,2.235)--(6.371,2.236)--(6.374,2.238)%
  --(6.377,2.239)--(6.380,2.240)--(6.383,2.242)--(6.386,2.243)--(6.389,2.244)--(6.392,2.246)%
  --(6.395,2.247)--(6.398,2.248)--(6.401,2.250)--(6.404,2.251)--(6.407,2.252)--(6.410,2.254)%
  --(6.413,2.255)--(6.416,2.256)--(6.419,2.258)--(6.422,2.259)--(6.425,2.260)--(6.428,2.262)%
  --(6.430,2.263)--(6.433,2.264)--(6.436,2.266)--(6.439,2.267)--(6.442,2.268)--(6.445,2.270)%
  --(6.448,2.271)--(6.451,2.272)--(6.454,2.274)--(6.457,2.275)--(6.460,2.276)--(6.463,2.278)%
  --(6.466,2.279)--(6.469,2.280)--(6.472,2.282)--(6.475,2.283)--(6.478,2.284)--(6.481,2.286)%
  --(6.484,2.287)--(6.487,2.288)--(6.490,2.290)--(6.493,2.291)--(6.496,2.292)--(6.499,2.294)%
  --(6.502,2.295)--(6.505,2.297)--(6.508,2.298)--(6.511,2.299)--(6.514,2.301)--(6.517,2.302)%
  --(6.520,2.303)--(6.523,2.305)--(6.526,2.306)--(6.529,2.307)--(6.532,2.309)--(6.535,2.310)%
  --(6.538,2.311)--(6.541,2.313)--(6.544,2.314)--(6.547,2.315)--(6.550,2.317)--(6.553,2.318)%
  --(6.556,2.319)--(6.559,2.321)--(6.562,2.322)--(6.565,2.323)--(6.568,2.325)--(6.571,2.326)%
  --(6.574,2.327)--(6.577,2.329)--(6.580,2.330)--(6.583,2.331)--(6.586,2.333)--(6.589,2.334)%
  --(6.592,2.336)--(6.595,2.337)--(6.598,2.338)--(6.601,2.340)--(6.604,2.341)--(6.607,2.342)%
  --(6.610,2.344)--(6.613,2.345)--(6.616,2.346)--(6.619,2.348)--(6.622,2.349)--(6.625,2.350)%
  --(6.628,2.352)--(6.631,2.353)--(6.634,2.354)--(6.637,2.356)--(6.639,2.357)--(6.642,2.359)%
  --(6.645,2.360)--(6.648,2.361)--(6.651,2.363)--(6.654,2.364)--(6.657,2.365)--(6.660,2.367)%
  --(6.663,2.368)--(6.666,2.369)--(6.669,2.371)--(6.672,2.372)--(6.675,2.373)--(6.678,2.375)%
  --(6.681,2.376)--(6.684,2.378)--(6.687,2.379)--(6.690,2.380)--(6.693,2.382)--(6.696,2.383)%
  --(6.699,2.384)--(6.702,2.386)--(6.705,2.387)--(6.708,2.388)--(6.711,2.390)--(6.714,2.391)%
  --(6.717,2.392)--(6.720,2.394)--(6.723,2.395)--(6.726,2.397)--(6.729,2.398)--(6.732,2.399)%
  --(6.735,2.401)--(6.738,2.402)--(6.741,2.403)--(6.744,2.405)--(6.747,2.406)--(6.750,2.407)%
  --(6.753,2.409)--(6.756,2.410)--(6.759,2.412)--(6.762,2.413)--(6.765,2.414)--(6.768,2.416)%
  --(6.771,2.417)--(6.774,2.418)--(6.777,2.420)--(6.780,2.421)--(6.783,2.422)--(6.786,2.424)%
  --(6.789,2.425)--(6.792,2.427)--(6.795,2.428)--(6.798,2.429)--(6.801,2.431)--(6.804,2.432)%
  --(6.807,2.433)--(6.810,2.435)--(6.813,2.436)--(6.816,2.438)--(6.819,2.439)--(6.822,2.440)%
  --(6.825,2.442)--(6.828,2.443)--(6.831,2.444)--(6.834,2.446)--(6.837,2.447)--(6.840,2.448)%
  --(6.843,2.450)--(6.846,2.451)--(6.848,2.453)--(6.851,2.454)--(6.854,2.455)--(6.857,2.457)%
  --(6.860,2.458)--(6.863,2.459)--(6.866,2.461)--(6.869,2.462)--(6.872,2.464)--(6.875,2.465)%
  --(6.878,2.466)--(6.881,2.468)--(6.884,2.469)--(6.887,2.470)--(6.890,2.472)--(6.893,2.473)%
  --(6.896,2.475)--(6.899,2.476)--(6.902,2.477)--(6.905,2.479)--(6.908,2.480)--(6.911,2.481)%
  --(6.914,2.483)--(6.917,2.484)--(6.920,2.486)--(6.923,2.487)--(6.926,2.488)--(6.929,2.490)%
  --(6.932,2.491)--(6.935,2.492)--(6.938,2.494)--(6.941,2.495)--(6.944,2.497)--(6.947,2.498)%
  --(6.950,2.499)--(6.953,2.501)--(6.956,2.502)--(6.959,2.504)--(6.962,2.505)--(6.965,2.506)%
  --(6.968,2.508)--(6.971,2.509)--(6.974,2.510)--(6.977,2.512)--(6.980,2.513)--(6.983,2.515)%
  --(6.986,2.516)--(6.989,2.517)--(6.992,2.519)--(6.995,2.520)--(6.998,2.521)--(7.001,2.523)%
  --(7.004,2.524)--(7.007,2.526)--(7.010,2.527)--(7.013,2.528)--(7.016,2.530)--(7.019,2.531)%
  --(7.022,2.533)--(7.025,2.534)--(7.028,2.535)--(7.031,2.537)--(7.034,2.538)--(7.037,2.539)%
  --(7.040,2.541)--(7.043,2.542)--(7.046,2.544)--(7.049,2.545)--(7.052,2.546)--(7.055,2.548)%
  --(7.057,2.549)--(7.060,2.551)--(7.063,2.552)--(7.066,2.553)--(7.069,2.555)--(7.072,2.556)%
  --(7.075,2.557)--(7.078,2.559)--(7.081,2.560)--(7.084,2.562)--(7.087,2.563)--(7.090,2.564)%
  --(7.093,2.566)--(7.096,2.567)--(7.099,2.569)--(7.102,2.570)--(7.105,2.571)--(7.108,2.573)%
  --(7.111,2.574)--(7.114,2.576)--(7.117,2.577)--(7.120,2.578)--(7.123,2.580)--(7.126,2.581)%
  --(7.129,2.583)--(7.132,2.584)--(7.135,2.585)--(7.138,2.587)--(7.141,2.588)--(7.144,2.589)%
  --(7.147,2.591)--(7.150,2.592)--(7.153,2.594)--(7.156,2.595)--(7.159,2.596)--(7.162,2.598)%
  --(7.165,2.599)--(7.168,2.601)--(7.171,2.602)--(7.174,2.603)--(7.177,2.605)--(7.180,2.606)%
  --(7.183,2.608)--(7.186,2.609)--(7.189,2.610)--(7.192,2.612)--(7.195,2.613)--(7.198,2.615)%
  --(7.201,2.616)--(7.204,2.617)--(7.207,2.619)--(7.210,2.620)--(7.213,2.622)--(7.216,2.623)%
  --(7.219,2.624)--(7.222,2.626)--(7.225,2.627)--(7.228,2.629)--(7.231,2.630)--(7.234,2.631)%
  --(7.237,2.633)--(7.240,2.634)--(7.243,2.636)--(7.246,2.637)--(7.249,2.638)--(7.252,2.640)%
  --(7.255,2.641)--(7.258,2.643)--(7.261,2.644)--(7.264,2.645)--(7.266,2.647)--(7.269,2.648)%
  --(7.272,2.650)--(7.275,2.651)--(7.278,2.652)--(7.281,2.654)--(7.284,2.655)--(7.287,2.657)%
  --(7.290,2.658)--(7.293,2.659)--(7.296,2.661)--(7.299,2.662)--(7.302,2.664)--(7.305,2.665)%
  --(7.308,2.666)--(7.311,2.668)--(7.314,2.669)--(7.317,2.671)--(7.320,2.672)--(7.323,2.673)%
  --(7.326,2.675)--(7.329,2.676)--(7.332,2.678)--(7.335,2.679)--(7.338,2.680)--(7.341,2.682)%
  --(7.344,2.683)--(7.347,2.685)--(7.350,2.686)--(7.353,2.687)--(7.356,2.689)--(7.359,2.690)%
  --(7.362,2.692)--(7.365,2.693)--(7.368,2.694)--(7.371,2.696)--(7.374,2.697)--(7.377,2.699)%
  --(7.380,2.700)--(7.383,2.702)--(7.386,2.703)--(7.389,2.704)--(7.392,2.706)--(7.395,2.707)%
  --(7.398,2.709)--(7.401,2.710)--(7.404,2.711)--(7.407,2.713)--(7.410,2.714)--(7.413,2.716)%
  --(7.416,2.717)--(7.419,2.718)--(7.422,2.720)--(7.425,2.721)--(7.428,2.723)--(7.431,2.724)%
  --(7.434,2.725)--(7.437,2.727)--(7.440,2.728)--(7.443,2.730)--(7.446,2.731)--(7.449,2.733)%
  --(7.452,2.734)--(7.455,2.735)--(7.458,2.737)--(7.461,2.738)--(7.464,2.740)--(7.467,2.741)%
  --(7.470,2.742)--(7.473,2.744)--(7.476,2.745)--(7.478,2.747)--(7.481,2.748)--(7.484,2.749)%
  --(7.487,2.751)--(7.490,2.752)--(7.493,2.754)--(7.496,2.755)--(7.499,2.757)--(7.502,2.758)%
  --(7.505,2.759)--(7.508,2.761)--(7.511,2.762)--(7.514,2.764)--(7.517,2.765)--(7.520,2.766)%
  --(7.523,2.768)--(7.526,2.769)--(7.529,2.771)--(7.532,2.772)--(7.535,2.773)--(7.538,2.775)%
  --(7.541,2.776)--(7.544,2.778)--(7.547,2.779)--(7.550,2.781)--(7.553,2.782)--(7.556,2.783)%
  --(7.559,2.785)--(7.562,2.786)--(7.565,2.788)--(7.568,2.789)--(7.571,2.790)--(7.574,2.792)%
  --(7.577,2.793)--(7.580,2.795)--(7.583,2.796)--(7.586,2.798)--(7.589,2.799)--(7.592,2.800)%
  --(7.595,2.802)--(7.598,2.803)--(7.601,2.805)--(7.604,2.806)--(7.607,2.807)--(7.610,2.809)%
  --(7.613,2.810)--(7.616,2.812)--(7.619,2.813)--(7.622,2.815)--(7.625,2.816)--(7.628,2.817)%
  --(7.631,2.819)--(7.634,2.820)--(7.637,2.822)--(7.640,2.823)--(7.643,2.825)--(7.646,2.826)%
  --(7.649,2.827)--(7.652,2.829)--(7.655,2.830)--(7.658,2.832)--(7.661,2.833)--(7.664,2.834)%
  --(7.667,2.836)--(7.670,2.837)--(7.673,2.839)--(7.676,2.840)--(7.679,2.842)--(7.682,2.843)%
  --(7.685,2.844)--(7.687,2.846)--(7.690,2.847)--(7.693,2.849)--(7.696,2.850)--(7.699,2.852)%
  --(7.702,2.853)--(7.705,2.854)--(7.708,2.856)--(7.711,2.857)--(7.714,2.859)--(7.717,2.860)%
  --(7.720,2.862)--(7.723,2.863)--(7.726,2.864)--(7.729,2.866)--(7.732,2.867)--(7.735,2.869)%
  --(7.738,2.870)--(7.741,2.871)--(7.744,2.873)--(7.747,2.874)--(7.750,2.876)--(7.753,2.877)%
  --(7.756,2.879)--(7.759,2.880)--(7.762,2.881)--(7.765,2.883)--(7.768,2.884)--(7.771,2.886)%
  --(7.774,2.887)--(7.777,2.889)--(7.780,2.890)--(7.783,2.891)--(7.786,2.893)--(7.789,2.894)%
  --(7.792,2.896)--(7.795,2.897)--(7.798,2.899)--(7.801,2.900)--(7.804,2.901)--(7.807,2.903)%
  --(7.810,2.904)--(7.813,2.906)--(7.816,2.907)--(7.819,2.909)--(7.822,2.910)--(7.825,2.911)%
  --(7.828,2.913)--(7.831,2.914)--(7.834,2.916)--(7.837,2.917)--(7.840,2.919)--(7.843,2.920)%
  --(7.846,2.921)--(7.849,2.923)--(7.852,2.924)--(7.855,2.926)--(7.858,2.927)--(7.861,2.929)%
  --(7.864,2.930)--(7.867,2.931)--(7.870,2.933)--(7.873,2.934)--(7.876,2.936)--(7.879,2.937)%
  --(7.882,2.939)--(7.885,2.940)--(7.888,2.942)--(7.891,2.943)--(7.894,2.944)--(7.896,2.946)%
  --(7.899,2.947)--(7.902,2.949)--(7.905,2.950)--(7.908,2.952)--(7.911,2.953)--(7.914,2.954)%
  --(7.917,2.956)--(7.920,2.957)--(7.923,2.959)--(7.926,2.960)--(7.929,2.962)--(7.932,2.963)%
  --(7.935,2.964)--(7.938,2.966)--(7.941,2.967)--(7.944,2.969)--(7.947,2.970)--(7.950,2.972)%
  --(7.953,2.973)--(7.956,2.974)--(7.959,2.976)--(7.962,2.977)--(7.965,2.979)--(7.968,2.980)%
  --(7.971,2.982)--(7.974,2.983)--(7.977,2.985)--(7.980,2.986)--(7.983,2.987)--(7.986,2.989)%
  --(7.989,2.990)--(7.992,2.992)--(7.995,2.993)--(7.998,2.995)--(8.001,2.996)--(8.004,2.997)%
  --(8.007,2.999)--(8.010,3.000)--(8.013,3.002)--(8.016,3.003)--(8.019,3.005)--(8.022,3.006)%
  --(8.025,3.007)--(8.028,3.009)--(8.031,3.010)--(8.034,3.012)--(8.037,3.013)--(8.040,3.015)%
  --(8.043,3.016)--(8.046,3.018)--(8.049,3.019)--(8.052,3.020)--(8.055,3.022)--(8.058,3.023)%
  --(8.061,3.025)--(8.064,3.026)--(8.067,3.028)--(8.070,3.029)--(8.073,3.031)--(8.076,3.032)%
  --(8.079,3.033)--(8.082,3.035)--(8.085,3.036)--(8.088,3.038)--(8.091,3.039)--(8.094,3.041)%
  --(8.097,3.042)--(8.100,3.043)--(8.103,3.045)--(8.105,3.046)--(8.108,3.048)--(8.111,3.049)%
  --(8.114,3.051)--(8.117,3.052)--(8.120,3.054)--(8.123,3.055)--(8.126,3.056)--(8.129,3.058)%
  --(8.132,3.059)--(8.135,3.061)--(8.138,3.062)--(8.141,3.064)--(8.144,3.065)--(8.147,3.067)%
  --(8.150,3.068)--(8.153,3.069)--(8.156,3.071)--(8.159,3.072)--(8.162,3.074)--(8.165,3.075)%
  --(8.168,3.077)--(8.171,3.078)--(8.174,3.080)--(8.177,3.081)--(8.180,3.082)--(8.183,3.084)%
  --(8.186,3.085)--(8.189,3.087)--(8.192,3.088)--(8.195,3.090)--(8.198,3.091)--(8.201,3.092)%
  --(8.204,3.094)--(8.207,3.095)--(8.210,3.097)--(8.213,3.098)--(8.216,3.100)--(8.219,3.101)%
  --(8.222,3.103)--(8.225,3.104)--(8.228,3.105)--(8.231,3.107)--(8.234,3.108)--(8.237,3.110)%
  --(8.240,3.111)--(8.243,3.113)--(8.246,3.114)--(8.249,3.116)--(8.252,3.117)--(8.255,3.118)%
  --(8.258,3.120)--(8.261,3.121)--(8.264,3.123)--(8.267,3.124)--(8.270,3.126)--(8.273,3.127)%
  --(8.276,3.129)--(8.279,3.130)--(8.282,3.132)--(8.285,3.133)--(8.288,3.134)--(8.291,3.136)%
  --(8.294,3.137)--(8.297,3.139)--(8.300,3.140)--(8.303,3.142)--(8.306,3.143)--(8.309,3.145)%
  --(8.312,3.146)--(8.314,3.147)--(8.317,3.149)--(8.320,3.150)--(8.323,3.152)--(8.326,3.153)%
  --(8.329,3.155)--(8.332,3.156)--(8.335,3.158)--(8.338,3.159)--(8.341,3.160)--(8.344,3.162)%
  --(8.347,3.163)--(8.350,3.165)--(8.353,3.166)--(8.356,3.168)--(8.359,3.169)--(8.362,3.171)%
  --(8.365,3.172)--(8.368,3.174)--(8.371,3.175)--(8.374,3.176)--(8.377,3.178)--(8.380,3.179)%
  --(8.383,3.181)--(8.386,3.182)--(8.389,3.184)--(8.392,3.185)--(8.395,3.187)--(8.398,3.188)%
  --(8.401,3.189)--(8.404,3.191)--(8.407,3.192)--(8.410,3.194)--(8.413,3.195)--(8.416,3.197)%
  --(8.419,3.198)--(8.422,3.200)--(8.425,3.201)--(8.428,3.203)--(8.431,3.204)--(8.434,3.205)%
  --(8.437,3.207)--(8.440,3.208)--(8.443,3.210)--(8.446,3.211)--(8.449,3.213)--(8.452,3.214)%
  --(8.455,3.216)--(8.458,3.217)--(8.461,3.218)--(8.464,3.220)--(8.467,3.221)--(8.470,3.223)%
  --(8.473,3.224)--(8.476,3.226)--(8.479,3.227)--(8.482,3.229)--(8.485,3.230)--(8.488,3.232)%
  --(8.491,3.233)--(8.494,3.234)--(8.497,3.236)--(8.500,3.237)--(8.503,3.239)--(8.506,3.240)%
  --(8.509,3.242)--(8.512,3.243)--(8.515,3.245)--(8.518,3.246)--(8.521,3.248)--(8.523,3.249)%
  --(8.526,3.250)--(8.529,3.252)--(8.532,3.253)--(8.535,3.255)--(8.538,3.256)--(8.541,3.258)%
  --(8.544,3.259)--(8.547,3.261)--(8.550,3.262)--(8.553,3.264)--(8.556,3.265)--(8.559,3.266)%
  --(8.562,3.268)--(8.565,3.269)--(8.568,3.271)--(8.571,3.272)--(8.574,3.274)--(8.577,3.275)%
  --(8.580,3.277)--(8.583,3.278)--(8.586,3.280)--(8.589,3.281)--(8.592,3.282)--(8.595,3.284)%
  --(8.598,3.285)--(8.601,3.287)--(8.604,3.288)--(8.607,3.290)--(8.610,3.291)--(8.613,3.293)%
  --(8.616,3.294)--(8.619,3.296)--(8.622,3.297)--(8.625,3.299)--(8.628,3.300)--(8.631,3.301)%
  --(8.634,3.303)--(8.637,3.304)--(8.640,3.306)--(8.643,3.307)--(8.646,3.309)--(8.649,3.310)%
  --(8.652,3.312)--(8.655,3.313)--(8.658,3.315)--(8.661,3.316)--(8.664,3.317)--(8.667,3.319)%
  --(8.670,3.320)--(8.673,3.322)--(8.676,3.323)--(8.679,3.325)--(8.682,3.326)--(8.685,3.328)%
  --(8.688,3.329)--(8.691,3.331)--(8.694,3.332)--(8.697,3.334)--(8.700,3.335)--(8.703,3.336)%
  --(8.706,3.338)--(8.709,3.339)--(8.712,3.341)--(8.715,3.342)--(8.718,3.344)--(8.721,3.345)%
  --(8.724,3.347)--(8.727,3.348)--(8.730,3.350)--(8.733,3.351)--(8.735,3.352)--(8.738,3.354)%
  --(8.741,3.355)--(8.744,3.357)--(8.747,3.358)--(8.750,3.360)--(8.753,3.361)--(8.756,3.363)%
  --(8.759,3.364)--(8.762,3.366)--(8.765,3.367)--(8.768,3.369)--(8.771,3.370)--(8.774,3.371)%
  --(8.777,3.373)--(8.780,3.374)--(8.783,3.376)--(8.786,3.377)--(8.789,3.379)--(8.792,3.380)%
  --(8.795,3.382)--(8.798,3.383)--(8.801,3.385)--(8.804,3.386)--(8.807,3.388)--(8.810,3.389)%
  --(8.813,3.390)--(8.816,3.392)--(8.819,3.393)--(8.822,3.395)--(8.825,3.396)--(8.828,3.398)%
  --(8.831,3.399)--(8.834,3.401)--(8.837,3.402)--(8.840,3.404)--(8.843,3.405)--(8.846,3.407)%
  --(8.849,3.408)--(8.852,3.409)--(8.855,3.411)--(8.858,3.412)--(8.861,3.414)--(8.864,3.415)%
  --(8.867,3.417)--(8.870,3.418)--(8.873,3.420)--(8.876,3.421)--(8.879,3.423)--(8.882,3.424)%
  --(8.885,3.426)--(8.888,3.427)--(8.891,3.429)--(8.894,3.430)--(8.897,3.431)--(8.900,3.433)%
  --(8.903,3.434)--(8.906,3.436)--(8.909,3.437)--(8.912,3.439)--(8.915,3.440)--(8.918,3.442)%
  --(8.921,3.443)--(8.924,3.445)--(8.927,3.446)--(8.930,3.448)--(8.933,3.449)--(8.936,3.450)%
  --(8.939,3.452)--(8.942,3.453)--(8.944,3.455)--(8.947,3.456)--(8.950,3.458)--(8.953,3.459)%
  --(8.956,3.461)--(8.959,3.462)--(8.962,3.464)--(8.965,3.465)--(8.968,3.467)--(8.971,3.468)%
  --(8.974,3.470)--(8.977,3.471)--(8.980,3.472)--(8.983,3.474)--(8.986,3.475)--(8.989,3.477)%
  --(8.992,3.478)--(8.995,3.480)--(8.998,3.481)--(9.001,3.483)--(9.004,3.484)--(9.007,3.486)%
  --(9.010,3.487)--(9.013,3.489)--(9.016,3.490)--(9.019,3.492)--(9.022,3.493)--(9.025,3.494)%
  --(9.028,3.496)--(9.031,3.497)--(9.034,3.499)--(9.037,3.500)--(9.040,3.502)--(9.043,3.503)%
  --(9.046,3.505)--(9.049,3.506)--(9.052,3.508)--(9.055,3.509)--(9.058,3.511)--(9.061,3.512)%
  --(9.064,3.514)--(9.067,3.515)--(9.070,3.517)--(9.073,3.518)--(9.076,3.519)--(9.079,3.521)%
  --(9.082,3.522)--(9.085,3.524)--(9.088,3.525)--(9.091,3.527)--(9.094,3.528)--(9.097,3.530)%
  --(9.100,3.531)--(9.103,3.533)--(9.106,3.534)--(9.109,3.536)--(9.112,3.537)--(9.115,3.539)%
  --(9.118,3.540)--(9.121,3.541)--(9.124,3.543)--(9.127,3.544)--(9.130,3.546)--(9.133,3.547)%
  --(9.136,3.549)--(9.139,3.550)--(9.142,3.552)--(9.145,3.553)--(9.148,3.555)--(9.151,3.556)%
  --(9.153,3.558)--(9.156,3.559)--(9.159,3.561)--(9.162,3.562)--(9.165,3.564)--(9.168,3.565)%
  --(9.171,3.566)--(9.174,3.568)--(9.177,3.569)--(9.180,3.571)--(9.183,3.572)--(9.186,3.574)%
  --(9.189,3.575)--(9.192,3.577)--(9.195,3.578)--(9.198,3.580)--(9.201,3.581)--(9.204,3.583)%
  --(9.207,3.584)--(9.210,3.586)--(9.213,3.587)--(9.216,3.589)--(9.219,3.590)--(9.222,3.591)%
  --(9.225,3.593)--(9.228,3.594)--(9.231,3.596)--(9.234,3.597)--(9.237,3.599)--(9.240,3.600)%
  --(9.243,3.602)--(9.246,3.603)--(9.249,3.605)--(9.252,3.606)--(9.255,3.608)--(9.258,3.609)%
  --(9.261,3.611)--(9.264,3.612)--(9.267,3.614)--(9.270,3.615)--(9.273,3.617)--(9.276,3.618)%
  --(9.279,3.619)--(9.282,3.621)--(9.285,3.622)--(9.288,3.624)--(9.291,3.625)--(9.294,3.627)%
  --(9.297,3.628)--(9.300,3.630)--(9.303,3.631)--(9.306,3.633)--(9.309,3.634)--(9.312,3.636)%
  --(9.315,3.637)--(9.318,3.639)--(9.321,3.640)--(9.324,3.642)--(9.327,3.643)--(9.330,3.645)%
  --(9.333,3.646)--(9.336,3.647)--(9.339,3.649)--(9.342,3.650)--(9.345,3.652)--(9.348,3.653)%
  --(9.351,3.655)--(9.354,3.656)--(9.357,3.658)--(9.360,3.659)--(9.362,3.661)--(9.365,3.662)%
  --(9.368,3.664)--(9.371,3.665)--(9.374,3.667)--(9.377,3.668)--(9.380,3.670)--(9.383,3.671)%
  --(9.386,3.673)--(9.389,3.674)--(9.392,3.675)--(9.395,3.677)--(9.398,3.678)--(9.401,3.680)%
  --(9.404,3.681)--(9.407,3.683)--(9.410,3.684)--(9.413,3.686)--(9.416,3.687)--(9.419,3.689)%
  --(9.422,3.690)--(9.425,3.692)--(9.428,3.693)--(9.431,3.695)--(9.434,3.696)--(9.437,3.698)%
  --(9.440,3.699)--(9.443,3.701)--(9.446,3.702)--(9.449,3.704)--(9.452,3.705)--(9.455,3.706)%
  --(9.458,3.708)--(9.461,3.709)--(9.464,3.711)--(9.467,3.712)--(9.470,3.714)--(9.473,3.715)%
  --(9.476,3.717)--(9.479,3.718)--(9.482,3.720)--(9.485,3.721)--(9.488,3.723)--(9.491,3.724)%
  --(9.494,3.726)--(9.497,3.727)--(9.500,3.729)--(9.503,3.730)--(9.506,3.732)--(9.509,3.733)%
  --(9.512,3.735)--(9.515,3.736)--(9.518,3.737)--(9.521,3.739)--(9.524,3.740)--(9.527,3.742)%
  --(9.530,3.743)--(9.533,3.745)--(9.536,3.746)--(9.539,3.748)--(9.542,3.749)--(9.545,3.751)%
  --(9.548,3.752)--(9.551,3.754)--(9.554,3.755)--(9.557,3.757)--(9.560,3.758)--(9.563,3.760)%
  --(9.566,3.761)--(9.569,3.763)--(9.571,3.764)--(9.574,3.766)--(9.577,3.767)--(9.580,3.769)%
  --(9.583,3.770)--(9.586,3.771)--(9.589,3.773)--(9.592,3.774)--(9.595,3.776)--(9.598,3.777)%
  --(9.601,3.779)--(9.604,3.780)--(9.607,3.782)--(9.610,3.783)--(9.613,3.785)--(9.616,3.786)%
  --(9.619,3.788)--(9.622,3.789)--(9.625,3.791)--(9.628,3.792)--(9.631,3.794)--(9.634,3.795)%
  --(9.637,3.797)--(9.640,3.798)--(9.643,3.800)--(9.646,3.801)--(9.649,3.803)--(9.652,3.804)%
  --(9.655,3.806)--(9.658,3.807)--(9.661,3.808)--(9.664,3.810)--(9.667,3.811)--(9.670,3.813)%
  --(9.673,3.814)--(9.676,3.816)--(9.679,3.817)--(9.682,3.819)--(9.685,3.820)--(9.688,3.822)%
  --(9.691,3.823)--(9.694,3.825)--(9.697,3.826)--(9.700,3.828)--(9.703,3.829)--(9.706,3.831)%
  --(9.709,3.832)--(9.712,3.834)--(9.715,3.835)--(9.718,3.837)--(9.721,3.838)--(9.724,3.840)%
  --(9.727,3.841)--(9.730,3.843)--(9.733,3.844)--(9.736,3.845)--(9.739,3.847)--(9.742,3.848)%
  --(9.745,3.850)--(9.748,3.851)--(9.751,3.853)--(9.754,3.854)--(9.757,3.856)--(9.760,3.857)%
  --(9.763,3.859)--(9.766,3.860)--(9.769,3.862)--(9.772,3.863)--(9.775,3.865)--(9.778,3.866)%
  --(9.780,3.868)--(9.783,3.869)--(9.786,3.871)--(9.789,3.872)--(9.792,3.874)--(9.795,3.875)%
  --(9.798,3.877)--(9.801,3.878)--(9.804,3.880)--(9.807,3.881)--(9.810,3.883)--(9.813,3.884)%
  --(9.816,3.885)--(9.819,3.887)--(9.822,3.888)--(9.825,3.890)--(9.828,3.891)--(9.831,3.893)%
  --(9.834,3.894)--(9.837,3.896)--(9.840,3.897)--(9.843,3.899)--(9.846,3.900)--(9.849,3.902)%
  --(9.852,3.903)--(9.855,3.905)--(9.858,3.906)--(9.861,3.908)--(9.864,3.909)--(9.867,3.911)%
  --(9.870,3.912)--(9.873,3.914)--(9.876,3.915)--(9.879,3.917)--(9.882,3.918)--(9.885,3.920)%
  --(9.888,3.921)--(9.891,3.923)--(9.894,3.924)--(9.897,3.926)--(9.900,3.927)--(9.903,3.929)%
  --(9.906,3.930)--(9.909,3.931)--(9.912,3.933)--(9.915,3.934)--(9.918,3.936)--(9.921,3.937)%
  --(9.924,3.939)--(9.927,3.940)--(9.930,3.942)--(9.933,3.943)--(9.936,3.945)--(9.939,3.946)%
  --(9.942,3.948)--(9.945,3.949)--(9.948,3.951)--(9.951,3.952)--(9.954,3.954)--(9.957,3.955)%
  --(9.960,3.957)--(9.963,3.958)--(9.966,3.960)--(9.969,3.961)--(9.972,3.963)--(9.975,3.964)%
  --(9.978,3.966)--(9.981,3.967)--(9.984,3.969)--(9.987,3.970)--(9.990,3.972)--(9.992,3.973)%
  --(9.995,3.975)--(9.998,3.976)--(10.001,3.978)--(10.004,3.979)--(10.007,3.980)--(10.010,3.982)%
  --(10.013,3.983)--(10.016,3.985)--(10.019,3.986)--(10.022,3.988)--(10.025,3.989)--(10.028,3.991)%
  --(10.031,3.992)--(10.034,3.994)--(10.037,3.995)--(10.040,3.997)--(10.043,3.998)--(10.046,4.000)%
  --(10.049,4.001)--(10.052,4.003)--(10.055,4.004)--(10.058,4.006)--(10.061,4.007)--(10.064,4.009)%
  --(10.067,4.010)--(10.070,4.012)--(10.073,4.013)--(10.076,4.015)--(10.079,4.016)--(10.082,4.018)%
  --(10.085,4.019)--(10.088,4.021)--(10.091,4.022)--(10.094,4.024)--(10.097,4.025)--(10.100,4.027)%
  --(10.103,4.028)--(10.106,4.030)--(10.109,4.031)--(10.112,4.033)--(10.115,4.034)--(10.118,4.035)%
  --(10.121,4.037)--(10.124,4.038)--(10.127,4.040)--(10.130,4.041)--(10.133,4.043)--(10.136,4.044)%
  --(10.139,4.046)--(10.142,4.047)--(10.145,4.049)--(10.148,4.050)--(10.151,4.052)--(10.154,4.053)%
  --(10.157,4.055)--(10.160,4.056)--(10.163,4.058)--(10.166,4.059)--(10.169,4.061)--(10.172,4.062)%
  --(10.175,4.064)--(10.178,4.065)--(10.181,4.067)--(10.184,4.068)--(10.187,4.070)--(10.190,4.071)%
  --(10.193,4.073)--(10.196,4.074)--(10.199,4.076)--(10.201,4.077)--(10.204,4.079)--(10.207,4.080)%
  --(10.210,4.082)--(10.213,4.083)--(10.216,4.085)--(10.219,4.086)--(10.222,4.088)--(10.225,4.089)%
  --(10.228,4.091)--(10.231,4.092)--(10.234,4.094)--(10.237,4.095)--(10.240,4.096)--(10.243,4.098)%
  --(10.246,4.099)--(10.249,4.101)--(10.252,4.102)--(10.255,4.104)--(10.258,4.105)--(10.261,4.107)%
  --(10.264,4.108)--(10.267,4.110)--(10.270,4.111)--(10.273,4.113)--(10.276,4.114)--(10.279,4.116)%
  --(10.282,4.117)--(10.285,4.119)--(10.288,4.120)--(10.291,4.122)--(10.294,4.123)--(10.297,4.125)%
  --(10.300,4.126)--(10.303,4.128)--(10.306,4.129)--(10.309,4.131)--(10.312,4.132)--(10.315,4.134)%
  --(10.318,4.135)--(10.321,4.137)--(10.324,4.138)--(10.327,4.140)--(10.330,4.141)--(10.333,4.143)%
  --(10.336,4.144)--(10.339,4.146)--(10.342,4.147)--(10.345,4.149)--(10.348,4.150)--(10.351,4.152)%
  --(10.354,4.153)--(10.357,4.155)--(10.360,4.156)--(10.363,4.158)--(10.366,4.159)--(10.369,4.161)%
  --(10.372,4.162)--(10.375,4.164)--(10.378,4.165)--(10.381,4.167)--(10.384,4.168)--(10.387,4.169)%
  --(10.390,4.171)--(10.393,4.172)--(10.396,4.174)--(10.399,4.175)--(10.402,4.177)--(10.405,4.178)%
  --(10.408,4.180)--(10.410,4.181)--(10.413,4.183)--(10.416,4.184)--(10.419,4.186)--(10.422,4.187)%
  --(10.425,4.189)--(10.428,4.190)--(10.431,4.192)--(10.434,4.193)--(10.437,4.195)--(10.440,4.196)%
  --(10.443,4.198)--(10.446,4.199)--(10.449,4.201)--(10.452,4.202)--(10.455,4.204)--(10.458,4.205)%
  --(10.461,4.207)--(10.464,4.208)--(10.467,4.210)--(10.470,4.211)--(10.473,4.213)--(10.476,4.214)%
  --(10.479,4.216)--(10.482,4.217)--(10.485,4.219)--(10.488,4.220)--(10.491,4.222)--(10.494,4.223)%
  --(10.497,4.225)--(10.500,4.226)--(10.503,4.228)--(10.506,4.229)--(10.509,4.231)--(10.512,4.232)%
  --(10.515,4.234)--(10.518,4.235)--(10.521,4.237)--(10.524,4.238)--(10.527,4.240)--(10.530,4.241)%
  --(10.533,4.243)--(10.536,4.244)--(10.539,4.246)--(10.542,4.247)--(10.545,4.249)--(10.548,4.250)%
  --(10.551,4.252)--(10.554,4.253)--(10.557,4.255)--(10.560,4.256)--(10.563,4.258)--(10.566,4.259)%
  --(10.569,4.260)--(10.572,4.262)--(10.575,4.263)--(10.578,4.265)--(10.581,4.266)--(10.584,4.268)%
  --(10.587,4.269)--(10.590,4.271)--(10.593,4.272)--(10.596,4.274)--(10.599,4.275)--(10.602,4.277)%
  --(10.605,4.278)--(10.608,4.280)--(10.611,4.281)--(10.614,4.283)--(10.617,4.284)--(10.619,4.286)%
  --(10.622,4.287)--(10.625,4.289)--(10.628,4.290)--(10.631,4.292)--(10.634,4.293)--(10.637,4.295)%
  --(10.640,4.296)--(10.643,4.298)--(10.646,4.299)--(10.649,4.301)--(10.652,4.302)--(10.655,4.304)%
  --(10.658,4.305)--(10.661,4.307)--(10.664,4.308)--(10.667,4.310)--(10.670,4.311)--(10.673,4.313)%
  --(10.676,4.314)--(10.679,4.316)--(10.682,4.317)--(10.685,4.319)--(10.688,4.320)--(10.691,4.322)%
  --(10.694,4.323)--(10.697,4.325)--(10.700,4.326)--(10.703,4.328)--(10.706,4.329)--(10.709,4.331)%
  --(10.712,4.332)--(10.715,4.334)--(10.718,4.335)--(10.721,4.337)--(10.724,4.338)--(10.727,4.340)%
  --(10.730,4.341)--(10.733,4.343)--(10.736,4.344)--(10.739,4.346)--(10.742,4.347)--(10.745,4.349)%
  --(10.748,4.350)--(10.751,4.352)--(10.754,4.353)--(10.757,4.355)--(10.760,4.356)--(10.763,4.358)%
  --(10.766,4.359)--(10.769,4.361)--(10.772,4.362)--(10.775,4.364)--(10.778,4.365)--(10.781,4.367)%
  --(10.784,4.368)--(10.787,4.370)--(10.790,4.371)--(10.793,4.373)--(10.796,4.374)--(10.799,4.376)%
  --(10.802,4.377)--(10.805,4.379)--(10.808,4.380)--(10.811,4.382)--(10.814,4.383)--(10.817,4.385)%
  --(10.820,4.386)--(10.823,4.388)--(10.826,4.389)--(10.828,4.390)--(10.831,4.392)--(10.834,4.393)%
  --(10.837,4.395)--(10.840,4.396)--(10.843,4.398)--(10.846,4.399)--(10.849,4.401)--(10.852,4.402)%
  --(10.855,4.404)--(10.858,4.405)--(10.861,4.407)--(10.864,4.408)--(10.867,4.410)--(10.870,4.411)%
  --(10.873,4.413)--(10.876,4.414)--(10.879,4.416)--(10.882,4.417)--(10.885,4.419)--(10.888,4.420)%
  --(10.891,4.422)--(10.894,4.423)--(10.897,4.425)--(10.900,4.426)--(10.903,4.428)--(10.906,4.429)%
  --(10.909,4.431)--(10.912,4.432)--(10.915,4.434)--(10.918,4.435)--(10.921,4.437)--(10.924,4.438)%
  --(10.927,4.440)--(10.930,4.441)--(10.933,4.443)--(10.936,4.444)--(10.939,4.446)--(10.942,4.447)%
  --(10.945,4.449)--(10.948,4.450)--(10.951,4.452)--(10.954,4.453)--(10.957,4.455)--(10.960,4.456)%
  --(10.963,4.458)--(10.966,4.459)--(10.969,4.461)--(10.972,4.462)--(10.975,4.464)--(10.978,4.465)%
  --(10.981,4.467)--(10.984,4.468)--(10.987,4.470)--(10.990,4.471)--(10.993,4.473)--(10.996,4.474)%
  --(10.999,4.476)--(11.002,4.477)--(11.005,4.479)--(11.008,4.480)--(11.011,4.482)--(11.014,4.483)%
  --(11.017,4.485)--(11.020,4.486)--(11.023,4.488)--(11.026,4.489)--(11.029,4.491)--(11.032,4.492)%
  --(11.035,4.494)--(11.037,4.495)--(11.040,4.497)--(11.043,4.498)--(11.046,4.500)--(11.049,4.501)%
  --(11.052,4.503)--(11.055,4.504)--(11.058,4.506)--(11.061,4.507)--(11.064,4.509)--(11.067,4.510)%
  --(11.070,4.512)--(11.073,4.513)--(11.076,4.515)--(11.079,4.516)--(11.082,4.518)--(11.085,4.519)%
  --(11.088,4.521)--(11.091,4.522)--(11.094,4.524)--(11.097,4.525)--(11.100,4.527)--(11.103,4.528)%
  --(11.106,4.530)--(11.109,4.531)--(11.112,4.533)--(11.115,4.534)--(11.118,4.536)--(11.121,4.537)%
  --(11.124,4.539)--(11.127,4.540)--(11.130,4.542)--(11.133,4.543)--(11.136,4.545)--(11.139,4.546)%
  --(11.142,4.548)--(11.145,4.549)--(11.148,4.551)--(11.151,4.552)--(11.154,4.554)--(11.157,4.555)%
  --(11.160,4.557)--(11.163,4.558)--(11.166,4.560)--(11.169,4.561)--(11.172,4.563)--(11.175,4.564)%
  --(11.178,4.566)--(11.181,4.567)--(11.184,4.569)--(11.187,4.570)--(11.190,4.572)--(11.193,4.573)%
  --(11.196,4.575)--(11.199,4.576)--(11.202,4.578)--(11.205,4.579)--(11.208,4.581)--(11.211,4.582)%
  --(11.214,4.584)--(11.217,4.585)--(11.220,4.587)--(11.223,4.588)--(11.226,4.590)--(11.229,4.591)%
  --(11.232,4.593)--(11.235,4.594)--(11.238,4.596)--(11.241,4.597)--(11.244,4.599)--(11.247,4.600)%
  --(11.249,4.602)--(11.252,4.603)--(11.255,4.605)--(11.258,4.606)--(11.261,4.608)--(11.264,4.609)%
  --(11.267,4.611)--(11.270,4.612)--(11.273,4.614)--(11.276,4.615)--(11.279,4.617)--(11.282,4.618)%
  --(11.285,4.620)--(11.288,4.621)--(11.291,4.623)--(11.294,4.624)--(11.297,4.626)--(11.300,4.627)%
  --(11.303,4.629)--(11.306,4.630)--(11.309,4.632)--(11.312,4.633)--(11.315,4.635)--(11.318,4.636)%
  --(11.321,4.638)--(11.324,4.639)--(11.327,4.641)--(11.330,4.642)--(11.333,4.644)--(11.336,4.645)%
  --(11.339,4.647)--(11.342,4.648)--(11.345,4.650)--(11.348,4.651)--(11.351,4.653)--(11.354,4.654)%
  --(11.357,4.656)--(11.360,4.657)--(11.363,4.659)--(11.366,4.660)--(11.369,4.662)--(11.372,4.663)%
  --(11.375,4.665)--(11.378,4.666)--(11.381,4.668)--(11.384,4.669)--(11.387,4.671)--(11.390,4.672)%
  --(11.393,4.674)--(11.396,4.675)--(11.399,4.677)--(11.402,4.678)--(11.405,4.680)--(11.408,4.681)%
  --(11.411,4.683)--(11.414,4.684)--(11.417,4.686)--(11.420,4.687)--(11.423,4.689)--(11.426,4.690)%
  --(11.429,4.692)--(11.432,4.693)--(11.435,4.695)--(11.438,4.696)--(11.441,4.698)--(11.444,4.699)%
  --(11.447,4.701)--(11.450,4.702)--(11.453,4.704)--(11.456,4.705)--(11.458,4.707)--(11.461,4.708)%
  --(11.464,4.710)--(11.467,4.711)--(11.470,4.713)--(11.473,4.714)--(11.476,4.716)--(11.479,4.717)%
  --(11.482,4.719)--(11.485,4.720)--(11.488,4.722)--(11.491,4.723)--(11.494,4.725)--(11.497,4.726)%
  --(11.500,4.728)--(11.503,4.729)--(11.506,4.731)--(11.509,4.732)--(11.512,4.734)--(11.515,4.735)%
  --(11.518,4.737)--(11.521,4.738)--(11.524,4.740)--(11.527,4.741)--(11.530,4.743)--(11.533,4.744)%
  --(11.536,4.746)--(11.539,4.747)--(11.542,4.749)--(11.545,4.750)--(11.548,4.752)--(11.551,4.753)%
  --(11.554,4.755)--(11.557,4.756)--(11.560,4.758)--(11.563,4.759)--(11.566,4.761)--(11.569,4.762)%
  --(11.572,4.764)--(11.575,4.765)--(11.578,4.767)--(11.581,4.768)--(11.584,4.770)--(11.587,4.771)%
  --(11.590,4.773)--(11.593,4.774)--(11.596,4.776)--(11.599,4.777)--(11.602,4.779)--(11.605,4.780)%
  --(11.608,4.782)--(11.611,4.783)--(11.614,4.785)--(11.617,4.786)--(11.620,4.788)--(11.623,4.789)%
  --(11.626,4.791)--(11.629,4.792)--(11.632,4.794)--(11.635,4.795)--(11.638,4.797)--(11.641,4.798)%
  --(11.644,4.800)--(11.647,4.801)--(11.650,4.803)--(11.653,4.804)--(11.656,4.806)--(11.659,4.807)%
  --(11.662,4.809)--(11.665,4.810)--(11.667,4.812)--(11.670,4.813)--(11.673,4.815)--(11.676,4.816)%
  --(11.679,4.818)--(11.682,4.819)--(11.685,4.821)--(11.688,4.822)--(11.691,4.824)--(11.694,4.825)%
  --(11.697,4.827)--(11.700,4.828)--(11.703,4.830)--(11.706,4.831)--(11.709,4.833)--(11.712,4.834)%
  --(11.715,4.836)--(11.718,4.837)--(11.721,4.839)--(11.724,4.840)--(11.727,4.842)--(11.730,4.843)%
  --(11.733,4.845)--(11.736,4.846)--(11.739,4.848)--(11.742,4.849)--(11.745,4.851)--(11.748,4.852)%
  --(11.751,4.854)--(11.754,4.855)--(11.757,4.857)--(11.760,4.858)--(11.763,4.860)--(11.766,4.861)%
  --(11.769,4.863)--(11.772,4.864)--(11.775,4.866)--(11.778,4.868)--(11.781,4.869)--(11.784,4.871)%
  --(11.787,4.872)--(11.790,4.874)--(11.793,4.875)--(11.796,4.877)--(11.799,4.878)--(11.802,4.880)%
  --(11.805,4.881)--(11.808,4.883)--(11.811,4.884)--(11.814,4.886)--(11.817,4.887)--(11.820,4.889)%
  --(11.823,4.890)--(11.826,4.892)--(11.829,4.893)--(11.832,4.895)--(11.835,4.896)--(11.838,4.898)%
  --(11.841,4.899)--(11.844,4.901)--(11.847,4.902)--(11.850,4.904)--(11.853,4.905)--(11.856,4.907)%
  --(11.859,4.908)--(11.862,4.910)--(11.865,4.911)--(11.868,4.913)--(11.871,4.914)--(11.874,4.916)%
  --(11.876,4.917)--(11.879,4.919)--(11.882,4.920)--(11.885,4.922)--(11.888,4.923)--(11.891,4.925)%
  --(11.894,4.926)--(11.897,4.928)--(11.900,4.929)--(11.903,4.931)--(11.906,4.932)--(11.909,4.934)%
  --(11.912,4.935)--(11.915,4.937)--(11.918,4.938)--(11.921,4.940)--(11.924,4.941)--(11.927,4.943)%
  --(11.930,4.944)--(11.933,4.946)--(11.936,4.947)--(11.939,4.949)--(11.942,4.950)--(11.945,4.952)%
  --(11.948,4.953)--(11.951,4.955)--(11.954,4.956)--(11.957,4.958)--(11.960,4.959)--(11.963,4.961)%
  --(11.966,4.962)--(11.969,4.964)--(11.972,4.965)--(11.975,4.967)--(11.978,4.968)--(11.981,4.970)%
  --(11.984,4.971)--(11.987,4.973)--(11.990,4.974)--(11.993,4.976)--(11.996,4.977)--(11.999,4.979)%
  --(12.002,4.980)--(12.005,4.982)--(12.008,4.983)--(12.011,4.985)--(12.014,4.986)--(12.017,4.988)%
  --(12.020,4.989)--(12.023,4.991)--(12.026,4.992)--(12.029,4.994)--(12.032,4.995)--(12.035,4.997)%
  --(12.038,4.998)--(12.041,5.000)--(12.044,5.001)--(12.047,5.003)--(12.050,5.004)--(12.053,5.006)%
  --(12.056,5.007)--(12.059,5.009)--(12.062,5.010)--(12.065,5.012)--(12.068,5.013)--(12.071,5.015)%
  --(12.074,5.016)--(12.077,5.018)--(12.080,5.020)--(12.083,5.021)--(12.085,5.023)--(12.088,5.024)%
  --(12.091,5.026)--(12.094,5.027)--(12.097,5.029)--(12.100,5.030)--(12.103,5.032)--(12.106,5.033)%
  --(12.109,5.035)--(12.112,5.036)--(12.115,5.038)--(12.118,5.039)--(12.121,5.041)--(12.124,5.042)%
  --(12.127,5.044)--(12.130,5.045)--(12.133,5.047)--(12.136,5.048)--(12.139,5.050)--(12.142,5.051)%
  --(12.145,5.053)--(12.148,5.054)--(12.151,5.056)--(12.154,5.057)--(12.157,5.059)--(12.160,5.060)%
  --(12.163,5.062)--(12.166,5.063)--(12.169,5.065)--(12.172,5.066)--(12.175,5.068)--(12.178,5.069)%
  --(12.181,5.071)--(12.184,5.072)--(12.187,5.074)--(12.190,5.075)--(12.193,5.077)--(12.196,5.078)%
  --(12.199,5.080)--(12.202,5.081)--(12.205,5.083)--(12.208,5.084)--(12.211,5.086)--(12.214,5.087)%
  --(12.217,5.089)--(12.220,5.090)--(12.223,5.092)--(12.226,5.093)--(12.229,5.095)--(12.232,5.096)%
  --(12.235,5.098)--(12.238,5.099)--(12.241,5.101)--(12.244,5.102)--(12.247,5.104)--(12.250,5.105)%
  --(12.253,5.107)--(12.256,5.108)--(12.259,5.110)--(12.262,5.111)--(12.265,5.113)--(12.268,5.114)%
  --(12.271,5.116)--(12.274,5.117)--(12.277,5.119)--(12.280,5.120)--(12.283,5.122)--(12.286,5.123)%
  --(12.289,5.125)--(12.292,5.126)--(12.295,5.128)--(12.297,5.129)--(12.300,5.131)--(12.303,5.132)%
  --(12.306,5.134)--(12.309,5.136)--(12.312,5.137)--(12.315,5.139)--(12.318,5.140)--(12.321,5.142)%
  --(12.324,5.143)--(12.327,5.145)--(12.330,5.146)--(12.333,5.148)--(12.336,5.149)--(12.339,5.151)%
  --(12.342,5.152)--(12.345,5.154)--(12.348,5.155)--(12.351,5.157)--(12.354,5.158)--(12.357,5.160)%
  --(12.360,5.161)--(12.363,5.163)--(12.366,5.164)--(12.369,5.166)--(12.372,5.167)--(12.375,5.169)%
  --(12.378,5.170)--(12.381,5.172)--(12.384,5.173)--(12.387,5.175)--(12.390,5.176)--(12.393,5.178)%
  --(12.396,5.179)--(12.399,5.181)--(12.402,5.182)--(12.405,5.184)--(12.408,5.185)--(12.411,5.187)%
  --(12.414,5.188)--(12.417,5.190)--(12.420,5.191)--(12.423,5.193)--(12.426,5.194)--(12.429,5.196)%
  --(12.432,5.197)--(12.435,5.199)--(12.438,5.200)--(12.441,5.202)--(12.444,5.203)--(12.447,5.205)%
  --(12.450,5.206)--(12.453,5.208)--(12.456,5.209)--(12.459,5.211)--(12.462,5.212)--(12.465,5.214)%
  --(12.468,5.215)--(12.471,5.217)--(12.474,5.218)--(12.477,5.220)--(12.480,5.221)--(12.483,5.223)%
  --(12.486,5.224)--(12.489,5.226)--(12.492,5.227)--(12.495,5.229)--(12.498,5.231)--(12.501,5.232)%
  --(12.504,5.234)--(12.506,5.235)--(12.509,5.237)--(12.512,5.238)--(12.515,5.240)--(12.518,5.241)%
  --(12.521,5.243)--(12.524,5.244)--(12.527,5.246)--(12.530,5.247)--(12.533,5.249)--(12.536,5.250)%
  --(12.539,5.252)--(12.542,5.253)--(12.545,5.255)--(12.548,5.256)--(12.551,5.258)--(12.554,5.259)%
  --(12.557,5.261)--(12.560,5.262)--(12.563,5.264)--(12.566,5.265)--(12.569,5.267)--(12.572,5.268)%
  --(12.575,5.270)--(12.578,5.271)--(12.581,5.273)--(12.584,5.274)--(12.587,5.276)--(12.590,5.277)%
  --(12.593,5.279)--(12.596,5.280)--(12.599,5.282)--(12.602,5.283)--(12.605,5.285)--(12.608,5.286)%
  --(12.611,5.288)--(12.614,5.289)--(12.617,5.291)--(12.620,5.292)--(12.623,5.294)--(12.626,5.295)%
  --(12.629,5.297)--(12.632,5.298)--(12.635,5.300)--(12.638,5.301)--(12.641,5.303)--(12.644,5.304)%
  --(12.647,5.306)--(12.650,5.307)--(12.653,5.309)--(12.656,5.310)--(12.659,5.312)--(12.662,5.313)%
  --(12.665,5.315)--(12.668,5.317)--(12.671,5.318)--(12.674,5.320)--(12.677,5.321)--(12.680,5.323)%
  --(12.683,5.324)--(12.686,5.326)--(12.689,5.327)--(12.692,5.329)--(12.695,5.330)--(12.698,5.332)%
  --(12.701,5.333)--(12.704,5.335)--(12.707,5.336)--(12.710,5.338)--(12.713,5.339)--(12.715,5.341)%
  --(12.718,5.342)--(12.721,5.344)--(12.724,5.345)--(12.727,5.347)--(12.730,5.348)--(12.733,5.350)%
  --(12.736,5.351)--(12.739,5.353)--(12.742,5.354)--(12.745,5.356)--(12.748,5.357)--(12.751,5.359)%
  --(12.754,5.360)--(12.757,5.362)--(12.760,5.363)--(12.763,5.365)--(12.766,5.366)--(12.769,5.368)%
  --(12.772,5.369)--(12.775,5.371)--(12.778,5.372)--(12.781,5.374)--(12.784,5.375)--(12.787,5.377)%
  --(12.790,5.378)--(12.793,5.380)--(12.796,5.381)--(12.799,5.383)--(12.802,5.384)--(12.805,5.386)%
  --(12.808,5.387)--(12.811,5.389)--(12.814,5.390)--(12.817,5.392)--(12.820,5.394)--(12.823,5.395)%
  --(12.826,5.397)--(12.829,5.398)--(12.832,5.400)--(12.835,5.401)--(12.838,5.403)--(12.841,5.404)%
  --(12.844,5.406)--(12.847,5.407)--(12.850,5.409)--(12.853,5.410)--(12.856,5.412)--(12.859,5.413)%
  --(12.862,5.415)--(12.865,5.416)--(12.868,5.418)--(12.871,5.419)--(12.874,5.421)--(12.877,5.422)%
  --(12.880,5.424)--(12.883,5.425)--(12.886,5.427)--(12.889,5.428)--(12.892,5.430)--(12.895,5.431)%
  --(12.898,5.433)--(12.901,5.434)--(12.904,5.436)--(12.907,5.437)--(12.910,5.439)--(12.913,5.440)%
  --(12.916,5.442)--(12.919,5.443)--(12.922,5.445)--(12.924,5.446)--(12.927,5.448)--(12.930,5.449)%
  --(12.933,5.451)--(12.936,5.452)--(12.939,5.454)--(12.942,5.455)--(12.945,5.457)--(12.948,5.458)%
  --(12.951,5.460)--(12.954,5.461)--(12.957,5.463)--(12.960,5.464)--(12.963,5.466)--(12.966,5.468)%
  --(12.969,5.469)--(12.972,5.471)--(12.975,5.472)--(12.978,5.474)--(12.981,5.475)--(12.984,5.477)%
  --(12.987,5.478)--(12.990,5.480)--(12.993,5.481)--(12.996,5.483)--(12.999,5.484)--(13.002,5.486)%
  --(13.005,5.487)--(13.008,5.489)--(13.011,5.490)--(13.014,5.492)--(13.017,5.493)--(13.020,5.495)%
  --(13.023,5.496)--(13.026,5.498)--(13.029,5.499)--(13.032,5.501)--(13.035,5.502)--(13.038,5.504)%
  --(13.041,5.505)--(13.044,5.507)--(13.047,5.508)--(13.050,5.510)--(13.053,5.511)--(13.056,5.513)%
  --(13.059,5.514)--(13.062,5.516)--(13.065,5.517)--(13.068,5.519)--(13.071,5.520)--(13.074,5.522)%
  --(13.077,5.523)--(13.080,5.525)--(13.083,5.526)--(13.086,5.528)--(13.089,5.529)--(13.092,5.531)%
  --(13.095,5.532)--(13.098,5.534)--(13.101,5.536)--(13.104,5.537)--(13.107,5.539)--(13.110,5.540)%
  --(13.113,5.542)--(13.116,5.543)--(13.119,5.545)--(13.122,5.546)--(13.125,5.548)--(13.128,5.549)%
  --(13.131,5.551)--(13.133,5.552)--(13.136,5.554)--(13.139,5.555)--(13.142,5.557)--(13.145,5.558)%
  --(13.148,5.560)--(13.151,5.561)--(13.154,5.563)--(13.157,5.564)--(13.160,5.566)--(13.163,5.567)%
  --(13.166,5.569)--(13.169,5.570)--(13.172,5.572)--(13.175,5.573)--(13.178,5.575)--(13.181,5.576)%
  --(13.184,5.578)--(13.187,5.579)--(13.190,5.581)--(13.193,5.582)--(13.196,5.584)--(13.199,5.585)%
  --(13.202,5.587)--(13.205,5.588)--(13.208,5.590)--(13.211,5.591)--(13.214,5.593)--(13.217,5.594)%
  --(13.220,5.596)--(13.223,5.597)--(13.226,5.599)--(13.229,5.601)--(13.232,5.602)--(13.235,5.604)%
  --(13.238,5.605)--(13.241,5.607)--(13.244,5.608)--(13.247,5.610)--(13.250,5.611)--(13.253,5.613)%
  --(13.256,5.614)--(13.259,5.616)--(13.262,5.617)--(13.265,5.619)--(13.268,5.620)--(13.271,5.622)%
  --(13.274,5.623)--(13.277,5.625)--(13.280,5.626)--(13.283,5.628)--(13.286,5.629)--(13.289,5.631)%
  --(13.292,5.632)--(13.295,5.634)--(13.298,5.635)--(13.301,5.637)--(13.304,5.638)--(13.307,5.640)%
  --(13.310,5.641)--(13.313,5.643)--(13.316,5.644)--(13.319,5.646)--(13.322,5.647)--(13.325,5.649)%
  --(13.328,5.650)--(13.331,5.652)--(13.334,5.653)--(13.337,5.655)--(13.340,5.656)--(13.342,5.658)%
  --(13.345,5.659)--(13.348,5.661)--(13.351,5.663)--(13.354,5.664)--(13.357,5.666)--(13.360,5.667)%
  --(13.363,5.669)--(13.366,5.670)--(13.369,5.672)--(13.372,5.673)--(13.375,5.675)--(13.378,5.676)%
  --(13.381,5.678)--(13.384,5.679)--(13.387,5.681)--(13.390,5.682)--(13.393,5.684)--(13.396,5.685)%
  --(13.399,5.687)--(13.402,5.688)--(13.405,5.690)--(13.408,5.691)--(13.411,5.693)--(13.414,5.694)%
  --(13.417,5.696)--(13.420,5.697)--(13.423,5.699)--(13.426,5.700)--(13.429,5.702)--(13.432,5.703)%
  --(13.435,5.705)--(13.438,5.706)--(13.441,5.708)--(13.444,5.709);
\gpcolor{color=gp lt color border}
\node[gp node left] at (2.972,7.989) {$\rho \approx \nicefrac{1}{3} \cdot \rho_{\rm{max}}$};
\gpcolor{rgb color={0.000,0.620,0.451}}
\draw[gp path] (1.872,7.989)--(2.788,7.989);
\draw[gp path] (1.507,2.512)--(1.510,2.510)--(1.513,2.508)--(1.516,2.506)--(1.519,2.504)%
  --(1.522,2.502)--(1.525,2.499)--(1.528,2.497)--(1.531,2.495)--(1.534,2.493)--(1.537,2.491)%
  --(1.540,2.489)--(1.543,2.487)--(1.546,2.485)--(1.549,2.482)--(1.552,2.480)--(1.555,2.478)%
  --(1.558,2.476)--(1.561,2.474)--(1.564,2.472)--(1.567,2.470)--(1.570,2.468)--(1.573,2.466)%
  --(1.576,2.463)--(1.579,2.461)--(1.582,2.459)--(1.585,2.457)--(1.588,2.455)--(1.591,2.453)%
  --(1.594,2.451)--(1.597,2.449)--(1.600,2.447)--(1.603,2.445)--(1.606,2.442)--(1.609,2.440)%
  --(1.611,2.438)--(1.614,2.436)--(1.617,2.434)--(1.620,2.432)--(1.623,2.430)--(1.626,2.428)%
  --(1.629,2.426)--(1.632,2.424)--(1.635,2.422)--(1.638,2.419)--(1.641,2.417)--(1.644,2.415)%
  --(1.647,2.413)--(1.650,2.411)--(1.653,2.409)--(1.656,2.407)--(1.659,2.405)--(1.662,2.403)%
  --(1.665,2.401)--(1.668,2.399)--(1.671,2.397)--(1.674,2.394)--(1.677,2.392)--(1.680,2.390)%
  --(1.683,2.388)--(1.686,2.386)--(1.689,2.384)--(1.692,2.382)--(1.695,2.380)--(1.698,2.378)%
  --(1.701,2.376)--(1.704,2.374)--(1.707,2.372)--(1.710,2.370)--(1.713,2.368)--(1.716,2.366)%
  --(1.719,2.364)--(1.722,2.362)--(1.725,2.360)--(1.728,2.357)--(1.731,2.355)--(1.734,2.353)%
  --(1.737,2.351)--(1.740,2.349)--(1.743,2.347)--(1.746,2.345)--(1.749,2.343)--(1.752,2.341)%
  --(1.755,2.339)--(1.758,2.337)--(1.761,2.335)--(1.764,2.333)--(1.767,2.331)--(1.770,2.329)%
  --(1.773,2.327)--(1.776,2.325)--(1.779,2.323)--(1.782,2.321)--(1.785,2.319)--(1.788,2.317)%
  --(1.791,2.315)--(1.794,2.313)--(1.797,2.311)--(1.800,2.309)--(1.803,2.307)--(1.806,2.305)%
  --(1.809,2.303)--(1.812,2.301)--(1.815,2.299)--(1.818,2.297)--(1.820,2.295)--(1.823,2.293)%
  --(1.826,2.291)--(1.829,2.290)--(1.832,2.288)--(1.835,2.286)--(1.838,2.284)--(1.841,2.282)%
  --(1.844,2.280)--(1.847,2.278)--(1.850,2.276)--(1.853,2.274)--(1.856,2.272)--(1.859,2.270)%
  --(1.862,2.268)--(1.865,2.266)--(1.868,2.264)--(1.871,2.262)--(1.874,2.261)--(1.877,2.259)%
  --(1.880,2.257)--(1.883,2.255)--(1.886,2.253)--(1.889,2.251)--(1.892,2.249)--(1.895,2.247)%
  --(1.898,2.245)--(1.901,2.243)--(1.904,2.242)--(1.907,2.240)--(1.910,2.238)--(1.913,2.236)%
  --(1.916,2.234)--(1.919,2.232)--(1.922,2.230)--(1.925,2.229)--(1.928,2.227)--(1.931,2.225)%
  --(1.934,2.223)--(1.937,2.221)--(1.940,2.219)--(1.943,2.217)--(1.946,2.216)--(1.949,2.214)%
  --(1.952,2.212)--(1.955,2.210)--(1.958,2.208)--(1.961,2.207)--(1.964,2.205)--(1.967,2.203)%
  --(1.970,2.201)--(1.973,2.199)--(1.976,2.198)--(1.979,2.196)--(1.982,2.194)--(1.985,2.192)%
  --(1.988,2.190)--(1.991,2.189)--(1.994,2.187)--(1.997,2.185)--(2.000,2.183)--(2.003,2.182)%
  --(2.006,2.180)--(2.009,2.178)--(2.012,2.176)--(2.015,2.175)--(2.018,2.173)--(2.021,2.171)%
  --(2.024,2.169)--(2.027,2.168)--(2.029,2.166)--(2.032,2.164)--(2.035,2.162)--(2.038,2.161)%
  --(2.041,2.159)--(2.044,2.157)--(2.047,2.156)--(2.050,2.154)--(2.053,2.152)--(2.056,2.151)%
  --(2.059,2.149)--(2.062,2.147)--(2.065,2.145)--(2.068,2.144)--(2.071,2.142)--(2.074,2.140)%
  --(2.077,2.139)--(2.080,2.137)--(2.083,2.136)--(2.086,2.134)--(2.089,2.132)--(2.092,2.131)%
  --(2.095,2.129)--(2.098,2.127)--(2.101,2.126)--(2.104,2.124)--(2.107,2.122)--(2.110,2.121)%
  --(2.113,2.119)--(2.116,2.118)--(2.119,2.116)--(2.122,2.114)--(2.125,2.113)--(2.128,2.111)%
  --(2.131,2.110)--(2.134,2.108)--(2.137,2.107)--(2.140,2.105)--(2.143,2.103)--(2.146,2.102)%
  --(2.149,2.100)--(2.152,2.099)--(2.155,2.097)--(2.158,2.096)--(2.161,2.094)--(2.164,2.093)%
  --(2.167,2.091)--(2.170,2.090)--(2.173,2.088)--(2.176,2.087)--(2.179,2.085)--(2.182,2.084)%
  --(2.185,2.082)--(2.188,2.081)--(2.191,2.079)--(2.194,2.078)--(2.197,2.076)--(2.200,2.075)%
  --(2.203,2.073)--(2.206,2.072)--(2.209,2.070)--(2.212,2.069)--(2.215,2.067)--(2.218,2.066)%
  --(2.221,2.064)--(2.224,2.063)--(2.227,2.062)--(2.230,2.060)--(2.233,2.059)--(2.236,2.057)%
  --(2.238,2.056)--(2.241,2.054)--(2.244,2.053)--(2.247,2.052)--(2.250,2.050)--(2.253,2.049)%
  --(2.256,2.047)--(2.259,2.046)--(2.262,2.045)--(2.265,2.043)--(2.268,2.042)--(2.271,2.041)%
  --(2.274,2.039)--(2.277,2.038)--(2.280,2.037)--(2.283,2.035)--(2.286,2.034)--(2.289,2.033)%
  --(2.292,2.031)--(2.295,2.030)--(2.298,2.029)--(2.301,2.027)--(2.304,2.026)--(2.307,2.025)%
  --(2.310,2.023)--(2.313,2.022)--(2.316,2.021)--(2.319,2.019)--(2.322,2.018)--(2.325,2.017)%
  --(2.328,2.016)--(2.331,2.014)--(2.334,2.013)--(2.337,2.012)--(2.340,2.011)--(2.343,2.009)%
  --(2.346,2.008)--(2.349,2.007)--(2.352,2.006)--(2.355,2.004)--(2.358,2.003)--(2.361,2.002)%
  --(2.364,2.001)--(2.367,2.000)--(2.370,1.998)--(2.373,1.997)--(2.376,1.996)--(2.379,1.995)%
  --(2.382,1.994)--(2.385,1.992)--(2.388,1.991)--(2.391,1.990)--(2.394,1.989)--(2.397,1.988)%
  --(2.400,1.987)--(2.403,1.985)--(2.406,1.984)--(2.409,1.983)--(2.412,1.982)--(2.415,1.981)%
  --(2.418,1.980)--(2.421,1.979)--(2.424,1.978)--(2.427,1.976)--(2.430,1.975)--(2.433,1.974)%
  --(2.436,1.973)--(2.439,1.972)--(2.442,1.971)--(2.445,1.970)--(2.447,1.969)--(2.450,1.968)%
  --(2.453,1.967)--(2.456,1.966)--(2.459,1.965)--(2.462,1.963)--(2.465,1.962)--(2.468,1.961)%
  --(2.471,1.960)--(2.474,1.959)--(2.477,1.958)--(2.480,1.957)--(2.483,1.956)--(2.486,1.955)%
  --(2.489,1.954)--(2.492,1.953)--(2.495,1.952)--(2.498,1.951)--(2.501,1.950)--(2.504,1.949)%
  --(2.507,1.948)--(2.510,1.947)--(2.513,1.946)--(2.516,1.945)--(2.519,1.944)--(2.522,1.943)%
  --(2.525,1.943)--(2.528,1.942)--(2.531,1.941)--(2.534,1.940)--(2.537,1.939)--(2.540,1.938)%
  --(2.543,1.937)--(2.546,1.936)--(2.549,1.935)--(2.552,1.934)--(2.555,1.933)--(2.558,1.932)%
  --(2.561,1.932)--(2.564,1.931)--(2.567,1.930)--(2.570,1.929)--(2.573,1.928)--(2.576,1.927)%
  --(2.579,1.926)--(2.582,1.925)--(2.585,1.925)--(2.588,1.924)--(2.591,1.923)--(2.594,1.922)%
  --(2.597,1.921)--(2.600,1.920)--(2.603,1.920)--(2.606,1.919)--(2.609,1.918)--(2.612,1.917)%
  --(2.615,1.916)--(2.618,1.915)--(2.621,1.915)--(2.624,1.914)--(2.627,1.913)--(2.630,1.912)%
  --(2.633,1.912)--(2.636,1.911)--(2.639,1.910)--(2.642,1.909)--(2.645,1.909)--(2.648,1.908)%
  --(2.651,1.907)--(2.654,1.906)--(2.656,1.906)--(2.659,1.905)--(2.662,1.904)--(2.665,1.903)%
  --(2.668,1.903)--(2.671,1.902)--(2.674,1.901)--(2.677,1.900)--(2.680,1.900)--(2.683,1.899)%
  --(2.686,1.898)--(2.689,1.898)--(2.692,1.897)--(2.695,1.896)--(2.698,1.896)--(2.701,1.895)%
  --(2.704,1.894)--(2.707,1.894)--(2.710,1.893)--(2.713,1.892)--(2.716,1.892)--(2.719,1.891)%
  --(2.722,1.890)--(2.725,1.890)--(2.728,1.889)--(2.731,1.889)--(2.734,1.888)--(2.737,1.887)%
  --(2.740,1.887)--(2.743,1.886)--(2.746,1.886)--(2.749,1.885)--(2.752,1.884)--(2.755,1.884)%
  --(2.758,1.883)--(2.761,1.883)--(2.764,1.882)--(2.767,1.881)--(2.770,1.881)--(2.773,1.880)%
  --(2.776,1.880)--(2.779,1.879)--(2.782,1.879)--(2.785,1.878)--(2.788,1.878)--(2.791,1.877)%
  --(2.794,1.877)--(2.797,1.876)--(2.800,1.876)--(2.803,1.875)--(2.806,1.875)--(2.809,1.874)%
  --(2.812,1.873)--(2.815,1.873)--(2.818,1.873)--(2.821,1.872)--(2.824,1.872)--(2.827,1.871)%
  --(2.830,1.871)--(2.833,1.870)--(2.836,1.870)--(2.839,1.869)--(2.842,1.869)--(2.845,1.868)%
  --(2.848,1.868)--(2.851,1.867)--(2.854,1.867)--(2.857,1.867)--(2.860,1.866)--(2.863,1.866)%
  --(2.866,1.865)--(2.868,1.865)--(2.871,1.864)--(2.874,1.864)--(2.877,1.864)--(2.880,1.863)%
  --(2.883,1.863)--(2.886,1.862)--(2.889,1.862)--(2.892,1.862)--(2.895,1.861)--(2.898,1.861)%
  --(2.901,1.860)--(2.904,1.860)--(2.907,1.860)--(2.910,1.859)--(2.913,1.859)--(2.916,1.859)%
  --(2.919,1.858)--(2.922,1.858)--(2.925,1.858)--(2.928,1.857)--(2.931,1.857)--(2.934,1.857)%
  --(2.937,1.856)--(2.940,1.856)--(2.943,1.856)--(2.946,1.855)--(2.949,1.855)--(2.952,1.855)%
  --(2.955,1.855)--(2.958,1.854)--(2.961,1.854)--(2.964,1.854)--(2.967,1.853)--(2.970,1.853)%
  --(2.973,1.853)--(2.976,1.853)--(2.979,1.852)--(2.982,1.852)--(2.985,1.852)--(2.988,1.852)%
  --(2.991,1.851)--(2.994,1.851)--(2.997,1.851)--(3.000,1.851)--(3.003,1.850)--(3.006,1.850)%
  --(3.009,1.850)--(3.012,1.850)--(3.015,1.849)--(3.018,1.849)--(3.021,1.849)--(3.024,1.849)%
  --(3.027,1.849)--(3.030,1.848)--(3.033,1.848)--(3.036,1.848)--(3.039,1.848)--(3.042,1.848)%
  --(3.045,1.847)--(3.048,1.847)--(3.051,1.847)--(3.054,1.847)--(3.057,1.847)--(3.060,1.847)%
  --(3.063,1.846)--(3.066,1.846)--(3.069,1.846)--(3.072,1.846)--(3.075,1.846)--(3.077,1.846)%
  --(3.080,1.846)--(3.083,1.845)--(3.086,1.845)--(3.089,1.845)--(3.092,1.845)--(3.095,1.845)%
  --(3.098,1.845)--(3.101,1.845)--(3.104,1.845)--(3.107,1.845)--(3.110,1.845)--(3.113,1.844)%
  --(3.116,1.844)--(3.119,1.844)--(3.122,1.844)--(3.125,1.844)--(3.128,1.844)--(3.131,1.844)%
  --(3.134,1.844)--(3.137,1.844)--(3.140,1.844)--(3.143,1.844)--(3.146,1.844)--(3.149,1.844)%
  --(3.152,1.844)--(3.155,1.844)--(3.158,1.843)--(3.161,1.843)--(3.164,1.843)--(3.167,1.843)%
  --(3.170,1.843)--(3.173,1.843)--(3.176,1.843)--(3.179,1.843)--(3.182,1.843)--(3.185,1.843)%
  --(3.188,1.843)--(3.191,1.843)--(3.194,1.843)--(3.197,1.843)--(3.200,1.843)--(3.203,1.843)%
  --(3.206,1.843)--(3.209,1.843)--(3.212,1.843)--(3.215,1.843)--(3.218,1.843)--(3.221,1.844)%
  --(3.224,1.844)--(3.227,1.844)--(3.230,1.844)--(3.233,1.844)--(3.236,1.844)--(3.239,1.844)%
  --(3.242,1.844)--(3.245,1.844)--(3.248,1.844)--(3.251,1.844)--(3.254,1.844)--(3.257,1.844)%
  --(3.260,1.844)--(3.263,1.844)--(3.266,1.844)--(3.269,1.845)--(3.272,1.845)--(3.275,1.845)%
  --(3.278,1.845)--(3.281,1.845)--(3.284,1.845)--(3.286,1.845)--(3.289,1.845)--(3.292,1.845)%
  --(3.295,1.846)--(3.298,1.846)--(3.301,1.846)--(3.304,1.846)--(3.307,1.846)--(3.310,1.846)%
  --(3.313,1.846)--(3.316,1.846)--(3.319,1.847)--(3.322,1.847)--(3.325,1.847)--(3.328,1.847)%
  --(3.331,1.847)--(3.334,1.847)--(3.337,1.848)--(3.340,1.848)--(3.343,1.848)--(3.346,1.848)%
  --(3.349,1.848)--(3.352,1.848)--(3.355,1.849)--(3.358,1.849)--(3.361,1.849)--(3.364,1.849)%
  --(3.367,1.849)--(3.370,1.850)--(3.373,1.850)--(3.376,1.850)--(3.379,1.850)--(3.382,1.850)%
  --(3.385,1.851)--(3.388,1.851)--(3.391,1.851)--(3.394,1.851)--(3.397,1.851)--(3.400,1.852)%
  --(3.403,1.852)--(3.406,1.852)--(3.409,1.852)--(3.412,1.853)--(3.415,1.853)--(3.418,1.853)%
  --(3.421,1.853)--(3.424,1.854)--(3.427,1.854)--(3.430,1.854)--(3.433,1.854)--(3.436,1.855)%
  --(3.439,1.855)--(3.442,1.855)--(3.445,1.856)--(3.448,1.856)--(3.451,1.856)--(3.454,1.856)%
  --(3.457,1.857)--(3.460,1.857)--(3.463,1.857)--(3.466,1.858)--(3.469,1.858)--(3.472,1.858)%
  --(3.475,1.858)--(3.478,1.859)--(3.481,1.859)--(3.484,1.859)--(3.487,1.860)--(3.490,1.860)%
  --(3.493,1.860)--(3.495,1.861)--(3.498,1.861)--(3.501,1.861)--(3.504,1.862)--(3.507,1.862)%
  --(3.510,1.862)--(3.513,1.863)--(3.516,1.863)--(3.519,1.863)--(3.522,1.864)--(3.525,1.864)%
  --(3.528,1.864)--(3.531,1.865)--(3.534,1.865)--(3.537,1.865)--(3.540,1.866)--(3.543,1.866)%
  --(3.546,1.866)--(3.549,1.867)--(3.552,1.867)--(3.555,1.868)--(3.558,1.868)--(3.561,1.868)%
  --(3.564,1.869)--(3.567,1.869)--(3.570,1.870)--(3.573,1.870)--(3.576,1.870)--(3.579,1.871)%
  --(3.582,1.871)--(3.585,1.871)--(3.588,1.872)--(3.591,1.872)--(3.594,1.873)--(3.597,1.873)%
  --(3.600,1.874)--(3.603,1.874)--(3.606,1.874)--(3.609,1.875)--(3.612,1.875)--(3.615,1.876)%
  --(3.618,1.876)--(3.621,1.876)--(3.624,1.877)--(3.627,1.877)--(3.630,1.878)--(3.633,1.878)%
  --(3.636,1.879)--(3.639,1.879)--(3.642,1.880)--(3.645,1.880)--(3.648,1.880)--(3.651,1.881)%
  --(3.654,1.881)--(3.657,1.882)--(3.660,1.882)--(3.663,1.883)--(3.666,1.883)--(3.669,1.884)%
  --(3.672,1.884)--(3.675,1.885)--(3.678,1.885)--(3.681,1.886)--(3.684,1.886)--(3.687,1.887)%
  --(3.690,1.887)--(3.693,1.888)--(3.696,1.888)--(3.699,1.889)--(3.702,1.889)--(3.704,1.890)%
  --(3.707,1.890)--(3.710,1.891)--(3.713,1.891)--(3.716,1.892)--(3.719,1.892)--(3.722,1.893)%
  --(3.725,1.893)--(3.728,1.894)--(3.731,1.894)--(3.734,1.895)--(3.737,1.895)--(3.740,1.896)%
  --(3.743,1.896)--(3.746,1.897)--(3.749,1.897)--(3.752,1.898)--(3.755,1.898)--(3.758,1.899)%
  --(3.761,1.900)--(3.764,1.900)--(3.767,1.901)--(3.770,1.901)--(3.773,1.902)--(3.776,1.902)%
  --(3.779,1.903)--(3.782,1.903)--(3.785,1.904)--(3.788,1.905)--(3.791,1.905)--(3.794,1.906)%
  --(3.797,1.906)--(3.800,1.907)--(3.803,1.907)--(3.806,1.908)--(3.809,1.909)--(3.812,1.909)%
  --(3.815,1.910)--(3.818,1.910)--(3.821,1.911)--(3.824,1.911)--(3.827,1.912)--(3.830,1.913)%
  --(3.833,1.913)--(3.836,1.914)--(3.839,1.914)--(3.842,1.915)--(3.845,1.916)--(3.848,1.916)%
  --(3.851,1.917)--(3.854,1.917)--(3.857,1.918)--(3.860,1.919)--(3.863,1.919)--(3.866,1.920)%
  --(3.869,1.921)--(3.872,1.921)--(3.875,1.922)--(3.878,1.922)--(3.881,1.923)--(3.884,1.924)%
  --(3.887,1.924)--(3.890,1.925)--(3.893,1.926)--(3.896,1.926)--(3.899,1.927)--(3.902,1.927)%
  --(3.905,1.928)--(3.908,1.929)--(3.911,1.929)--(3.914,1.930)--(3.916,1.931)--(3.919,1.931)%
  --(3.922,1.932)--(3.925,1.933)--(3.928,1.933)--(3.931,1.934)--(3.934,1.935)--(3.937,1.935)%
  --(3.940,1.936)--(3.943,1.937)--(3.946,1.937)--(3.949,1.938)--(3.952,1.939)--(3.955,1.939)%
  --(3.958,1.940)--(3.961,1.941)--(3.964,1.941)--(3.967,1.942)--(3.970,1.943)--(3.973,1.943)%
  --(3.976,1.944)--(3.979,1.945)--(3.982,1.946)--(3.985,1.946)--(3.988,1.947)--(3.991,1.948)%
  --(3.994,1.948)--(3.997,1.949)--(4.000,1.950)--(4.003,1.950)--(4.006,1.951)--(4.009,1.952)%
  --(4.012,1.953)--(4.015,1.953)--(4.018,1.954)--(4.021,1.955)--(4.024,1.955)--(4.027,1.956)%
  --(4.030,1.957)--(4.033,1.958)--(4.036,1.958)--(4.039,1.959)--(4.042,1.960)--(4.045,1.961)%
  --(4.048,1.961)--(4.051,1.962)--(4.054,1.963)--(4.057,1.963)--(4.060,1.964)--(4.063,1.965)%
  --(4.066,1.966)--(4.069,1.966)--(4.072,1.967)--(4.075,1.968)--(4.078,1.969)--(4.081,1.969)%
  --(4.084,1.970)--(4.087,1.971)--(4.090,1.972)--(4.093,1.972)--(4.096,1.973)--(4.099,1.974)%
  --(4.102,1.975)--(4.105,1.976)--(4.108,1.976)--(4.111,1.977)--(4.114,1.978)--(4.117,1.979)%
  --(4.120,1.979)--(4.123,1.980)--(4.125,1.981)--(4.128,1.982)--(4.131,1.982)--(4.134,1.983)%
  --(4.137,1.984)--(4.140,1.985)--(4.143,1.986)--(4.146,1.986)--(4.149,1.987)--(4.152,1.988)%
  --(4.155,1.989)--(4.158,1.990)--(4.161,1.990)--(4.164,1.991)--(4.167,1.992)--(4.170,1.993)%
  --(4.173,1.994)--(4.176,1.994)--(4.179,1.995)--(4.182,1.996)--(4.185,1.997)--(4.188,1.998)%
  --(4.191,1.998)--(4.194,1.999)--(4.197,2.000)--(4.200,2.001)--(4.203,2.002)--(4.206,2.003)%
  --(4.209,2.003)--(4.212,2.004)--(4.215,2.005)--(4.218,2.006)--(4.221,2.007)--(4.224,2.007)%
  --(4.227,2.008)--(4.230,2.009)--(4.233,2.010)--(4.236,2.011)--(4.239,2.012)--(4.242,2.012)%
  --(4.245,2.013)--(4.248,2.014)--(4.251,2.015)--(4.254,2.016)--(4.257,2.017)--(4.260,2.018)%
  --(4.263,2.018)--(4.266,2.019)--(4.269,2.020)--(4.272,2.021)--(4.275,2.022)--(4.278,2.023)%
  --(4.281,2.024)--(4.284,2.024)--(4.287,2.025)--(4.290,2.026)--(4.293,2.027)--(4.296,2.028)%
  --(4.299,2.029)--(4.302,2.030)--(4.305,2.030)--(4.308,2.031)--(4.311,2.032)--(4.314,2.033)%
  --(4.317,2.034)--(4.320,2.035)--(4.323,2.036)--(4.326,2.037)--(4.329,2.037)--(4.332,2.038)%
  --(4.334,2.039)--(4.337,2.040)--(4.340,2.041)--(4.343,2.042)--(4.346,2.043)--(4.349,2.044)%
  --(4.352,2.045)--(4.355,2.045)--(4.358,2.046)--(4.361,2.047)--(4.364,2.048)--(4.367,2.049)%
  --(4.370,2.050)--(4.373,2.051)--(4.376,2.052)--(4.379,2.053)--(4.382,2.054)--(4.385,2.054)%
  --(4.388,2.055)--(4.391,2.056)--(4.394,2.057)--(4.397,2.058)--(4.400,2.059)--(4.403,2.060)%
  --(4.406,2.061)--(4.409,2.062)--(4.412,2.063)--(4.415,2.064)--(4.418,2.064)--(4.421,2.065)%
  --(4.424,2.066)--(4.427,2.067)--(4.430,2.068)--(4.433,2.069)--(4.436,2.070)--(4.439,2.071)%
  --(4.442,2.072)--(4.445,2.073)--(4.448,2.074)--(4.451,2.075)--(4.454,2.076)--(4.457,2.077)%
  --(4.460,2.077)--(4.463,2.078)--(4.466,2.079)--(4.469,2.080)--(4.472,2.081)--(4.475,2.082)%
  --(4.478,2.083)--(4.481,2.084)--(4.484,2.085)--(4.487,2.086)--(4.490,2.087)--(4.493,2.088)%
  --(4.496,2.089)--(4.499,2.090)--(4.502,2.091)--(4.505,2.092)--(4.508,2.093)--(4.511,2.094)%
  --(4.514,2.095)--(4.517,2.096)--(4.520,2.096)--(4.523,2.097)--(4.526,2.098)--(4.529,2.099)%
  --(4.532,2.100)--(4.535,2.101)--(4.538,2.102)--(4.541,2.103)--(4.543,2.104)--(4.546,2.105)%
  --(4.549,2.106)--(4.552,2.107)--(4.555,2.108)--(4.558,2.109)--(4.561,2.110)--(4.564,2.111)%
  --(4.567,2.112)--(4.570,2.113)--(4.573,2.114)--(4.576,2.115)--(4.579,2.116)--(4.582,2.117)%
  --(4.585,2.118)--(4.588,2.119)--(4.591,2.120)--(4.594,2.121)--(4.597,2.122)--(4.600,2.123)%
  --(4.603,2.124)--(4.606,2.125)--(4.609,2.126)--(4.612,2.127)--(4.615,2.128)--(4.618,2.129)%
  --(4.621,2.130)--(4.624,2.131)--(4.627,2.132)--(4.630,2.133)--(4.633,2.134)--(4.636,2.135)%
  --(4.639,2.136)--(4.642,2.137)--(4.645,2.138)--(4.648,2.139)--(4.651,2.140)--(4.654,2.141)%
  --(4.657,2.142)--(4.660,2.143)--(4.663,2.144)--(4.666,2.145)--(4.669,2.146)--(4.672,2.147)%
  --(4.675,2.148)--(4.678,2.149)--(4.681,2.150)--(4.684,2.151)--(4.687,2.152)--(4.690,2.153)%
  --(4.693,2.154)--(4.696,2.155)--(4.699,2.156)--(4.702,2.157)--(4.705,2.158)--(4.708,2.159)%
  --(4.711,2.160)--(4.714,2.161)--(4.717,2.162)--(4.720,2.163)--(4.723,2.164)--(4.726,2.166)%
  --(4.729,2.167)--(4.732,2.168)--(4.735,2.169)--(4.738,2.170)--(4.741,2.171)--(4.744,2.172)%
  --(4.747,2.173)--(4.750,2.174)--(4.752,2.175)--(4.755,2.176)--(4.758,2.177)--(4.761,2.178)%
  --(4.764,2.179)--(4.767,2.180)--(4.770,2.181)--(4.773,2.182)--(4.776,2.183)--(4.779,2.184)%
  --(4.782,2.185)--(4.785,2.186)--(4.788,2.188)--(4.791,2.189)--(4.794,2.190)--(4.797,2.191)%
  --(4.800,2.192)--(4.803,2.193)--(4.806,2.194)--(4.809,2.195)--(4.812,2.196)--(4.815,2.197)%
  --(4.818,2.198)--(4.821,2.199)--(4.824,2.200)--(4.827,2.201)--(4.830,2.202)--(4.833,2.203)%
  --(4.836,2.205)--(4.839,2.206)--(4.842,2.207)--(4.845,2.208)--(4.848,2.209)--(4.851,2.210)%
  --(4.854,2.211)--(4.857,2.212)--(4.860,2.213)--(4.863,2.214)--(4.866,2.215)--(4.869,2.216)%
  --(4.872,2.217)--(4.875,2.219)--(4.878,2.220)--(4.881,2.221)--(4.884,2.222)--(4.887,2.223)%
  --(4.890,2.224)--(4.893,2.225)--(4.896,2.226)--(4.899,2.227)--(4.902,2.228)--(4.905,2.229)%
  --(4.908,2.231)--(4.911,2.232)--(4.914,2.233)--(4.917,2.234)--(4.920,2.235)--(4.923,2.236)%
  --(4.926,2.237)--(4.929,2.238)--(4.932,2.239)--(4.935,2.240)--(4.938,2.242)--(4.941,2.243)%
  --(4.944,2.244)--(4.947,2.245)--(4.950,2.246)--(4.953,2.247)--(4.956,2.248)--(4.959,2.249)%
  --(4.961,2.250)--(4.964,2.251)--(4.967,2.253)--(4.970,2.254)--(4.973,2.255)--(4.976,2.256)%
  --(4.979,2.257)--(4.982,2.258)--(4.985,2.259)--(4.988,2.260)--(4.991,2.261)--(4.994,2.263)%
  --(4.997,2.264)--(5.000,2.265)--(5.003,2.266)--(5.006,2.267)--(5.009,2.268)--(5.012,2.269)%
  --(5.015,2.270)--(5.018,2.272)--(5.021,2.273)--(5.024,2.274)--(5.027,2.275)--(5.030,2.276)%
  --(5.033,2.277)--(5.036,2.278)--(5.039,2.279)--(5.042,2.281)--(5.045,2.282)--(5.048,2.283)%
  --(5.051,2.284)--(5.054,2.285)--(5.057,2.286)--(5.060,2.287)--(5.063,2.289)--(5.066,2.290)%
  --(5.069,2.291)--(5.072,2.292)--(5.075,2.293)--(5.078,2.294)--(5.081,2.295)--(5.084,2.296)%
  --(5.087,2.298)--(5.090,2.299)--(5.093,2.300)--(5.096,2.301)--(5.099,2.302)--(5.102,2.303)%
  --(5.105,2.304)--(5.108,2.306)--(5.111,2.307)--(5.114,2.308)--(5.117,2.309)--(5.120,2.310)%
  --(5.123,2.311)--(5.126,2.313)--(5.129,2.314)--(5.132,2.315)--(5.135,2.316)--(5.138,2.317)%
  --(5.141,2.318)--(5.144,2.319)--(5.147,2.321)--(5.150,2.322)--(5.153,2.323)--(5.156,2.324)%
  --(5.159,2.325)--(5.162,2.326)--(5.165,2.328)--(5.168,2.329)--(5.171,2.330)--(5.173,2.331)%
  --(5.176,2.332)--(5.179,2.333)--(5.182,2.335)--(5.185,2.336)--(5.188,2.337)--(5.191,2.338)%
  --(5.194,2.339)--(5.197,2.340)--(5.200,2.342)--(5.203,2.343)--(5.206,2.344)--(5.209,2.345)%
  --(5.212,2.346)--(5.215,2.347)--(5.218,2.349)--(5.221,2.350)--(5.224,2.351)--(5.227,2.352)%
  --(5.230,2.353)--(5.233,2.354)--(5.236,2.356)--(5.239,2.357)--(5.242,2.358)--(5.245,2.359)%
  --(5.248,2.360)--(5.251,2.361)--(5.254,2.363)--(5.257,2.364)--(5.260,2.365)--(5.263,2.366)%
  --(5.266,2.367)--(5.269,2.369)--(5.272,2.370)--(5.275,2.371)--(5.278,2.372)--(5.281,2.373)%
  --(5.284,2.374)--(5.287,2.376)--(5.290,2.377)--(5.293,2.378)--(5.296,2.379)--(5.299,2.380)%
  --(5.302,2.382)--(5.305,2.383)--(5.308,2.384)--(5.311,2.385)--(5.314,2.386)--(5.317,2.388)%
  --(5.320,2.389)--(5.323,2.390)--(5.326,2.391)--(5.329,2.392)--(5.332,2.394)--(5.335,2.395)%
  --(5.338,2.396)--(5.341,2.397)--(5.344,2.398)--(5.347,2.400)--(5.350,2.401)--(5.353,2.402)%
  --(5.356,2.403)--(5.359,2.404)--(5.362,2.406)--(5.365,2.407)--(5.368,2.408)--(5.371,2.409)%
  --(5.374,2.410)--(5.377,2.412)--(5.380,2.413)--(5.382,2.414)--(5.385,2.415)--(5.388,2.416)%
  --(5.391,2.418)--(5.394,2.419)--(5.397,2.420)--(5.400,2.421)--(5.403,2.422)--(5.406,2.424)%
  --(5.409,2.425)--(5.412,2.426)--(5.415,2.427)--(5.418,2.428)--(5.421,2.430)--(5.424,2.431)%
  --(5.427,2.432)--(5.430,2.433)--(5.433,2.435)--(5.436,2.436)--(5.439,2.437)--(5.442,2.438)%
  --(5.445,2.439)--(5.448,2.441)--(5.451,2.442)--(5.454,2.443)--(5.457,2.444)--(5.460,2.446)%
  --(5.463,2.447)--(5.466,2.448)--(5.469,2.449)--(5.472,2.450)--(5.475,2.452)--(5.478,2.453)%
  --(5.481,2.454)--(5.484,2.455)--(5.487,2.457)--(5.490,2.458)--(5.493,2.459)--(5.496,2.460)%
  --(5.499,2.461)--(5.502,2.463)--(5.505,2.464)--(5.508,2.465)--(5.511,2.466)--(5.514,2.468)%
  --(5.517,2.469)--(5.520,2.470)--(5.523,2.471)--(5.526,2.473)--(5.529,2.474)--(5.532,2.475)%
  --(5.535,2.476)--(5.538,2.477)--(5.541,2.479)--(5.544,2.480)--(5.547,2.481)--(5.550,2.482)%
  --(5.553,2.484)--(5.556,2.485)--(5.559,2.486)--(5.562,2.487)--(5.565,2.489)--(5.568,2.490)%
  --(5.571,2.491)--(5.574,2.492)--(5.577,2.494)--(5.580,2.495)--(5.583,2.496)--(5.586,2.497)%
  --(5.589,2.499)--(5.591,2.500)--(5.594,2.501)--(5.597,2.502)--(5.600,2.504)--(5.603,2.505)%
  --(5.606,2.506)--(5.609,2.507)--(5.612,2.509)--(5.615,2.510)--(5.618,2.511)--(5.621,2.512)%
  --(5.624,2.514)--(5.627,2.515)--(5.630,2.516)--(5.633,2.517)--(5.636,2.519)--(5.639,2.520)%
  --(5.642,2.521)--(5.645,2.522)--(5.648,2.524)--(5.651,2.525)--(5.654,2.526)--(5.657,2.527)%
  --(5.660,2.529)--(5.663,2.530)--(5.666,2.531)--(5.669,2.532)--(5.672,2.534)--(5.675,2.535)%
  --(5.678,2.536)--(5.681,2.537)--(5.684,2.539)--(5.687,2.540)--(5.690,2.541)--(5.693,2.542)%
  --(5.696,2.544)--(5.699,2.545)--(5.702,2.546)--(5.705,2.547)--(5.708,2.549)--(5.711,2.550)%
  --(5.714,2.551)--(5.717,2.553)--(5.720,2.554)--(5.723,2.555)--(5.726,2.556)--(5.729,2.558)%
  --(5.732,2.559)--(5.735,2.560)--(5.738,2.561)--(5.741,2.563)--(5.744,2.564)--(5.747,2.565)%
  --(5.750,2.566)--(5.753,2.568)--(5.756,2.569)--(5.759,2.570)--(5.762,2.572)--(5.765,2.573)%
  --(5.768,2.574)--(5.771,2.575)--(5.774,2.577)--(5.777,2.578)--(5.780,2.579)--(5.783,2.580)%
  --(5.786,2.582)--(5.789,2.583)--(5.792,2.584)--(5.795,2.586)--(5.798,2.587)--(5.800,2.588)%
  --(5.803,2.589)--(5.806,2.591)--(5.809,2.592)--(5.812,2.593)--(5.815,2.594)--(5.818,2.596)%
  --(5.821,2.597)--(5.824,2.598)--(5.827,2.600)--(5.830,2.601)--(5.833,2.602)--(5.836,2.603)%
  --(5.839,2.605)--(5.842,2.606)--(5.845,2.607)--(5.848,2.609)--(5.851,2.610)--(5.854,2.611)%
  --(5.857,2.612)--(5.860,2.614)--(5.863,2.615)--(5.866,2.616)--(5.869,2.618)--(5.872,2.619)%
  --(5.875,2.620)--(5.878,2.621)--(5.881,2.623)--(5.884,2.624)--(5.887,2.625)--(5.890,2.627)%
  --(5.893,2.628)--(5.896,2.629)--(5.899,2.630)--(5.902,2.632)--(5.905,2.633)--(5.908,2.634)%
  --(5.911,2.636)--(5.914,2.637)--(5.917,2.638)--(5.920,2.640)--(5.923,2.641)--(5.926,2.642)%
  --(5.929,2.643)--(5.932,2.645)--(5.935,2.646)--(5.938,2.647)--(5.941,2.649)--(5.944,2.650)%
  --(5.947,2.651)--(5.950,2.652)--(5.953,2.654)--(5.956,2.655)--(5.959,2.656)--(5.962,2.658)%
  --(5.965,2.659)--(5.968,2.660)--(5.971,2.662)--(5.974,2.663)--(5.977,2.664)--(5.980,2.665)%
  --(5.983,2.667)--(5.986,2.668)--(5.989,2.669)--(5.992,2.671)--(5.995,2.672)--(5.998,2.673)%
  --(6.001,2.675)--(6.004,2.676)--(6.007,2.677)--(6.009,2.679)--(6.012,2.680)--(6.015,2.681)%
  --(6.018,2.682)--(6.021,2.684)--(6.024,2.685)--(6.027,2.686)--(6.030,2.688)--(6.033,2.689)%
  --(6.036,2.690)--(6.039,2.692)--(6.042,2.693)--(6.045,2.694)--(6.048,2.696)--(6.051,2.697)%
  --(6.054,2.698)--(6.057,2.699)--(6.060,2.701)--(6.063,2.702)--(6.066,2.703)--(6.069,2.705)%
  --(6.072,2.706)--(6.075,2.707)--(6.078,2.709)--(6.081,2.710)--(6.084,2.711)--(6.087,2.713)%
  --(6.090,2.714)--(6.093,2.715)--(6.096,2.717)--(6.099,2.718)--(6.102,2.719)--(6.105,2.720)%
  --(6.108,2.722)--(6.111,2.723)--(6.114,2.724)--(6.117,2.726)--(6.120,2.727)--(6.123,2.728)%
  --(6.126,2.730)--(6.129,2.731)--(6.132,2.732)--(6.135,2.734)--(6.138,2.735)--(6.141,2.736)%
  --(6.144,2.738)--(6.147,2.739)--(6.150,2.740)--(6.153,2.742)--(6.156,2.743)--(6.159,2.744)%
  --(6.162,2.746)--(6.165,2.747)--(6.168,2.748)--(6.171,2.750)--(6.174,2.751)--(6.177,2.752)%
  --(6.180,2.754)--(6.183,2.755)--(6.186,2.756)--(6.189,2.758)--(6.192,2.759)--(6.195,2.760)%
  --(6.198,2.761)--(6.201,2.763)--(6.204,2.764)--(6.207,2.765)--(6.210,2.767)--(6.213,2.768)%
  --(6.216,2.769)--(6.218,2.771)--(6.221,2.772)--(6.224,2.773)--(6.227,2.775)--(6.230,2.776)%
  --(6.233,2.777)--(6.236,2.779)--(6.239,2.780)--(6.242,2.781)--(6.245,2.783)--(6.248,2.784)%
  --(6.251,2.785)--(6.254,2.787)--(6.257,2.788)--(6.260,2.789)--(6.263,2.791)--(6.266,2.792)%
  --(6.269,2.793)--(6.272,2.795)--(6.275,2.796)--(6.278,2.797)--(6.281,2.799)--(6.284,2.800)%
  --(6.287,2.801)--(6.290,2.803)--(6.293,2.804)--(6.296,2.805)--(6.299,2.807)--(6.302,2.808)%
  --(6.305,2.809)--(6.308,2.811)--(6.311,2.812)--(6.314,2.813)--(6.317,2.815)--(6.320,2.816)%
  --(6.323,2.818)--(6.326,2.819)--(6.329,2.820)--(6.332,2.822)--(6.335,2.823)--(6.338,2.824)%
  --(6.341,2.826)--(6.344,2.827)--(6.347,2.828)--(6.350,2.830)--(6.353,2.831)--(6.356,2.832)%
  --(6.359,2.834)--(6.362,2.835)--(6.365,2.836)--(6.368,2.838)--(6.371,2.839)--(6.374,2.840)%
  --(6.377,2.842)--(6.380,2.843)--(6.383,2.844)--(6.386,2.846)--(6.389,2.847)--(6.392,2.848)%
  --(6.395,2.850)--(6.398,2.851)--(6.401,2.852)--(6.404,2.854)--(6.407,2.855)--(6.410,2.857)%
  --(6.413,2.858)--(6.416,2.859)--(6.419,2.861)--(6.422,2.862)--(6.425,2.863)--(6.428,2.865)%
  --(6.430,2.866)--(6.433,2.867)--(6.436,2.869)--(6.439,2.870)--(6.442,2.871)--(6.445,2.873)%
  --(6.448,2.874)--(6.451,2.875)--(6.454,2.877)--(6.457,2.878)--(6.460,2.879)--(6.463,2.881)%
  --(6.466,2.882)--(6.469,2.884)--(6.472,2.885)--(6.475,2.886)--(6.478,2.888)--(6.481,2.889)%
  --(6.484,2.890)--(6.487,2.892)--(6.490,2.893)--(6.493,2.894)--(6.496,2.896)--(6.499,2.897)%
  --(6.502,2.898)--(6.505,2.900)--(6.508,2.901)--(6.511,2.903)--(6.514,2.904)--(6.517,2.905)%
  --(6.520,2.907)--(6.523,2.908)--(6.526,2.909)--(6.529,2.911)--(6.532,2.912)--(6.535,2.913)%
  --(6.538,2.915)--(6.541,2.916)--(6.544,2.917)--(6.547,2.919)--(6.550,2.920)--(6.553,2.922)%
  --(6.556,2.923)--(6.559,2.924)--(6.562,2.926)--(6.565,2.927)--(6.568,2.928)--(6.571,2.930)%
  --(6.574,2.931)--(6.577,2.932)--(6.580,2.934)--(6.583,2.935)--(6.586,2.937)--(6.589,2.938)%
  --(6.592,2.939)--(6.595,2.941)--(6.598,2.942)--(6.601,2.943)--(6.604,2.945)--(6.607,2.946)%
  --(6.610,2.948)--(6.613,2.949)--(6.616,2.950)--(6.619,2.952)--(6.622,2.953)--(6.625,2.954)%
  --(6.628,2.956)--(6.631,2.957)--(6.634,2.958)--(6.637,2.960)--(6.639,2.961)--(6.642,2.963)%
  --(6.645,2.964)--(6.648,2.965)--(6.651,2.967)--(6.654,2.968)--(6.657,2.969)--(6.660,2.971)%
  --(6.663,2.972)--(6.666,2.974)--(6.669,2.975)--(6.672,2.976)--(6.675,2.978)--(6.678,2.979)%
  --(6.681,2.980)--(6.684,2.982)--(6.687,2.983)--(6.690,2.984)--(6.693,2.986)--(6.696,2.987)%
  --(6.699,2.989)--(6.702,2.990)--(6.705,2.991)--(6.708,2.993)--(6.711,2.994)--(6.714,2.995)%
  --(6.717,2.997)--(6.720,2.998)--(6.723,3.000)--(6.726,3.001)--(6.729,3.002)--(6.732,3.004)%
  --(6.735,3.005)--(6.738,3.006)--(6.741,3.008)--(6.744,3.009)--(6.747,3.011)--(6.750,3.012)%
  --(6.753,3.013)--(6.756,3.015)--(6.759,3.016)--(6.762,3.018)--(6.765,3.019)--(6.768,3.020)%
  --(6.771,3.022)--(6.774,3.023)--(6.777,3.024)--(6.780,3.026)--(6.783,3.027)--(6.786,3.029)%
  --(6.789,3.030)--(6.792,3.031)--(6.795,3.033)--(6.798,3.034)--(6.801,3.035)--(6.804,3.037)%
  --(6.807,3.038)--(6.810,3.040)--(6.813,3.041)--(6.816,3.042)--(6.819,3.044)--(6.822,3.045)%
  --(6.825,3.047)--(6.828,3.048)--(6.831,3.049)--(6.834,3.051)--(6.837,3.052)--(6.840,3.053)%
  --(6.843,3.055)--(6.846,3.056)--(6.848,3.058)--(6.851,3.059)--(6.854,3.060)--(6.857,3.062)%
  --(6.860,3.063)--(6.863,3.065)--(6.866,3.066)--(6.869,3.067)--(6.872,3.069)--(6.875,3.070)%
  --(6.878,3.071)--(6.881,3.073)--(6.884,3.074)--(6.887,3.076)--(6.890,3.077)--(6.893,3.078)%
  --(6.896,3.080)--(6.899,3.081)--(6.902,3.083)--(6.905,3.084)--(6.908,3.085)--(6.911,3.087)%
  --(6.914,3.088)--(6.917,3.089)--(6.920,3.091)--(6.923,3.092)--(6.926,3.094)--(6.929,3.095)%
  --(6.932,3.096)--(6.935,3.098)--(6.938,3.099)--(6.941,3.101)--(6.944,3.102)--(6.947,3.103)%
  --(6.950,3.105)--(6.953,3.106)--(6.956,3.108)--(6.959,3.109)--(6.962,3.110)--(6.965,3.112)%
  --(6.968,3.113)--(6.971,3.115)--(6.974,3.116)--(6.977,3.117)--(6.980,3.119)--(6.983,3.120)%
  --(6.986,3.122)--(6.989,3.123)--(6.992,3.124)--(6.995,3.126)--(6.998,3.127)--(7.001,3.128)%
  --(7.004,3.130)--(7.007,3.131)--(7.010,3.133)--(7.013,3.134)--(7.016,3.135)--(7.019,3.137)%
  --(7.022,3.138)--(7.025,3.140)--(7.028,3.141)--(7.031,3.142)--(7.034,3.144)--(7.037,3.145)%
  --(7.040,3.147)--(7.043,3.148)--(7.046,3.149)--(7.049,3.151)--(7.052,3.152)--(7.055,3.154)%
  --(7.057,3.155)--(7.060,3.156)--(7.063,3.158)--(7.066,3.159)--(7.069,3.161)--(7.072,3.162)%
  --(7.075,3.163)--(7.078,3.165)--(7.081,3.166)--(7.084,3.168)--(7.087,3.169)--(7.090,3.170)%
  --(7.093,3.172)--(7.096,3.173)--(7.099,3.175)--(7.102,3.176)--(7.105,3.177)--(7.108,3.179)%
  --(7.111,3.180)--(7.114,3.182)--(7.117,3.183)--(7.120,3.184)--(7.123,3.186)--(7.126,3.187)%
  --(7.129,3.189)--(7.132,3.190)--(7.135,3.191)--(7.138,3.193)--(7.141,3.194)--(7.144,3.196)%
  --(7.147,3.197)--(7.150,3.198)--(7.153,3.200)--(7.156,3.201)--(7.159,3.203)--(7.162,3.204)%
  --(7.165,3.206)--(7.168,3.207)--(7.171,3.208)--(7.174,3.210)--(7.177,3.211)--(7.180,3.213)%
  --(7.183,3.214)--(7.186,3.215)--(7.189,3.217)--(7.192,3.218)--(7.195,3.220)--(7.198,3.221)%
  --(7.201,3.222)--(7.204,3.224)--(7.207,3.225)--(7.210,3.227)--(7.213,3.228)--(7.216,3.229)%
  --(7.219,3.231)--(7.222,3.232)--(7.225,3.234)--(7.228,3.235)--(7.231,3.236)--(7.234,3.238)%
  --(7.237,3.239)--(7.240,3.241)--(7.243,3.242)--(7.246,3.244)--(7.249,3.245)--(7.252,3.246)%
  --(7.255,3.248)--(7.258,3.249)--(7.261,3.251)--(7.264,3.252)--(7.266,3.253)--(7.269,3.255)%
  --(7.272,3.256)--(7.275,3.258)--(7.278,3.259)--(7.281,3.260)--(7.284,3.262)--(7.287,3.263)%
  --(7.290,3.265)--(7.293,3.266)--(7.296,3.268)--(7.299,3.269)--(7.302,3.270)--(7.305,3.272)%
  --(7.308,3.273)--(7.311,3.275)--(7.314,3.276)--(7.317,3.277)--(7.320,3.279)--(7.323,3.280)%
  --(7.326,3.282)--(7.329,3.283)--(7.332,3.284)--(7.335,3.286)--(7.338,3.287)--(7.341,3.289)%
  --(7.344,3.290)--(7.347,3.292)--(7.350,3.293)--(7.353,3.294)--(7.356,3.296)--(7.359,3.297)%
  --(7.362,3.299)--(7.365,3.300)--(7.368,3.301)--(7.371,3.303)--(7.374,3.304)--(7.377,3.306)%
  --(7.380,3.307)--(7.383,3.309)--(7.386,3.310)--(7.389,3.311)--(7.392,3.313)--(7.395,3.314)%
  --(7.398,3.316)--(7.401,3.317)--(7.404,3.318)--(7.407,3.320)--(7.410,3.321)--(7.413,3.323)%
  --(7.416,3.324)--(7.419,3.326)--(7.422,3.327)--(7.425,3.328)--(7.428,3.330)--(7.431,3.331)%
  --(7.434,3.333)--(7.437,3.334)--(7.440,3.336)--(7.443,3.337)--(7.446,3.338)--(7.449,3.340)%
  --(7.452,3.341)--(7.455,3.343)--(7.458,3.344)--(7.461,3.345)--(7.464,3.347)--(7.467,3.348)%
  --(7.470,3.350)--(7.473,3.351)--(7.476,3.353)--(7.478,3.354)--(7.481,3.355)--(7.484,3.357)%
  --(7.487,3.358)--(7.490,3.360)--(7.493,3.361)--(7.496,3.363)--(7.499,3.364)--(7.502,3.365)%
  --(7.505,3.367)--(7.508,3.368)--(7.511,3.370)--(7.514,3.371)--(7.517,3.372)--(7.520,3.374)%
  --(7.523,3.375)--(7.526,3.377)--(7.529,3.378)--(7.532,3.380)--(7.535,3.381)--(7.538,3.382)%
  --(7.541,3.384)--(7.544,3.385)--(7.547,3.387)--(7.550,3.388)--(7.553,3.390)--(7.556,3.391)%
  --(7.559,3.392)--(7.562,3.394)--(7.565,3.395)--(7.568,3.397)--(7.571,3.398)--(7.574,3.400)%
  --(7.577,3.401)--(7.580,3.402)--(7.583,3.404)--(7.586,3.405)--(7.589,3.407)--(7.592,3.408)%
  --(7.595,3.410)--(7.598,3.411)--(7.601,3.412)--(7.604,3.414)--(7.607,3.415)--(7.610,3.417)%
  --(7.613,3.418)--(7.616,3.420)--(7.619,3.421)--(7.622,3.422)--(7.625,3.424)--(7.628,3.425)%
  --(7.631,3.427)--(7.634,3.428)--(7.637,3.430)--(7.640,3.431)--(7.643,3.432)--(7.646,3.434)%
  --(7.649,3.435)--(7.652,3.437)--(7.655,3.438)--(7.658,3.440)--(7.661,3.441)--(7.664,3.442)%
  --(7.667,3.444)--(7.670,3.445)--(7.673,3.447)--(7.676,3.448)--(7.679,3.450)--(7.682,3.451)%
  --(7.685,3.452)--(7.687,3.454)--(7.690,3.455)--(7.693,3.457)--(7.696,3.458)--(7.699,3.460)%
  --(7.702,3.461)--(7.705,3.462)--(7.708,3.464)--(7.711,3.465)--(7.714,3.467)--(7.717,3.468)%
  --(7.720,3.470)--(7.723,3.471)--(7.726,3.472)--(7.729,3.474)--(7.732,3.475)--(7.735,3.477)%
  --(7.738,3.478)--(7.741,3.480)--(7.744,3.481)--(7.747,3.483)--(7.750,3.484)--(7.753,3.485)%
  --(7.756,3.487)--(7.759,3.488)--(7.762,3.490)--(7.765,3.491)--(7.768,3.493)--(7.771,3.494)%
  --(7.774,3.495)--(7.777,3.497)--(7.780,3.498)--(7.783,3.500)--(7.786,3.501)--(7.789,3.503)%
  --(7.792,3.504)--(7.795,3.505)--(7.798,3.507)--(7.801,3.508)--(7.804,3.510)--(7.807,3.511)%
  --(7.810,3.513)--(7.813,3.514)--(7.816,3.516)--(7.819,3.517)--(7.822,3.518)--(7.825,3.520)%
  --(7.828,3.521)--(7.831,3.523)--(7.834,3.524)--(7.837,3.526)--(7.840,3.527)--(7.843,3.528)%
  --(7.846,3.530)--(7.849,3.531)--(7.852,3.533)--(7.855,3.534)--(7.858,3.536)--(7.861,3.537)%
  --(7.864,3.539)--(7.867,3.540)--(7.870,3.541)--(7.873,3.543)--(7.876,3.544)--(7.879,3.546)%
  --(7.882,3.547)--(7.885,3.549)--(7.888,3.550)--(7.891,3.551)--(7.894,3.553)--(7.896,3.554)%
  --(7.899,3.556)--(7.902,3.557)--(7.905,3.559)--(7.908,3.560)--(7.911,3.562)--(7.914,3.563)%
  --(7.917,3.564)--(7.920,3.566)--(7.923,3.567)--(7.926,3.569)--(7.929,3.570)--(7.932,3.572)%
  --(7.935,3.573)--(7.938,3.575)--(7.941,3.576)--(7.944,3.577)--(7.947,3.579)--(7.950,3.580)%
  --(7.953,3.582)--(7.956,3.583)--(7.959,3.585)--(7.962,3.586)--(7.965,3.588)--(7.968,3.589)%
  --(7.971,3.590)--(7.974,3.592)--(7.977,3.593)--(7.980,3.595)--(7.983,3.596)--(7.986,3.598)%
  --(7.989,3.599)--(7.992,3.601)--(7.995,3.602)--(7.998,3.603)--(8.001,3.605)--(8.004,3.606)%
  --(8.007,3.608)--(8.010,3.609)--(8.013,3.611)--(8.016,3.612)--(8.019,3.614)--(8.022,3.615)%
  --(8.025,3.616)--(8.028,3.618)--(8.031,3.619)--(8.034,3.621)--(8.037,3.622)--(8.040,3.624)%
  --(8.043,3.625)--(8.046,3.627)--(8.049,3.628)--(8.052,3.629)--(8.055,3.631)--(8.058,3.632)%
  --(8.061,3.634)--(8.064,3.635)--(8.067,3.637)--(8.070,3.638)--(8.073,3.640)--(8.076,3.641)%
  --(8.079,3.642)--(8.082,3.644)--(8.085,3.645)--(8.088,3.647)--(8.091,3.648)--(8.094,3.650)%
  --(8.097,3.651)--(8.100,3.653)--(8.103,3.654)--(8.105,3.655)--(8.108,3.657)--(8.111,3.658)%
  --(8.114,3.660)--(8.117,3.661)--(8.120,3.663)--(8.123,3.664)--(8.126,3.666)--(8.129,3.667)%
  --(8.132,3.668)--(8.135,3.670)--(8.138,3.671)--(8.141,3.673)--(8.144,3.674)--(8.147,3.676)%
  --(8.150,3.677)--(8.153,3.679)--(8.156,3.680)--(8.159,3.682)--(8.162,3.683)--(8.165,3.684)%
  --(8.168,3.686)--(8.171,3.687)--(8.174,3.689)--(8.177,3.690)--(8.180,3.692)--(8.183,3.693)%
  --(8.186,3.695)--(8.189,3.696)--(8.192,3.697)--(8.195,3.699)--(8.198,3.700)--(8.201,3.702)%
  --(8.204,3.703)--(8.207,3.705)--(8.210,3.706)--(8.213,3.708)--(8.216,3.709)--(8.219,3.711)%
  --(8.222,3.712)--(8.225,3.713)--(8.228,3.715)--(8.231,3.716)--(8.234,3.718)--(8.237,3.719)%
  --(8.240,3.721)--(8.243,3.722)--(8.246,3.724)--(8.249,3.725)--(8.252,3.726)--(8.255,3.728)%
  --(8.258,3.729)--(8.261,3.731)--(8.264,3.732)--(8.267,3.734)--(8.270,3.735)--(8.273,3.737)%
  --(8.276,3.738)--(8.279,3.740)--(8.282,3.741)--(8.285,3.742)--(8.288,3.744)--(8.291,3.745)%
  --(8.294,3.747)--(8.297,3.748)--(8.300,3.750)--(8.303,3.751)--(8.306,3.753)--(8.309,3.754)%
  --(8.312,3.756)--(8.314,3.757)--(8.317,3.758)--(8.320,3.760)--(8.323,3.761)--(8.326,3.763)%
  --(8.329,3.764)--(8.332,3.766)--(8.335,3.767)--(8.338,3.769)--(8.341,3.770)--(8.344,3.772)%
  --(8.347,3.773)--(8.350,3.774)--(8.353,3.776)--(8.356,3.777)--(8.359,3.779)--(8.362,3.780)%
  --(8.365,3.782)--(8.368,3.783)--(8.371,3.785)--(8.374,3.786)--(8.377,3.788)--(8.380,3.789)%
  --(8.383,3.790)--(8.386,3.792)--(8.389,3.793)--(8.392,3.795)--(8.395,3.796)--(8.398,3.798)%
  --(8.401,3.799)--(8.404,3.801)--(8.407,3.802)--(8.410,3.804)--(8.413,3.805)--(8.416,3.806)%
  --(8.419,3.808)--(8.422,3.809)--(8.425,3.811)--(8.428,3.812)--(8.431,3.814)--(8.434,3.815)%
  --(8.437,3.817)--(8.440,3.818)--(8.443,3.820)--(8.446,3.821)--(8.449,3.823)--(8.452,3.824)%
  --(8.455,3.825)--(8.458,3.827)--(8.461,3.828)--(8.464,3.830)--(8.467,3.831)--(8.470,3.833)%
  --(8.473,3.834)--(8.476,3.836)--(8.479,3.837)--(8.482,3.839)--(8.485,3.840)--(8.488,3.841)%
  --(8.491,3.843)--(8.494,3.844)--(8.497,3.846)--(8.500,3.847)--(8.503,3.849)--(8.506,3.850)%
  --(8.509,3.852)--(8.512,3.853)--(8.515,3.855)--(8.518,3.856)--(8.521,3.858)--(8.523,3.859)%
  --(8.526,3.860)--(8.529,3.862)--(8.532,3.863)--(8.535,3.865)--(8.538,3.866)--(8.541,3.868)%
  --(8.544,3.869)--(8.547,3.871)--(8.550,3.872)--(8.553,3.874)--(8.556,3.875)--(8.559,3.877)%
  --(8.562,3.878)--(8.565,3.879)--(8.568,3.881)--(8.571,3.882)--(8.574,3.884)--(8.577,3.885)%
  --(8.580,3.887)--(8.583,3.888)--(8.586,3.890)--(8.589,3.891)--(8.592,3.893)--(8.595,3.894)%
  --(8.598,3.896)--(8.601,3.897)--(8.604,3.898)--(8.607,3.900)--(8.610,3.901)--(8.613,3.903)%
  --(8.616,3.904)--(8.619,3.906)--(8.622,3.907)--(8.625,3.909)--(8.628,3.910)--(8.631,3.912)%
  --(8.634,3.913)--(8.637,3.915)--(8.640,3.916)--(8.643,3.917)--(8.646,3.919)--(8.649,3.920)%
  --(8.652,3.922)--(8.655,3.923)--(8.658,3.925)--(8.661,3.926)--(8.664,3.928)--(8.667,3.929)%
  --(8.670,3.931)--(8.673,3.932)--(8.676,3.934)--(8.679,3.935)--(8.682,3.937)--(8.685,3.938)%
  --(8.688,3.939)--(8.691,3.941)--(8.694,3.942)--(8.697,3.944)--(8.700,3.945)--(8.703,3.947)%
  --(8.706,3.948)--(8.709,3.950)--(8.712,3.951)--(8.715,3.953)--(8.718,3.954)--(8.721,3.956)%
  --(8.724,3.957)--(8.727,3.958)--(8.730,3.960)--(8.733,3.961)--(8.735,3.963)--(8.738,3.964)%
  --(8.741,3.966)--(8.744,3.967)--(8.747,3.969)--(8.750,3.970)--(8.753,3.972)--(8.756,3.973)%
  --(8.759,3.975)--(8.762,3.976)--(8.765,3.978)--(8.768,3.979)--(8.771,3.980)--(8.774,3.982)%
  --(8.777,3.983)--(8.780,3.985)--(8.783,3.986)--(8.786,3.988)--(8.789,3.989)--(8.792,3.991)%
  --(8.795,3.992)--(8.798,3.994)--(8.801,3.995)--(8.804,3.997)--(8.807,3.998)--(8.810,4.000)%
  --(8.813,4.001)--(8.816,4.002)--(8.819,4.004)--(8.822,4.005)--(8.825,4.007)--(8.828,4.008)%
  --(8.831,4.010)--(8.834,4.011)--(8.837,4.013)--(8.840,4.014)--(8.843,4.016)--(8.846,4.017)%
  --(8.849,4.019)--(8.852,4.020)--(8.855,4.022)--(8.858,4.023)--(8.861,4.025)--(8.864,4.026)%
  --(8.867,4.027)--(8.870,4.029)--(8.873,4.030)--(8.876,4.032)--(8.879,4.033)--(8.882,4.035)%
  --(8.885,4.036)--(8.888,4.038)--(8.891,4.039)--(8.894,4.041)--(8.897,4.042)--(8.900,4.044)%
  --(8.903,4.045)--(8.906,4.047)--(8.909,4.048)--(8.912,4.049)--(8.915,4.051)--(8.918,4.052)%
  --(8.921,4.054)--(8.924,4.055)--(8.927,4.057)--(8.930,4.058)--(8.933,4.060)--(8.936,4.061)%
  --(8.939,4.063)--(8.942,4.064)--(8.944,4.066)--(8.947,4.067)--(8.950,4.069)--(8.953,4.070)%
  --(8.956,4.072)--(8.959,4.073)--(8.962,4.074)--(8.965,4.076)--(8.968,4.077)--(8.971,4.079)%
  --(8.974,4.080)--(8.977,4.082)--(8.980,4.083)--(8.983,4.085)--(8.986,4.086)--(8.989,4.088)%
  --(8.992,4.089)--(8.995,4.091)--(8.998,4.092)--(9.001,4.094)--(9.004,4.095)--(9.007,4.097)%
  --(9.010,4.098)--(9.013,4.100)--(9.016,4.101)--(9.019,4.102)--(9.022,4.104)--(9.025,4.105)%
  --(9.028,4.107)--(9.031,4.108)--(9.034,4.110)--(9.037,4.111)--(9.040,4.113)--(9.043,4.114)%
  --(9.046,4.116)--(9.049,4.117)--(9.052,4.119)--(9.055,4.120)--(9.058,4.122)--(9.061,4.123)%
  --(9.064,4.125)--(9.067,4.126)--(9.070,4.127)--(9.073,4.129)--(9.076,4.130)--(9.079,4.132)%
  --(9.082,4.133)--(9.085,4.135)--(9.088,4.136)--(9.091,4.138)--(9.094,4.139)--(9.097,4.141)%
  --(9.100,4.142)--(9.103,4.144)--(9.106,4.145)--(9.109,4.147)--(9.112,4.148)--(9.115,4.150)%
  --(9.118,4.151)--(9.121,4.153)--(9.124,4.154)--(9.127,4.155)--(9.130,4.157)--(9.133,4.158)%
  --(9.136,4.160)--(9.139,4.161)--(9.142,4.163)--(9.145,4.164)--(9.148,4.166)--(9.151,4.167)%
  --(9.153,4.169)--(9.156,4.170)--(9.159,4.172)--(9.162,4.173)--(9.165,4.175)--(9.168,4.176)%
  --(9.171,4.178)--(9.174,4.179)--(9.177,4.181)--(9.180,4.182)--(9.183,4.184)--(9.186,4.185)%
  --(9.189,4.186)--(9.192,4.188)--(9.195,4.189)--(9.198,4.191)--(9.201,4.192)--(9.204,4.194)%
  --(9.207,4.195)--(9.210,4.197)--(9.213,4.198)--(9.216,4.200)--(9.219,4.201)--(9.222,4.203)%
  --(9.225,4.204)--(9.228,4.206)--(9.231,4.207)--(9.234,4.209)--(9.237,4.210)--(9.240,4.212)%
  --(9.243,4.213)--(9.246,4.215)--(9.249,4.216)--(9.252,4.217)--(9.255,4.219)--(9.258,4.220)%
  --(9.261,4.222)--(9.264,4.223)--(9.267,4.225)--(9.270,4.226)--(9.273,4.228)--(9.276,4.229)%
  --(9.279,4.231)--(9.282,4.232)--(9.285,4.234)--(9.288,4.235)--(9.291,4.237)--(9.294,4.238)%
  --(9.297,4.240)--(9.300,4.241)--(9.303,4.243)--(9.306,4.244)--(9.309,4.246)--(9.312,4.247)%
  --(9.315,4.248)--(9.318,4.250)--(9.321,4.251)--(9.324,4.253)--(9.327,4.254)--(9.330,4.256)%
  --(9.333,4.257)--(9.336,4.259)--(9.339,4.260)--(9.342,4.262)--(9.345,4.263)--(9.348,4.265)%
  --(9.351,4.266)--(9.354,4.268)--(9.357,4.269)--(9.360,4.271)--(9.362,4.272)--(9.365,4.274)%
  --(9.368,4.275)--(9.371,4.277)--(9.374,4.278)--(9.377,4.280)--(9.380,4.281)--(9.383,4.282)%
  --(9.386,4.284)--(9.389,4.285)--(9.392,4.287)--(9.395,4.288)--(9.398,4.290)--(9.401,4.291)%
  --(9.404,4.293)--(9.407,4.294)--(9.410,4.296)--(9.413,4.297)--(9.416,4.299)--(9.419,4.300)%
  --(9.422,4.302)--(9.425,4.303)--(9.428,4.305)--(9.431,4.306)--(9.434,4.308)--(9.437,4.309)%
  --(9.440,4.311)--(9.443,4.312)--(9.446,4.314)--(9.449,4.315)--(9.452,4.317)--(9.455,4.318)%
  --(9.458,4.319)--(9.461,4.321)--(9.464,4.322)--(9.467,4.324)--(9.470,4.325)--(9.473,4.327)%
  --(9.476,4.328)--(9.479,4.330)--(9.482,4.331)--(9.485,4.333)--(9.488,4.334)--(9.491,4.336)%
  --(9.494,4.337)--(9.497,4.339)--(9.500,4.340)--(9.503,4.342)--(9.506,4.343)--(9.509,4.345)%
  --(9.512,4.346)--(9.515,4.348)--(9.518,4.349)--(9.521,4.351)--(9.524,4.352)--(9.527,4.354)%
  --(9.530,4.355)--(9.533,4.357)--(9.536,4.358)--(9.539,4.359)--(9.542,4.361)--(9.545,4.362)%
  --(9.548,4.364)--(9.551,4.365)--(9.554,4.367)--(9.557,4.368)--(9.560,4.370)--(9.563,4.371)%
  --(9.566,4.373)--(9.569,4.374)--(9.571,4.376)--(9.574,4.377)--(9.577,4.379)--(9.580,4.380)%
  --(9.583,4.382)--(9.586,4.383)--(9.589,4.385)--(9.592,4.386)--(9.595,4.388)--(9.598,4.389)%
  --(9.601,4.391)--(9.604,4.392)--(9.607,4.394)--(9.610,4.395)--(9.613,4.397)--(9.616,4.398)%
  --(9.619,4.400)--(9.622,4.401)--(9.625,4.402)--(9.628,4.404)--(9.631,4.405)--(9.634,4.407)%
  --(9.637,4.408)--(9.640,4.410)--(9.643,4.411)--(9.646,4.413)--(9.649,4.414)--(9.652,4.416)%
  --(9.655,4.417)--(9.658,4.419)--(9.661,4.420)--(9.664,4.422)--(9.667,4.423)--(9.670,4.425)%
  --(9.673,4.426)--(9.676,4.428)--(9.679,4.429)--(9.682,4.431)--(9.685,4.432)--(9.688,4.434)%
  --(9.691,4.435)--(9.694,4.437)--(9.697,4.438)--(9.700,4.440)--(9.703,4.441)--(9.706,4.443)%
  --(9.709,4.444)--(9.712,4.446)--(9.715,4.447)--(9.718,4.448)--(9.721,4.450)--(9.724,4.451)%
  --(9.727,4.453)--(9.730,4.454)--(9.733,4.456)--(9.736,4.457)--(9.739,4.459)--(9.742,4.460)%
  --(9.745,4.462)--(9.748,4.463)--(9.751,4.465)--(9.754,4.466)--(9.757,4.468)--(9.760,4.469)%
  --(9.763,4.471)--(9.766,4.472)--(9.769,4.474)--(9.772,4.475)--(9.775,4.477)--(9.778,4.478)%
  --(9.780,4.480)--(9.783,4.481)--(9.786,4.483)--(9.789,4.484)--(9.792,4.486)--(9.795,4.487)%
  --(9.798,4.489)--(9.801,4.490)--(9.804,4.492)--(9.807,4.493)--(9.810,4.495)--(9.813,4.496)%
  --(9.816,4.498)--(9.819,4.499)--(9.822,4.500)--(9.825,4.502)--(9.828,4.503)--(9.831,4.505)%
  --(9.834,4.506)--(9.837,4.508)--(9.840,4.509)--(9.843,4.511)--(9.846,4.512)--(9.849,4.514)%
  --(9.852,4.515)--(9.855,4.517)--(9.858,4.518)--(9.861,4.520)--(9.864,4.521)--(9.867,4.523)%
  --(9.870,4.524)--(9.873,4.526)--(9.876,4.527)--(9.879,4.529)--(9.882,4.530)--(9.885,4.532)%
  --(9.888,4.533)--(9.891,4.535)--(9.894,4.536)--(9.897,4.538)--(9.900,4.539)--(9.903,4.541)%
  --(9.906,4.542)--(9.909,4.544)--(9.912,4.545)--(9.915,4.547)--(9.918,4.548)--(9.921,4.550)%
  --(9.924,4.551)--(9.927,4.553)--(9.930,4.554)--(9.933,4.556)--(9.936,4.557)--(9.939,4.558)%
  --(9.942,4.560)--(9.945,4.561)--(9.948,4.563)--(9.951,4.564)--(9.954,4.566)--(9.957,4.567)%
  --(9.960,4.569)--(9.963,4.570)--(9.966,4.572)--(9.969,4.573)--(9.972,4.575)--(9.975,4.576)%
  --(9.978,4.578)--(9.981,4.579)--(9.984,4.581)--(9.987,4.582)--(9.990,4.584)--(9.992,4.585)%
  --(9.995,4.587)--(9.998,4.588)--(10.001,4.590)--(10.004,4.591)--(10.007,4.593)--(10.010,4.594)%
  --(10.013,4.596)--(10.016,4.597)--(10.019,4.599)--(10.022,4.600)--(10.025,4.602)--(10.028,4.603)%
  --(10.031,4.605)--(10.034,4.606)--(10.037,4.608)--(10.040,4.609)--(10.043,4.611)--(10.046,4.612)%
  --(10.049,4.614)--(10.052,4.615)--(10.055,4.617)--(10.058,4.618)--(10.061,4.620)--(10.064,4.621)%
  --(10.067,4.623)--(10.070,4.624)--(10.073,4.625)--(10.076,4.627)--(10.079,4.628)--(10.082,4.630)%
  --(10.085,4.631)--(10.088,4.633)--(10.091,4.634)--(10.094,4.636)--(10.097,4.637)--(10.100,4.639)%
  --(10.103,4.640)--(10.106,4.642)--(10.109,4.643)--(10.112,4.645)--(10.115,4.646)--(10.118,4.648)%
  --(10.121,4.649)--(10.124,4.651)--(10.127,4.652)--(10.130,4.654)--(10.133,4.655)--(10.136,4.657)%
  --(10.139,4.658)--(10.142,4.660)--(10.145,4.661)--(10.148,4.663)--(10.151,4.664)--(10.154,4.666)%
  --(10.157,4.667)--(10.160,4.669)--(10.163,4.670)--(10.166,4.672)--(10.169,4.673)--(10.172,4.675)%
  --(10.175,4.676)--(10.178,4.678)--(10.181,4.679)--(10.184,4.681)--(10.187,4.682)--(10.190,4.684)%
  --(10.193,4.685)--(10.196,4.687)--(10.199,4.688)--(10.201,4.690)--(10.204,4.691)--(10.207,4.693)%
  --(10.210,4.694)--(10.213,4.696)--(10.216,4.697)--(10.219,4.699)--(10.222,4.700)--(10.225,4.702)%
  --(10.228,4.703)--(10.231,4.705)--(10.234,4.706)--(10.237,4.707)--(10.240,4.709)--(10.243,4.710)%
  --(10.246,4.712)--(10.249,4.713)--(10.252,4.715)--(10.255,4.716)--(10.258,4.718)--(10.261,4.719)%
  --(10.264,4.721)--(10.267,4.722)--(10.270,4.724)--(10.273,4.725)--(10.276,4.727)--(10.279,4.728)%
  --(10.282,4.730)--(10.285,4.731)--(10.288,4.733)--(10.291,4.734)--(10.294,4.736)--(10.297,4.737)%
  --(10.300,4.739)--(10.303,4.740)--(10.306,4.742)--(10.309,4.743)--(10.312,4.745)--(10.315,4.746)%
  --(10.318,4.748)--(10.321,4.749)--(10.324,4.751)--(10.327,4.752)--(10.330,4.754)--(10.333,4.755)%
  --(10.336,4.757)--(10.339,4.758)--(10.342,4.760)--(10.345,4.761)--(10.348,4.763)--(10.351,4.764)%
  --(10.354,4.766)--(10.357,4.767)--(10.360,4.769)--(10.363,4.770)--(10.366,4.772)--(10.369,4.773)%
  --(10.372,4.775)--(10.375,4.776)--(10.378,4.778)--(10.381,4.779)--(10.384,4.781)--(10.387,4.782)%
  --(10.390,4.784)--(10.393,4.785)--(10.396,4.787)--(10.399,4.788)--(10.402,4.790)--(10.405,4.791)%
  --(10.408,4.793)--(10.410,4.794)--(10.413,4.796)--(10.416,4.797)--(10.419,4.799)--(10.422,4.800)%
  --(10.425,4.802)--(10.428,4.803)--(10.431,4.805)--(10.434,4.806)--(10.437,4.808)--(10.440,4.809)%
  --(10.443,4.811)--(10.446,4.812)--(10.449,4.814)--(10.452,4.815)--(10.455,4.816)--(10.458,4.818)%
  --(10.461,4.819)--(10.464,4.821)--(10.467,4.822)--(10.470,4.824)--(10.473,4.825)--(10.476,4.827)%
  --(10.479,4.828)--(10.482,4.830)--(10.485,4.831)--(10.488,4.833)--(10.491,4.834)--(10.494,4.836)%
  --(10.497,4.837)--(10.500,4.839)--(10.503,4.840)--(10.506,4.842)--(10.509,4.843)--(10.512,4.845)%
  --(10.515,4.846)--(10.518,4.848)--(10.521,4.849)--(10.524,4.851)--(10.527,4.852)--(10.530,4.854)%
  --(10.533,4.855)--(10.536,4.857)--(10.539,4.858)--(10.542,4.860)--(10.545,4.861)--(10.548,4.863)%
  --(10.551,4.864)--(10.554,4.866)--(10.557,4.867)--(10.560,4.869)--(10.563,4.870)--(10.566,4.872)%
  --(10.569,4.873)--(10.572,4.875)--(10.575,4.876)--(10.578,4.878)--(10.581,4.879)--(10.584,4.881)%
  --(10.587,4.882)--(10.590,4.884)--(10.593,4.885)--(10.596,4.887)--(10.599,4.888)--(10.602,4.890)%
  --(10.605,4.891)--(10.608,4.893)--(10.611,4.894)--(10.614,4.896)--(10.617,4.897)--(10.619,4.899)%
  --(10.622,4.900)--(10.625,4.902)--(10.628,4.903)--(10.631,4.905)--(10.634,4.906)--(10.637,4.908)%
  --(10.640,4.909)--(10.643,4.911)--(10.646,4.912)--(10.649,4.914)--(10.652,4.915)--(10.655,4.917)%
  --(10.658,4.918)--(10.661,4.920)--(10.664,4.921)--(10.667,4.923)--(10.670,4.924)--(10.673,4.926)%
  --(10.676,4.927)--(10.679,4.929)--(10.682,4.930)--(10.685,4.932)--(10.688,4.933)--(10.691,4.935)%
  --(10.694,4.936)--(10.697,4.938)--(10.700,4.939)--(10.703,4.941)--(10.706,4.942)--(10.709,4.944)%
  --(10.712,4.945)--(10.715,4.947)--(10.718,4.948)--(10.721,4.950)--(10.724,4.951)--(10.727,4.953)%
  --(10.730,4.954)--(10.733,4.956)--(10.736,4.957)--(10.739,4.959)--(10.742,4.960)--(10.745,4.962)%
  --(10.748,4.963)--(10.751,4.965)--(10.754,4.966)--(10.757,4.968)--(10.760,4.969)--(10.763,4.971)%
  --(10.766,4.972)--(10.769,4.974)--(10.772,4.975)--(10.775,4.977)--(10.778,4.978)--(10.781,4.980)%
  --(10.784,4.981)--(10.787,4.983)--(10.790,4.984)--(10.793,4.986)--(10.796,4.987)--(10.799,4.989)%
  --(10.802,4.990)--(10.805,4.992)--(10.808,4.993)--(10.811,4.995)--(10.814,4.996)--(10.817,4.998)%
  --(10.820,4.999)--(10.823,5.001)--(10.826,5.002)--(10.828,5.004)--(10.831,5.005)--(10.834,5.007)%
  --(10.837,5.008)--(10.840,5.010)--(10.843,5.011)--(10.846,5.013)--(10.849,5.014)--(10.852,5.016)%
  --(10.855,5.017)--(10.858,5.019)--(10.861,5.020)--(10.864,5.022)--(10.867,5.023)--(10.870,5.025)%
  --(10.873,5.026)--(10.876,5.028)--(10.879,5.029)--(10.882,5.031)--(10.885,5.032)--(10.888,5.034)%
  --(10.891,5.035)--(10.894,5.037)--(10.897,5.038)--(10.900,5.040)--(10.903,5.041)--(10.906,5.043)%
  --(10.909,5.044)--(10.912,5.046)--(10.915,5.047)--(10.918,5.049)--(10.921,5.050)--(10.924,5.052)%
  --(10.927,5.053)--(10.930,5.055)--(10.933,5.056)--(10.936,5.058)--(10.939,5.059)--(10.942,5.060)%
  --(10.945,5.062)--(10.948,5.063)--(10.951,5.065)--(10.954,5.066)--(10.957,5.068)--(10.960,5.069)%
  --(10.963,5.071)--(10.966,5.072)--(10.969,5.074)--(10.972,5.075)--(10.975,5.077)--(10.978,5.078)%
  --(10.981,5.080)--(10.984,5.081)--(10.987,5.083)--(10.990,5.084)--(10.993,5.086)--(10.996,5.087)%
  --(10.999,5.089)--(11.002,5.090)--(11.005,5.092)--(11.008,5.093)--(11.011,5.095)--(11.014,5.096)%
  --(11.017,5.098)--(11.020,5.099)--(11.023,5.101)--(11.026,5.102)--(11.029,5.104)--(11.032,5.105)%
  --(11.035,5.107)--(11.037,5.108)--(11.040,5.110)--(11.043,5.111)--(11.046,5.113)--(11.049,5.114)%
  --(11.052,5.116)--(11.055,5.117)--(11.058,5.119)--(11.061,5.120)--(11.064,5.122)--(11.067,5.123)%
  --(11.070,5.125)--(11.073,5.126)--(11.076,5.128)--(11.079,5.129)--(11.082,5.131)--(11.085,5.132)%
  --(11.088,5.134)--(11.091,5.135)--(11.094,5.137)--(11.097,5.138)--(11.100,5.140)--(11.103,5.142)%
  --(11.106,5.143)--(11.109,5.145)--(11.112,5.146)--(11.115,5.148)--(11.118,5.149)--(11.121,5.151)%
  --(11.124,5.152)--(11.127,5.154)--(11.130,5.155)--(11.133,5.157)--(11.136,5.158)--(11.139,5.160)%
  --(11.142,5.161)--(11.145,5.163)--(11.148,5.164)--(11.151,5.166)--(11.154,5.167)--(11.157,5.169)%
  --(11.160,5.170)--(11.163,5.172)--(11.166,5.173)--(11.169,5.175)--(11.172,5.176)--(11.175,5.178)%
  --(11.178,5.179)--(11.181,5.181)--(11.184,5.182)--(11.187,5.184)--(11.190,5.185)--(11.193,5.187)%
  --(11.196,5.188)--(11.199,5.190)--(11.202,5.191)--(11.205,5.193)--(11.208,5.194)--(11.211,5.196)%
  --(11.214,5.197)--(11.217,5.199)--(11.220,5.200)--(11.223,5.202)--(11.226,5.203)--(11.229,5.205)%
  --(11.232,5.206)--(11.235,5.208)--(11.238,5.209)--(11.241,5.211)--(11.244,5.212)--(11.247,5.214)%
  --(11.249,5.215)--(11.252,5.217)--(11.255,5.218)--(11.258,5.220)--(11.261,5.221)--(11.264,5.223)%
  --(11.267,5.224)--(11.270,5.226)--(11.273,5.227)--(11.276,5.229)--(11.279,5.230)--(11.282,5.232)%
  --(11.285,5.233)--(11.288,5.235)--(11.291,5.236)--(11.294,5.238)--(11.297,5.239)--(11.300,5.241)%
  --(11.303,5.242)--(11.306,5.244)--(11.309,5.245)--(11.312,5.247)--(11.315,5.248)--(11.318,5.250)%
  --(11.321,5.251)--(11.324,5.253)--(11.327,5.254)--(11.330,5.256)--(11.333,5.257)--(11.336,5.259)%
  --(11.339,5.260)--(11.342,5.262)--(11.345,5.263)--(11.348,5.265)--(11.351,5.266)--(11.354,5.268)%
  --(11.357,5.269)--(11.360,5.271)--(11.363,5.272)--(11.366,5.274)--(11.369,5.275)--(11.372,5.277)%
  --(11.375,5.278)--(11.378,5.280)--(11.381,5.281)--(11.384,5.283)--(11.387,5.284)--(11.390,5.286)%
  --(11.393,5.287)--(11.396,5.289)--(11.399,5.290)--(11.402,5.292)--(11.405,5.293)--(11.408,5.295)%
  --(11.411,5.296)--(11.414,5.298)--(11.417,5.299)--(11.420,5.301)--(11.423,5.302)--(11.426,5.304)%
  --(11.429,5.305)--(11.432,5.307)--(11.435,5.308)--(11.438,5.310)--(11.441,5.311)--(11.444,5.313)%
  --(11.447,5.314)--(11.450,5.316)--(11.453,5.317)--(11.456,5.319)--(11.458,5.320)--(11.461,5.322)%
  --(11.464,5.323)--(11.467,5.325)--(11.470,5.326)--(11.473,5.328)--(11.476,5.329)--(11.479,5.331)%
  --(11.482,5.332)--(11.485,5.334)--(11.488,5.335)--(11.491,5.337)--(11.494,5.338)--(11.497,5.340)%
  --(11.500,5.341)--(11.503,5.343)--(11.506,5.344)--(11.509,5.346)--(11.512,5.347)--(11.515,5.349)%
  --(11.518,5.350)--(11.521,5.352)--(11.524,5.353)--(11.527,5.355)--(11.530,5.356)--(11.533,5.358)%
  --(11.536,5.359)--(11.539,5.361)--(11.542,5.362)--(11.545,5.364)--(11.548,5.365)--(11.551,5.367)%
  --(11.554,5.368)--(11.557,5.370)--(11.560,5.371)--(11.563,5.373)--(11.566,5.374)--(11.569,5.376)%
  --(11.572,5.377)--(11.575,5.379)--(11.578,5.380)--(11.581,5.382)--(11.584,5.383)--(11.587,5.385)%
  --(11.590,5.386)--(11.593,5.388)--(11.596,5.389)--(11.599,5.391)--(11.602,5.392)--(11.605,5.394)%
  --(11.608,5.395)--(11.611,5.397)--(11.614,5.398)--(11.617,5.400)--(11.620,5.401)--(11.623,5.403)%
  --(11.626,5.404)--(11.629,5.406)--(11.632,5.408)--(11.635,5.409)--(11.638,5.411)--(11.641,5.412)%
  --(11.644,5.414)--(11.647,5.415)--(11.650,5.417)--(11.653,5.418)--(11.656,5.420)--(11.659,5.421)%
  --(11.662,5.423)--(11.665,5.424)--(11.667,5.426)--(11.670,5.427)--(11.673,5.429)--(11.676,5.430)%
  --(11.679,5.432)--(11.682,5.433)--(11.685,5.435)--(11.688,5.436)--(11.691,5.438)--(11.694,5.439)%
  --(11.697,5.441)--(11.700,5.442)--(11.703,5.444)--(11.706,5.445)--(11.709,5.447)--(11.712,5.448)%
  --(11.715,5.450)--(11.718,5.451)--(11.721,5.453)--(11.724,5.454)--(11.727,5.456)--(11.730,5.457)%
  --(11.733,5.459)--(11.736,5.460)--(11.739,5.462)--(11.742,5.463)--(11.745,5.465)--(11.748,5.466)%
  --(11.751,5.468)--(11.754,5.469)--(11.757,5.471)--(11.760,5.472)--(11.763,5.474)--(11.766,5.475)%
  --(11.769,5.477)--(11.772,5.478)--(11.775,5.480)--(11.778,5.481)--(11.781,5.483)--(11.784,5.484)%
  --(11.787,5.486)--(11.790,5.487)--(11.793,5.489)--(11.796,5.490)--(11.799,5.492)--(11.802,5.493)%
  --(11.805,5.495)--(11.808,5.496)--(11.811,5.498)--(11.814,5.499)--(11.817,5.501)--(11.820,5.502)%
  --(11.823,5.504)--(11.826,5.505)--(11.829,5.507)--(11.832,5.508)--(11.835,5.510)--(11.838,5.511)%
  --(11.841,5.513)--(11.844,5.514)--(11.847,5.516)--(11.850,5.517)--(11.853,5.519)--(11.856,5.520)%
  --(11.859,5.522)--(11.862,5.523)--(11.865,5.525)--(11.868,5.526)--(11.871,5.528)--(11.874,5.529)%
  --(11.876,5.531)--(11.879,5.532)--(11.882,5.534)--(11.885,5.535)--(11.888,5.537)--(11.891,5.538)%
  --(11.894,5.540)--(11.897,5.542)--(11.900,5.543)--(11.903,5.545)--(11.906,5.546)--(11.909,5.548)%
  --(11.912,5.549)--(11.915,5.551)--(11.918,5.552)--(11.921,5.554)--(11.924,5.555)--(11.927,5.557)%
  --(11.930,5.558)--(11.933,5.560)--(11.936,5.561)--(11.939,5.563)--(11.942,5.564)--(11.945,5.566)%
  --(11.948,5.567)--(11.951,5.569)--(11.954,5.570)--(11.957,5.572)--(11.960,5.573)--(11.963,5.575)%
  --(11.966,5.576)--(11.969,5.578)--(11.972,5.579)--(11.975,5.581)--(11.978,5.582)--(11.981,5.584)%
  --(11.984,5.585)--(11.987,5.587)--(11.990,5.588)--(11.993,5.590)--(11.996,5.591)--(11.999,5.593)%
  --(12.002,5.594)--(12.005,5.596)--(12.008,5.597)--(12.011,5.599)--(12.014,5.600)--(12.017,5.602)%
  --(12.020,5.603)--(12.023,5.605)--(12.026,5.606)--(12.029,5.608)--(12.032,5.609)--(12.035,5.611)%
  --(12.038,5.612)--(12.041,5.614)--(12.044,5.615)--(12.047,5.617)--(12.050,5.618)--(12.053,5.620)%
  --(12.056,5.621)--(12.059,5.623)--(12.062,5.624)--(12.065,5.626)--(12.068,5.627)--(12.071,5.629)%
  --(12.074,5.630)--(12.077,5.632)--(12.080,5.633)--(12.083,5.635)--(12.085,5.636)--(12.088,5.638)%
  --(12.091,5.639)--(12.094,5.641)--(12.097,5.642)--(12.100,5.644)--(12.103,5.646)--(12.106,5.647)%
  --(12.109,5.649)--(12.112,5.650)--(12.115,5.652)--(12.118,5.653)--(12.121,5.655)--(12.124,5.656)%
  --(12.127,5.658)--(12.130,5.659)--(12.133,5.661)--(12.136,5.662)--(12.139,5.664)--(12.142,5.665)%
  --(12.145,5.667)--(12.148,5.668)--(12.151,5.670)--(12.154,5.671)--(12.157,5.673)--(12.160,5.674)%
  --(12.163,5.676)--(12.166,5.677)--(12.169,5.679)--(12.172,5.680)--(12.175,5.682)--(12.178,5.683)%
  --(12.181,5.685)--(12.184,5.686)--(12.187,5.688)--(12.190,5.689)--(12.193,5.691)--(12.196,5.692)%
  --(12.199,5.694)--(12.202,5.695)--(12.205,5.697)--(12.208,5.698)--(12.211,5.700)--(12.214,5.701)%
  --(12.217,5.703)--(12.220,5.704)--(12.223,5.706)--(12.226,5.707)--(12.229,5.709)--(12.232,5.710)%
  --(12.235,5.712)--(12.238,5.713)--(12.241,5.715)--(12.244,5.716)--(12.247,5.718)--(12.250,5.719)%
  --(12.253,5.721)--(12.256,5.722)--(12.259,5.724)--(12.262,5.725)--(12.265,5.727)--(12.268,5.728)%
  --(12.271,5.730)--(12.274,5.731)--(12.277,5.733)--(12.280,5.735)--(12.283,5.736)--(12.286,5.738)%
  --(12.289,5.739)--(12.292,5.741)--(12.295,5.742)--(12.297,5.744)--(12.300,5.745)--(12.303,5.747)%
  --(12.306,5.748)--(12.309,5.750)--(12.312,5.751)--(12.315,5.753)--(12.318,5.754)--(12.321,5.756)%
  --(12.324,5.757)--(12.327,5.759)--(12.330,5.760)--(12.333,5.762)--(12.336,5.763)--(12.339,5.765)%
  --(12.342,5.766)--(12.345,5.768)--(12.348,5.769)--(12.351,5.771)--(12.354,5.772)--(12.357,5.774)%
  --(12.360,5.775)--(12.363,5.777)--(12.366,5.778)--(12.369,5.780)--(12.372,5.781)--(12.375,5.783)%
  --(12.378,5.784)--(12.381,5.786)--(12.384,5.787)--(12.387,5.789)--(12.390,5.790)--(12.393,5.792)%
  --(12.396,5.793)--(12.399,5.795)--(12.402,5.796)--(12.405,5.798)--(12.408,5.799)--(12.411,5.801)%
  --(12.414,5.802)--(12.417,5.804)--(12.420,5.805)--(12.423,5.807)--(12.426,5.808)--(12.429,5.810)%
  --(12.432,5.811)--(12.435,5.813)--(12.438,5.814)--(12.441,5.816)--(12.444,5.818)--(12.447,5.819)%
  --(12.450,5.821)--(12.453,5.822)--(12.456,5.824)--(12.459,5.825)--(12.462,5.827)--(12.465,5.828)%
  --(12.468,5.830)--(12.471,5.831)--(12.474,5.833)--(12.477,5.834)--(12.480,5.836)--(12.483,5.837)%
  --(12.486,5.839)--(12.489,5.840)--(12.492,5.842)--(12.495,5.843)--(12.498,5.845)--(12.501,5.846)%
  --(12.504,5.848)--(12.506,5.849)--(12.509,5.851)--(12.512,5.852)--(12.515,5.854)--(12.518,5.855)%
  --(12.521,5.857)--(12.524,5.858)--(12.527,5.860)--(12.530,5.861)--(12.533,5.863)--(12.536,5.864)%
  --(12.539,5.866)--(12.542,5.867)--(12.545,5.869)--(12.548,5.870)--(12.551,5.872)--(12.554,5.873)%
  --(12.557,5.875)--(12.560,5.876)--(12.563,5.878)--(12.566,5.879)--(12.569,5.881)--(12.572,5.882)%
  --(12.575,5.884)--(12.578,5.885)--(12.581,5.887)--(12.584,5.888)--(12.587,5.890)--(12.590,5.892)%
  --(12.593,5.893)--(12.596,5.895)--(12.599,5.896)--(12.602,5.898)--(12.605,5.899)--(12.608,5.901)%
  --(12.611,5.902)--(12.614,5.904)--(12.617,5.905)--(12.620,5.907)--(12.623,5.908)--(12.626,5.910)%
  --(12.629,5.911)--(12.632,5.913)--(12.635,5.914)--(12.638,5.916)--(12.641,5.917)--(12.644,5.919)%
  --(12.647,5.920)--(12.650,5.922)--(12.653,5.923)--(12.656,5.925)--(12.659,5.926)--(12.662,5.928)%
  --(12.665,5.929)--(12.668,5.931)--(12.671,5.932)--(12.674,5.934)--(12.677,5.935)--(12.680,5.937)%
  --(12.683,5.938)--(12.686,5.940)--(12.689,5.941)--(12.692,5.943)--(12.695,5.944)--(12.698,5.946)%
  --(12.701,5.947)--(12.704,5.949)--(12.707,5.950)--(12.710,5.952)--(12.713,5.953)--(12.715,5.955)%
  --(12.718,5.956)--(12.721,5.958)--(12.724,5.959)--(12.727,5.961)--(12.730,5.963)--(12.733,5.964)%
  --(12.736,5.966)--(12.739,5.967)--(12.742,5.969)--(12.745,5.970)--(12.748,5.972)--(12.751,5.973)%
  --(12.754,5.975)--(12.757,5.976)--(12.760,5.978)--(12.763,5.979)--(12.766,5.981)--(12.769,5.982)%
  --(12.772,5.984)--(12.775,5.985)--(12.778,5.987)--(12.781,5.988)--(12.784,5.990)--(12.787,5.991)%
  --(12.790,5.993)--(12.793,5.994)--(12.796,5.996)--(12.799,5.997)--(12.802,5.999)--(12.805,6.000)%
  --(12.808,6.002)--(12.811,6.003)--(12.814,6.005)--(12.817,6.006)--(12.820,6.008)--(12.823,6.009)%
  --(12.826,6.011)--(12.829,6.012)--(12.832,6.014)--(12.835,6.015)--(12.838,6.017)--(12.841,6.018)%
  --(12.844,6.020)--(12.847,6.021)--(12.850,6.023)--(12.853,6.024)--(12.856,6.026)--(12.859,6.027)%
  --(12.862,6.029)--(12.865,6.031)--(12.868,6.032)--(12.871,6.034)--(12.874,6.035)--(12.877,6.037)%
  --(12.880,6.038)--(12.883,6.040)--(12.886,6.041)--(12.889,6.043)--(12.892,6.044)--(12.895,6.046)%
  --(12.898,6.047)--(12.901,6.049)--(12.904,6.050)--(12.907,6.052)--(12.910,6.053)--(12.913,6.055)%
  --(12.916,6.056)--(12.919,6.058)--(12.922,6.059)--(12.924,6.061)--(12.927,6.062)--(12.930,6.064)%
  --(12.933,6.065)--(12.936,6.067)--(12.939,6.068)--(12.942,6.070)--(12.945,6.071)--(12.948,6.073)%
  --(12.951,6.074)--(12.954,6.076)--(12.957,6.077)--(12.960,6.079)--(12.963,6.080)--(12.966,6.082)%
  --(12.969,6.083)--(12.972,6.085)--(12.975,6.086)--(12.978,6.088)--(12.981,6.089)--(12.984,6.091)%
  --(12.987,6.093)--(12.990,6.094)--(12.993,6.096)--(12.996,6.097)--(12.999,6.099)--(13.002,6.100)%
  --(13.005,6.102)--(13.008,6.103)--(13.011,6.105)--(13.014,6.106)--(13.017,6.108)--(13.020,6.109)%
  --(13.023,6.111)--(13.026,6.112)--(13.029,6.114)--(13.032,6.115)--(13.035,6.117)--(13.038,6.118)%
  --(13.041,6.120)--(13.044,6.121)--(13.047,6.123)--(13.050,6.124)--(13.053,6.126)--(13.056,6.127)%
  --(13.059,6.129)--(13.062,6.130)--(13.065,6.132)--(13.068,6.133)--(13.071,6.135)--(13.074,6.136)%
  --(13.077,6.138)--(13.080,6.139)--(13.083,6.141)--(13.086,6.142)--(13.089,6.144)--(13.092,6.145)%
  --(13.095,6.147)--(13.098,6.148)--(13.101,6.150)--(13.104,6.152)--(13.107,6.153)--(13.110,6.155)%
  --(13.113,6.156)--(13.116,6.158)--(13.119,6.159)--(13.122,6.161)--(13.125,6.162)--(13.128,6.164)%
  --(13.131,6.165)--(13.133,6.167)--(13.136,6.168)--(13.139,6.170)--(13.142,6.171)--(13.145,6.173)%
  --(13.148,6.174)--(13.151,6.176)--(13.154,6.177)--(13.157,6.179)--(13.160,6.180)--(13.163,6.182)%
  --(13.166,6.183)--(13.169,6.185)--(13.172,6.186)--(13.175,6.188)--(13.178,6.189)--(13.181,6.191)%
  --(13.184,6.192)--(13.187,6.194)--(13.190,6.195)--(13.193,6.197)--(13.196,6.198)--(13.199,6.200)%
  --(13.202,6.201)--(13.205,6.203)--(13.208,6.204)--(13.211,6.206)--(13.214,6.207)--(13.217,6.209)%
  --(13.220,6.211)--(13.223,6.212)--(13.226,6.214)--(13.229,6.215)--(13.232,6.217)--(13.235,6.218)%
  --(13.238,6.220)--(13.241,6.221)--(13.244,6.223)--(13.247,6.224)--(13.250,6.226)--(13.253,6.227)%
  --(13.256,6.229)--(13.259,6.230)--(13.262,6.232)--(13.265,6.233)--(13.268,6.235)--(13.271,6.236)%
  --(13.274,6.238)--(13.277,6.239)--(13.280,6.241)--(13.283,6.242)--(13.286,6.244)--(13.289,6.245)%
  --(13.292,6.247)--(13.295,6.248)--(13.298,6.250)--(13.301,6.251)--(13.304,6.253)--(13.307,6.254)%
  --(13.310,6.256)--(13.313,6.257)--(13.316,6.259)--(13.319,6.260)--(13.322,6.262)--(13.325,6.263)%
  --(13.328,6.265)--(13.331,6.267)--(13.334,6.268)--(13.337,6.270)--(13.340,6.271)--(13.342,6.273)%
  --(13.345,6.274)--(13.348,6.276)--(13.351,6.277)--(13.354,6.279)--(13.357,6.280)--(13.360,6.282)%
  --(13.363,6.283)--(13.366,6.285)--(13.369,6.286)--(13.372,6.288)--(13.375,6.289)--(13.378,6.291)%
  --(13.381,6.292)--(13.384,6.294)--(13.387,6.295)--(13.390,6.297)--(13.393,6.298)--(13.396,6.300)%
  --(13.399,6.301)--(13.402,6.303)--(13.405,6.304)--(13.408,6.306)--(13.411,6.307)--(13.414,6.309)%
  --(13.417,6.310)--(13.420,6.312)--(13.423,6.313)--(13.426,6.315)--(13.429,6.316)--(13.432,6.318)%
  --(13.435,6.319)--(13.438,6.321)--(13.441,6.323)--(13.444,6.324);
\gpcolor{color=gp lt color border}
\node[gp node left] at (2.972,7.681) {$\rho \approx \nicefrac{2}{3} \cdot \rho_{\rm{max}}$};
\gpcolor{rgb color={0.337,0.706,0.914}}
\draw[gp path] (1.872,7.681)--(2.788,7.681);
\draw[gp path] (1.507,2.513)--(1.510,2.512)--(1.513,2.511)--(1.516,2.510)--(1.519,2.509)%
  --(1.522,2.508)--(1.525,2.507)--(1.528,2.506)--(1.531,2.505)--(1.534,2.504)--(1.537,2.503)%
  --(1.540,2.502)--(1.543,2.501)--(1.546,2.500)--(1.549,2.499)--(1.552,2.498)--(1.555,2.497)%
  --(1.558,2.496)--(1.561,2.495)--(1.564,2.494)--(1.567,2.493)--(1.570,2.492)--(1.573,2.491)%
  --(1.576,2.490)--(1.579,2.489)--(1.582,2.488)--(1.585,2.487)--(1.588,2.486)--(1.591,2.485)%
  --(1.594,2.484)--(1.597,2.483)--(1.600,2.482)--(1.603,2.481)--(1.606,2.480)--(1.609,2.479)%
  --(1.611,2.478)--(1.614,2.477)--(1.617,2.476)--(1.620,2.475)--(1.623,2.474)--(1.626,2.473)%
  --(1.629,2.472)--(1.632,2.471)--(1.635,2.471)--(1.638,2.470)--(1.641,2.469)--(1.644,2.468)%
  --(1.647,2.467)--(1.650,2.466)--(1.653,2.465)--(1.656,2.464)--(1.659,2.463)--(1.662,2.462)%
  --(1.665,2.461)--(1.668,2.460)--(1.671,2.459)--(1.674,2.458)--(1.677,2.457)--(1.680,2.456)%
  --(1.683,2.455)--(1.686,2.454)--(1.689,2.453)--(1.692,2.452)--(1.695,2.451)--(1.698,2.450)%
  --(1.701,2.449)--(1.704,2.448)--(1.707,2.447)--(1.710,2.446)--(1.713,2.445)--(1.716,2.444)%
  --(1.719,2.443)--(1.722,2.442)--(1.725,2.441)--(1.728,2.440)--(1.731,2.440)--(1.734,2.439)%
  --(1.737,2.438)--(1.740,2.437)--(1.743,2.436)--(1.746,2.435)--(1.749,2.434)--(1.752,2.433)%
  --(1.755,2.432)--(1.758,2.431)--(1.761,2.430)--(1.764,2.429)--(1.767,2.428)--(1.770,2.427)%
  --(1.773,2.426)--(1.776,2.425)--(1.779,2.424)--(1.782,2.424)--(1.785,2.423)--(1.788,2.422)%
  --(1.791,2.421)--(1.794,2.420)--(1.797,2.419)--(1.800,2.418)--(1.803,2.417)--(1.806,2.416)%
  --(1.809,2.415)--(1.812,2.414)--(1.815,2.413)--(1.818,2.412)--(1.820,2.412)--(1.823,2.411)%
  --(1.826,2.410)--(1.829,2.409)--(1.832,2.408)--(1.835,2.407)--(1.838,2.406)--(1.841,2.405)%
  --(1.844,2.404)--(1.847,2.403)--(1.850,2.403)--(1.853,2.402)--(1.856,2.401)--(1.859,2.400)%
  --(1.862,2.399)--(1.865,2.398)--(1.868,2.397)--(1.871,2.396)--(1.874,2.395)--(1.877,2.395)%
  --(1.880,2.394)--(1.883,2.393)--(1.886,2.392)--(1.889,2.391)--(1.892,2.390)--(1.895,2.389)%
  --(1.898,2.388)--(1.901,2.388)--(1.904,2.387)--(1.907,2.386)--(1.910,2.385)--(1.913,2.384)%
  --(1.916,2.383)--(1.919,2.382)--(1.922,2.382)--(1.925,2.381)--(1.928,2.380)--(1.931,2.379)%
  --(1.934,2.378)--(1.937,2.377)--(1.940,2.377)--(1.943,2.376)--(1.946,2.375)--(1.949,2.374)%
  --(1.952,2.373)--(1.955,2.372)--(1.958,2.372)--(1.961,2.371)--(1.964,2.370)--(1.967,2.369)%
  --(1.970,2.368)--(1.973,2.367)--(1.976,2.367)--(1.979,2.366)--(1.982,2.365)--(1.985,2.364)%
  --(1.988,2.363)--(1.991,2.363)--(1.994,2.362)--(1.997,2.361)--(2.000,2.360)--(2.003,2.359)%
  --(2.006,2.359)--(2.009,2.358)--(2.012,2.357)--(2.015,2.356)--(2.018,2.356)--(2.021,2.355)%
  --(2.024,2.354)--(2.027,2.353)--(2.029,2.352)--(2.032,2.352)--(2.035,2.351)--(2.038,2.350)%
  --(2.041,2.349)--(2.044,2.349)--(2.047,2.348)--(2.050,2.347)--(2.053,2.346)--(2.056,2.346)%
  --(2.059,2.345)--(2.062,2.344)--(2.065,2.343)--(2.068,2.343)--(2.071,2.342)--(2.074,2.341)%
  --(2.077,2.340)--(2.080,2.340)--(2.083,2.339)--(2.086,2.338)--(2.089,2.338)--(2.092,2.337)%
  --(2.095,2.336)--(2.098,2.335)--(2.101,2.335)--(2.104,2.334)--(2.107,2.333)--(2.110,2.333)%
  --(2.113,2.332)--(2.116,2.331)--(2.119,2.331)--(2.122,2.330)--(2.125,2.329)--(2.128,2.328)%
  --(2.131,2.328)--(2.134,2.327)--(2.137,2.326)--(2.140,2.326)--(2.143,2.325)--(2.146,2.324)%
  --(2.149,2.324)--(2.152,2.323)--(2.155,2.322)--(2.158,2.322)--(2.161,2.321)--(2.164,2.320)%
  --(2.167,2.320)--(2.170,2.319)--(2.173,2.319)--(2.176,2.318)--(2.179,2.317)--(2.182,2.317)%
  --(2.185,2.316)--(2.188,2.315)--(2.191,2.315)--(2.194,2.314)--(2.197,2.313)--(2.200,2.313)%
  --(2.203,2.312)--(2.206,2.312)--(2.209,2.311)--(2.212,2.310)--(2.215,2.310)--(2.218,2.309)%
  --(2.221,2.309)--(2.224,2.308)--(2.227,2.307)--(2.230,2.307)--(2.233,2.306)--(2.236,2.306)%
  --(2.238,2.305)--(2.241,2.304)--(2.244,2.304)--(2.247,2.303)--(2.250,2.303)--(2.253,2.302)%
  --(2.256,2.302)--(2.259,2.301)--(2.262,2.301)--(2.265,2.300)--(2.268,2.299)--(2.271,2.299)%
  --(2.274,2.298)--(2.277,2.298)--(2.280,2.297)--(2.283,2.297)--(2.286,2.296)--(2.289,2.296)%
  --(2.292,2.295)--(2.295,2.295)--(2.298,2.294)--(2.301,2.294)--(2.304,2.293)--(2.307,2.293)%
  --(2.310,2.292)--(2.313,2.291)--(2.316,2.291)--(2.319,2.290)--(2.322,2.290)--(2.325,2.289)%
  --(2.328,2.289)--(2.331,2.289)--(2.334,2.288)--(2.337,2.288)--(2.340,2.287)--(2.343,2.287)%
  --(2.346,2.286)--(2.349,2.286)--(2.352,2.285)--(2.355,2.285)--(2.358,2.284)--(2.361,2.284)%
  --(2.364,2.283)--(2.367,2.283)--(2.370,2.282)--(2.373,2.282)--(2.376,2.282)--(2.379,2.281)%
  --(2.382,2.281)--(2.385,2.280)--(2.388,2.280)--(2.391,2.279)--(2.394,2.279)--(2.397,2.279)%
  --(2.400,2.278)--(2.403,2.278)--(2.406,2.277)--(2.409,2.277)--(2.412,2.276)--(2.415,2.276)%
  --(2.418,2.276)--(2.421,2.275)--(2.424,2.275)--(2.427,2.274)--(2.430,2.274)--(2.433,2.274)%
  --(2.436,2.273)--(2.439,2.273)--(2.442,2.273)--(2.445,2.272)--(2.447,2.272)--(2.450,2.271)%
  --(2.453,2.271)--(2.456,2.271)--(2.459,2.270)--(2.462,2.270)--(2.465,2.270)--(2.468,2.269)%
  --(2.471,2.269)--(2.474,2.269)--(2.477,2.268)--(2.480,2.268)--(2.483,2.268)--(2.486,2.267)%
  --(2.489,2.267)--(2.492,2.267)--(2.495,2.266)--(2.498,2.266)--(2.501,2.266)--(2.504,2.265)%
  --(2.507,2.265)--(2.510,2.265)--(2.513,2.265)--(2.516,2.264)--(2.519,2.264)--(2.522,2.264)%
  --(2.525,2.263)--(2.528,2.263)--(2.531,2.263)--(2.534,2.263)--(2.537,2.262)--(2.540,2.262)%
  --(2.543,2.262)--(2.546,2.261)--(2.549,2.261)--(2.552,2.261)--(2.555,2.261)--(2.558,2.260)%
  --(2.561,2.260)--(2.564,2.260)--(2.567,2.260)--(2.570,2.259)--(2.573,2.259)--(2.576,2.259)%
  --(2.579,2.259)--(2.582,2.258)--(2.585,2.258)--(2.588,2.258)--(2.591,2.258)--(2.594,2.258)%
  --(2.597,2.257)--(2.600,2.257)--(2.603,2.257)--(2.606,2.257)--(2.609,2.257)--(2.612,2.256)%
  --(2.615,2.256)--(2.618,2.256)--(2.621,2.256)--(2.624,2.256)--(2.627,2.255)--(2.630,2.255)%
  --(2.633,2.255)--(2.636,2.255)--(2.639,2.255)--(2.642,2.255)--(2.645,2.254)--(2.648,2.254)%
  --(2.651,2.254)--(2.654,2.254)--(2.656,2.254)--(2.659,2.254)--(2.662,2.254)--(2.665,2.253)%
  --(2.668,2.253)--(2.671,2.253)--(2.674,2.253)--(2.677,2.253)--(2.680,2.253)--(2.683,2.253)%
  --(2.686,2.252)--(2.689,2.252)--(2.692,2.252)--(2.695,2.252)--(2.698,2.252)--(2.701,2.252)%
  --(2.704,2.252)--(2.707,2.252)--(2.710,2.252)--(2.713,2.252)--(2.716,2.251)--(2.719,2.251)%
  --(2.722,2.251)--(2.725,2.251)--(2.728,2.251)--(2.731,2.251)--(2.734,2.251)--(2.737,2.251)%
  --(2.740,2.251)--(2.743,2.251)--(2.746,2.251)--(2.749,2.251)--(2.752,2.251)--(2.755,2.251)%
  --(2.758,2.251)--(2.761,2.251)--(2.764,2.250)--(2.767,2.250)--(2.770,2.250)--(2.773,2.250)%
  --(2.776,2.250)--(2.779,2.250)--(2.782,2.250)--(2.785,2.250)--(2.788,2.250)--(2.791,2.250)%
  --(2.794,2.250)--(2.797,2.250)--(2.800,2.250)--(2.803,2.250)--(2.806,2.250)--(2.809,2.250)%
  --(2.812,2.250)--(2.815,2.250)--(2.818,2.250)--(2.821,2.250)--(2.824,2.250)--(2.827,2.250)%
  --(2.830,2.250)--(2.833,2.250)--(2.836,2.250)--(2.839,2.250)--(2.842,2.251)--(2.845,2.251)%
  --(2.848,2.251)--(2.851,2.251)--(2.854,2.251)--(2.857,2.251)--(2.860,2.251)--(2.863,2.251)%
  --(2.866,2.251)--(2.868,2.251)--(2.871,2.251)--(2.874,2.251)--(2.877,2.251)--(2.880,2.251)%
  --(2.883,2.251)--(2.886,2.251)--(2.889,2.252)--(2.892,2.252)--(2.895,2.252)--(2.898,2.252)%
  --(2.901,2.252)--(2.904,2.252)--(2.907,2.252)--(2.910,2.252)--(2.913,2.252)--(2.916,2.252)%
  --(2.919,2.253)--(2.922,2.253)--(2.925,2.253)--(2.928,2.253)--(2.931,2.253)--(2.934,2.253)%
  --(2.937,2.253)--(2.940,2.253)--(2.943,2.254)--(2.946,2.254)--(2.949,2.254)--(2.952,2.254)%
  --(2.955,2.254)--(2.958,2.254)--(2.961,2.254)--(2.964,2.255)--(2.967,2.255)--(2.970,2.255)%
  --(2.973,2.255)--(2.976,2.255)--(2.979,2.255)--(2.982,2.256)--(2.985,2.256)--(2.988,2.256)%
  --(2.991,2.256)--(2.994,2.256)--(2.997,2.256)--(3.000,2.257)--(3.003,2.257)--(3.006,2.257)%
  --(3.009,2.257)--(3.012,2.257)--(3.015,2.258)--(3.018,2.258)--(3.021,2.258)--(3.024,2.258)%
  --(3.027,2.259)--(3.030,2.259)--(3.033,2.259)--(3.036,2.259)--(3.039,2.259)--(3.042,2.260)%
  --(3.045,2.260)--(3.048,2.260)--(3.051,2.260)--(3.054,2.261)--(3.057,2.261)--(3.060,2.261)%
  --(3.063,2.261)--(3.066,2.262)--(3.069,2.262)--(3.072,2.262)--(3.075,2.262)--(3.077,2.263)%
  --(3.080,2.263)--(3.083,2.263)--(3.086,2.263)--(3.089,2.264)--(3.092,2.264)--(3.095,2.264)%
  --(3.098,2.264)--(3.101,2.265)--(3.104,2.265)--(3.107,2.265)--(3.110,2.266)--(3.113,2.266)%
  --(3.116,2.266)--(3.119,2.266)--(3.122,2.267)--(3.125,2.267)--(3.128,2.267)--(3.131,2.268)%
  --(3.134,2.268)--(3.137,2.268)--(3.140,2.269)--(3.143,2.269)--(3.146,2.269)--(3.149,2.270)%
  --(3.152,2.270)--(3.155,2.270)--(3.158,2.271)--(3.161,2.271)--(3.164,2.271)--(3.167,2.272)%
  --(3.170,2.272)--(3.173,2.272)--(3.176,2.273)--(3.179,2.273)--(3.182,2.273)--(3.185,2.274)%
  --(3.188,2.274)--(3.191,2.274)--(3.194,2.275)--(3.197,2.275)--(3.200,2.275)--(3.203,2.276)%
  --(3.206,2.276)--(3.209,2.277)--(3.212,2.277)--(3.215,2.277)--(3.218,2.278)--(3.221,2.278)%
  --(3.224,2.278)--(3.227,2.279)--(3.230,2.279)--(3.233,2.280)--(3.236,2.280)--(3.239,2.280)%
  --(3.242,2.281)--(3.245,2.281)--(3.248,2.282)--(3.251,2.282)--(3.254,2.282)--(3.257,2.283)%
  --(3.260,2.283)--(3.263,2.284)--(3.266,2.284)--(3.269,2.285)--(3.272,2.285)--(3.275,2.285)%
  --(3.278,2.286)--(3.281,2.286)--(3.284,2.287)--(3.286,2.287)--(3.289,2.288)--(3.292,2.288)%
  --(3.295,2.288)--(3.298,2.289)--(3.301,2.289)--(3.304,2.290)--(3.307,2.290)--(3.310,2.291)%
  --(3.313,2.291)--(3.316,2.292)--(3.319,2.292)--(3.322,2.292)--(3.325,2.293)--(3.328,2.293)%
  --(3.331,2.294)--(3.334,2.294)--(3.337,2.295)--(3.340,2.295)--(3.343,2.296)--(3.346,2.296)%
  --(3.349,2.297)--(3.352,2.297)--(3.355,2.298)--(3.358,2.298)--(3.361,2.299)--(3.364,2.299)%
  --(3.367,2.300)--(3.370,2.300)--(3.373,2.301)--(3.376,2.301)--(3.379,2.302)--(3.382,2.302)%
  --(3.385,2.303)--(3.388,2.303)--(3.391,2.304)--(3.394,2.304)--(3.397,2.305)--(3.400,2.305)%
  --(3.403,2.306)--(3.406,2.306)--(3.409,2.307)--(3.412,2.307)--(3.415,2.308)--(3.418,2.308)%
  --(3.421,2.309)--(3.424,2.310)--(3.427,2.310)--(3.430,2.311)--(3.433,2.311)--(3.436,2.312)%
  --(3.439,2.312)--(3.442,2.313)--(3.445,2.313)--(3.448,2.314)--(3.451,2.314)--(3.454,2.315)%
  --(3.457,2.316)--(3.460,2.316)--(3.463,2.317)--(3.466,2.317)--(3.469,2.318)--(3.472,2.318)%
  --(3.475,2.319)--(3.478,2.320)--(3.481,2.320)--(3.484,2.321)--(3.487,2.321)--(3.490,2.322)%
  --(3.493,2.322)--(3.495,2.323)--(3.498,2.324)--(3.501,2.324)--(3.504,2.325)--(3.507,2.325)%
  --(3.510,2.326)--(3.513,2.327)--(3.516,2.327)--(3.519,2.328)--(3.522,2.328)--(3.525,2.329)%
  --(3.528,2.330)--(3.531,2.330)--(3.534,2.331)--(3.537,2.332)--(3.540,2.332)--(3.543,2.333)%
  --(3.546,2.333)--(3.549,2.334)--(3.552,2.335)--(3.555,2.335)--(3.558,2.336)--(3.561,2.337)%
  --(3.564,2.337)--(3.567,2.338)--(3.570,2.338)--(3.573,2.339)--(3.576,2.340)--(3.579,2.340)%
  --(3.582,2.341)--(3.585,2.342)--(3.588,2.342)--(3.591,2.343)--(3.594,2.344)--(3.597,2.344)%
  --(3.600,2.345)--(3.603,2.346)--(3.606,2.346)--(3.609,2.347)--(3.612,2.348)--(3.615,2.348)%
  --(3.618,2.349)--(3.621,2.350)--(3.624,2.350)--(3.627,2.351)--(3.630,2.352)--(3.633,2.352)%
  --(3.636,2.353)--(3.639,2.354)--(3.642,2.354)--(3.645,2.355)--(3.648,2.356)--(3.651,2.356)%
  --(3.654,2.357)--(3.657,2.358)--(3.660,2.358)--(3.663,2.359)--(3.666,2.360)--(3.669,2.360)%
  --(3.672,2.361)--(3.675,2.362)--(3.678,2.363)--(3.681,2.363)--(3.684,2.364)--(3.687,2.365)%
  --(3.690,2.365)--(3.693,2.366)--(3.696,2.367)--(3.699,2.368)--(3.702,2.368)--(3.704,2.369)%
  --(3.707,2.370)--(3.710,2.370)--(3.713,2.371)--(3.716,2.372)--(3.719,2.373)--(3.722,2.373)%
  --(3.725,2.374)--(3.728,2.375)--(3.731,2.376)--(3.734,2.376)--(3.737,2.377)--(3.740,2.378)%
  --(3.743,2.378)--(3.746,2.379)--(3.749,2.380)--(3.752,2.381)--(3.755,2.381)--(3.758,2.382)%
  --(3.761,2.383)--(3.764,2.384)--(3.767,2.384)--(3.770,2.385)--(3.773,2.386)--(3.776,2.387)%
  --(3.779,2.388)--(3.782,2.388)--(3.785,2.389)--(3.788,2.390)--(3.791,2.391)--(3.794,2.391)%
  --(3.797,2.392)--(3.800,2.393)--(3.803,2.394)--(3.806,2.394)--(3.809,2.395)--(3.812,2.396)%
  --(3.815,2.397)--(3.818,2.398)--(3.821,2.398)--(3.824,2.399)--(3.827,2.400)--(3.830,2.401)%
  --(3.833,2.402)--(3.836,2.402)--(3.839,2.403)--(3.842,2.404)--(3.845,2.405)--(3.848,2.405)%
  --(3.851,2.406)--(3.854,2.407)--(3.857,2.408)--(3.860,2.409)--(3.863,2.410)--(3.866,2.410)%
  --(3.869,2.411)--(3.872,2.412)--(3.875,2.413)--(3.878,2.414)--(3.881,2.414)--(3.884,2.415)%
  --(3.887,2.416)--(3.890,2.417)--(3.893,2.418)--(3.896,2.418)--(3.899,2.419)--(3.902,2.420)%
  --(3.905,2.421)--(3.908,2.422)--(3.911,2.423)--(3.914,2.423)--(3.916,2.424)--(3.919,2.425)%
  --(3.922,2.426)--(3.925,2.427)--(3.928,2.428)--(3.931,2.428)--(3.934,2.429)--(3.937,2.430)%
  --(3.940,2.431)--(3.943,2.432)--(3.946,2.433)--(3.949,2.434)--(3.952,2.434)--(3.955,2.435)%
  --(3.958,2.436)--(3.961,2.437)--(3.964,2.438)--(3.967,2.439)--(3.970,2.440)--(3.973,2.440)%
  --(3.976,2.441)--(3.979,2.442)--(3.982,2.443)--(3.985,2.444)--(3.988,2.445)--(3.991,2.446)%
  --(3.994,2.446)--(3.997,2.447)--(4.000,2.448)--(4.003,2.449)--(4.006,2.450)--(4.009,2.451)%
  --(4.012,2.452)--(4.015,2.453)--(4.018,2.453)--(4.021,2.454)--(4.024,2.455)--(4.027,2.456)%
  --(4.030,2.457)--(4.033,2.458)--(4.036,2.459)--(4.039,2.460)--(4.042,2.461)--(4.045,2.461)%
  --(4.048,2.462)--(4.051,2.463)--(4.054,2.464)--(4.057,2.465)--(4.060,2.466)--(4.063,2.467)%
  --(4.066,2.468)--(4.069,2.469)--(4.072,2.470)--(4.075,2.470)--(4.078,2.471)--(4.081,2.472)%
  --(4.084,2.473)--(4.087,2.474)--(4.090,2.475)--(4.093,2.476)--(4.096,2.477)--(4.099,2.478)%
  --(4.102,2.479)--(4.105,2.480)--(4.108,2.481)--(4.111,2.481)--(4.114,2.482)--(4.117,2.483)%
  --(4.120,2.484)--(4.123,2.485)--(4.125,2.486)--(4.128,2.487)--(4.131,2.488)--(4.134,2.489)%
  --(4.137,2.490)--(4.140,2.491)--(4.143,2.492)--(4.146,2.493)--(4.149,2.494)--(4.152,2.495)%
  --(4.155,2.495)--(4.158,2.496)--(4.161,2.497)--(4.164,2.498)--(4.167,2.499)--(4.170,2.500)%
  --(4.173,2.501)--(4.176,2.502)--(4.179,2.503)--(4.182,2.504)--(4.185,2.505)--(4.188,2.506)%
  --(4.191,2.507)--(4.194,2.508)--(4.197,2.509)--(4.200,2.510)--(4.203,2.511)--(4.206,2.512)%
  --(4.209,2.513)--(4.212,2.514)--(4.215,2.515)--(4.218,2.516)--(4.221,2.516)--(4.224,2.517)%
  --(4.227,2.518)--(4.230,2.519)--(4.233,2.520)--(4.236,2.521)--(4.239,2.522)--(4.242,2.523)%
  --(4.245,2.524)--(4.248,2.525)--(4.251,2.526)--(4.254,2.527)--(4.257,2.528)--(4.260,2.529)%
  --(4.263,2.530)--(4.266,2.531)--(4.269,2.532)--(4.272,2.533)--(4.275,2.534)--(4.278,2.535)%
  --(4.281,2.536)--(4.284,2.537)--(4.287,2.538)--(4.290,2.539)--(4.293,2.540)--(4.296,2.541)%
  --(4.299,2.542)--(4.302,2.543)--(4.305,2.544)--(4.308,2.545)--(4.311,2.546)--(4.314,2.547)%
  --(4.317,2.548)--(4.320,2.549)--(4.323,2.550)--(4.326,2.551)--(4.329,2.552)--(4.332,2.553)%
  --(4.334,2.554)--(4.337,2.555)--(4.340,2.556)--(4.343,2.557)--(4.346,2.558)--(4.349,2.559)%
  --(4.352,2.560)--(4.355,2.561)--(4.358,2.562)--(4.361,2.563)--(4.364,2.564)--(4.367,2.565)%
  --(4.370,2.566)--(4.373,2.567)--(4.376,2.568)--(4.379,2.569)--(4.382,2.570)--(4.385,2.571)%
  --(4.388,2.572)--(4.391,2.573)--(4.394,2.575)--(4.397,2.576)--(4.400,2.577)--(4.403,2.578)%
  --(4.406,2.579)--(4.409,2.580)--(4.412,2.581)--(4.415,2.582)--(4.418,2.583)--(4.421,2.584)%
  --(4.424,2.585)--(4.427,2.586)--(4.430,2.587)--(4.433,2.588)--(4.436,2.589)--(4.439,2.590)%
  --(4.442,2.591)--(4.445,2.592)--(4.448,2.593)--(4.451,2.594)--(4.454,2.595)--(4.457,2.596)%
  --(4.460,2.597)--(4.463,2.599)--(4.466,2.600)--(4.469,2.601)--(4.472,2.602)--(4.475,2.603)%
  --(4.478,2.604)--(4.481,2.605)--(4.484,2.606)--(4.487,2.607)--(4.490,2.608)--(4.493,2.609)%
  --(4.496,2.610)--(4.499,2.611)--(4.502,2.612)--(4.505,2.613)--(4.508,2.614)--(4.511,2.616)%
  --(4.514,2.617)--(4.517,2.618)--(4.520,2.619)--(4.523,2.620)--(4.526,2.621)--(4.529,2.622)%
  --(4.532,2.623)--(4.535,2.624)--(4.538,2.625)--(4.541,2.626)--(4.543,2.627)--(4.546,2.628)%
  --(4.549,2.629)--(4.552,2.631)--(4.555,2.632)--(4.558,2.633)--(4.561,2.634)--(4.564,2.635)%
  --(4.567,2.636)--(4.570,2.637)--(4.573,2.638)--(4.576,2.639)--(4.579,2.640)--(4.582,2.641)%
  --(4.585,2.643)--(4.588,2.644)--(4.591,2.645)--(4.594,2.646)--(4.597,2.647)--(4.600,2.648)%
  --(4.603,2.649)--(4.606,2.650)--(4.609,2.651)--(4.612,2.652)--(4.615,2.653)--(4.618,2.655)%
  --(4.621,2.656)--(4.624,2.657)--(4.627,2.658)--(4.630,2.659)--(4.633,2.660)--(4.636,2.661)%
  --(4.639,2.662)--(4.642,2.663)--(4.645,2.665)--(4.648,2.666)--(4.651,2.667)--(4.654,2.668)%
  --(4.657,2.669)--(4.660,2.670)--(4.663,2.671)--(4.666,2.672)--(4.669,2.673)--(4.672,2.675)%
  --(4.675,2.676)--(4.678,2.677)--(4.681,2.678)--(4.684,2.679)--(4.687,2.680)--(4.690,2.681)%
  --(4.693,2.682)--(4.696,2.683)--(4.699,2.685)--(4.702,2.686)--(4.705,2.687)--(4.708,2.688)%
  --(4.711,2.689)--(4.714,2.690)--(4.717,2.691)--(4.720,2.693)--(4.723,2.694)--(4.726,2.695)%
  --(4.729,2.696)--(4.732,2.697)--(4.735,2.698)--(4.738,2.699)--(4.741,2.700)--(4.744,2.702)%
  --(4.747,2.703)--(4.750,2.704)--(4.752,2.705)--(4.755,2.706)--(4.758,2.707)--(4.761,2.708)%
  --(4.764,2.710)--(4.767,2.711)--(4.770,2.712)--(4.773,2.713)--(4.776,2.714)--(4.779,2.715)%
  --(4.782,2.716)--(4.785,2.718)--(4.788,2.719)--(4.791,2.720)--(4.794,2.721)--(4.797,2.722)%
  --(4.800,2.723)--(4.803,2.724)--(4.806,2.726)--(4.809,2.727)--(4.812,2.728)--(4.815,2.729)%
  --(4.818,2.730)--(4.821,2.731)--(4.824,2.732)--(4.827,2.734)--(4.830,2.735)--(4.833,2.736)%
  --(4.836,2.737)--(4.839,2.738)--(4.842,2.739)--(4.845,2.741)--(4.848,2.742)--(4.851,2.743)%
  --(4.854,2.744)--(4.857,2.745)--(4.860,2.746)--(4.863,2.748)--(4.866,2.749)--(4.869,2.750)%
  --(4.872,2.751)--(4.875,2.752)--(4.878,2.753)--(4.881,2.755)--(4.884,2.756)--(4.887,2.757)%
  --(4.890,2.758)--(4.893,2.759)--(4.896,2.760)--(4.899,2.762)--(4.902,2.763)--(4.905,2.764)%
  --(4.908,2.765)--(4.911,2.766)--(4.914,2.767)--(4.917,2.769)--(4.920,2.770)--(4.923,2.771)%
  --(4.926,2.772)--(4.929,2.773)--(4.932,2.775)--(4.935,2.776)--(4.938,2.777)--(4.941,2.778)%
  --(4.944,2.779)--(4.947,2.780)--(4.950,2.782)--(4.953,2.783)--(4.956,2.784)--(4.959,2.785)%
  --(4.961,2.786)--(4.964,2.788)--(4.967,2.789)--(4.970,2.790)--(4.973,2.791)--(4.976,2.792)%
  --(4.979,2.793)--(4.982,2.795)--(4.985,2.796)--(4.988,2.797)--(4.991,2.798)--(4.994,2.799)%
  --(4.997,2.801)--(5.000,2.802)--(5.003,2.803)--(5.006,2.804)--(5.009,2.805)--(5.012,2.807)%
  --(5.015,2.808)--(5.018,2.809)--(5.021,2.810)--(5.024,2.811)--(5.027,2.813)--(5.030,2.814)%
  --(5.033,2.815)--(5.036,2.816)--(5.039,2.817)--(5.042,2.819)--(5.045,2.820)--(5.048,2.821)%
  --(5.051,2.822)--(5.054,2.823)--(5.057,2.825)--(5.060,2.826)--(5.063,2.827)--(5.066,2.828)%
  --(5.069,2.830)--(5.072,2.831)--(5.075,2.832)--(5.078,2.833)--(5.081,2.834)--(5.084,2.836)%
  --(5.087,2.837)--(5.090,2.838)--(5.093,2.839)--(5.096,2.840)--(5.099,2.842)--(5.102,2.843)%
  --(5.105,2.844)--(5.108,2.845)--(5.111,2.846)--(5.114,2.848)--(5.117,2.849)--(5.120,2.850)%
  --(5.123,2.851)--(5.126,2.853)--(5.129,2.854)--(5.132,2.855)--(5.135,2.856)--(5.138,2.857)%
  --(5.141,2.859)--(5.144,2.860)--(5.147,2.861)--(5.150,2.862)--(5.153,2.864)--(5.156,2.865)%
  --(5.159,2.866)--(5.162,2.867)--(5.165,2.868)--(5.168,2.870)--(5.171,2.871)--(5.173,2.872)%
  --(5.176,2.873)--(5.179,2.875)--(5.182,2.876)--(5.185,2.877)--(5.188,2.878)--(5.191,2.880)%
  --(5.194,2.881)--(5.197,2.882)--(5.200,2.883)--(5.203,2.885)--(5.206,2.886)--(5.209,2.887)%
  --(5.212,2.888)--(5.215,2.889)--(5.218,2.891)--(5.221,2.892)--(5.224,2.893)--(5.227,2.894)%
  --(5.230,2.896)--(5.233,2.897)--(5.236,2.898)--(5.239,2.899)--(5.242,2.901)--(5.245,2.902)%
  --(5.248,2.903)--(5.251,2.904)--(5.254,2.906)--(5.257,2.907)--(5.260,2.908)--(5.263,2.909)%
  --(5.266,2.911)--(5.269,2.912)--(5.272,2.913)--(5.275,2.914)--(5.278,2.916)--(5.281,2.917)%
  --(5.284,2.918)--(5.287,2.919)--(5.290,2.921)--(5.293,2.922)--(5.296,2.923)--(5.299,2.924)%
  --(5.302,2.926)--(5.305,2.927)--(5.308,2.928)--(5.311,2.929)--(5.314,2.931)--(5.317,2.932)%
  --(5.320,2.933)--(5.323,2.934)--(5.326,2.936)--(5.329,2.937)--(5.332,2.938)--(5.335,2.939)%
  --(5.338,2.941)--(5.341,2.942)--(5.344,2.943)--(5.347,2.944)--(5.350,2.946)--(5.353,2.947)%
  --(5.356,2.948)--(5.359,2.949)--(5.362,2.951)--(5.365,2.952)--(5.368,2.953)--(5.371,2.954)%
  --(5.374,2.956)--(5.377,2.957)--(5.380,2.958)--(5.382,2.960)--(5.385,2.961)--(5.388,2.962)%
  --(5.391,2.963)--(5.394,2.965)--(5.397,2.966)--(5.400,2.967)--(5.403,2.968)--(5.406,2.970)%
  --(5.409,2.971)--(5.412,2.972)--(5.415,2.973)--(5.418,2.975)--(5.421,2.976)--(5.424,2.977)%
  --(5.427,2.979)--(5.430,2.980)--(5.433,2.981)--(5.436,2.982)--(5.439,2.984)--(5.442,2.985)%
  --(5.445,2.986)--(5.448,2.987)--(5.451,2.989)--(5.454,2.990)--(5.457,2.991)--(5.460,2.993)%
  --(5.463,2.994)--(5.466,2.995)--(5.469,2.996)--(5.472,2.998)--(5.475,2.999)--(5.478,3.000)%
  --(5.481,3.002)--(5.484,3.003)--(5.487,3.004)--(5.490,3.005)--(5.493,3.007)--(5.496,3.008)%
  --(5.499,3.009)--(5.502,3.011)--(5.505,3.012)--(5.508,3.013)--(5.511,3.014)--(5.514,3.016)%
  --(5.517,3.017)--(5.520,3.018)--(5.523,3.019)--(5.526,3.021)--(5.529,3.022)--(5.532,3.023)%
  --(5.535,3.025)--(5.538,3.026)--(5.541,3.027)--(5.544,3.029)--(5.547,3.030)--(5.550,3.031)%
  --(5.553,3.032)--(5.556,3.034)--(5.559,3.035)--(5.562,3.036)--(5.565,3.038)--(5.568,3.039)%
  --(5.571,3.040)--(5.574,3.041)--(5.577,3.043)--(5.580,3.044)--(5.583,3.045)--(5.586,3.047)%
  --(5.589,3.048)--(5.591,3.049)--(5.594,3.050)--(5.597,3.052)--(5.600,3.053)--(5.603,3.054)%
  --(5.606,3.056)--(5.609,3.057)--(5.612,3.058)--(5.615,3.060)--(5.618,3.061)--(5.621,3.062)%
  --(5.624,3.063)--(5.627,3.065)--(5.630,3.066)--(5.633,3.067)--(5.636,3.069)--(5.639,3.070)%
  --(5.642,3.071)--(5.645,3.073)--(5.648,3.074)--(5.651,3.075)--(5.654,3.076)--(5.657,3.078)%
  --(5.660,3.079)--(5.663,3.080)--(5.666,3.082)--(5.669,3.083)--(5.672,3.084)--(5.675,3.086)%
  --(5.678,3.087)--(5.681,3.088)--(5.684,3.090)--(5.687,3.091)--(5.690,3.092)--(5.693,3.093)%
  --(5.696,3.095)--(5.699,3.096)--(5.702,3.097)--(5.705,3.099)--(5.708,3.100)--(5.711,3.101)%
  --(5.714,3.103)--(5.717,3.104)--(5.720,3.105)--(5.723,3.107)--(5.726,3.108)--(5.729,3.109)%
  --(5.732,3.111)--(5.735,3.112)--(5.738,3.113)--(5.741,3.114)--(5.744,3.116)--(5.747,3.117)%
  --(5.750,3.118)--(5.753,3.120)--(5.756,3.121)--(5.759,3.122)--(5.762,3.124)--(5.765,3.125)%
  --(5.768,3.126)--(5.771,3.128)--(5.774,3.129)--(5.777,3.130)--(5.780,3.132)--(5.783,3.133)%
  --(5.786,3.134)--(5.789,3.136)--(5.792,3.137)--(5.795,3.138)--(5.798,3.140)--(5.800,3.141)%
  --(5.803,3.142)--(5.806,3.143)--(5.809,3.145)--(5.812,3.146)--(5.815,3.147)--(5.818,3.149)%
  --(5.821,3.150)--(5.824,3.151)--(5.827,3.153)--(5.830,3.154)--(5.833,3.155)--(5.836,3.157)%
  --(5.839,3.158)--(5.842,3.159)--(5.845,3.161)--(5.848,3.162)--(5.851,3.163)--(5.854,3.165)%
  --(5.857,3.166)--(5.860,3.167)--(5.863,3.169)--(5.866,3.170)--(5.869,3.171)--(5.872,3.173)%
  --(5.875,3.174)--(5.878,3.175)--(5.881,3.177)--(5.884,3.178)--(5.887,3.179)--(5.890,3.181)%
  --(5.893,3.182)--(5.896,3.183)--(5.899,3.185)--(5.902,3.186)--(5.905,3.187)--(5.908,3.189)%
  --(5.911,3.190)--(5.914,3.191)--(5.917,3.193)--(5.920,3.194)--(5.923,3.195)--(5.926,3.197)%
  --(5.929,3.198)--(5.932,3.199)--(5.935,3.201)--(5.938,3.202)--(5.941,3.203)--(5.944,3.205)%
  --(5.947,3.206)--(5.950,3.207)--(5.953,3.209)--(5.956,3.210)--(5.959,3.211)--(5.962,3.213)%
  --(5.965,3.214)--(5.968,3.215)--(5.971,3.217)--(5.974,3.218)--(5.977,3.219)--(5.980,3.221)%
  --(5.983,3.222)--(5.986,3.223)--(5.989,3.225)--(5.992,3.226)--(5.995,3.227)--(5.998,3.229)%
  --(6.001,3.230)--(6.004,3.232)--(6.007,3.233)--(6.009,3.234)--(6.012,3.236)--(6.015,3.237)%
  --(6.018,3.238)--(6.021,3.240)--(6.024,3.241)--(6.027,3.242)--(6.030,3.244)--(6.033,3.245)%
  --(6.036,3.246)--(6.039,3.248)--(6.042,3.249)--(6.045,3.250)--(6.048,3.252)--(6.051,3.253)%
  --(6.054,3.254)--(6.057,3.256)--(6.060,3.257)--(6.063,3.258)--(6.066,3.260)--(6.069,3.261)%
  --(6.072,3.263)--(6.075,3.264)--(6.078,3.265)--(6.081,3.267)--(6.084,3.268)--(6.087,3.269)%
  --(6.090,3.271)--(6.093,3.272)--(6.096,3.273)--(6.099,3.275)--(6.102,3.276)--(6.105,3.277)%
  --(6.108,3.279)--(6.111,3.280)--(6.114,3.281)--(6.117,3.283)--(6.120,3.284)--(6.123,3.286)%
  --(6.126,3.287)--(6.129,3.288)--(6.132,3.290)--(6.135,3.291)--(6.138,3.292)--(6.141,3.294)%
  --(6.144,3.295)--(6.147,3.296)--(6.150,3.298)--(6.153,3.299)--(6.156,3.300)--(6.159,3.302)%
  --(6.162,3.303)--(6.165,3.305)--(6.168,3.306)--(6.171,3.307)--(6.174,3.309)--(6.177,3.310)%
  --(6.180,3.311)--(6.183,3.313)--(6.186,3.314)--(6.189,3.315)--(6.192,3.317)--(6.195,3.318)%
  --(6.198,3.320)--(6.201,3.321)--(6.204,3.322)--(6.207,3.324)--(6.210,3.325)--(6.213,3.326)%
  --(6.216,3.328)--(6.218,3.329)--(6.221,3.330)--(6.224,3.332)--(6.227,3.333)--(6.230,3.335)%
  --(6.233,3.336)--(6.236,3.337)--(6.239,3.339)--(6.242,3.340)--(6.245,3.341)--(6.248,3.343)%
  --(6.251,3.344)--(6.254,3.345)--(6.257,3.347)--(6.260,3.348)--(6.263,3.350)--(6.266,3.351)%
  --(6.269,3.352)--(6.272,3.354)--(6.275,3.355)--(6.278,3.356)--(6.281,3.358)--(6.284,3.359)%
  --(6.287,3.361)--(6.290,3.362)--(6.293,3.363)--(6.296,3.365)--(6.299,3.366)--(6.302,3.367)%
  --(6.305,3.369)--(6.308,3.370)--(6.311,3.371)--(6.314,3.373)--(6.317,3.374)--(6.320,3.376)%
  --(6.323,3.377)--(6.326,3.378)--(6.329,3.380)--(6.332,3.381)--(6.335,3.382)--(6.338,3.384)%
  --(6.341,3.385)--(6.344,3.387)--(6.347,3.388)--(6.350,3.389)--(6.353,3.391)--(6.356,3.392)%
  --(6.359,3.393)--(6.362,3.395)--(6.365,3.396)--(6.368,3.398)--(6.371,3.399)--(6.374,3.400)%
  --(6.377,3.402)--(6.380,3.403)--(6.383,3.404)--(6.386,3.406)--(6.389,3.407)--(6.392,3.409)%
  --(6.395,3.410)--(6.398,3.411)--(6.401,3.413)--(6.404,3.414)--(6.407,3.416)--(6.410,3.417)%
  --(6.413,3.418)--(6.416,3.420)--(6.419,3.421)--(6.422,3.422)--(6.425,3.424)--(6.428,3.425)%
  --(6.430,3.427)--(6.433,3.428)--(6.436,3.429)--(6.439,3.431)--(6.442,3.432)--(6.445,3.433)%
  --(6.448,3.435)--(6.451,3.436)--(6.454,3.438)--(6.457,3.439)--(6.460,3.440)--(6.463,3.442)%
  --(6.466,3.443)--(6.469,3.445)--(6.472,3.446)--(6.475,3.447)--(6.478,3.449)--(6.481,3.450)%
  --(6.484,3.451)--(6.487,3.453)--(6.490,3.454)--(6.493,3.456)--(6.496,3.457)--(6.499,3.458)%
  --(6.502,3.460)--(6.505,3.461)--(6.508,3.463)--(6.511,3.464)--(6.514,3.465)--(6.517,3.467)%
  --(6.520,3.468)--(6.523,3.470)--(6.526,3.471)--(6.529,3.472)--(6.532,3.474)--(6.535,3.475)%
  --(6.538,3.476)--(6.541,3.478)--(6.544,3.479)--(6.547,3.481)--(6.550,3.482)--(6.553,3.483)%
  --(6.556,3.485)--(6.559,3.486)--(6.562,3.488)--(6.565,3.489)--(6.568,3.490)--(6.571,3.492)%
  --(6.574,3.493)--(6.577,3.495)--(6.580,3.496)--(6.583,3.497)--(6.586,3.499)--(6.589,3.500)%
  --(6.592,3.502)--(6.595,3.503)--(6.598,3.504)--(6.601,3.506)--(6.604,3.507)--(6.607,3.508)%
  --(6.610,3.510)--(6.613,3.511)--(6.616,3.513)--(6.619,3.514)--(6.622,3.515)--(6.625,3.517)%
  --(6.628,3.518)--(6.631,3.520)--(6.634,3.521)--(6.637,3.522)--(6.639,3.524)--(6.642,3.525)%
  --(6.645,3.527)--(6.648,3.528)--(6.651,3.529)--(6.654,3.531)--(6.657,3.532)--(6.660,3.534)%
  --(6.663,3.535)--(6.666,3.536)--(6.669,3.538)--(6.672,3.539)--(6.675,3.541)--(6.678,3.542)%
  --(6.681,3.543)--(6.684,3.545)--(6.687,3.546)--(6.690,3.548)--(6.693,3.549)--(6.696,3.550)%
  --(6.699,3.552)--(6.702,3.553)--(6.705,3.555)--(6.708,3.556)--(6.711,3.557)--(6.714,3.559)%
  --(6.717,3.560)--(6.720,3.562)--(6.723,3.563)--(6.726,3.564)--(6.729,3.566)--(6.732,3.567)%
  --(6.735,3.569)--(6.738,3.570)--(6.741,3.571)--(6.744,3.573)--(6.747,3.574)--(6.750,3.576)%
  --(6.753,3.577)--(6.756,3.578)--(6.759,3.580)--(6.762,3.581)--(6.765,3.583)--(6.768,3.584)%
  --(6.771,3.585)--(6.774,3.587)--(6.777,3.588)--(6.780,3.590)--(6.783,3.591)--(6.786,3.592)%
  --(6.789,3.594)--(6.792,3.595)--(6.795,3.597)--(6.798,3.598)--(6.801,3.600)--(6.804,3.601)%
  --(6.807,3.602)--(6.810,3.604)--(6.813,3.605)--(6.816,3.607)--(6.819,3.608)--(6.822,3.609)%
  --(6.825,3.611)--(6.828,3.612)--(6.831,3.614)--(6.834,3.615)--(6.837,3.616)--(6.840,3.618)%
  --(6.843,3.619)--(6.846,3.621)--(6.848,3.622)--(6.851,3.623)--(6.854,3.625)--(6.857,3.626)%
  --(6.860,3.628)--(6.863,3.629)--(6.866,3.631)--(6.869,3.632)--(6.872,3.633)--(6.875,3.635)%
  --(6.878,3.636)--(6.881,3.638)--(6.884,3.639)--(6.887,3.640)--(6.890,3.642)--(6.893,3.643)%
  --(6.896,3.645)--(6.899,3.646)--(6.902,3.647)--(6.905,3.649)--(6.908,3.650)--(6.911,3.652)%
  --(6.914,3.653)--(6.917,3.655)--(6.920,3.656)--(6.923,3.657)--(6.926,3.659)--(6.929,3.660)%
  --(6.932,3.662)--(6.935,3.663)--(6.938,3.664)--(6.941,3.666)--(6.944,3.667)--(6.947,3.669)%
  --(6.950,3.670)--(6.953,3.672)--(6.956,3.673)--(6.959,3.674)--(6.962,3.676)--(6.965,3.677)%
  --(6.968,3.679)--(6.971,3.680)--(6.974,3.681)--(6.977,3.683)--(6.980,3.684)--(6.983,3.686)%
  --(6.986,3.687)--(6.989,3.688)--(6.992,3.690)--(6.995,3.691)--(6.998,3.693)--(7.001,3.694)%
  --(7.004,3.696)--(7.007,3.697)--(7.010,3.698)--(7.013,3.700)--(7.016,3.701)--(7.019,3.703)%
  --(7.022,3.704)--(7.025,3.706)--(7.028,3.707)--(7.031,3.708)--(7.034,3.710)--(7.037,3.711)%
  --(7.040,3.713)--(7.043,3.714)--(7.046,3.715)--(7.049,3.717)--(7.052,3.718)--(7.055,3.720)%
  --(7.057,3.721)--(7.060,3.723)--(7.063,3.724)--(7.066,3.725)--(7.069,3.727)--(7.072,3.728)%
  --(7.075,3.730)--(7.078,3.731)--(7.081,3.733)--(7.084,3.734)--(7.087,3.735)--(7.090,3.737)%
  --(7.093,3.738)--(7.096,3.740)--(7.099,3.741)--(7.102,3.742)--(7.105,3.744)--(7.108,3.745)%
  --(7.111,3.747)--(7.114,3.748)--(7.117,3.750)--(7.120,3.751)--(7.123,3.752)--(7.126,3.754)%
  --(7.129,3.755)--(7.132,3.757)--(7.135,3.758)--(7.138,3.760)--(7.141,3.761)--(7.144,3.762)%
  --(7.147,3.764)--(7.150,3.765)--(7.153,3.767)--(7.156,3.768)--(7.159,3.770)--(7.162,3.771)%
  --(7.165,3.772)--(7.168,3.774)--(7.171,3.775)--(7.174,3.777)--(7.177,3.778)--(7.180,3.780)%
  --(7.183,3.781)--(7.186,3.782)--(7.189,3.784)--(7.192,3.785)--(7.195,3.787)--(7.198,3.788)%
  --(7.201,3.790)--(7.204,3.791)--(7.207,3.792)--(7.210,3.794)--(7.213,3.795)--(7.216,3.797)%
  --(7.219,3.798)--(7.222,3.800)--(7.225,3.801)--(7.228,3.802)--(7.231,3.804)--(7.234,3.805)%
  --(7.237,3.807)--(7.240,3.808)--(7.243,3.810)--(7.246,3.811)--(7.249,3.812)--(7.252,3.814)%
  --(7.255,3.815)--(7.258,3.817)--(7.261,3.818)--(7.264,3.820)--(7.266,3.821)--(7.269,3.822)%
  --(7.272,3.824)--(7.275,3.825)--(7.278,3.827)--(7.281,3.828)--(7.284,3.830)--(7.287,3.831)%
  --(7.290,3.832)--(7.293,3.834)--(7.296,3.835)--(7.299,3.837)--(7.302,3.838)--(7.305,3.840)%
  --(7.308,3.841)--(7.311,3.842)--(7.314,3.844)--(7.317,3.845)--(7.320,3.847)--(7.323,3.848)%
  --(7.326,3.850)--(7.329,3.851)--(7.332,3.852)--(7.335,3.854)--(7.338,3.855)--(7.341,3.857)%
  --(7.344,3.858)--(7.347,3.860)--(7.350,3.861)--(7.353,3.863)--(7.356,3.864)--(7.359,3.865)%
  --(7.362,3.867)--(7.365,3.868)--(7.368,3.870)--(7.371,3.871)--(7.374,3.873)--(7.377,3.874)%
  --(7.380,3.875)--(7.383,3.877)--(7.386,3.878)--(7.389,3.880)--(7.392,3.881)--(7.395,3.883)%
  --(7.398,3.884)--(7.401,3.886)--(7.404,3.887)--(7.407,3.888)--(7.410,3.890)--(7.413,3.891)%
  --(7.416,3.893)--(7.419,3.894)--(7.422,3.896)--(7.425,3.897)--(7.428,3.898)--(7.431,3.900)%
  --(7.434,3.901)--(7.437,3.903)--(7.440,3.904)--(7.443,3.906)--(7.446,3.907)--(7.449,3.909)%
  --(7.452,3.910)--(7.455,3.911)--(7.458,3.913)--(7.461,3.914)--(7.464,3.916)--(7.467,3.917)%
  --(7.470,3.919)--(7.473,3.920)--(7.476,3.921)--(7.478,3.923)--(7.481,3.924)--(7.484,3.926)%
  --(7.487,3.927)--(7.490,3.929)--(7.493,3.930)--(7.496,3.932)--(7.499,3.933)--(7.502,3.934)%
  --(7.505,3.936)--(7.508,3.937)--(7.511,3.939)--(7.514,3.940)--(7.517,3.942)--(7.520,3.943)%
  --(7.523,3.945)--(7.526,3.946)--(7.529,3.947)--(7.532,3.949)--(7.535,3.950)--(7.538,3.952)%
  --(7.541,3.953)--(7.544,3.955)--(7.547,3.956)--(7.550,3.958)--(7.553,3.959)--(7.556,3.960)%
  --(7.559,3.962)--(7.562,3.963)--(7.565,3.965)--(7.568,3.966)--(7.571,3.968)--(7.574,3.969)%
  --(7.577,3.970)--(7.580,3.972)--(7.583,3.973)--(7.586,3.975)--(7.589,3.976)--(7.592,3.978)%
  --(7.595,3.979)--(7.598,3.981)--(7.601,3.982)--(7.604,3.983)--(7.607,3.985)--(7.610,3.986)%
  --(7.613,3.988)--(7.616,3.989)--(7.619,3.991)--(7.622,3.992)--(7.625,3.994)--(7.628,3.995)%
  --(7.631,3.997)--(7.634,3.998)--(7.637,3.999)--(7.640,4.001)--(7.643,4.002)--(7.646,4.004)%
  --(7.649,4.005)--(7.652,4.007)--(7.655,4.008)--(7.658,4.010)--(7.661,4.011)--(7.664,4.012)%
  --(7.667,4.014)--(7.670,4.015)--(7.673,4.017)--(7.676,4.018)--(7.679,4.020)--(7.682,4.021)%
  --(7.685,4.023)--(7.687,4.024)--(7.690,4.025)--(7.693,4.027)--(7.696,4.028)--(7.699,4.030)%
  --(7.702,4.031)--(7.705,4.033)--(7.708,4.034)--(7.711,4.036)--(7.714,4.037)--(7.717,4.038)%
  --(7.720,4.040)--(7.723,4.041)--(7.726,4.043)--(7.729,4.044)--(7.732,4.046)--(7.735,4.047)%
  --(7.738,4.049)--(7.741,4.050)--(7.744,4.052)--(7.747,4.053)--(7.750,4.054)--(7.753,4.056)%
  --(7.756,4.057)--(7.759,4.059)--(7.762,4.060)--(7.765,4.062)--(7.768,4.063)--(7.771,4.065)%
  --(7.774,4.066)--(7.777,4.067)--(7.780,4.069)--(7.783,4.070)--(7.786,4.072)--(7.789,4.073)%
  --(7.792,4.075)--(7.795,4.076)--(7.798,4.078)--(7.801,4.079)--(7.804,4.081)--(7.807,4.082)%
  --(7.810,4.083)--(7.813,4.085)--(7.816,4.086)--(7.819,4.088)--(7.822,4.089)--(7.825,4.091)%
  --(7.828,4.092)--(7.831,4.094)--(7.834,4.095)--(7.837,4.097)--(7.840,4.098)--(7.843,4.099)%
  --(7.846,4.101)--(7.849,4.102)--(7.852,4.104)--(7.855,4.105)--(7.858,4.107)--(7.861,4.108)%
  --(7.864,4.110)--(7.867,4.111)--(7.870,4.113)--(7.873,4.114)--(7.876,4.115)--(7.879,4.117)%
  --(7.882,4.118)--(7.885,4.120)--(7.888,4.121)--(7.891,4.123)--(7.894,4.124)--(7.896,4.126)%
  --(7.899,4.127)--(7.902,4.129)--(7.905,4.130)--(7.908,4.131)--(7.911,4.133)--(7.914,4.134)%
  --(7.917,4.136)--(7.920,4.137)--(7.923,4.139)--(7.926,4.140)--(7.929,4.142)--(7.932,4.143)%
  --(7.935,4.145)--(7.938,4.146)--(7.941,4.147)--(7.944,4.149)--(7.947,4.150)--(7.950,4.152)%
  --(7.953,4.153)--(7.956,4.155)--(7.959,4.156)--(7.962,4.158)--(7.965,4.159)--(7.968,4.161)%
  --(7.971,4.162)--(7.974,4.163)--(7.977,4.165)--(7.980,4.166)--(7.983,4.168)--(7.986,4.169)%
  --(7.989,4.171)--(7.992,4.172)--(7.995,4.174)--(7.998,4.175)--(8.001,4.177)--(8.004,4.178)%
  --(8.007,4.179)--(8.010,4.181)--(8.013,4.182)--(8.016,4.184)--(8.019,4.185)--(8.022,4.187)%
  --(8.025,4.188)--(8.028,4.190)--(8.031,4.191)--(8.034,4.193)--(8.037,4.194)--(8.040,4.196)%
  --(8.043,4.197)--(8.046,4.198)--(8.049,4.200)--(8.052,4.201)--(8.055,4.203)--(8.058,4.204)%
  --(8.061,4.206)--(8.064,4.207)--(8.067,4.209)--(8.070,4.210)--(8.073,4.212)--(8.076,4.213)%
  --(8.079,4.215)--(8.082,4.216)--(8.085,4.217)--(8.088,4.219)--(8.091,4.220)--(8.094,4.222)%
  --(8.097,4.223)--(8.100,4.225)--(8.103,4.226)--(8.105,4.228)--(8.108,4.229)--(8.111,4.231)%
  --(8.114,4.232)--(8.117,4.234)--(8.120,4.235)--(8.123,4.236)--(8.126,4.238)--(8.129,4.239)%
  --(8.132,4.241)--(8.135,4.242)--(8.138,4.244)--(8.141,4.245)--(8.144,4.247)--(8.147,4.248)%
  --(8.150,4.250)--(8.153,4.251)--(8.156,4.253)--(8.159,4.254)--(8.162,4.255)--(8.165,4.257)%
  --(8.168,4.258)--(8.171,4.260)--(8.174,4.261)--(8.177,4.263)--(8.180,4.264)--(8.183,4.266)%
  --(8.186,4.267)--(8.189,4.269)--(8.192,4.270)--(8.195,4.272)--(8.198,4.273)--(8.201,4.274)%
  --(8.204,4.276)--(8.207,4.277)--(8.210,4.279)--(8.213,4.280)--(8.216,4.282)--(8.219,4.283)%
  --(8.222,4.285)--(8.225,4.286)--(8.228,4.288)--(8.231,4.289)--(8.234,4.291)--(8.237,4.292)%
  --(8.240,4.293)--(8.243,4.295)--(8.246,4.296)--(8.249,4.298)--(8.252,4.299)--(8.255,4.301)%
  --(8.258,4.302)--(8.261,4.304)--(8.264,4.305)--(8.267,4.307)--(8.270,4.308)--(8.273,4.310)%
  --(8.276,4.311)--(8.279,4.313)--(8.282,4.314)--(8.285,4.315)--(8.288,4.317)--(8.291,4.318)%
  --(8.294,4.320)--(8.297,4.321)--(8.300,4.323)--(8.303,4.324)--(8.306,4.326)--(8.309,4.327)%
  --(8.312,4.329)--(8.314,4.330)--(8.317,4.332)--(8.320,4.333)--(8.323,4.335)--(8.326,4.336)%
  --(8.329,4.337)--(8.332,4.339)--(8.335,4.340)--(8.338,4.342)--(8.341,4.343)--(8.344,4.345)%
  --(8.347,4.346)--(8.350,4.348)--(8.353,4.349)--(8.356,4.351)--(8.359,4.352)--(8.362,4.354)%
  --(8.365,4.355)--(8.368,4.357)--(8.371,4.358)--(8.374,4.359)--(8.377,4.361)--(8.380,4.362)%
  --(8.383,4.364)--(8.386,4.365)--(8.389,4.367)--(8.392,4.368)--(8.395,4.370)--(8.398,4.371)%
  --(8.401,4.373)--(8.404,4.374)--(8.407,4.376)--(8.410,4.377)--(8.413,4.379)--(8.416,4.380)%
  --(8.419,4.382)--(8.422,4.383)--(8.425,4.384)--(8.428,4.386)--(8.431,4.387)--(8.434,4.389)%
  --(8.437,4.390)--(8.440,4.392)--(8.443,4.393)--(8.446,4.395)--(8.449,4.396)--(8.452,4.398)%
  --(8.455,4.399)--(8.458,4.401)--(8.461,4.402)--(8.464,4.404)--(8.467,4.405)--(8.470,4.407)%
  --(8.473,4.408)--(8.476,4.409)--(8.479,4.411)--(8.482,4.412)--(8.485,4.414)--(8.488,4.415)%
  --(8.491,4.417)--(8.494,4.418)--(8.497,4.420)--(8.500,4.421)--(8.503,4.423)--(8.506,4.424)%
  --(8.509,4.426)--(8.512,4.427)--(8.515,4.429)--(8.518,4.430)--(8.521,4.432)--(8.523,4.433)%
  --(8.526,4.434)--(8.529,4.436)--(8.532,4.437)--(8.535,4.439)--(8.538,4.440)--(8.541,4.442)%
  --(8.544,4.443)--(8.547,4.445)--(8.550,4.446)--(8.553,4.448)--(8.556,4.449)--(8.559,4.451)%
  --(8.562,4.452)--(8.565,4.454)--(8.568,4.455)--(8.571,4.457)--(8.574,4.458)--(8.577,4.459)%
  --(8.580,4.461)--(8.583,4.462)--(8.586,4.464)--(8.589,4.465)--(8.592,4.467)--(8.595,4.468)%
  --(8.598,4.470)--(8.601,4.471)--(8.604,4.473)--(8.607,4.474)--(8.610,4.476)--(8.613,4.477)%
  --(8.616,4.479)--(8.619,4.480)--(8.622,4.482)--(8.625,4.483)--(8.628,4.485)--(8.631,4.486)%
  --(8.634,4.487)--(8.637,4.489)--(8.640,4.490)--(8.643,4.492)--(8.646,4.493)--(8.649,4.495)%
  --(8.652,4.496)--(8.655,4.498)--(8.658,4.499)--(8.661,4.501)--(8.664,4.502)--(8.667,4.504)%
  --(8.670,4.505)--(8.673,4.507)--(8.676,4.508)--(8.679,4.510)--(8.682,4.511)--(8.685,4.513)%
  --(8.688,4.514)--(8.691,4.516)--(8.694,4.517)--(8.697,4.518)--(8.700,4.520)--(8.703,4.521)%
  --(8.706,4.523)--(8.709,4.524)--(8.712,4.526)--(8.715,4.527)--(8.718,4.529)--(8.721,4.530)%
  --(8.724,4.532)--(8.727,4.533)--(8.730,4.535)--(8.733,4.536)--(8.735,4.538)--(8.738,4.539)%
  --(8.741,4.541)--(8.744,4.542)--(8.747,4.544)--(8.750,4.545)--(8.753,4.546)--(8.756,4.548)%
  --(8.759,4.549)--(8.762,4.551)--(8.765,4.552)--(8.768,4.554)--(8.771,4.555)--(8.774,4.557)%
  --(8.777,4.558)--(8.780,4.560)--(8.783,4.561)--(8.786,4.563)--(8.789,4.564)--(8.792,4.566)%
  --(8.795,4.567)--(8.798,4.569)--(8.801,4.570)--(8.804,4.572)--(8.807,4.573)--(8.810,4.575)%
  --(8.813,4.576)--(8.816,4.578)--(8.819,4.579)--(8.822,4.580)--(8.825,4.582)--(8.828,4.583)%
  --(8.831,4.585)--(8.834,4.586)--(8.837,4.588)--(8.840,4.589)--(8.843,4.591)--(8.846,4.592)%
  --(8.849,4.594)--(8.852,4.595)--(8.855,4.597)--(8.858,4.598)--(8.861,4.600)--(8.864,4.601)%
  --(8.867,4.603)--(8.870,4.604)--(8.873,4.606)--(8.876,4.607)--(8.879,4.609)--(8.882,4.610)%
  --(8.885,4.612)--(8.888,4.613)--(8.891,4.614)--(8.894,4.616)--(8.897,4.617)--(8.900,4.619)%
  --(8.903,4.620)--(8.906,4.622)--(8.909,4.623)--(8.912,4.625)--(8.915,4.626)--(8.918,4.628)%
  --(8.921,4.629)--(8.924,4.631)--(8.927,4.632)--(8.930,4.634)--(8.933,4.635)--(8.936,4.637)%
  --(8.939,4.638)--(8.942,4.640)--(8.944,4.641)--(8.947,4.643)--(8.950,4.644)--(8.953,4.646)%
  --(8.956,4.647)--(8.959,4.649)--(8.962,4.650)--(8.965,4.651)--(8.968,4.653)--(8.971,4.654)%
  --(8.974,4.656)--(8.977,4.657)--(8.980,4.659)--(8.983,4.660)--(8.986,4.662)--(8.989,4.663)%
  --(8.992,4.665)--(8.995,4.666)--(8.998,4.668)--(9.001,4.669)--(9.004,4.671)--(9.007,4.672)%
  --(9.010,4.674)--(9.013,4.675)--(9.016,4.677)--(9.019,4.678)--(9.022,4.680)--(9.025,4.681)%
  --(9.028,4.683)--(9.031,4.684)--(9.034,4.686)--(9.037,4.687)--(9.040,4.689)--(9.043,4.690)%
  --(9.046,4.691)--(9.049,4.693)--(9.052,4.694)--(9.055,4.696)--(9.058,4.697)--(9.061,4.699)%
  --(9.064,4.700)--(9.067,4.702)--(9.070,4.703)--(9.073,4.705)--(9.076,4.706)--(9.079,4.708)%
  --(9.082,4.709)--(9.085,4.711)--(9.088,4.712)--(9.091,4.714)--(9.094,4.715)--(9.097,4.717)%
  --(9.100,4.718)--(9.103,4.720)--(9.106,4.721)--(9.109,4.723)--(9.112,4.724)--(9.115,4.726)%
  --(9.118,4.727)--(9.121,4.729)--(9.124,4.730)--(9.127,4.732)--(9.130,4.733)--(9.133,4.734)%
  --(9.136,4.736)--(9.139,4.737)--(9.142,4.739)--(9.145,4.740)--(9.148,4.742)--(9.151,4.743)%
  --(9.153,4.745)--(9.156,4.746)--(9.159,4.748)--(9.162,4.749)--(9.165,4.751)--(9.168,4.752)%
  --(9.171,4.754)--(9.174,4.755)--(9.177,4.757)--(9.180,4.758)--(9.183,4.760)--(9.186,4.761)%
  --(9.189,4.763)--(9.192,4.764)--(9.195,4.766)--(9.198,4.767)--(9.201,4.769)--(9.204,4.770)%
  --(9.207,4.772)--(9.210,4.773)--(9.213,4.775)--(9.216,4.776)--(9.219,4.778)--(9.222,4.779)%
  --(9.225,4.780)--(9.228,4.782)--(9.231,4.783)--(9.234,4.785)--(9.237,4.786)--(9.240,4.788)%
  --(9.243,4.789)--(9.246,4.791)--(9.249,4.792)--(9.252,4.794)--(9.255,4.795)--(9.258,4.797)%
  --(9.261,4.798)--(9.264,4.800)--(9.267,4.801)--(9.270,4.803)--(9.273,4.804)--(9.276,4.806)%
  --(9.279,4.807)--(9.282,4.809)--(9.285,4.810)--(9.288,4.812)--(9.291,4.813)--(9.294,4.815)%
  --(9.297,4.816)--(9.300,4.818)--(9.303,4.819)--(9.306,4.821)--(9.309,4.822)--(9.312,4.824)%
  --(9.315,4.825)--(9.318,4.827)--(9.321,4.828)--(9.324,4.830)--(9.327,4.831)--(9.330,4.832)%
  --(9.333,4.834)--(9.336,4.835)--(9.339,4.837)--(9.342,4.838)--(9.345,4.840)--(9.348,4.841)%
  --(9.351,4.843)--(9.354,4.844)--(9.357,4.846)--(9.360,4.847)--(9.362,4.849)--(9.365,4.850)%
  --(9.368,4.852)--(9.371,4.853)--(9.374,4.855)--(9.377,4.856)--(9.380,4.858)--(9.383,4.859)%
  --(9.386,4.861)--(9.389,4.862)--(9.392,4.864)--(9.395,4.865)--(9.398,4.867)--(9.401,4.868)%
  --(9.404,4.870)--(9.407,4.871)--(9.410,4.873)--(9.413,4.874)--(9.416,4.876)--(9.419,4.877)%
  --(9.422,4.879)--(9.425,4.880)--(9.428,4.882)--(9.431,4.883)--(9.434,4.885)--(9.437,4.886)%
  --(9.440,4.888)--(9.443,4.889)--(9.446,4.891)--(9.449,4.892)--(9.452,4.893)--(9.455,4.895)%
  --(9.458,4.896)--(9.461,4.898)--(9.464,4.899)--(9.467,4.901)--(9.470,4.902)--(9.473,4.904)%
  --(9.476,4.905)--(9.479,4.907)--(9.482,4.908)--(9.485,4.910)--(9.488,4.911)--(9.491,4.913)%
  --(9.494,4.914)--(9.497,4.916)--(9.500,4.917)--(9.503,4.919)--(9.506,4.920)--(9.509,4.922)%
  --(9.512,4.923)--(9.515,4.925)--(9.518,4.926)--(9.521,4.928)--(9.524,4.929)--(9.527,4.931)%
  --(9.530,4.932)--(9.533,4.934)--(9.536,4.935)--(9.539,4.937)--(9.542,4.938)--(9.545,4.940)%
  --(9.548,4.941)--(9.551,4.943)--(9.554,4.944)--(9.557,4.946)--(9.560,4.947)--(9.563,4.949)%
  --(9.566,4.950)--(9.569,4.952)--(9.571,4.953)--(9.574,4.955)--(9.577,4.956)--(9.580,4.958)%
  --(9.583,4.959)--(9.586,4.961)--(9.589,4.962)--(9.592,4.963)--(9.595,4.965)--(9.598,4.966)%
  --(9.601,4.968)--(9.604,4.969)--(9.607,4.971)--(9.610,4.972)--(9.613,4.974)--(9.616,4.975)%
  --(9.619,4.977)--(9.622,4.978)--(9.625,4.980)--(9.628,4.981)--(9.631,4.983)--(9.634,4.984)%
  --(9.637,4.986)--(9.640,4.987)--(9.643,4.989)--(9.646,4.990)--(9.649,4.992)--(9.652,4.993)%
  --(9.655,4.995)--(9.658,4.996)--(9.661,4.998)--(9.664,4.999)--(9.667,5.001)--(9.670,5.002)%
  --(9.673,5.004)--(9.676,5.005)--(9.679,5.007)--(9.682,5.008)--(9.685,5.010)--(9.688,5.011)%
  --(9.691,5.013)--(9.694,5.014)--(9.697,5.016)--(9.700,5.017)--(9.703,5.019)--(9.706,5.020)%
  --(9.709,5.022)--(9.712,5.023)--(9.715,5.025)--(9.718,5.026)--(9.721,5.028)--(9.724,5.029)%
  --(9.727,5.031)--(9.730,5.032)--(9.733,5.034)--(9.736,5.035)--(9.739,5.037)--(9.742,5.038)%
  --(9.745,5.040)--(9.748,5.041)--(9.751,5.043)--(9.754,5.044)--(9.757,5.046)--(9.760,5.047)%
  --(9.763,5.049)--(9.766,5.050)--(9.769,5.051)--(9.772,5.053)--(9.775,5.054)--(9.778,5.056)%
  --(9.780,5.057)--(9.783,5.059)--(9.786,5.060)--(9.789,5.062)--(9.792,5.063)--(9.795,5.065)%
  --(9.798,5.066)--(9.801,5.068)--(9.804,5.069)--(9.807,5.071)--(9.810,5.072)--(9.813,5.074)%
  --(9.816,5.075)--(9.819,5.077)--(9.822,5.078)--(9.825,5.080)--(9.828,5.081)--(9.831,5.083)%
  --(9.834,5.084)--(9.837,5.086)--(9.840,5.087)--(9.843,5.089)--(9.846,5.090)--(9.849,5.092)%
  --(9.852,5.093)--(9.855,5.095)--(9.858,5.096)--(9.861,5.098)--(9.864,5.099)--(9.867,5.101)%
  --(9.870,5.102)--(9.873,5.104)--(9.876,5.105)--(9.879,5.107)--(9.882,5.108)--(9.885,5.110)%
  --(9.888,5.111)--(9.891,5.113)--(9.894,5.114)--(9.897,5.116)--(9.900,5.117)--(9.903,5.119)%
  --(9.906,5.120)--(9.909,5.122)--(9.912,5.123)--(9.915,5.125)--(9.918,5.126)--(9.921,5.128)%
  --(9.924,5.129)--(9.927,5.131)--(9.930,5.132)--(9.933,5.134)--(9.936,5.135)--(9.939,5.137)%
  --(9.942,5.138)--(9.945,5.140)--(9.948,5.141)--(9.951,5.143)--(9.954,5.144)--(9.957,5.146)%
  --(9.960,5.147)--(9.963,5.149)--(9.966,5.150)--(9.969,5.152)--(9.972,5.153)--(9.975,5.155)%
  --(9.978,5.156)--(9.981,5.158)--(9.984,5.159)--(9.987,5.161)--(9.990,5.162)--(9.992,5.164)%
  --(9.995,5.165)--(9.998,5.167)--(10.001,5.168)--(10.004,5.170)--(10.007,5.171)--(10.010,5.173)%
  --(10.013,5.174)--(10.016,5.176)--(10.019,5.177)--(10.022,5.179)--(10.025,5.180)--(10.028,5.181)%
  --(10.031,5.183)--(10.034,5.184)--(10.037,5.186)--(10.040,5.187)--(10.043,5.189)--(10.046,5.190)%
  --(10.049,5.192)--(10.052,5.193)--(10.055,5.195)--(10.058,5.196)--(10.061,5.198)--(10.064,5.199)%
  --(10.067,5.201)--(10.070,5.202)--(10.073,5.204)--(10.076,5.205)--(10.079,5.207)--(10.082,5.208)%
  --(10.085,5.210)--(10.088,5.211)--(10.091,5.213)--(10.094,5.214)--(10.097,5.216)--(10.100,5.217)%
  --(10.103,5.219)--(10.106,5.220)--(10.109,5.222)--(10.112,5.223)--(10.115,5.225)--(10.118,5.226)%
  --(10.121,5.228)--(10.124,5.229)--(10.127,5.231)--(10.130,5.232)--(10.133,5.234)--(10.136,5.235)%
  --(10.139,5.237)--(10.142,5.238)--(10.145,5.240)--(10.148,5.241)--(10.151,5.243)--(10.154,5.244)%
  --(10.157,5.246)--(10.160,5.247)--(10.163,5.249)--(10.166,5.250)--(10.169,5.252)--(10.172,5.253)%
  --(10.175,5.255)--(10.178,5.256)--(10.181,5.258)--(10.184,5.259)--(10.187,5.261)--(10.190,5.262)%
  --(10.193,5.264)--(10.196,5.265)--(10.199,5.267)--(10.201,5.268)--(10.204,5.270)--(10.207,5.271)%
  --(10.210,5.273)--(10.213,5.274)--(10.216,5.276)--(10.219,5.277)--(10.222,5.279)--(10.225,5.280)%
  --(10.228,5.282)--(10.231,5.283)--(10.234,5.285)--(10.237,5.286)--(10.240,5.288)--(10.243,5.289)%
  --(10.246,5.291)--(10.249,5.292)--(10.252,5.294)--(10.255,5.295)--(10.258,5.297)--(10.261,5.298)%
  --(10.264,5.300)--(10.267,5.301)--(10.270,5.303)--(10.273,5.304)--(10.276,5.306)--(10.279,5.307)%
  --(10.282,5.309)--(10.285,5.310)--(10.288,5.312)--(10.291,5.313)--(10.294,5.315)--(10.297,5.316)%
  --(10.300,5.318)--(10.303,5.319)--(10.306,5.321)--(10.309,5.322)--(10.312,5.324)--(10.315,5.325)%
  --(10.318,5.327)--(10.321,5.328)--(10.324,5.330)--(10.327,5.331)--(10.330,5.333)--(10.333,5.334)%
  --(10.336,5.336)--(10.339,5.337)--(10.342,5.339)--(10.345,5.340)--(10.348,5.342)--(10.351,5.343)%
  --(10.354,5.345)--(10.357,5.346)--(10.360,5.348)--(10.363,5.349)--(10.366,5.351)--(10.369,5.352)%
  --(10.372,5.354)--(10.375,5.355)--(10.378,5.357)--(10.381,5.358)--(10.384,5.360)--(10.387,5.361)%
  --(10.390,5.363)--(10.393,5.364)--(10.396,5.366)--(10.399,5.367)--(10.402,5.369)--(10.405,5.370)%
  --(10.408,5.372)--(10.410,5.373)--(10.413,5.375)--(10.416,5.376)--(10.419,5.378)--(10.422,5.379)%
  --(10.425,5.381)--(10.428,5.382)--(10.431,5.384)--(10.434,5.385)--(10.437,5.387)--(10.440,5.388)%
  --(10.443,5.390)--(10.446,5.391)--(10.449,5.393)--(10.452,5.394)--(10.455,5.396)--(10.458,5.397)%
  --(10.461,5.399)--(10.464,5.400)--(10.467,5.402)--(10.470,5.403)--(10.473,5.405)--(10.476,5.406)%
  --(10.479,5.408)--(10.482,5.409)--(10.485,5.411)--(10.488,5.412)--(10.491,5.414)--(10.494,5.415)%
  --(10.497,5.417)--(10.500,5.418)--(10.503,5.420)--(10.506,5.421)--(10.509,5.423)--(10.512,5.424)%
  --(10.515,5.426)--(10.518,5.427)--(10.521,5.429)--(10.524,5.430)--(10.527,5.432)--(10.530,5.433)%
  --(10.533,5.435)--(10.536,5.436)--(10.539,5.438)--(10.542,5.439)--(10.545,5.441)--(10.548,5.442)%
  --(10.551,5.444)--(10.554,5.445)--(10.557,5.447)--(10.560,5.448)--(10.563,5.450)--(10.566,5.451)%
  --(10.569,5.453)--(10.572,5.454)--(10.575,5.456)--(10.578,5.457)--(10.581,5.459)--(10.584,5.460)%
  --(10.587,5.462)--(10.590,5.463)--(10.593,5.465)--(10.596,5.466)--(10.599,5.468)--(10.602,5.469)%
  --(10.605,5.471)--(10.608,5.472)--(10.611,5.474)--(10.614,5.475)--(10.617,5.477)--(10.619,5.478)%
  --(10.622,5.480)--(10.625,5.481)--(10.628,5.483)--(10.631,5.484)--(10.634,5.486)--(10.637,5.487)%
  --(10.640,5.489)--(10.643,5.490)--(10.646,5.492)--(10.649,5.493)--(10.652,5.495)--(10.655,5.496)%
  --(10.658,5.498)--(10.661,5.499)--(10.664,5.501)--(10.667,5.502)--(10.670,5.504)--(10.673,5.505)%
  --(10.676,5.507)--(10.679,5.508)--(10.682,5.510)--(10.685,5.511)--(10.688,5.513)--(10.691,5.514)%
  --(10.694,5.516)--(10.697,5.517)--(10.700,5.519)--(10.703,5.520)--(10.706,5.522)--(10.709,5.523)%
  --(10.712,5.525)--(10.715,5.526)--(10.718,5.528)--(10.721,5.529)--(10.724,5.531)--(10.727,5.532)%
  --(10.730,5.534)--(10.733,5.535)--(10.736,5.537)--(10.739,5.538)--(10.742,5.540)--(10.745,5.541)%
  --(10.748,5.543)--(10.751,5.544)--(10.754,5.546)--(10.757,5.547)--(10.760,5.549)--(10.763,5.550)%
  --(10.766,5.552)--(10.769,5.553)--(10.772,5.555)--(10.775,5.556)--(10.778,5.558)--(10.781,5.559)%
  --(10.784,5.561)--(10.787,5.562)--(10.790,5.564)--(10.793,5.565)--(10.796,5.567)--(10.799,5.568)%
  --(10.802,5.570)--(10.805,5.571)--(10.808,5.573)--(10.811,5.574)--(10.814,5.576)--(10.817,5.577)%
  --(10.820,5.579)--(10.823,5.580)--(10.826,5.582)--(10.828,5.583)--(10.831,5.585)--(10.834,5.586)%
  --(10.837,5.588)--(10.840,5.589)--(10.843,5.591)--(10.846,5.592)--(10.849,5.594)--(10.852,5.595)%
  --(10.855,5.597)--(10.858,5.598)--(10.861,5.600)--(10.864,5.601)--(10.867,5.603)--(10.870,5.604)%
  --(10.873,5.606)--(10.876,5.607)--(10.879,5.609)--(10.882,5.610)--(10.885,5.612)--(10.888,5.613)%
  --(10.891,5.615)--(10.894,5.616)--(10.897,5.618)--(10.900,5.620)--(10.903,5.621)--(10.906,5.623)%
  --(10.909,5.624)--(10.912,5.626)--(10.915,5.627)--(10.918,5.629)--(10.921,5.630)--(10.924,5.632)%
  --(10.927,5.633)--(10.930,5.635)--(10.933,5.636)--(10.936,5.638)--(10.939,5.639)--(10.942,5.641)%
  --(10.945,5.642)--(10.948,5.644)--(10.951,5.645)--(10.954,5.647)--(10.957,5.648)--(10.960,5.650)%
  --(10.963,5.651)--(10.966,5.653)--(10.969,5.654)--(10.972,5.656)--(10.975,5.657)--(10.978,5.659)%
  --(10.981,5.660)--(10.984,5.662)--(10.987,5.663)--(10.990,5.665)--(10.993,5.666)--(10.996,5.668)%
  --(10.999,5.669)--(11.002,5.671)--(11.005,5.672)--(11.008,5.674)--(11.011,5.675)--(11.014,5.677)%
  --(11.017,5.678)--(11.020,5.680)--(11.023,5.681)--(11.026,5.683)--(11.029,5.684)--(11.032,5.686)%
  --(11.035,5.687)--(11.037,5.689)--(11.040,5.690)--(11.043,5.692)--(11.046,5.693)--(11.049,5.695)%
  --(11.052,5.696)--(11.055,5.698)--(11.058,5.699)--(11.061,5.701)--(11.064,5.702)--(11.067,5.704)%
  --(11.070,5.705)--(11.073,5.707)--(11.076,5.708)--(11.079,5.710)--(11.082,5.711)--(11.085,5.713)%
  --(11.088,5.714)--(11.091,5.716)--(11.094,5.717)--(11.097,5.719)--(11.100,5.720)--(11.103,5.722)%
  --(11.106,5.723)--(11.109,5.725)--(11.112,5.726)--(11.115,5.728)--(11.118,5.729)--(11.121,5.731)%
  --(11.124,5.732)--(11.127,5.734)--(11.130,5.735)--(11.133,5.737)--(11.136,5.738)--(11.139,5.740)%
  --(11.142,5.741)--(11.145,5.743)--(11.148,5.744)--(11.151,5.746)--(11.154,5.747)--(11.157,5.749)%
  --(11.160,5.750)--(11.163,5.752)--(11.166,5.753)--(11.169,5.755)--(11.172,5.756)--(11.175,5.758)%
  --(11.178,5.759)--(11.181,5.761)--(11.184,5.762)--(11.187,5.764)--(11.190,5.765)--(11.193,5.767)%
  --(11.196,5.768)--(11.199,5.770)--(11.202,5.772)--(11.205,5.773)--(11.208,5.775)--(11.211,5.776)%
  --(11.214,5.778)--(11.217,5.779)--(11.220,5.781)--(11.223,5.782)--(11.226,5.784)--(11.229,5.785)%
  --(11.232,5.787)--(11.235,5.788)--(11.238,5.790)--(11.241,5.791)--(11.244,5.793)--(11.247,5.794)%
  --(11.249,5.796)--(11.252,5.797)--(11.255,5.799)--(11.258,5.800)--(11.261,5.802)--(11.264,5.803)%
  --(11.267,5.805)--(11.270,5.806)--(11.273,5.808)--(11.276,5.809)--(11.279,5.811)--(11.282,5.812)%
  --(11.285,5.814)--(11.288,5.815)--(11.291,5.817)--(11.294,5.818)--(11.297,5.820)--(11.300,5.821)%
  --(11.303,5.823)--(11.306,5.824)--(11.309,5.826)--(11.312,5.827)--(11.315,5.829)--(11.318,5.830)%
  --(11.321,5.832)--(11.324,5.833)--(11.327,5.835)--(11.330,5.836)--(11.333,5.838)--(11.336,5.839)%
  --(11.339,5.841)--(11.342,5.842)--(11.345,5.844)--(11.348,5.845)--(11.351,5.847)--(11.354,5.848)%
  --(11.357,5.850)--(11.360,5.851)--(11.363,5.853)--(11.366,5.854)--(11.369,5.856)--(11.372,5.857)%
  --(11.375,5.859)--(11.378,5.860)--(11.381,5.862)--(11.384,5.863)--(11.387,5.865)--(11.390,5.866)%
  --(11.393,5.868)--(11.396,5.869)--(11.399,5.871)--(11.402,5.872)--(11.405,5.874)--(11.408,5.875)%
  --(11.411,5.877)--(11.414,5.878)--(11.417,5.880)--(11.420,5.881)--(11.423,5.883)--(11.426,5.885)%
  --(11.429,5.886)--(11.432,5.888)--(11.435,5.889)--(11.438,5.891)--(11.441,5.892)--(11.444,5.894)%
  --(11.447,5.895)--(11.450,5.897)--(11.453,5.898)--(11.456,5.900)--(11.458,5.901)--(11.461,5.903)%
  --(11.464,5.904)--(11.467,5.906)--(11.470,5.907)--(11.473,5.909)--(11.476,5.910)--(11.479,5.912)%
  --(11.482,5.913)--(11.485,5.915)--(11.488,5.916)--(11.491,5.918)--(11.494,5.919)--(11.497,5.921)%
  --(11.500,5.922)--(11.503,5.924)--(11.506,5.925)--(11.509,5.927)--(11.512,5.928)--(11.515,5.930)%
  --(11.518,5.931)--(11.521,5.933)--(11.524,5.934)--(11.527,5.936)--(11.530,5.937)--(11.533,5.939)%
  --(11.536,5.940)--(11.539,5.942)--(11.542,5.943)--(11.545,5.945)--(11.548,5.946)--(11.551,5.948)%
  --(11.554,5.949)--(11.557,5.951)--(11.560,5.952)--(11.563,5.954)--(11.566,5.955)--(11.569,5.957)%
  --(11.572,5.958)--(11.575,5.960)--(11.578,5.961)--(11.581,5.963)--(11.584,5.964)--(11.587,5.966)%
  --(11.590,5.967)--(11.593,5.969)--(11.596,5.970)--(11.599,5.972)--(11.602,5.973)--(11.605,5.975)%
  --(11.608,5.977)--(11.611,5.978)--(11.614,5.980)--(11.617,5.981)--(11.620,5.983)--(11.623,5.984)%
  --(11.626,5.986)--(11.629,5.987)--(11.632,5.989)--(11.635,5.990)--(11.638,5.992)--(11.641,5.993)%
  --(11.644,5.995)--(11.647,5.996)--(11.650,5.998)--(11.653,5.999)--(11.656,6.001)--(11.659,6.002)%
  --(11.662,6.004)--(11.665,6.005)--(11.667,6.007)--(11.670,6.008)--(11.673,6.010)--(11.676,6.011)%
  --(11.679,6.013)--(11.682,6.014)--(11.685,6.016)--(11.688,6.017)--(11.691,6.019)--(11.694,6.020)%
  --(11.697,6.022)--(11.700,6.023)--(11.703,6.025)--(11.706,6.026)--(11.709,6.028)--(11.712,6.029)%
  --(11.715,6.031)--(11.718,6.032)--(11.721,6.034)--(11.724,6.035)--(11.727,6.037)--(11.730,6.038)%
  --(11.733,6.040)--(11.736,6.041)--(11.739,6.043)--(11.742,6.044)--(11.745,6.046)--(11.748,6.047)%
  --(11.751,6.049)--(11.754,6.050)--(11.757,6.052)--(11.760,6.053)--(11.763,6.055)--(11.766,6.056)%
  --(11.769,6.058)--(11.772,6.060)--(11.775,6.061)--(11.778,6.063)--(11.781,6.064)--(11.784,6.066)%
  --(11.787,6.067)--(11.790,6.069)--(11.793,6.070)--(11.796,6.072)--(11.799,6.073)--(11.802,6.075)%
  --(11.805,6.076)--(11.808,6.078)--(11.811,6.079)--(11.814,6.081)--(11.817,6.082)--(11.820,6.084)%
  --(11.823,6.085)--(11.826,6.087)--(11.829,6.088)--(11.832,6.090)--(11.835,6.091)--(11.838,6.093)%
  --(11.841,6.094)--(11.844,6.096)--(11.847,6.097)--(11.850,6.099)--(11.853,6.100)--(11.856,6.102)%
  --(11.859,6.103)--(11.862,6.105)--(11.865,6.106)--(11.868,6.108)--(11.871,6.109)--(11.874,6.111)%
  --(11.876,6.112)--(11.879,6.114)--(11.882,6.115)--(11.885,6.117)--(11.888,6.118)--(11.891,6.120)%
  --(11.894,6.121)--(11.897,6.123)--(11.900,6.124)--(11.903,6.126)--(11.906,6.127)--(11.909,6.129)%
  --(11.912,6.130)--(11.915,6.132)--(11.918,6.133)--(11.921,6.135)--(11.924,6.137)--(11.927,6.138)%
  --(11.930,6.140)--(11.933,6.141)--(11.936,6.143)--(11.939,6.144)--(11.942,6.146)--(11.945,6.147)%
  --(11.948,6.149)--(11.951,6.150)--(11.954,6.152)--(11.957,6.153)--(11.960,6.155)--(11.963,6.156)%
  --(11.966,6.158)--(11.969,6.159)--(11.972,6.161)--(11.975,6.162)--(11.978,6.164)--(11.981,6.165)%
  --(11.984,6.167)--(11.987,6.168)--(11.990,6.170)--(11.993,6.171)--(11.996,6.173)--(11.999,6.174)%
  --(12.002,6.176)--(12.005,6.177)--(12.008,6.179)--(12.011,6.180)--(12.014,6.182)--(12.017,6.183)%
  --(12.020,6.185)--(12.023,6.186)--(12.026,6.188)--(12.029,6.189)--(12.032,6.191)--(12.035,6.192)%
  --(12.038,6.194)--(12.041,6.195)--(12.044,6.197)--(12.047,6.198)--(12.050,6.200)--(12.053,6.201)%
  --(12.056,6.203)--(12.059,6.204)--(12.062,6.206)--(12.065,6.208)--(12.068,6.209)--(12.071,6.211)%
  --(12.074,6.212)--(12.077,6.214)--(12.080,6.215)--(12.083,6.217)--(12.085,6.218)--(12.088,6.220)%
  --(12.091,6.221)--(12.094,6.223)--(12.097,6.224)--(12.100,6.226)--(12.103,6.227)--(12.106,6.229)%
  --(12.109,6.230)--(12.112,6.232)--(12.115,6.233)--(12.118,6.235)--(12.121,6.236)--(12.124,6.238)%
  --(12.127,6.239)--(12.130,6.241)--(12.133,6.242)--(12.136,6.244)--(12.139,6.245)--(12.142,6.247)%
  --(12.145,6.248)--(12.148,6.250)--(12.151,6.251)--(12.154,6.253)--(12.157,6.254)--(12.160,6.256)%
  --(12.163,6.257)--(12.166,6.259)--(12.169,6.260)--(12.172,6.262)--(12.175,6.263)--(12.178,6.265)%
  --(12.181,6.266)--(12.184,6.268)--(12.187,6.269)--(12.190,6.271)--(12.193,6.273)--(12.196,6.274)%
  --(12.199,6.276)--(12.202,6.277)--(12.205,6.279)--(12.208,6.280)--(12.211,6.282)--(12.214,6.283)%
  --(12.217,6.285)--(12.220,6.286)--(12.223,6.288)--(12.226,6.289)--(12.229,6.291)--(12.232,6.292)%
  --(12.235,6.294)--(12.238,6.295)--(12.241,6.297)--(12.244,6.298)--(12.247,6.300)--(12.250,6.301)%
  --(12.253,6.303)--(12.256,6.304)--(12.259,6.306)--(12.262,6.307)--(12.265,6.309)--(12.268,6.310)%
  --(12.271,6.312)--(12.274,6.313)--(12.277,6.315)--(12.280,6.316)--(12.283,6.318)--(12.286,6.319)%
  --(12.289,6.321)--(12.292,6.322)--(12.295,6.324)--(12.297,6.325)--(12.300,6.327)--(12.303,6.328)%
  --(12.306,6.330)--(12.309,6.331)--(12.312,6.333)--(12.315,6.334)--(12.318,6.336)--(12.321,6.338)%
  --(12.324,6.339)--(12.327,6.341)--(12.330,6.342)--(12.333,6.344)--(12.336,6.345)--(12.339,6.347)%
  --(12.342,6.348)--(12.345,6.350)--(12.348,6.351)--(12.351,6.353)--(12.354,6.354)--(12.357,6.356)%
  --(12.360,6.357)--(12.363,6.359)--(12.366,6.360)--(12.369,6.362)--(12.372,6.363)--(12.375,6.365)%
  --(12.378,6.366)--(12.381,6.368)--(12.384,6.369)--(12.387,6.371)--(12.390,6.372)--(12.393,6.374)%
  --(12.396,6.375)--(12.399,6.377)--(12.402,6.378)--(12.405,6.380)--(12.408,6.381)--(12.411,6.383)%
  --(12.414,6.384)--(12.417,6.386)--(12.420,6.387)--(12.423,6.389)--(12.426,6.390)--(12.429,6.392)%
  --(12.432,6.393)--(12.435,6.395)--(12.438,6.397)--(12.441,6.398)--(12.444,6.400)--(12.447,6.401)%
  --(12.450,6.403)--(12.453,6.404)--(12.456,6.406)--(12.459,6.407)--(12.462,6.409)--(12.465,6.410)%
  --(12.468,6.412)--(12.471,6.413)--(12.474,6.415)--(12.477,6.416)--(12.480,6.418)--(12.483,6.419)%
  --(12.486,6.421)--(12.489,6.422)--(12.492,6.424)--(12.495,6.425)--(12.498,6.427)--(12.501,6.428)%
  --(12.504,6.430)--(12.506,6.431)--(12.509,6.433)--(12.512,6.434)--(12.515,6.436)--(12.518,6.437)%
  --(12.521,6.439)--(12.524,6.440)--(12.527,6.442)--(12.530,6.443)--(12.533,6.445)--(12.536,6.446)%
  --(12.539,6.448)--(12.542,6.449)--(12.545,6.451)--(12.548,6.452)--(12.551,6.454)--(12.554,6.456)%
  --(12.557,6.457)--(12.560,6.459)--(12.563,6.460)--(12.566,6.462)--(12.569,6.463)--(12.572,6.465)%
  --(12.575,6.466)--(12.578,6.468)--(12.581,6.469)--(12.584,6.471)--(12.587,6.472)--(12.590,6.474)%
  --(12.593,6.475)--(12.596,6.477)--(12.599,6.478)--(12.602,6.480)--(12.605,6.481)--(12.608,6.483)%
  --(12.611,6.484)--(12.614,6.486)--(12.617,6.487)--(12.620,6.489)--(12.623,6.490)--(12.626,6.492)%
  --(12.629,6.493)--(12.632,6.495)--(12.635,6.496)--(12.638,6.498)--(12.641,6.499)--(12.644,6.501)%
  --(12.647,6.502)--(12.650,6.504)--(12.653,6.505)--(12.656,6.507)--(12.659,6.508)--(12.662,6.510)%
  --(12.665,6.512)--(12.668,6.513)--(12.671,6.515)--(12.674,6.516)--(12.677,6.518)--(12.680,6.519)%
  --(12.683,6.521)--(12.686,6.522)--(12.689,6.524)--(12.692,6.525)--(12.695,6.527)--(12.698,6.528)%
  --(12.701,6.530)--(12.704,6.531)--(12.707,6.533)--(12.710,6.534)--(12.713,6.536)--(12.715,6.537)%
  --(12.718,6.539)--(12.721,6.540)--(12.724,6.542)--(12.727,6.543)--(12.730,6.545)--(12.733,6.546)%
  --(12.736,6.548)--(12.739,6.549)--(12.742,6.551)--(12.745,6.552)--(12.748,6.554)--(12.751,6.555)%
  --(12.754,6.557)--(12.757,6.558)--(12.760,6.560)--(12.763,6.561)--(12.766,6.563)--(12.769,6.565)%
  --(12.772,6.566)--(12.775,6.568)--(12.778,6.569)--(12.781,6.571)--(12.784,6.572)--(12.787,6.574)%
  --(12.790,6.575)--(12.793,6.577)--(12.796,6.578)--(12.799,6.580)--(12.802,6.581)--(12.805,6.583)%
  --(12.808,6.584)--(12.811,6.586)--(12.814,6.587)--(12.817,6.589)--(12.820,6.590)--(12.823,6.592)%
  --(12.826,6.593)--(12.829,6.595)--(12.832,6.596)--(12.835,6.598)--(12.838,6.599)--(12.841,6.601)%
  --(12.844,6.602)--(12.847,6.604)--(12.850,6.605)--(12.853,6.607)--(12.856,6.608)--(12.859,6.610)%
  --(12.862,6.611)--(12.865,6.613)--(12.868,6.614)--(12.871,6.616)--(12.874,6.618)--(12.877,6.619)%
  --(12.880,6.621)--(12.883,6.622)--(12.886,6.624)--(12.889,6.625)--(12.892,6.627)--(12.895,6.628)%
  --(12.898,6.630)--(12.901,6.631)--(12.904,6.633)--(12.907,6.634)--(12.910,6.636)--(12.913,6.637)%
  --(12.916,6.639)--(12.919,6.640)--(12.922,6.642)--(12.924,6.643)--(12.927,6.645)--(12.930,6.646)%
  --(12.933,6.648)--(12.936,6.649)--(12.939,6.651)--(12.942,6.652)--(12.945,6.654)--(12.948,6.655)%
  --(12.951,6.657)--(12.954,6.658)--(12.957,6.660)--(12.960,6.661)--(12.963,6.663)--(12.966,6.664)%
  --(12.969,6.666)--(12.972,6.667)--(12.975,6.669)--(12.978,6.671)--(12.981,6.672)--(12.984,6.674)%
  --(12.987,6.675)--(12.990,6.677)--(12.993,6.678)--(12.996,6.680)--(12.999,6.681)--(13.002,6.683)%
  --(13.005,6.684)--(13.008,6.686)--(13.011,6.687)--(13.014,6.689)--(13.017,6.690)--(13.020,6.692)%
  --(13.023,6.693)--(13.026,6.695)--(13.029,6.696)--(13.032,6.698)--(13.035,6.699)--(13.038,6.701)%
  --(13.041,6.702)--(13.044,6.704)--(13.047,6.705)--(13.050,6.707)--(13.053,6.708)--(13.056,6.710)%
  --(13.059,6.711)--(13.062,6.713)--(13.065,6.714)--(13.068,6.716)--(13.071,6.717)--(13.074,6.719)%
  --(13.077,6.721)--(13.080,6.722)--(13.083,6.724)--(13.086,6.725)--(13.089,6.727)--(13.092,6.728)%
  --(13.095,6.730)--(13.098,6.731)--(13.101,6.733)--(13.104,6.734)--(13.107,6.736)--(13.110,6.737)%
  --(13.113,6.739)--(13.116,6.740)--(13.119,6.742)--(13.122,6.743)--(13.125,6.745)--(13.128,6.746)%
  --(13.131,6.748)--(13.133,6.749)--(13.136,6.751)--(13.139,6.752)--(13.142,6.754)--(13.145,6.755)%
  --(13.148,6.757)--(13.151,6.758)--(13.154,6.760)--(13.157,6.761)--(13.160,6.763)--(13.163,6.764)%
  --(13.166,6.766)--(13.169,6.768)--(13.172,6.769)--(13.175,6.771)--(13.178,6.772)--(13.181,6.774)%
  --(13.184,6.775)--(13.187,6.777)--(13.190,6.778)--(13.193,6.780)--(13.196,6.781)--(13.199,6.783)%
  --(13.202,6.784)--(13.205,6.786)--(13.208,6.787)--(13.211,6.789)--(13.214,6.790)--(13.217,6.792)%
  --(13.220,6.793)--(13.223,6.795)--(13.226,6.796)--(13.229,6.798)--(13.232,6.799)--(13.235,6.801)%
  --(13.238,6.802)--(13.241,6.804)--(13.244,6.805)--(13.247,6.807)--(13.250,6.808)--(13.253,6.810)%
  --(13.256,6.811)--(13.259,6.813)--(13.262,6.814)--(13.265,6.816)--(13.268,6.818)--(13.271,6.819)%
  --(13.274,6.821)--(13.277,6.822)--(13.280,6.824)--(13.283,6.825)--(13.286,6.827)--(13.289,6.828)%
  --(13.292,6.830)--(13.295,6.831)--(13.298,6.833)--(13.301,6.834)--(13.304,6.836)--(13.307,6.837)%
  --(13.310,6.839)--(13.313,6.840)--(13.316,6.842)--(13.319,6.843)--(13.322,6.845)--(13.325,6.846)%
  --(13.328,6.848)--(13.331,6.849)--(13.334,6.851)--(13.337,6.852)--(13.340,6.854)--(13.342,6.855)%
  --(13.345,6.857)--(13.348,6.858)--(13.351,6.860)--(13.354,6.861)--(13.357,6.863)--(13.360,6.865)%
  --(13.363,6.866)--(13.366,6.868)--(13.369,6.869)--(13.372,6.871)--(13.375,6.872)--(13.378,6.874)%
  --(13.381,6.875)--(13.384,6.877)--(13.387,6.878)--(13.390,6.880)--(13.393,6.881)--(13.396,6.883)%
  --(13.399,6.884)--(13.402,6.886)--(13.405,6.887)--(13.408,6.889)--(13.411,6.890)--(13.414,6.892)%
  --(13.417,6.893)--(13.420,6.895)--(13.423,6.896)--(13.426,6.898)--(13.429,6.899)--(13.432,6.901)%
  --(13.435,6.902)--(13.438,6.904)--(13.441,6.905)--(13.444,6.907);
\gpcolor{color=gp lt color border}
\node[gp node left] at (2.972,7.373) {$\rho \approx \rho_{\rm{max}}$};
\gpcolor{rgb color={0.902,0.624,0.000}}
\draw[gp path] (1.872,7.373)--(2.788,7.373);
\draw[gp path] (1.507,2.514)--(1.510,2.514)--(1.513,2.514)--(1.516,2.514)--(1.519,2.514)%
  --(1.522,2.514)--(1.525,2.514)--(1.528,2.514)--(1.531,2.514)--(1.534,2.514)--(1.537,2.514)%
  --(1.540,2.514)--(1.543,2.514)--(1.546,2.514)--(1.549,2.514)--(1.552,2.514)--(1.555,2.514)%
  --(1.558,2.514)--(1.561,2.514)--(1.564,2.514)--(1.567,2.514)--(1.570,2.514)--(1.573,2.514)%
  --(1.576,2.514)--(1.579,2.514)--(1.582,2.514)--(1.585,2.514)--(1.588,2.514)--(1.591,2.514)%
  --(1.594,2.514)--(1.597,2.514)--(1.600,2.514)--(1.603,2.514)--(1.606,2.514)--(1.609,2.514)%
  --(1.611,2.514)--(1.614,2.514)--(1.617,2.514)--(1.620,2.514)--(1.623,2.514)--(1.626,2.514)%
  --(1.629,2.514)--(1.632,2.514)--(1.635,2.514)--(1.638,2.514)--(1.641,2.514)--(1.644,2.514)%
  --(1.647,2.514)--(1.650,2.514)--(1.653,2.514)--(1.656,2.514)--(1.659,2.514)--(1.662,2.514)%
  --(1.665,2.514)--(1.668,2.514)--(1.671,2.514)--(1.674,2.514)--(1.677,2.514)--(1.680,2.514)%
  --(1.683,2.514)--(1.686,2.514)--(1.689,2.514)--(1.692,2.514)--(1.695,2.514)--(1.698,2.514)%
  --(1.701,2.514)--(1.704,2.514)--(1.707,2.514)--(1.710,2.514)--(1.713,2.514)--(1.716,2.514)%
  --(1.719,2.514)--(1.722,2.514)--(1.725,2.514)--(1.728,2.514)--(1.731,2.515)--(1.734,2.515)%
  --(1.737,2.515)--(1.740,2.515)--(1.743,2.515)--(1.746,2.515)--(1.749,2.515)--(1.752,2.515)%
  --(1.755,2.515)--(1.758,2.515)--(1.761,2.515)--(1.764,2.515)--(1.767,2.515)--(1.770,2.515)%
  --(1.773,2.515)--(1.776,2.515)--(1.779,2.515)--(1.782,2.515)--(1.785,2.515)--(1.788,2.515)%
  --(1.791,2.515)--(1.794,2.515)--(1.797,2.515)--(1.800,2.515)--(1.803,2.515)--(1.806,2.515)%
  --(1.809,2.515)--(1.812,2.515)--(1.815,2.515)--(1.818,2.515)--(1.820,2.515)--(1.823,2.515)%
  --(1.826,2.516)--(1.829,2.516)--(1.832,2.516)--(1.835,2.516)--(1.838,2.516)--(1.841,2.516)%
  --(1.844,2.516)--(1.847,2.516)--(1.850,2.516)--(1.853,2.516)--(1.856,2.516)--(1.859,2.516)%
  --(1.862,2.516)--(1.865,2.516)--(1.868,2.516)--(1.871,2.516)--(1.874,2.516)--(1.877,2.516)%
  --(1.880,2.517)--(1.883,2.517)--(1.886,2.517)--(1.889,2.517)--(1.892,2.517)--(1.895,2.517)%
  --(1.898,2.517)--(1.901,2.517)--(1.904,2.517)--(1.907,2.517)--(1.910,2.517)--(1.913,2.517)%
  --(1.916,2.517)--(1.919,2.517)--(1.922,2.517)--(1.925,2.518)--(1.928,2.518)--(1.931,2.518)%
  --(1.934,2.518)--(1.937,2.518)--(1.940,2.518)--(1.943,2.518)--(1.946,2.518)--(1.949,2.518)%
  --(1.952,2.518)--(1.955,2.518)--(1.958,2.518)--(1.961,2.519)--(1.964,2.519)--(1.967,2.519)%
  --(1.970,2.519)--(1.973,2.519)--(1.976,2.519)--(1.979,2.519)--(1.982,2.519)--(1.985,2.519)%
  --(1.988,2.519)--(1.991,2.520)--(1.994,2.520)--(1.997,2.520)--(2.000,2.520)--(2.003,2.520)%
  --(2.006,2.520)--(2.009,2.520)--(2.012,2.520)--(2.015,2.520)--(2.018,2.521)--(2.021,2.521)%
  --(2.024,2.521)--(2.027,2.521)--(2.029,2.521)--(2.032,2.521)--(2.035,2.521)--(2.038,2.521)%
  --(2.041,2.522)--(2.044,2.522)--(2.047,2.522)--(2.050,2.522)--(2.053,2.522)--(2.056,2.522)%
  --(2.059,2.522)--(2.062,2.522)--(2.065,2.523)--(2.068,2.523)--(2.071,2.523)--(2.074,2.523)%
  --(2.077,2.523)--(2.080,2.523)--(2.083,2.523)--(2.086,2.524)--(2.089,2.524)--(2.092,2.524)%
  --(2.095,2.524)--(2.098,2.524)--(2.101,2.524)--(2.104,2.524)--(2.107,2.525)--(2.110,2.525)%
  --(2.113,2.525)--(2.116,2.525)--(2.119,2.525)--(2.122,2.525)--(2.125,2.526)--(2.128,2.526)%
  --(2.131,2.526)--(2.134,2.526)--(2.137,2.526)--(2.140,2.526)--(2.143,2.527)--(2.146,2.527)%
  --(2.149,2.527)--(2.152,2.527)--(2.155,2.527)--(2.158,2.527)--(2.161,2.528)--(2.164,2.528)%
  --(2.167,2.528)--(2.170,2.528)--(2.173,2.528)--(2.176,2.529)--(2.179,2.529)--(2.182,2.529)%
  --(2.185,2.529)--(2.188,2.529)--(2.191,2.530)--(2.194,2.530)--(2.197,2.530)--(2.200,2.530)%
  --(2.203,2.530)--(2.206,2.531)--(2.209,2.531)--(2.212,2.531)--(2.215,2.531)--(2.218,2.531)%
  --(2.221,2.532)--(2.224,2.532)--(2.227,2.532)--(2.230,2.532)--(2.233,2.533)--(2.236,2.533)%
  --(2.238,2.533)--(2.241,2.533)--(2.244,2.533)--(2.247,2.534)--(2.250,2.534)--(2.253,2.534)%
  --(2.256,2.534)--(2.259,2.535)--(2.262,2.535)--(2.265,2.535)--(2.268,2.535)--(2.271,2.536)%
  --(2.274,2.536)--(2.277,2.536)--(2.280,2.536)--(2.283,2.537)--(2.286,2.537)--(2.289,2.537)%
  --(2.292,2.537)--(2.295,2.538)--(2.298,2.538)--(2.301,2.538)--(2.304,2.538)--(2.307,2.539)%
  --(2.310,2.539)--(2.313,2.539)--(2.316,2.539)--(2.319,2.540)--(2.322,2.540)--(2.325,2.540)%
  --(2.328,2.540)--(2.331,2.541)--(2.334,2.541)--(2.337,2.541)--(2.340,2.542)--(2.343,2.542)%
  --(2.346,2.542)--(2.349,2.542)--(2.352,2.543)--(2.355,2.543)--(2.358,2.543)--(2.361,2.544)%
  --(2.364,2.544)--(2.367,2.544)--(2.370,2.544)--(2.373,2.545)--(2.376,2.545)--(2.379,2.545)%
  --(2.382,2.546)--(2.385,2.546)--(2.388,2.546)--(2.391,2.547)--(2.394,2.547)--(2.397,2.547)%
  --(2.400,2.548)--(2.403,2.548)--(2.406,2.548)--(2.409,2.548)--(2.412,2.549)--(2.415,2.549)%
  --(2.418,2.549)--(2.421,2.550)--(2.424,2.550)--(2.427,2.550)--(2.430,2.551)--(2.433,2.551)%
  --(2.436,2.551)--(2.439,2.552)--(2.442,2.552)--(2.445,2.552)--(2.447,2.553)--(2.450,2.553)%
  --(2.453,2.554)--(2.456,2.554)--(2.459,2.554)--(2.462,2.555)--(2.465,2.555)--(2.468,2.555)%
  --(2.471,2.556)--(2.474,2.556)--(2.477,2.556)--(2.480,2.557)--(2.483,2.557)--(2.486,2.557)%
  --(2.489,2.558)--(2.492,2.558)--(2.495,2.559)--(2.498,2.559)--(2.501,2.559)--(2.504,2.560)%
  --(2.507,2.560)--(2.510,2.560)--(2.513,2.561)--(2.516,2.561)--(2.519,2.562)--(2.522,2.562)%
  --(2.525,2.562)--(2.528,2.563)--(2.531,2.563)--(2.534,2.564)--(2.537,2.564)--(2.540,2.564)%
  --(2.543,2.565)--(2.546,2.565)--(2.549,2.566)--(2.552,2.566)--(2.555,2.566)--(2.558,2.567)%
  --(2.561,2.567)--(2.564,2.568)--(2.567,2.568)--(2.570,2.568)--(2.573,2.569)--(2.576,2.569)%
  --(2.579,2.570)--(2.582,2.570)--(2.585,2.571)--(2.588,2.571)--(2.591,2.571)--(2.594,2.572)%
  --(2.597,2.572)--(2.600,2.573)--(2.603,2.573)--(2.606,2.574)--(2.609,2.574)--(2.612,2.574)%
  --(2.615,2.575)--(2.618,2.575)--(2.621,2.576)--(2.624,2.576)--(2.627,2.577)--(2.630,2.577)%
  --(2.633,2.578)--(2.636,2.578)--(2.639,2.579)--(2.642,2.579)--(2.645,2.579)--(2.648,2.580)%
  --(2.651,2.580)--(2.654,2.581)--(2.656,2.581)--(2.659,2.582)--(2.662,2.582)--(2.665,2.583)%
  --(2.668,2.583)--(2.671,2.584)--(2.674,2.584)--(2.677,2.585)--(2.680,2.585)--(2.683,2.586)%
  --(2.686,2.586)--(2.689,2.587)--(2.692,2.587)--(2.695,2.588)--(2.698,2.588)--(2.701,2.589)%
  --(2.704,2.589)--(2.707,2.590)--(2.710,2.590)--(2.713,2.591)--(2.716,2.591)--(2.719,2.592)%
  --(2.722,2.592)--(2.725,2.593)--(2.728,2.593)--(2.731,2.594)--(2.734,2.594)--(2.737,2.595)%
  --(2.740,2.595)--(2.743,2.596)--(2.746,2.596)--(2.749,2.597)--(2.752,2.597)--(2.755,2.598)%
  --(2.758,2.598)--(2.761,2.599)--(2.764,2.599)--(2.767,2.600)--(2.770,2.600)--(2.773,2.601)%
  --(2.776,2.602)--(2.779,2.602)--(2.782,2.603)--(2.785,2.603)--(2.788,2.604)--(2.791,2.604)%
  --(2.794,2.605)--(2.797,2.605)--(2.800,2.606)--(2.803,2.607)--(2.806,2.607)--(2.809,2.608)%
  --(2.812,2.608)--(2.815,2.609)--(2.818,2.609)--(2.821,2.610)--(2.824,2.610)--(2.827,2.611)%
  --(2.830,2.612)--(2.833,2.612)--(2.836,2.613)--(2.839,2.613)--(2.842,2.614)--(2.845,2.614)%
  --(2.848,2.615)--(2.851,2.616)--(2.854,2.616)--(2.857,2.617)--(2.860,2.617)--(2.863,2.618)%
  --(2.866,2.619)--(2.868,2.619)--(2.871,2.620)--(2.874,2.620)--(2.877,2.621)--(2.880,2.622)%
  --(2.883,2.622)--(2.886,2.623)--(2.889,2.623)--(2.892,2.624)--(2.895,2.625)--(2.898,2.625)%
  --(2.901,2.626)--(2.904,2.626)--(2.907,2.627)--(2.910,2.628)--(2.913,2.628)--(2.916,2.629)%
  --(2.919,2.630)--(2.922,2.630)--(2.925,2.631)--(2.928,2.631)--(2.931,2.632)--(2.934,2.633)%
  --(2.937,2.633)--(2.940,2.634)--(2.943,2.635)--(2.946,2.635)--(2.949,2.636)--(2.952,2.637)%
  --(2.955,2.637)--(2.958,2.638)--(2.961,2.638)--(2.964,2.639)--(2.967,2.640)--(2.970,2.640)%
  --(2.973,2.641)--(2.976,2.642)--(2.979,2.642)--(2.982,2.643)--(2.985,2.644)--(2.988,2.644)%
  --(2.991,2.645)--(2.994,2.646)--(2.997,2.646)--(3.000,2.647)--(3.003,2.648)--(3.006,2.648)%
  --(3.009,2.649)--(3.012,2.650)--(3.015,2.650)--(3.018,2.651)--(3.021,2.652)--(3.024,2.652)%
  --(3.027,2.653)--(3.030,2.654)--(3.033,2.655)--(3.036,2.655)--(3.039,2.656)--(3.042,2.657)%
  --(3.045,2.657)--(3.048,2.658)--(3.051,2.659)--(3.054,2.659)--(3.057,2.660)--(3.060,2.661)%
  --(3.063,2.661)--(3.066,2.662)--(3.069,2.663)--(3.072,2.664)--(3.075,2.664)--(3.077,2.665)%
  --(3.080,2.666)--(3.083,2.666)--(3.086,2.667)--(3.089,2.668)--(3.092,2.669)--(3.095,2.669)%
  --(3.098,2.670)--(3.101,2.671)--(3.104,2.671)--(3.107,2.672)--(3.110,2.673)--(3.113,2.674)%
  --(3.116,2.674)--(3.119,2.675)--(3.122,2.676)--(3.125,2.677)--(3.128,2.677)--(3.131,2.678)%
  --(3.134,2.679)--(3.137,2.680)--(3.140,2.680)--(3.143,2.681)--(3.146,2.682)--(3.149,2.683)%
  --(3.152,2.683)--(3.155,2.684)--(3.158,2.685)--(3.161,2.686)--(3.164,2.686)--(3.167,2.687)%
  --(3.170,2.688)--(3.173,2.689)--(3.176,2.689)--(3.179,2.690)--(3.182,2.691)--(3.185,2.692)%
  --(3.188,2.692)--(3.191,2.693)--(3.194,2.694)--(3.197,2.695)--(3.200,2.695)--(3.203,2.696)%
  --(3.206,2.697)--(3.209,2.698)--(3.212,2.699)--(3.215,2.699)--(3.218,2.700)--(3.221,2.701)%
  --(3.224,2.702)--(3.227,2.703)--(3.230,2.703)--(3.233,2.704)--(3.236,2.705)--(3.239,2.706)%
  --(3.242,2.706)--(3.245,2.707)--(3.248,2.708)--(3.251,2.709)--(3.254,2.710)--(3.257,2.710)%
  --(3.260,2.711)--(3.263,2.712)--(3.266,2.713)--(3.269,2.714)--(3.272,2.715)--(3.275,2.715)%
  --(3.278,2.716)--(3.281,2.717)--(3.284,2.718)--(3.286,2.719)--(3.289,2.719)--(3.292,2.720)%
  --(3.295,2.721)--(3.298,2.722)--(3.301,2.723)--(3.304,2.724)--(3.307,2.724)--(3.310,2.725)%
  --(3.313,2.726)--(3.316,2.727)--(3.319,2.728)--(3.322,2.729)--(3.325,2.729)--(3.328,2.730)%
  --(3.331,2.731)--(3.334,2.732)--(3.337,2.733)--(3.340,2.734)--(3.343,2.734)--(3.346,2.735)%
  --(3.349,2.736)--(3.352,2.737)--(3.355,2.738)--(3.358,2.739)--(3.361,2.739)--(3.364,2.740)%
  --(3.367,2.741)--(3.370,2.742)--(3.373,2.743)--(3.376,2.744)--(3.379,2.745)--(3.382,2.745)%
  --(3.385,2.746)--(3.388,2.747)--(3.391,2.748)--(3.394,2.749)--(3.397,2.750)--(3.400,2.751)%
  --(3.403,2.752)--(3.406,2.752)--(3.409,2.753)--(3.412,2.754)--(3.415,2.755)--(3.418,2.756)%
  --(3.421,2.757)--(3.424,2.758)--(3.427,2.759)--(3.430,2.759)--(3.433,2.760)--(3.436,2.761)%
  --(3.439,2.762)--(3.442,2.763)--(3.445,2.764)--(3.448,2.765)--(3.451,2.766)--(3.454,2.767)%
  --(3.457,2.767)--(3.460,2.768)--(3.463,2.769)--(3.466,2.770)--(3.469,2.771)--(3.472,2.772)%
  --(3.475,2.773)--(3.478,2.774)--(3.481,2.775)--(3.484,2.776)--(3.487,2.776)--(3.490,2.777)%
  --(3.493,2.778)--(3.495,2.779)--(3.498,2.780)--(3.501,2.781)--(3.504,2.782)--(3.507,2.783)%
  --(3.510,2.784)--(3.513,2.785)--(3.516,2.786)--(3.519,2.787)--(3.522,2.787)--(3.525,2.788)%
  --(3.528,2.789)--(3.531,2.790)--(3.534,2.791)--(3.537,2.792)--(3.540,2.793)--(3.543,2.794)%
  --(3.546,2.795)--(3.549,2.796)--(3.552,2.797)--(3.555,2.798)--(3.558,2.799)--(3.561,2.800)%
  --(3.564,2.800)--(3.567,2.801)--(3.570,2.802)--(3.573,2.803)--(3.576,2.804)--(3.579,2.805)%
  --(3.582,2.806)--(3.585,2.807)--(3.588,2.808)--(3.591,2.809)--(3.594,2.810)--(3.597,2.811)%
  --(3.600,2.812)--(3.603,2.813)--(3.606,2.814)--(3.609,2.815)--(3.612,2.816)--(3.615,2.817)%
  --(3.618,2.818)--(3.621,2.819)--(3.624,2.820)--(3.627,2.821)--(3.630,2.821)--(3.633,2.822)%
  --(3.636,2.823)--(3.639,2.824)--(3.642,2.825)--(3.645,2.826)--(3.648,2.827)--(3.651,2.828)%
  --(3.654,2.829)--(3.657,2.830)--(3.660,2.831)--(3.663,2.832)--(3.666,2.833)--(3.669,2.834)%
  --(3.672,2.835)--(3.675,2.836)--(3.678,2.837)--(3.681,2.838)--(3.684,2.839)--(3.687,2.840)%
  --(3.690,2.841)--(3.693,2.842)--(3.696,2.843)--(3.699,2.844)--(3.702,2.845)--(3.704,2.846)%
  --(3.707,2.847)--(3.710,2.848)--(3.713,2.849)--(3.716,2.850)--(3.719,2.851)--(3.722,2.852)%
  --(3.725,2.853)--(3.728,2.854)--(3.731,2.855)--(3.734,2.856)--(3.737,2.857)--(3.740,2.858)%
  --(3.743,2.859)--(3.746,2.860)--(3.749,2.861)--(3.752,2.862)--(3.755,2.863)--(3.758,2.864)%
  --(3.761,2.865)--(3.764,2.866)--(3.767,2.867)--(3.770,2.868)--(3.773,2.869)--(3.776,2.870)%
  --(3.779,2.871)--(3.782,2.872)--(3.785,2.873)--(3.788,2.874)--(3.791,2.875)--(3.794,2.876)%
  --(3.797,2.877)--(3.800,2.878)--(3.803,2.879)--(3.806,2.880)--(3.809,2.881)--(3.812,2.882)%
  --(3.815,2.883)--(3.818,2.885)--(3.821,2.886)--(3.824,2.887)--(3.827,2.888)--(3.830,2.889)%
  --(3.833,2.890)--(3.836,2.891)--(3.839,2.892)--(3.842,2.893)--(3.845,2.894)--(3.848,2.895)%
  --(3.851,2.896)--(3.854,2.897)--(3.857,2.898)--(3.860,2.899)--(3.863,2.900)--(3.866,2.901)%
  --(3.869,2.902)--(3.872,2.903)--(3.875,2.904)--(3.878,2.905)--(3.881,2.906)--(3.884,2.908)%
  --(3.887,2.909)--(3.890,2.910)--(3.893,2.911)--(3.896,2.912)--(3.899,2.913)--(3.902,2.914)%
  --(3.905,2.915)--(3.908,2.916)--(3.911,2.917)--(3.914,2.918)--(3.916,2.919)--(3.919,2.920)%
  --(3.922,2.921)--(3.925,2.922)--(3.928,2.923)--(3.931,2.925)--(3.934,2.926)--(3.937,2.927)%
  --(3.940,2.928)--(3.943,2.929)--(3.946,2.930)--(3.949,2.931)--(3.952,2.932)--(3.955,2.933)%
  --(3.958,2.934)--(3.961,2.935)--(3.964,2.936)--(3.967,2.937)--(3.970,2.939)--(3.973,2.940)%
  --(3.976,2.941)--(3.979,2.942)--(3.982,2.943)--(3.985,2.944)--(3.988,2.945)--(3.991,2.946)%
  --(3.994,2.947)--(3.997,2.948)--(4.000,2.949)--(4.003,2.951)--(4.006,2.952)--(4.009,2.953)%
  --(4.012,2.954)--(4.015,2.955)--(4.018,2.956)--(4.021,2.957)--(4.024,2.958)--(4.027,2.959)%
  --(4.030,2.960)--(4.033,2.961)--(4.036,2.963)--(4.039,2.964)--(4.042,2.965)--(4.045,2.966)%
  --(4.048,2.967)--(4.051,2.968)--(4.054,2.969)--(4.057,2.970)--(4.060,2.971)--(4.063,2.973)%
  --(4.066,2.974)--(4.069,2.975)--(4.072,2.976)--(4.075,2.977)--(4.078,2.978)--(4.081,2.979)%
  --(4.084,2.980)--(4.087,2.981)--(4.090,2.983)--(4.093,2.984)--(4.096,2.985)--(4.099,2.986)%
  --(4.102,2.987)--(4.105,2.988)--(4.108,2.989)--(4.111,2.990)--(4.114,2.992)--(4.117,2.993)%
  --(4.120,2.994)--(4.123,2.995)--(4.125,2.996)--(4.128,2.997)--(4.131,2.998)--(4.134,2.999)%
  --(4.137,3.001)--(4.140,3.002)--(4.143,3.003)--(4.146,3.004)--(4.149,3.005)--(4.152,3.006)%
  --(4.155,3.007)--(4.158,3.009)--(4.161,3.010)--(4.164,3.011)--(4.167,3.012)--(4.170,3.013)%
  --(4.173,3.014)--(4.176,3.015)--(4.179,3.016)--(4.182,3.018)--(4.185,3.019)--(4.188,3.020)%
  --(4.191,3.021)--(4.194,3.022)--(4.197,3.023)--(4.200,3.025)--(4.203,3.026)--(4.206,3.027)%
  --(4.209,3.028)--(4.212,3.029)--(4.215,3.030)--(4.218,3.031)--(4.221,3.033)--(4.224,3.034)%
  --(4.227,3.035)--(4.230,3.036)--(4.233,3.037)--(4.236,3.038)--(4.239,3.039)--(4.242,3.041)%
  --(4.245,3.042)--(4.248,3.043)--(4.251,3.044)--(4.254,3.045)--(4.257,3.046)--(4.260,3.048)%
  --(4.263,3.049)--(4.266,3.050)--(4.269,3.051)--(4.272,3.052)--(4.275,3.053)--(4.278,3.055)%
  --(4.281,3.056)--(4.284,3.057)--(4.287,3.058)--(4.290,3.059)--(4.293,3.060)--(4.296,3.062)%
  --(4.299,3.063)--(4.302,3.064)--(4.305,3.065)--(4.308,3.066)--(4.311,3.067)--(4.314,3.069)%
  --(4.317,3.070)--(4.320,3.071)--(4.323,3.072)--(4.326,3.073)--(4.329,3.075)--(4.332,3.076)%
  --(4.334,3.077)--(4.337,3.078)--(4.340,3.079)--(4.343,3.080)--(4.346,3.082)--(4.349,3.083)%
  --(4.352,3.084)--(4.355,3.085)--(4.358,3.086)--(4.361,3.088)--(4.364,3.089)--(4.367,3.090)%
  --(4.370,3.091)--(4.373,3.092)--(4.376,3.093)--(4.379,3.095)--(4.382,3.096)--(4.385,3.097)%
  --(4.388,3.098)--(4.391,3.099)--(4.394,3.101)--(4.397,3.102)--(4.400,3.103)--(4.403,3.104)%
  --(4.406,3.105)--(4.409,3.107)--(4.412,3.108)--(4.415,3.109)--(4.418,3.110)--(4.421,3.111)%
  --(4.424,3.113)--(4.427,3.114)--(4.430,3.115)--(4.433,3.116)--(4.436,3.117)--(4.439,3.119)%
  --(4.442,3.120)--(4.445,3.121)--(4.448,3.122)--(4.451,3.123)--(4.454,3.125)--(4.457,3.126)%
  --(4.460,3.127)--(4.463,3.128)--(4.466,3.129)--(4.469,3.131)--(4.472,3.132)--(4.475,3.133)%
  --(4.478,3.134)--(4.481,3.136)--(4.484,3.137)--(4.487,3.138)--(4.490,3.139)--(4.493,3.140)%
  --(4.496,3.142)--(4.499,3.143)--(4.502,3.144)--(4.505,3.145)--(4.508,3.146)--(4.511,3.148)%
  --(4.514,3.149)--(4.517,3.150)--(4.520,3.151)--(4.523,3.153)--(4.526,3.154)--(4.529,3.155)%
  --(4.532,3.156)--(4.535,3.157)--(4.538,3.159)--(4.541,3.160)--(4.543,3.161)--(4.546,3.162)%
  --(4.549,3.164)--(4.552,3.165)--(4.555,3.166)--(4.558,3.167)--(4.561,3.168)--(4.564,3.170)%
  --(4.567,3.171)--(4.570,3.172)--(4.573,3.173)--(4.576,3.175)--(4.579,3.176)--(4.582,3.177)%
  --(4.585,3.178)--(4.588,3.180)--(4.591,3.181)--(4.594,3.182)--(4.597,3.183)--(4.600,3.184)%
  --(4.603,3.186)--(4.606,3.187)--(4.609,3.188)--(4.612,3.189)--(4.615,3.191)--(4.618,3.192)%
  --(4.621,3.193)--(4.624,3.194)--(4.627,3.196)--(4.630,3.197)--(4.633,3.198)--(4.636,3.199)%
  --(4.639,3.201)--(4.642,3.202)--(4.645,3.203)--(4.648,3.204)--(4.651,3.206)--(4.654,3.207)%
  --(4.657,3.208)--(4.660,3.209)--(4.663,3.211)--(4.666,3.212)--(4.669,3.213)--(4.672,3.214)%
  --(4.675,3.216)--(4.678,3.217)--(4.681,3.218)--(4.684,3.219)--(4.687,3.221)--(4.690,3.222)%
  --(4.693,3.223)--(4.696,3.224)--(4.699,3.226)--(4.702,3.227)--(4.705,3.228)--(4.708,3.229)%
  --(4.711,3.231)--(4.714,3.232)--(4.717,3.233)--(4.720,3.234)--(4.723,3.236)--(4.726,3.237)%
  --(4.729,3.238)--(4.732,3.239)--(4.735,3.241)--(4.738,3.242)--(4.741,3.243)--(4.744,3.244)%
  --(4.747,3.246)--(4.750,3.247)--(4.752,3.248)--(4.755,3.250)--(4.758,3.251)--(4.761,3.252)%
  --(4.764,3.253)--(4.767,3.255)--(4.770,3.256)--(4.773,3.257)--(4.776,3.258)--(4.779,3.260)%
  --(4.782,3.261)--(4.785,3.262)--(4.788,3.263)--(4.791,3.265)--(4.794,3.266)--(4.797,3.267)%
  --(4.800,3.269)--(4.803,3.270)--(4.806,3.271)--(4.809,3.272)--(4.812,3.274)--(4.815,3.275)%
  --(4.818,3.276)--(4.821,3.277)--(4.824,3.279)--(4.827,3.280)--(4.830,3.281)--(4.833,3.283)%
  --(4.836,3.284)--(4.839,3.285)--(4.842,3.286)--(4.845,3.288)--(4.848,3.289)--(4.851,3.290)%
  --(4.854,3.291)--(4.857,3.293)--(4.860,3.294)--(4.863,3.295)--(4.866,3.297)--(4.869,3.298)%
  --(4.872,3.299)--(4.875,3.300)--(4.878,3.302)--(4.881,3.303)--(4.884,3.304)--(4.887,3.306)%
  --(4.890,3.307)--(4.893,3.308)--(4.896,3.309)--(4.899,3.311)--(4.902,3.312)--(4.905,3.313)%
  --(4.908,3.315)--(4.911,3.316)--(4.914,3.317)--(4.917,3.318)--(4.920,3.320)--(4.923,3.321)%
  --(4.926,3.322)--(4.929,3.324)--(4.932,3.325)--(4.935,3.326)--(4.938,3.327)--(4.941,3.329)%
  --(4.944,3.330)--(4.947,3.331)--(4.950,3.333)--(4.953,3.334)--(4.956,3.335)--(4.959,3.337)%
  --(4.961,3.338)--(4.964,3.339)--(4.967,3.340)--(4.970,3.342)--(4.973,3.343)--(4.976,3.344)%
  --(4.979,3.346)--(4.982,3.347)--(4.985,3.348)--(4.988,3.350)--(4.991,3.351)--(4.994,3.352)%
  --(4.997,3.353)--(5.000,3.355)--(5.003,3.356)--(5.006,3.357)--(5.009,3.359)--(5.012,3.360)%
  --(5.015,3.361)--(5.018,3.363)--(5.021,3.364)--(5.024,3.365)--(5.027,3.366)--(5.030,3.368)%
  --(5.033,3.369)--(5.036,3.370)--(5.039,3.372)--(5.042,3.373)--(5.045,3.374)--(5.048,3.376)%
  --(5.051,3.377)--(5.054,3.378)--(5.057,3.380)--(5.060,3.381)--(5.063,3.382)--(5.066,3.383)%
  --(5.069,3.385)--(5.072,3.386)--(5.075,3.387)--(5.078,3.389)--(5.081,3.390)--(5.084,3.391)%
  --(5.087,3.393)--(5.090,3.394)--(5.093,3.395)--(5.096,3.397)--(5.099,3.398)--(5.102,3.399)%
  --(5.105,3.401)--(5.108,3.402)--(5.111,3.403)--(5.114,3.404)--(5.117,3.406)--(5.120,3.407)%
  --(5.123,3.408)--(5.126,3.410)--(5.129,3.411)--(5.132,3.412)--(5.135,3.414)--(5.138,3.415)%
  --(5.141,3.416)--(5.144,3.418)--(5.147,3.419)--(5.150,3.420)--(5.153,3.422)--(5.156,3.423)%
  --(5.159,3.424)--(5.162,3.426)--(5.165,3.427)--(5.168,3.428)--(5.171,3.430)--(5.173,3.431)%
  --(5.176,3.432)--(5.179,3.434)--(5.182,3.435)--(5.185,3.436)--(5.188,3.438)--(5.191,3.439)%
  --(5.194,3.440)--(5.197,3.441)--(5.200,3.443)--(5.203,3.444)--(5.206,3.445)--(5.209,3.447)%
  --(5.212,3.448)--(5.215,3.449)--(5.218,3.451)--(5.221,3.452)--(5.224,3.453)--(5.227,3.455)%
  --(5.230,3.456)--(5.233,3.457)--(5.236,3.459)--(5.239,3.460)--(5.242,3.461)--(5.245,3.463)%
  --(5.248,3.464)--(5.251,3.465)--(5.254,3.467)--(5.257,3.468)--(5.260,3.469)--(5.263,3.471)%
  --(5.266,3.472)--(5.269,3.473)--(5.272,3.475)--(5.275,3.476)--(5.278,3.477)--(5.281,3.479)%
  --(5.284,3.480)--(5.287,3.481)--(5.290,3.483)--(5.293,3.484)--(5.296,3.485)--(5.299,3.487)%
  --(5.302,3.488)--(5.305,3.490)--(5.308,3.491)--(5.311,3.492)--(5.314,3.494)--(5.317,3.495)%
  --(5.320,3.496)--(5.323,3.498)--(5.326,3.499)--(5.329,3.500)--(5.332,3.502)--(5.335,3.503)%
  --(5.338,3.504)--(5.341,3.506)--(5.344,3.507)--(5.347,3.508)--(5.350,3.510)--(5.353,3.511)%
  --(5.356,3.512)--(5.359,3.514)--(5.362,3.515)--(5.365,3.516)--(5.368,3.518)--(5.371,3.519)%
  --(5.374,3.520)--(5.377,3.522)--(5.380,3.523)--(5.382,3.524)--(5.385,3.526)--(5.388,3.527)%
  --(5.391,3.528)--(5.394,3.530)--(5.397,3.531)--(5.400,3.533)--(5.403,3.534)--(5.406,3.535)%
  --(5.409,3.537)--(5.412,3.538)--(5.415,3.539)--(5.418,3.541)--(5.421,3.542)--(5.424,3.543)%
  --(5.427,3.545)--(5.430,3.546)--(5.433,3.547)--(5.436,3.549)--(5.439,3.550)--(5.442,3.551)%
  --(5.445,3.553)--(5.448,3.554)--(5.451,3.556)--(5.454,3.557)--(5.457,3.558)--(5.460,3.560)%
  --(5.463,3.561)--(5.466,3.562)--(5.469,3.564)--(5.472,3.565)--(5.475,3.566)--(5.478,3.568)%
  --(5.481,3.569)--(5.484,3.570)--(5.487,3.572)--(5.490,3.573)--(5.493,3.575)--(5.496,3.576)%
  --(5.499,3.577)--(5.502,3.579)--(5.505,3.580)--(5.508,3.581)--(5.511,3.583)--(5.514,3.584)%
  --(5.517,3.585)--(5.520,3.587)--(5.523,3.588)--(5.526,3.590)--(5.529,3.591)--(5.532,3.592)%
  --(5.535,3.594)--(5.538,3.595)--(5.541,3.596)--(5.544,3.598)--(5.547,3.599)--(5.550,3.600)%
  --(5.553,3.602)--(5.556,3.603)--(5.559,3.605)--(5.562,3.606)--(5.565,3.607)--(5.568,3.609)%
  --(5.571,3.610)--(5.574,3.611)--(5.577,3.613)--(5.580,3.614)--(5.583,3.615)--(5.586,3.617)%
  --(5.589,3.618)--(5.591,3.620)--(5.594,3.621)--(5.597,3.622)--(5.600,3.624)--(5.603,3.625)%
  --(5.606,3.626)--(5.609,3.628)--(5.612,3.629)--(5.615,3.631)--(5.618,3.632)--(5.621,3.633)%
  --(5.624,3.635)--(5.627,3.636)--(5.630,3.637)--(5.633,3.639)--(5.636,3.640)--(5.639,3.642)%
  --(5.642,3.643)--(5.645,3.644)--(5.648,3.646)--(5.651,3.647)--(5.654,3.648)--(5.657,3.650)%
  --(5.660,3.651)--(5.663,3.652)--(5.666,3.654)--(5.669,3.655)--(5.672,3.657)--(5.675,3.658)%
  --(5.678,3.659)--(5.681,3.661)--(5.684,3.662)--(5.687,3.664)--(5.690,3.665)--(5.693,3.666)%
  --(5.696,3.668)--(5.699,3.669)--(5.702,3.670)--(5.705,3.672)--(5.708,3.673)--(5.711,3.675)%
  --(5.714,3.676)--(5.717,3.677)--(5.720,3.679)--(5.723,3.680)--(5.726,3.681)--(5.729,3.683)%
  --(5.732,3.684)--(5.735,3.686)--(5.738,3.687)--(5.741,3.688)--(5.744,3.690)--(5.747,3.691)%
  --(5.750,3.692)--(5.753,3.694)--(5.756,3.695)--(5.759,3.697)--(5.762,3.698)--(5.765,3.699)%
  --(5.768,3.701)--(5.771,3.702)--(5.774,3.704)--(5.777,3.705)--(5.780,3.706)--(5.783,3.708)%
  --(5.786,3.709)--(5.789,3.710)--(5.792,3.712)--(5.795,3.713)--(5.798,3.715)--(5.800,3.716)%
  --(5.803,3.717)--(5.806,3.719)--(5.809,3.720)--(5.812,3.722)--(5.815,3.723)--(5.818,3.724)%
  --(5.821,3.726)--(5.824,3.727)--(5.827,3.729)--(5.830,3.730)--(5.833,3.731)--(5.836,3.733)%
  --(5.839,3.734)--(5.842,3.735)--(5.845,3.737)--(5.848,3.738)--(5.851,3.740)--(5.854,3.741)%
  --(5.857,3.742)--(5.860,3.744)--(5.863,3.745)--(5.866,3.747)--(5.869,3.748)--(5.872,3.749)%
  --(5.875,3.751)--(5.878,3.752)--(5.881,3.754)--(5.884,3.755)--(5.887,3.756)--(5.890,3.758)%
  --(5.893,3.759)--(5.896,3.761)--(5.899,3.762)--(5.902,3.763)--(5.905,3.765)--(5.908,3.766)%
  --(5.911,3.767)--(5.914,3.769)--(5.917,3.770)--(5.920,3.772)--(5.923,3.773)--(5.926,3.774)%
  --(5.929,3.776)--(5.932,3.777)--(5.935,3.779)--(5.938,3.780)--(5.941,3.781)--(5.944,3.783)%
  --(5.947,3.784)--(5.950,3.786)--(5.953,3.787)--(5.956,3.788)--(5.959,3.790)--(5.962,3.791)%
  --(5.965,3.793)--(5.968,3.794)--(5.971,3.795)--(5.974,3.797)--(5.977,3.798)--(5.980,3.800)%
  --(5.983,3.801)--(5.986,3.802)--(5.989,3.804)--(5.992,3.805)--(5.995,3.807)--(5.998,3.808)%
  --(6.001,3.809)--(6.004,3.811)--(6.007,3.812)--(6.009,3.814)--(6.012,3.815)--(6.015,3.816)%
  --(6.018,3.818)--(6.021,3.819)--(6.024,3.821)--(6.027,3.822)--(6.030,3.823)--(6.033,3.825)%
  --(6.036,3.826)--(6.039,3.828)--(6.042,3.829)--(6.045,3.830)--(6.048,3.832)--(6.051,3.833)%
  --(6.054,3.835)--(6.057,3.836)--(6.060,3.838)--(6.063,3.839)--(6.066,3.840)--(6.069,3.842)%
  --(6.072,3.843)--(6.075,3.845)--(6.078,3.846)--(6.081,3.847)--(6.084,3.849)--(6.087,3.850)%
  --(6.090,3.852)--(6.093,3.853)--(6.096,3.854)--(6.099,3.856)--(6.102,3.857)--(6.105,3.859)%
  --(6.108,3.860)--(6.111,3.861)--(6.114,3.863)--(6.117,3.864)--(6.120,3.866)--(6.123,3.867)%
  --(6.126,3.868)--(6.129,3.870)--(6.132,3.871)--(6.135,3.873)--(6.138,3.874)--(6.141,3.876)%
  --(6.144,3.877)--(6.147,3.878)--(6.150,3.880)--(6.153,3.881)--(6.156,3.883)--(6.159,3.884)%
  --(6.162,3.885)--(6.165,3.887)--(6.168,3.888)--(6.171,3.890)--(6.174,3.891)--(6.177,3.892)%
  --(6.180,3.894)--(6.183,3.895)--(6.186,3.897)--(6.189,3.898)--(6.192,3.900)--(6.195,3.901)%
  --(6.198,3.902)--(6.201,3.904)--(6.204,3.905)--(6.207,3.907)--(6.210,3.908)--(6.213,3.909)%
  --(6.216,3.911)--(6.218,3.912)--(6.221,3.914)--(6.224,3.915)--(6.227,3.917)--(6.230,3.918)%
  --(6.233,3.919)--(6.236,3.921)--(6.239,3.922)--(6.242,3.924)--(6.245,3.925)--(6.248,3.926)%
  --(6.251,3.928)--(6.254,3.929)--(6.257,3.931)--(6.260,3.932)--(6.263,3.934)--(6.266,3.935)%
  --(6.269,3.936)--(6.272,3.938)--(6.275,3.939)--(6.278,3.941)--(6.281,3.942)--(6.284,3.943)%
  --(6.287,3.945)--(6.290,3.946)--(6.293,3.948)--(6.296,3.949)--(6.299,3.951)--(6.302,3.952)%
  --(6.305,3.953)--(6.308,3.955)--(6.311,3.956)--(6.314,3.958)--(6.317,3.959)--(6.320,3.960)%
  --(6.323,3.962)--(6.326,3.963)--(6.329,3.965)--(6.332,3.966)--(6.335,3.968)--(6.338,3.969)%
  --(6.341,3.970)--(6.344,3.972)--(6.347,3.973)--(6.350,3.975)--(6.353,3.976)--(6.356,3.978)%
  --(6.359,3.979)--(6.362,3.980)--(6.365,3.982)--(6.368,3.983)--(6.371,3.985)--(6.374,3.986)%
  --(6.377,3.988)--(6.380,3.989)--(6.383,3.990)--(6.386,3.992)--(6.389,3.993)--(6.392,3.995)%
  --(6.395,3.996)--(6.398,3.997)--(6.401,3.999)--(6.404,4.000)--(6.407,4.002)--(6.410,4.003)%
  --(6.413,4.005)--(6.416,4.006)--(6.419,4.007)--(6.422,4.009)--(6.425,4.010)--(6.428,4.012)%
  --(6.430,4.013)--(6.433,4.015)--(6.436,4.016)--(6.439,4.017)--(6.442,4.019)--(6.445,4.020)%
  --(6.448,4.022)--(6.451,4.023)--(6.454,4.025)--(6.457,4.026)--(6.460,4.027)--(6.463,4.029)%
  --(6.466,4.030)--(6.469,4.032)--(6.472,4.033)--(6.475,4.035)--(6.478,4.036)--(6.481,4.037)%
  --(6.484,4.039)--(6.487,4.040)--(6.490,4.042)--(6.493,4.043)--(6.496,4.045)--(6.499,4.046)%
  --(6.502,4.047)--(6.505,4.049)--(6.508,4.050)--(6.511,4.052)--(6.514,4.053)--(6.517,4.055)%
  --(6.520,4.056)--(6.523,4.058)--(6.526,4.059)--(6.529,4.060)--(6.532,4.062)--(6.535,4.063)%
  --(6.538,4.065)--(6.541,4.066)--(6.544,4.068)--(6.547,4.069)--(6.550,4.070)--(6.553,4.072)%
  --(6.556,4.073)--(6.559,4.075)--(6.562,4.076)--(6.565,4.078)--(6.568,4.079)--(6.571,4.080)%
  --(6.574,4.082)--(6.577,4.083)--(6.580,4.085)--(6.583,4.086)--(6.586,4.088)--(6.589,4.089)%
  --(6.592,4.090)--(6.595,4.092)--(6.598,4.093)--(6.601,4.095)--(6.604,4.096)--(6.607,4.098)%
  --(6.610,4.099)--(6.613,4.101)--(6.616,4.102)--(6.619,4.103)--(6.622,4.105)--(6.625,4.106)%
  --(6.628,4.108)--(6.631,4.109)--(6.634,4.111)--(6.637,4.112)--(6.639,4.113)--(6.642,4.115)%
  --(6.645,4.116)--(6.648,4.118)--(6.651,4.119)--(6.654,4.121)--(6.657,4.122)--(6.660,4.124)%
  --(6.663,4.125)--(6.666,4.126)--(6.669,4.128)--(6.672,4.129)--(6.675,4.131)--(6.678,4.132)%
  --(6.681,4.134)--(6.684,4.135)--(6.687,4.136)--(6.690,4.138)--(6.693,4.139)--(6.696,4.141)%
  --(6.699,4.142)--(6.702,4.144)--(6.705,4.145)--(6.708,4.147)--(6.711,4.148)--(6.714,4.149)%
  --(6.717,4.151)--(6.720,4.152)--(6.723,4.154)--(6.726,4.155)--(6.729,4.157)--(6.732,4.158)%
  --(6.735,4.160)--(6.738,4.161)--(6.741,4.162)--(6.744,4.164)--(6.747,4.165)--(6.750,4.167)%
  --(6.753,4.168)--(6.756,4.170)--(6.759,4.171)--(6.762,4.172)--(6.765,4.174)--(6.768,4.175)%
  --(6.771,4.177)--(6.774,4.178)--(6.777,4.180)--(6.780,4.181)--(6.783,4.183)--(6.786,4.184)%
  --(6.789,4.185)--(6.792,4.187)--(6.795,4.188)--(6.798,4.190)--(6.801,4.191)--(6.804,4.193)%
  --(6.807,4.194)--(6.810,4.196)--(6.813,4.197)--(6.816,4.198)--(6.819,4.200)--(6.822,4.201)%
  --(6.825,4.203)--(6.828,4.204)--(6.831,4.206)--(6.834,4.207)--(6.837,4.209)--(6.840,4.210)%
  --(6.843,4.211)--(6.846,4.213)--(6.848,4.214)--(6.851,4.216)--(6.854,4.217)--(6.857,4.219)%
  --(6.860,4.220)--(6.863,4.222)--(6.866,4.223)--(6.869,4.225)--(6.872,4.226)--(6.875,4.227)%
  --(6.878,4.229)--(6.881,4.230)--(6.884,4.232)--(6.887,4.233)--(6.890,4.235)--(6.893,4.236)%
  --(6.896,4.238)--(6.899,4.239)--(6.902,4.240)--(6.905,4.242)--(6.908,4.243)--(6.911,4.245)%
  --(6.914,4.246)--(6.917,4.248)--(6.920,4.249)--(6.923,4.251)--(6.926,4.252)--(6.929,4.253)%
  --(6.932,4.255)--(6.935,4.256)--(6.938,4.258)--(6.941,4.259)--(6.944,4.261)--(6.947,4.262)%
  --(6.950,4.264)--(6.953,4.265)--(6.956,4.267)--(6.959,4.268)--(6.962,4.269)--(6.965,4.271)%
  --(6.968,4.272)--(6.971,4.274)--(6.974,4.275)--(6.977,4.277)--(6.980,4.278)--(6.983,4.280)%
  --(6.986,4.281)--(6.989,4.282)--(6.992,4.284)--(6.995,4.285)--(6.998,4.287)--(7.001,4.288)%
  --(7.004,4.290)--(7.007,4.291)--(7.010,4.293)--(7.013,4.294)--(7.016,4.296)--(7.019,4.297)%
  --(7.022,4.298)--(7.025,4.300)--(7.028,4.301)--(7.031,4.303)--(7.034,4.304)--(7.037,4.306)%
  --(7.040,4.307)--(7.043,4.309)--(7.046,4.310)--(7.049,4.312)--(7.052,4.313)--(7.055,4.314)%
  --(7.057,4.316)--(7.060,4.317)--(7.063,4.319)--(7.066,4.320)--(7.069,4.322)--(7.072,4.323)%
  --(7.075,4.325)--(7.078,4.326)--(7.081,4.328)--(7.084,4.329)--(7.087,4.330)--(7.090,4.332)%
  --(7.093,4.333)--(7.096,4.335)--(7.099,4.336)--(7.102,4.338)--(7.105,4.339)--(7.108,4.341)%
  --(7.111,4.342)--(7.114,4.344)--(7.117,4.345)--(7.120,4.346)--(7.123,4.348)--(7.126,4.349)%
  --(7.129,4.351)--(7.132,4.352)--(7.135,4.354)--(7.138,4.355)--(7.141,4.357)--(7.144,4.358)%
  --(7.147,4.360)--(7.150,4.361)--(7.153,4.362)--(7.156,4.364)--(7.159,4.365)--(7.162,4.367)%
  --(7.165,4.368)--(7.168,4.370)--(7.171,4.371)--(7.174,4.373)--(7.177,4.374)--(7.180,4.376)%
  --(7.183,4.377)--(7.186,4.379)--(7.189,4.380)--(7.192,4.381)--(7.195,4.383)--(7.198,4.384)%
  --(7.201,4.386)--(7.204,4.387)--(7.207,4.389)--(7.210,4.390)--(7.213,4.392)--(7.216,4.393)%
  --(7.219,4.395)--(7.222,4.396)--(7.225,4.397)--(7.228,4.399)--(7.231,4.400)--(7.234,4.402)%
  --(7.237,4.403)--(7.240,4.405)--(7.243,4.406)--(7.246,4.408)--(7.249,4.409)--(7.252,4.411)%
  --(7.255,4.412)--(7.258,4.414)--(7.261,4.415)--(7.264,4.416)--(7.266,4.418)--(7.269,4.419)%
  --(7.272,4.421)--(7.275,4.422)--(7.278,4.424)--(7.281,4.425)--(7.284,4.427)--(7.287,4.428)%
  --(7.290,4.430)--(7.293,4.431)--(7.296,4.433)--(7.299,4.434)--(7.302,4.435)--(7.305,4.437)%
  --(7.308,4.438)--(7.311,4.440)--(7.314,4.441)--(7.317,4.443)--(7.320,4.444)--(7.323,4.446)%
  --(7.326,4.447)--(7.329,4.449)--(7.332,4.450)--(7.335,4.452)--(7.338,4.453)--(7.341,4.454)%
  --(7.344,4.456)--(7.347,4.457)--(7.350,4.459)--(7.353,4.460)--(7.356,4.462)--(7.359,4.463)%
  --(7.362,4.465)--(7.365,4.466)--(7.368,4.468)--(7.371,4.469)--(7.374,4.471)--(7.377,4.472)%
  --(7.380,4.474)--(7.383,4.475)--(7.386,4.476)--(7.389,4.478)--(7.392,4.479)--(7.395,4.481)%
  --(7.398,4.482)--(7.401,4.484)--(7.404,4.485)--(7.407,4.487)--(7.410,4.488)--(7.413,4.490)%
  --(7.416,4.491)--(7.419,4.493)--(7.422,4.494)--(7.425,4.495)--(7.428,4.497)--(7.431,4.498)%
  --(7.434,4.500)--(7.437,4.501)--(7.440,4.503)--(7.443,4.504)--(7.446,4.506)--(7.449,4.507)%
  --(7.452,4.509)--(7.455,4.510)--(7.458,4.512)--(7.461,4.513)--(7.464,4.515)--(7.467,4.516)%
  --(7.470,4.517)--(7.473,4.519)--(7.476,4.520)--(7.478,4.522)--(7.481,4.523)--(7.484,4.525)%
  --(7.487,4.526)--(7.490,4.528)--(7.493,4.529)--(7.496,4.531)--(7.499,4.532)--(7.502,4.534)%
  --(7.505,4.535)--(7.508,4.537)--(7.511,4.538)--(7.514,4.539)--(7.517,4.541)--(7.520,4.542)%
  --(7.523,4.544)--(7.526,4.545)--(7.529,4.547)--(7.532,4.548)--(7.535,4.550)--(7.538,4.551)%
  --(7.541,4.553)--(7.544,4.554)--(7.547,4.556)--(7.550,4.557)--(7.553,4.559)--(7.556,4.560)%
  --(7.559,4.562)--(7.562,4.563)--(7.565,4.564)--(7.568,4.566)--(7.571,4.567)--(7.574,4.569)%
  --(7.577,4.570)--(7.580,4.572)--(7.583,4.573)--(7.586,4.575)--(7.589,4.576)--(7.592,4.578)%
  --(7.595,4.579)--(7.598,4.581)--(7.601,4.582)--(7.604,4.584)--(7.607,4.585)--(7.610,4.587)%
  --(7.613,4.588)--(7.616,4.589)--(7.619,4.591)--(7.622,4.592)--(7.625,4.594)--(7.628,4.595)%
  --(7.631,4.597)--(7.634,4.598)--(7.637,4.600)--(7.640,4.601)--(7.643,4.603)--(7.646,4.604)%
  --(7.649,4.606)--(7.652,4.607)--(7.655,4.609)--(7.658,4.610)--(7.661,4.612)--(7.664,4.613)%
  --(7.667,4.614)--(7.670,4.616)--(7.673,4.617)--(7.676,4.619)--(7.679,4.620)--(7.682,4.622)%
  --(7.685,4.623)--(7.687,4.625)--(7.690,4.626)--(7.693,4.628)--(7.696,4.629)--(7.699,4.631)%
  --(7.702,4.632)--(7.705,4.634)--(7.708,4.635)--(7.711,4.637)--(7.714,4.638)--(7.717,4.640)%
  --(7.720,4.641)--(7.723,4.642)--(7.726,4.644)--(7.729,4.645)--(7.732,4.647)--(7.735,4.648)%
  --(7.738,4.650)--(7.741,4.651)--(7.744,4.653)--(7.747,4.654)--(7.750,4.656)--(7.753,4.657)%
  --(7.756,4.659)--(7.759,4.660)--(7.762,4.662)--(7.765,4.663)--(7.768,4.665)--(7.771,4.666)%
  --(7.774,4.668)--(7.777,4.669)--(7.780,4.670)--(7.783,4.672)--(7.786,4.673)--(7.789,4.675)%
  --(7.792,4.676)--(7.795,4.678)--(7.798,4.679)--(7.801,4.681)--(7.804,4.682)--(7.807,4.684)%
  --(7.810,4.685)--(7.813,4.687)--(7.816,4.688)--(7.819,4.690)--(7.822,4.691)--(7.825,4.693)%
  --(7.828,4.694)--(7.831,4.696)--(7.834,4.697)--(7.837,4.698)--(7.840,4.700)--(7.843,4.701)%
  --(7.846,4.703)--(7.849,4.704)--(7.852,4.706)--(7.855,4.707)--(7.858,4.709)--(7.861,4.710)%
  --(7.864,4.712)--(7.867,4.713)--(7.870,4.715)--(7.873,4.716)--(7.876,4.718)--(7.879,4.719)%
  --(7.882,4.721)--(7.885,4.722)--(7.888,4.724)--(7.891,4.725)--(7.894,4.727)--(7.896,4.728)%
  --(7.899,4.730)--(7.902,4.731)--(7.905,4.732)--(7.908,4.734)--(7.911,4.735)--(7.914,4.737)%
  --(7.917,4.738)--(7.920,4.740)--(7.923,4.741)--(7.926,4.743)--(7.929,4.744)--(7.932,4.746)%
  --(7.935,4.747)--(7.938,4.749)--(7.941,4.750)--(7.944,4.752)--(7.947,4.753)--(7.950,4.755)%
  --(7.953,4.756)--(7.956,4.758)--(7.959,4.759)--(7.962,4.761)--(7.965,4.762)--(7.968,4.764)%
  --(7.971,4.765)--(7.974,4.766)--(7.977,4.768)--(7.980,4.769)--(7.983,4.771)--(7.986,4.772)%
  --(7.989,4.774)--(7.992,4.775)--(7.995,4.777)--(7.998,4.778)--(8.001,4.780)--(8.004,4.781)%
  --(8.007,4.783)--(8.010,4.784)--(8.013,4.786)--(8.016,4.787)--(8.019,4.789)--(8.022,4.790)%
  --(8.025,4.792)--(8.028,4.793)--(8.031,4.795)--(8.034,4.796)--(8.037,4.798)--(8.040,4.799)%
  --(8.043,4.800)--(8.046,4.802)--(8.049,4.803)--(8.052,4.805)--(8.055,4.806)--(8.058,4.808)%
  --(8.061,4.809)--(8.064,4.811)--(8.067,4.812)--(8.070,4.814)--(8.073,4.815)--(8.076,4.817)%
  --(8.079,4.818)--(8.082,4.820)--(8.085,4.821)--(8.088,4.823)--(8.091,4.824)--(8.094,4.826)%
  --(8.097,4.827)--(8.100,4.829)--(8.103,4.830)--(8.105,4.832)--(8.108,4.833)--(8.111,4.835)%
  --(8.114,4.836)--(8.117,4.838)--(8.120,4.839)--(8.123,4.840)--(8.126,4.842)--(8.129,4.843)%
  --(8.132,4.845)--(8.135,4.846)--(8.138,4.848)--(8.141,4.849)--(8.144,4.851)--(8.147,4.852)%
  --(8.150,4.854)--(8.153,4.855)--(8.156,4.857)--(8.159,4.858)--(8.162,4.860)--(8.165,4.861)%
  --(8.168,4.863)--(8.171,4.864)--(8.174,4.866)--(8.177,4.867)--(8.180,4.869)--(8.183,4.870)%
  --(8.186,4.872)--(8.189,4.873)--(8.192,4.875)--(8.195,4.876)--(8.198,4.878)--(8.201,4.879)%
  --(8.204,4.881)--(8.207,4.882)--(8.210,4.883)--(8.213,4.885)--(8.216,4.886)--(8.219,4.888)%
  --(8.222,4.889)--(8.225,4.891)--(8.228,4.892)--(8.231,4.894)--(8.234,4.895)--(8.237,4.897)%
  --(8.240,4.898)--(8.243,4.900)--(8.246,4.901)--(8.249,4.903)--(8.252,4.904)--(8.255,4.906)%
  --(8.258,4.907)--(8.261,4.909)--(8.264,4.910)--(8.267,4.912)--(8.270,4.913)--(8.273,4.915)%
  --(8.276,4.916)--(8.279,4.918)--(8.282,4.919)--(8.285,4.921)--(8.288,4.922)--(8.291,4.924)%
  --(8.294,4.925)--(8.297,4.927)--(8.300,4.928)--(8.303,4.930)--(8.306,4.931)--(8.309,4.932)%
  --(8.312,4.934)--(8.314,4.935)--(8.317,4.937)--(8.320,4.938)--(8.323,4.940)--(8.326,4.941)%
  --(8.329,4.943)--(8.332,4.944)--(8.335,4.946)--(8.338,4.947)--(8.341,4.949)--(8.344,4.950)%
  --(8.347,4.952)--(8.350,4.953)--(8.353,4.955)--(8.356,4.956)--(8.359,4.958)--(8.362,4.959)%
  --(8.365,4.961)--(8.368,4.962)--(8.371,4.964)--(8.374,4.965)--(8.377,4.967)--(8.380,4.968)%
  --(8.383,4.970)--(8.386,4.971)--(8.389,4.973)--(8.392,4.974)--(8.395,4.976)--(8.398,4.977)%
  --(8.401,4.979)--(8.404,4.980)--(8.407,4.982)--(8.410,4.983)--(8.413,4.985)--(8.416,4.986)%
  --(8.419,4.987)--(8.422,4.989)--(8.425,4.990)--(8.428,4.992)--(8.431,4.993)--(8.434,4.995)%
  --(8.437,4.996)--(8.440,4.998)--(8.443,4.999)--(8.446,5.001)--(8.449,5.002)--(8.452,5.004)%
  --(8.455,5.005)--(8.458,5.007)--(8.461,5.008)--(8.464,5.010)--(8.467,5.011)--(8.470,5.013)%
  --(8.473,5.014)--(8.476,5.016)--(8.479,5.017)--(8.482,5.019)--(8.485,5.020)--(8.488,5.022)%
  --(8.491,5.023)--(8.494,5.025)--(8.497,5.026)--(8.500,5.028)--(8.503,5.029)--(8.506,5.031)%
  --(8.509,5.032)--(8.512,5.034)--(8.515,5.035)--(8.518,5.037)--(8.521,5.038)--(8.523,5.040)%
  --(8.526,5.041)--(8.529,5.043)--(8.532,5.044)--(8.535,5.046)--(8.538,5.047)--(8.541,5.048)%
  --(8.544,5.050)--(8.547,5.051)--(8.550,5.053)--(8.553,5.054)--(8.556,5.056)--(8.559,5.057)%
  --(8.562,5.059)--(8.565,5.060)--(8.568,5.062)--(8.571,5.063)--(8.574,5.065)--(8.577,5.066)%
  --(8.580,5.068)--(8.583,5.069)--(8.586,5.071)--(8.589,5.072)--(8.592,5.074)--(8.595,5.075)%
  --(8.598,5.077)--(8.601,5.078)--(8.604,5.080)--(8.607,5.081)--(8.610,5.083)--(8.613,5.084)%
  --(8.616,5.086)--(8.619,5.087)--(8.622,5.089)--(8.625,5.090)--(8.628,5.092)--(8.631,5.093)%
  --(8.634,5.095)--(8.637,5.096)--(8.640,5.098)--(8.643,5.099)--(8.646,5.101)--(8.649,5.102)%
  --(8.652,5.104)--(8.655,5.105)--(8.658,5.107)--(8.661,5.108)--(8.664,5.110)--(8.667,5.111)%
  --(8.670,5.113)--(8.673,5.114)--(8.676,5.116)--(8.679,5.117)--(8.682,5.119)--(8.685,5.120)%
  --(8.688,5.122)--(8.691,5.123)--(8.694,5.125)--(8.697,5.126)--(8.700,5.127)--(8.703,5.129)%
  --(8.706,5.130)--(8.709,5.132)--(8.712,5.133)--(8.715,5.135)--(8.718,5.136)--(8.721,5.138)%
  --(8.724,5.139)--(8.727,5.141)--(8.730,5.142)--(8.733,5.144)--(8.735,5.145)--(8.738,5.147)%
  --(8.741,5.148)--(8.744,5.150)--(8.747,5.151)--(8.750,5.153)--(8.753,5.154)--(8.756,5.156)%
  --(8.759,5.157)--(8.762,5.159)--(8.765,5.160)--(8.768,5.162)--(8.771,5.163)--(8.774,5.165)%
  --(8.777,5.166)--(8.780,5.168)--(8.783,5.169)--(8.786,5.171)--(8.789,5.172)--(8.792,5.174)%
  --(8.795,5.175)--(8.798,5.177)--(8.801,5.178)--(8.804,5.180)--(8.807,5.181)--(8.810,5.183)%
  --(8.813,5.184)--(8.816,5.186)--(8.819,5.187)--(8.822,5.189)--(8.825,5.190)--(8.828,5.192)%
  --(8.831,5.193)--(8.834,5.195)--(8.837,5.196)--(8.840,5.198)--(8.843,5.199)--(8.846,5.201)%
  --(8.849,5.202)--(8.852,5.204)--(8.855,5.205)--(8.858,5.207)--(8.861,5.208)--(8.864,5.210)%
  --(8.867,5.211)--(8.870,5.213)--(8.873,5.214)--(8.876,5.216)--(8.879,5.217)--(8.882,5.219)%
  --(8.885,5.220)--(8.888,5.222)--(8.891,5.223)--(8.894,5.225)--(8.897,5.226)--(8.900,5.228)%
  --(8.903,5.229)--(8.906,5.231)--(8.909,5.232)--(8.912,5.234)--(8.915,5.235)--(8.918,5.236)%
  --(8.921,5.238)--(8.924,5.239)--(8.927,5.241)--(8.930,5.242)--(8.933,5.244)--(8.936,5.245)%
  --(8.939,5.247)--(8.942,5.248)--(8.944,5.250)--(8.947,5.251)--(8.950,5.253)--(8.953,5.254)%
  --(8.956,5.256)--(8.959,5.257)--(8.962,5.259)--(8.965,5.260)--(8.968,5.262)--(8.971,5.263)%
  --(8.974,5.265)--(8.977,5.266)--(8.980,5.268)--(8.983,5.269)--(8.986,5.271)--(8.989,5.272)%
  --(8.992,5.274)--(8.995,5.275)--(8.998,5.277)--(9.001,5.278)--(9.004,5.280)--(9.007,5.281)%
  --(9.010,5.283)--(9.013,5.284)--(9.016,5.286)--(9.019,5.287)--(9.022,5.289)--(9.025,5.290)%
  --(9.028,5.292)--(9.031,5.293)--(9.034,5.295)--(9.037,5.296)--(9.040,5.298)--(9.043,5.299)%
  --(9.046,5.301)--(9.049,5.302)--(9.052,5.304)--(9.055,5.305)--(9.058,5.307)--(9.061,5.308)%
  --(9.064,5.310)--(9.067,5.311)--(9.070,5.313)--(9.073,5.314)--(9.076,5.316)--(9.079,5.317)%
  --(9.082,5.319)--(9.085,5.320)--(9.088,5.322)--(9.091,5.323)--(9.094,5.325)--(9.097,5.326)%
  --(9.100,5.328)--(9.103,5.329)--(9.106,5.331)--(9.109,5.332)--(9.112,5.334)--(9.115,5.335)%
  --(9.118,5.337)--(9.121,5.338)--(9.124,5.340)--(9.127,5.341)--(9.130,5.343)--(9.133,5.344)%
  --(9.136,5.346)--(9.139,5.347)--(9.142,5.349)--(9.145,5.350)--(9.148,5.352)--(9.151,5.353)%
  --(9.153,5.355)--(9.156,5.356)--(9.159,5.358)--(9.162,5.359)--(9.165,5.361)--(9.168,5.362)%
  --(9.171,5.364)--(9.174,5.365)--(9.177,5.367)--(9.180,5.368)--(9.183,5.370)--(9.186,5.371)%
  --(9.189,5.373)--(9.192,5.374)--(9.195,5.376)--(9.198,5.377)--(9.201,5.379)--(9.204,5.380)%
  --(9.207,5.382)--(9.210,5.383)--(9.213,5.385)--(9.216,5.386)--(9.219,5.388)--(9.222,5.389)%
  --(9.225,5.391)--(9.228,5.392)--(9.231,5.394)--(9.234,5.395)--(9.237,5.397)--(9.240,5.398)%
  --(9.243,5.400)--(9.246,5.401)--(9.249,5.403)--(9.252,5.404)--(9.255,5.406)--(9.258,5.407)%
  --(9.261,5.409)--(9.264,5.410)--(9.267,5.412)--(9.270,5.413)--(9.273,5.415)--(9.276,5.416)%
  --(9.279,5.418)--(9.282,5.419)--(9.285,5.421)--(9.288,5.422)--(9.291,5.424)--(9.294,5.425)%
  --(9.297,5.427)--(9.300,5.428)--(9.303,5.430)--(9.306,5.431)--(9.309,5.433)--(9.312,5.434)%
  --(9.315,5.436)--(9.318,5.437)--(9.321,5.439)--(9.324,5.440)--(9.327,5.442)--(9.330,5.443)%
  --(9.333,5.445)--(9.336,5.446)--(9.339,5.448)--(9.342,5.449)--(9.345,5.451)--(9.348,5.452)%
  --(9.351,5.454)--(9.354,5.455)--(9.357,5.457)--(9.360,5.458)--(9.362,5.460)--(9.365,5.461)%
  --(9.368,5.463)--(9.371,5.464)--(9.374,5.466)--(9.377,5.467)--(9.380,5.469)--(9.383,5.470)%
  --(9.386,5.472)--(9.389,5.473)--(9.392,5.475)--(9.395,5.476)--(9.398,5.478)--(9.401,5.479)%
  --(9.404,5.481)--(9.407,5.482)--(9.410,5.484)--(9.413,5.485)--(9.416,5.487)--(9.419,5.488)%
  --(9.422,5.490)--(9.425,5.491)--(9.428,5.493)--(9.431,5.494)--(9.434,5.496)--(9.437,5.497)%
  --(9.440,5.499)--(9.443,5.500)--(9.446,5.502)--(9.449,5.503)--(9.452,5.505)--(9.455,5.506)%
  --(9.458,5.508)--(9.461,5.509)--(9.464,5.511)--(9.467,5.512)--(9.470,5.514)--(9.473,5.515)%
  --(9.476,5.517)--(9.479,5.518)--(9.482,5.520)--(9.485,5.521)--(9.488,5.523)--(9.491,5.524)%
  --(9.494,5.526)--(9.497,5.527)--(9.500,5.529)--(9.503,5.530)--(9.506,5.532)--(9.509,5.533)%
  --(9.512,5.535)--(9.515,5.536)--(9.518,5.538)--(9.521,5.539)--(9.524,5.541)--(9.527,5.542)%
  --(9.530,5.544)--(9.533,5.545)--(9.536,5.547)--(9.539,5.548)--(9.542,5.550)--(9.545,5.551)%
  --(9.548,5.553)--(9.551,5.554)--(9.554,5.556)--(9.557,5.557)--(9.560,5.559)--(9.563,5.560)%
  --(9.566,5.562)--(9.569,5.563)--(9.571,5.565)--(9.574,5.566)--(9.577,5.568)--(9.580,5.569)%
  --(9.583,5.571)--(9.586,5.572)--(9.589,5.574)--(9.592,5.575)--(9.595,5.577)--(9.598,5.578)%
  --(9.601,5.580)--(9.604,5.581)--(9.607,5.583)--(9.610,5.584)--(9.613,5.586)--(9.616,5.587)%
  --(9.619,5.589)--(9.622,5.590)--(9.625,5.592)--(9.628,5.593)--(9.631,5.595)--(9.634,5.596)%
  --(9.637,5.598)--(9.640,5.599)--(9.643,5.601)--(9.646,5.602)--(9.649,5.604)--(9.652,5.605)%
  --(9.655,5.607)--(9.658,5.608)--(9.661,5.610)--(9.664,5.611)--(9.667,5.613)--(9.670,5.614)%
  --(9.673,5.616)--(9.676,5.617)--(9.679,5.619)--(9.682,5.620)--(9.685,5.622)--(9.688,5.623)%
  --(9.691,5.625)--(9.694,5.626)--(9.697,5.628)--(9.700,5.629)--(9.703,5.631)--(9.706,5.632)%
  --(9.709,5.634)--(9.712,5.635)--(9.715,5.637)--(9.718,5.638)--(9.721,5.640)--(9.724,5.641)%
  --(9.727,5.643)--(9.730,5.644)--(9.733,5.646)--(9.736,5.647)--(9.739,5.649)--(9.742,5.650)%
  --(9.745,5.652)--(9.748,5.653)--(9.751,5.655)--(9.754,5.656)--(9.757,5.658)--(9.760,5.659)%
  --(9.763,5.661)--(9.766,5.662)--(9.769,5.664)--(9.772,5.665)--(9.775,5.667)--(9.778,5.668)%
  --(9.780,5.670)--(9.783,5.671)--(9.786,5.673)--(9.789,5.674)--(9.792,5.676)--(9.795,5.677)%
  --(9.798,5.679)--(9.801,5.680)--(9.804,5.682)--(9.807,5.683)--(9.810,5.685)--(9.813,5.686)%
  --(9.816,5.688)--(9.819,5.689)--(9.822,5.691)--(9.825,5.692)--(9.828,5.694)--(9.831,5.695)%
  --(9.834,5.697)--(9.837,5.698)--(9.840,5.700)--(9.843,5.701)--(9.846,5.703)--(9.849,5.704)%
  --(9.852,5.706)--(9.855,5.707)--(9.858,5.709)--(9.861,5.710)--(9.864,5.712)--(9.867,5.713)%
  --(9.870,5.715)--(9.873,5.716)--(9.876,5.718)--(9.879,5.719)--(9.882,5.721)--(9.885,5.722)%
  --(9.888,5.724)--(9.891,5.725)--(9.894,5.727)--(9.897,5.728)--(9.900,5.730)--(9.903,5.732)%
  --(9.906,5.733)--(9.909,5.735)--(9.912,5.736)--(9.915,5.738)--(9.918,5.739)--(9.921,5.741)%
  --(9.924,5.742)--(9.927,5.744)--(9.930,5.745)--(9.933,5.747)--(9.936,5.748)--(9.939,5.750)%
  --(9.942,5.751)--(9.945,5.753)--(9.948,5.754)--(9.951,5.756)--(9.954,5.757)--(9.957,5.759)%
  --(9.960,5.760)--(9.963,5.762)--(9.966,5.763)--(9.969,5.765)--(9.972,5.766)--(9.975,5.768)%
  --(9.978,5.769)--(9.981,5.771)--(9.984,5.772)--(9.987,5.774)--(9.990,5.775)--(9.992,5.777)%
  --(9.995,5.778)--(9.998,5.780)--(10.001,5.781)--(10.004,5.783)--(10.007,5.784)--(10.010,5.786)%
  --(10.013,5.787)--(10.016,5.789)--(10.019,5.790)--(10.022,5.792)--(10.025,5.793)--(10.028,5.795)%
  --(10.031,5.796)--(10.034,5.798)--(10.037,5.799)--(10.040,5.801)--(10.043,5.802)--(10.046,5.804)%
  --(10.049,5.805)--(10.052,5.807)--(10.055,5.808)--(10.058,5.810)--(10.061,5.811)--(10.064,5.813)%
  --(10.067,5.814)--(10.070,5.816)--(10.073,5.817)--(10.076,5.819)--(10.079,5.820)--(10.082,5.822)%
  --(10.085,5.823)--(10.088,5.825)--(10.091,5.826)--(10.094,5.828)--(10.097,5.829)--(10.100,5.831)%
  --(10.103,5.832)--(10.106,5.834)--(10.109,5.835)--(10.112,5.837)--(10.115,5.838)--(10.118,5.840)%
  --(10.121,5.841)--(10.124,5.843)--(10.127,5.844)--(10.130,5.846)--(10.133,5.847)--(10.136,5.849)%
  --(10.139,5.850)--(10.142,5.852)--(10.145,5.853)--(10.148,5.855)--(10.151,5.856)--(10.154,5.858)%
  --(10.157,5.859)--(10.160,5.861)--(10.163,5.863)--(10.166,5.864)--(10.169,5.866)--(10.172,5.867)%
  --(10.175,5.869)--(10.178,5.870)--(10.181,5.872)--(10.184,5.873)--(10.187,5.875)--(10.190,5.876)%
  --(10.193,5.878)--(10.196,5.879)--(10.199,5.881)--(10.201,5.882)--(10.204,5.884)--(10.207,5.885)%
  --(10.210,5.887)--(10.213,5.888)--(10.216,5.890)--(10.219,5.891)--(10.222,5.893)--(10.225,5.894)%
  --(10.228,5.896)--(10.231,5.897)--(10.234,5.899)--(10.237,5.900)--(10.240,5.902)--(10.243,5.903)%
  --(10.246,5.905)--(10.249,5.906)--(10.252,5.908)--(10.255,5.909)--(10.258,5.911)--(10.261,5.912)%
  --(10.264,5.914)--(10.267,5.915)--(10.270,5.917)--(10.273,5.918)--(10.276,5.920)--(10.279,5.921)%
  --(10.282,5.923)--(10.285,5.924)--(10.288,5.926)--(10.291,5.927)--(10.294,5.929)--(10.297,5.930)%
  --(10.300,5.932)--(10.303,5.933)--(10.306,5.935)--(10.309,5.936)--(10.312,5.938)--(10.315,5.939)%
  --(10.318,5.941)--(10.321,5.942)--(10.324,5.944)--(10.327,5.945)--(10.330,5.947)--(10.333,5.948)%
  --(10.336,5.950)--(10.339,5.951)--(10.342,5.953)--(10.345,5.954)--(10.348,5.956)--(10.351,5.957)%
  --(10.354,5.959)--(10.357,5.960)--(10.360,5.962)--(10.363,5.964)--(10.366,5.965)--(10.369,5.967)%
  --(10.372,5.968)--(10.375,5.970)--(10.378,5.971)--(10.381,5.973)--(10.384,5.974)--(10.387,5.976)%
  --(10.390,5.977)--(10.393,5.979)--(10.396,5.980)--(10.399,5.982)--(10.402,5.983)--(10.405,5.985)%
  --(10.408,5.986)--(10.410,5.988)--(10.413,5.989)--(10.416,5.991)--(10.419,5.992)--(10.422,5.994)%
  --(10.425,5.995)--(10.428,5.997)--(10.431,5.998)--(10.434,6.000)--(10.437,6.001)--(10.440,6.003)%
  --(10.443,6.004)--(10.446,6.006)--(10.449,6.007)--(10.452,6.009)--(10.455,6.010)--(10.458,6.012)%
  --(10.461,6.013)--(10.464,6.015)--(10.467,6.016)--(10.470,6.018)--(10.473,6.019)--(10.476,6.021)%
  --(10.479,6.022)--(10.482,6.024)--(10.485,6.025)--(10.488,6.027)--(10.491,6.028)--(10.494,6.030)%
  --(10.497,6.031)--(10.500,6.033)--(10.503,6.034)--(10.506,6.036)--(10.509,6.037)--(10.512,6.039)%
  --(10.515,6.040)--(10.518,6.042)--(10.521,6.043)--(10.524,6.045)--(10.527,6.046)--(10.530,6.048)%
  --(10.533,6.050)--(10.536,6.051)--(10.539,6.053)--(10.542,6.054)--(10.545,6.056)--(10.548,6.057)%
  --(10.551,6.059)--(10.554,6.060)--(10.557,6.062)--(10.560,6.063)--(10.563,6.065)--(10.566,6.066)%
  --(10.569,6.068)--(10.572,6.069)--(10.575,6.071)--(10.578,6.072)--(10.581,6.074)--(10.584,6.075)%
  --(10.587,6.077)--(10.590,6.078)--(10.593,6.080)--(10.596,6.081)--(10.599,6.083)--(10.602,6.084)%
  --(10.605,6.086)--(10.608,6.087)--(10.611,6.089)--(10.614,6.090)--(10.617,6.092)--(10.619,6.093)%
  --(10.622,6.095)--(10.625,6.096)--(10.628,6.098)--(10.631,6.099)--(10.634,6.101)--(10.637,6.102)%
  --(10.640,6.104)--(10.643,6.105)--(10.646,6.107)--(10.649,6.108)--(10.652,6.110)--(10.655,6.111)%
  --(10.658,6.113)--(10.661,6.114)--(10.664,6.116)--(10.667,6.117)--(10.670,6.119)--(10.673,6.120)%
  --(10.676,6.122)--(10.679,6.123)--(10.682,6.125)--(10.685,6.127)--(10.688,6.128)--(10.691,6.130)%
  --(10.694,6.131)--(10.697,6.133)--(10.700,6.134)--(10.703,6.136)--(10.706,6.137)--(10.709,6.139)%
  --(10.712,6.140)--(10.715,6.142)--(10.718,6.143)--(10.721,6.145)--(10.724,6.146)--(10.727,6.148)%
  --(10.730,6.149)--(10.733,6.151)--(10.736,6.152)--(10.739,6.154)--(10.742,6.155)--(10.745,6.157)%
  --(10.748,6.158)--(10.751,6.160)--(10.754,6.161)--(10.757,6.163)--(10.760,6.164)--(10.763,6.166)%
  --(10.766,6.167)--(10.769,6.169)--(10.772,6.170)--(10.775,6.172)--(10.778,6.173)--(10.781,6.175)%
  --(10.784,6.176)--(10.787,6.178)--(10.790,6.179)--(10.793,6.181)--(10.796,6.182)--(10.799,6.184)%
  --(10.802,6.185)--(10.805,6.187)--(10.808,6.188)--(10.811,6.190)--(10.814,6.191)--(10.817,6.193)%
  --(10.820,6.194)--(10.823,6.196)--(10.826,6.198)--(10.828,6.199)--(10.831,6.201)--(10.834,6.202)%
  --(10.837,6.204)--(10.840,6.205)--(10.843,6.207)--(10.846,6.208)--(10.849,6.210)--(10.852,6.211)%
  --(10.855,6.213)--(10.858,6.214)--(10.861,6.216)--(10.864,6.217)--(10.867,6.219)--(10.870,6.220)%
  --(10.873,6.222)--(10.876,6.223)--(10.879,6.225)--(10.882,6.226)--(10.885,6.228)--(10.888,6.229)%
  --(10.891,6.231)--(10.894,6.232)--(10.897,6.234)--(10.900,6.235)--(10.903,6.237)--(10.906,6.238)%
  --(10.909,6.240)--(10.912,6.241)--(10.915,6.243)--(10.918,6.244)--(10.921,6.246)--(10.924,6.247)%
  --(10.927,6.249)--(10.930,6.250)--(10.933,6.252)--(10.936,6.253)--(10.939,6.255)--(10.942,6.256)%
  --(10.945,6.258)--(10.948,6.259)--(10.951,6.261)--(10.954,6.262)--(10.957,6.264)--(10.960,6.266)%
  --(10.963,6.267)--(10.966,6.269)--(10.969,6.270)--(10.972,6.272)--(10.975,6.273)--(10.978,6.275)%
  --(10.981,6.276)--(10.984,6.278)--(10.987,6.279)--(10.990,6.281)--(10.993,6.282)--(10.996,6.284)%
  --(10.999,6.285)--(11.002,6.287)--(11.005,6.288)--(11.008,6.290)--(11.011,6.291)--(11.014,6.293)%
  --(11.017,6.294)--(11.020,6.296)--(11.023,6.297)--(11.026,6.299)--(11.029,6.300)--(11.032,6.302)%
  --(11.035,6.303)--(11.037,6.305)--(11.040,6.306)--(11.043,6.308)--(11.046,6.309)--(11.049,6.311)%
  --(11.052,6.312)--(11.055,6.314)--(11.058,6.315)--(11.061,6.317)--(11.064,6.318)--(11.067,6.320)%
  --(11.070,6.321)--(11.073,6.323)--(11.076,6.324)--(11.079,6.326)--(11.082,6.328)--(11.085,6.329)%
  --(11.088,6.331)--(11.091,6.332)--(11.094,6.334)--(11.097,6.335)--(11.100,6.337)--(11.103,6.338)%
  --(11.106,6.340)--(11.109,6.341)--(11.112,6.343)--(11.115,6.344)--(11.118,6.346)--(11.121,6.347)%
  --(11.124,6.349)--(11.127,6.350)--(11.130,6.352)--(11.133,6.353)--(11.136,6.355)--(11.139,6.356)%
  --(11.142,6.358)--(11.145,6.359)--(11.148,6.361)--(11.151,6.362)--(11.154,6.364)--(11.157,6.365)%
  --(11.160,6.367)--(11.163,6.368)--(11.166,6.370)--(11.169,6.371)--(11.172,6.373)--(11.175,6.374)%
  --(11.178,6.376)--(11.181,6.377)--(11.184,6.379)--(11.187,6.380)--(11.190,6.382)--(11.193,6.383)%
  --(11.196,6.385)--(11.199,6.386)--(11.202,6.388)--(11.205,6.390)--(11.208,6.391)--(11.211,6.393)%
  --(11.214,6.394)--(11.217,6.396)--(11.220,6.397)--(11.223,6.399)--(11.226,6.400)--(11.229,6.402)%
  --(11.232,6.403)--(11.235,6.405)--(11.238,6.406)--(11.241,6.408)--(11.244,6.409)--(11.247,6.411)%
  --(11.249,6.412)--(11.252,6.414)--(11.255,6.415)--(11.258,6.417)--(11.261,6.418)--(11.264,6.420)%
  --(11.267,6.421)--(11.270,6.423)--(11.273,6.424)--(11.276,6.426)--(11.279,6.427)--(11.282,6.429)%
  --(11.285,6.430)--(11.288,6.432)--(11.291,6.433)--(11.294,6.435)--(11.297,6.436)--(11.300,6.438)%
  --(11.303,6.439)--(11.306,6.441)--(11.309,6.442)--(11.312,6.444)--(11.315,6.446)--(11.318,6.447)%
  --(11.321,6.449)--(11.324,6.450)--(11.327,6.452)--(11.330,6.453)--(11.333,6.455)--(11.336,6.456)%
  --(11.339,6.458)--(11.342,6.459)--(11.345,6.461)--(11.348,6.462)--(11.351,6.464)--(11.354,6.465)%
  --(11.357,6.467)--(11.360,6.468)--(11.363,6.470)--(11.366,6.471)--(11.369,6.473)--(11.372,6.474)%
  --(11.375,6.476)--(11.378,6.477)--(11.381,6.479)--(11.384,6.480)--(11.387,6.482)--(11.390,6.483)%
  --(11.393,6.485)--(11.396,6.486)--(11.399,6.488)--(11.402,6.489)--(11.405,6.491)--(11.408,6.492)%
  --(11.411,6.494)--(11.414,6.495)--(11.417,6.497)--(11.420,6.498)--(11.423,6.500)--(11.426,6.502)%
  --(11.429,6.503)--(11.432,6.505)--(11.435,6.506)--(11.438,6.508)--(11.441,6.509)--(11.444,6.511)%
  --(11.447,6.512)--(11.450,6.514)--(11.453,6.515)--(11.456,6.517)--(11.458,6.518)--(11.461,6.520)%
  --(11.464,6.521)--(11.467,6.523)--(11.470,6.524)--(11.473,6.526)--(11.476,6.527)--(11.479,6.529)%
  --(11.482,6.530)--(11.485,6.532)--(11.488,6.533)--(11.491,6.535)--(11.494,6.536)--(11.497,6.538)%
  --(11.500,6.539)--(11.503,6.541)--(11.506,6.542)--(11.509,6.544)--(11.512,6.545)--(11.515,6.547)%
  --(11.518,6.548)--(11.521,6.550)--(11.524,6.551)--(11.527,6.553)--(11.530,6.555)--(11.533,6.556)%
  --(11.536,6.558)--(11.539,6.559)--(11.542,6.561)--(11.545,6.562)--(11.548,6.564)--(11.551,6.565)%
  --(11.554,6.567)--(11.557,6.568)--(11.560,6.570)--(11.563,6.571)--(11.566,6.573)--(11.569,6.574)%
  --(11.572,6.576)--(11.575,6.577)--(11.578,6.579)--(11.581,6.580)--(11.584,6.582)--(11.587,6.583)%
  --(11.590,6.585)--(11.593,6.586)--(11.596,6.588)--(11.599,6.589)--(11.602,6.591)--(11.605,6.592)%
  --(11.608,6.594)--(11.611,6.595)--(11.614,6.597)--(11.617,6.598)--(11.620,6.600)--(11.623,6.601)%
  --(11.626,6.603)--(11.629,6.605)--(11.632,6.606)--(11.635,6.608)--(11.638,6.609)--(11.641,6.611)%
  --(11.644,6.612)--(11.647,6.614)--(11.650,6.615)--(11.653,6.617)--(11.656,6.618)--(11.659,6.620)%
  --(11.662,6.621)--(11.665,6.623)--(11.667,6.624)--(11.670,6.626)--(11.673,6.627)--(11.676,6.629)%
  --(11.679,6.630)--(11.682,6.632)--(11.685,6.633)--(11.688,6.635)--(11.691,6.636)--(11.694,6.638)%
  --(11.697,6.639)--(11.700,6.641)--(11.703,6.642)--(11.706,6.644)--(11.709,6.645)--(11.712,6.647)%
  --(11.715,6.648)--(11.718,6.650)--(11.721,6.651)--(11.724,6.653)--(11.727,6.654)--(11.730,6.656)%
  --(11.733,6.658)--(11.736,6.659)--(11.739,6.661)--(11.742,6.662)--(11.745,6.664)--(11.748,6.665)%
  --(11.751,6.667)--(11.754,6.668)--(11.757,6.670)--(11.760,6.671)--(11.763,6.673)--(11.766,6.674)%
  --(11.769,6.676)--(11.772,6.677)--(11.775,6.679)--(11.778,6.680)--(11.781,6.682)--(11.784,6.683)%
  --(11.787,6.685)--(11.790,6.686)--(11.793,6.688)--(11.796,6.689)--(11.799,6.691)--(11.802,6.692)%
  --(11.805,6.694)--(11.808,6.695)--(11.811,6.697)--(11.814,6.698)--(11.817,6.700)--(11.820,6.701)%
  --(11.823,6.703)--(11.826,6.705)--(11.829,6.706)--(11.832,6.708)--(11.835,6.709)--(11.838,6.711)%
  --(11.841,6.712)--(11.844,6.714)--(11.847,6.715)--(11.850,6.717)--(11.853,6.718)--(11.856,6.720)%
  --(11.859,6.721)--(11.862,6.723)--(11.865,6.724)--(11.868,6.726)--(11.871,6.727)--(11.874,6.729)%
  --(11.876,6.730)--(11.879,6.732)--(11.882,6.733)--(11.885,6.735)--(11.888,6.736)--(11.891,6.738)%
  --(11.894,6.739)--(11.897,6.741)--(11.900,6.742)--(11.903,6.744)--(11.906,6.745)--(11.909,6.747)%
  --(11.912,6.748)--(11.915,6.750)--(11.918,6.751)--(11.921,6.753)--(11.924,6.755)--(11.927,6.756)%
  --(11.930,6.758)--(11.933,6.759)--(11.936,6.761)--(11.939,6.762)--(11.942,6.764)--(11.945,6.765)%
  --(11.948,6.767)--(11.951,6.768)--(11.954,6.770)--(11.957,6.771)--(11.960,6.773)--(11.963,6.774)%
  --(11.966,6.776)--(11.969,6.777)--(11.972,6.779)--(11.975,6.780)--(11.978,6.782)--(11.981,6.783)%
  --(11.984,6.785)--(11.987,6.786)--(11.990,6.788)--(11.993,6.789)--(11.996,6.791)--(11.999,6.792)%
  --(12.002,6.794)--(12.005,6.795)--(12.008,6.797)--(12.011,6.798)--(12.014,6.800)--(12.017,6.802)%
  --(12.020,6.803)--(12.023,6.805)--(12.026,6.806)--(12.029,6.808)--(12.032,6.809)--(12.035,6.811)%
  --(12.038,6.812)--(12.041,6.814)--(12.044,6.815)--(12.047,6.817)--(12.050,6.818)--(12.053,6.820)%
  --(12.056,6.821)--(12.059,6.823)--(12.062,6.824)--(12.065,6.826)--(12.068,6.827)--(12.071,6.829)%
  --(12.074,6.830)--(12.077,6.832)--(12.080,6.833)--(12.083,6.835)--(12.085,6.836)--(12.088,6.838)%
  --(12.091,6.839)--(12.094,6.841)--(12.097,6.842)--(12.100,6.844)--(12.103,6.846)--(12.106,6.847)%
  --(12.109,6.849)--(12.112,6.850)--(12.115,6.852)--(12.118,6.853)--(12.121,6.855)--(12.124,6.856)%
  --(12.127,6.858)--(12.130,6.859)--(12.133,6.861)--(12.136,6.862)--(12.139,6.864)--(12.142,6.865)%
  --(12.145,6.867)--(12.148,6.868)--(12.151,6.870)--(12.154,6.871)--(12.157,6.873)--(12.160,6.874)%
  --(12.163,6.876)--(12.166,6.877)--(12.169,6.879)--(12.172,6.880)--(12.175,6.882)--(12.178,6.883)%
  --(12.181,6.885)--(12.184,6.886)--(12.187,6.888)--(12.190,6.889)--(12.193,6.891)--(12.196,6.893)%
  --(12.199,6.894)--(12.202,6.896)--(12.205,6.897)--(12.208,6.899)--(12.211,6.900)--(12.214,6.902)%
  --(12.217,6.903)--(12.220,6.905)--(12.223,6.906)--(12.226,6.908)--(12.229,6.909)--(12.232,6.911)%
  --(12.235,6.912)--(12.238,6.914)--(12.241,6.915)--(12.244,6.917)--(12.247,6.918)--(12.250,6.920)%
  --(12.253,6.921)--(12.256,6.923)--(12.259,6.924)--(12.262,6.926)--(12.265,6.927)--(12.268,6.929)%
  --(12.271,6.930)--(12.274,6.932)--(12.277,6.933)--(12.280,6.935)--(12.283,6.937)--(12.286,6.938)%
  --(12.289,6.940)--(12.292,6.941)--(12.295,6.943)--(12.297,6.944)--(12.300,6.946)--(12.303,6.947)%
  --(12.306,6.949)--(12.309,6.950)--(12.312,6.952)--(12.315,6.953)--(12.318,6.955)--(12.321,6.956)%
  --(12.324,6.958)--(12.327,6.959)--(12.330,6.961)--(12.333,6.962)--(12.336,6.964)--(12.339,6.965)%
  --(12.342,6.967)--(12.345,6.968)--(12.348,6.970)--(12.351,6.971)--(12.354,6.973)--(12.357,6.974)%
  --(12.360,6.976)--(12.363,6.977)--(12.366,6.979)--(12.369,6.981)--(12.372,6.982)--(12.375,6.984)%
  --(12.378,6.985)--(12.381,6.987)--(12.384,6.988)--(12.387,6.990)--(12.390,6.991)--(12.393,6.993)%
  --(12.396,6.994)--(12.399,6.996)--(12.402,6.997)--(12.405,6.999)--(12.408,7.000)--(12.411,7.002)%
  --(12.414,7.003)--(12.417,7.005)--(12.420,7.006)--(12.423,7.008)--(12.426,7.009)--(12.429,7.011)%
  --(12.432,7.012)--(12.435,7.014)--(12.438,7.015)--(12.441,7.017)--(12.444,7.018)--(12.447,7.020)%
  --(12.450,7.021)--(12.453,7.023)--(12.456,7.025)--(12.459,7.026)--(12.462,7.028)--(12.465,7.029)%
  --(12.468,7.031)--(12.471,7.032)--(12.474,7.034)--(12.477,7.035)--(12.480,7.037)--(12.483,7.038)%
  --(12.486,7.040)--(12.489,7.041)--(12.492,7.043)--(12.495,7.044)--(12.498,7.046)--(12.501,7.047)%
  --(12.504,7.049)--(12.506,7.050)--(12.509,7.052)--(12.512,7.053)--(12.515,7.055)--(12.518,7.056)%
  --(12.521,7.058)--(12.524,7.059)--(12.527,7.061)--(12.530,7.062)--(12.533,7.064)--(12.536,7.066)%
  --(12.539,7.067)--(12.542,7.069)--(12.545,7.070)--(12.548,7.072)--(12.551,7.073)--(12.554,7.075)%
  --(12.557,7.076)--(12.560,7.078)--(12.563,7.079)--(12.566,7.081)--(12.569,7.082)--(12.572,7.084)%
  --(12.575,7.085)--(12.578,7.087)--(12.581,7.088)--(12.584,7.090)--(12.587,7.091)--(12.590,7.093)%
  --(12.593,7.094)--(12.596,7.096)--(12.599,7.097)--(12.602,7.099)--(12.605,7.100)--(12.608,7.102)%
  --(12.611,7.103)--(12.614,7.105)--(12.617,7.106)--(12.620,7.108)--(12.623,7.110)--(12.626,7.111)%
  --(12.629,7.113)--(12.632,7.114)--(12.635,7.116)--(12.638,7.117)--(12.641,7.119)--(12.644,7.120)%
  --(12.647,7.122)--(12.650,7.123)--(12.653,7.125)--(12.656,7.126)--(12.659,7.128)--(12.662,7.129)%
  --(12.665,7.131)--(12.668,7.132)--(12.671,7.134)--(12.674,7.135)--(12.677,7.137)--(12.680,7.138)%
  --(12.683,7.140)--(12.686,7.141)--(12.689,7.143)--(12.692,7.144)--(12.695,7.146)--(12.698,7.147)%
  --(12.701,7.149)--(12.704,7.151)--(12.707,7.152)--(12.710,7.154)--(12.713,7.155)--(12.715,7.157)%
  --(12.718,7.158)--(12.721,7.160)--(12.724,7.161)--(12.727,7.163)--(12.730,7.164)--(12.733,7.166)%
  --(12.736,7.167)--(12.739,7.169)--(12.742,7.170)--(12.745,7.172)--(12.748,7.173)--(12.751,7.175)%
  --(12.754,7.176)--(12.757,7.178)--(12.760,7.179)--(12.763,7.181)--(12.766,7.182)--(12.769,7.184)%
  --(12.772,7.185)--(12.775,7.187)--(12.778,7.188)--(12.781,7.190)--(12.784,7.192)--(12.787,7.193)%
  --(12.790,7.195)--(12.793,7.196)--(12.796,7.198)--(12.799,7.199)--(12.802,7.201)--(12.805,7.202)%
  --(12.808,7.204)--(12.811,7.205)--(12.814,7.207)--(12.817,7.208)--(12.820,7.210)--(12.823,7.211)%
  --(12.826,7.213)--(12.829,7.214)--(12.832,7.216)--(12.835,7.217)--(12.838,7.219)--(12.841,7.220)%
  --(12.844,7.222)--(12.847,7.223)--(12.850,7.225)--(12.853,7.226)--(12.856,7.228)--(12.859,7.229)%
  --(12.862,7.231)--(12.865,7.233)--(12.868,7.234)--(12.871,7.236)--(12.874,7.237)--(12.877,7.239)%
  --(12.880,7.240)--(12.883,7.242)--(12.886,7.243)--(12.889,7.245)--(12.892,7.246)--(12.895,7.248)%
  --(12.898,7.249)--(12.901,7.251)--(12.904,7.252)--(12.907,7.254)--(12.910,7.255)--(12.913,7.257)%
  --(12.916,7.258)--(12.919,7.260)--(12.922,7.261)--(12.924,7.263)--(12.927,7.264)--(12.930,7.266)%
  --(12.933,7.267)--(12.936,7.269)--(12.939,7.271)--(12.942,7.272)--(12.945,7.274)--(12.948,7.275)%
  --(12.951,7.277)--(12.954,7.278)--(12.957,7.280)--(12.960,7.281)--(12.963,7.283)--(12.966,7.284)%
  --(12.969,7.286)--(12.972,7.287)--(12.975,7.289)--(12.978,7.290)--(12.981,7.292)--(12.984,7.293)%
  --(12.987,7.295)--(12.990,7.296)--(12.993,7.298)--(12.996,7.299)--(12.999,7.301)--(13.002,7.302)%
  --(13.005,7.304)--(13.008,7.305)--(13.011,7.307)--(13.014,7.308)--(13.017,7.310)--(13.020,7.312)%
  --(13.023,7.313)--(13.026,7.315)--(13.029,7.316)--(13.032,7.318)--(13.035,7.319)--(13.038,7.321)%
  --(13.041,7.322)--(13.044,7.324)--(13.047,7.325)--(13.050,7.327)--(13.053,7.328)--(13.056,7.330)%
  --(13.059,7.331)--(13.062,7.333)--(13.065,7.334)--(13.068,7.336)--(13.071,7.337)--(13.074,7.339)%
  --(13.077,7.340)--(13.080,7.342)--(13.083,7.343)--(13.086,7.345)--(13.089,7.346)--(13.092,7.348)%
  --(13.095,7.350)--(13.098,7.351)--(13.101,7.353)--(13.104,7.354)--(13.107,7.356)--(13.110,7.357)%
  --(13.113,7.359)--(13.116,7.360)--(13.119,7.362)--(13.122,7.363)--(13.125,7.365)--(13.128,7.366)%
  --(13.131,7.368)--(13.133,7.369)--(13.136,7.371)--(13.139,7.372)--(13.142,7.374)--(13.145,7.375)%
  --(13.148,7.377)--(13.151,7.378)--(13.154,7.380)--(13.157,7.381)--(13.160,7.383)--(13.163,7.384)%
  --(13.166,7.386)--(13.169,7.388)--(13.172,7.389)--(13.175,7.391)--(13.178,7.392)--(13.181,7.394)%
  --(13.184,7.395)--(13.187,7.397)--(13.190,7.398)--(13.193,7.400)--(13.196,7.401)--(13.199,7.403)%
  --(13.202,7.404)--(13.205,7.406)--(13.208,7.407)--(13.211,7.409)--(13.214,7.410)--(13.217,7.412)%
  --(13.220,7.413)--(13.223,7.415)--(13.226,7.416)--(13.229,7.418)--(13.232,7.419)--(13.235,7.421)%
  --(13.238,7.422)--(13.241,7.424)--(13.244,7.425)--(13.247,7.427)--(13.250,7.429)--(13.253,7.430)%
  --(13.256,7.432)--(13.259,7.433)--(13.262,7.435)--(13.265,7.436)--(13.268,7.438)--(13.271,7.439)%
  --(13.274,7.441)--(13.277,7.442)--(13.280,7.444)--(13.283,7.445)--(13.286,7.447)--(13.289,7.448)%
  --(13.292,7.450)--(13.295,7.451)--(13.298,7.453)--(13.301,7.454)--(13.304,7.456)--(13.307,7.457)%
  --(13.310,7.459)--(13.313,7.460)--(13.316,7.462)--(13.319,7.463)--(13.322,7.465)--(13.325,7.467)%
  --(13.328,7.468)--(13.331,7.470)--(13.334,7.471)--(13.337,7.473)--(13.340,7.474)--(13.342,7.476)%
  --(13.345,7.477)--(13.348,7.479)--(13.351,7.480)--(13.354,7.482)--(13.357,7.483)--(13.360,7.485)%
  --(13.363,7.486)--(13.366,7.488)--(13.369,7.489)--(13.372,7.491)--(13.375,7.492)--(13.378,7.494)%
  --(13.381,7.495)--(13.384,7.497)--(13.387,7.498)--(13.390,7.500)--(13.393,7.501)--(13.396,7.503)%
  --(13.399,7.505)--(13.402,7.506)--(13.405,7.508)--(13.408,7.509)--(13.411,7.511)--(13.414,7.512)%
  --(13.417,7.514)--(13.420,7.515)--(13.423,7.517)--(13.426,7.518)--(13.429,7.520)--(13.432,7.521)%
  --(13.435,7.523)--(13.438,7.524)--(13.441,7.526)--(13.444,7.527);
\gpcolor{rgb color={0.941,0.894,0.259}}
\draw[gp path] (1.504,2.514)--(1.625,2.514)--(1.745,2.514)--(1.866,2.514)--(1.986,2.514)%
  --(2.107,2.514)--(2.228,2.514)--(2.348,2.514)--(2.469,2.514)--(2.589,2.514)--(2.710,2.514)%
  --(2.831,2.514)--(2.951,2.514)--(3.072,2.514)--(3.192,2.514)--(3.313,2.514)--(3.434,2.514)%
  --(3.554,2.514)--(3.675,2.514)--(3.796,2.514)--(3.916,2.514)--(4.037,2.514)--(4.157,2.514)%
  --(4.278,2.514)--(4.399,2.514)--(4.519,2.514)--(4.640,2.514)--(4.760,2.514)--(4.881,2.514)%
  --(5.002,2.514)--(5.122,2.514)--(5.243,2.514)--(5.363,2.514)--(5.484,2.514)--(5.605,2.514)%
  --(5.725,2.514)--(5.846,2.514)--(5.966,2.514)--(6.087,2.514)--(6.208,2.514)--(6.328,2.514)%
  --(6.449,2.514)--(6.569,2.514)--(6.690,2.514)--(6.811,2.514)--(6.931,2.514)--(7.052,2.514)%
  --(7.172,2.514)--(7.293,2.514)--(7.414,2.514)--(7.534,2.514)--(7.655,2.514)--(7.776,2.514)%
  --(7.896,2.514)--(8.017,2.514)--(8.137,2.514)--(8.258,2.514)--(8.379,2.514)--(8.499,2.514)%
  --(8.620,2.514)--(8.740,2.514)--(8.861,2.514)--(8.982,2.514)--(9.102,2.514)--(9.223,2.514)%
  --(9.343,2.514)--(9.464,2.514)--(9.585,2.514)--(9.705,2.514)--(9.826,2.514)--(9.946,2.514)%
  --(10.067,2.514)--(10.188,2.514)--(10.308,2.514)--(10.429,2.514)--(10.549,2.514)--(10.670,2.514)%
  --(10.791,2.514)--(10.911,2.514)--(11.032,2.514)--(11.152,2.514)--(11.273,2.514)--(11.394,2.514)%
  --(11.514,2.514)--(11.635,2.514)--(11.756,2.514)--(11.876,2.514)--(11.997,2.514)--(12.117,2.514)%
  --(12.238,2.514)--(12.359,2.514)--(12.479,2.514)--(12.600,2.514)--(12.720,2.514)--(12.841,2.514)%
  --(12.962,2.514)--(13.082,2.514)--(13.203,2.514)--(13.323,2.514)--(13.444,2.514);
\gpcolor{color=gp lt color border}
\draw[gp path] (1.504,8.631)--(1.504,0.985)--(13.447,0.985)--(13.447,8.631)--cycle;
%% coordinates of the plot area
\gpdefrectangularnode{gp plot 1}{\pgfpoint{1.504cm}{0.985cm}}{\pgfpoint{13.447cm}{8.631cm}}
\end{tikzpicture}
%% gnuplot variables

	\caption{Gráfico da equação cuja raíz determina o valor de $m$ (``Equação do gap zerada''). \protect[Parameters: NJL $\rm{D}_1$, $m_0 = \np[MeV]{5.6}$]}
	\label{Fig:gap_NJL-Buballa_Set_1}
\end{figure*}

\begin{figure*}
	\begin{tikzpicture}[gnuplot]
%% generated with GNUPLOT 5.0p2 (Lua 5.2; terminal rev. 99, script rev. 100)
%% Tue Apr  5 14:26:13 2016
\path (0.000,0.000) rectangle (14.000,9.000);
\gpcolor{color=gp lt color border}
\gpsetlinetype{gp lt border}
\gpsetdashtype{gp dt solid}
\gpsetlinewidth{1.00}
\draw[gp path] (1.504,0.985)--(1.684,0.985);
\draw[gp path] (13.447,0.985)--(13.267,0.985);
\node[gp node right] at (1.320,0.985) {$-100$};
\draw[gp path] (1.504,2.259)--(1.684,2.259);
\draw[gp path] (13.447,2.259)--(13.267,2.259);
\node[gp node right] at (1.320,2.259) {$0$};
\draw[gp path] (1.504,3.534)--(1.684,3.534);
\draw[gp path] (13.447,3.534)--(13.267,3.534);
\node[gp node right] at (1.320,3.534) {$100$};
\draw[gp path] (1.504,4.808)--(1.684,4.808);
\draw[gp path] (13.447,4.808)--(13.267,4.808);
\node[gp node right] at (1.320,4.808) {$200$};
\draw[gp path] (1.504,6.082)--(1.684,6.082);
\draw[gp path] (13.447,6.082)--(13.267,6.082);
\node[gp node right] at (1.320,6.082) {$300$};
\draw[gp path] (1.504,7.357)--(1.684,7.357);
\draw[gp path] (13.447,7.357)--(13.267,7.357);
\node[gp node right] at (1.320,7.357) {$400$};
\draw[gp path] (1.504,8.631)--(1.684,8.631);
\draw[gp path] (13.447,8.631)--(13.267,8.631);
\node[gp node right] at (1.320,8.631) {$500$};
\draw[gp path] (1.504,0.985)--(1.504,1.165);
\draw[gp path] (1.504,8.631)--(1.504,8.451);
\node[gp node center] at (1.504,0.677) {$0$};
\draw[gp path] (2.698,0.985)--(2.698,1.165);
\draw[gp path] (2.698,8.631)--(2.698,8.451);
\node[gp node center] at (2.698,0.677) {$100$};
\draw[gp path] (3.893,0.985)--(3.893,1.165);
\draw[gp path] (3.893,8.631)--(3.893,8.451);
\node[gp node center] at (3.893,0.677) {$200$};
\draw[gp path] (5.087,0.985)--(5.087,1.165);
\draw[gp path] (5.087,8.631)--(5.087,8.451);
\node[gp node center] at (5.087,0.677) {$300$};
\draw[gp path] (6.281,0.985)--(6.281,1.165);
\draw[gp path] (6.281,8.631)--(6.281,8.451);
\node[gp node center] at (6.281,0.677) {$400$};
\draw[gp path] (7.476,0.985)--(7.476,1.165);
\draw[gp path] (7.476,8.631)--(7.476,8.451);
\node[gp node center] at (7.476,0.677) {$500$};
\draw[gp path] (8.670,0.985)--(8.670,1.165);
\draw[gp path] (8.670,8.631)--(8.670,8.451);
\node[gp node center] at (8.670,0.677) {$600$};
\draw[gp path] (9.864,0.985)--(9.864,1.165);
\draw[gp path] (9.864,8.631)--(9.864,8.451);
\node[gp node center] at (9.864,0.677) {$700$};
\draw[gp path] (11.058,0.985)--(11.058,1.165);
\draw[gp path] (11.058,8.631)--(11.058,8.451);
\node[gp node center] at (11.058,0.677) {$800$};
\draw[gp path] (12.253,0.985)--(12.253,1.165);
\draw[gp path] (12.253,8.631)--(12.253,8.451);
\node[gp node center] at (12.253,0.677) {$900$};
\draw[gp path] (13.447,0.985)--(13.447,1.165);
\draw[gp path] (13.447,8.631)--(13.447,8.451);
\node[gp node center] at (13.447,0.677) {$1000$};
\draw[gp path] (1.504,8.631)--(1.504,0.985)--(13.447,0.985)--(13.447,8.631)--cycle;
\node[gp node center,rotate=-270] at (0.246,4.808) {$F(m) = m - m_0 + 2G_S\rho_s$ (MeV)};
\node[gp node center] at (7.475,0.215) {$m$ (MeV)};
\gpcolor{rgb color={0.580,0.000,0.827}}
\draw[gp path] (1.516,2.255)--(1.528,2.252)--(1.540,2.248)--(1.552,2.244)--(1.564,2.240)%
  --(1.576,2.236)--(1.588,2.233)--(1.600,2.229)--(1.612,2.225)--(1.624,2.221)--(1.636,2.217)%
  --(1.647,2.214)--(1.659,2.210)--(1.671,2.206)--(1.683,2.202)--(1.695,2.199)--(1.707,2.195)%
  --(1.719,2.191)--(1.731,2.187)--(1.743,2.184)--(1.755,2.180)--(1.767,2.176)--(1.779,2.173)%
  --(1.791,2.169)--(1.803,2.165)--(1.815,2.162)--(1.827,2.158)--(1.839,2.155)--(1.851,2.151)%
  --(1.863,2.148)--(1.875,2.144)--(1.887,2.141)--(1.899,2.137)--(1.910,2.134)--(1.922,2.130)%
  --(1.934,2.127)--(1.946,2.123)--(1.958,2.120)--(1.970,2.117)--(1.982,2.113)--(1.994,2.110)%
  --(2.006,2.107)--(2.018,2.103)--(2.030,2.100)--(2.042,2.097)--(2.054,2.094)--(2.066,2.090)%
  --(2.078,2.087)--(2.090,2.084)--(2.102,2.081)--(2.114,2.078)--(2.126,2.075)--(2.138,2.072)%
  --(2.150,2.069)--(2.162,2.066)--(2.173,2.063)--(2.185,2.060)--(2.197,2.057)--(2.209,2.054)%
  --(2.221,2.051)--(2.233,2.048)--(2.245,2.045)--(2.257,2.042)--(2.269,2.040)--(2.281,2.037)%
  --(2.293,2.034)--(2.305,2.031)--(2.317,2.029)--(2.329,2.026)--(2.341,2.023)--(2.353,2.021)%
  --(2.365,2.018)--(2.377,2.016)--(2.389,2.013)--(2.401,2.011)--(2.413,2.008)--(2.425,2.006)%
  --(2.436,2.003)--(2.448,2.001)--(2.460,1.998)--(2.472,1.996)--(2.484,1.994)--(2.496,1.991)%
  --(2.508,1.989)--(2.520,1.987)--(2.532,1.985)--(2.544,1.983)--(2.556,1.980)--(2.568,1.978)%
  --(2.580,1.976)--(2.592,1.974)--(2.604,1.972)--(2.616,1.970)--(2.628,1.968)--(2.640,1.966)%
  --(2.652,1.964)--(2.664,1.962)--(2.676,1.961)--(2.688,1.959)--(2.699,1.957)--(2.711,1.955)%
  --(2.723,1.954)--(2.735,1.952)--(2.747,1.950)--(2.759,1.948)--(2.771,1.947)--(2.783,1.945)%
  --(2.795,1.944)--(2.807,1.942)--(2.819,1.941)--(2.831,1.939)--(2.843,1.938)--(2.855,1.936)%
  --(2.867,1.935)--(2.879,1.934)--(2.891,1.932)--(2.903,1.931)--(2.915,1.930)--(2.927,1.929)%
  --(2.939,1.927)--(2.951,1.926)--(2.963,1.925)--(2.974,1.924)--(2.986,1.923)--(2.998,1.922)%
  --(3.010,1.921)--(3.022,1.920)--(3.034,1.919)--(3.046,1.918)--(3.058,1.917)--(3.070,1.916)%
  --(3.082,1.916)--(3.094,1.915)--(3.106,1.914)--(3.118,1.913)--(3.130,1.913)--(3.142,1.912)%
  --(3.154,1.911)--(3.166,1.911)--(3.178,1.910)--(3.190,1.910)--(3.202,1.909)--(3.214,1.909)%
  --(3.226,1.908)--(3.237,1.908)--(3.249,1.908)--(3.261,1.907)--(3.273,1.907)--(3.285,1.907)%
  --(3.297,1.906)--(3.309,1.906)--(3.321,1.906)--(3.333,1.906)--(3.345,1.906)--(3.357,1.906)%
  --(3.369,1.906)--(3.381,1.906)--(3.393,1.906)--(3.405,1.906)--(3.417,1.906)--(3.429,1.906)%
  --(3.441,1.906)--(3.453,1.906)--(3.465,1.906)--(3.477,1.906)--(3.489,1.907)--(3.500,1.907)%
  --(3.512,1.907)--(3.524,1.908)--(3.536,1.908)--(3.548,1.908)--(3.560,1.909)--(3.572,1.909)%
  --(3.584,1.910)--(3.596,1.910)--(3.608,1.911)--(3.620,1.912)--(3.632,1.912)--(3.644,1.913)%
  --(3.656,1.914)--(3.668,1.914)--(3.680,1.915)--(3.692,1.916)--(3.704,1.917)--(3.716,1.917)%
  --(3.728,1.918)--(3.740,1.919)--(3.752,1.920)--(3.763,1.921)--(3.775,1.922)--(3.787,1.923)%
  --(3.799,1.924)--(3.811,1.925)--(3.823,1.926)--(3.835,1.928)--(3.847,1.929)--(3.859,1.930)%
  --(3.871,1.931)--(3.883,1.933)--(3.895,1.934)--(3.907,1.935)--(3.919,1.937)--(3.931,1.938)%
  --(3.943,1.939)--(3.955,1.941)--(3.967,1.942)--(3.979,1.944)--(3.991,1.945)--(4.003,1.947)%
  --(4.015,1.949)--(4.026,1.950)--(4.038,1.952)--(4.050,1.954)--(4.062,1.955)--(4.074,1.957)%
  --(4.086,1.959)--(4.098,1.961)--(4.110,1.963)--(4.122,1.965)--(4.134,1.966)--(4.146,1.968)%
  --(4.158,1.970)--(4.170,1.972)--(4.182,1.974)--(4.194,1.977)--(4.206,1.979)--(4.218,1.981)%
  --(4.230,1.983)--(4.242,1.985)--(4.254,1.987)--(4.266,1.990)--(4.278,1.992)--(4.290,1.994)%
  --(4.301,1.996)--(4.313,1.999)--(4.325,2.001)--(4.337,2.004)--(4.349,2.006)--(4.361,2.008)%
  --(4.373,2.011)--(4.385,2.013)--(4.397,2.016)--(4.409,2.019)--(4.421,2.021)--(4.433,2.024)%
  --(4.445,2.027)--(4.457,2.029)--(4.469,2.032)--(4.481,2.035)--(4.493,2.038)--(4.505,2.040)%
  --(4.517,2.043)--(4.529,2.046)--(4.541,2.049)--(4.553,2.052)--(4.564,2.055)--(4.576,2.058)%
  --(4.588,2.061)--(4.600,2.064)--(4.612,2.067)--(4.624,2.070)--(4.636,2.073)--(4.648,2.076)%
  --(4.660,2.080)--(4.672,2.083)--(4.684,2.086)--(4.696,2.089)--(4.708,2.092)--(4.720,2.096)%
  --(4.732,2.099)--(4.744,2.103)--(4.756,2.106)--(4.768,2.109)--(4.780,2.113)--(4.792,2.116)%
  --(4.804,2.120)--(4.816,2.123)--(4.827,2.127)--(4.839,2.130)--(4.851,2.134)--(4.863,2.138)%
  --(4.875,2.141)--(4.887,2.145)--(4.899,2.149)--(4.911,2.152)--(4.923,2.156)--(4.935,2.160)%
  --(4.947,2.164)--(4.959,2.168)--(4.971,2.172)--(4.983,2.175)--(4.995,2.179)--(5.007,2.183)%
  --(5.019,2.187)--(5.031,2.191)--(5.043,2.195)--(5.055,2.199)--(5.067,2.203)--(5.079,2.208)%
  --(5.090,2.212)--(5.102,2.216)--(5.114,2.220)--(5.126,2.224)--(5.138,2.228)--(5.150,2.233)%
  --(5.162,2.237)--(5.174,2.241)--(5.186,2.246)--(5.198,2.250)--(5.210,2.254)--(5.222,2.259)%
  --(5.234,2.263)--(5.246,2.268)--(5.258,2.272)--(5.270,2.277)--(5.282,2.281)--(5.294,2.286)%
  --(5.306,2.290)--(5.318,2.295)--(5.330,2.300)--(5.342,2.304)--(5.353,2.309)--(5.365,2.314)%
  --(5.377,2.318)--(5.389,2.323)--(5.401,2.328)--(5.413,2.333)--(5.425,2.337)--(5.437,2.342)%
  --(5.449,2.347)--(5.461,2.352)--(5.473,2.357)--(5.485,2.362)--(5.497,2.367)--(5.509,2.372)%
  --(5.521,2.377)--(5.533,2.382)--(5.545,2.387)--(5.557,2.392)--(5.569,2.397)--(5.581,2.402)%
  --(5.593,2.407)--(5.605,2.412)--(5.617,2.418)--(5.628,2.423)--(5.640,2.428)--(5.652,2.433)%
  --(5.664,2.439)--(5.676,2.444)--(5.688,2.449)--(5.700,2.455)--(5.712,2.460)--(5.724,2.465)%
  --(5.736,2.471)--(5.748,2.476)--(5.760,2.482)--(5.772,2.487)--(5.784,2.493)--(5.796,2.498)%
  --(5.808,2.504)--(5.820,2.509)--(5.832,2.515)--(5.844,2.520)--(5.856,2.526)--(5.868,2.532)%
  --(5.880,2.537)--(5.891,2.543)--(5.903,2.549)--(5.915,2.554)--(5.927,2.560)--(5.939,2.566)%
  --(5.951,2.572)--(5.963,2.578)--(5.975,2.583)--(5.987,2.589)--(5.999,2.595)--(6.011,2.601)%
  --(6.023,2.607)--(6.035,2.613)--(6.047,2.619)--(6.059,2.625)--(6.071,2.631)--(6.083,2.637)%
  --(6.095,2.643)--(6.107,2.649)--(6.119,2.655)--(6.131,2.661)--(6.143,2.667)--(6.154,2.673)%
  --(6.166,2.680)--(6.178,2.686)--(6.190,2.692)--(6.202,2.698)--(6.214,2.705)--(6.226,2.711)%
  --(6.238,2.717)--(6.250,2.723)--(6.262,2.730)--(6.274,2.736)--(6.286,2.742)--(6.298,2.749)%
  --(6.310,2.755)--(6.322,2.762)--(6.334,2.768)--(6.346,2.774)--(6.358,2.781)--(6.370,2.787)%
  --(6.382,2.794)--(6.394,2.800)--(6.406,2.807)--(6.417,2.814)--(6.429,2.820)--(6.441,2.827)%
  --(6.453,2.833)--(6.465,2.840)--(6.477,2.847)--(6.489,2.853)--(6.501,2.860)--(6.513,2.867)%
  --(6.525,2.874)--(6.537,2.880)--(6.549,2.887)--(6.561,2.894)--(6.573,2.901)--(6.585,2.908)%
  --(6.597,2.914)--(6.609,2.921)--(6.621,2.928)--(6.633,2.935)--(6.645,2.942)--(6.657,2.949)%
  --(6.669,2.956)--(6.680,2.963)--(6.692,2.970)--(6.704,2.977)--(6.716,2.984)--(6.728,2.991)%
  --(6.740,2.998)--(6.752,3.005)--(6.764,3.012)--(6.776,3.019)--(6.788,3.026)--(6.800,3.033)%
  --(6.812,3.041)--(6.824,3.048)--(6.836,3.055)--(6.848,3.062)--(6.860,3.069)--(6.872,3.077)%
  --(6.884,3.084)--(6.896,3.091)--(6.908,3.098)--(6.920,3.106)--(6.932,3.113)--(6.944,3.120)%
  --(6.955,3.128)--(6.967,3.135)--(6.979,3.142)--(6.991,3.150)--(7.003,3.157)--(7.015,3.165)%
  --(7.027,3.172)--(7.039,3.180)--(7.051,3.187)--(7.063,3.195)--(7.075,3.202)--(7.087,3.210)%
  --(7.099,3.217)--(7.111,3.225)--(7.123,3.232)--(7.135,3.240)--(7.147,3.248)--(7.159,3.255)%
  --(7.171,3.263)--(7.183,3.270)--(7.195,3.278)--(7.207,3.286)--(7.218,3.294)--(7.230,3.301)%
  --(7.242,3.309)--(7.254,3.317)--(7.266,3.324)--(7.278,3.332)--(7.290,3.340)--(7.302,3.348)%
  --(7.314,3.356)--(7.326,3.363)--(7.338,3.371)--(7.350,3.379)--(7.362,3.387)--(7.374,3.395)%
  --(7.386,3.403)--(7.398,3.411)--(7.410,3.419)--(7.422,3.427)--(7.434,3.435)--(7.446,3.443)%
  --(7.458,3.451)--(7.470,3.459)--(7.481,3.467)--(7.493,3.475)--(7.505,3.483)--(7.517,3.491)%
  --(7.529,3.499)--(7.541,3.507)--(7.553,3.515)--(7.565,3.523)--(7.577,3.531)--(7.589,3.539)%
  --(7.601,3.548)--(7.613,3.556)--(7.625,3.564)--(7.637,3.572)--(7.649,3.580)--(7.661,3.589)%
  --(7.673,3.597)--(7.685,3.605)--(7.697,3.613)--(7.709,3.622)--(7.721,3.630)--(7.733,3.638)%
  --(7.744,3.647)--(7.756,3.655)--(7.768,3.663)--(7.780,3.672)--(7.792,3.680)--(7.804,3.688)%
  --(7.816,3.697)--(7.828,3.705)--(7.840,3.714)--(7.852,3.722)--(7.864,3.731)--(7.876,3.739)%
  --(7.888,3.748)--(7.900,3.756)--(7.912,3.765)--(7.924,3.773)--(7.936,3.782)--(7.948,3.790)%
  --(7.960,3.799)--(7.972,3.807)--(7.984,3.816)--(7.996,3.824)--(8.007,3.833)--(8.019,3.842)%
  --(8.031,3.850)--(8.043,3.859)--(8.055,3.868)--(8.067,3.876)--(8.079,3.885)--(8.091,3.894)%
  --(8.103,3.902)--(8.115,3.911)--(8.127,3.920)--(8.139,3.928)--(8.151,3.937)--(8.163,3.946)%
  --(8.175,3.955)--(8.187,3.963)--(8.199,3.972)--(8.211,3.981)--(8.223,3.990)--(8.235,3.999)%
  --(8.247,4.008)--(8.259,4.016)--(8.271,4.025)--(8.282,4.034)--(8.294,4.043)--(8.306,4.052)%
  --(8.318,4.061)--(8.330,4.070)--(8.342,4.079)--(8.354,4.088)--(8.366,4.097)--(8.378,4.106)%
  --(8.390,4.115)--(8.402,4.123)--(8.414,4.132)--(8.426,4.142)--(8.438,4.151)--(8.450,4.160)%
  --(8.462,4.169)--(8.474,4.178)--(8.486,4.187)--(8.498,4.196)--(8.510,4.205)--(8.522,4.214)%
  --(8.534,4.223)--(8.545,4.232)--(8.557,4.241)--(8.569,4.250)--(8.581,4.260)--(8.593,4.269)%
  --(8.605,4.278)--(8.617,4.287)--(8.629,4.296)--(8.641,4.305)--(8.653,4.315)--(8.665,4.324)%
  --(8.677,4.333)--(8.689,4.342)--(8.701,4.352)--(8.713,4.361)--(8.725,4.370)--(8.737,4.379)%
  --(8.749,4.389)--(8.761,4.398)--(8.773,4.407)--(8.785,4.417)--(8.797,4.426)--(8.808,4.435)%
  --(8.820,4.445)--(8.832,4.454)--(8.844,4.463)--(8.856,4.473)--(8.868,4.482)--(8.880,4.492)%
  --(8.892,4.501)--(8.904,4.510)--(8.916,4.520)--(8.928,4.529)--(8.940,4.539)--(8.952,4.548)%
  --(8.964,4.558)--(8.976,4.567)--(8.988,4.577)--(9.000,4.586)--(9.012,4.596)--(9.024,4.605)%
  --(9.036,4.615)--(9.048,4.624)--(9.060,4.634)--(9.071,4.643)--(9.083,4.653)--(9.095,4.662)%
  --(9.107,4.672)--(9.119,4.681)--(9.131,4.691)--(9.143,4.701)--(9.155,4.710)--(9.167,4.720)%
  --(9.179,4.730)--(9.191,4.739)--(9.203,4.749)--(9.215,4.758)--(9.227,4.768)--(9.239,4.778)%
  --(9.251,4.787)--(9.263,4.797)--(9.275,4.807)--(9.287,4.817)--(9.299,4.826)--(9.311,4.836)%
  --(9.323,4.846)--(9.334,4.855)--(9.346,4.865)--(9.358,4.875)--(9.370,4.885)--(9.382,4.895)%
  --(9.394,4.904)--(9.406,4.914)--(9.418,4.924)--(9.430,4.934)--(9.442,4.944)--(9.454,4.953)%
  --(9.466,4.963)--(9.478,4.973)--(9.490,4.983)--(9.502,4.993)--(9.514,5.003)--(9.526,5.012)%
  --(9.538,5.022)--(9.550,5.032)--(9.562,5.042)--(9.574,5.052)--(9.586,5.062)--(9.598,5.072)%
  --(9.609,5.082)--(9.621,5.092)--(9.633,5.102)--(9.645,5.112)--(9.657,5.122)--(9.669,5.132)%
  --(9.681,5.142)--(9.693,5.152)--(9.705,5.162)--(9.717,5.172)--(9.729,5.182)--(9.741,5.192)%
  --(9.753,5.202)--(9.765,5.212)--(9.777,5.222)--(9.789,5.232)--(9.801,5.242)--(9.813,5.252)%
  --(9.825,5.262)--(9.837,5.272)--(9.849,5.282)--(9.861,5.292)--(9.872,5.302)--(9.884,5.312)%
  --(9.896,5.322)--(9.908,5.333)--(9.920,5.343)--(9.932,5.353)--(9.944,5.363)--(9.956,5.373)%
  --(9.968,5.383)--(9.980,5.393)--(9.992,5.404)--(10.004,5.414)--(10.016,5.424)--(10.028,5.434)%
  --(10.040,5.444)--(10.052,5.455)--(10.064,5.465)--(10.076,5.475)--(10.088,5.485)--(10.100,5.495)%
  --(10.112,5.506)--(10.124,5.516)--(10.135,5.526)--(10.147,5.536)--(10.159,5.547)--(10.171,5.557)%
  --(10.183,5.567)--(10.195,5.578)--(10.207,5.588)--(10.219,5.598)--(10.231,5.608)--(10.243,5.619)%
  --(10.255,5.629)--(10.267,5.639)--(10.279,5.650)--(10.291,5.660)--(10.303,5.670)--(10.315,5.681)%
  --(10.327,5.691)--(10.339,5.702)--(10.351,5.712)--(10.363,5.722)--(10.375,5.733)--(10.387,5.743)%
  --(10.398,5.753)--(10.410,5.764)--(10.422,5.774)--(10.434,5.785)--(10.446,5.795)--(10.458,5.806)%
  --(10.470,5.816)--(10.482,5.826)--(10.494,5.837)--(10.506,5.847)--(10.518,5.858)--(10.530,5.868)%
  --(10.542,5.879)--(10.554,5.889)--(10.566,5.900)--(10.578,5.910)--(10.590,5.921)--(10.602,5.931)%
  --(10.614,5.942)--(10.626,5.952)--(10.638,5.963)--(10.650,5.973)--(10.661,5.984)--(10.673,5.994)%
  --(10.685,6.005)--(10.697,6.015)--(10.709,6.026)--(10.721,6.037)--(10.733,6.047)--(10.745,6.058)%
  --(10.757,6.068)--(10.769,6.079)--(10.781,6.089)--(10.793,6.100)--(10.805,6.111)--(10.817,6.121)%
  --(10.829,6.132)--(10.841,6.143)--(10.853,6.153)--(10.865,6.164)--(10.877,6.174)--(10.889,6.185)%
  --(10.901,6.196)--(10.913,6.206)--(10.925,6.217)--(10.936,6.228)--(10.948,6.238)--(10.960,6.249)%
  --(10.972,6.260)--(10.984,6.270)--(10.996,6.281)--(11.008,6.292)--(11.020,6.302)--(11.032,6.313)%
  --(11.044,6.324)--(11.056,6.335)--(11.068,6.345)--(11.080,6.356)--(11.092,6.367)--(11.104,6.378)%
  --(11.116,6.388)--(11.128,6.399)--(11.140,6.410)--(11.152,6.421)--(11.164,6.431)--(11.176,6.442)%
  --(11.188,6.453)--(11.199,6.464)--(11.211,6.474)--(11.223,6.485)--(11.235,6.496)--(11.247,6.507)%
  --(11.259,6.518)--(11.271,6.528)--(11.283,6.539)--(11.295,6.550)--(11.307,6.561)--(11.319,6.572)%
  --(11.331,6.583)--(11.343,6.593)--(11.355,6.604)--(11.367,6.615)--(11.379,6.626)--(11.391,6.637)%
  --(11.403,6.648)--(11.415,6.659)--(11.427,6.669)--(11.439,6.680)--(11.451,6.691)--(11.462,6.702)%
  --(11.474,6.713)--(11.486,6.724)--(11.498,6.735)--(11.510,6.746)--(11.522,6.757)--(11.534,6.768)%
  --(11.546,6.779)--(11.558,6.789)--(11.570,6.800)--(11.582,6.811)--(11.594,6.822)--(11.606,6.833)%
  --(11.618,6.844)--(11.630,6.855)--(11.642,6.866)--(11.654,6.877)--(11.666,6.888)--(11.678,6.899)%
  --(11.690,6.910)--(11.702,6.921)--(11.714,6.932)--(11.725,6.943)--(11.737,6.954)--(11.749,6.965)%
  --(11.761,6.976)--(11.773,6.987)--(11.785,6.998)--(11.797,7.009)--(11.809,7.020)--(11.821,7.031)%
  --(11.833,7.042)--(11.845,7.053)--(11.857,7.064)--(11.869,7.075)--(11.881,7.086)--(11.893,7.097)%
  --(11.905,7.108)--(11.917,7.119)--(11.929,7.131)--(11.941,7.142)--(11.953,7.153)--(11.965,7.164)%
  --(11.977,7.175)--(11.988,7.186)--(12.000,7.197)--(12.012,7.208)--(12.024,7.219)--(12.036,7.230)%
  --(12.048,7.241)--(12.060,7.253)--(12.072,7.264)--(12.084,7.275)--(12.096,7.286)--(12.108,7.297)%
  --(12.120,7.308)--(12.132,7.319)--(12.144,7.330)--(12.156,7.342)--(12.168,7.353)--(12.180,7.364)%
  --(12.192,7.375)--(12.204,7.386)--(12.216,7.397)--(12.228,7.409)--(12.240,7.420)--(12.252,7.431)%
  --(12.263,7.442)--(12.275,7.453)--(12.287,7.464)--(12.299,7.476)--(12.311,7.487)--(12.323,7.498)%
  --(12.335,7.509)--(12.347,7.520)--(12.359,7.532)--(12.371,7.543)--(12.383,7.554)--(12.395,7.565)%
  --(12.407,7.577)--(12.419,7.588)--(12.431,7.599)--(12.443,7.610)--(12.455,7.622)--(12.467,7.633)%
  --(12.479,7.644)--(12.491,7.655)--(12.503,7.667)--(12.515,7.678)--(12.526,7.689)--(12.538,7.700)%
  --(12.550,7.712)--(12.562,7.723)--(12.574,7.734)--(12.586,7.745)--(12.598,7.757)--(12.610,7.768)%
  --(12.622,7.779)--(12.634,7.791)--(12.646,7.802)--(12.658,7.813)--(12.670,7.825)--(12.682,7.836)%
  --(12.694,7.847)--(12.706,7.858)--(12.718,7.870)--(12.730,7.881)--(12.742,7.892)--(12.754,7.904)%
  --(12.766,7.915)--(12.778,7.926)--(12.789,7.938)--(12.801,7.949)--(12.813,7.960)--(12.825,7.972)%
  --(12.837,7.983)--(12.849,7.995)--(12.861,8.006)--(12.873,8.017)--(12.885,8.029)--(12.897,8.040)%
  --(12.909,8.051)--(12.921,8.063)--(12.933,8.074)--(12.945,8.086)--(12.957,8.097)--(12.969,8.108)%
  --(12.981,8.120)--(12.993,8.131)--(13.005,8.143)--(13.017,8.154)--(13.029,8.165)--(13.041,8.177)%
  --(13.052,8.188)--(13.064,8.200)--(13.076,8.211)--(13.088,8.222)--(13.100,8.234)--(13.112,8.245)%
  --(13.124,8.257)--(13.136,8.268)--(13.148,8.280)--(13.160,8.291)--(13.172,8.302)--(13.184,8.314)%
  --(13.196,8.325)--(13.208,8.337)--(13.220,8.348)--(13.232,8.360)--(13.244,8.371)--(13.256,8.383)%
  --(13.268,8.394)--(13.280,8.406)--(13.292,8.417)--(13.304,8.429)--(13.315,8.440)--(13.327,8.452)%
  --(13.339,8.463)--(13.351,8.475)--(13.363,8.486)--(13.375,8.498)--(13.387,8.509)--(13.399,8.521)%
  --(13.411,8.532)--(13.423,8.544)--(13.435,8.555)--(13.447,8.567);
\gpcolor{rgb color={0.000,0.620,0.451}}
\draw[gp path] (1.504,2.259)--(1.625,2.259)--(1.745,2.259)--(1.866,2.259)--(1.987,2.259)%
  --(2.107,2.259)--(2.228,2.259)--(2.348,2.259)--(2.469,2.259)--(2.590,2.259)--(2.710,2.259)%
  --(2.831,2.259)--(2.952,2.259)--(3.072,2.259)--(3.193,2.259)--(3.314,2.259)--(3.434,2.259)%
  --(3.555,2.259)--(3.675,2.259)--(3.796,2.259)--(3.917,2.259)--(4.037,2.259)--(4.158,2.259)%
  --(4.279,2.259)--(4.399,2.259)--(4.520,2.259)--(4.641,2.259)--(4.761,2.259)--(4.882,2.259)%
  --(5.002,2.259)--(5.123,2.259)--(5.244,2.259)--(5.364,2.259)--(5.485,2.259)--(5.606,2.259)%
  --(5.726,2.259)--(5.847,2.259)--(5.968,2.259)--(6.088,2.259)--(6.209,2.259)--(6.329,2.259)%
  --(6.450,2.259)--(6.571,2.259)--(6.691,2.259)--(6.812,2.259)--(6.933,2.259)--(7.053,2.259)%
  --(7.174,2.259)--(7.295,2.259)--(7.415,2.259)--(7.536,2.259)--(7.656,2.259)--(7.777,2.259)%
  --(7.898,2.259)--(8.018,2.259)--(8.139,2.259)--(8.260,2.259)--(8.380,2.259)--(8.501,2.259)%
  --(8.622,2.259)--(8.742,2.259)--(8.863,2.259)--(8.983,2.259)--(9.104,2.259)--(9.225,2.259)%
  --(9.345,2.259)--(9.466,2.259)--(9.587,2.259)--(9.707,2.259)--(9.828,2.259)--(9.949,2.259)%
  --(10.069,2.259)--(10.190,2.259)--(10.310,2.259)--(10.431,2.259)--(10.552,2.259)--(10.672,2.259)%
  --(10.793,2.259)--(10.914,2.259)--(11.034,2.259)--(11.155,2.259)--(11.276,2.259)--(11.396,2.259)%
  --(11.517,2.259)--(11.637,2.259)--(11.758,2.259)--(11.879,2.259)--(11.999,2.259)--(12.120,2.259)%
  --(12.241,2.259)--(12.361,2.259)--(12.482,2.259)--(12.603,2.259)--(12.723,2.259)--(12.844,2.259)%
  --(12.964,2.259)--(13.085,2.259)--(13.206,2.259)--(13.326,2.259)--(13.447,2.259);
\gpcolor{color=gp lt color border}
\draw[gp path] (1.504,8.631)--(1.504,0.985)--(13.447,0.985)--(13.447,8.631)--cycle;
%% coordinates of the plot area
\gpdefrectangularnode{gp plot 1}{\pgfpoint{1.504cm}{0.985cm}}{\pgfpoint{13.447cm}{8.631cm}}
\end{tikzpicture}
%% gnuplot variables

	\caption{Gráfico da mass $m$ em função da densidade bariônica $\rho_B$. \protect[Parameters: NJL $\rm{D}_1$, $m_0 = \np[MeV]{5.6}$]}
	\label{Fig:mass_NJL-Buballa_Set_1}
\end{figure*}


\begin{figure*}
	\begin{tikzpicture}[gnuplot]
%% generated with GNUPLOT 5.0p2 (Lua 5.2; terminal rev. 99, script rev. 100)
%% Mon Mar  7 16:54:51 2016
\path (0.000,0.000) rectangle (14.000,9.000);
\gpcolor{color=gp lt color border}
\gpsetlinetype{gp lt border}
\gpsetdashtype{gp dt solid}
\gpsetlinewidth{1.00}
\draw[gp path] (1.688,0.985)--(1.868,0.985);
\draw[gp path] (13.447,0.985)--(13.267,0.985);
\node[gp node right] at (1.504,0.985) {$-0.9$};
\draw[gp path] (1.688,1.941)--(1.868,1.941);
\draw[gp path] (13.447,1.941)--(13.267,1.941);
\node[gp node right] at (1.504,1.941) {$-0.85$};
\draw[gp path] (1.688,2.897)--(1.868,2.897);
\draw[gp path] (13.447,2.897)--(13.267,2.897);
\node[gp node right] at (1.504,2.897) {$-0.8$};
\draw[gp path] (1.688,3.852)--(1.868,3.852);
\draw[gp path] (13.447,3.852)--(13.267,3.852);
\node[gp node right] at (1.504,3.852) {$-0.75$};
\draw[gp path] (1.688,4.808)--(1.868,4.808);
\draw[gp path] (13.447,4.808)--(13.267,4.808);
\node[gp node right] at (1.504,4.808) {$-0.7$};
\draw[gp path] (1.688,5.764)--(1.868,5.764);
\draw[gp path] (13.447,5.764)--(13.267,5.764);
\node[gp node right] at (1.504,5.764) {$-0.65$};
\draw[gp path] (1.688,6.720)--(1.868,6.720);
\draw[gp path] (13.447,6.720)--(13.267,6.720);
\node[gp node right] at (1.504,6.720) {$-0.6$};
\draw[gp path] (1.688,7.675)--(1.868,7.675);
\draw[gp path] (13.447,7.675)--(13.267,7.675);
\node[gp node right] at (1.504,7.675) {$-0.55$};
\draw[gp path] (1.688,8.631)--(1.868,8.631);
\draw[gp path] (13.447,8.631)--(13.267,8.631);
\node[gp node right] at (1.504,8.631) {$-0.5$};
\draw[gp path] (1.688,0.985)--(1.688,1.165);
\draw[gp path] (1.688,8.631)--(1.688,8.451);
\node[gp node center] at (1.688,0.677) {$0$};
\draw[gp path] (3.158,0.985)--(3.158,1.165);
\draw[gp path] (3.158,8.631)--(3.158,8.451);
\node[gp node center] at (3.158,0.677) {$0.05$};
\draw[gp path] (4.628,0.985)--(4.628,1.165);
\draw[gp path] (4.628,8.631)--(4.628,8.451);
\node[gp node center] at (4.628,0.677) {$0.1$};
\draw[gp path] (6.098,0.985)--(6.098,1.165);
\draw[gp path] (6.098,8.631)--(6.098,8.451);
\node[gp node center] at (6.098,0.677) {$0.15$};
\draw[gp path] (7.568,0.985)--(7.568,1.165);
\draw[gp path] (7.568,8.631)--(7.568,8.451);
\node[gp node center] at (7.568,0.677) {$0.2$};
\draw[gp path] (9.037,0.985)--(9.037,1.165);
\draw[gp path] (9.037,8.631)--(9.037,8.451);
\node[gp node center] at (9.037,0.677) {$0.25$};
\draw[gp path] (10.507,0.985)--(10.507,1.165);
\draw[gp path] (10.507,8.631)--(10.507,8.451);
\node[gp node center] at (10.507,0.677) {$0.3$};
\draw[gp path] (11.977,0.985)--(11.977,1.165);
\draw[gp path] (11.977,8.631)--(11.977,8.451);
\node[gp node center] at (11.977,0.677) {$0.35$};
\draw[gp path] (13.447,0.985)--(13.447,1.165);
\draw[gp path] (13.447,8.631)--(13.447,8.451);
\node[gp node center] at (13.447,0.677) {$0.4$};
\draw[gp path] (1.688,8.631)--(1.688,0.985)--(13.447,0.985)--(13.447,8.631)--cycle;
\node[gp node center,rotate=-270] at (0.246,4.808) {$\rho_s$ ($\rm{fm}^{-3}$)};
\node[gp node center] at (7.567,0.215) {$\rho$ ($\rm{fm}^{-3}$)};
\gpcolor{rgb color={0.580,0.000,0.827}}
\draw[gp path] (1.992,1.278)--(2.002,1.284)--(2.012,1.292)--(2.022,1.298)--(2.032,1.306)%
  --(2.042,1.312)--(2.052,1.319)--(2.062,1.326)--(2.072,1.333)--(2.082,1.340)--(2.092,1.347)%
  --(2.102,1.353)--(2.112,1.361)--(2.122,1.367)--(2.132,1.375)--(2.142,1.381)--(2.152,1.388)%
  --(2.162,1.395)--(2.172,1.402)--(2.182,1.409)--(2.192,1.416)--(2.202,1.422)--(2.212,1.430)%
  --(2.222,1.436)--(2.232,1.444)--(2.242,1.450)--(2.252,1.457)--(2.262,1.464)--(2.272,1.471)%
  --(2.282,1.478)--(2.292,1.485)--(2.302,1.491)--(2.312,1.499)--(2.322,1.505)--(2.332,1.512)%
  --(2.342,1.519)--(2.352,1.526)--(2.362,1.533)--(2.372,1.540)--(2.382,1.546)--(2.392,1.554)%
  --(2.402,1.560)--(2.412,1.567)--(2.422,1.574)--(2.432,1.581)--(2.442,1.588)--(2.452,1.595)%
  --(2.462,1.601)--(2.472,1.609)--(2.482,1.615)--(2.492,1.623)--(2.502,1.629)--(2.512,1.636)%
  --(2.522,1.643)--(2.532,1.650)--(2.542,1.656)--(2.552,1.664)--(2.562,1.670)--(2.572,1.678)%
  --(2.582,1.684)--(2.592,1.690)--(2.602,1.698)--(2.612,1.704)--(2.622,1.711)--(2.632,1.718)%
  --(2.642,1.725)--(2.652,1.732)--(2.662,1.739)--(2.672,1.745)--(2.682,1.753)--(2.692,1.759)%
  --(2.702,1.766)--(2.712,1.773)--(2.722,1.780)--(2.732,1.787)--(2.742,1.794)--(2.752,1.800)%
  --(2.762,1.808)--(2.772,1.814)--(2.782,1.820)--(2.792,1.828)--(2.802,1.834)--(2.812,1.841)%
  --(2.822,1.848)--(2.832,1.855)--(2.842,1.862)--(2.852,1.869)--(2.862,1.875)--(2.872,1.883)%
  --(2.882,1.889)--(2.892,1.896)--(2.902,1.903)--(2.912,1.909)--(2.922,1.916)--(2.932,1.923)%
  --(2.942,1.930)--(2.952,1.937)--(2.962,1.944)--(2.972,1.950)--(2.982,1.958)--(2.992,1.964)%
  --(3.003,1.971)--(3.013,1.978)--(3.023,1.984)--(3.033,1.991)--(3.043,1.998)--(3.053,2.005)%
  --(3.063,2.011)--(3.073,2.019)--(3.083,2.025)--(3.093,2.032)--(3.103,2.039)--(3.113,2.045)%
  --(3.123,2.053)--(3.133,2.059)--(3.143,2.066)--(3.153,2.073)--(3.163,2.080)--(3.173,2.086)%
  --(3.183,2.094)--(3.193,2.100)--(3.203,2.106)--(3.213,2.114)--(3.223,2.120)--(3.233,2.127)%
  --(3.243,2.134)--(3.253,2.141)--(3.263,2.147)--(3.273,2.155)--(3.283,2.161)--(3.293,2.168)%
  --(3.303,2.175)--(3.313,2.181)--(3.323,2.189)--(3.333,2.195)--(3.343,2.202)--(3.353,2.209)%
  --(3.363,2.215)--(3.373,2.222)--(3.383,2.229)--(3.393,2.236)--(3.403,2.242)--(3.413,2.250)%
  --(3.423,2.256)--(3.433,2.262)--(3.443,2.270)--(3.453,2.276)--(3.463,2.283)--(3.473,2.290)%
  --(3.483,2.296)--(3.493,2.303)--(3.503,2.310)--(3.513,2.317)--(3.523,2.323)--(3.533,2.331)%
  --(3.543,2.337)--(3.553,2.343)--(3.563,2.351)--(3.573,2.357)--(3.583,2.364)--(3.593,2.371)%
  --(3.603,2.378)--(3.613,2.384)--(3.623,2.391)--(3.633,2.398)--(3.643,2.404)--(3.653,2.412)%
  --(3.663,2.418)--(3.673,2.424)--(3.683,2.432)--(3.693,2.438)--(3.703,2.445)--(3.713,2.452)%
  --(3.723,2.458)--(3.733,2.465)--(3.743,2.472)--(3.753,2.479)--(3.763,2.485)--(3.773,2.492)%
  --(3.783,2.499)--(3.793,2.505)--(3.803,2.513)--(3.813,2.519)--(3.823,2.526)--(3.833,2.533)%
  --(3.843,2.539)--(3.853,2.546)--(3.863,2.553)--(3.873,2.560)--(3.883,2.566)--(3.893,2.573)%
  --(3.903,2.580)--(3.913,2.586)--(3.923,2.593)--(3.933,2.600)--(3.943,2.606)--(3.953,2.613)%
  --(3.963,2.620)--(3.973,2.626)--(3.983,2.633)--(3.993,2.640)--(4.003,2.647)--(4.013,2.653)%
  --(4.023,2.660)--(4.033,2.667)--(4.043,2.673)--(4.053,2.681)--(4.063,2.687)--(4.073,2.693)%
  --(4.083,2.701)--(4.093,2.707)--(4.103,2.714)--(4.113,2.721)--(4.123,2.727)--(4.133,2.734)%
  --(4.143,2.741)--(4.153,2.747)--(4.163,2.754)--(4.173,2.760)--(4.183,2.768)--(4.193,2.774)%
  --(4.203,2.780)--(4.213,2.788)--(4.223,2.794)--(4.233,2.801)--(4.243,2.808)--(4.253,2.814)%
  --(4.263,2.821)--(4.273,2.828)--(4.283,2.834)--(4.293,2.841)--(4.303,2.848)--(4.313,2.855)%
  --(4.323,2.861)--(4.333,2.867)--(4.343,2.875)--(4.353,2.881)--(4.363,2.887)--(4.373,2.895)%
  --(4.383,2.901)--(4.393,2.908)--(4.403,2.915)--(4.413,2.921)--(4.423,2.928)--(4.433,2.934)%
  --(4.443,2.941)--(4.453,2.948)--(4.463,2.954)--(4.473,2.961)--(4.483,2.968)--(4.493,2.974)%
  --(4.503,2.982)--(4.513,2.988)--(4.523,2.994)--(4.533,3.001)--(4.543,3.008)--(4.553,3.014)%
  --(4.563,3.021)--(4.573,3.028)--(4.583,3.034)--(4.593,3.041)--(4.603,3.048)--(4.613,3.055)%
  --(4.623,3.061)--(4.633,3.067)--(4.643,3.075)--(4.653,3.081)--(4.663,3.087)--(4.673,3.095)%
  --(4.683,3.101)--(4.693,3.107)--(4.703,3.114)--(4.713,3.121)--(4.723,3.127)--(4.733,3.134)%
  --(4.743,3.141)--(4.753,3.148)--(4.763,3.154)--(4.773,3.161)--(4.783,3.168)--(4.793,3.174)%
  --(4.803,3.180)--(4.813,3.188)--(4.823,3.194)--(4.833,3.200)--(4.843,3.208)--(4.853,3.214)%
  --(4.863,3.220)--(4.873,3.227)--(4.883,3.234)--(4.893,3.240)--(4.903,3.247)--(4.913,3.254)%
  --(4.923,3.260)--(4.933,3.267)--(4.944,3.273)--(4.954,3.281)--(4.964,3.287)--(4.974,3.293)%
  --(4.984,3.300)--(4.994,3.307)--(5.004,3.313)--(5.014,3.320)--(5.024,3.327)--(5.034,3.333)%
  --(5.044,3.340)--(5.054,3.346)--(5.064,3.353)--(5.074,3.360)--(5.084,3.366)--(5.094,3.373)%
  --(5.104,3.380)--(5.114,3.386)--(5.124,3.392)--(5.134,3.400)--(5.144,3.406)--(5.154,3.412)%
  --(5.164,3.419)--(5.174,3.426)--(5.184,3.432)--(5.194,3.439)--(5.204,3.446)--(5.214,3.452)%
  --(5.224,3.459)--(5.234,3.465)--(5.244,3.472)--(5.254,3.479)--(5.264,3.485)--(5.274,3.491)%
  --(5.284,3.499)--(5.294,3.505)--(5.304,3.511)--(5.314,3.519)--(5.324,3.525)--(5.334,3.531)%
  --(5.344,3.538)--(5.354,3.545)--(5.364,3.551)--(5.374,3.558)--(5.384,3.564)--(5.394,3.571)%
  --(5.404,3.577)--(5.414,3.584)--(5.424,3.590)--(5.434,3.597)--(5.444,3.604)--(5.454,3.610)%
  --(5.464,3.617)--(5.474,3.624)--(5.484,3.630)--(5.494,3.636)--(5.504,3.644)--(5.514,3.650)%
  --(5.524,3.656)--(5.534,3.663)--(5.544,3.670)--(5.554,3.676)--(5.564,3.683)--(5.574,3.689)%
  --(5.584,3.696)--(5.594,3.703)--(5.604,3.709)--(5.614,3.715)--(5.624,3.722)--(5.634,3.729)%
  --(5.644,3.735)--(5.654,3.741)--(5.664,3.749)--(5.674,3.755)--(5.684,3.761)--(5.694,3.768)%
  --(5.704,3.775)--(5.714,3.781)--(5.724,3.788)--(5.734,3.794)--(5.744,3.801)--(5.754,3.808)%
  --(5.764,3.814)--(5.774,3.820)--(5.784,3.827)--(5.794,3.834)--(5.804,3.840)--(5.814,3.846)%
  --(5.824,3.854)--(5.834,3.860)--(5.844,3.866)--(5.854,3.873)--(5.864,3.880)--(5.874,3.886)%
  --(5.884,3.892)--(5.894,3.899)--(5.904,3.906)--(5.914,3.912)--(5.924,3.919)--(5.934,3.925)%
  --(5.944,3.932)--(5.954,3.939)--(5.964,3.945)--(5.974,3.951)--(5.984,3.958)--(5.994,3.965)%
  --(6.004,3.971)--(6.014,3.977)--(6.024,3.984)--(6.034,3.991)--(6.044,3.997)--(6.054,4.004)%
  --(6.064,4.010)--(6.074,4.017)--(6.084,4.023)--(6.094,4.030)--(6.104,4.036)--(6.114,4.043)%
  --(6.124,4.050)--(6.134,4.056)--(6.144,4.062)--(6.154,4.069)--(6.164,4.076)--(6.174,4.082)%
  --(6.184,4.088)--(6.194,4.095)--(6.204,4.102)--(6.214,4.108)--(6.224,4.115)--(6.234,4.121)%
  --(6.244,4.128)--(6.254,4.134)--(6.264,4.141)--(6.274,4.147)--(6.284,4.154)--(6.294,4.161)%
  --(6.304,4.167)--(6.314,4.173)--(6.324,4.179)--(6.334,4.187)--(6.344,4.193)--(6.354,4.199)%
  --(6.364,4.206)--(6.374,4.213)--(6.384,4.219)--(6.394,4.225)--(6.404,4.232)--(6.414,4.238)%
  --(6.424,4.245)--(6.434,4.252)--(6.444,4.258)--(6.454,4.264)--(6.464,4.271)--(6.474,4.278)%
  --(6.484,4.284)--(6.494,4.290)--(6.504,4.296)--(6.514,4.304)--(6.524,4.310)--(6.534,4.316)%
  --(6.544,4.323)--(6.554,4.330)--(6.564,4.336)--(6.574,4.342)--(6.584,4.349)--(6.594,4.355)%
  --(6.604,4.362)--(6.614,4.369)--(6.624,4.375)--(6.634,4.381)--(6.644,4.388)--(6.654,4.395)%
  --(6.664,4.401)--(6.674,4.407)--(6.684,4.413)--(6.694,4.421)--(6.704,4.427)--(6.714,4.433)%
  --(6.724,4.440)--(6.734,4.446)--(6.744,4.453)--(6.754,4.459)--(6.764,4.466)--(6.774,4.472)%
  --(6.784,4.478)--(6.794,4.485)--(6.804,4.492)--(6.814,4.498)--(6.824,4.504)--(6.834,4.510)%
  --(6.844,4.518)--(6.854,4.524)--(6.864,4.530)--(6.874,4.537)--(6.885,4.544)--(6.895,4.550)%
  --(6.905,4.556)--(6.915,4.563)--(6.925,4.569)--(6.935,4.576)--(6.945,4.582)--(6.955,4.589)%
  --(6.965,4.595)--(6.975,4.601)--(6.985,4.608)--(6.995,4.615)--(7.005,4.621)--(7.015,4.627)%
  --(7.025,4.633)--(7.035,4.641)--(7.045,4.647)--(7.055,4.653)--(7.065,4.660)--(7.075,4.666)%
  --(7.085,4.673)--(7.095,4.679)--(7.105,4.686)--(7.115,4.692)--(7.125,4.698)--(7.135,4.704)%
  --(7.145,4.712)--(7.155,4.718)--(7.165,4.724)--(7.175,4.730)--(7.185,4.737)--(7.195,4.744)%
  --(7.205,4.750)--(7.215,4.756)--(7.225,4.763)--(7.235,4.769)--(7.245,4.776)--(7.255,4.782)%
  --(7.265,4.789)--(7.275,4.795)--(7.285,4.801)--(7.295,4.808)--(7.305,4.815)--(7.315,4.821)%
  --(7.325,4.827)--(7.335,4.833)--(7.345,4.840)--(7.355,4.847)--(7.365,4.853)--(7.375,4.859)%
  --(7.385,4.866)--(7.395,4.872)--(7.405,4.879)--(7.415,4.885)--(7.425,4.892)--(7.435,4.898)%
  --(7.445,4.904)--(7.455,4.910)--(7.465,4.918)--(7.475,4.924)--(7.485,4.930)--(7.495,4.936)%
  --(7.505,4.943)--(7.515,4.950)--(7.525,4.956)--(7.535,4.962)--(7.545,4.969)--(7.555,4.975)%
  --(7.565,4.981)--(7.575,4.988)--(7.585,4.995)--(7.595,5.001)--(7.605,5.007)--(7.615,5.013)%
  --(7.625,5.020)--(7.635,5.027)--(7.645,5.033)--(7.655,5.039)--(7.665,5.046)--(7.675,5.052)%
  --(7.685,5.058)--(7.695,5.065)--(7.705,5.072)--(7.715,5.078)--(7.725,5.084)--(7.735,5.090)%
  --(7.745,5.098)--(7.755,5.104)--(7.765,5.110)--(7.775,5.116)--(7.785,5.122)--(7.795,5.129)%
  --(7.805,5.136)--(7.815,5.142)--(7.825,5.148)--(7.835,5.155)--(7.845,5.161)--(7.855,5.167)%
  --(7.865,5.174)--(7.875,5.181)--(7.885,5.187)--(7.895,5.193)--(7.905,5.199)--(7.915,5.206)%
  --(7.925,5.212)--(7.935,5.219)--(7.945,5.225)--(7.955,5.232)--(7.965,5.238)--(7.975,5.244)%
  --(7.985,5.250)--(7.995,5.257)--(8.005,5.264)--(8.015,5.270)--(8.025,5.276)--(8.035,5.282)%
  --(8.045,5.289)--(8.055,5.296)--(8.065,5.302)--(8.075,5.308)--(8.085,5.315)--(8.095,5.321)%
  --(8.105,5.327)--(8.115,5.333)--(8.125,5.340)--(8.135,5.347)--(8.145,5.353)--(8.155,5.359)%
  --(8.165,5.365)--(8.175,5.372)--(8.185,5.379)--(8.195,5.385)--(8.205,5.391)--(8.215,5.398)%
  --(8.225,5.404)--(8.235,5.410)--(8.245,5.416)--(8.255,5.423)--(8.265,5.430)--(8.275,5.436)%
  --(8.285,5.442)--(8.295,5.448)--(8.305,5.455)--(8.315,5.461)--(8.325,5.468)--(8.335,5.474)%
  --(8.345,5.480)--(8.355,5.487)--(8.365,5.493)--(8.375,5.499)--(8.385,5.505)--(8.395,5.513)%
  --(8.405,5.519)--(8.415,5.525)--(8.425,5.531)--(8.435,5.538)--(8.445,5.544)--(8.455,5.550)%
  --(8.465,5.557)--(8.475,5.563)--(8.485,5.570)--(8.495,5.576)--(8.505,5.582)--(8.515,5.588)%
  --(8.525,5.595)--(8.535,5.602)--(8.545,5.608)--(8.555,5.614)--(8.565,5.620)--(8.575,5.627)%
  --(8.585,5.633)--(8.595,5.639)--(8.605,5.646)--(8.615,5.653)--(8.625,5.659)--(8.635,5.665)%
  --(8.645,5.671)--(8.655,5.677)--(8.665,5.684)--(8.675,5.690)--(8.685,5.697)--(8.695,5.703)%
  --(8.705,5.710)--(8.715,5.716)--(8.725,5.722)--(8.735,5.728)--(8.745,5.734)--(8.755,5.741)%
  --(8.765,5.748)--(8.775,5.754)--(8.785,5.760)--(8.795,5.766)--(8.805,5.773)--(8.815,5.779)%
  --(8.826,5.785)--(8.836,5.792)--(8.846,5.799)--(8.856,5.805)--(8.866,5.811)--(8.876,5.817)%
  --(8.886,5.823)--(8.896,5.830)--(8.906,5.836)--(8.916,5.842)--(8.926,5.849)--(8.936,5.855)%
  --(8.946,5.862)--(8.956,5.868)--(8.966,5.874)--(8.976,5.880)--(8.986,5.887)--(8.996,5.893)%
  --(9.006,5.900)--(9.016,5.906)--(9.026,5.912)--(9.036,5.919)--(9.046,5.925)--(9.056,5.931)%
  --(9.066,5.937)--(9.076,5.943)--(9.086,5.951)--(9.096,5.957)--(9.106,5.963)--(9.116,5.969)%
  --(9.126,5.976)--(9.136,5.982)--(9.146,5.988)--(9.156,5.994)--(9.166,6.000)--(9.176,6.008)%
  --(9.186,6.014)--(9.196,6.020)--(9.206,6.026)--(9.216,6.032)--(9.226,6.039)--(9.236,6.045)%
  --(9.246,6.051)--(9.256,6.057)--(9.266,6.064)--(9.276,6.071)--(9.286,6.077)--(9.296,6.083)%
  --(9.306,6.089)--(9.316,6.096)--(9.326,6.102)--(9.336,6.108)--(9.346,6.114)--(9.356,6.120)%
  --(9.366,6.128)--(9.376,6.134)--(9.386,6.140)--(9.396,6.146)--(9.406,6.152)--(9.416,6.159)%
  --(9.426,6.165)--(9.436,6.171)--(9.446,6.177)--(9.456,6.183)--(9.466,6.191)--(9.476,6.197)%
  --(9.486,6.203)--(9.496,6.209)--(9.506,6.215)--(9.516,6.222)--(9.526,6.228)--(9.536,6.234)%
  --(9.546,6.240)--(9.556,6.246)--(9.566,6.254)--(9.576,6.260)--(9.586,6.266)--(9.596,6.272)%
  --(9.606,6.279)--(9.616,6.285)--(9.626,6.291)--(9.636,6.297)--(9.646,6.303)--(9.656,6.310)%
  --(9.666,6.316)--(9.676,6.323)--(9.686,6.329)--(9.696,6.335)--(9.706,6.342)--(9.716,6.348)%
  --(9.726,6.354)--(9.736,6.360)--(9.746,6.366)--(9.756,6.373)--(9.766,6.379)--(9.776,6.385)%
  --(9.786,6.392)--(9.796,6.398)--(9.806,6.405)--(9.816,6.411)--(9.826,6.417)--(9.836,6.423)%
  --(9.846,6.429)--(9.856,6.436)--(9.866,6.442)--(9.876,6.448)--(9.886,6.454)--(9.896,6.461)%
  --(9.906,6.468)--(9.916,6.474)--(9.926,6.480)--(9.936,6.486)--(9.946,6.492)--(9.956,6.499)%
  --(9.966,6.505)--(9.976,6.511)--(9.986,6.517)--(9.996,6.523)--(10.006,6.530)--(10.016,6.536)%
  --(10.026,6.543)--(10.036,6.549)--(10.046,6.555)--(10.056,6.562)--(10.066,6.568)--(10.076,6.574)%
  --(10.086,6.580)--(10.096,6.586)--(10.106,6.593)--(10.116,6.599)--(10.126,6.605)--(10.136,6.611)%
  --(10.146,6.617)--(10.156,6.625)--(10.166,6.631)--(10.176,6.637)--(10.186,6.643)--(10.196,6.649)%
  --(10.206,6.656)--(10.216,6.662)--(10.226,6.668)--(10.236,6.674)--(10.246,6.680)--(10.256,6.687)%
  --(10.266,6.693)--(10.276,6.699)--(10.286,6.705)--(10.296,6.712)--(10.306,6.719)--(10.316,6.725)%
  --(10.326,6.731)--(10.336,6.737)--(10.346,6.743)--(10.356,6.750)--(10.366,6.756)--(10.376,6.762)%
  --(10.386,6.768)--(10.396,6.774)--(10.406,6.781)--(10.416,6.787)--(10.426,6.793)--(10.436,6.799)%
  --(10.446,6.805)--(10.456,6.812)--(10.466,6.819)--(10.476,6.825)--(10.486,6.831)--(10.496,6.837)%
  --(10.506,6.844)--(10.516,6.850)--(10.526,6.856)--(10.536,6.862)--(10.546,6.868)--(10.556,6.875)%
  --(10.566,6.881)--(10.576,6.887)--(10.586,6.893)--(10.596,6.899)--(10.606,6.906)--(10.616,6.912)%
  --(10.626,6.918)--(10.636,6.925)--(10.646,6.931)--(10.656,6.937)--(10.666,6.944)--(10.676,6.950)%
  --(10.686,6.956)--(10.696,6.962)--(10.706,6.968)--(10.716,6.975)--(10.726,6.981)--(10.736,6.987)%
  --(10.746,6.993)--(10.756,6.999)--(10.767,7.006)--(10.777,7.012)--(10.787,7.018)--(10.797,7.024)%
  --(10.807,7.030)--(10.817,7.037)--(10.827,7.043)--(10.837,7.050)--(10.847,7.056)--(10.857,7.062)%
  --(10.867,7.069)--(10.877,7.075)--(10.887,7.081)--(10.897,7.087)--(10.907,7.093)--(10.917,7.100)%
  --(10.927,7.106)--(10.937,7.112)--(10.947,7.118)--(10.957,7.124)--(10.967,7.131)--(10.977,7.137)%
  --(10.987,7.143)--(10.997,7.149)--(11.007,7.155)--(11.017,7.162)--(11.027,7.168)--(11.037,7.174)%
  --(11.047,7.180)--(11.057,7.186)--(11.067,7.193)--(11.077,7.199)--(11.087,7.206)--(11.097,7.212)%
  --(11.107,7.218)--(11.117,7.224)--(11.127,7.231)--(11.137,7.237)--(11.147,7.243)--(11.157,7.249)%
  --(11.167,7.255)--(11.177,7.262)--(11.187,7.268)--(11.197,7.274)--(11.207,7.280)--(11.217,7.286)%
  --(11.227,7.293)--(11.237,7.299)--(11.247,7.305)--(11.257,7.311)--(11.267,7.317)--(11.277,7.324)%
  --(11.287,7.330)--(11.297,7.336)--(11.307,7.342)--(11.317,7.348)--(11.327,7.355)--(11.337,7.361)%
  --(11.347,7.367)--(11.357,7.373)--(11.367,7.379)--(11.377,7.386)--(11.387,7.392)--(11.397,7.398)%
  --(11.407,7.404)--(11.417,7.411)--(11.427,7.417)--(11.437,7.424)--(11.447,7.430)--(11.457,7.436)%
  --(11.467,7.442)--(11.477,7.448)--(11.487,7.455)--(11.497,7.461)--(11.507,7.467)--(11.517,7.473)%
  --(11.527,7.479)--(11.537,7.486)--(11.547,7.492)--(11.557,7.498)--(11.567,7.504)--(11.577,7.510)%
  --(11.587,7.517)--(11.597,7.523)--(11.607,7.529)--(11.617,7.535)--(11.627,7.541)--(11.637,7.548)%
  --(11.647,7.554)--(11.657,7.560)--(11.667,7.566)--(11.677,7.572)--(11.687,7.579)--(11.697,7.585)%
  --(11.707,7.591)--(11.717,7.597)--(11.727,7.603)--(11.737,7.610)--(11.747,7.616)--(11.757,7.622)%
  --(11.767,7.628)--(11.777,7.634)--(11.787,7.641)--(11.797,7.647)--(11.807,7.653)--(11.817,7.659)%
  --(11.827,7.665)--(11.837,7.672)--(11.847,7.678)--(11.857,7.684)--(11.867,7.690)--(11.877,7.696)%
  --(11.887,7.703)--(11.897,7.709)--(11.907,7.715)--(11.917,7.721)--(11.927,7.727)--(11.937,7.734)%
  --(11.947,7.740)--(11.957,7.746)--(11.967,7.752)--(11.977,7.758)--(11.987,7.764);
\gpcolor{color=gp lt color border}
\draw[gp path] (1.688,8.631)--(1.688,0.985)--(13.447,0.985)--(13.447,8.631)--cycle;
%% coordinates of the plot area
\gpdefrectangularnode{gp plot 1}{\pgfpoint{1.688cm}{0.985cm}}{\pgfpoint{13.447cm}{8.631cm}}
\end{tikzpicture}
%% gnuplot variables

	\caption{Gráfico da densidade escalar. \protect[Parameters: NJL $\rm{D}_1$, $m_0 = \np[MeV]{5.6}$]}
	\label{Fig:scalar_density_NJL-Buballa_Set_1}
\end{figure*}

\begin{figure*}
	\begin{tikzpicture}[gnuplot]
%% generated with GNUPLOT 5.0p2 (Lua 5.2; terminal rev. 99, script rev. 100)
%% Mon Apr  4 20:39:05 2016
\path (0.000,0.000) rectangle (14.000,9.000);
\gpcolor{color=gp lt color border}
\gpsetlinetype{gp lt border}
\gpsetdashtype{gp dt solid}
\gpsetlinewidth{1.00}
\draw[gp path] (1.320,0.985)--(1.500,0.985);
\draw[gp path] (13.447,0.985)--(13.267,0.985);
\node[gp node right] at (1.136,0.985) {$312$};
\draw[gp path] (1.320,1.750)--(1.500,1.750);
\draw[gp path] (13.447,1.750)--(13.267,1.750);
\node[gp node right] at (1.136,1.750) {$314$};
\draw[gp path] (1.320,2.514)--(1.500,2.514);
\draw[gp path] (13.447,2.514)--(13.267,2.514);
\node[gp node right] at (1.136,2.514) {$316$};
\draw[gp path] (1.320,3.279)--(1.500,3.279);
\draw[gp path] (13.447,3.279)--(13.267,3.279);
\node[gp node right] at (1.136,3.279) {$318$};
\draw[gp path] (1.320,4.043)--(1.500,4.043);
\draw[gp path] (13.447,4.043)--(13.267,4.043);
\node[gp node right] at (1.136,4.043) {$320$};
\draw[gp path] (1.320,4.808)--(1.500,4.808);
\draw[gp path] (13.447,4.808)--(13.267,4.808);
\node[gp node right] at (1.136,4.808) {$322$};
\draw[gp path] (1.320,5.573)--(1.500,5.573);
\draw[gp path] (13.447,5.573)--(13.267,5.573);
\node[gp node right] at (1.136,5.573) {$324$};
\draw[gp path] (1.320,6.337)--(1.500,6.337);
\draw[gp path] (13.447,6.337)--(13.267,6.337);
\node[gp node right] at (1.136,6.337) {$326$};
\draw[gp path] (1.320,7.102)--(1.500,7.102);
\draw[gp path] (13.447,7.102)--(13.267,7.102);
\node[gp node right] at (1.136,7.102) {$328$};
\draw[gp path] (1.320,7.866)--(1.500,7.866);
\draw[gp path] (13.447,7.866)--(13.267,7.866);
\node[gp node right] at (1.136,7.866) {$330$};
\draw[gp path] (1.320,8.631)--(1.500,8.631);
\draw[gp path] (13.447,8.631)--(13.267,8.631);
\node[gp node right] at (1.136,8.631) {$332$};
\draw[gp path] (1.320,0.985)--(1.320,1.165);
\draw[gp path] (1.320,8.631)--(1.320,8.451);
\node[gp node center] at (1.320,0.677) {$0$};
\draw[gp path] (3.341,0.985)--(3.341,1.165);
\draw[gp path] (3.341,8.631)--(3.341,8.451);
\node[gp node center] at (3.341,0.677) {$0.05$};
\draw[gp path] (5.362,0.985)--(5.362,1.165);
\draw[gp path] (5.362,8.631)--(5.362,8.451);
\node[gp node center] at (5.362,0.677) {$0.1$};
\draw[gp path] (7.384,0.985)--(7.384,1.165);
\draw[gp path] (7.384,8.631)--(7.384,8.451);
\node[gp node center] at (7.384,0.677) {$0.15$};
\draw[gp path] (9.405,0.985)--(9.405,1.165);
\draw[gp path] (9.405,8.631)--(9.405,8.451);
\node[gp node center] at (9.405,0.677) {$0.2$};
\draw[gp path] (11.426,0.985)--(11.426,1.165);
\draw[gp path] (11.426,8.631)--(11.426,8.451);
\node[gp node center] at (11.426,0.677) {$0.25$};
\draw[gp path] (13.447,0.985)--(13.447,1.165);
\draw[gp path] (13.447,8.631)--(13.447,8.451);
\node[gp node center] at (13.447,0.677) {$0.3$};
\draw[gp path] (1.320,8.631)--(1.320,0.985)--(13.447,0.985)--(13.447,8.631)--cycle;
\node[gp node center,rotate=-270] at (0.246,4.808) {$\mu_R$ (MeV)};
\node[gp node center] at (7.383,0.215) {$\rho$ ($\rm{fm}^{-3}$)};
\gpcolor{rgb color={0.580,0.000,0.827}}
\draw[gp path] (1.734,4.912)--(1.745,4.953)--(1.755,4.994)--(1.765,5.034)--(1.775,5.138)%
  --(1.786,5.177)--(1.796,5.214)--(1.806,5.251)--(1.816,5.287)--(1.826,5.322)--(1.837,5.357)%
  --(1.847,5.391)--(1.857,5.425)--(1.867,5.458)--(1.877,5.490)--(1.888,5.522)--(1.898,5.618)%
  --(1.908,5.649)--(1.918,5.679)--(1.929,5.708)--(1.939,5.737)--(1.949,5.766)--(1.959,5.794)%
  --(1.969,5.822)--(1.980,5.849)--(1.990,5.876)--(2.000,5.902)--(2.010,5.928)--(2.021,6.018)%
  --(2.031,6.043)--(2.041,6.068)--(2.051,6.092)--(2.061,6.116)--(2.072,6.140)--(2.082,6.163)%
  --(2.092,6.186)--(2.102,6.208)--(2.112,6.231)--(2.123,6.252)--(2.133,6.338)--(2.143,6.359)%
  --(2.153,6.380)--(2.164,6.400)--(2.174,6.420)--(2.184,6.440)--(2.194,6.460)--(2.204,6.479)%
  --(2.215,6.498)--(2.225,6.517)--(2.235,6.535)--(2.245,6.554)--(2.256,6.635)--(2.266,6.653)%
  --(2.276,6.670)--(2.286,6.687)--(2.296,6.704)--(2.307,6.721)--(2.317,6.737)--(2.327,6.753)%
  --(2.337,6.769)--(2.347,6.785)--(2.358,6.800)--(2.368,6.815)--(2.378,6.830)--(2.388,6.908)%
  --(2.399,6.923)--(2.409,6.937)--(2.419,6.951)--(2.429,6.965)--(2.439,6.979)--(2.450,6.993)%
  --(2.460,7.006)--(2.470,7.019)--(2.480,7.032)--(2.491,7.045)--(2.501,7.058)--(2.511,7.070)%
  --(2.521,7.145)--(2.531,7.157)--(2.542,7.169)--(2.552,7.180)--(2.562,7.192)--(2.572,7.203)%
  --(2.582,7.214)--(2.593,7.226)--(2.603,7.236)--(2.613,7.247)--(2.623,7.258)--(2.634,7.268)%
  --(2.644,7.279)--(2.654,7.350)--(2.664,7.360)--(2.674,7.370)--(2.685,7.380)--(2.695,7.390)%
  --(2.705,7.399)--(2.715,7.408)--(2.726,7.417)--(2.736,7.426)--(2.746,7.435)--(2.756,7.444)%
  --(2.766,7.453)--(2.777,7.461)--(2.787,7.470)--(2.797,7.539)--(2.807,7.547)--(2.817,7.555)%
  --(2.828,7.563)--(2.838,7.571)--(2.848,7.578)--(2.858,7.586)--(2.869,7.593)--(2.879,7.600)%
  --(2.889,7.607)--(2.899,7.615)--(2.909,7.622)--(2.920,7.628)--(2.930,7.635)--(2.940,7.642)%
  --(2.950,7.709)--(2.961,7.715)--(2.971,7.721)--(2.981,7.728)--(2.991,7.734)--(3.001,7.740)%
  --(3.012,7.746)--(3.022,7.751)--(3.032,7.757)--(3.042,7.763)--(3.052,7.768)--(3.063,7.774)%
  --(3.073,7.779)--(3.083,7.784)--(3.093,7.790)--(3.104,7.855)--(3.114,7.859)--(3.124,7.864)%
  --(3.134,7.869)--(3.144,7.874)--(3.155,7.879)--(3.165,7.883)--(3.175,7.888)--(3.185,7.892)%
  --(3.195,7.896)--(3.206,7.901)--(3.216,7.905)--(3.226,7.909)--(3.236,7.913)--(3.247,7.917)%
  --(3.257,7.921)--(3.267,7.924)--(3.277,7.928)--(3.287,7.991)--(3.298,7.994)--(3.308,7.998)%
  --(3.318,8.001)--(3.328,8.005)--(3.339,8.008)--(3.349,8.011)--(3.359,8.014)--(3.369,8.017)%
  --(3.379,8.020)--(3.390,8.023)--(3.400,8.026)--(3.410,8.029)--(3.420,8.032)--(3.430,8.034)%
  --(3.441,8.037)--(3.451,8.040)--(3.461,8.042)--(3.471,8.045)--(3.482,8.105)--(3.492,8.108)%
  --(3.502,8.110)--(3.512,8.112)--(3.522,8.114)--(3.533,8.116)--(3.543,8.119)--(3.553,8.120)%
  --(3.563,8.122)--(3.574,8.124)--(3.584,8.126)--(3.594,8.128)--(3.604,8.130)--(3.614,8.131)%
  --(3.625,8.133)--(3.635,8.134)--(3.645,8.136)--(3.655,8.138)--(3.665,8.139)--(3.676,8.140)%
  --(3.686,8.142)--(3.696,8.143)--(3.706,8.144)--(3.717,8.203)--(3.727,8.204)--(3.737,8.205)%
  --(3.747,8.206)--(3.757,8.207)--(3.768,8.208)--(3.778,8.209)--(3.788,8.210)--(3.798,8.211)%
  --(3.809,8.211)--(3.819,8.212)--(3.829,8.213)--(3.839,8.213)--(3.849,8.214)--(3.860,8.215)%
  --(3.870,8.215)--(3.880,8.216)--(3.890,8.216)--(3.900,8.217)--(3.911,8.217)--(3.921,8.217)%
  --(3.931,8.218)--(3.941,8.218)--(3.952,8.218)--(3.962,8.218)--(3.972,8.219)--(3.982,8.219)%
  --(3.992,8.219)--(4.003,8.219)--(4.013,8.219)--(4.023,8.233)--(4.033,8.261)--(4.044,8.261)%
  --(4.054,8.261)--(4.064,8.261)--(4.074,8.261)--(4.084,8.260)--(4.095,8.260)--(4.105,8.260)%
  --(4.115,8.260)--(4.125,8.259)--(4.135,8.259)--(4.146,8.259)--(4.156,8.258)--(4.166,8.258)%
  --(4.176,8.257)--(4.187,8.257)--(4.197,8.256)--(4.207,8.256)--(4.217,8.255)--(4.227,8.255)%
  --(4.238,8.254)--(4.248,8.253)--(4.258,8.253)--(4.268,8.280)--(4.279,8.279)--(4.289,8.278)%
  --(4.299,8.277)--(4.309,8.277)--(4.319,8.276)--(4.330,8.275)--(4.340,8.274)--(4.350,8.273)%
  --(4.360,8.272)--(4.370,8.271)--(4.381,8.270)--(4.391,8.269)--(4.401,8.268)--(4.411,8.267)%
  --(4.422,8.266)--(4.432,8.265)--(4.442,8.264)--(4.452,8.263)--(4.462,8.262)--(4.473,8.261)%
  --(4.483,8.259)--(4.493,8.258)--(4.503,8.257)--(4.514,8.256)--(4.524,8.255)--(4.534,8.253)%
  --(4.544,8.252)--(4.554,8.251)--(4.565,8.249)--(4.575,8.248)--(4.585,8.247)--(4.595,8.245)%
  --(4.605,8.244)--(4.616,8.243)--(4.626,8.241)--(4.636,8.240)--(4.646,8.238)--(4.657,8.237)%
  --(4.667,8.235)--(4.677,8.234)--(4.687,8.232)--(4.697,8.231)--(4.708,8.229)--(4.718,8.228)%
  --(4.728,8.226)--(4.738,8.225)--(4.748,8.223)--(4.759,8.221)--(4.769,8.220)--(4.779,8.218)%
  --(4.789,8.217)--(4.800,8.215)--(4.810,8.213)--(4.820,8.212)--(4.830,8.210)--(4.840,8.208)%
  --(4.851,8.207)--(4.861,8.205)--(4.871,8.203)--(4.881,8.201)--(4.892,8.200)--(4.902,8.171)%
  --(4.912,8.170)--(4.922,8.168)--(4.932,8.166)--(4.943,8.164)--(4.953,8.163)--(4.963,8.161)%
  --(4.973,8.159)--(4.983,8.157)--(4.994,8.155)--(5.004,8.153)--(5.014,8.152)--(5.024,8.150)%
  --(5.035,8.148)--(5.045,8.146)--(5.055,8.144)--(5.065,8.142)--(5.075,8.140)--(5.086,8.139)%
  --(5.096,8.137)--(5.106,8.135)--(5.116,8.133)--(5.127,8.105)--(5.137,8.103)--(5.147,8.101)%
  --(5.157,8.099)--(5.167,8.097)--(5.178,8.095)--(5.188,8.093)--(5.198,8.091)--(5.208,8.089)%
  --(5.218,8.087)--(5.229,8.085)--(5.239,8.084)--(5.249,8.082)--(5.259,8.080)--(5.270,8.078)%
  --(5.280,8.050)--(5.290,8.048)--(5.300,8.046)--(5.310,8.044)--(5.321,8.042)--(5.331,8.040)%
  --(5.341,8.038)--(5.351,8.036)--(5.362,8.034)--(5.372,8.032)--(5.382,8.030)--(5.392,8.028)%
  --(5.402,8.001)--(5.413,7.999)--(5.423,7.997)--(5.433,7.995)--(5.443,7.993)--(5.453,7.991)%
  --(5.464,7.989)--(5.474,7.987)--(5.484,7.985)--(5.494,7.983)--(5.505,7.955)--(5.515,7.953)%
  --(5.525,7.951)--(5.535,7.949)--(5.545,7.947)--(5.556,7.945)--(5.566,7.943)--(5.576,7.941)%
  --(5.586,7.939)--(5.597,7.912)--(5.607,7.910)--(5.617,7.908)--(5.627,7.906)--(5.637,7.904)%
  --(5.648,7.903)--(5.658,7.901)--(5.668,7.899)--(5.678,7.872)--(5.688,7.870)--(5.699,7.868)%
  --(5.709,7.866)--(5.719,7.864)--(5.729,7.862)--(5.740,7.860)--(5.750,7.858)--(5.760,7.831)%
  --(5.770,7.829)--(5.780,7.828)--(5.791,7.826)--(5.801,7.824)--(5.811,7.822)--(5.821,7.820)%
  --(5.832,7.793)--(5.842,7.791)--(5.852,7.790)--(5.862,7.788)--(5.872,7.786)--(5.883,7.784)%
  --(5.893,7.782)--(5.903,7.756)--(5.913,7.754)--(5.923,7.752)--(5.934,7.750)--(5.944,7.749)%
  --(5.954,7.747)--(5.964,7.720)--(5.975,7.719)--(5.985,7.717)--(5.995,7.715)--(6.005,7.713)%
  --(6.015,7.712)--(6.026,7.685)--(6.036,7.684)--(6.046,7.682)--(6.056,7.680)--(6.067,7.679)%
  --(6.077,7.653)--(6.087,7.651)--(6.097,7.649)--(6.107,7.648)--(6.118,7.646)--(6.128,7.620)%
  --(6.138,7.618)--(6.148,7.617)--(6.158,7.615)--(6.169,7.614)--(6.179,7.612)--(6.189,7.586)%
  --(6.199,7.585)--(6.210,7.583)--(6.220,7.582)--(6.230,7.556)--(6.240,7.554)--(6.250,7.553)%
  --(6.261,7.551)--(6.271,7.550)--(6.281,7.525)--(6.291,7.523)--(6.301,7.522)--(6.312,7.520)%
  --(6.322,7.519)--(6.332,7.493)--(6.342,7.492)--(6.353,7.491)--(6.363,7.489)--(6.373,7.464)%
  --(6.383,7.463)--(6.393,7.461)--(6.404,7.460)--(6.414,7.435)--(6.424,7.434)--(6.434,7.432)%
  --(6.445,7.431)--(6.455,7.430)--(6.465,7.405)--(6.475,7.404)--(6.485,7.403)--(6.496,7.401)%
  --(6.506,7.377)--(6.516,7.375)--(6.526,7.374)--(6.536,7.350)--(6.547,7.349)--(6.557,7.347)%
  --(6.567,7.346)--(6.577,7.322)--(6.588,7.321)--(6.598,7.320)--(6.608,7.319)--(6.618,7.294)%
  --(6.628,7.293)--(6.639,7.292)--(6.649,7.291)--(6.659,7.267)--(6.669,7.266)--(6.680,7.265)%
  --(6.690,7.241)--(6.700,7.240)--(6.710,7.239)--(6.720,7.238)--(6.731,7.214)--(6.741,7.214)%
  --(6.751,7.213)--(6.761,7.189)--(6.771,7.188)--(6.782,7.187)--(6.792,7.164)--(6.802,7.163)%
  --(6.812,7.162)--(6.823,7.161)--(6.833,7.138)--(6.843,7.137)--(6.853,7.137)--(6.863,7.113)%
  --(6.874,7.113)--(6.884,7.112)--(6.894,7.089)--(6.904,7.088)--(6.915,7.088)--(6.925,7.064)%
  --(6.935,7.064)--(6.945,7.063)--(6.955,7.040)--(6.966,7.040)--(6.976,7.039)--(6.986,7.016)%
  --(6.996,7.016)--(7.006,7.016)--(7.017,6.993)--(7.027,6.993)--(7.037,6.992)--(7.047,6.970)%
  --(7.058,6.969)--(7.068,6.969)--(7.078,6.946)--(7.088,6.946)--(7.098,6.924)--(7.109,6.923)%
  --(7.119,6.923)--(7.129,6.901)--(7.139,6.901)--(7.150,6.901)--(7.160,6.879)--(7.170,6.878)%
  --(7.180,6.856)--(7.190,6.856)--(7.201,6.856)--(7.211,6.834)--(7.221,6.834)--(7.231,6.834)%
  --(7.241,6.813)--(7.252,6.813)--(7.262,6.791)--(7.272,6.791)--(7.282,6.791)--(7.293,6.770)%
  --(7.303,6.770)--(7.313,6.748)--(7.323,6.748)--(7.333,6.727)--(7.344,6.727)--(7.354,6.728)%
  --(7.364,6.706)--(7.374,6.707)--(7.385,6.685)--(7.395,6.686)--(7.405,6.686)--(7.415,6.665)%
  --(7.425,6.666)--(7.436,6.645)--(7.446,6.645)--(7.456,6.624)--(7.466,6.625)--(7.476,6.604)%
  --(7.487,6.605)--(7.497,6.605)--(7.507,6.585)--(7.517,6.585)--(7.528,6.565)--(7.538,6.565)%
  --(7.548,6.545)--(7.558,6.546)--(7.568,6.525)--(7.579,6.526)--(7.589,6.506)--(7.599,6.507)%
  --(7.609,6.487)--(7.620,6.488)--(7.630,6.467)--(7.640,6.468)--(7.650,6.469)--(7.660,6.450)%
  --(7.671,6.451)--(7.681,6.431)--(7.691,6.432)--(7.701,6.412)--(7.711,6.413)--(7.722,6.394)%
  --(7.732,6.395)--(7.742,6.375)--(7.752,6.377)--(7.763,6.357)--(7.773,6.338)--(7.783,6.339)%
  --(7.793,6.320)--(7.803,6.321)--(7.814,6.302)--(7.824,6.303)--(7.834,6.284)--(7.844,6.286)%
  --(7.854,6.267)--(7.865,6.268)--(7.875,6.250)--(7.885,6.251)--(7.895,6.232)--(7.906,6.234)%
  --(7.916,6.215)--(7.926,6.197)--(7.936,6.198)--(7.946,6.180)--(7.957,6.182)--(7.967,6.163)%
  --(7.977,6.165)--(7.987,6.147)--(7.998,6.129)--(8.008,6.131)--(8.018,6.113)--(8.028,6.115)%
  --(8.038,6.096)--(8.049,6.078)--(8.059,6.081)--(8.069,6.063)--(8.079,6.065)--(8.089,6.047)%
  --(8.100,6.029)--(8.110,6.032)--(8.120,6.014)--(8.130,6.016)--(8.141,5.999)--(8.151,5.981)%
  --(8.161,5.984)--(8.171,5.967)--(8.181,5.969)--(8.192,5.952)--(8.202,5.935)--(8.212,5.937)%
  --(8.222,5.920)--(8.233,5.903)--(8.243,5.906)--(8.253,5.889)--(8.263,5.872)--(8.273,5.875)%
  --(8.284,5.858)--(8.294,5.841)--(8.304,5.844)--(8.314,5.828)--(8.324,5.811)--(8.335,5.814)%
  --(8.345,5.798)--(8.355,5.782)--(8.365,5.785)--(8.376,5.768)--(8.386,5.752)--(8.396,5.755)%
  --(8.406,5.739)--(8.416,5.723)--(8.427,5.708)--(8.437,5.711)--(8.447,5.695)--(8.457,5.679)%
  --(8.468,5.683)--(8.478,5.667)--(8.488,5.652)--(8.498,5.636)--(8.508,5.640)--(8.519,5.624)%
  --(8.529,5.609)--(8.539,5.594)--(8.549,5.597)--(8.559,5.582)--(8.570,5.567)--(8.580,5.552)%
  --(8.590,5.556)--(8.600,5.541)--(8.611,5.526)--(8.621,5.511)--(8.631,5.515)--(8.641,5.501)%
  --(8.651,5.486)--(8.662,5.472)--(8.672,5.457)--(8.682,5.461)--(8.692,5.447)--(8.703,5.433)%
  --(8.713,5.419)--(8.723,5.404)--(8.733,5.409)--(8.743,5.395)--(8.754,5.381)--(8.764,5.367)%
  --(8.774,5.353)--(8.784,5.339)--(8.794,5.344)--(8.805,5.330)--(8.815,5.317)--(8.825,5.303)%
  --(8.835,5.290)--(8.846,5.276)--(8.856,5.263)--(8.866,5.268)--(8.876,5.255)--(8.886,5.242)%
  --(8.897,5.229)--(8.907,5.216)--(8.917,5.203)--(8.927,5.190)--(8.938,5.177)--(8.948,5.165)%
  --(8.958,5.170)--(8.968,5.157)--(8.978,5.145)--(8.989,5.132)--(8.999,5.120)--(9.009,5.108)%
  --(9.019,5.095)--(9.029,5.083)--(9.040,5.071)--(9.050,5.059)--(9.060,5.047)--(9.070,5.035)%
  --(9.081,5.023)--(9.091,5.012)--(9.101,5.000)--(9.111,4.988)--(9.121,4.977)--(9.132,4.965)%
  --(9.142,4.971)--(9.152,4.960)--(9.162,4.949)--(9.173,4.938)--(9.183,4.926)--(9.193,4.915)%
  --(9.203,4.887)--(9.213,4.876)--(9.224,4.866)--(9.234,4.855)--(9.244,4.844)--(9.254,4.833)%
  --(9.264,4.823)--(9.275,4.813)--(9.285,4.802)--(9.295,4.792)--(9.305,4.782)--(9.316,4.771)%
  --(9.326,4.761)--(9.336,4.751)--(9.346,4.741)--(9.356,4.731)--(9.367,4.722)--(9.377,4.695)%
  --(9.387,4.686)--(9.397,4.676)--(9.407,4.666)--(9.418,4.657)--(9.428,4.648)--(9.438,4.638)%
  --(9.448,4.629)--(9.459,4.604)--(9.469,4.595)--(9.479,4.586)--(9.489,4.577)--(9.499,4.568)%
  --(9.510,4.559)--(9.520,4.534)--(9.530,4.526)--(9.540,4.517)--(9.551,4.509)--(9.561,4.500)%
  --(9.571,4.476)--(9.581,4.468)--(9.591,4.460)--(9.602,4.452)--(9.612,4.444)--(9.622,4.420)%
  --(9.632,4.412)--(9.642,4.404)--(9.653,4.397)--(9.663,4.373)--(9.673,4.366)--(9.683,4.359)%
  --(9.694,4.351)--(9.704,4.328)--(9.714,4.321)--(9.724,4.314)--(9.734,4.292)--(9.745,4.285)%
  --(9.755,4.278)--(9.765,4.256)--(9.775,4.249)--(9.786,4.242)--(9.796,4.220)--(9.806,4.214)%
  --(9.816,4.208)--(9.826,4.186)--(9.837,4.180)--(9.847,4.174)--(9.857,4.153)--(9.867,4.147)%
  --(9.877,4.126)--(9.888,4.120)--(9.898,4.099)--(9.908,4.094)--(9.918,4.088)--(9.929,4.068)%
  --(9.939,4.063)--(9.949,4.043)--(9.959,4.038)--(9.969,4.018)--(9.980,4.013)--(9.990,3.993)%
  --(10.000,3.988)--(10.010,3.969)--(10.021,3.964)--(10.031,3.945)--(10.041,3.941)--(10.051,3.922)%
  --(10.061,3.918)--(10.072,3.899)--(10.082,3.895)--(10.092,3.877)--(10.102,3.873)--(10.112,3.855)%
  --(10.123,3.837)--(10.133,3.833)--(10.143,3.815)--(10.153,3.812)--(10.164,3.794)--(10.174,3.777)%
  --(10.184,3.774)--(10.194,3.757)--(10.204,3.740)--(10.215,3.737)--(10.225,3.720)--(10.235,3.704)%
  --(10.245,3.701)--(10.256,3.685)--(10.266,3.669)--(10.276,3.653)--(10.286,3.651)--(10.296,3.635)%
  --(10.307,3.619)--(10.317,3.604)--(10.327,3.602)--(10.337,3.587)--(10.347,3.571)--(10.358,3.556)%
  --(10.368,3.542)--(10.378,3.540)--(10.388,3.526)--(10.399,3.511)--(10.409,3.497)--(10.419,3.483)%
  --(10.429,3.469)--(10.439,3.455)--(10.450,3.455)--(10.460,3.441)--(10.470,3.427)--(10.480,3.414)%
  --(10.491,3.401)--(10.501,3.388)--(10.511,3.375)--(10.521,3.362)--(10.531,3.350)--(10.542,3.337)%
  --(10.552,3.325)--(10.562,3.313)--(10.572,3.301)--(10.582,3.289)--(10.593,3.277)--(10.603,3.266)%
  --(10.613,3.254)--(10.623,3.231)--(10.634,3.219)--(10.644,3.208)--(10.654,3.198)--(10.664,3.187)%
  --(10.674,3.177)--(10.685,3.154)--(10.695,3.144)--(10.705,3.134)--(10.715,3.124)--(10.726,3.114)%
  --(10.736,3.092)--(10.746,3.083)--(10.756,3.074)--(10.766,3.053)--(10.777,3.044)--(10.787,3.035)%
  --(10.797,3.015)--(10.807,3.006)--(10.817,2.998)--(10.828,2.978)--(10.838,2.970)--(10.848,2.951)%
  --(10.858,2.943)--(10.869,2.936)--(10.879,2.917)--(10.889,2.910)--(10.899,2.891)--(10.909,2.885)%
  --(10.920,2.867)--(10.930,2.849)--(10.940,2.843)--(10.950,2.825)--(10.960,2.819)--(10.971,2.803)%
  --(10.981,2.786)--(10.991,2.781)--(11.001,2.764)--(11.012,2.748)--(11.022,2.743)--(11.032,2.728)%
  --(11.042,2.713)--(11.052,2.698)--(11.063,2.683)--(11.073,2.679)--(11.083,2.664)--(11.093,2.650)%
  --(11.104,2.636)--(11.114,2.623)--(11.124,2.609)--(11.134,2.596)--(11.144,2.583)--(11.155,2.570)%
  --(11.165,2.558)--(11.175,2.546)--(11.185,2.534)--(11.195,2.522)--(11.206,2.501)--(11.216,2.489)%
  --(11.226,2.478)--(11.236,2.468)--(11.247,2.457)--(11.257,2.437)--(11.267,2.427)--(11.277,2.417)%
  --(11.287,2.398)--(11.298,2.389)--(11.308,2.371)--(11.318,2.362)--(11.328,2.344)--(11.339,2.336)%
  --(11.349,2.321)--(11.359,2.309)--(11.369,2.293)--(11.379,2.281)--(11.390,2.270)--(11.400,2.254)%
  --(11.410,2.244)--(11.420,2.229)--(11.430,2.215)--(11.441,2.201)--(11.451,2.191)--(11.461,2.178)%
  --(11.471,2.165)--(11.482,2.148)--(11.492,2.135)--(11.502,2.123)--(11.512,2.108)--(11.522,2.097)%
  --(11.533,2.082)--(11.543,2.071)--(11.553,2.057)--(11.563,2.043)--(11.574,2.030)--(11.584,2.017)%
  --(11.594,2.005)--(11.604,1.989)--(11.614,1.977)--(11.625,1.962)--(11.635,1.952)--(11.645,1.937)%
  --(11.655,1.924)--(11.665,1.911)--(11.676,1.895)--(11.686,1.883)--(11.696,1.871)--(11.706,1.857)%
  --(11.717,1.843)--(11.727,1.829)--(11.737,1.816)--(11.747,1.801)--(11.757,1.789)--(11.768,1.775)%
  --(11.778,1.761)--(11.788,1.748)--(11.798,1.736)--(11.809,1.721)--(11.819,1.709)--(11.829,1.696)%
  --(11.839,1.680)--(11.849,1.668)--(11.860,1.654)--(11.870,1.641)--(11.880,1.628)--(11.890,1.614)%
  --(11.900,1.600)--(11.911,1.588)--(11.921,1.574)--(11.931,1.560)--(11.941,1.546);
\gpcolor{color=gp lt color border}
\draw[gp path] (1.320,8.631)--(1.320,0.985)--(13.447,0.985)--(13.447,8.631)--cycle;
%% coordinates of the plot area
\gpdefrectangularnode{gp plot 1}{\pgfpoint{1.320cm}{0.985cm}}{\pgfpoint{13.447cm}{8.631cm}}
\end{tikzpicture}
%% gnuplot variables

	\caption{Gráfico do potencial químico. \protect[Parameters: NJL $\rm{D}_1$, $m_0 = \np[MeV]{5.6}$]}
	\label{Fig:chemical_potential_NJL-Buballa_Set_1}
\end{figure*}

\begin{figure*}
	\begin{tikzpicture}[gnuplot]
%% generated with GNUPLOT 5.0p2 (Lua 5.2; terminal rev. 99, script rev. 100)
%% Fri Mar  4 16:19:36 2016
\path (0.000,0.000) rectangle (14.000,9.000);
\gpcolor{color=gp lt color border}
\gpsetlinetype{gp lt border}
\gpsetdashtype{gp dt solid}
\gpsetlinewidth{1.00}
\draw[gp path] (1.688,0.985)--(1.868,0.985);
\draw[gp path] (13.447,0.985)--(13.267,0.985);
\node[gp node right] at (1.504,0.985) {$-4500$};
\draw[gp path] (1.688,1.835)--(1.868,1.835);
\draw[gp path] (13.447,1.835)--(13.267,1.835);
\node[gp node right] at (1.504,1.835) {$-4000$};
\draw[gp path] (1.688,2.684)--(1.868,2.684);
\draw[gp path] (13.447,2.684)--(13.267,2.684);
\node[gp node right] at (1.504,2.684) {$-3500$};
\draw[gp path] (1.688,3.534)--(1.868,3.534);
\draw[gp path] (13.447,3.534)--(13.267,3.534);
\node[gp node right] at (1.504,3.534) {$-3000$};
\draw[gp path] (1.688,4.383)--(1.868,4.383);
\draw[gp path] (13.447,4.383)--(13.267,4.383);
\node[gp node right] at (1.504,4.383) {$-2500$};
\draw[gp path] (1.688,5.233)--(1.868,5.233);
\draw[gp path] (13.447,5.233)--(13.267,5.233);
\node[gp node right] at (1.504,5.233) {$-2000$};
\draw[gp path] (1.688,6.082)--(1.868,6.082);
\draw[gp path] (13.447,6.082)--(13.267,6.082);
\node[gp node right] at (1.504,6.082) {$-1500$};
\draw[gp path] (1.688,6.932)--(1.868,6.932);
\draw[gp path] (13.447,6.932)--(13.267,6.932);
\node[gp node right] at (1.504,6.932) {$-1000$};
\draw[gp path] (1.688,7.781)--(1.868,7.781);
\draw[gp path] (13.447,7.781)--(13.267,7.781);
\node[gp node right] at (1.504,7.781) {$-500$};
\draw[gp path] (1.688,8.631)--(1.868,8.631);
\draw[gp path] (13.447,8.631)--(13.267,8.631);
\node[gp node right] at (1.504,8.631) {$0$};
\draw[gp path] (1.688,0.985)--(1.688,1.165);
\draw[gp path] (1.688,8.631)--(1.688,8.451);
\node[gp node center] at (1.688,0.677) {$0$};
\draw[gp path] (3.368,0.985)--(3.368,1.165);
\draw[gp path] (3.368,8.631)--(3.368,8.451);
\node[gp node center] at (3.368,0.677) {$0.5$};
\draw[gp path] (5.048,0.985)--(5.048,1.165);
\draw[gp path] (5.048,8.631)--(5.048,8.451);
\node[gp node center] at (5.048,0.677) {$1$};
\draw[gp path] (6.728,0.985)--(6.728,1.165);
\draw[gp path] (6.728,8.631)--(6.728,8.451);
\node[gp node center] at (6.728,0.677) {$1.5$};
\draw[gp path] (8.407,0.985)--(8.407,1.165);
\draw[gp path] (8.407,8.631)--(8.407,8.451);
\node[gp node center] at (8.407,0.677) {$2$};
\draw[gp path] (10.087,0.985)--(10.087,1.165);
\draw[gp path] (10.087,8.631)--(10.087,8.451);
\node[gp node center] at (10.087,0.677) {$2.5$};
\draw[gp path] (11.767,0.985)--(11.767,1.165);
\draw[gp path] (11.767,8.631)--(11.767,8.451);
\node[gp node center] at (11.767,0.677) {$3$};
\draw[gp path] (13.447,0.985)--(13.447,1.165);
\draw[gp path] (13.447,8.631)--(13.447,8.451);
\node[gp node center] at (13.447,0.677) {$3.5$};
\draw[gp path] (1.688,8.631)--(1.688,0.985)--(13.447,0.985)--(13.447,8.631)--cycle;
\node[gp node center,rotate=-270] at (0.246,4.808) {$\omega$ ($\rm{MeV}/\rm{fm}^{3}$)};
\node[gp node center] at (7.567,0.215) {$\rho$ ($\rm{fm}^{-3}$)};
\gpcolor{rgb color={0.580,0.000,0.827}}
\draw[gp path] (1.732,8.029)--(1.742,8.026)--(1.752,8.024)--(1.762,8.022)--(1.772,8.019)%
  --(1.782,8.017)--(1.792,8.015)--(1.802,8.012)--(1.812,8.010)--(1.822,8.007)--(1.832,8.005)%
  --(1.842,8.003)--(1.852,8.000)--(1.862,7.998)--(1.872,7.996)--(1.882,7.994)--(1.893,7.991)%
  --(1.903,7.989)--(1.913,7.987)--(1.923,7.984)--(1.933,7.982)--(1.943,7.980)--(1.953,7.977)%
  --(1.963,7.975)--(1.973,7.973)--(1.983,7.970)--(1.993,7.968)--(2.003,7.966)--(2.013,7.963)%
  --(2.023,7.961)--(2.033,7.959)--(2.043,7.956)--(2.053,7.954)--(2.063,7.952)--(2.074,7.949)%
  --(2.084,7.947)--(2.094,7.945)--(2.104,7.942)--(2.114,7.940)--(2.124,7.938)--(2.134,7.935)%
  --(2.144,7.933)--(2.154,7.930)--(2.164,7.928)--(2.174,7.926)--(2.184,7.923)--(2.194,7.921)%
  --(2.204,7.918)--(2.214,7.916)--(2.224,7.913)--(2.234,7.911)--(2.244,7.908)--(2.255,7.906)%
  --(2.265,7.903)--(2.275,7.901)--(2.285,7.898)--(2.295,7.896)--(2.305,7.893)--(2.315,7.891)%
  --(2.325,7.888)--(2.335,7.886)--(2.345,7.883)--(2.355,7.880)--(2.365,7.878)--(2.375,7.875)%
  --(2.385,7.872)--(2.395,7.870)--(2.405,7.867)--(2.415,7.864)--(2.425,7.862)--(2.436,7.859)%
  --(2.446,7.856)--(2.456,7.853)--(2.466,7.851)--(2.476,7.848)--(2.486,7.845)--(2.496,7.842)%
  --(2.506,7.839)--(2.516,7.837)--(2.526,7.834)--(2.536,7.831)--(2.546,7.828)--(2.556,7.825)%
  --(2.566,7.822)--(2.576,7.819)--(2.586,7.816)--(2.596,7.813)--(2.606,7.810)--(2.617,7.807)%
  --(2.627,7.804)--(2.637,7.801)--(2.647,7.798)--(2.657,7.795)--(2.667,7.792)--(2.677,7.789)%
  --(2.687,7.786)--(2.697,7.783)--(2.707,7.779)--(2.717,7.776)--(2.727,7.773)--(2.737,7.770)%
  --(2.747,7.767)--(2.757,7.763)--(2.767,7.760)--(2.777,7.757)--(2.787,7.753)--(2.798,7.750)%
  --(2.808,7.747)--(2.818,7.743)--(2.828,7.740)--(2.838,7.737)--(2.848,7.733)--(2.858,7.730)%
  --(2.868,7.726)--(2.878,7.723)--(2.888,7.719)--(2.898,7.716)--(2.908,7.713)--(2.918,7.709)%
  --(2.928,7.705)--(2.938,7.702)--(2.948,7.698)--(2.958,7.695)--(2.968,7.691)--(2.979,7.687)%
  --(2.989,7.684)--(2.999,7.680)--(3.009,7.677)--(3.019,7.673)--(3.029,7.669)--(3.039,7.666)%
  --(3.049,7.662)--(3.059,7.658)--(3.069,7.654)--(3.079,7.650)--(3.089,7.647)--(3.099,7.643)%
  --(3.109,7.639)--(3.119,7.635)--(3.129,7.631)--(3.139,7.627)--(3.149,7.624)--(3.160,7.620)%
  --(3.170,7.616)--(3.180,7.612)--(3.190,7.608)--(3.200,7.604)--(3.210,7.600)--(3.220,7.596)%
  --(3.230,7.592)--(3.240,7.588)--(3.250,7.584)--(3.260,7.580)--(3.270,7.576)--(3.280,7.572)%
  --(3.290,7.568)--(3.300,7.564)--(3.310,7.560)--(3.320,7.556)--(3.330,7.551)--(3.341,7.547)%
  --(3.351,7.543)--(3.361,7.539)--(3.371,7.535)--(3.381,7.531)--(3.391,7.527)--(3.401,7.523)%
  --(3.411,7.518)--(3.421,7.514)--(3.431,7.510)--(3.441,7.506)--(3.451,7.502)--(3.461,7.498)%
  --(3.471,7.493)--(3.481,7.489)--(3.491,7.485)--(3.501,7.481)--(3.511,7.477)--(3.522,7.472)%
  --(3.532,7.468)--(3.542,7.464)--(3.552,7.460)--(3.562,7.455)--(3.572,7.451)--(3.582,7.447)%
  --(3.592,7.443)--(3.602,7.438)--(3.612,7.434)--(3.622,7.430)--(3.632,7.425)--(3.642,7.421)%
  --(3.652,7.417)--(3.662,7.413)--(3.672,7.409)--(3.682,7.404)--(3.692,7.400)--(3.703,7.396)%
  --(3.713,7.391)--(3.723,7.387)--(3.733,7.383)--(3.743,7.379)--(3.753,7.374)--(3.763,7.370)%
  --(3.773,7.366)--(3.783,7.362)--(3.793,7.358)--(3.803,7.354)--(3.813,7.349)--(3.823,7.345)%
  --(3.833,7.341)--(3.843,7.337)--(3.853,7.332)--(3.863,7.328)--(3.873,7.324)--(3.884,7.320)%
  --(3.894,7.316)--(3.904,7.312)--(3.914,7.308)--(3.924,7.304)--(3.934,7.299)--(3.944,7.295)%
  --(3.954,7.291)--(3.964,7.287)--(3.974,7.283)--(3.984,7.279)--(3.994,7.275)--(4.004,7.271)%
  --(4.014,7.267)--(4.024,7.263)--(4.034,7.259)--(4.044,7.255)--(4.054,7.251)--(4.065,7.248)%
  --(4.075,7.244)--(4.085,7.240)--(4.095,7.236)--(4.105,7.232)--(4.115,7.228)--(4.125,7.225)%
  --(4.135,7.221)--(4.145,7.217)--(4.155,7.213)--(4.165,7.210)--(4.175,7.206)--(4.185,7.202)%
  --(4.195,7.199)--(4.205,7.195)--(4.215,7.191)--(4.225,7.188)--(4.235,7.184)--(4.246,7.181)%
  --(4.256,7.177)--(4.266,7.174)--(4.276,7.171)--(4.286,7.167)--(4.296,7.164)--(4.306,7.160)%
  --(4.316,7.157)--(4.326,7.154)--(4.336,7.150)--(4.346,7.147)--(4.356,7.144)--(4.366,7.141)%
  --(4.376,7.137)--(4.386,7.134)--(4.396,7.131)--(4.406,7.128)--(4.416,7.125)--(4.427,7.122)%
  --(4.437,7.119)--(4.447,7.116)--(4.457,7.113)--(4.467,7.110)--(4.477,7.107)--(4.487,7.104)%
  --(4.497,7.101)--(4.507,7.099)--(4.517,7.096)--(4.527,7.093)--(4.537,7.090)--(4.547,7.088)%
  --(4.557,7.085)--(4.567,7.082)--(4.577,7.080)--(4.587,7.077)--(4.597,7.075)--(4.608,7.072)%
  --(4.618,7.070)--(4.628,7.067)--(4.638,7.065)--(4.648,7.062)--(4.658,7.060)--(4.668,7.058)%
  --(4.678,7.055)--(4.688,7.053)--(4.698,7.051)--(4.708,7.048)--(4.718,7.046)--(4.728,7.044)%
  --(4.738,7.042)--(4.748,7.040)--(4.758,7.037)--(4.768,7.035)--(4.778,7.033)--(4.789,7.031)%
  --(4.799,7.029)--(4.809,7.027)--(4.819,7.025)--(4.829,7.023)--(4.839,7.021)--(4.849,7.019)%
  --(4.859,7.017)--(4.869,7.015)--(4.879,7.013)--(4.889,7.011)--(4.899,7.009)--(4.909,7.008)%
  --(4.919,7.006)--(4.929,7.004)--(4.939,7.002)--(4.949,7.000)--(4.960,6.998)--(4.970,6.996)%
  --(4.980,6.995)--(4.990,6.993)--(5.000,6.991)--(5.010,6.989)--(5.020,6.987)--(5.030,6.985)%
  --(5.040,6.984)--(5.050,6.982)--(5.060,6.980)--(5.070,6.978)--(5.080,6.976)--(5.090,6.974)%
  --(5.100,6.973)--(5.110,6.971)--(5.120,6.969)--(5.130,6.967)--(5.141,6.965)--(5.151,6.963)%
  --(5.161,6.961)--(5.171,6.959)--(5.181,6.957)--(5.191,6.955)--(5.201,6.953)--(5.211,6.951)%
  --(5.221,6.949)--(5.231,6.947)--(5.241,6.945)--(5.251,6.943)--(5.261,6.941)--(5.271,6.939)%
  --(5.281,6.937)--(5.291,6.935)--(5.301,6.932)--(5.311,6.930)--(5.322,6.928)--(5.332,6.926)%
  --(5.342,6.923)--(5.352,6.921)--(5.362,6.919)--(5.372,6.916)--(5.382,6.914)--(5.392,6.911)%
  --(5.402,6.909)--(5.412,6.906)--(5.422,6.904)--(5.432,6.901)--(5.442,6.899)--(5.452,6.896)%
  --(5.462,6.894)--(5.472,6.891)--(5.482,6.888)--(5.492,6.885)--(5.503,6.883)--(5.513,6.880)%
  --(5.523,6.877)--(5.533,6.874)--(5.543,6.871)--(5.553,6.868)--(5.563,6.865)--(5.573,6.862)%
  --(5.583,6.859)--(5.593,6.856)--(5.603,6.853)--(5.613,6.850)--(5.623,6.847)--(5.633,6.843)%
  --(5.643,6.840)--(5.653,6.837)--(5.663,6.834)--(5.673,6.830)--(5.684,6.827)--(5.694,6.824)%
  --(5.704,6.820)--(5.714,6.817)--(5.724,6.813)--(5.734,6.810)--(5.744,6.806)--(5.754,6.803)%
  --(5.764,6.799)--(5.774,6.795)--(5.784,6.792)--(5.794,6.788)--(5.804,6.784)--(5.814,6.780)%
  --(5.824,6.777)--(5.834,6.773)--(5.844,6.769)--(5.854,6.765)--(5.865,6.761)--(5.875,6.757)%
  --(5.885,6.753)--(5.895,6.749)--(5.905,6.745)--(5.915,6.741)--(5.925,6.737)--(5.935,6.733)%
  --(5.945,6.729)--(5.955,6.725)--(5.965,6.720)--(5.975,6.716)--(5.985,6.712)--(5.995,6.708)%
  --(6.005,6.703)--(6.015,6.699)--(6.025,6.695)--(6.035,6.690)--(6.046,6.686)--(6.056,6.681)%
  --(6.066,6.677)--(6.076,6.672)--(6.086,6.668)--(6.096,6.663)--(6.106,6.659)--(6.116,6.654)%
  --(6.126,6.649)--(6.136,6.645)--(6.146,6.640)--(6.156,6.635)--(6.166,6.631)--(6.176,6.626)%
  --(6.186,6.621)--(6.196,6.616)--(6.206,6.612)--(6.216,6.607)--(6.227,6.602)--(6.237,6.597)%
  --(6.247,6.592)--(6.257,6.587)--(6.267,6.582)--(6.277,6.577)--(6.287,6.572)--(6.297,6.567)%
  --(6.307,6.562)--(6.317,6.557)--(6.327,6.552)--(6.337,6.547)--(6.347,6.542)--(6.357,6.537)%
  --(6.367,6.531)--(6.377,6.526)--(6.387,6.521)--(6.397,6.516)--(6.408,6.510)--(6.418,6.505)%
  --(6.428,6.500)--(6.438,6.495)--(6.448,6.489)--(6.458,6.484)--(6.468,6.479)--(6.478,6.473)%
  --(6.488,6.468)--(6.498,6.462)--(6.508,6.457)--(6.518,6.451)--(6.528,6.446)--(6.538,6.440)%
  --(6.548,6.435)--(6.558,6.429)--(6.568,6.424)--(6.578,6.418)--(6.589,6.413)--(6.599,6.407)%
  --(6.609,6.401)--(6.619,6.396)--(6.629,6.390)--(6.639,6.384)--(6.649,6.378)--(6.659,6.373)%
  --(6.669,6.367)--(6.679,6.361)--(6.689,6.355)--(6.699,6.350)--(6.709,6.344)--(6.719,6.338)%
  --(6.729,6.332)--(6.739,6.326)--(6.749,6.320)--(6.759,6.314)--(6.770,6.308)--(6.780,6.303)%
  --(6.790,6.297)--(6.800,6.291)--(6.810,6.285)--(6.820,6.279)--(6.830,6.273)--(6.840,6.267)%
  --(6.850,6.260)--(6.860,6.254)--(6.870,6.248)--(6.880,6.242)--(6.890,6.236)--(6.900,6.230)%
  --(6.910,6.224)--(6.920,6.218)--(6.930,6.211)--(6.940,6.205)--(6.951,6.199)--(6.961,6.193)%
  --(6.971,6.187)--(6.981,6.180)--(6.991,6.174)--(7.001,6.168)--(7.011,6.161)--(7.021,6.155)%
  --(7.031,6.149)--(7.041,6.142)--(7.051,6.136)--(7.061,6.130)--(7.071,6.123)--(7.081,6.117)%
  --(7.091,6.110)--(7.101,6.104)--(7.111,6.097)--(7.121,6.091)--(7.132,6.084)--(7.142,6.078)%
  --(7.152,6.071)--(7.162,6.065)--(7.172,6.058)--(7.182,6.052)--(7.192,6.045)--(7.202,6.039)%
  --(7.212,6.032)--(7.222,6.025)--(7.232,6.019)--(7.242,6.012)--(7.252,6.005)--(7.262,5.999)%
  --(7.272,5.992)--(7.282,5.985)--(7.292,5.979)--(7.302,5.972)--(7.313,5.965)--(7.323,5.958)%
  --(7.333,5.952)--(7.343,5.945)--(7.353,5.938)--(7.363,5.931)--(7.373,5.924)--(7.383,5.917)%
  --(7.393,5.911)--(7.403,5.904)--(7.413,5.897)--(7.423,5.890)--(7.433,5.883)--(7.443,5.876)%
  --(7.453,5.869)--(7.463,5.862)--(7.473,5.855)--(7.483,5.848)--(7.494,5.841)--(7.504,5.834)%
  --(7.514,5.827)--(7.524,5.820)--(7.534,5.813)--(7.544,5.806)--(7.554,5.799)--(7.564,5.792)%
  --(7.574,5.785)--(7.584,5.778)--(7.594,5.770)--(7.604,5.763)--(7.614,5.756)--(7.624,5.749)%
  --(7.634,5.742)--(7.644,5.735)--(7.654,5.727)--(7.664,5.720)--(7.675,5.713)--(7.685,5.706)%
  --(7.695,5.698)--(7.705,5.691)--(7.715,5.684)--(7.725,5.677)--(7.735,5.669)--(7.745,5.662)%
  --(7.755,5.655)--(7.765,5.647)--(7.775,5.640)--(7.785,5.633)--(7.795,5.625)--(7.805,5.618)%
  --(7.815,5.610)--(7.825,5.603)--(7.835,5.595)--(7.845,5.588)--(7.856,5.581)--(7.866,5.573)%
  --(7.876,5.566)--(7.886,5.558)--(7.896,5.551)--(7.906,5.543)--(7.916,5.535)--(7.926,5.528)%
  --(7.936,5.520)--(7.946,5.513)--(7.956,5.505)--(7.966,5.498)--(7.976,5.490)--(7.986,5.482)%
  --(7.996,5.475)--(8.006,5.467)--(8.016,5.459)--(8.026,5.452)--(8.037,5.444)--(8.047,5.436)%
  --(8.057,5.429)--(8.067,5.421)--(8.077,5.413)--(8.087,5.405)--(8.097,5.398)--(8.107,5.390)%
  --(8.117,5.382)--(8.127,5.374)--(8.137,5.366)--(8.147,5.359)--(8.157,5.351)--(8.167,5.343)%
  --(8.177,5.335)--(8.187,5.327)--(8.197,5.319)--(8.207,5.311)--(8.218,5.303)--(8.228,5.296)%
  --(8.238,5.288)--(8.248,5.280)--(8.258,5.272)--(8.268,5.264)--(8.278,5.256)--(8.288,5.248)%
  --(8.298,5.240)--(8.308,5.232)--(8.318,5.224)--(8.328,5.216)--(8.338,5.208)--(8.348,5.200)%
  --(8.358,5.191)--(8.368,5.183)--(8.378,5.175)--(8.388,5.167)--(8.399,5.159)--(8.409,5.151)%
  --(8.419,5.143)--(8.429,5.135)--(8.439,5.126)--(8.449,5.118)--(8.459,5.110)--(8.469,5.102)%
  --(8.479,5.094)--(8.489,5.085)--(8.499,5.077)--(8.509,5.069)--(8.519,5.061)--(8.529,5.052)%
  --(8.539,5.044)--(8.549,5.036)--(8.559,5.027)--(8.569,5.019)--(8.580,5.011)--(8.590,5.002)%
  --(8.600,4.994)--(8.610,4.986)--(8.620,4.977)--(8.630,4.969)--(8.640,4.960)--(8.650,4.952)%
  --(8.660,4.943)--(8.670,4.935)--(8.680,4.927)--(8.690,4.918)--(8.700,4.910)--(8.710,4.901)%
  --(8.720,4.893)--(8.730,4.884)--(8.740,4.876)--(8.750,4.867)--(8.761,4.858)--(8.771,4.850)%
  --(8.781,4.841)--(8.791,4.833)--(8.801,4.824)--(8.811,4.816)--(8.821,4.807)--(8.831,4.798)%
  --(8.841,4.790)--(8.851,4.781)--(8.861,4.772)--(8.871,4.764)--(8.881,4.755)--(8.891,4.746)%
  --(8.901,4.738)--(8.911,4.729)--(8.921,4.720)--(8.931,4.711)--(8.942,4.703)--(8.952,4.694)%
  --(8.962,4.685)--(8.972,4.676)--(8.982,4.667)--(8.992,4.659)--(9.002,4.650)--(9.012,4.641)%
  --(9.022,4.632)--(9.032,4.623)--(9.042,4.614)--(9.052,4.605)--(9.062,4.597)--(9.072,4.588)%
  --(9.082,4.579)--(9.092,4.570)--(9.102,4.561)--(9.112,4.552)--(9.123,4.543)--(9.133,4.534)%
  --(9.143,4.525)--(9.153,4.516)--(9.163,4.507)--(9.173,4.498)--(9.183,4.489)--(9.193,4.480)%
  --(9.203,4.471)--(9.213,4.462)--(9.223,4.453)--(9.233,4.443)--(9.243,4.434)--(9.253,4.425)%
  --(9.263,4.416)--(9.273,4.407)--(9.283,4.398)--(9.293,4.389)--(9.304,4.379)--(9.314,4.370)%
  --(9.324,4.361)--(9.334,4.352)--(9.344,4.343)--(9.354,4.333)--(9.364,4.324)--(9.374,4.315)%
  --(9.384,4.306)--(9.394,4.296)--(9.404,4.287)--(9.414,4.278)--(9.424,4.268)--(9.434,4.259)%
  --(9.444,4.250)--(9.454,4.240)--(9.464,4.231)--(9.474,4.222)--(9.485,4.212)--(9.495,4.203)%
  --(9.505,4.194)--(9.515,4.184)--(9.525,4.175)--(9.535,4.165)--(9.545,4.156)--(9.555,4.146)%
  --(9.565,4.137)--(9.575,4.127)--(9.585,4.118)--(9.595,4.108)--(9.605,4.099)--(9.615,4.089)%
  --(9.625,4.080)--(9.635,4.070)--(9.645,4.061)--(9.655,4.051)--(9.666,4.042)--(9.676,4.032)%
  --(9.686,4.022)--(9.696,4.013)--(9.706,4.003)--(9.716,3.994)--(9.726,3.984)--(9.736,3.974)%
  --(9.746,3.965)--(9.756,3.955)--(9.766,3.945)--(9.776,3.935)--(9.786,3.926)--(9.796,3.916)%
  --(9.806,3.906)--(9.816,3.897)--(9.826,3.887)--(9.836,3.877)--(9.847,3.867)--(9.857,3.857)%
  --(9.867,3.848)--(9.877,3.838)--(9.887,3.828)--(9.897,3.818)--(9.907,3.808)--(9.917,3.798)%
  --(9.927,3.788)--(9.937,3.779)--(9.947,3.769)--(9.957,3.759)--(9.967,3.749)--(9.977,3.739)%
  --(9.987,3.729)--(9.997,3.719)--(10.007,3.709)--(10.017,3.699)--(10.028,3.689)--(10.038,3.679)%
  --(10.048,3.669)--(10.058,3.659)--(10.068,3.649)--(10.078,3.639)--(10.088,3.629)--(10.098,3.619)%
  --(10.108,3.609)--(10.118,3.599)--(10.128,3.589)--(10.138,3.578)--(10.148,3.568)--(10.158,3.558)%
  --(10.168,3.548)--(10.178,3.538)--(10.188,3.528)--(10.198,3.518)--(10.209,3.507)--(10.219,3.497)%
  --(10.229,3.487)--(10.239,3.477)--(10.249,3.466)--(10.259,3.456)--(10.269,3.446)--(10.279,3.436)%
  --(10.289,3.425)--(10.299,3.415)--(10.309,3.405)--(10.319,3.395)--(10.329,3.384)--(10.339,3.374)%
  --(10.349,3.363)--(10.359,3.353)--(10.369,3.343)--(10.379,3.332)--(10.390,3.322)--(10.400,3.312)%
  --(10.410,3.301)--(10.420,3.291)--(10.430,3.280)--(10.440,3.270)--(10.450,3.259)--(10.460,3.249)%
  --(10.470,3.238)--(10.480,3.228)--(10.490,3.217)--(10.500,3.207)--(10.510,3.196)--(10.520,3.186)%
  --(10.530,3.175)--(10.540,3.165)--(10.550,3.154)--(10.560,3.144)--(10.571,3.133)--(10.581,3.122)%
  --(10.591,3.112)--(10.601,3.101)--(10.611,3.090)--(10.621,3.080)--(10.631,3.069)--(10.641,3.058)%
  --(10.651,3.048)--(10.661,3.037)--(10.671,3.026)--(10.681,3.016)--(10.691,3.005)--(10.701,2.994)%
  --(10.711,2.984)--(10.721,2.973)--(10.731,2.962)--(10.741,2.951)--(10.752,2.940)--(10.762,2.930)%
  --(10.772,2.919)--(10.782,2.908)--(10.792,2.897)--(10.802,2.886)--(10.812,2.875)--(10.822,2.864)%
  --(10.832,2.854)--(10.842,2.843)--(10.852,2.832)--(10.862,2.821)--(10.872,2.810)--(10.882,2.799)%
  --(10.892,2.788)--(10.902,2.777)--(10.912,2.766)--(10.922,2.755)--(10.933,2.744)--(10.943,2.733)%
  --(10.953,2.722)--(10.963,2.711)--(10.973,2.700)--(10.983,2.689)--(10.993,2.678)--(11.003,2.667)%
  --(11.013,2.656)--(11.023,2.645)--(11.033,2.634)--(11.043,2.622)--(11.053,2.611)--(11.063,2.600)%
  --(11.073,2.589)--(11.083,2.578)--(11.093,2.567)--(11.103,2.555)--(11.114,2.544)--(11.124,2.533)%
  --(11.134,2.522)--(11.144,2.511)--(11.154,2.499)--(11.164,2.488)--(11.174,2.477)--(11.184,2.466)%
  --(11.194,2.454)--(11.204,2.443)--(11.214,2.432)--(11.224,2.420)--(11.234,2.409)--(11.244,2.398)%
  --(11.254,2.386)--(11.264,2.375)--(11.274,2.364)--(11.284,2.352)--(11.295,2.341)--(11.305,2.329)%
  --(11.315,2.318)--(11.325,2.307)--(11.335,2.295)--(11.345,2.284)--(11.355,2.272)--(11.365,2.261)%
  --(11.375,2.249)--(11.385,2.238)--(11.395,2.226)--(11.405,2.215)--(11.415,2.203)--(11.425,2.192)%
  --(11.435,2.180)--(11.445,2.169)--(11.455,2.157)--(11.465,2.145)--(11.476,2.134)--(11.486,2.122)%
  --(11.496,2.111)--(11.506,2.099)--(11.516,2.087)--(11.526,2.076)--(11.536,2.064)--(11.546,2.052)%
  --(11.556,2.041)--(11.566,2.029)--(11.576,2.017)--(11.586,2.006)--(11.596,1.994)--(11.606,1.982)%
  --(11.616,1.970)--(11.626,1.959)--(11.636,1.947)--(11.646,1.935)--(11.657,1.923)--(11.667,1.911)%
  --(11.677,1.900)--(11.687,1.888)--(11.697,1.876)--(11.707,1.864)--(11.717,1.852)--(11.727,1.840)%
  --(11.737,1.828)--(11.747,1.816)--(11.757,1.805)--(11.767,1.793)--(11.777,1.781);
\gpcolor{color=gp lt color border}
\draw[gp path] (1.688,8.631)--(1.688,0.985)--(13.447,0.985)--(13.447,8.631)--cycle;
%% coordinates of the plot area
\gpdefrectangularnode{gp plot 1}{\pgfpoint{1.688cm}{0.985cm}}{\pgfpoint{13.447cm}{8.631cm}}
\end{tikzpicture}
%% gnuplot variables

	\caption{Gráfico do potencial termodinâmico $\tilde{\omega}$. \protect[Parameters: NJL $\rm{D}_1$, $m_0 = \np[MeV]{5.6}$]}
	\label{Fig:thermodynamic_potential_NJL-Buballa_Set_1}
\end{figure*}

\begin{figure*}
	\begin{tikzpicture}[gnuplot]
%% generated with GNUPLOT 5.0p2 (Lua 5.2; terminal rev. 99, script rev. 100)
%% Mon Mar  7 16:54:51 2016
\path (0.000,0.000) rectangle (14.000,9.000);
\gpcolor{color=gp lt color border}
\gpsetlinetype{gp lt border}
\gpsetdashtype{gp dt solid}
\gpsetlinewidth{1.00}
\draw[gp path] (1.320,0.985)--(1.500,0.985);
\draw[gp path] (13.447,0.985)--(13.267,0.985);
\node[gp node right] at (1.136,0.985) {$400$};
\draw[gp path] (1.320,2.259)--(1.500,2.259);
\draw[gp path] (13.447,2.259)--(13.267,2.259);
\node[gp node right] at (1.136,2.259) {$405$};
\draw[gp path] (1.320,3.534)--(1.500,3.534);
\draw[gp path] (13.447,3.534)--(13.267,3.534);
\node[gp node right] at (1.136,3.534) {$410$};
\draw[gp path] (1.320,4.808)--(1.500,4.808);
\draw[gp path] (13.447,4.808)--(13.267,4.808);
\node[gp node right] at (1.136,4.808) {$415$};
\draw[gp path] (1.320,6.082)--(1.500,6.082);
\draw[gp path] (13.447,6.082)--(13.267,6.082);
\node[gp node right] at (1.136,6.082) {$420$};
\draw[gp path] (1.320,7.357)--(1.500,7.357);
\draw[gp path] (13.447,7.357)--(13.267,7.357);
\node[gp node right] at (1.136,7.357) {$425$};
\draw[gp path] (1.320,8.631)--(1.500,8.631);
\draw[gp path] (13.447,8.631)--(13.267,8.631);
\node[gp node right] at (1.136,8.631) {$430$};
\draw[gp path] (1.320,0.985)--(1.320,1.165);
\draw[gp path] (1.320,8.631)--(1.320,8.451);
\node[gp node center] at (1.320,0.677) {$0$};
\draw[gp path] (2.836,0.985)--(2.836,1.165);
\draw[gp path] (2.836,8.631)--(2.836,8.451);
\node[gp node center] at (2.836,0.677) {$0.05$};
\draw[gp path] (4.352,0.985)--(4.352,1.165);
\draw[gp path] (4.352,8.631)--(4.352,8.451);
\node[gp node center] at (4.352,0.677) {$0.1$};
\draw[gp path] (5.868,0.985)--(5.868,1.165);
\draw[gp path] (5.868,8.631)--(5.868,8.451);
\node[gp node center] at (5.868,0.677) {$0.15$};
\draw[gp path] (7.384,0.985)--(7.384,1.165);
\draw[gp path] (7.384,8.631)--(7.384,8.451);
\node[gp node center] at (7.384,0.677) {$0.2$};
\draw[gp path] (8.899,0.985)--(8.899,1.165);
\draw[gp path] (8.899,8.631)--(8.899,8.451);
\node[gp node center] at (8.899,0.677) {$0.25$};
\draw[gp path] (10.415,0.985)--(10.415,1.165);
\draw[gp path] (10.415,8.631)--(10.415,8.451);
\node[gp node center] at (10.415,0.677) {$0.3$};
\draw[gp path] (11.931,0.985)--(11.931,1.165);
\draw[gp path] (11.931,8.631)--(11.931,8.451);
\node[gp node center] at (11.931,0.677) {$0.35$};
\draw[gp path] (13.447,0.985)--(13.447,1.165);
\draw[gp path] (13.447,8.631)--(13.447,8.451);
\node[gp node center] at (13.447,0.677) {$0.4$};
\draw[gp path] (1.320,8.631)--(1.320,0.985)--(13.447,0.985)--(13.447,8.631)--cycle;
\node[gp node center,rotate=-270] at (0.246,4.808) {$P$ ($\rm{MeV}/\rm{fm}^3$)};
\node[gp node center] at (7.383,0.215) {$\rho$ ($\rm{fm}^{-3}$)};
\gpcolor{rgb color={0.580,0.000,0.827}}
\draw[gp path] (1.633,1.765)--(1.644,1.766)--(1.654,1.765)--(1.664,1.765)--(1.675,1.764)%
  --(1.685,1.764)--(1.695,1.763)--(1.706,1.763)--(1.716,1.762)--(1.726,1.763)--(1.737,1.762)%
  --(1.747,1.762)--(1.757,1.761)--(1.768,1.761)--(1.778,1.760)--(1.788,1.760)--(1.799,1.759)%
  --(1.809,1.759)--(1.819,1.758)--(1.830,1.758)--(1.840,1.757)--(1.850,1.757)--(1.860,1.756)%
  --(1.871,1.756)--(1.881,1.754)--(1.891,1.755)--(1.902,1.753)--(1.912,1.754)--(1.922,1.752)%
  --(1.933,1.752)--(1.943,1.751)--(1.953,1.751)--(1.964,1.749)--(1.974,1.750)--(1.984,1.748)%
  --(1.995,1.748)--(2.005,1.747)--(2.015,1.747)--(2.026,1.745)--(2.036,1.745)--(2.046,1.744)%
  --(2.057,1.744)--(2.067,1.742)--(2.077,1.742)--(2.087,1.740)--(2.098,1.741)--(2.108,1.739)%
  --(2.118,1.739)--(2.129,1.737)--(2.139,1.737)--(2.149,1.735)--(2.160,1.736)--(2.170,1.734)%
  --(2.180,1.734)--(2.191,1.732)--(2.201,1.732)--(2.211,1.730)--(2.222,1.730)--(2.232,1.728)%
  --(2.242,1.728)--(2.253,1.729)--(2.263,1.727)--(2.273,1.727)--(2.284,1.725)--(2.294,1.725)%
  --(2.304,1.723)--(2.314,1.723)--(2.325,1.720)--(2.335,1.721)--(2.345,1.718)--(2.356,1.719)%
  --(2.366,1.716)--(2.376,1.717)--(2.387,1.714)--(2.397,1.715)--(2.407,1.712)--(2.418,1.712)%
  --(2.428,1.710)--(2.438,1.710)--(2.449,1.711)--(2.459,1.708)--(2.469,1.709)--(2.480,1.706)%
  --(2.490,1.706)--(2.500,1.703)--(2.511,1.704)--(2.521,1.701)--(2.531,1.702)--(2.542,1.699)%
  --(2.552,1.699)--(2.562,1.696)--(2.572,1.697)--(2.583,1.697)--(2.593,1.694)--(2.603,1.695)%
  --(2.614,1.692)--(2.624,1.693)--(2.634,1.689)--(2.645,1.690)--(2.655,1.687)--(2.665,1.688)%
  --(2.676,1.684)--(2.686,1.685)--(2.696,1.686)--(2.707,1.682)--(2.717,1.683)--(2.727,1.680)%
  --(2.738,1.680)--(2.748,1.677)--(2.758,1.678)--(2.769,1.674)--(2.779,1.675)--(2.789,1.676)%
  --(2.799,1.672)--(2.810,1.673)--(2.820,1.670)--(2.830,1.671)--(2.841,1.667)--(2.851,1.668)%
  --(2.861,1.664)--(2.872,1.665)--(2.882,1.666)--(2.892,1.662)--(2.903,1.663)--(2.913,1.659)%
  --(2.923,1.660)--(2.934,1.656)--(2.944,1.657)--(2.954,1.654)--(2.965,1.655)--(2.975,1.656)%
  --(2.985,1.652)--(2.996,1.653)--(3.006,1.649)--(3.016,1.650)--(3.026,1.646)--(3.037,1.647)%
  --(3.047,1.648)--(3.057,1.644)--(3.068,1.645)--(3.078,1.641)--(3.088,1.642)--(3.099,1.638)%
  --(3.109,1.639)--(3.119,1.640)--(3.130,1.636)--(3.140,1.637)--(3.150,1.633)--(3.161,1.634)%
  --(3.171,1.635)--(3.181,1.631)--(3.192,1.632)--(3.202,1.628)--(3.212,1.629)--(3.223,1.625)%
  --(3.233,1.626)--(3.243,1.627)--(3.253,1.623)--(3.264,1.624)--(3.274,1.620)--(3.284,1.621)%
  --(3.295,1.617)--(3.305,1.618)--(3.315,1.619)--(3.326,1.615)--(3.336,1.616)--(3.346,1.612)%
  --(3.357,1.613)--(3.367,1.615)--(3.377,1.610)--(3.388,1.612)--(3.398,1.607)--(3.408,1.608)%
  --(3.419,1.610)--(3.429,1.605)--(3.439,1.607)--(3.450,1.602)--(3.460,1.604)--(3.470,1.605)%
  --(3.480,1.600)--(3.491,1.602)--(3.501,1.597)--(3.511,1.599)--(3.522,1.594)--(3.532,1.596)%
  --(3.542,1.597)--(3.553,1.592)--(3.563,1.594)--(3.573,1.589)--(3.584,1.591)--(3.594,1.593)%
  --(3.604,1.588)--(3.615,1.589)--(3.625,1.591)--(3.635,1.586)--(3.646,1.588)--(3.656,1.583)%
  --(3.666,1.585)--(3.677,1.587)--(3.687,1.582)--(3.697,1.583)--(3.707,1.578)--(3.718,1.580)%
  --(3.728,1.582)--(3.738,1.577)--(3.749,1.579)--(3.759,1.574)--(3.769,1.576)--(3.780,1.578)%
  --(3.790,1.572)--(3.800,1.575)--(3.811,1.569)--(3.821,1.571)--(3.831,1.573)--(3.842,1.568)%
  --(3.852,1.570)--(3.862,1.572)--(3.873,1.567)--(3.883,1.569)--(3.893,1.564)--(3.904,1.566)%
  --(3.914,1.568)--(3.924,1.563)--(3.934,1.565)--(3.945,1.560)--(3.955,1.562)--(3.965,1.564)%
  --(3.976,1.559)--(3.986,1.561)--(3.996,1.563)--(4.007,1.558)--(4.017,1.560)--(4.027,1.555)%
  --(4.038,1.557)--(4.048,1.559)--(4.058,1.554)--(4.069,1.556)--(4.079,1.559)--(4.089,1.553)%
  --(4.100,1.556)--(4.110,1.550)--(4.120,1.552)--(4.131,1.555)--(4.141,1.549)--(4.151,1.552)%
  --(4.161,1.554)--(4.172,1.549)--(4.182,1.551)--(4.192,1.554)--(4.203,1.548)--(4.213,1.551)%
  --(4.223,1.545)--(4.234,1.548)--(4.244,1.551)--(4.254,1.545)--(4.265,1.548)--(4.275,1.550)%
  --(4.285,1.545)--(4.296,1.547)--(4.306,1.550)--(4.316,1.544)--(4.327,1.547)--(4.337,1.542)%
  --(4.347,1.544)--(4.358,1.547)--(4.368,1.541)--(4.378,1.544)--(4.388,1.547)--(4.399,1.541)%
  --(4.409,1.544)--(4.419,1.548)--(4.430,1.542)--(4.440,1.545)--(4.450,1.548)--(4.461,1.542)%
  --(4.471,1.545)--(4.481,1.539)--(4.492,1.542)--(4.502,1.545)--(4.512,1.539)--(4.523,1.542)%
  --(4.533,1.546)--(4.543,1.540)--(4.554,1.543)--(4.564,1.546)--(4.574,1.540)--(4.585,1.543)%
  --(4.595,1.547)--(4.605,1.541)--(4.615,1.544)--(4.626,1.548)--(4.636,1.541)--(4.646,1.545)%
  --(4.657,1.548)--(4.667,1.542)--(4.677,1.546)--(4.688,1.540)--(4.698,1.543)--(4.708,1.547)%
  --(4.719,1.540)--(4.729,1.544)--(4.739,1.548)--(4.750,1.541)--(4.760,1.545)--(4.770,1.549)%
  --(4.781,1.543)--(4.791,1.546)--(4.801,1.550)--(4.812,1.544)--(4.822,1.548)--(4.832,1.551)%
  --(4.842,1.545)--(4.853,1.549)--(4.863,1.553)--(4.873,1.547)--(4.884,1.550)--(4.894,1.554)%
  --(4.904,1.548)--(4.915,1.552)--(4.925,1.556)--(4.935,1.550)--(4.946,1.554)--(4.956,1.558)%
  --(4.966,1.551)--(4.977,1.556)--(4.987,1.560)--(4.997,1.553)--(5.008,1.558)--(5.018,1.562)%
  --(5.028,1.555)--(5.039,1.560)--(5.049,1.564)--(5.059,1.557)--(5.069,1.562)--(5.080,1.566)%
  --(5.090,1.560)--(5.100,1.564)--(5.111,1.568)--(5.121,1.562)--(5.131,1.566)--(5.142,1.571)%
  --(5.152,1.575)--(5.162,1.569)--(5.173,1.573)--(5.183,1.578)--(5.193,1.571)--(5.204,1.576)%
  --(5.214,1.581)--(5.224,1.574)--(5.235,1.579)--(5.245,1.584)--(5.255,1.577)--(5.266,1.582)%
  --(5.276,1.587)--(5.286,1.580)--(5.296,1.585)--(5.307,1.590)--(5.317,1.583)--(5.327,1.588)%
  --(5.338,1.593)--(5.348,1.586)--(5.358,1.591)--(5.369,1.596)--(5.379,1.601)--(5.389,1.595)%
  --(5.400,1.600)--(5.410,1.605)--(5.420,1.598)--(5.431,1.603)--(5.441,1.608)--(5.451,1.602)%
  --(5.462,1.607)--(5.472,1.612)--(5.482,1.605)--(5.493,1.611)--(5.503,1.616)--(5.513,1.609)%
  --(5.523,1.615)--(5.534,1.620)--(5.544,1.625)--(5.554,1.619)--(5.565,1.624)--(5.575,1.630)%
  --(5.585,1.623)--(5.596,1.628)--(5.606,1.634)--(5.616,1.627)--(5.627,1.633)--(5.637,1.639)%
  --(5.647,1.644)--(5.658,1.637)--(5.668,1.643)--(5.678,1.649)--(5.689,1.642)--(5.699,1.648)%
  --(5.709,1.654)--(5.720,1.647)--(5.730,1.653)--(5.740,1.659)--(5.750,1.665)--(5.761,1.658)%
  --(5.771,1.664)--(5.781,1.670)--(5.792,1.663)--(5.802,1.669)--(5.812,1.675)--(5.823,1.668)%
  --(5.833,1.674)--(5.843,1.680)--(5.854,1.687)--(5.864,1.680)--(5.874,1.686)--(5.885,1.692)%
  --(5.895,1.686)--(5.905,1.692)--(5.916,1.698)--(5.926,1.704)--(5.936,1.698)--(5.947,1.704)%
  --(5.957,1.710)--(5.967,1.704)--(5.977,1.710)--(5.988,1.717)--(5.998,1.710)--(6.008,1.716)%
  --(6.019,1.723)--(6.029,1.729)--(6.039,1.723)--(6.050,1.729)--(6.060,1.736)--(6.070,1.729)%
  --(6.081,1.736)--(6.091,1.743)--(6.101,1.749)--(6.112,1.743)--(6.122,1.749)--(6.132,1.756)%
  --(6.143,1.750)--(6.153,1.756)--(6.163,1.763)--(6.174,1.770)--(6.184,1.764)--(6.194,1.771)%
  --(6.204,1.778)--(6.215,1.771)--(6.225,1.778)--(6.235,1.785)--(6.246,1.792)--(6.256,1.785)%
  --(6.266,1.793)--(6.277,1.800)--(6.287,1.807)--(6.297,1.800)--(6.308,1.808)--(6.318,1.815)%
  --(6.328,1.808)--(6.339,1.816)--(6.349,1.823)--(6.359,1.830)--(6.370,1.824)--(6.380,1.831)%
  --(6.390,1.839)--(6.401,1.832)--(6.411,1.840)--(6.421,1.847)--(6.431,1.855)--(6.442,1.848)%
  --(6.452,1.856)--(6.462,1.863)--(6.473,1.871)--(6.483,1.865)--(6.493,1.872)--(6.504,1.880)%
  --(6.514,1.873)--(6.524,1.881)--(6.535,1.889)--(6.545,1.897)--(6.555,1.890)--(6.566,1.898)%
  --(6.576,1.906)--(6.586,1.915)--(6.597,1.908)--(6.607,1.916)--(6.617,1.924)--(6.628,1.932)%
  --(6.638,1.926)--(6.648,1.934)--(6.658,1.942)--(6.669,1.935)--(6.679,1.944)--(6.689,1.952)%
  --(6.700,1.960)--(6.710,1.954)--(6.720,1.962)--(6.731,1.971)--(6.741,1.979)--(6.751,1.973)%
  --(6.762,1.981)--(6.772,1.990)--(6.782,1.998)--(6.793,1.992)--(6.803,2.000)--(6.813,2.009)%
  --(6.824,2.018)--(6.834,2.011)--(6.844,2.020)--(6.855,2.029)--(6.865,2.022)--(6.875,2.031)%
  --(6.885,2.040)--(6.896,2.049)--(6.906,2.042)--(6.916,2.051)--(6.927,2.060)--(6.937,2.069)%
  --(6.947,2.063)--(6.958,2.072)--(6.968,2.081)--(6.978,2.090)--(6.989,2.084)--(6.999,2.093)%
  --(7.009,2.102)--(7.020,2.111)--(7.030,2.105)--(7.040,2.114)--(7.051,2.124)--(7.061,2.133)%
  --(7.071,2.126)--(7.082,2.136)--(7.092,2.145)--(7.102,2.155)--(7.112,2.148)--(7.123,2.158)%
  --(7.133,2.168)--(7.143,2.177)--(7.154,2.171)--(7.164,2.181)--(7.174,2.190)--(7.185,2.200)%
  --(7.195,2.194)--(7.205,2.204)--(7.216,2.213)--(7.226,2.223)--(7.236,2.217)--(7.247,2.227)%
  --(7.257,2.237)--(7.267,2.247)--(7.278,2.240)--(7.288,2.251)--(7.298,2.261)--(7.309,2.271)%
  --(7.319,2.264)--(7.329,2.275)--(7.339,2.285)--(7.350,2.295)--(7.360,2.289)--(7.370,2.299)%
  --(7.381,2.310)--(7.391,2.320)--(7.401,2.314)--(7.412,2.324)--(7.422,2.335)--(7.432,2.345)%
  --(7.443,2.339)--(7.453,2.349)--(7.463,2.360)--(7.474,2.371)--(7.484,2.364)--(7.494,2.375)%
  --(7.505,2.386)--(7.515,2.397)--(7.525,2.390)--(7.536,2.401)--(7.546,2.412)--(7.556,2.423)%
  --(7.566,2.417)--(7.577,2.428)--(7.587,2.439)--(7.597,2.450)--(7.608,2.461)--(7.618,2.455)%
  --(7.628,2.466)--(7.639,2.477)--(7.649,2.488)--(7.659,2.482)--(7.670,2.493)--(7.680,2.505)%
  --(7.690,2.516)--(7.701,2.510)--(7.711,2.521)--(7.721,2.533)--(7.732,2.544)--(7.742,2.538)%
  --(7.752,2.550)--(7.763,2.561)--(7.773,2.573)--(7.783,2.567)--(7.793,2.578)--(7.804,2.590)%
  --(7.814,2.602)--(7.824,2.614)--(7.835,2.607)--(7.845,2.619)--(7.855,2.631)--(7.866,2.643)%
  --(7.876,2.637)--(7.886,2.649)--(7.897,2.661)--(7.907,2.673)--(7.917,2.667)--(7.928,2.679)%
  --(7.938,2.691)--(7.948,2.704)--(7.959,2.716)--(7.969,2.710)--(7.979,2.722)--(7.990,2.734)%
  --(8.000,2.747)--(8.010,2.741)--(8.021,2.753)--(8.031,2.766)--(8.041,2.778)--(8.051,2.772)%
  --(8.062,2.785)--(8.072,2.797)--(8.082,2.810)--(8.093,2.823)--(8.103,2.817)--(8.113,2.829)%
  --(8.124,2.842)--(8.134,2.855)--(8.144,2.849)--(8.155,2.862)--(8.165,2.875)--(8.175,2.888)%
  --(8.186,2.901)--(8.196,2.895)--(8.206,2.908)--(8.217,2.921)--(8.227,2.934)--(8.237,2.928)%
  --(8.248,2.941)--(8.258,2.955)--(8.268,2.968)--(8.278,2.962)--(8.289,2.975)--(8.299,2.989)%
  --(8.309,3.002)--(8.320,3.016)--(8.330,3.010)--(8.340,3.023)--(8.351,3.037)--(8.361,3.051)%
  --(8.371,3.045)--(8.382,3.058)--(8.392,3.072)--(8.402,3.086)--(8.413,3.100)--(8.423,3.094)%
  --(8.433,3.108)--(8.444,3.122)--(8.454,3.135)--(8.464,3.130)--(8.475,3.144)--(8.485,3.158)%
  --(8.495,3.172)--(8.505,3.186)--(8.516,3.180)--(8.526,3.194)--(8.536,3.208)--(8.547,3.223)%
  --(8.557,3.217)--(8.567,3.231)--(8.578,3.246)--(8.588,3.260)--(8.598,3.274)--(8.609,3.269)%
  --(8.619,3.283)--(8.629,3.298)--(8.640,3.312)--(8.650,3.327)--(8.660,3.321)--(8.671,3.336)%
  --(8.681,3.351)--(8.691,3.365)--(8.702,3.360)--(8.712,3.375)--(8.722,3.390)--(8.732,3.404)%
  --(8.743,3.419)--(8.753,3.414)--(8.763,3.429)--(8.774,3.444)--(8.784,3.459)--(8.794,3.474)%
  --(8.805,3.469)--(8.815,3.484)--(8.825,3.499)--(8.836,3.514)--(8.846,3.509)--(8.856,3.524)%
  --(8.867,3.540)--(8.877,3.555)--(8.887,3.570)--(8.898,3.565)--(8.908,3.580)--(8.918,3.596)%
  --(8.929,3.612)--(8.939,3.627)--(8.949,3.622)--(8.959,3.638)--(8.970,3.653)--(8.980,3.669)%
  --(8.990,3.664)--(9.001,3.680)--(9.011,3.695)--(9.021,3.711)--(9.032,3.727)--(9.042,3.722)%
  --(9.052,3.738)--(9.063,3.754)--(9.073,3.770)--(9.083,3.786)--(9.094,3.781)--(9.104,3.797)%
  --(9.114,3.814)--(9.125,3.830)--(9.135,3.846)--(9.145,3.841)--(9.156,3.857)--(9.166,3.874)%
  --(9.176,3.890)--(9.186,3.885)--(9.197,3.902)--(9.207,3.918)--(9.217,3.935)--(9.228,3.952)%
  --(9.238,3.947)--(9.248,3.963)--(9.259,3.980)--(9.269,3.997)--(9.279,4.014)--(9.290,4.009)%
  --(9.300,4.026)--(9.310,4.043)--(9.321,4.060)--(9.331,4.077)--(9.341,4.072)--(9.352,4.089)%
  --(9.362,4.106)--(9.372,4.123)--(9.383,4.140)--(9.393,4.135)--(9.403,4.153)--(9.413,4.170)%
  --(9.424,4.187)--(9.434,4.205)--(9.444,4.200)--(9.455,4.217)--(9.465,4.235)--(9.475,4.253)%
  --(9.486,4.248)--(9.496,4.265)--(9.506,4.283)--(9.517,4.301)--(9.527,4.318)--(9.537,4.314)%
  --(9.548,4.332)--(9.558,4.349)--(9.568,4.367)--(9.579,4.385)--(9.589,4.381)--(9.599,4.399)%
  --(9.610,4.417)--(9.620,4.435)--(9.630,4.453)--(9.640,4.448)--(9.651,4.467)--(9.661,4.485)%
  --(9.671,4.503)--(9.682,4.521)--(9.692,4.517)--(9.702,4.535)--(9.713,4.554)--(9.723,4.572)%
  --(9.733,4.591)--(9.744,4.586)--(9.754,4.605)--(9.764,4.623)--(9.775,4.642)--(9.785,4.661)%
  --(9.795,4.656)--(9.806,4.675)--(9.816,4.694)--(9.826,4.713)--(9.837,4.732)--(9.847,4.727)%
  --(9.857,4.746)--(9.867,4.765)--(9.878,4.784)--(9.888,4.803)--(9.898,4.799)--(9.909,4.818)%
  --(9.919,4.838)--(9.929,4.857)--(9.940,4.876)--(9.950,4.872)--(9.960,4.891)--(9.971,4.911)%
  --(9.981,4.930)--(9.991,4.949)--(10.002,4.945)--(10.012,4.965)--(10.022,4.984)--(10.033,5.004)%
  --(10.043,5.024)--(10.053,5.020)--(10.064,5.039)--(10.074,5.059)--(10.084,5.079)--(10.094,5.099)%
  --(10.105,5.095)--(10.115,5.115)--(10.125,5.135)--(10.136,5.155)--(10.146,5.175)--(10.156,5.171)%
  --(10.167,5.191)--(10.177,5.211)--(10.187,5.231)--(10.198,5.252)--(10.208,5.248)--(10.218,5.268)%
  --(10.229,5.289)--(10.239,5.309)--(10.249,5.329)--(10.260,5.326)--(10.270,5.346)--(10.280,5.367)%
  --(10.291,5.387)--(10.301,5.408)--(10.311,5.404)--(10.321,5.425)--(10.332,5.446)--(10.342,5.467)%
  --(10.352,5.487)--(10.363,5.484)--(10.373,5.505)--(10.383,5.526)--(10.394,5.547)--(10.404,5.568)%
  --(10.414,5.564)--(10.425,5.585)--(10.435,5.606)--(10.445,5.628)--(10.456,5.649)--(10.466,5.645)%
  --(10.476,5.667)--(10.487,5.688)--(10.497,5.709)--(10.507,5.731)--(10.518,5.727)--(10.528,5.749)%
  --(10.538,5.770)--(10.548,5.792)--(10.559,5.814)--(10.569,5.835)--(10.579,5.832)--(10.590,5.854)%
  --(10.600,5.876)--(10.610,5.897)--(10.621,5.919)--(10.631,5.916)--(10.641,5.938)--(10.652,5.960)%
  --(10.662,5.982)--(10.672,6.004)--(10.683,6.001)--(10.693,6.023)--(10.703,6.045)--(10.714,6.067)%
  --(10.724,6.090)--(10.734,6.087)--(10.745,6.109)--(10.755,6.131)--(10.765,6.154)--(10.775,6.176)%
  --(10.786,6.173)--(10.796,6.196)--(10.806,6.218)--(10.817,6.241)--(10.827,6.264)--(10.837,6.261)%
  --(10.848,6.284)--(10.858,6.306)--(10.868,6.329)--(10.879,6.352)--(10.889,6.349)--(10.899,6.372)%
  --(10.910,6.395)--(10.920,6.418)--(10.930,6.441)--(10.941,6.438)--(10.951,6.462)--(10.961,6.485)%
  --(10.972,6.508)--(10.982,6.531)--(10.992,6.529)--(11.002,6.552)--(11.013,6.575)--(11.023,6.599)%
  --(11.033,6.622)--(11.044,6.646)--(11.054,6.643)--(11.064,6.667)--(11.075,6.690)--(11.085,6.714)%
  --(11.095,6.738)--(11.106,6.735)--(11.116,6.759)--(11.126,6.783)--(11.137,6.807)--(11.147,6.831)%
  --(11.157,6.828)--(11.168,6.852)--(11.178,6.876)--(11.188,6.900)--(11.199,6.924)--(11.209,6.922)%
  --(11.219,6.946)--(11.229,6.971)--(11.240,6.995)--(11.250,7.019)--(11.260,7.017)--(11.271,7.041)%
  --(11.281,7.066)--(11.291,7.090)--(11.302,7.115)--(11.312,7.113)--(11.322,7.137)--(11.333,7.162)%
  --(11.343,7.187)--(11.353,7.211)--(11.364,7.209)--(11.374,7.234)--(11.384,7.259)--(11.395,7.284)%
  --(11.405,7.309)--(11.415,7.334)--(11.426,7.332)--(11.436,7.357)--(11.446,7.382)--(11.456,7.407)%
  --(11.467,7.432)--(11.477,7.430)--(11.487,7.455)--(11.498,7.481)--(11.508,7.506)--(11.518,7.532)%
  --(11.529,7.530)--(11.539,7.555)--(11.549,7.581)--(11.560,7.606)--(11.570,7.632)--(11.580,7.630)%
  --(11.591,7.656)--(11.601,7.681)--(11.611,7.707)--(11.622,7.733)--(11.632,7.731)--(11.642,7.757)%
  --(11.653,7.783)--(11.663,7.809)--(11.673,7.835)--(11.683,7.834)--(11.694,7.860)--(11.704,7.886)%
  --(11.714,7.912)--(11.725,7.938)--(11.735,7.937)--(11.745,7.963)--(11.756,7.989)--(11.766,8.016)%
  --(11.776,8.042)--(11.787,8.041)--(11.797,8.067)--(11.807,8.094)--(11.818,8.120)--(11.828,8.147)%
  --(11.838,8.146)--(11.849,8.172)--(11.859,8.199)--(11.869,8.226)--(11.880,8.253)--(11.890,8.251)%
  --(11.900,8.278)--(11.910,8.305)--(11.921,8.332)--(11.931,8.359)--(11.941,8.386);
\gpcolor{color=gp lt color border}
\draw[gp path] (1.320,8.631)--(1.320,0.985)--(13.447,0.985)--(13.447,8.631)--cycle;
%% coordinates of the plot area
\gpdefrectangularnode{gp plot 1}{\pgfpoint{1.320cm}{0.985cm}}{\pgfpoint{13.447cm}{8.631cm}}
\end{tikzpicture}
%% gnuplot variables

	\caption{Gráfico da pressão. \protect[Parameters: NJL $\rm{D}_1$, $m_0 = \np[MeV]{5.6}$]}
	\label{Fig:pressure_NJL-Buballa_Set_1}
\end{figure*}

\begin{figure*}
	\begin{tikzpicture}[gnuplot]
%% generated with GNUPLOT 5.0p2 (Lua 5.2; terminal rev. 99, script rev. 100)
%% Mon Apr 11 17:50:14 2016
\path (0.000,0.000) rectangle (14.000,9.000);
\gpcolor{color=gp lt color border}
\gpsetlinetype{gp lt border}
\gpsetdashtype{gp dt solid}
\gpsetlinewidth{1.00}
\draw[gp path] (1.320,0.985)--(1.500,0.985);
\draw[gp path] (13.447,0.985)--(13.267,0.985);
\node[gp node right] at (1.136,0.985) {$955$};
\draw[gp path] (1.320,2.077)--(1.500,2.077);
\draw[gp path] (13.447,2.077)--(13.267,2.077);
\node[gp node right] at (1.136,2.077) {$960$};
\draw[gp path] (1.320,3.170)--(1.500,3.170);
\draw[gp path] (13.447,3.170)--(13.267,3.170);
\node[gp node right] at (1.136,3.170) {$965$};
\draw[gp path] (1.320,4.262)--(1.500,4.262);
\draw[gp path] (13.447,4.262)--(13.267,4.262);
\node[gp node right] at (1.136,4.262) {$970$};
\draw[gp path] (1.320,5.354)--(1.500,5.354);
\draw[gp path] (13.447,5.354)--(13.267,5.354);
\node[gp node right] at (1.136,5.354) {$975$};
\draw[gp path] (1.320,6.446)--(1.500,6.446);
\draw[gp path] (13.447,6.446)--(13.267,6.446);
\node[gp node right] at (1.136,6.446) {$980$};
\draw[gp path] (1.320,7.539)--(1.500,7.539);
\draw[gp path] (13.447,7.539)--(13.267,7.539);
\node[gp node right] at (1.136,7.539) {$985$};
\draw[gp path] (1.320,8.631)--(1.500,8.631);
\draw[gp path] (13.447,8.631)--(13.267,8.631);
\node[gp node right] at (1.136,8.631) {$990$};
\draw[gp path] (1.320,0.985)--(1.320,1.165);
\draw[gp path] (1.320,8.631)--(1.320,8.451);
\node[gp node center] at (1.320,0.677) {$0$};
\draw[gp path] (3.341,0.985)--(3.341,1.165);
\draw[gp path] (3.341,8.631)--(3.341,8.451);
\node[gp node center] at (3.341,0.677) {$0.05$};
\draw[gp path] (5.362,0.985)--(5.362,1.165);
\draw[gp path] (5.362,8.631)--(5.362,8.451);
\node[gp node center] at (5.362,0.677) {$0.1$};
\draw[gp path] (7.384,0.985)--(7.384,1.165);
\draw[gp path] (7.384,8.631)--(7.384,8.451);
\node[gp node center] at (7.384,0.677) {$0.15$};
\draw[gp path] (9.405,0.985)--(9.405,1.165);
\draw[gp path] (9.405,8.631)--(9.405,8.451);
\node[gp node center] at (9.405,0.677) {$0.2$};
\draw[gp path] (11.426,0.985)--(11.426,1.165);
\draw[gp path] (11.426,8.631)--(11.426,8.451);
\node[gp node center] at (11.426,0.677) {$0.25$};
\draw[gp path] (13.447,0.985)--(13.447,1.165);
\draw[gp path] (13.447,8.631)--(13.447,8.451);
\node[gp node center] at (13.447,0.677) {$0.3$};
\draw[gp path] (1.320,8.631)--(1.320,0.985)--(13.447,0.985)--(13.447,8.631)--cycle;
\node[gp node center,rotate=-270] at (0.246,4.808) {$\varepsilon/\rho$ (MeV)};
\node[gp node center] at (7.383,0.215) {$\rho$ $\rm{fm}^{-3}$};
\gpcolor{rgb color={0.580,0.000,0.827}}
\draw[gp path] (1.734,1.175)--(1.745,1.234)--(1.755,1.292)--(1.765,1.349)--(1.775,1.406)%
  --(1.786,1.461)--(1.796,1.516)--(1.806,1.570)--(1.816,1.623)--(1.826,1.676)--(1.837,1.727)%
  --(1.847,1.779)--(1.857,1.829)--(1.867,1.879)--(1.877,1.928)--(1.888,1.977)--(1.898,2.025)%
  --(1.908,2.072)--(1.918,2.119)--(1.929,2.166)--(1.939,2.211)--(1.949,2.257)--(1.959,2.301)%
  --(1.969,2.346)--(1.980,2.389)--(1.990,2.433)--(2.000,2.476)--(2.010,2.518)--(2.021,2.560)%
  --(2.031,2.601)--(2.041,2.642)--(2.051,2.683)--(2.061,2.723)--(2.072,2.763)--(2.082,2.802)%
  --(2.092,2.841)--(2.102,2.880)--(2.112,2.918)--(2.123,2.956)--(2.133,2.993)--(2.143,3.030)%
  --(2.153,3.067)--(2.164,3.103)--(2.174,3.139)--(2.184,3.175)--(2.194,3.210)--(2.204,3.245)%
  --(2.215,3.280)--(2.225,3.314)--(2.235,3.348)--(2.245,3.382)--(2.256,3.416)--(2.266,3.449)%
  --(2.276,3.482)--(2.286,3.514)--(2.296,3.547)--(2.307,3.579)--(2.317,3.610)--(2.327,3.642)%
  --(2.337,3.673)--(2.347,3.704)--(2.358,3.735)--(2.368,3.765)--(2.378,3.795)--(2.388,3.825)%
  --(2.399,3.855)--(2.409,3.884)--(2.419,3.913)--(2.429,3.942)--(2.439,3.971)--(2.450,4.000)%
  --(2.460,4.028)--(2.470,4.056)--(2.480,4.084)--(2.491,4.111)--(2.501,4.139)--(2.511,4.166)%
  --(2.521,4.193)--(2.531,4.219)--(2.542,4.246)--(2.552,4.272)--(2.562,4.298)--(2.572,4.324)%
  --(2.582,4.350)--(2.593,4.376)--(2.603,4.401)--(2.613,4.426)--(2.623,4.451)--(2.634,4.476)%
  --(2.644,4.500)--(2.654,4.525)--(2.664,4.549)--(2.674,4.573)--(2.685,4.597)--(2.695,4.621)%
  --(2.705,4.644)--(2.715,4.667)--(2.726,4.691)--(2.736,4.714)--(2.746,4.736)--(2.756,4.759)%
  --(2.766,4.782)--(2.777,4.804)--(2.787,4.826)--(2.797,4.848)--(2.807,4.870)--(2.817,4.892)%
  --(2.828,4.913)--(2.838,4.935)--(2.848,4.956)--(2.858,4.977)--(2.869,4.998)--(2.879,5.019)%
  --(2.889,5.040)--(2.899,5.060)--(2.909,5.081)--(2.920,5.101)--(2.930,5.121)--(2.940,5.141)%
  --(2.950,5.161)--(2.961,5.181)--(2.971,5.200)--(2.981,5.220)--(2.991,5.239)--(3.001,5.258)%
  --(3.012,5.277)--(3.022,5.296)--(3.032,5.315)--(3.042,5.333)--(3.052,5.352)--(3.063,5.370)%
  --(3.073,5.389)--(3.083,5.407)--(3.093,5.425)--(3.104,5.443)--(3.114,5.461)--(3.124,5.478)%
  --(3.134,5.496)--(3.144,5.513)--(3.155,5.531)--(3.165,5.548)--(3.175,5.565)--(3.185,5.582)%
  --(3.195,5.599)--(3.206,5.615)--(3.216,5.632)--(3.226,5.649)--(3.236,5.665)--(3.247,5.681)%
  --(3.257,5.698)--(3.267,5.714)--(3.277,5.730)--(3.287,5.746)--(3.298,5.762)--(3.308,5.777)%
  --(3.318,5.793)--(3.328,5.808)--(3.339,5.824)--(3.349,5.839)--(3.359,5.854)--(3.369,5.869)%
  --(3.379,5.884)--(3.390,5.899)--(3.400,5.914)--(3.410,5.929)--(3.420,5.943)--(3.430,5.958)%
  --(3.441,5.972)--(3.451,5.987)--(3.461,6.001)--(3.471,6.015)--(3.482,6.029)--(3.492,6.043)%
  --(3.502,6.057)--(3.512,6.071)--(3.522,6.084)--(3.533,6.098)--(3.543,6.111)--(3.553,6.125)%
  --(3.563,6.138)--(3.574,6.151)--(3.584,6.165)--(3.594,6.178)--(3.604,6.191)--(3.614,6.204)%
  --(3.625,6.216)--(3.635,6.229)--(3.645,6.242)--(3.655,6.255)--(3.665,6.267)--(3.676,6.279)%
  --(3.686,6.292)--(3.696,6.304)--(3.706,6.316)--(3.717,6.328)--(3.727,6.340)--(3.737,6.352)%
  --(3.747,6.364)--(3.757,6.376)--(3.768,6.388)--(3.778,6.399)--(3.788,6.411)--(3.798,6.423)%
  --(3.809,6.434)--(3.819,6.445)--(3.829,6.457)--(3.839,6.468)--(3.849,6.479)--(3.860,6.490)%
  --(3.870,6.501)--(3.880,6.512)--(3.890,6.523)--(3.900,6.533)--(3.911,6.544)--(3.921,6.555)%
  --(3.931,6.565)--(3.941,6.576)--(3.952,6.586)--(3.962,6.597)--(3.972,6.607)--(3.982,6.617)%
  --(3.992,6.627)--(4.003,6.637)--(4.013,6.647)--(4.023,6.657)--(4.033,6.667)--(4.044,6.677)%
  --(4.054,6.687)--(4.064,6.697)--(4.074,6.706)--(4.084,6.716)--(4.095,6.725)--(4.105,6.735)%
  --(4.115,6.744)--(4.125,6.753)--(4.135,6.763)--(4.146,6.772)--(4.156,6.781)--(4.166,6.790)%
  --(4.176,6.799)--(4.187,6.808)--(4.197,6.817)--(4.207,6.826)--(4.217,6.835)--(4.227,6.843)%
  --(4.238,6.852)--(4.248,6.861)--(4.258,6.869)--(4.268,6.878)--(4.279,6.886)--(4.289,6.894)%
  --(4.299,6.903)--(4.309,6.911)--(4.319,6.919)--(4.330,6.927)--(4.340,6.935)--(4.350,6.943)%
  --(4.360,6.951)--(4.370,6.959)--(4.381,6.967)--(4.391,6.975)--(4.401,6.983)--(4.411,6.990)%
  --(4.422,6.998)--(4.432,7.006)--(4.442,7.013)--(4.452,7.021)--(4.462,7.028)--(4.473,7.035)%
  --(4.483,7.043)--(4.493,7.050)--(4.503,7.057)--(4.514,7.064)--(4.524,7.071)--(4.534,7.079)%
  --(4.544,7.086)--(4.554,7.093)--(4.565,7.099)--(4.575,7.106)--(4.585,7.113)--(4.595,7.120)%
  --(4.605,7.127)--(4.616,7.133)--(4.626,7.140)--(4.636,7.146)--(4.646,7.153)--(4.657,7.159)%
  --(4.667,7.166)--(4.677,7.172)--(4.687,7.179)--(4.697,7.185)--(4.708,7.191)--(4.718,7.197)%
  --(4.728,7.203)--(4.738,7.209)--(4.748,7.216)--(4.759,7.222)--(4.769,7.227)--(4.779,7.233)%
  --(4.789,7.239)--(4.800,7.245)--(4.810,7.251)--(4.820,7.257)--(4.830,7.262)--(4.840,7.268)%
  --(4.851,7.273)--(4.861,7.279)--(4.871,7.284)--(4.881,7.290)--(4.892,7.295)--(4.902,7.301)%
  --(4.912,7.306)--(4.922,7.311)--(4.932,7.317)--(4.943,7.322)--(4.953,7.327)--(4.963,7.332)%
  --(4.973,7.337)--(4.983,7.342)--(4.994,7.347)--(5.004,7.352)--(5.014,7.357)--(5.024,7.362)%
  --(5.035,7.367)--(5.045,7.371)--(5.055,7.376)--(5.065,7.381)--(5.075,7.386)--(5.086,7.390)%
  --(5.096,7.395)--(5.106,7.399)--(5.116,7.404)--(5.127,7.408)--(5.137,7.413)--(5.147,7.417)%
  --(5.157,7.421)--(5.167,7.426)--(5.178,7.430)--(5.188,7.434)--(5.198,7.438)--(5.208,7.442)%
  --(5.218,7.447)--(5.229,7.451)--(5.239,7.455)--(5.249,7.459)--(5.259,7.463)--(5.270,7.467)%
  --(5.280,7.470)--(5.290,7.474)--(5.300,7.478)--(5.310,7.482)--(5.321,7.486)--(5.331,7.489)%
  --(5.341,7.493)--(5.351,7.497)--(5.362,7.500)--(5.372,7.504)--(5.382,7.507)--(5.392,7.511)%
  --(5.402,7.514)--(5.413,7.517)--(5.423,7.521)--(5.433,7.524)--(5.443,7.527)--(5.453,7.531)%
  --(5.464,7.534)--(5.474,7.537)--(5.484,7.540)--(5.494,7.543)--(5.505,7.547)--(5.515,7.550)%
  --(5.525,7.553)--(5.535,7.556)--(5.545,7.559)--(5.556,7.561)--(5.566,7.564)--(5.576,7.567)%
  --(5.586,7.570)--(5.597,7.573)--(5.607,7.576)--(5.617,7.578)--(5.627,7.581)--(5.637,7.584)%
  --(5.648,7.586)--(5.658,7.589)--(5.668,7.591)--(5.678,7.594)--(5.688,7.596)--(5.699,7.599)%
  --(5.709,7.601)--(5.719,7.604)--(5.729,7.606)--(5.740,7.608)--(5.750,7.611)--(5.760,7.613)%
  --(5.770,7.615)--(5.780,7.617)--(5.791,7.619)--(5.801,7.622)--(5.811,7.624)--(5.821,7.626)%
  --(5.832,7.628)--(5.842,7.630)--(5.852,7.632)--(5.862,7.634)--(5.872,7.636)--(5.883,7.638)%
  --(5.893,7.639)--(5.903,7.641)--(5.913,7.643)--(5.923,7.645)--(5.934,7.647)--(5.944,7.648)%
  --(5.954,7.650)--(5.964,7.652)--(5.975,7.653)--(5.985,7.655)--(5.995,7.656)--(6.005,7.658)%
  --(6.015,7.659)--(6.026,7.661)--(6.036,7.662)--(6.046,7.664)--(6.056,7.665)--(6.067,7.667)%
  --(6.077,7.668)--(6.087,7.669)--(6.097,7.670)--(6.107,7.672)--(6.118,7.673)--(6.128,7.674)%
  --(6.138,7.675)--(6.148,7.676)--(6.158,7.678)--(6.169,7.679)--(6.179,7.680)--(6.189,7.681)%
  --(6.199,7.682)--(6.210,7.683)--(6.220,7.684)--(6.230,7.685)--(6.240,7.685)--(6.250,7.686)%
  --(6.261,7.687)--(6.271,7.688)--(6.281,7.689)--(6.291,7.689)--(6.301,7.690)--(6.312,7.691)%
  --(6.322,7.692)--(6.332,7.692)--(6.342,7.693)--(6.353,7.693)--(6.363,7.694)--(6.373,7.695)%
  --(6.383,7.695)--(6.393,7.696)--(6.404,7.696)--(6.414,7.696)--(6.424,7.697)--(6.434,7.697)%
  --(6.445,7.698)--(6.455,7.698)--(6.465,7.698)--(6.475,7.699)--(6.485,7.699)--(6.496,7.699)%
  --(6.506,7.699)--(6.516,7.699)--(6.526,7.700)--(6.536,7.700)--(6.547,7.700)--(6.557,7.700)%
  --(6.567,7.700)--(6.577,7.700)--(6.588,7.700)--(6.598,7.700)--(6.608,7.700)--(6.618,7.700)%
  --(6.628,7.700)--(6.639,7.700)--(6.649,7.700)--(6.659,7.700)--(6.669,7.699)--(6.680,7.699)%
  --(6.690,7.699)--(6.700,7.699)--(6.710,7.698)--(6.720,7.698)--(6.731,7.698)--(6.741,7.697)%
  --(6.751,7.697)--(6.761,7.697)--(6.771,7.696)--(6.782,7.696)--(6.792,7.695)--(6.802,7.695)%
  --(6.812,7.694)--(6.823,7.694)--(6.833,7.693)--(6.843,7.693)--(6.853,7.692)--(6.863,7.692)%
  --(6.874,7.691)--(6.884,7.690)--(6.894,7.690)--(6.904,7.689)--(6.915,7.688)--(6.925,7.687)%
  --(6.935,7.687)--(6.945,7.686)--(6.955,7.685)--(6.966,7.684)--(6.976,7.683)--(6.986,7.682)%
  --(6.996,7.682)--(7.006,7.681)--(7.017,7.680)--(7.027,7.679)--(7.037,7.678)--(7.047,7.677)%
  --(7.058,7.676)--(7.068,7.675)--(7.078,7.673)--(7.088,7.672)--(7.098,7.671)--(7.109,7.670)%
  --(7.119,7.669)--(7.129,7.668)--(7.139,7.667)--(7.150,7.665)--(7.160,7.664)--(7.170,7.663)%
  --(7.180,7.662)--(7.190,7.660)--(7.201,7.659)--(7.211,7.658)--(7.221,7.656)--(7.231,7.655)%
  --(7.241,7.653)--(7.252,7.652)--(7.262,7.651)--(7.272,7.649)--(7.282,7.648)--(7.293,7.646)%
  --(7.303,7.644)--(7.313,7.643)--(7.323,7.641)--(7.333,7.640)--(7.344,7.638)--(7.354,7.637)%
  --(7.364,7.635)--(7.374,7.633)--(7.385,7.632)--(7.395,7.630)--(7.405,7.628)--(7.415,7.626)%
  --(7.425,7.625)--(7.436,7.623)--(7.446,7.621)--(7.456,7.619)--(7.466,7.617)--(7.476,7.615)%
  --(7.487,7.614)--(7.497,7.612)--(7.507,7.610)--(7.517,7.608)--(7.528,7.606)--(7.538,7.604)%
  --(7.548,7.602)--(7.558,7.600)--(7.568,7.598)--(7.579,7.596)--(7.589,7.594)--(7.599,7.591)%
  --(7.609,7.589)--(7.620,7.587)--(7.630,7.585)--(7.640,7.583)--(7.650,7.581)--(7.660,7.579)%
  --(7.671,7.576)--(7.681,7.574)--(7.691,7.572)--(7.701,7.570)--(7.711,7.567)--(7.722,7.565)%
  --(7.732,7.563)--(7.742,7.560)--(7.752,7.558)--(7.763,7.555)--(7.773,7.553)--(7.783,7.551)%
  --(7.793,7.548)--(7.803,7.546)--(7.814,7.543)--(7.824,7.541)--(7.834,7.538)--(7.844,7.536)%
  --(7.854,7.533)--(7.865,7.531)--(7.875,7.528)--(7.885,7.525)--(7.895,7.523)--(7.906,7.520)%
  --(7.916,7.518)--(7.926,7.515)--(7.936,7.512)--(7.946,7.509)--(7.957,7.507)--(7.967,7.504)%
  --(7.977,7.501)--(7.987,7.498)--(7.998,7.496)--(8.008,7.493)--(8.018,7.490)--(8.028,7.487)%
  --(8.038,7.484)--(8.049,7.481)--(8.059,7.479)--(8.069,7.476)--(8.079,7.473)--(8.089,7.470)%
  --(8.100,7.467)--(8.110,7.464)--(8.120,7.461)--(8.130,7.458)--(8.141,7.455)--(8.151,7.452)%
  --(8.161,7.449)--(8.171,7.446)--(8.181,7.443)--(8.192,7.439)--(8.202,7.436)--(8.212,7.433)%
  --(8.222,7.430)--(8.233,7.427)--(8.243,7.424)--(8.253,7.420)--(8.263,7.417)--(8.273,7.414)%
  --(8.284,7.411)--(8.294,7.408)--(8.304,7.404)--(8.314,7.401)--(8.324,7.398)--(8.335,7.394)%
  --(8.345,7.391)--(8.355,7.388)--(8.365,7.384)--(8.376,7.381)--(8.386,7.377)--(8.396,7.374)%
  --(8.406,7.371)--(8.416,7.367)--(8.427,7.364)--(8.437,7.360)--(8.447,7.357)--(8.457,7.353)%
  --(8.468,7.350)--(8.478,7.346)--(8.488,7.342)--(8.498,7.339)--(8.508,7.335)--(8.519,7.332)%
  --(8.529,7.328)--(8.539,7.324)--(8.549,7.321)--(8.559,7.317)--(8.570,7.313)--(8.580,7.310)%
  --(8.590,7.306)--(8.600,7.302)--(8.611,7.299)--(8.621,7.295)--(8.631,7.291)--(8.641,7.287)%
  --(8.651,7.283)--(8.662,7.280)--(8.672,7.276)--(8.682,7.272)--(8.692,7.268)--(8.703,7.264)%
  --(8.713,7.260)--(8.723,7.256)--(8.733,7.252)--(8.743,7.249)--(8.754,7.245)--(8.764,7.241)%
  --(8.774,7.237)--(8.784,7.233)--(8.794,7.229)--(8.805,7.225)--(8.815,7.221)--(8.825,7.217)%
  --(8.835,7.212)--(8.846,7.208)--(8.856,7.204)--(8.866,7.200)--(8.876,7.196)--(8.886,7.192)%
  --(8.897,7.188)--(8.907,7.184)--(8.917,7.179)--(8.927,7.175)--(8.938,7.171)--(8.948,7.167)%
  --(8.958,7.163)--(8.968,7.158)--(8.978,7.154)--(8.989,7.150)--(8.999,7.146)--(9.009,7.141)%
  --(9.019,7.137)--(9.029,7.133)--(9.040,7.128)--(9.050,7.124)--(9.060,7.120)--(9.070,7.115)%
  --(9.081,7.111)--(9.091,7.106)--(9.101,7.102)--(9.111,7.098)--(9.121,7.093)--(9.132,7.089)%
  --(9.142,7.084)--(9.152,7.080)--(9.162,7.075)--(9.173,7.071)--(9.183,7.066)--(9.193,7.062)%
  --(9.203,7.057)--(9.213,7.052)--(9.224,7.048)--(9.234,7.043)--(9.244,7.039)--(9.254,7.034)%
  --(9.264,7.029)--(9.275,7.025)--(9.285,7.020)--(9.295,7.015)--(9.305,7.011)--(9.316,7.006)%
  --(9.326,7.001)--(9.336,6.997)--(9.346,6.992)--(9.356,6.987)--(9.367,6.982)--(9.377,6.978)%
  --(9.387,6.973)--(9.397,6.968)--(9.407,6.963)--(9.418,6.958)--(9.428,6.954)--(9.438,6.949)%
  --(9.448,6.944)--(9.459,6.939)--(9.469,6.934)--(9.479,6.929)--(9.489,6.924)--(9.499,6.919)%
  --(9.510,6.914)--(9.520,6.909)--(9.530,6.905)--(9.540,6.900)--(9.551,6.895)--(9.561,6.890)%
  --(9.571,6.885)--(9.581,6.880)--(9.591,6.874)--(9.602,6.869)--(9.612,6.864)--(9.622,6.859)%
  --(9.632,6.854)--(9.642,6.849)--(9.653,6.844)--(9.663,6.839)--(9.673,6.834)--(9.683,6.829)%
  --(9.694,6.823)--(9.704,6.818)--(9.714,6.813)--(9.724,6.808)--(9.734,6.803)--(9.745,6.797)%
  --(9.755,6.792)--(9.765,6.787)--(9.775,6.782)--(9.786,6.776)--(9.796,6.771)--(9.806,6.766)%
  --(9.816,6.761)--(9.826,6.755)--(9.837,6.750)--(9.847,6.745)--(9.857,6.739)--(9.867,6.734)%
  --(9.877,6.729)--(9.888,6.723)--(9.898,6.718)--(9.908,6.712)--(9.918,6.707)--(9.929,6.702)%
  --(9.939,6.696)--(9.949,6.691)--(9.959,6.685)--(9.969,6.680)--(9.980,6.674)--(9.990,6.669)%
  --(10.000,6.663)--(10.010,6.658)--(10.021,6.652)--(10.031,6.647)--(10.041,6.641)--(10.051,6.636)%
  --(10.061,6.630)--(10.072,6.624)--(10.082,6.619)--(10.092,6.613)--(10.102,6.608)--(10.112,6.602)%
  --(10.123,6.596)--(10.133,6.591)--(10.143,6.585)--(10.153,6.579)--(10.164,6.574)--(10.174,6.568)%
  --(10.184,6.562)--(10.194,6.557)--(10.204,6.551)--(10.215,6.545)--(10.225,6.539)--(10.235,6.534)%
  --(10.245,6.528)--(10.256,6.522)--(10.266,6.516)--(10.276,6.510)--(10.286,6.505)--(10.296,6.499)%
  --(10.307,6.493)--(10.317,6.487)--(10.327,6.481)--(10.337,6.475)--(10.347,6.469)--(10.358,6.464)%
  --(10.368,6.458)--(10.378,6.452)--(10.388,6.446)--(10.399,6.440)--(10.409,6.434)--(10.419,6.428)%
  --(10.429,6.422)--(10.439,6.416)--(10.450,6.410)--(10.460,6.404)--(10.470,6.398)--(10.480,6.392)%
  --(10.491,6.386)--(10.501,6.380)--(10.511,6.374)--(10.521,6.368)--(10.531,6.362)--(10.542,6.356)%
  --(10.552,6.350)--(10.562,6.344)--(10.572,6.337)--(10.582,6.331)--(10.593,6.325)--(10.603,6.319)%
  --(10.613,6.313)--(10.623,6.307)--(10.634,6.301)--(10.644,6.294)--(10.654,6.288)--(10.664,6.282)%
  --(10.674,6.276)--(10.685,6.270)--(10.695,6.263)--(10.705,6.257)--(10.715,6.251)--(10.726,6.245)%
  --(10.736,6.238)--(10.746,6.232)--(10.756,6.226)--(10.766,6.219)--(10.777,6.213)--(10.787,6.207)%
  --(10.797,6.200)--(10.807,6.194)--(10.817,6.188)--(10.828,6.181)--(10.838,6.175)--(10.848,6.169)%
  --(10.858,6.162)--(10.869,6.156)--(10.879,6.149)--(10.889,6.143)--(10.899,6.137)--(10.909,6.130)%
  --(10.920,6.124)--(10.930,6.117)--(10.940,6.111)--(10.950,6.104)--(10.960,6.098)--(10.971,6.091)%
  --(10.981,6.085)--(10.991,6.078)--(11.001,6.072)--(11.012,6.065)--(11.022,6.059)--(11.032,6.052)%
  --(11.042,6.046)--(11.052,6.039)--(11.063,6.032)--(11.073,6.026)--(11.083,6.019)--(11.093,6.013)%
  --(11.104,6.006)--(11.114,5.999)--(11.124,5.993)--(11.134,5.986)--(11.144,5.979)--(11.155,5.973)%
  --(11.165,5.966)--(11.175,5.959)--(11.185,5.953)--(11.195,5.946)--(11.206,5.939)--(11.216,5.933)%
  --(11.226,5.926)--(11.236,5.919)--(11.247,5.912)--(11.257,5.906)--(11.267,5.899)--(11.277,5.892)%
  --(11.287,5.885)--(11.298,5.878)--(11.308,5.872)--(11.318,5.865)--(11.328,5.858)--(11.339,5.851)%
  --(11.349,5.844)--(11.359,5.837)--(11.369,5.831)--(11.379,5.824)--(11.390,5.817)--(11.400,5.810)%
  --(11.410,5.803)--(11.420,5.796)--(11.430,5.789)--(11.441,5.782)--(11.451,5.775)--(11.461,5.769)%
  --(11.471,5.762)--(11.482,5.755)--(11.492,5.748)--(11.502,5.741)--(11.512,5.734)--(11.522,5.727)%
  --(11.533,5.720)--(11.543,5.713)--(11.553,5.706)--(11.563,5.699)--(11.574,5.692)--(11.584,5.685)%
  --(11.594,5.677)--(11.604,5.670)--(11.614,5.663)--(11.625,5.656)--(11.635,5.649)--(11.645,5.642)%
  --(11.655,5.635)--(11.665,5.628)--(11.676,5.621)--(11.686,5.614)--(11.696,5.606)--(11.706,5.599)%
  --(11.717,5.592)--(11.727,5.585)--(11.737,5.578)--(11.747,5.571)--(11.757,5.563)--(11.768,5.556)%
  --(11.778,5.549)--(11.788,5.542)--(11.798,5.535)--(11.809,5.527)--(11.819,5.520)--(11.829,5.513)%
  --(11.839,5.506)--(11.849,5.498)--(11.860,5.491)--(11.870,5.484)--(11.880,5.476)--(11.890,5.469)%
  --(11.900,5.462)--(11.911,5.455)--(11.921,5.447)--(11.931,5.440)--(11.941,5.433);
\gpcolor{color=gp lt color border}
\draw[gp path] (1.320,8.631)--(1.320,0.985)--(13.447,0.985)--(13.447,8.631)--cycle;
%% coordinates of the plot area
\gpdefrectangularnode{gp plot 1}{\pgfpoint{1.320cm}{0.985cm}}{\pgfpoint{13.447cm}{8.631cm}}
\end{tikzpicture}
%% gnuplot variables

	\caption{Gráfico da razão entre a densidade de energia $\varepsilon$ e a densidade bariônica $\rho_B$ (energia por partícula). \protect[Parameters: NJL $\rm{D}_1$, $m_0 = \np[MeV]{5.6}$]}
	\label{Fig:energy_density_per_particle_NJL-Buballa_Set_1}
\end{figure*}

\FloatBarrier