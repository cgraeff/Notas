%%%%%%%%%%%%%%%%%%%%%%%%%%%%%%%%%%%%%%%%%%%%%
\chapter{Modelo de Nambu--Jona-Lasinio (NJL)}
%%%%%%%%%%%%%%%%%%%%%%%%%%%%%%%%%%%%%%%%%%%%%

\begin{fullwidth}\it
Discutimos a origem e motivação do modelo NJL, bem como sua expressão e propriedades. Discutimos posteriormente a extensão de tal modelo, bem como as razões para essas modificações.
\end{fullwidth}

%%%%%%%%%%%%%%%%%%%%%%%%%%%%%%%%%%%%%%
\section{Motivação e origem do modelo}
%%%%%%%%%%%%%%%%%%%%%%%%%%%%%%%%%%%%%%

Sobre a origem do modelo\cite{Buballa}:
\begin{quote}
Historically, it goes back to two papers by Nambu and Jona-Lasinio in 1961 [...] to a time when QCD and even quarks were still unknown. In its original version, the NJL model was therefore a model of interacting \emph{nucleons}, and obviously, confinement [\dots] was not an issue. On the other hand, even in the pre-QCD era there were already indications for the existence of a (partially) conserved axial vector current (PCAC), i.e., chiral symmetry. Since (approximate) chiral symmetry implies (almost) massless fermions on the Lagrangian level, the problem was to find a mechanism which explains the large nucleon mass without destroying the symmetry.

It was the pioneering idea of Nambu and Jona-Lasinio that the gap in the Dirac spectrum of the nucleon\footnote{\emph{the gap in the Dirac spectrum of the nucleon}, do que se trata?} can be generated quite analogously to the energy gap of a superconductor in BCS theory, which has been developed a few years earlier. To that end they introduced a Lagrangian for a nucleon field $\psi$ with a point-like, chirally symmetric four-fermion interaction\footnote{\emph{point-like}: acho que isto está ligado ao fato de que o termo que corresponderia a um potencial é simplesmente uma constante vezes uma coisa que depende do $\psi$, não algo que saia de uma outra equação que dependa do $\psi$, como a Eq. de Klein-Gordon no modelo de Walecka para o núcleo que eu trabalhei. \emph{chirally symmetric}: imagino que seja por que não há diferenciação entre as duas quiralidaes (helicidades para uma partícula sem massa). \emph{four-fermion interaction}: não tenho ideia de como é possivel ver isso nos termos.},
\begin{equation}\label{Eq:LagOriginalNJL}
	\mathcal{L} = \bar{\psi}(i\slashed{\partial} - m)\psi + G\{(\bar{\psi}i\gamma_5\vec{\tau}\psi)^2\}.
\end{equation}
\end{quote}
%
Na lagrangiana acima,
\begin{itemize}
	\item $m$ é a massa constituinte do nucleon, que deve ser pequena devido ao fato de que se não fosse, não haveria possibilidade de haver simetria quiral. Simetria quiral implica em massa pequena.
	\item $\vec{\tau}$ é a matriz de Pauli atuando no espaço de isospin.
	\item $G$ é uma constante de acoplamento que possui alguma dimensão (termos diferentes tem dimensões diferentes).
\end{itemize}

Ainda temos\cite{Buballa}
\begin{quote}
[T]he self-energy induced by the interaction generates an effective mass $M$ which can be considerably larger than $m$ and stays large, even when $m$ is taken to zero (``chiral limit''). At the same time, there are light collective nucleon-antinucleon excitations which become massless in the chiral limit: The pion emerges as the Goldstone boson of the spontaneously broken chiral symmetry. In fact, this discovery was an important milestone on the way to the general derivation of the Goldstone theorem in the same year.\footnote{\emph{large collective excitatations}: o que é isso? \emph{Goldstone boson}: o que é isso? \emph{spontaneously broken}: o que é isso?}

After the development of QCD, the NJL model was reinterpreted as a schematic quark model. At that point, of course, the lack of confinement became a problem\footnote{Se não ter confinamento é um problema para os quarks, pq temos que adicionar termos ao modelo para a fase de hadrons para poder descrevê-la adequadamente? Não deveria descrever direito pois foi para isso que o modelo foi criado? Ou adicionamos termos só para descrever adequadamente as propriedades da fase de hádrons (principalmente: qual é o papel do termo proporcional a $G_{sv}$?). Na fase de quarks, não tem importância o confinamento pois estamos a uma densidade muito alta, então eles desconfinam de qualquer forma. Devemos tratar as duas separadamente e depois descobrir onde fica a transição usando condições de Gibbs}, severely limiting the applicability of the model. On the other hand, there are many situations where chiral symmetry is the relevant feature of QCD, confinement being less important. The most prominent example is again the Goldstone nature of the pion. In this aspect the NJL model is superior to the MIT bag, which [\dots] fails to explain the low pion mass.\footnote{Qual é a relação entre ser o boson de Goldstone (?) e ter massa pequena? É por que de alguma forma ele está ligado à simetria quiral?}
\end{quote}


%%%%%%%%%%%%%%%%%%%%%%%%%%%%%%%%%%%
\section{Características do modelo}
%%%%%%%%%%%%%%%%%%%%%%%%%%%%%%%%%%%

Simetria quiral, Extensibilidade;

Por que uma massa não nula implica em uma simetria quiral quebrada: Ver discussão em Ref.\cite{Vogl}
% Vogl, U., and W. Weise. - Progress in Particle and Nuclear Physics 27 (1991): 195-272 - The Nambu and Jona-Lasinio model: its implications for hadrons and nuclei

\begin{description}
	\item[Limite quiral] No limite quiral, a massa constituinte vai a zero:
		\begin{equation}
			m \to 0.
		\end{equation}
\end{description}

%%%%%%%%%%%%%%%%%%%%%%%%%%%%%%%%%%%%%%%%%%%%
\section{Lagrangiana do modelo NJL em SU(2)}
%%%%%%%%%%%%%%%%%%%%%%%%%%%%%%%%%%%%%%%%%%%%

Considerando a lagrangiana do tipo NJL em SU(2) dada por Buballa\cite{Buballa1996}
\begin{equation}
	\mathcal{L} = \bar{\psi}(i\gamma^\mu\partial_\mu - m_0)\psi + G_S[(\bar{\psi}\psi)^2 + (\bar{\psi}i\gamma_5\vec{\tau}\psi)^2] - G_V(\bar{\psi}\gamma^\mu \psi)^2,
\end{equation}
%
onde
\begin{itemize}
	\item As constantes $G_S$ e $G_V$ na referência são dadas em unidades de $\rm{massa}^{-2}$. Estamos usando unidades de $\rm{fm}^2$ para tais constantes, portanto devemos multiplicar os valores contidos na referência por $(\hbar c)^2$ para manter a convenção adotada.
	\item ``Here $\psi$ is a fermion field with $n_f = 2$ flavors and $n_c$ colors. It may be interpreted [\dots] as a quark field ($n_c = 3$).''
	\item ``Apart from the bare mass $m_0$, the Lagrangian [above] is chirally symmetric.''
	\item $(\bar{\psi}\psi)^2$ e $(\bar{\psi}i\gamma_5\vec{\tau}\psi)^2$ são os canais scalar\footnote{scalar-isoscalar} e pseudoscalar-isovector, ambos com acoplamento $G_s$;
	\item $(\bar{\psi}\gamma^\mu \psi)^2$ é o canal vector-isoscalar. 
	\item ``It is known, e.g., from Walecka model [109], that this channel [(vector-isoscalar)] is quite important at non-zero densities.''; ``In principle we allow for further channels [\dots], which, however, do not contribute at mean-field level as long as we have only one common quark chemical potential.''\cite{Buballa}
\end{itemize}

%%%%%%%%%%%%%%%%%%%%%%%%%%%
\subsection{Equação do gap}
%%%%%%%%%%%%%%%%%%%%%%%%%%%

A equação do gap tem a ver com a auto-energia, de alguma forma. Como há interação, existe uma massa efetiva, mas não sei como isso funciona. Não sei como sai da lagrangiana e se chega na expressão para $M$ ($m^*$), não consigo nem ter uma vaga ideia do que está sendo feito nas contas da Débora. Não sei se a adição de um termo vetorial (caso NJL) ou outros termos (eNJL) têm algum efeito na equação do gap.

Pelo que pude ver em Menezes\cite{NJLv}, não há nenhum termo em $G_V$ na equação do gap. Acredito que por alguma razão a massa diminui quando não estamos em uma situação de vácuo, devendo depender da densidade através das funções de onda dos férmions. No entanto, isso deve ter alguma influência nas equações de estado calculadas através da estatística, pois elas vem do potencial termodinâmico e esse vem da hamiltoniana/lagrangiana, onde temos justamente uma dependência em $M$ ($m^*$).

O que determina a equação do gap é a condição $\partial \omega / \partial M = 0$. Discutir aqui sobre auto energia, mas tem que determinar na parte de Termodinâmica.

%%%%%%%%%%%%%%%%%%%%%%%%%%%%%%%%%%%%%
\subsection{Adição do termo vetorial}
%%%%%%%%%%%%%%%%%%%%%%%%%%%%%%%%%%%%%

Em primeiro lugar, por quê colocamos esse termo? Descrever o que muda e o que não muda com a adição desse termo.

%%%%%%%%%%%%%%%%%%%%%%%%%%%%%%%%%%
\section{Lagrangiana NJL em SU(3)}
%%%%%%%%%%%%%%%%%%%%%%%%%%%%%%%%%%

Apresentar e discutir a lagrangiana em SU(3).

%%%%%%%%%%%%%%%%%%%%%%%%%%%
\subsection{Equação do gap}
%%%%%%%%%%%%%%%%%%%%%%%%%%%

%%%%%%%%%%%%%%%%%%%%%%%%%%%%%%%%%%%%%%%%%
\section{Sucessos e insucessos do modelo}
%%%%%%%%%%%%%%%%%%%%%%%%%%%%%%%%%%%%%%%%%

Só sei que não descreve a saturação da matéria de nucleons. Discutir tanto para SU(2) como para SU(3).

%%%%%%%%%%%%%%%%%%%%%%%%%%%%%%%%%%%%%%%%%%%%%%
\section{Fase de hádrons: extended NJL (eNJL)}
%%%%%%%%%%%%%%%%%%%%%%%%%%%%%%%%%%%%%%%%%%%%%%

O modelo NJL não é capaz de descrever a energia de saturação para a matéria de nucleons, e por isso não podemos utilizar tal modelo para a fase de hádrons. No entanto, com a adição de termos adequados, é possível obter resultados adequados. Discutiremos tais termos nas seções subsequentes.

%%%%%%%%%%%%%%%%%%%%%%%%%%%%%%%%%%%%%%%%%%%%%%%%%%%%%%%%%%%%%%%%%%%%%%%%%
\subsection{Introdução de outros termos à lagrangiana original}
\label{Sec:Introducao_termos_lag_NJL}
%%%%%%%%%%%%%%%%%%%%%%%%%%%%%%%%%%%%%%%%%%%%%%%%%%%%%%%%%%%%%%%%%%%%%%%%%

A simetria quiral é uma simetria importante da QCD\cite{Koch}. Como não sabemos resolver tal teoria, é comum que se utilizem teorias alternativas, como o Modelo de Walecka, com o intuito de se obter resultados que descrevam bem os dados experimentais. O modelo de Walecka consegue descrever bem os dados experimentais, mas não possui simetria quiral. O Modelo de Nambu--Jona-Lasinio, em contrapartida, possui simetria quiral, porém em sua versão original não é capaz de descrever características importantes da QCD, como a saturação da matéria nuclear. Porém, a adição de outros termos é possível:\cite{Buballa}
\begin{quote}
After the reinterpretation of the NJL model as a quark model, many authors kept the original form of the Lagrangian Eq.~\eqref{Eq:LagOriginalNJL}, with $\psi$ now being a quark field with two flavor and three color degrees of freedom. However, this choice is not unique and we can write down many other chirally symmetric interaction terms.
\end{quote}
%
Com a adição de um termo scalar-vector, é possível obter resultados adequados\cite{Koch}:
\begin{quote}
	In recent years\footnote{Década de 80!} a lot of progress has been made in describing the nucleus as a relativistic system [1\footnote{Serot, Walecka; Advances in Nuclear Physics 16},2]. The success of this approach is mainly due to the interplay of a large vector and scalar potential, the cancelation of which leads to the comparatively weak nuclear potential. In most models these forces are mediated by the exchange of effective scalar- and vector mesons. In the mean field approximation several data are reproduced reasonably well [1].
	However, most models assume an explicit mass for the nucleon and hence do not posses chiral symmetry. Since chirality is an important symmetry of the strong interaction [3], one would like to incorporate this symmetry into these models. This then changes the situation drastically. Although the saturation properties of infinite nuclear matter may be fitted [4,5], these models suffer from deficiencies and problems occur onde finite systems are considered [6]. It is hence an open problem to construct an effective, chirally invariant, relativistic field theory for the nuclear interactions.
	In this letter we present a novel approach by considering a chirally invariant model, which contains only fermionic degrees of freedom. Although it is well known that the nucleon-nucleon interaction is due to meson exchange, we believe it is interesting to construct an effective field theory without explicit mesonic degrees of freedom. [\dots]
	[\dots]
	[\dots] if we introduce an additional term in the lagrangian in order to change the density dependence of the binding energy, it should also influence the density dependence of the mass. A chirally invariant scalar-vector interaction term meets these requirements and [\dots] cures the binding problem.
\end{quote}
%
De Lee\cite{japoneses}:
\begin{quote}
	When dealing with symmetric nuclear matter, it is necessary to reproduce the property of nuclear saturation, as the Walecka model [15] has succeeded in describing phenomenologically the saturation property o symmetric nuclear matter without chiral symmetry, in which the nucleon is treated as a fundamental particle, not as a composite one. In the original NJL model with chiral symmetry, however, if the nucleon field is treated not as a composite but as a fundamental fermion field, it is unable to reproduce the nuclear matter saturation property. Here, it has been observed that the nuclear saturation property is well reproduced by introducing scalar-vector and isoscalar-vector\footnote{Ambas com acoplamento $G_{sv}^i$ na lagrangiana dada pela Eq.~(2.1) do artigo.} eight-point interaction in the original NJL model [8], the nucleon then being a fundamental fermion. Thus, it is possible to consider an NJL-type model in which the nuclear saturation property is satisfied as a possible model for nuclear matter [16-18]. For this reason, in this paper, an NJL-type model for nuclear matter with rather reasonable nuclear saturation properties is adopted.
	
	[\dots]
	
	[\dots] for nuclear matter, we adopt the extended NJL model with the scalar-vector eight-point interaction and treat the nucleon field as a fundamental fermion field with $N_c = 1$, in which $N_c$ is the number of colors. As for quark matter, we adopt the extended NJL model with $N_c = 3$ and treat the quark field as a fundamental fermion field.\footnote{Talvez tirar essa parte dos quarks quando houver uma seção para discutir essa fase.}
\end{quote}


%%%%%%%%%%%%%%%%%%%%%%%%%%%%%%%%%%%%%%%%%%%%%%%%
\subsection{Lagrangiana do modelo eNJL em SU(2)}
%%%%%%%%%%%%%%%%%%%%%%%%%%%%%%%%%%%%%%%%%%%%%%%%

Sobre o modelo\cite{Pais}:
\begin{quote}
An advantage of these models is the fact that, since they satisfy chiral symmetry, the EoS is also valid at higher densities, as the ones present in the center of compact objects.

[\dots]

The NJL model can be extended [\dots] to yield reasonable saturation properties of nuclear matter, the field $\psi$ being the nucleon field. An effective density dependent coupling constant is obtained if the following extended NJL (eNJL) Lagrangian density, which actually pushes chiral symmetry restoration to higher densities, is considered,
\begin{equation}\label{Eq:Lagrangiana_eNLJ_Pais}
\begin{split}
	\mathcal{L} =&~ \bar{\psi}(i\gamma^\mu\partial_\mu)\psi + G_s[(\bar{\psi}\psi)^2 + (\bar{\psi}i\gamma_5\vec{\tau}\psi)^2] \\
	& - G_v(\bar{\psi}\gamma^\mu\psi)^2 - G_{sv}[(\bar{\psi}\psi)^2 + (\bar{\psi}i\gamma_5\vec{\tau}\psi)^2](\bar{\psi}\gamma^\mu\psi)^2 \\
	& - G_\rho[(\bar{\psi}\gamma^\mu\vec{\tau}\psi)^2 + (\bar{\psi}\gamma_5\gamma^\mu\vec{\tau}\psi)^2] \\
	& - G_{v\rho}(\bar{\psi}\gamma^\mu\psi)^2[(\bar{\psi}\gamma^\mu\vec{\tau}\psi)^2 + (\bar{\psi}\gamma_5\gamma^\mu\vec{\tau}\psi)^2] \\
	& - G_{s\rho} [(\bar{\psi}\psi)^2 + (\bar{\psi}i\gamma_5\vec{\tau}\psi)^2][(\bar{\psi}\gamma^\mu\vec{\tau}\psi)^2 + (\bar{\psi}\gamma_5\gamma^\mu\vec{\tau}\psi)^2].
\end{split}
\end{equation}
\end{quote}

Outras informações relevantes (ainda do artigo):
\begin{itemize}
	\item Para a matéria nuclear, a degenerescência é $2 N_f$;
	\item O \emph{cutoff} $\Lambda$ é tal que a massa do nucleon no vácuo seja de 939 MeV, determinada variacionalmente;
	\item O termo proporcional a $G_v$ simula uma repulsão de curto alcance quiralmente invariante entre dois nucleons;
	\item O termo proporcional a $G_s$ simula uma repulsão de curto alcance entre os nucleons (chiral invariant)\footnote{Acredito que seja em $G_v$, pois é o que consta na última versão do artigo. De qq forma, se um é isso, o outro é o que?};
	\item ``The term in $G_{sv}$ accounts for the density dependence of the scalar coupling. For the nuclear matter, the NJL model leads to binding, but the binding energy per particle does no have a minimum except at a rather high density where the nucleon mass is small or vanishing. The introduction of the $G_{sv}$ coupling term is required to correct this.''
	\item O termo proporcional a $G_\rho$ (isovetor-vetor) é incluido para descrever a matéria nuclear assimétrica (em isospin);
	\item ``The terms $G_{v\rho}$ and $G_{s\rho}$ make the symmetry energy softer.''
	\item Os valores dos parâmetros são dados na Tab.~\ref{Tab:Parametros_eNJL} (em Tsue~\cite{japoneses} há uma discussão de como obter parametrizações para o caso em que existem os parâmetros $G_s$, $G_v$, e $G_{sv}$).
\end{itemize}

\begin{table*}
\begin{tabular}{lcccccccc}
\toprule
Model & $G_s$ ($\rm{fm}^2$) & $G_v$ ($\rm{fm}^2$) & $G_{sv}$ ($\rm{fm}^8$) & $G_\rho$ ($\rm{fm}^2$) & $G_{v\rho}$ ($\rm{fm}^8$) & $G_{s\rho}$ ($\rm{fm}^8$) & $\Lambda$ (MeV) & $m$ (MeV) \\
\midrule
eNJL1 & 4.855 & 4.65 & -6.583 & 0.5876 & 0 & 0 & 388.189 & 0 \\
eNJL1$\omega\rho$1 & 4.855 & 4.65 & -6.583 & 0.5976 & -1 & 0 & 388.189 & 0 \\
eNJL1$\omega\rho$2 & 4.855 & 4.65 & -6.583 & 0.6476 & -6 & 0 & 388.189 & 0 \\
eNJL2 & 3.8 & 3.8 & -4.228 & 0.6313 & 0 & 0 & 422.384 & 0 \\
eNJL2$\omega\rho$1 & 3.8 & 3.8 & -4.228 & 0.6413 & -1 & 0 & 422.384 & 0 \\
eNJL3 & 1.93 & 3.0 & -1.8 & 0.65 & 0 & 0 & 534.815 & 0 \\
eNJL3$\sigma\rho$1 & 1.93 & 3.0 & -1.8 & 0.0269 & 0 & 0.5 & 534.815 & 0 \\
eNJL1m & 1.3833 & 1.781 & -2.943 & 0.7 & 0 & 0 & 478.248 & 450 \\
eNJL1m$\sigma\rho$1 & 1.3833 & 1.781 & -2.943 & 0.0739 & 0 & 1 & 478.248 & 450 \\
eNJL2m & 1.078 & 1.955 & -2.74 & 0.75 & 0 & 0 & 502.466 & 450 \\
eNJL2m$\sigma\rho$1 & 1.078 & 1.955 & -2.74 & -0.1114 & 0 & 1 & 502.466 & 450 \\
\bottomrule
\end{tabular}
\caption{Conjuntos de parâmetros para a lagrangiana~\eqref{Eq:Lagrangiana_eNLJ_Pais}. \label{Tab:Parametros_eNJL}}
\end{table*}

%%%%%%%%%%%%%%%%%%%%%%%%%%%%%%%%%%%%%%
\subsection{Lagrangiana eNJL em SU(3)}
%%%%%%%%%%%%%%%%%%%%%%%%%%%%%%%%%%%%%%

Apresentar e discutir a lagrangiana em SU(3).


%%%%%%%%%%%%%%%%%%%%%%%%%%%%%%%%%%%
%%%%%%%%%%%%%%%%%%%%%%%%%%%%%%%%%%%
%%%%%%%%%%%%%%%%%%%%%%%%%%%%%%%%%%%

%%%%%%%%%%%%%%%%%%%%%%%%
\section{Próximas ações}
%%%%%%%%%%%%%%%%%%%%%%%%

\begin{itemize}
	\item Discutir:
		\begin{itemize}
			\item Motivação e origens do modelo;
			\item O modelo, com discussão do que cada coisa é na lagrangiana; Copiar tabela de parâmetros do Buballa;
			\item Determinar a Equação do Gap (tem a ver com auto-energia).
			\item Características do modelo (e.g. simetria quiral, extensibilidade);
			\item Sucessos e insucessos do modelo;
			\item Extensão eNJL: Motivação;
			\item Extensão eNJL: termos adicionais comentados;
			\item Extensão eNJL: Sucessos da versão extendida;
		\end{itemize}
	\item Problemas:
		\begin{itemize}
			\item Não sei encontrar a Eq. do gap, ou o que ela significa;
			\item A adição de mais termos faz alguma diferença na equação do gap? Depende. A interação vetorial aparentemente não, mas não sei as outras.
		\end{itemize}
\end{itemize}


