%%%%%%%%%%%%%%%%%%%%%%%%%%%%%
\section{NJL: Fase de quarks}
%%%%%%%%%%%%%%%%%%%%%%%%%%%%%

Considerando a lagrangiana do tipo NJL em SU(2) dada por
\begin{equation}
	\mathcal{L} = \bar{q}(i\slashed{\partial} - m)q + G_s[(\bar{q}q)^2 + (\bar{q}i\gamma_5\vec{\tau}q)^2] - G_v(\bar{q}\gamma^\mu q)^2,
\end{equation}
%
onde
\begin{itemize}
	\item $(\bar{q}q)^2$ e $(\bar{q}i\gamma_5\vec{\tau}q)^2$ são os canais scalar\footnote{scalar-isoscalar} e pseudoscalar-isovector, ambos com acoplamento $G_s$;
	\item $(\bar{q}\gamma^\mu q)^2$ é o canal vector-isoscalar. ``It is known, e.g., from Walecka model [109], that this channel is quite important at non-zero densities.'';
	\item ``In principle we allow for further channels [\dots], which, however, do not contribute at mean-field level as long as we have only one common quark chemical potential.''
\end{itemize}

%%%%%%%%%%%%%%%%%%%%%%%%%%%%
\subsection{Programa Débora}
%%%%%%%%%%%%%%%%%%%%%%%%%%%%

\begin{itemize}
	\item Roda variando o potencial químico $\mu$. É igual em ambos os quarks?
	\item Usa $g = 3/\pi^2$
	\item $\rho_s$ é igual ao caso para hádrons
	\item $p_F^2 = \mu^2 - M^2$; Essa equação aparece no Buballa.
\end{itemize}
