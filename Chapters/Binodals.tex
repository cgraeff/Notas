\chapter{Binodais}

Em \textcite{japoneses} e \textcite{japoneses2}, a determinação da binodal é feita de maneira relativamente simples, verificando a pressão em função de $\mu_B = \mu_N = 3\mu_q$. Temos só um ponto $(p, \mu_B)|_{\textrm{coexist}}$ de coexistência entre as fases.

Observações:
\begin{itemize}
	\item Como consideramos que as massas dos quarks são as mesmas, temos densidades e potenciais químicos iguais para o quarks $u$ e $d$. Isso significa que a fração de prótons $y_p$ é uma quantidade fixa em $y_p = 0.5$ Assim, temos que usar $y_p = 0.5$ para os hádrons também.
	\item Para o caso $T = 0$, o que poderíamos variar para fazer um gráfico? 
\end{itemize}