\chapter{Binodais}

\begin{description}
\item[Condições de Gibbs:]\footnote{Por que essas condições garantem a coexistência de fases?}
\begin{align}
	\mu_u^H &= \mu_u^Q \\
	\mu_d^H &= \mu_d^Q \\
	T^H &= T^Q \\
	P^H &= P^Q
\end{align}

\item[Binodal:] Generalização da linha de transição de fases: quando determinamos o diagrama de fases no plano $T \times \mu$, obtemos uma linha onde duas fases coexistem. Ao adicionarmos uma terceira variável, verifica-se (ou pelo menos é uma possibilidade) que esta variável altera a ``posição'' da lina. Temos então que existe uma \emph{superfície} onde ocorre a coexistência de fases. A linha que ocorre no diagrama de fases $T \times \mu$ é uma projeção da superfície no plano $T \times \mu$. \footnote{Essa interpretação está correta?}
	\begin{itemize}
		\item Acredito que as binodais não necessariamente sejam superfícies fechadas. Não consigo imaginar algo que seja fechado e tenha a projeção esperada para o plano $T \times \mu$, exceto se incluir os eixos, ou for algo similar o que tem na tese do James.
		\item Que $\mu$ é o do gráfico? $\mu_B$?
		\item Como relacionar $\mu_i^j$ a $\mu$?
		\item Para determinar uma binodal, precisamos de mais uma variável. Por enquanto, não podemos usar $T$ pois não tem na versão hadrônica, nem $y_p$, pois não temos como variar isso para a versão de quarks.
	\end{itemize}
\end{description}



Em \textcite{japoneses} e \textcite{japoneses2}, a determinação da binodal é feita de maneira relativamente simples, verificando a pressão em função de $\mu_B = \mu_N = 3\mu_q$. Temos só um ponto $(p, \mu_B)|_{\textrm{coexist}}$ de coexistência entre as fases.

Observações:
\begin{itemize}
	\item Como consideramos que as massas dos quarks são as mesmas, temos densidades e potenciais químicos iguais para o quarks $u$ e $d$. Isso significa que a fração de prótons $y_p$ é uma quantidade fixa em $y_p = 0.5$ Assim, temos que usar $y_p = 0.5$ para os hádrons também.
	\item Para o caso $T = 0$, o que poderíamos variar para fazer um gráfico? 
\end{itemize}