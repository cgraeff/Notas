%%%%%%%%%%%%%%%%%%%%%%%
\chapter{Termodinâmica}
%%%%%%%%%%%%%%%%%%%%%%%

%\begin{tikzpicture}[place/.style={circle,draw=blue!50,fill=blue!20,thick}]
%\node[place] (QCD) {};
%\node[place] (sec) [below=of QCD, label=right:test] {};
%\node[place] (thi) [below left=of sec] {};
%\node[place] (for) [below right=of sec] {};
%
%\draw [->] (QCD) to node[anchor=east] {Lab} (sec);
%\end{tikzpicture}

%%%%%%%%%%%%%%%%%%%%%%%
\section{Termodinâmica}
%%%%%%%%%%%%%%%%%%%%%%%

%%%%%%%%%%%%%%%%%%%%%%%%%%%%%%%%%
\section{Ensamble Grand Canônico}
%%%%%%%%%%%%%%%%%%%%%%%%%%%%%%%%%

Função partição:
\begin{equation}
	Z(\mu, T, V) = \Tr \exp\left[-\beta(H-\mu\hat{N})\right]
\end{equation}

Potencial termodinâmico por unidade de volume:
\begin{equation}
	\Omega(\mu, T) = -\frac{-\Tr\ln Z}{V} = -P(\mu,T)
\end{equation}

\begin{equation}
	Vd\Omega = -S dT - PdV - Nd\mu
\end{equation}

\begin{align}
	\frac{S}{V} &= \left(\frac{\partial P}{\partial T}\right)_\mu \\
	\frac{N}{V} &= \left(\frac{\partial P}{\partial\mu}\right)_T
\end{align}
